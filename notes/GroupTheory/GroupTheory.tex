\documentclass{article}

\usepackage{amsmath}
\usepackage{amssymb}
\usepackage{parskip}
\usepackage{fullpage}
\usepackage{hyperref}

\hypersetup{
    colorlinks=true,
    linkcolor=black,
    urlcolor=blue,
    pdftitle={GroupTheory},
    pdfpagemode=FullScreen,
}

\title{Group Theory}
\author{Paolo Bettelini}
\date{}

\begin{document}

\maketitle
\tableofcontents
\pagebreak

\section{Groups}

\subsection{Binary operations}

Let \(G\) be a set. A \textit{binary operation} \(\circ\) on \(G\) is a map
\[
    G \times G \to G,
    \quad\quad\quad\quad
    (x,y) \to x \circ y
\]

% closure

\subsection{Cayley tables}

A binary operation \(\circ\) on a finite set \(G\) can be
visualized using a \textit{Cayley table}.

Example: \(G=\{0,1\}\) and \(\circ \equiv \text{multiplication}\).
\begin{tabular}{|c|c|c|}
    \hline
    \circ{} & 0 & 1 \\
    \hline
    0 & 0 & 0 \\
    \hline
    1 & 0 & 1 \\
    \hline
\end{tabular}

\subsection{Definition}

A \textit{group} \((G,\circ)\) is a tuple containing a set \(G\) and
a binary operation \(\circ\) where \(\circ\) satisfies.

\begin{enumerate}
    \item \textbf{Associative}: \(\forall a,b,c\in G a \circ (b \circ c) = (a \circ b) \circ c\)
    \item \textbf{Identity}: \(\exists e \,|\, \forall a \in G, ea=ae=a\) 
    \item \textbf{Inverse}: \(\forall a\in G \exists a^{-1} | a^{-1}a = aa^{-1} = e\)
\end{enumerate}

The element \(e\) is unique whereas \(a^{-1}\) depends on \(a\).

% https://cameroncounts.files.wordpress.com/2016/11/groups.pdf

\end{document}

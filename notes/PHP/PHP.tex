\documentclass{article}

\usepackage{fullpage}
\usepackage{listings,xcolor}
\usepackage{hyperref}

\hypersetup{
    colorlinks=true,
    linkcolor=black,
    urlcolor=blue,
    pdftitle={PHP},
    pdfpagemode=FullScreen,
}

\definecolor{dkgreen}{rgb}{0,.6,0}
\definecolor{dkblue}{rgb}{0,0,.6}
\definecolor{dkyellow}{cmyk}{0,0,.8,.3}

\lstset{
  language        = php,
  basicstyle      = \small\ttfamily,
  showstringspaces=false,
  keywordstyle    = \color{dkblue},
  stringstyle     = \color{red},
  identifierstyle = \color{dkgreen},
  commentstyle    = \color{gray},
  emph            =[1]{php},
  emphstyle       =[1]\color{black},
  backgroundcolor=\color{lightgray},
  emph            =[2]{if,and,or,else},
  emphstyle       =[2]\color{dkblue}}

\title{PHP}
\author{Paolo Bettelini}
\date{}

\begin{document}

\maketitle
\tableofcontents
\pagebreak

\section{Variable types and constants}

\subsection{Types}

\begin{itemize}
    \item String
    \item Integer
    \item Float
    \item Boolean
    \item Array
    \item Object
    \item NULL
    \item Resource
\end{itemize}

\subsection{Examples}

\begin{lstlisting}
<?php
    $x = 41976;
    $y = 'Hello world!';
    $xx = "Hello world!";
    $z = 10.365;
    $cars = array("Volvo","BMW","Toyota");
    $n = null;
    $f = false;
    
    var_dump($x); // prints to the stdout variable type and content
?>
\end{lstlisting}

\subsection{Constants}

The syntax to create a contant value is as follows

\begin{lstlisting}
define(name, value, case-insensitive);
\end{lstlisting}

\begin{lstlisting}
<?php
    define("GREETING", "Welcome to W3Schools.com!", true);
    echo greeting;
?>
\end{lstlisting}

\pagebreak

\section{Comparisons}

\subsection{Operator}

\begin{itemize}
    \item == \ \ compares two values \(\rightarrow\) 2.0 == 2 (true)
    \item === compares two values \&\& data type \(\rightarrow\)  2.0 === 2 (false)
\end{itemize}

\subsection{String comparisons}

the \textbf{strcmp} function returns \(0\) if the two strings are equal.
It return another number if they are different. It performs a calculation with the ASCII values of all characters in the strings.
\\
For example \textbf{strcmp("a", "b") == -1} since the ASCII values of \(A-B = -1\) \\ 
Or \textbf{strcmp ("aa", "ab") == -256}, in this case the additional letters have a weight of \(2 ^ 8 \) and so on. 

\begin{lstlisting}
<?php
    $var1 = "Hello";
    $var2 = "HEllo";
    if (strcmp($var1, $var2) == 0) {
        echo '$var1 is equal to $var2 in a case-sensitive string comparison';
    } elseif (strcasecmp($var1, $var2) == 0) {
        echo '$var1 is not equal to $var2 in a case insensitive string comparison';
    }
?>
\end{lstlisting}

\pagebreak

\section{Control flow}

\subsection{Conditions}

\begin{lstlisting}
<?php
    $t = date("H");
    
    if ($t < "10") {
      echo "Have a good morning!";
    } elseif ($t < "20") {
      echo "Have a good day!";
    } else {
      echo "Have a good night!";
    }
?>
\end{lstlisting}

\subsection{Loops}

\begin{lstlisting}
<?php
    while (condition is true) {
      code to be executed;
    }
    
    do {
      code to be executed;
    } while (condition is true);
    
    for ($x = 0; $x < 10; $x++) {
      code to be executed; // 10 times
    }
    
    foreach ($array as $value) {
      code to be executed;
    }
?>
\end{lstlisting}

\subsection{Switches}

\begin{lstlisting}
<?php
    switch (n) {
      case label1:
        code to be executed if n=label1;
        break;
      case label2:
        code to be executed if n=label2;
        break;
      case label3:
        code to be executed if n=label3;
        break;
        ...
      default:
        code to be executed if n is different from all labels;
    }
?>
\end{lstlisting}

\pagebreak

\section{Require and include}

\subsection{Definition}

You can use commands to import code from another file
\textbf{require} and \textbf{include}. They both serve the same purpose, however
there is a difference in behavior in case of error:

\begin{itemize}
    \item \textbf{require} FATAL ERROR \(\rightarrow\) execution stops 
    \item \textbf{include} WARNING \(\rightarrow\) execution continues
\end{itemize}

so you have to use \textit{require} when the file is essential.

The syntax is as follows: 

\begin{lstlisting}
<?php
    require 'filename';
    include 'filename';
?>
\end{lstlisting}

\subsection{Require Once}

The require only imports itself once.
If the file already exists, the file is not imported again.
\\
Importing a file multiple times will execute its code.
This ensures that you do not override functions (which cannot be redeclared).

\begin{lstlisting}
<?php
    require_once('file.php');
    include_once('file.php');
?>
\end{lstlisting}

The \textit{require} should be used when the file code needs to be rerun.

\section{Namespaces}

Namespace are like packages for files. \\

\textbf{File1:}
\begin{lstlisting}
<?php namespace foo;
    class Cat {
        static function says() {echo 'meoow';} 
    }
?>
\end{lstlisting}

or to simplify their use in classes.

\textbf{File2:}
\begin{lstlisting}
<?php namespace main
  include 'file1.php';
  use foo\Cat as feline
  echo feline::says(), "<br />\n";
?>
\end{lstlisting}

\pagebreak

\section{Functions}

\subsection{Definition}

\begin{lstlisting}
<?php
    function functionName() {
      do things;
    }
    
    functionName(); // call the function
?>
\end{lstlisting}

\subsection{Parameters}

\subsubsection{Mandatory parameters}

\begin{lstlisting}
<?php
    function greetName($fname) {
      echo "hello $fname.<br>";
    }
    
    greetName("Giecinz");
    greetName("Paola");
?>
\end{lstlisting}

\subsubsection{Optional parameters}

\begin{lstlisting}
<?php
    function setHeight(int $minheight = 50) {
      echo "The height is : $minheight <br>";
    }
    setHeight(350);
    setHeight(); // default value = 50
?>
\end{lstlisting}

\subsubsection{Type declaration}

The type declaration is used to force a parameter type, return value,
or property of a class. 

If the type is not respected, a \textbf{TypeError} is raised.

\begin{lstlisting}
<?php
  function test1(boolean $param) {}
  test1(true);

  function test2(): int {
    return 1;
  }
?>
\end{lstlisting}

\pagebreak

\subsubsection{Return}

\begin{lstlisting}
<?php
    function sum(int $x, int $y) {
        return $x + $y;
    }
    echo "5 + 10 = " . sum(5, 10) . "<br>";
?>
\end{lstlisting}

\subsubsection{Explicit return}

\begin{lstlisting}
<?php
    function sum(int $x, int $y) : int {
      return $x + $y;
    }
?>
\end{lstlisting}

\section{Scopes and global variables}

\begin{itemize}
    \item \textbf{local} a variable declared inside a function etc.
    \item \textbf{global} a variable declared with the keyword global or outside a function.
    \item \textbf{static} a static variable.
\end{itemize}

\begin{lstlisting}
<?php
    $x = 5;
    $y = 10;
    
    function myTest() {
      global $x, $y;
      $y = $x + $y;
    }

    class Foo {
      static $my_var = 'Foo'; // static var
    }
    
    myTest();
    echo $y; // outputs 15
?>
\end{lstlisting}

\pagebreak

\section{Variables}

\subsection{Strings}

\subsubsection{Multi-line string}

\begin{lstlisting}
    print <<< END
        <p style="background-color: yellow">
        Four score and seven years ago<br/>
        our fathers set onto this continent<br/>
        (and so on ...)<br/>
        </p>
    END;
\end{lstlisting}

\subsubsection{Concatenation}

\begin{lstlisting}
    $a = "str1" . "str2";
    $a .= "str3";
\end{lstlisting}

\subsubsection{End of line}

The constant \textbf{PHP\_EOL} represents a new line.

\subsubsection{Some functions}

\begin{lstlisting}
    echo strlen(" ciao");
    echo strlen(ltrim(" ciao"));
    echo strtoupper("Anghilotto");
    echo strtolower("AngHiLotto");
    echo strcmp("a", "a") . "<br>";
    echo substr("ciao", 2, 1);
    echo str_replace("i", "a", "ciao");
\end{lstlisting}

\subsection{Casting}

\begin{lstlisting}
    echo (int)(9.2);
    echo (int)("text"); // returns 0
\end{lstlisting}

\subsection{Nested variables}

You can allocate a variable whose name is the value of another variable.

\begin{lstlisting}
    $my_var= 'variablename';
    $$my_var = 'Some text';
    echo $variablename . "<br><br>";
\end{lstlisting}

\subsection{Dates}

\begin{lstlisting}
    // current date
    echo date("Y/m/d");
    echo date("Y.m.d");
    echo date("Y-m-d");

    $date = new DateTime('now');
    echo $date->format('Y-m-d H:i:s');

    $diff = date_diff(date_create('2003-05-28'), date_create('2022-05-28));
\end{lstlisting}

\subsection{Arrays}

\subsubsection{Declaration}

\begin{lstlisting}
    // array
    $cars = array("Volvo", "BMW", "Toyota");
    $cars[0] = "Panda";

    $arr = array();
    array_push($colors,"blue","yellow");
    
    // associative array
    $age = ["Peter"=>"35", "Ben"=>"37", "Joe"=>"43"];
\end{lstlisting}

\subsubsection{Iterating}

\begin{lstlisting}
    $colors = array("red", "green", "blue", "yellow");
    foreach ($colors as $value) {
        echo "$value <br>";
    }
\end{lstlisting}

\subsubsection{Some functions}

\begin{lstlisting}
    $arr = array('Hello','World!','Beautiful','Day!');
    
    $str = implode(" ", $arr);
    $c = count($arr);
    array_reverse($arr$)

    $arr = explode(" ", $str$);

    sort($arr$); // sort array

    // Associative arrays
    asort($girl); // sort in ascending order according to value
    ksort($girl); // sort in ascending order according to key
    arsort($girl); // sort in descending order according to value
    krsort($girl); // sort in descending order according to key
\end{lstlisting}

\section{Files}

\subsection{Read and write}

\begin{lstlisting}
    $content = file_get_contents($inFile);
    file_put_contents($outFile, $result);
\end{lstlisting}

\subsection{Csv}

\begin{lstlisting}
    function writeToCsv(string $out, $values) {
        $fp = fopen($out, 'w');
        foreach ($values as $key => $val) {
            fputcsv($fp, [$key,$val], ";");
        }
        fclose($fp);
    }
\end{lstlisting}

\begin{lstlisting}
    if ($handle = fopen("parole.csv", "r")) {
        while ($data = fgetcsv($handle, 1000, ";")) {
            // ...
        }
    
        fclose($handle);
    }

    $fp = fopen($out, 'w');
    foreach ($values as $key => $val) {
        fputcsv($fp, [$key,$val], ";");
    }
    fclose($fp);
\end{lstlisting}

\section{Requests}

HTML form

\begin{lstlisting}[language=HTML]
    <form action=file.php” method=“POST”>
    <input type=“text” name=field1>
    <input type=“submit”>
    </form>
\end{lstlisting}

Check page request type

\begin{lstlisting}
    if ($_SERVER['REQUEST_METHOD'] == 'POST') {
        // POST
    } else {
        // GET
    }
\end{lstlisting}

All the request-variables are in the \textbf{\$\_POST} array or  \textbf{\$\_GET} array.

\begin{lstlisting}
    if(isset($_POST['field1'])) {
        $a = $_POST['field1']
    }
\end{lstlisting}

\end{document}
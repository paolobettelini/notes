\documentclass{article}

\usepackage{amsmath}
\usepackage{amssymb}
\usepackage{parskip}
\usepackage{fullpage}
\usepackage{hyperref}
\usepackage{tikz}
\usepackage{makecell}

\usetikzlibrary{ % tikz packages
	cd % tikz-cd communitative diagrams
}

\hypersetup{
    colorlinks=true,
    linkcolor=black,
    urlcolor=blue,
    pdftitle={CategoryTheory},
    pdfpagemode=FullScreen,
}

\title{Category Theory}
\author{Paolo Bettelini}
\date{}

\begin{document}

\maketitle
\tableofcontents
\pagebreak

% after abstraction we notice that everything is the same. Every theory is the same thing.

\section{Category}

A category consists of \textit{objects} and \textit{morphism} or \textit{arrows}.
\\
An arrow has a beginning and an ending, and it goes from one object to another.
\\
Objects serve the purpose of marking the beginning and ending of a morphism.

\[
    \begin{tikzcd}
        a \arrow[r, bend left] \arrow[r, bend left=49, shift left] \arrow[loop, distance=2em, in=215, out=145] & b \arrow[loop, distance=2em, in=35, out=325] \arrow[l, bend left]
    \end{tikzcd}
    \\
    \makecell[l] {
        \text{An example of}
        \\
        \text{objects and morphisms}
    }
\]

\subsection{Composition}

Composition is a property that says that if there is an arrow from
\(a\) to \(b\), and an arrow from \(b\) to \(c\), there must exist an arrow
from \(a\) to \(c\).

\begin{center}
    \begin{tikzcd}
        a \arrow[r, "f"]
        \arrow[rr, "f \circ g", bend left=49] & b
        \arrow[r, "g"] & c
    \end{tikzcd}
\end{center}

\subsection{Identity}

For every object there is an identity arrow.

\begin{center}
    \begin{tikzcd}
        a \arrow["\text{id}_a"', loop, distance=2em, in=35, out=325]
    \end{tikzcd}
\end{center}

The composition of an arrow with an identity is the arrow itself

\begin{center}
    \begin{tikzcd}
        a \arrow[r, "f"] &
        b \arrow["\text{id}_b"', loop, distance=2em, in=35, out=325]
    \end{tikzcd}
\end{center}

\[
    f \circ \text{id}_b = f
\]

and also vice versa

\[
    \text{id}_b \circ f = f
\]

\subsection{Associativity}

Compositions have the associative property

\begin{center}
    \begin{tikzcd}
        a \arrow[r, "f"]
        \arrow[rr, "g \circ f", bend left, shift left=2]
        \arrow[rrr, "h \circ (g \circ f)", bend left=49, shift left=2]
        \arrow[rrr, "(h \circ g) \circ f", bend right=49, shift right=2] & b
        \arrow[r, "g"] \arrow[rr, "h \circ g", bend right, shift right=2] & c
        \arrow[r, "h"] & d
    \end{tikzcd}
\end{center}

\[
    h \circ (g \circ f) = (h \circ g) \circ f
\]

\pagebreak

\section{Homomorphism}

An Homomorphism is a map between two structures of the same type.

\subsection{Isomorphisms}

A function \(f\) going from \(a\) to \(b\)

\[
    f:\quad a \rightarrow b
\]

is invertible if there is a function \(g\) that goes from \(b\) to \(a\)

\[
    b:\quad b \rightarrow a
\]

such that

\begin{align*}
    g \circ f &= \text{id}_b
    \\
    f \circ g &= \text{id}_a
\end{align*}

\begin{center}
    \begin{tikzcd}
        a \arrow[r, "f", bend left] & b \arrow[l, "g", bend left]
    \end{tikzcd}
\end{center}

This is a \textit{bijective} homomorphism and it's called isomorphism.
An isomorphism is labelled \(\xrightarrow{\sim}\).

\subsection{Epimorphisms}

\begin{center}
    \begin{tikzcd}
        a \arrow[r, "f"] &
        b \arrow[r, "g_1", shift left]
        \arrow[r, "g_2"', shift right] & c
    \end{tikzcd}
\end{center}

% https://www.youtube.com/watch?v=O2lZkr-aAqk&t=1072s

\subsection{Monomorphisms}

% https://www.youtube.com/watch?v=p54Hd7AmVFU&t=1344s
% https://tikzcd.yichuanshen.de/

\end{document}

\documentclass{article}

\usepackage{amsmath}
\usepackage{amssymb}
\usepackage{parskip}
\usepackage{fullpage}
\usepackage{hyperref}

% lilypond is not an classic sty package.
% First you run lilypond-book <file.tex> or <file.lytex>
% lilypond-book --pdf file.tex --out=target
% lilypond will parse every lilypond enviroment in the tex file
% for each enviroment it will generate a pdf image
% then it will generate another tex file in the target folder,
% where it will replace the lilypond enviroments (which do not exist in LaTeX,
% compiling a document normally would give an error) with the pre compiled images
% So you're basically using an enviroment that doesn't exist, then using an external tool
% to parse the document and create another identical TeX file except it points
% to the precompiled music sheets
% I have yet to understand why they couldn't just \usepackage{lilypond} and that's it.

% lilypond-book --pdf *.tex --out=target
% lilypond-book --pdf *.tex --out=target
% cd target
% lualatex *.tex
% mv *.pdf ../
% cd ..
% # these files stack up, they have a diffent name each time
% find . -type f -name "tmp*.pdf" -delete
% find . -type f -name "tmp*.out" -delete

\hypersetup{
    colorlinks=true,
    linkcolor=black,
    urlcolor=blue,
    pdftitle={Western Music Notation},
    pdfpagemode=FullScreen,
}

\title{Western Music Notation}
\author{Paolo Bettelini}
\date{}

\begin{document}

\maketitle
\tableofcontents
\pagebreak

\section{Staff}

The \textbf{staff} or \textbf{stave} is a set of five horizontal lines and four spaces that each represent a different
note.

% empty staff
\begin{lilypond}
    \version "2.22.2"
    \paper {
      #(set-paper-size "letter")
      top-margin = 0.7\cm
    }
    \layout { 
      indent = 0.0\cm
      pagenumber = no
    }
    \new Score \with {
      \override TimeSignature #'transparent = ##t
      \override Clef #'transparent = ##t
      defaultBarType = #""
      \remove Bar_number_engraver
      \remove Clef_engraver
    } <<
        \context Staff {
            \repeat unfold 3 {
                s1
            }
        }
    >>
\end{lilypond}

\section{Clef}

A \textbf{clef} is a musical symbol used to indice which notes are represented
by the lines and spaces on a musical stave.
There are mainly 3 types of clefs.

% types of clefs
\begin{center}
    \begin{lilypond}
        \version "2.22.2"
        \score {
            \new StaffGroup <<
                \new Staff \with {
                    \override TimeSignature #'transparent = ##t
                    instrumentName = \markup \center-column {
                        "Treble"
                        "Clef"
                    }
                } {
                    \clef treble << \gtr >>
                }
                \new Staff \with {
                    \override TimeSignature #'transparent = ##t
                    instrumentName = \markup \center-column {
                        "Alto"
                        "Clef"
                    }
                } {
                    \clef alto << \gtr >>
                }
                \new Staff \with {
                    \override TimeSignature #'transparent = ##t
                    instrumentName = \markup \center-column {
                        "Bass"
                        "Clef"
                    }
                } {
                    \clef bass << \gtr >>
                }
            >>
            \layout {
                indent = 2\cm
                
            }
        }
    \end{lilypond}
\end{center}

\section{Key Signature}

A \textbf{key signature} is a set of symbols placed
on the staff at the beginning of a section of music.
The initial key signature is placed right after the clef.

The symbols are \(\sharp\), \(\flat\) and \(\natural\).

% JSON

\end{document}

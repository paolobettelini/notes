\documentclass{article}
\usepackage{amsmath}
\usepackage{amssymb}
\usepackage{parskip}
\usepackage{fullpage}
\usepackage{hyperref}
\usepackage{listings,listings-rust}

\hypersetup{
    colorlinks=true,
    linkcolor=black,
    urlcolor=blue,
    pdftitle={Rust},
    pdfpagemode=FullScreen,
}

\title{The Rust programming language}
\author{Paolo Bettelini}
\date{}

\begin{document}

\maketitle
\tableofcontents
\pagebreak

\section{Basic Types}

%\subsection{Boolean}
%\subsection{Numbers}
%\subsubsection{Signed integers}
%\subsubsection{Unsigned integers}
%\subsubsection{Floating points}
%\subsection{Text}

\begin{lstlisting}[language=Rust, style=boxed, numbers=none]
// boolean
bool

// signed integers
i8, i16, i32, i64, i128, isize

// unsigned integers
u8, u16, u32, u64, u128, usize

// floating points
f32, f64

// Text
char, String, str
\end{lstlisting}

\section{Tuples}

Tuples are a combination of multiple types.
Tuples can contain any number of types and/or other tuples.

\begin{lstlisting}[language=Rust, style=boxed, numbers=none]
let coordinates = (101, 3, 4);
let person = ("Paolo", "Bettelini", 18);
let status: (bool, (u128, i32)) = (true, (1u128, 2));
\end{lstlisting}

\subsection{Returning from loops}

\begin{lstlisting}[language=Rust, style=boxed, numbers=none]
let mut counter = 0;

let result = loop {
    counter += 1;

    if counter == 10 {
        break counter;
    }
};
\end{lstlisting}

\end{document}

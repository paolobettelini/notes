\documentclass[a4paper]{article}

\usepackage{amsmath}
\usepackage{amssymb}
\usepackage{parskip}
\usepackage{fullpage}
\usepackage{hyperref}

\hypersetup{
    colorlinks=true,
    linkcolor=black,
    urlcolor=blue,
    pdftitle={Integers},
    pdfpagemode=FullScreen,
}

\title{Integers}
\author{Paolo Bettelini}
\date{}

\newcommand{\divides}{\,|\,}

\begin{document}

\maketitle
\tableofcontents
\pagebreak

\section{Divide operator}

\subsection{Definition}

Given two integers \(a\) and \(b\),
we say that \(a \divides b\) if \(a\) divides \(b\),
meaning that
\[
    \exists x \,|\, ax = b
\].

\subsection{Properties}

Given the integers \(a\), \(b\) and \(c\)

\begin{align*}
    a \divides b \iff -a \divides b \iff a \divides -b \\
    |a| \leq |b|, \quad b \neq 0 \\
    a \divides b \implies a \divides bc \\
    a \divides b \land b \divides c \implies a \divides c
\end{align*}

\subsection{Division with remainder}

Given two integers \(a\) and \(b\) with \(b > 0\),
\[
    \exists_{=1} q,r \,|\, a=bq+r, \quad 0 \leq r < b
\]

% TODO proof

Let \(q\) and \(r\) be the quotient and remainder of the division of \(b\)
by \(a\).
The common divisors of \(a\) and \(b\) are equivalent to the common divisors of \(r\) and \(q\).

% TODO proof

\subsection{Euclidean algorithm}

Euclid's algorithm, is an efficient method for computing the greatest common divisor of two integers
\(a\) and \(b\) where \(b > 0\).

\subsection{Bézout's identity}

Let \(a\) and \(b\) be integers with greatest common divisor \(d\).
Then, there exist integers \(x\) and \(y\) such that
\[
    ax+by=d
\]
Furthermore, the integers \(az+bt\) are multiples of \(d\).

\pagebreak

\end{document}

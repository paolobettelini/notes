\documentclass[a4paper]{article}

\usepackage{amsmath}
\usepackage{amssymb}
\usepackage{parskip}
\usepackage{fullpage}
\usepackage{hyperref}

\hypersetup{
    colorlinks=true,
    linkcolor=black,
    urlcolor=blue,
    pdftitle={Integers},
    pdfpagemode=FullScreen,
}

\title{Integers}
\author{Paolo Bettelini}
\date{}

\newcommand{\divides}{\,|\,}

\begin{document}

\maketitle
\tableofcontents
\pagebreak

\section{Divides operator}

\subsection{Definition}

Given two integers \(a\) and \(b\),
we say that \(a \divides b\) if \(a\) divides \(b\),
meaning that
\[
    \exists x \,|\, ax = b
\].

\subsection{Properties}

Given the integers \(a\), \(b\) and \(c\)

\begin{align*}
    a \divides b \iff -a \divides b \iff a \divides -b \\
    |a| \leq |b|, \quad b \neq 0 \\
    a \divides b \implies a \divides bc \\
    a \divides b \land b \divides c \implies a \divides c
\end{align*}

\section{Division with remainder}

Given two integers \(a\) and \(b\) with \(b > 0\),
\[
    \exists_{=1} q,r \,|\, a=bq+r, \quad 0 \leq r < b
\]

% TODO proof

Let \(q\) and \(r\) be the quotient and remainder of the division of \(b\)
by \(a\).
The common divisors of \(a\) and \(b\) are equivalent to the common divisors of \(r\) and \(q\).

% TODO proof

\section{Euclidean algorithm}

Euclid's algorithm, is an efficient method for computing the greatest common divisor of two integers
\(a\) and \(b\) where \(b > 0\).

Consider
\[
    a = bq + r
\]
The process is iterative.
For each iteration take the coefficient of the quotient (\(b\)) and divide it by the remainder.

\begin{align*}
    &a = bq + r, &0 \leq r < b \\
    &b = rq_1 + r_1, &0 \leq r_1 < r \\
    &r = r_1q_2 + r_2, &0 \leq r_2 < r_1 \\
    \phantom{ } &\qquad \vdots & \\
    &r_n = r_{n+1}q_{n+2} + r_{n+1}, &0 \leq r_{n+2} < r_{n+1} \\
    &r_{n+1} = r_{n+2}q_{n+3} + 0&
\end{align*}
This sequence is stricly decreasing and will terminate with a null remainder.
The last remainder \(r_{n+2}\) is then the greatest common divisor between \(a\) and \(b\).

\section{Bézout's identity}

Let \(a\) and \(b\) be integers with greatest common divisor \(d\).
Then, there exist integers \(x\) and \(y\) such that
\[
    ax+by=d
\]
Furthermore, the integers \(az+bt\) are multiples of \(d\).

\section{Greatest common divisor of multiple integers}

The greatest common divisors of \(a_0, a_1, \cdots, a_n\), denoted \(\gcd(a_0, a_1, \cdots, a_n)\),
is the greatest integer \(n\) such that \(n \divides a_k\).

There exists integers \(u_k\) such that
\[
    a_0u_0 + \cdots a_nu_n = \gcd(a_0, a_1, \cdots, a_n)
\]

For \(n \geq 2\), \(\gcd(\gcd(a_0, \cdots, a_{n-1}), a_n) = \gcd(a_0, \cdots, a_n)\).

Given an integer \(c\), \(\gcd(ca_0, ca_1, \cdots, ca_n) = c \cdot \gcd(a_0, a_1, \cdots, a_n)\).

% TODO proof

% TODO proof

\subsection{Coprime numbers}

Two integers \(a\) and \(b\) are said to be \textbf{coprime}
if they have no common divisor other than \(1\), meaning that \(\gcd(a,b)=1\).

Let \(d = \gcd(a, b) \neq 0\). Then, the integers \(a'\) and \(b'\) where \(a = da'\) and \(b = db'\)
are coprime because \(d = \gcd(da', db') = d\cdot \gcd(a', b') \implies \gcd(a', b') = 1\).

% TODO pag 35 fundamental theorem of arithmetic

\pagebreak

\section{Linear diophantine equations}

\subsection{Definition}

A linear diophantine equations is an equation with 2 or more integer unknowns of the following form.
\[
    a_1x_1 + a_2+x_2 + \cdots + a_nx_n = b
\]
where \(a_i,x_i,b \in \mathbb{Z}\) and \(x_i\) are unknowns.

The equation is solvable iff \(\gcd(a_1,a_2,\cdots,a_n) \divides b\).
This is because the left-hand side will always be a value that is a multiple
of \(\gcd(a_1,a_2,\cdots,a_n)\).

In fact, if \(\gcd(a_1,a_2,\cdots,a_n) \divides b\),
then \(b = \gcd(a_1,a_2,\cdots,a_n)e\) for some \(e\). \\
By the Bezout identity, which can be find using Euclid's algorithm, we have
\[a_1v_1+a_2v_2+\cdots+a_nv_n=\gcd(a_1,a_2,\cdots,a_n)\]
meaning that an integer solution is given by \(x_n=ev_n\).

\subsection{Two unkowns}

Let \[ax+by=c\] be a solvable diophantine equation
and let \(d = \gcd(a,b)\).
If \(d=0\), this means that \(a=b=0\) and \(c=0\) since \(d \divides c\) (identity).
Otherwise, let \(a=da'\), \(b=db'\) and \(c=dc'\), then the equation is equivalent to
\[
    a'x + b'y = c'
\]
Consider any solution to the equation \(x=\overline{x}\) and \(y=\overline{y}\),
then the equation has infinitely many other solutions given by
\begin{align*}
    x &= \overline{x} + b'h
    y &= \overline{y} + a'h
\end{align*}
for any \(h \in \mathbb{Z}\).

% PROOF of d != 0
% e da lì in poi

% https://algebrainsubria.altervista.org/algebra.pdf

\end{document}

\documentclass{article}

\usepackage{amsmath}
\usepackage{amssymb}
\usepackage{parskip}
\usepackage{fullpage}
\usepackage{hyperref}

\hypersetup{
    colorlinks=true,
    linkcolor=black,
    urlcolor=blue,
    pdftitle={TheoryOfComputation},
    pdfpagemode=FullScreen,
}

\title{Theory of Computation}
\author{Paolo Bettelini}
\date{}

\begin{document}

\maketitle
\tableofcontents
\pagebreak

\section{Alphabet}

An alphabet is a set of values which represents the solutions
to a certain problem. \\
The set \(\{0,1\}\) is the binary set. The set \(\{0,1\}^*\) is the set of
all binary strings (union of all \(n\)-permutations of \(\{0,1\}\) and an empty string).
In general, if \(\Sigma\) is an alphabet \(\Sigma^*\) is the set
of all strings over \(\Sigma\)
\[
    \Sigma^* = \lambda \cup \bigcup_{n\in\mathbb{N}} \Sigma^n
\]
where \(\lambda\) is the empty string.
Note that \(\lambda \neq \varnothing \neq \{\lambda\}\).
\section{Turing Machines}

% https://web.cs.ucdavis.edu/~doty/fall2015ecs120/notes.pdf

\pagebreak

\end{document}

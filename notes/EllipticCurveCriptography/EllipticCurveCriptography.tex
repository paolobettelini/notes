\documentclass{article}
\usepackage[utf8]{inputenc}
\usepackage{amsmath}
\usepackage{amssymb}
\usepackage{parskip}
\usepackage{dsfont}
\usepackage{fullpage}
\usepackage{pgfplots}

\title{Elliptic Curve Criptography}
\author{Paolo Bettelini}
\date{}

\pgfplotsset{compat=1.18}

\begin{document}

\maketitle
\tableofcontents
\pagebreak

\section{Elliptic Curves}

\subsection{Definition}

An elliptic curve \(E\) is a set of points such that
\[
	E=\{(x,y)\,|\,y^2=x^3+ax+b\}\cup\{O\},\quad 4a^3+27b^2\neq 0
\]

\begin{center}
	\begin{tikzpicture}
		\begin{axis}[
			xmin=-6,
			xmax=6,
			ymin=-7,
			ymax=7,
			xlabel={\(x\)},
			ylabel={\(y\)},
			scale only axis,
			axis lines=middle,
			domain=-1.912931:3,
			samples=100,
			smooth,
			clip=false,
			axis equal image=true,
			scale=0.9
		]
			\addplot [red] {sqrt(x^3+7)}
				node[right, color=black] {\(y^2=x^3+7\)};
			\addplot [red] {-sqrt(x^3+7)};
		\end{axis}
	\end{tikzpicture}
\end{center}
Where \(O\) is a point at infinity.

The elliptic curve is symmetrical about the x-axis.\\
The opposite of a point \(P\) is its reflection \(-P\).

The coefficients \(a,b\) be part of
\begin{itemize}
	\item \(\mathbb{R}\) Real numbers
	\item \(\mathbb{Q}\) Rational numbers
	\item \(\mathbb{C}\) Complex numbers
	\item \(\mathbb{Z}/p\mathbb{Z}\) Finite field
\end{itemize}

\subsection{Addition}

Given two points \(P,Q\in E\) we can describe a unique third point.\\
We take the line that intersects \(P\) and \(Q\), the opposite of the third intersection with the curve is out point.
\[
	P+Q=-R
\]

If \(P=Q\), the intersection line will be given by the tangent at that point.\\
If \(P=-Q\), \(P+Q=O\).\\
If \(P=-P\) (inflection point, the concavity of the curve changes) \(R=P\), \(P+P=-P=P\).\\
We consider \(-O\) to be \(O\).

The intersection line \(mx+q\) is given by
\[
	m=\frac{P_y-Q_y}{P_y-Q_x}
\]
and
\[
	q=P_y-mP_x
\]

\begin{center}
	\begin{tikzpicture}
		\begin{axis}[
			xmin=-6,
			xmax=6,
			ymin=-6,
			ymax=6,
			xlabel={\(x\)},
			ylabel={\(y\)},
			scale only axis,
			axis lines=middle,
			domain=-1.912931:3,
			samples=200,
			smooth,
		]
			\addplot [draw=red] {sqrt(x^3+7)};
			\addplot [draw=red] {-sqrt(x^3+7)};
			\addplot [draw=blue][domain=-1.75:1.63] {0.617*x+2.361};

			\draw [color=black, fill=black] (-1.75,1.281) circle (0.1) node [above left] {\(P\)};
			\draw [color=black, fill=black] (0.5,2.669) circle (0.1) node [above] {\(Q\)};
			\draw [color=black, fill=black] (1.631,3.367) circle (0.1) node [below right] {\(R\)};
			\draw [dashed] (1.631,3.367) -- (1.631,-3.367);
			\draw [color=black, fill=black] (1.631,-3.367) circle (0.1) node [right] {\(-R\)};
		\end{axis}
	\end{tikzpicture}
\end{center}

\subsection{Scalar Multiplication}

Given a point \(P\in E\), multiplying \(kP\) where \(k\in\mathbb{Z}\) is equivalent to adding \(P\) to itself \(k\) times.\\
Computing \(2P\) is the equivalent of \(P+P\) which can be calculated as \(P+Q=-R\).

\section{Finite field}

\section{Diffie Hellman}

\subsection{Definition}

Diffie–Hellman key exchange is a method of securely exchanging cryptographic keys over a public channel.
[...]

\subsection{Using elliptic curves}

[...]
TODO

\end{document}
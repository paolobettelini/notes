\documentclass{article}

\usepackage{amsmath}
\usepackage{amssymb}
\usepackage{parskip}
\usepackage{fullpage}
\usepackage{hyperref}
\usepackage{tikz}

\usepackage[backend=bibtex]{biblatex}

\addbibresource{resources/references.bib}

\usetikzlibrary{angles,quotes}

\hypersetup{
    colorlinks=true,
    linkcolor=black,
    urlcolor=blue,
    pdftitle={Physical Rendering},
    pdfpagemode=FullScreen,
}

\title{Physical Rendering}
\author{Paolo Bettelini}
\date{}

\begin{document}

\maketitle
\tableofcontents
\pagebreak

\section{Measurements}

\subsection{Radiant Flux}

The radiant flux (or power) \(\Phi\) is the total amount of energy passing
through a surface per second and is measured in \([W]\) (watts) as \(\frac{J}{s}\).

\subsection{Irradiance}

The irradiance \(E\) is the measurements of the radiant flux per \textit{unit area}
and is measured in \([W]{[M]}^{-2}\) as \(\frac{\Phi}{m^2}\).

\subsection{Radiance}

The radiance \(L\) is the irradiance per unit solid angle (steradian) and is
measured in \([W]{[M]}^{-2}{[M]}^{-2}{[sr]}^{-1}\) as \(\frac{E}{sr}\).

\section{Terminology}

% tikz picture
\begin{center}
    \begin{tikzpicture}
        \draw (-4,0) -- (4, 0);
        \draw (0,0) coordinate (a) -- (-3, 3) coordinate (b);
        \draw[->] (a) -- (0, 3) coordinate (c);
        \draw[->] (a) -- (3, 3) coordinate (d);

        \node[below] at (a) {\(x\)};
        \node[right] at (d) {\({\hat{V}}^t\)};
        \node[left] at (b) {\(\hat{V}\)};
        \node[left] at (c) {\(\hat{N}\)};
        \node[right] at (c) {\(\hat{L},\hat{R}\)};

        \pic["\({\theta}_r\)", draw=black, -, angle eccentricity=1.2, angle radius=1cm] {angle=d--a--c};
        \pic["\({\theta}_i\)", draw=black, -, angle eccentricity=1.2, angle radius=1cm] {angle=c--a--b};
    \end{tikzpicture}
\end{center}

\begin{itemize}
    \item \(\hat{V}\) direction torwards the camera
    \item \(\hat{N}\) surface normal
    \item \(\hat{L}\) vector pointing torward the light source
    \item \(\hat{R}\) reflected ray direction
    \item \({\theta}_i, {\theta}_r\) incident and reflected angles
\end{itemize}

\(\hat{R}=\hat{L}-2\hat{N}(\hat{L}\cdot\hat{N})\)

% #2 08:00

\pagebreak

\section{Rendering equation}

The rendering equation tells us how much light is exiting a \textit{surface point}
in a given direction

\pagebreak

\nocite{*} % cite all entries

\printbibliography

\end{document}

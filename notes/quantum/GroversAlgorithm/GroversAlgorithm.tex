\documentclass{article}
\usepackage{amsmath}
\usepackage{parskip}
\usepackage{fullpage}

\title{Grover's Algorithm}
\author{Paolo Bettelini}
\date{}

\begin{document}

\maketitle
\tableofcontents
\pagebreak

\section{Introduction}

Given a list of \(N\) element an item \(\omega\) with a unique properties, on average we will need to check \(\frac{N}{2}\) elements before finding \(\omega\).
This classical computation is \(O(N)\) in time complexity.

Grover's algorithm reduces this time complexity to \(O(\sqrt{N})\), meaning that if we have a list of size \(100\) it will take \(10\) steps to find \(\omega\) instead of \(50\) on average.

This quantum algorithm uses amplitude amplification of a superposition to have a near perfect probability of finding \(\omega\).

\pagebreak

\section{Algorithm}

\begin{align*}
    U_\omega|x\rangle=
    \begin{cases}
        -|x\rangle,\quad \text{if } x=\omega \\
        +|x\rangle,\quad \text{if } x\neq\omega
    \end{cases}
\end{align*}

\begin{align*}
    U_\omega=
    \begin{bmatrix}
        (-1)^{f(0)} & 0 & \cdots & 0 \\
        0 & (-1)^{f(1)} & \cdots & 0 \\
        \vdots & 0 & \ddots & \vdots \\
        0 & 0 & \cdots & (-1)^{f(2^n-1)}
    \end{bmatrix}
\end{align*}

\end{document}
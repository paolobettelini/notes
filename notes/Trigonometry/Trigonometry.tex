\documentclass[a4paper]{article}

\usepackage{amsmath}
\usepackage{amssymb}
\usepackage{parskip}
\usepackage{fullpage}
\usepackage{hyperref}

\hypersetup{
    colorlinks=true,
    linkcolor=black,
    urlcolor=blue,
    pdftitle={Trigonometry},
    pdfpagemode=FullScreen,
}

\title{Trigonometry}
\author{Paolo Bettelini}
\date{}

\begin{document}

\maketitle
\tableofcontents
\pagebreak

\section{Law of sines}

Given a triangle with sides \(a\), \(b\) and \(c\) and their respective opposite angles
\(\alpha\), \(\beta\) and \(\gamma\)
\[
    \frac{\sin(\alpha)}{a} =
    \frac{\sin(\beta)}{b} =
    \frac{\sin(\gamma)}{c}
\]

\section{Law of cosines}

Given a triangle with sides \(a\), \(b\) and \(c\)
\[
    c^2 = a^2 + b^2 - 2ab\cos\gamma
\]
where \(\gamma\) is the angle between \(a\) and \(b\) (opposite of \(c\)).

\section{Pythagorean identities}

An arbitrary angle \(\theta\) on the unit circle forms a triangle with sides
\(1\), \(\sin\theta\) and \(\cos\theta\).
\\
Give the Pythagorean theorem we have
\[
    \sin^2\theta + \cos^2\theta = 1
\]
which implies
\begin{align*}
    \sin\theta &= \pm \sqrt{1 - \cos^2\theta} \\
    \cos\theta &= \pm \sqrt{1 - \sin^2\theta}
\end{align*}

\pagebreak

\end{document}

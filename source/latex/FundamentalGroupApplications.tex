\documentclass[preview]{standalone}

\usepackage{amsmath}
\usepackage{amssymb}
\usepackage{stellar}
\usepackage{definitions}
\usepackage{bettelini}

\begin{document}

\id{fundamental-group-applications}
\genpage

\section{Applications of the fundamental group}

\subsection{Brouwer fixed point theorem}

\begin{snippettheorem}{brouwer-fixed-point-d2}{Brouwer fixed point theorem for D²}
    Every continuous map \(f \colon D^2 \fromto D^2\) has a fixed point.
\end{snippettheorem}

\begin{snippetproof}{brouwer-fixed-point-d2-proof}{brouwer-fixed-point-d2}{Brouwer fixed point}
    Suppose for contradiction that \(f(x) \neq x\) for all \(x \in D^2\).
    Define \(r \colon D^2 \fromto S^1\) by taking
    \(r(x)\) to be the intersection of the ray from \(f(x)\) through \(x\)
    with \(S^1\).
    
    This map is continuous and satisfies \(r(x) = x\) for \(x \in S^1\),
    so \(r\) is a \snippetref[retract-definition][retraction] of \(D^2\) onto \(S^1\).
    
    But then \(\pi_1(r) \colon \pi_1(D^2) \fromto \pi_1(S^1)\)
    would be surjective.
    Since \(\pi_1(D^2) = \{e\}\) (as \(D^2\) is contractible)
    and \(\pi_1(S^1) \cong \integers \neq \{e\}\),
    this is a contradiction \lightning.
\end{snippetproof}

\subsection{Fundamental theorem of algebra}

\includesnpt{fundamental-theorem-of-algebra}

\begin{snippetproof}{fundamental-theorem-algebra-proof}{fundamental-theorem-algebra}{Fundamental theorem of algebra}
    It suffices to show that \(f\) has at least one root, since if \(\alpha\) is a root
    then \(f\) is divisible by \(x - \alpha\) and \(f/(x-\alpha)\) has degree \(n-1\),
    so we can proceed by induction.
    
    Suppose for contradiction that \(f\) has no complex roots.
    Consider the function \(g_r \colon [0,1] \fromto S^1\) defined by
    \[
        g_r(s) \triangleq \frac{f(re^{2\pi i s}) / f(r)}{|f(re^{2\pi i s}) / f(r)|}
    \]
    By construction, \(g_r\) takes values in \(S^1\) and is well-defined and continuous
    since \(f\) never vanishes by hypothesis.
    
    Define \(g \colon [0,1] \cartesianprod \realnumbers \fromto S^1\) by \((s,r) \mapsto g_r(s)\).
    Then
    \[
        g_r(0) = \frac{f(r) / f(r)}{|f(r)/f(r)|} = 1 = g_r(1)
    \]
    So \(g_r\) is a closed path from \(1\) to \(1\) in \(S^1\), hence \([g_r] \in \pi_1(S^1, 1)\).
    
    We have \(g_0(s) = 1\), so \(g_0\) is the constant path at \(1\).
    For all \(r, r'\), the loops \(g_r\) and \(g_{r'}\) are homotopic with fixed endpoints
    (we can take \(g\) as the homotopy between them).
    Thus \(g_r\) is homotopic to the constant loop \(g_0\).
    
    We now choose an \(r\) for which this leads to a contradiction.
    Choose \(r_0\) large enough that \(r_0 > 1\) and \(r_0 > \sum |c_i|\)
    (the coefficients of the polynomial).
    
    For \(\|x\| = r_0\), by the triangle inequality:
    \begin{align*}
        \left|\sum c_i x^{n-i}\right|
        &\leq \sum |c_i| \cdot |x^{n-i}| \leq
        |x^{n-1}| \cdot \left(\sum_{i=1}^{n-1}|c_i|\right)
    \end{align*}
    
    So for all \(t \in [0,1]\) we have
    \[
        |x^n| > t \cdot \left|\sum_{i=1}^{n-1} c_i x^{n-i}\right|
    \]
    
    Considering
    \[
        f_t(x) \triangleq x^n + t\sum c_i x^{n-i}
    \]
    we have \(f_t(x) \neq 0\) for \(t \in [0,1]\).
    Clearly \(f(x) = f_1(x)\) and \(f_0(x) = x^n\).
    
    Substituting this into \(g_{r_0}\):
    \[
        g_{r_0,t}(s) = \frac{f_t(r_0 e^{2\pi i s}) / f_t(r_0)}{|f_t(r_0 e^{2\pi i s}) / f_t(r_0)|}
    \]
    
    For \(t=1\) we have \(f_1(x) = f(x)\) so \(g_{r_0, 1} = g_{r_0}\),
    a closed path in \(S^1\) homotopic to the constant loop.
    
    For \(t=0\) we have
    \[
        g_{r_0, 0}(s) = \frac{{r_0}^n e^{2\pi i n s} / {r_0}^n}{|{r_0}^n / {r_0}^n|} = e^{2\pi i n s}
    \]
    
    So we have found a path that winds \(n\) times around \(1\),
    i.e., its degree is \(n\), but \(g_{r_0, 0}\) and \(g_{r_0, 1}\) are homotopic
    with fixed endpoints \lightning.
\end{snippetproof}

\end{document}

\documentclass[preview]{standalone}

\usepackage{amsmath}
\usepackage{amssymb}
\usepackage{stellar}
\usepackage{definitions}
\usepackage{bettelini}

\begin{document}

\id{thermodynamics-introduction}
\genpage

\section{Thermodynamics}

\begin{snippetdefinition}{temperature-definition}{Temperature}
    \emph{Temperature} is a measure of the average kinetic energy of the particles in a system. It quantifies how hot or cold a system is relative to a reference.
\end{snippetdefinition}

\begin{snippetdefinition}{thermal-equilibrium-definition}{Thermal equilibrium}
    \emph{Thermal equilibrium} is a state in which two or
    more systems in thermal contact no longer exchange heat energy
    because they have reached the same temperature.
    In this condition, there is no net flow of thermal energy between the systems.
\end{snippetdefinition}

\plain{Thermal equilibrium is transitive.}

\begin{snippet}{thermo-tmp1}
    Consideriamo un fluido omogeneo, costituito da un'unica sostanza
    molecolare allo stato liquido o gassoso.
    All'equilibrio termico abbiamo un'equazione di stato
    \[
        f(V, p, T) = 0
    \]
    Quando due di queste variabili sono note, il valore della terza risulta univocamente definito.
    \\
    Possiamo combinare le variabili nel coefficiente di
    dilatazione termina:
    \[
        \alpha = \frac{1}{V} {\left(\frac{\partial V}{\partial T}\right)}_p
    \]
    e nel coefficiente di tensione:
    \[
        \beta = \frac{1}{p} {\left(\frac{\partial p}{\partial T}\right)}_V
    \]
    Determinano la variazione relativa del volume e della pressione corrispondente a un
    aumento unitario della temperatura. I suffissi indicano che nel processo di derivazione
    rispetto a T si mantiene costante p nel primo caso e il volume V nel secondo caso.

    Il lavoro compiuto dal gas durante una trasformazione
    da uno stato \(A\) a uno stato \(B\) è dato da
    \[
        W = \integral[A][B][p][V]
    \]
    Questo è anche l'area sottesa alla curva della trasformazione nel piano p/V di Clapeyron.
\end{snippet}

\begin{snippetdefinition}{thermodynamic cycle-definition}{Thermodynamic cycle}
    A \emph{thermodynamic cycle} is a series of processes that a system undergoes,
    returning to its initial state at the end of the cycle.
    During the cycle, the system may exchange heat and work with its surroundings.
\end{snippetdefinition}

\includesnpt[width=75\%|src=/snippet/static/thermo-cycle-illustration.png]{centered-img}

\plain{The left one clearly has a positive work, while the other has a negative work}.

\begin{snippetdefinition}{trasformazione-reversibile-definition}{Trasformazione reversibile}
    Una trasformazione viene detta \emph{reversibile} se:
    \begin{enumerate}
        \item avviene attraverso una successione
        continua di stati di equilibrio (si dice per questo motivo \emph{quasi-statica});
        \item avviene in assenza di qualsiasi forza dissipativa.
    \end{enumerate}
\end{snippetdefinition}

\section{Tipi di trasformazione}

\subsection{Trasformazione isobarica}

\includesnpt{trasformazione-isobarica-illustration}

\subsection{Trasformazione isocòra}

\includesnpt{trasformazione-isocora-illustration}

\subsection{Trasformazione isotermica}

\includesnpt{trasformazione-isotermica-illustration}

\section{Ideal gasses}

\begin{snippetdefinition}{ideal-gas-definition}{Ideal gas}
    An \emph{ideal gas} is a theoretical gas composed of many randomly moving point
    particles that do not interact with each other, except through elastic collisions.
    An idea gas follows the \emph{ideal gas law}, which is the equation of state
    \[
        pV = nRT
    \]
    where
    \begin{itemize}
        \item \(p\): pressure;
        \item \(V\): volume;
        \item \(n\): amount of moles;
        \item \(R\): universal gas constant \(8.314 \frac{J}{\text{mol} \cdot K}\);
        \item \(T\): temperature.
    \end{itemize}
\end{snippetdefinition}

\section{Dilatazione termica}

\begin{snippet}{thermo-temp-2}
    Partendo da
    \[
        \alpha = \frac{1}{V} {\left(\frac{\partial V}{\partial T}\right)}_p
    \]
    Integrando abbiamo
    \[
        \integral[T_1][T_f][\alpha][T] = \integral[V_i][V_f][\frac{1}{V}][V]
    \]
    Se \(\alpha\) non dipende dalla temperatura otteniamo
    \[
        V_f = V_i e^{\alpha(T_f - T_i)}
    \]
\end{snippet}

\begin{snippetdefinition}{dilatazione-solido-definition}{Dilatazione solido}
    La dilatazione di un oggetto in una direzione \(\Delta l\) in funzione del
    cambio di temperatura \(\Delta T\) è proporzionale e data da
    \[
        \Delta l = l_0 \cdot \alpha \cdot \Delta T
    \]
    dove \(l_0\) è la lunghezza iniziale e \(\alpha\) è il coefficiente di dilatazione lineare
    \((\frac{1}{K})\). Questo funziona solamente per certi intervalli di temperatura,
    ossia il solido non deve cambiare stato.
    
    Un solido si dilata in tutte le direzioni.
    La dilatazione dell'area o del volume di un solito possono essere \textit{approssimate}
    nella seguente maniera
    \begin{align*}
        \Delta A &\approx 2 \cdot A_0 \cdot \alpha \cdot \Delta T \\
        \Delta V &\approx 3 \cdot V_0 \cdot \alpha \cdot \Delta T
    \end{align*}
\end{snippetdefinition}

\begin{snippetdefinition}{dilatazione-volumetrica-definition}{Dilatazione volumetrica}
    Un liquido/gas, non avendo forma propria, ha una dilatazione che può essere
    quantificata solo in termini volumetrici
    \[
        \Delta V = V_0 \cdot \gamma \cdot \Delta T
    \]
    dove \(\gamma\) è il coefficiente di dilatazione cubica.
\end{snippetdefinition}

\begin{snippettheorem}{avogadro-law}{Avogadro's law}
    Avogadro's law states that equal volumes of all gases, at the same temperature and pressure, have
    the same number of molecules.
    \[
        \frac{V_1}{n_1} = \frac{V_2}{n_2}
    \]
\end{snippettheorem}

\section{Calore}

\begin{snippetdefinition}{asddsaioid-definition}{Capacità terminca}
    \todo
\end{snippetdefinition}

% estensiva
% intensiva

% viene divisa per la massa per farla diventare intensiva
\begin{snippetdefinition}{asddsaioidff-definition}{Calore specifico}
    \todo
\end{snippetdefinition}

\end{document}
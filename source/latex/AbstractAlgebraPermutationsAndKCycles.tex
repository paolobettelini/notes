\documentclass[preview]{standalone}

\usepackage{amsmath}
\usepackage{amssymb}
\usepackage{stellar}
\usepackage{definitions}

\begin{document}

\id{permutations-and-k-cycles}
\genpage

\section{Permutations}

\begin{snippet}{permutation-notation-example}
    Let \(\sigma \in \text{Sym}_n\), we can explicit the permutation as
    \[
        \sigma \triangleq \begin{pmatrix}
            1 & 2 & \cdots & n \\
            \sigma(1) & \sigma(2) & \cdots & \sigma(n)
        \end{pmatrix}
    \]
    where \(1,2,\cdots, n\) are the letters.
\end{snippet}

\begin{snippetproposition}{permutation-group-non-abelian}{}
    The \group \(\permgrp_n\) is an \abeliangroup \ifandonlyif \(n<3\).
\end{snippetproposition}

\begin{snippetproof}{permutation-group-non-abelian-proof}{permutation-group-non-abelian}{}
    If \(n=1\), the case is trivial. If \(n=2\), the permutation can only exchange two letters.
    For \(n\geq 3\), consider
    \[
        \sigma \triangleq \begin{pmatrix}
            1 & 2 & 3 & \cdots & n \\
            2 & 1 & 3 & \cdots & n
        \end{pmatrix}
    \]
    and
    \[
        \tau \triangleq \begin{pmatrix}
            1 & 2 & 3 & 4 & \cdots & n \\
            3 & 2 & 1 & 4 & \cdots & n
        \end{pmatrix}
    \]
    Then,
    \begin{align*}
        \tau(\sigma) &= \begin{pmatrix}
            1 & 2 & 3 & 4 & \cdots & n \\
            2 & 3 & 1 & 4 & \cdots & n
        \end{pmatrix} \\
        \sigma(\tau) &= \begin{pmatrix}
            1 & 2 & 3 & 4 & \cdots & n \\
            3 & 1 & 2 & 4 & \cdots & n
        \end{pmatrix}
    \end{align*}
\end{snippetproof}

\begin{snippetdefinition}{permutation-support-definition}{Support of a permutation}
    Let \(\sigma \in \permgrp_n\). Then, the \emph{support of \(\sigma\)}
    is defined as
    \[
        \text{Supp}(\sigma) \triangleq \{ i \in \{1,2,\cdots,n\} \suchthat \sigma(i) \neq i \}
    \]
\end{snippetdefinition}

\plain{The support is the empty set if and only if the permutation is the identity function.
Indeed, if the permutation is not the identity function, then its cycle is greater or equal than 2.}

\begin{snippetproposition}{permutation-supp-closure}{}
    Let \(\sigma \in \permgrp_n\).
    Then,
    \[
        i \in \supp(\sigma) \implies \sigma(i) \in \supp(\sigma)
    \]
\end{snippetproposition}

\begin{snippetproof}{permutation-supp-closure-proof}{permutation-supp-closure}{}
    By the hypothesis, we have \(i \neq \sigma(i)\).
    We need to show that \(\sigma(i) \neq \sigma(\sigma(i))\).
    However, this is true since \(\sigma\) is \injective.
\end{snippetproof}

\begin{snippetproposition}{properties-of-supp}{}
    Let \(\sigma, \tau \in \permgrp_n\). Then:
    \begin{enumerate}
        \item \(\supp(\tau(\sigma)) \subseteq \supp(\sigma) \union \supp(\tau)\);
        \item \(\supp(\sigma^{-1}) = \supp(\sigma)\).
    \end{enumerate}
\end{snippetproposition}

\begin{snippetproof}{properties-of-supp-proof}{properties-of-supp}{}
    \begin{enumerate}
        \item Let \(i\in\supp(\tau(\sigma))\) (i.e. \(\tau(\sigma(i)) \neq i\)).
            We need to show that \(i\in\supp(\sigma)\) (meaning \(\sigma(i) \neq i\)) or
            \(i\in \supp(\tau)\) (meaning \(\tau(i) \neq i\)).
            Assume that this is not true, meaning \(i\notin\supp(\sigma) \land i\notin\supp(\tau)\).
            Thus, \(\sigma(i)=i\) and \(\tau(i)=i\) implying \(\tau(\sigma(i)) = \tau(i) = i\),
            which is a contradiction \lightning.
        \item Let \(i\in\supp(\sigma)\), meaning \(\sigma(i) \neq i\) or \(i\neq \sigma^\inversefunction(i)\)
            which also means \(i\in\supp(\sigma^\inversefunction)\).
            Thus, \(\supp(\sigma)\subseteq \supp(\sigma^\inversefunction)\).
            The reverse inclusion is obtained by swapping \(\sigma\) and \(\sigma^\inversefunction\).
    \end{enumerate}
\end{snippetproof}

\begin{snippetdefinition}{disjoint-permutation-definition}{Disjoint permutations}
    Let \(\sigma, \tau \in \permgrp_n\). Then, \(\sigma\) and \(\tau\)
    are said to be \emph{disjoint permutations} if \(\supp(\sigma)\) and \(\supp(\tau)\)
    are \disjoint.
\end{snippetdefinition}

\begin{snippetproposition}{disjoint-permutations-commute}{Disjoint permutations commute}
    Let \(\sigma, \tau \in \permgrp_n\) where \(\sigma\) and \(\tau\) are \disjointperm. Then,
    \(\sigma(\tau) = \tau(\sigma)\).
\end{snippetproposition}

\begin{snippetproof}{disjoint-permutations-commute-proof}{disjoint-permutations-commute}{Disjoint permutations commute}
    We need to show that for every \(i\in\{1,2,\cdots. n\}\), we have \(\tau(\sigma(i)) = \sigma(\tau(i))\).
    Consider the cases:
    \begin{enumerate}
        \item \(i\in\supp(\sigma) \land i\notin\supp(\tau)\): since \(i\in\supp(\sigma)\)
            we have \(\sigma(i)\in\supp(\sigma)\) and, thus, \(\sigma(i)\notin\supp(\tau)\),
            meaning \(\tau(\sigma(i)) \neq \sigma(i)\).
            On the other hand, \(\tau(i) = i\) and, thus, \(\tau(\sigma(i)) = \sigma(i)\).
            Thus, \(\tau(\sigma(i)) = \sigma(i) = \sigma(\tau(i))\);
        \item \(i\notin\supp(\sigma) \land i\in\supp(\tau)\): analogous to the last case;
        \item \(i\notin\supp(\sigma) \land i\notin\supp(\tau)\): we have that \(\tau(\sigma(i)) = \tau(i) = i\)
            and \(\sigma(\tau(i))=\sigma(i)=i\).
    \end{enumerate}
\end{snippetproof}

\begin{snippetproposition}{disjoint-permutation-period}{Period of disjoint permutations}
    Let \(\sigma, \tau \in \permgrp_n\) where \(\sigma\) and \(\tau\) are \disjointperm.
    Then,
    \[
        |\tau(\sigma)| = \text{lcm}(|\tau|, |\sigma|)
    \]
\end{snippetproposition}

\begin{snippetproof}{disjoint-permutation-period-proof}{disjoint-permutation-period}{Period of disjoint permutations}
    Let \(|\sigma| = m\), \(|\tau| = n\) and \(t = \text{lmc}(m,n)\).
    Since the permutations are \disjointperm, we know that the commute and thus
    \({(\tau(\sigma))}^r = \tau^r(\sigma^r)\) for every \(r\in\integers\).
    In particular, \[{(\tau(\sigma))}^t = \tau^t(\sigma^t) = \text{Id}(\text{Id}) = \text{Id}\].
    This means that \(|\tau(\sigma)|\) is a divisor of \(t\).
    Let \(d=|\tau(\sigma)|\). We need to show that \(d=t\).
    In order to do this, we show that \(d\) is a multiple of \(m\) and \(n\).
    We ave that \({(\tau(\sigma))}^d = \text{Id}\). Then, \(\tau^d(\sigma^d) = \text{Id}\),
    from which \(\sigma^d = \tau^{-d}\). Now, \(\supp(\sigma^d) \subseteq \supp(\sigma)\)
    and likewise \(\supp(\tau^{-d}) \subseteq \supp(\tau)\).
    Since \(\sigma^d = \tau^{-d}\), obviously \(\supp(\sigma^d) = \supp(\tau^{-d})\),
    but \[\supp(\sigma^d) \intersection \supp(\tau^{-d}) \subseteq \supp(\sigma) \intersection \supp(\tau) = \emptyset\]
    Therefore, \(\supp(\sigma^d) = \supp(\tau^{-d}) = \emptyset\), meaning \(\sigma^d = \text{Id}\)
    and \(\tau^{-d} = \text{Id}\), from which it follows that \(d\) is a multiple of \(|\sigma|\)
    and \(-d\) is a multiple of \(|\tau|\) (and thus multiple of \(m\) and \(n\)).
\end{snippetproof}

\begin{snippetdefinition}{k-cycle-definition}{\(k\)-cycle}
    Let \(\sigma \in \permgrp_n\). Then, \(\sigma\) is said to be a \emph{\(k\)-cycle}
    if there exist \(a_0, a_1, \cdots, a_k\) distinct letters
    such that \[
        \bigwedge\limits_{i=0}^{k-1} \sigma(a_k) = a_{k+1 \bmod{k}}
    \]
    and \(\sigma(i) = i\) for all \(i\notin \{a_0, a_1, \cdots, a_k\}\).
    \emph{Notation:} such permutation is denoted \((a_1 a_2 \cdots a_k)\).
\end{snippetdefinition}

\plain{Any k-cycle can be written in k different ways.}

\end{document}
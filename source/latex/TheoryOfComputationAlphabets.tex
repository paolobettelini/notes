\documentclass[preview]{standalone}

\usepackage{amsmath}
\usepackage{amssymb}
\usepackage{tikz}
\usepackage{wrapfig}
\usepackage{bettelini}
\usepackage{stellar}
\usepackage{definitions}

% =======
\usetikzlibrary{ % tikz packages
    automata,positioning,
    arrows.meta,bending
}
\tikzset{every state/.style={
    inner sep=2pt,
    minimum size=4pt
}}
\tikzset{>=stealth}  %latex, to, stealth
% Empty string symbol.
\newcommand{\emptyString}{\lambda}
% =======

\begin{document}

\id{theoryofcomputation-alphabets}
\genpage

\section{Alphabets and Languages}

\begin{snippetdefinition}{alphabet-definition}{Alphabet}
    An \textit{alphabet} is a finite set of \textit{symbols},
    including an empty string denoted \(\emptyString\).
    The length of a string \(w\) is denoted as \(|w|\).
\end{snippetdefinition}

\begin{snippet}{alphabet-stuff1}
    Note that \(\emptyString \neq \varnothing \neq \{\emptyString\}\).

    If \(\Sigma\) is an alphabet,
    \[
        \Sigma_\emptyString = \emptyString \union \Sigma
    \]

    A set of strings is called a \textit{language}.

    The set \(\{0,1\}\) is the binary set.
\end{snippet}

\section{Operations}

\begin{snippetdefinition}{language-concatenation-definition}{Concatenation}
    If \(A\) and \(B\) are two languages over the same alphabet,
    the \textit{concatenation} of \(A\) and \(B\) is defined as
    \[
        AB = \{ab \suchthat a \in A \land b \in B\}
    \]
\end{snippetdefinition}

\begin{snippetdefinition}{kleene-star-operator-definition}{Kleene star operator}
    The \textit{kleene star operator} can be applied to alphabets or languages.
    It represent the union of all \(n\)-permutations of the set.
    If \(\Sigma\) is an alphabet, \(\Sigma^*\) is the set of all strings over \(\Sigma\).
    \[
        \Sigma^* = \emptyString \cup \bigcup_{n\in\naturalnumbers} \Sigma^n
    \]
\end{snippetdefinition}

\plain{The kleene star of the alphabet with 0 and 1 is the set of all binary strings.}

\end{document}
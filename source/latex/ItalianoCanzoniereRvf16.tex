\documentclass[preview]{standalone}

\usepackage{amsmath}
\usepackage{amssymb}
\usepackage{stellar}
\usepackage{definitions}
\usepackage{bettelini}

\begin{document}

\id{italiano-canzoniere-rvf-16}
\genpage

\section{Rvf 16: Movesi il vecchierel canuto et biancho}

\begin{snippet}{canzoniere-rvf-16-parte1}
    La poesia è un sonetto con schema delle rime ABBA ABBA CDE CDE.
    \\\\
    \StellarPoetry{1}{
        Movesi il vecchierel \textbf{canuto et biancho} \\
        del dolce loco ov'à sua età \textbf{fornita} \\
        et da la famigliuola sbigottita \\
        che vede il caro padre \textbf{venir manco};
    }{
        Il vecchietto canuto e imbiancato dall'età si separa
        \bfslash dal suo tranquillo paese natale, che lo ha nutrito per l'intero numero dei suoi anni (dove ha trascorso tutta la vita),
        \bfslash e dalla sua famiglia, commossa
        \bfslash che vede il caro padre partire;
    }
    \StellarPoetry{5}{
        indi trahendo poi l'antiquo fianco \\
        per l'extreme giornate di sua vita, \\
        quanto piú pò, col buon voler s'aita, \\
        rotto dagli anni, et dal camino stanco;
    }{
        da lì, trascinando a fatica le vecchie membra (\quotes{l'antiquo fianco})
        \bfslash attraverso le ultime giornate della sua vita,
        \bfslash si fa coraggio quanto più può con la buona volontà,
        \bfslash sfiancato dagli acciacchi dell'età e affaticato dal cammino;
    }
    \StellarPoetry{9}{
        et viene a Roma, seguendo 'l desio, \\
        per mirar la sembianza di colui \\
        ch'ancor lassú nel ciel vedere spera:
    }{
        e viene a Roma, seguendo il suo desiderio
        \bfslash di vedere il cosiddetto velo della Veronica, l'immagine di Cristo
        \bfslash che spera di poter vedere di nuovo in cielo, dopo la morte (\quotes{ancor}):
    }
    \StellarPoetry{12}{
        \textbf{così}, lasso, talor vo cerchand'io, \\
        donna, quanto è possibile, in altrui \\
        la disïata vostra forma vera.
    }{
        allo stesso modo, a volte cerco io, sfinito,
        \bfslash Laura, per quanto possibile, in altre donne
        \bfslash il vostro desiderato volto in carne e ossa.
    }

    % caduta e biancho sono una ditologia sinonimica (sono due sinonimi)
    Vi è una sproporzionata similitudine fra i primi undici versi e gli ultimi tre.
    Questa similitudine delinea quindi una separazione del testo.

    La prima parte può essere letta come una storia: un vecchietto si allontana
    dalla sua città dove è cresciuto e dalla sua famiglia, la quale ne rimane stupita.
    Il termine fornita può implicare che la sua vita sia ormai compiuta.
    Infatti, la famiglia è sbigottita perché è preoccupata che non torni più (venir manco).
    Questa partenza è connotata da un sentimento di affetto (vezzeggiativi vecchierel e famigliuola, dolce, caro).
    Infatti, dolce e caro vengono valorizzati dall'inversione data dal fatto che siano prima del nome.

    Da quel punto, il vecchio si trascina a fatica nelle sue ultime giornate,
    spinto dalla sua volontà, sfaticato dall'età e dal cammino.
    Il viaggio sta arrivando alla fine come la sua vita sta terminando.\\
    Vi è un chiasmo con rotto: \textbf{rotto}, \textbf{anni}, \textbf{cammino}, \textbf{stanco},
    dove gli aggettivi della fatica sono agli estremi.

    Seguendo il desiderio, viene a Roma per ammirare la sembianza di Cristo (del Velo della Veronica),
    che spera di poter vedere di nuovo in cielo, dopo la morte.
\end{snippet}

\begin{snippetdefinition}{il-velo-della-veronica-definition}{Il Velo della Veronica}
    Il \textit{Velo della Veronica} è un panno usato da Gesù per asciugarsi le lacrime e il sudore
    durante la Via Crucis.
    Si dice che il panno abbia impressa la faccia di Gesù. Il panno prende il nome della donna che lo aiutò ad alzarsi.
\end{snippetdefinition}

\begin{snippet}{canzoniere-rvf-16-parte2}
    % lasso = ahimè
    Allo stesso modo, io (scrittore) come il vecchietto, cerco nelle altre donne la vostra forma perfetta
    (la forma di Laura).
    Questa similitudine compara le sembianze di Laura e quelle di Cristo, e le donne con Santa Veronica.
    Come il vecchierello è stanco, anche Petrarca è sfiatato (stanco - lasso).
    Ciò che entrambi hanno in comune è l'idea di \textbf{ricerca} e di \textbf{lontananza}.

    Gli elementi che hanno fatto discutere sono
    \begin{itemize}
        \item la similitudine quasi blasfema, dove per rigor di logica Laura viene paragonata a Dio;
        \item la fortissima sproporzione del sonetto.
    \end{itemize}

    % Dante, Par. XXXI, vv. 103-111
\end{snippet}

\end{document}
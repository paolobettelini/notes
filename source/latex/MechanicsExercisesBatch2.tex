\documentclass[preview]{standalone}

\usepackage{amsmath}
\usepackage{amssymb}
\usepackage{stellar}
\usepackage{definitions}
\usepackage{bettelini}

\begin{document}

\id{mechanics-ex-2}
\genpage

\section{Exercises - Batch 2}

\begin{snippetexercise}{mechanics-ex-2.1}{\underline{2.1}}
    An observer drops a stone into a well to measure its depth \(h\).
    The time interval between the initial moment and the instant the sound of the stone hitting the bottom
    is heard is \(\Delta t\). How much is \(h\)? (Take into account the speed of sound).
\end{snippetexercise}

\begin{snippetsolution}{mechanics-ex-2.1-sol}{\underline{2.1}}
    \todo
\end{snippetsolution}

\begin{snippetexercise}{mechanics-ex-2.2}{\underline{2.2}}
    A projectile is fired at a target initially placed at a height \(h\) and dropped simultaneously with the shot.
    Show that the condition for the projectile to hit the target is that it is initially aimed at the target itself.
\end{snippetexercise}

\begin{snippetsolution}{mechanics-ex-2.2-sol}{\underline{2.2}}
    \todo
\end{snippetsolution}

\begin{snippetexercise}{mechanics-ex-2.3}{\underline{2.3}}
    A material point moves along a circular arc of radius \(R\) with the following equation of motion:
    \[
        s = s_0 \cos \omega t,
    \]
    where \(s\) is the curvilinear abscissa and \(s_0\) and \(\omega\) are given constants.
    Find the angular velocity and the normal and tangential components of the acceleration.
\end{snippetexercise}

\begin{snippetsolution}{mechanics-ex-2.3-sol}{\underline{2.3}}
    \todo
\end{snippetsolution}

\begin{snippetexercise}{mechanics-ex-2.4}{\underline{2.4}}
    Two airplanes \(A\) and \(B\) have opposite velocities of magnitude \(v\) and their trajectories are two
    parallel straight lines separated by a distance \(d\).
    Let \(t=0\) be the instant when the line \(AB\) is perpendicular to the trajectories.
    The axis of the cannon mounted on \(A\) forms an angle \(\alpha\) with the axis of the airplane,
    and the projectiles are fired with a velocity of magnitude \(v_R\) relative to \(A\).
    At what instant \(t^*\) should airplane \(A\) fire to hit airplane \(B\)?
    [Do not consider gravitational acceleration].
\end{snippetexercise}

\begin{snippetsolution}{mechanics-ex-2.4-sol}{\underline{2.4}}
    \todo
\end{snippetsolution}

\begin{snippetexercise}{mechanics-ex-2.5}{\underline{2.5}}
    A car starts from rest with uniformly accelerated motion with acceleration \(a\).
    After a time \(\tau\), a projectile is launched, assumed to move with constant velocity \(v_0\).
    Determine the minimum velocity \(v_0\) required to hit the car, as a function of \(a\) and \(\tau\).
    The motion can be considered purely one-dimensional.
\end{snippetexercise}

\begin{snippetsolution}{mechanics-ex-2.5-sol}{\underline{2.5}}
    \todo
\end{snippetsolution}

\begin{snippetexercise}{mechanics-ex-2.6}{\underline{2.6}}
    A train moving in uniform rectilinear motion with velocity of magnitude \(v_0\) suddenly decelerates with
    constant deceleration of magnitude \(A\); as a consequence, a suitcase, precariously placed on the luggage rack,
    falls and lands on the floor of the train.
    Determine the trajectory of the suitcase as seen by an observer \(O\) stationary on the ground and by an
    observer \(O'\) on the train.
\end{snippetexercise}

\begin{snippetsolution}{mechanics-ex-2.6-sol}{\underline{2.6}}
    \todo
\end{snippetsolution}

\end{document}
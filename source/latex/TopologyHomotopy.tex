\documentclass[preview]{standalone}

\usepackage{amsmath}
\usepackage{amssymb}
\usepackage{stellar}
\usepackage{definitions}
\usepackage{bettelini}

\begin{document}

\id{topology-homotopy}
\genpage

\section{Homotopy}

\begin{snippetdefinition}{homotopy-definition}{Homotopy}
    Let \(X, Y\) be \topologicalspace[topological spaces]
    and \(f_0, f_1 \colon X \fromto Y\) \topologycontinuous[continuous maps]
    between \(X\) and \(Y\). Then, \(f_0\) and \(f_1\) are said to be
    \emph{homotopic} if there exists a \topologycontinuous[continuous map]
    \[
       H \colon X \cartesianprod [0,1] \fromto Y
    \]
    such that:
    \begin{enumerate}
        \item \(H(x, 0) = f_0(x)\);
        \item \(H(x, 1) = f_1(x)\)
    \end{enumerate}
    for all \(x\in X\).
    Such a map is said to be a \emph{homotopy} from \(f_0\) to \(f_1\).
    We write \(f_0 \simeq f_1\).
\end{snippetdefinition}

\begin{snippetproposition}{homotopy-is-equivalence-relation}{Homotopy is equivalence relation}
    Let \(X, Y\) be \topologicalspace[topological spaces].
    The relation ``being homotopic'' on continuous maps \(X \fromto Y\)
    is an equivalence relation.
\end{snippetproposition}

\begin{snippetproof}{homotopy-is-equivalence-relation-proof}{homotopy-is-equivalence-relation}{Homotopy is equivalence}
    \begin{enumerate}
        \item \emph{Reflexive:} \(f \simeq f\) via the constant homotopy \(H(x,t) = f(x)\).
        \item \emph{Symmetric:} If \(f \simeq g\) via \(H(x,t)\),
            then \(g \simeq f\) via \(G(x,t) = H(x, 1-t)\).
        \item \emph{Transitive:} If \(f \simeq g\) via \(H\)
            and \(g \simeq h\) via \(K\), then \(f \simeq h\) via
            \[
                L(x, t) = \begin{cases}
                    H(x, 2t) & 0 \leq t \leq \frac{1}{2} \\
                    K(x, 2t-1) & \frac{1}{2} \leq t \leq 1
                \end{cases}
            \]
    \end{enumerate}
\end{snippetproof}

\begin{snippetlemma}{homotopic-maps-compose}{Composition of homotopic maps}
    Let \(f_0, f_1 \colon X \fromto Y\) and \(g_0, g_1 \colon Y \fromto Z\)
    be continuous maps with \(f_0 \simeq f_1\) and \(g_0 \simeq g_1\).
    Then \(g_0 \circ f_0 \simeq g_1 \circ f_1\).
\end{snippetlemma}

\begin{snippetdefinition}{homotopy-equivalence-definition}{Homotopy equivalence}
    A \topologycontinuous[continuous map] \(f \colon X \fromto Y\)
    is called a \emph{homotopy equivalence} if there exists
    a continuous map \(g \colon Y \fromto X\) such that
    \(f \circ g \simeq \identity_Y\) and \(g \circ f \simeq \identity_X\).
    
    Two \topologicalspace[topological spaces] are said to be
    \emph{homotopy equivalent} (or have the same \emph{homotopy type})
    if there exists a homotopy equivalence between them.
\end{snippetdefinition}

\begin{snippetdefinition}{contractible-space-definition}{Contractible space}
    A \topologicalspace \(X\) is said to be \emph{contractible}
    if it is homotopy equivalent to a single point.
\end{snippetdefinition}

\plain{Note that any convex non-empty subset of the reals is contractible.}

\subsection{Gluing of continuous maps}

\begin{snippetproposition}{gluing-lemma}{Gluing lemma}
    Let \(X, Y\) be \topologicalspace[topological spaces]
    and let \(\{A_i\}_{i \in I}\) be a family of subsets of \(X\) with \(X = \bigcup_{i \in I} A_i\).
    Let \(f_i \colon A_i \fromto Y\) be continuous maps such that
    for all \(i, j \in I\),
    \[
        \restr{f_i}{A_i \intersection A_j} = \restr{f_j}{A_i \intersection A_j}
    \]
    Then the map \(f \colon X \fromto Y\) defined by \(f(x) = f_i(x)\) for \(x \in A_i\)
    is well-defined and continuous if:
    \begin{enumerate}
        \item all \(A_i\) are \openset[open] in \(X\); or
        \item \(I\) is finite and all \(A_i\) are \closedset[closed] in \(X\).
    \end{enumerate}
\end{snippetproposition}

\end{document}
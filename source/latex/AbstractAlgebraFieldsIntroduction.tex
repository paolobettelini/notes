\documentclass[preview]{standalone}

\usepackage{amsmath}
\usepackage{amssymb}
\usepackage{stellar}
\usepackage{definitions}

\begin{document}

\id{fields-introduction}
\genpage

\section{Definition}

\begin{snippetdefinition}{field-definition}{Field}
    A \textit{field} \((F, +, \circ)\) is a triple containing a \set \(F\) and two \binoperation[binary operations]
    \(+\) and \(\circ\) on \(F\) such that:
    \begin{enumerate}
        \item \((F, +)\) is an \abeliangroup;
        \item \((F \difference \{0\}, \circ)\) is an \abeliangroup;
        \item \textit{distributivity of multiplication over addition}:
            \(\forall a,b,c \in F, a\circ (b+c) = a\circ b + a\circ c\).
    \end{enumerate}
\end{snippetdefinition}

\plain{This is a ring where each element has a multiplicative inverse except zero.}

\begin{snippetdefinition}{totally-ordered-field-definition}{Totally ordered field}
    A \field \((F, +, \cdot)\) together with a \totalorder \(\leq\) on \(R\) is a
    \textit{totally ordered field} if it satisfies the following properties:
    \begin{enumerate}
        \item \(\forall a,b,c \in F, a \leq b \implies a+c \leq b+c\);
        \item \(\forall a,b,c \in F, 0 \leq a \land 0 \leq b \implies 0 \leq a \cdot b\).
    \end{enumerate}
\end{snippetdefinition}

\begin{snippetdefinition}{field-homomorphism-definition}{Field homomorphism}
    Let \((F, +, \cdot)\) and \((K, \oplus, \otimes)\) be \field[fields].
    A \function \(\varphi\colon F \fromto K\)
    such that:
    \begin{enumerate}
        \item \emph{additive morphism property:} \(\forall a,b \in F, \varphi(a + b) = \varphi(a) \oplus \varphi(b)\)
        \item \emph{multiplicative morphism property:} \(\forall a,b \in F, \varphi(a \cdot b) = \varphi(a) \otimes \varphi(b)\)
        \item \emph{unity preservation:} \(\varphi(1_F) = 1_K\)
    \end{enumerate}
    is said to be a \emph{field homomorphism}.
\end{snippetdefinition}

\begin{snippetdefinition}{field-isomorphism-definition}{Field isomorphism}
    Let \((F, +, \cdot)\) and \((K, \oplus, \otimes)\) be \field[fields] and let
    \(\varphi\colon F \fromto K\) be a \fieldhomomorphism.
    Then, \(\varphi\) is said to be a \emph{field isomorphism} if it is \bijective.
    \[
        (F, +, \cdot) \cong (K, \oplus, \otimes)
    \]
\end{snippetdefinition}

\end{document}
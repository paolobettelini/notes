\documentclass[preview]{standalone}

\usepackage{amsmath}
\usepackage{amssymb}
\usepackage{stellar}
\usepackage{bettelini}

\hypersetup{
    colorlinks=true,
    linkcolor=black,
    urlcolor=blue,
    pdftitle={Stellar},
    pdfpagemode=FullScreen,
}

\begin{document}

\title{Geografia economica}
\id{geoeconomica-secolo-breve}
\genpage

\section{Il secolo breve}

\begin{snippetdefinition}{secolo-definizione}{Secolo}
    Un \textit{secolo} può essere definito in maniera stretta come 100 anni,
    oppure come un periodo di circa 100 anni con caratteristiche omogenee.
\end{snippetdefinition}

\begin{snippetdefinition}{secolo-breve-definizione}{Secolo breve}
    Con il termine il \textit{secolo breve} ci si riferisce al novecento.
    Esso inizia con la Prima Guerra Mondiale (1914) e termina con la fine della guerra fredda.
    Questa secolo è caratterizzato da una serie di avvenimenti che hanno avuto conseguenze molto importanti
    per la società odienrna, in particolare le guerre mondiali.
\end{snippetdefinition}

\begin{snippet}{hobsbawn-elementi-novecento}
    Secondo l'autore E. J. Hobsbawn, il secolo breve da lui ampiamente studiato inizia nell'anno
    1914, con lo scoppio della Prima guerra mondiale, e termina nel 1991, con la fine della
    Guerra fredda e il crollo dell'Unione Sovietica.\\
    Questo periodo racchiude uno dei periodi fondamentali della recente storia dell'umanità e 
    rappresentano fasi di passaggio molto rapide ma allo stesso tempo molto violente.
    \\\\
    Il secolo breve di Hobsbawm si suddivide in tre periodi chiave:
    \begin{itemize}
        \item \textbf{Età della catastrofe (1914-1945)}: i due conflitti mondiali in un'unica Guerra dei trent'anni.
            La Prima guerra mondiale segna la fine della società ottocentesca e la definitiva
            dissoluzione degli imperi millenari;
        \item \textbf{Età dell'oro (1946-1973):} la decolonizzazione pone fine agli ultimi imperi.
    
            È l'epoca del Boom e si affronta un bipolarismo delle due potenze mondiali: Capitalismo vs Comunismo.
            {\footnotesize Capitalismo: Società basata sull'acquisizione del capitale, ossia gli
            averi dei cittadini e delle aziende. (USA).}\\
            {\footnotesize Comunismo: Società basata sulla condivisione del capitale (URSS).}
    
            Nel 1973 finisce la crescita economica, la quale pone fine all'età dell'oro.
        \item \textbf{Età della crisi (1973-1991)}: inizia la globalizzazione ed il potere economico
            è sempre più nelle mani di Stati Uniti d'America e Giappone.\\
            Gli eventi che portarono alla crisi sono riportati di seguito:
            \begin{itemize}
                \item 1975: crisi petrolifera che causò una grande inflazione sul prezzo della benzina e dei gas;
                \item 1989: crollo del muro di Berlino che segnò la fine della divisione tra
                    Est e Ovest in Germania.
                \item 1991: definitiva dissoluzione dell'Unione Sovietica (URSS).
            \end{itemize}
    \end{itemize}
    Secondo Hobsbawm vi sono tre elementi chiave del Novecento:
    \begin{itemize}
        \item \textbf{La fine dell'eurocentrismo}:
            fine della tendenza a considerare l'Europa e i valori europei come il centro o la norma
            in vari ambiti globali come la storia, la cultura, la politica e l'economia;
        \item \textbf{Il carattere sempre più unitario del mondo}:
            intensificazione della globalizzazione e dell'indipendenza tra le nazioni;
        \item \textbf{La disintegrazione dei vecchi modelli di relazioni umane e sociali e la rottura
            dei legami tra le generazioni, specialmente nei paesi avanzati}:
            deterioramento delle tradizionali strutture familiari e comunitarie causato dai 
            cambiamenti sociali, economici e tecnologici, i quali portano una maggiore individualizzazione.
    \end{itemize}
\end{snippet}

\end{document}
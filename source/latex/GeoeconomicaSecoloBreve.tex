\documentclass[preview]{standalone}

\usepackage{amsmath}
\usepackage{amssymb}
\usepackage{stellar}
\usepackage{bettelini}

\hypersetup{
    colorlinks=true,
    linkcolor=black,
    urlcolor=blue,
    pdftitle={Stellar},
    pdfpagemode=FullScreen,
}

\begin{document}

\title{Geografia economica}
\id{geoeconomica-secolo-breve}
\genpage

\begin{snippetdefinition}{secolo-definizione}{Secolo}
    Un \textit{secolo} può essere definito in mainera stretta come 100 anni,
    oppure come un periodo di circa 100 anni con caratteristiche omogenee.
\end{snippetdefinition}

\begin{snippetdefinition}{secolo-breve-definizione}{Secolo breve}
    Con il termine il \textit{secolo breve} ci si riferisce al novecento.
    Esso inizia con la Prima Guerra Mondiale (1914) e termina con la fine della guerra fredda.
    Questa secolo è caratterizzato da una serie di avvenimenti che hanno avuto conseguenze molto importanti
    per la società odienrna, in particolare le guerre mondiali.
\end{snippetdefinition}

\begin{snippet}{hobsbawn-elementi-novecento}
    Secondo Hobsbawm vi sono tre elementi chiave del Novecento:
    \begin{itemize}
        \item la fine dell'eurocentrismo;
        \item l'unitarietà del mondo;
        \item la disintegrazione dei vecchi modelli di relazioni umane.
    \end{itemize}
\end{snippet}

\end{document}
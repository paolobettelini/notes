\documentclass[preview]{standalone}

\usepackage{amsmath}
\usepackage{amssymb}
\usepackage{stellar}
\usepackage{definitions}

\begin{document}

\id{matrices-systems-of-linear-equations}
\genpage


\section{Elementary row operations}

\includesnpt{linearalgebra-elementary-row-operations}

\section{Systems of linear equations}

\begin{snippetproposition}{trivial-system-of-linear-equations-solution}{Trivial system of linear equations}
    Let \(a,b \in \realnumbers\). Then, the solution to
    \[
        ax = b
    \]
    is given by
    \[
        \begin{cases}
            x = \frac{b}{a} & a \neq 0 \\
            x \in \emptyset & a = 0 \land b \neq 0 \\
            x \in \realnumbers & a = 0 \land b = 0
        \end{cases}
    \]
\end{snippetproposition}

\begin{snippet}{systems-of-linear-equations}
    A system of linear equations
    \begin{align*}
        a_1x + b_1y + c_1z &= d_1 \\
        a_2x + b_2y + c_2z &= d_2 \\
        a_3x + b_3y + c_3z &= d_3 \\
    \end{align*}

    Can be represented by a matrix multiplication \(M\vec{x}=\vec{d}\)

    \[
        \begin{bmatrix} 
            a_1 && b_1 && c_1 \\
            a_2 && b_2 && c_2 \\
            a_3 && b_3 && c_3
        \end{bmatrix}
        \begin{bmatrix}
            x \\ y \\ z
        \end{bmatrix}
        =
        \begin{bmatrix}
            d_1 \\ d_2 \\ d_3
        \end{bmatrix}
    \]

    The geometrical interpretation is to find the vector \(\vec{x}\)
    such that when the matrix \(M\) is applied to it, the resulting vector is \(\vec{d}\).

    We may represent the whole system just by
    \[
        \begin{bmatrix} 
            a_1 && b_1 && c_1 && d_1 \\
            a_2 && b_2 && c_2 && d_2 \\
            a_3 && b_3 && c_3 && d_3
        \end{bmatrix}
    \]
\end{snippet}

\subsection{Using elementary row operations}

\begin{snippet}{linearalgebra-expl1}
    Applying \snippetref[linearalgebra-elementary-row-operations][elementary row operations]
    does not change the solution of the linear system.
\end{snippet}

\includesnpt{linearalgebra-lin-sys-sol-amount}

\begin{snippet}{linearalgebra-expl2}
By applying these operations, our goal is to make the system matrix
look like the following:
\[
    \begin{bmatrix} 
        1 && 0 && 0 && e_1 \\
        0 && 1 && 0 && e_2 \\
        0 && 0 && 1 && e_3
    \end{bmatrix}
\]
which is the implicit solution
\(x=e_1\), \(y=e_2\) and \(z=e_3\).
If this is possible, then this is the only solution to our system.

\vspace{.25cm}

If it is possible to create a row whose elements are all \(0\)s,
then there are infinitely many solutions, because
there are infinitely many solutions for the equation
\(0x_1+0x_2+\cdots+0x_n = 0\).

\vspace{.25cm}

If it is possible to create a whose elements elements
are all \(0\)s expect for the last one there are zero solutions,
because there are zero solutions for the equation
\(0x_1+0x_2+\cdots+0x_n = a\) with \(a \neq 0\).
\end{snippet}

\subsubsection{Examples}

\includesnpt{linearalgebra-matrix-linear-system-1-sol-example-1}
\includesnpt{linearalgebra-matrix-linear-system-inf-sol-example-1}
\includesnpt{linearalgebra-matrix-linear-system-0-sol-example-1}

\section{Cramer's rule}

\begin{snippettheorem}{cramer-rule-theorem}{Cramer's rule}
    Consider a system of \(n\) equations and unknowns
    \[
        A\vec{x}=\vec{b}
    \]
    The solution is given by
    \[
        \vec{x}_i = \frac{\det(A_i)}{\det(A)}
    \]
    where \(A_i\) is formed by replacing the \(i\)-th column
    of \(A\) by \(\vec{b}\).
\end{snippettheorem}

\end{document}
\documentclass[preview]{standalone}

\usepackage{amsmath}
\usepackage{amssymb}
\usepackage{stellar}
\usepackage{definitions}

\begin{document}

\id{sequences-functions-introduction}
\genpage

\section{Sequence of Functions}

\begin{snippetdefinition}{sequence-of-functions-definition}{Sequence of functions}
    A \emph{sequence of functions} is a family of \function[functions] \(\{f_n\}_{n\in\naturalnumbers}\)
    defined on a common domain \(f_n \colon D \to \realnumbers\).
\end{snippetdefinition}

\begin{snippetdefinition}{function-sequence-convergence-definition}{Convergence}
    Let \(\{f_n\}_{n\in\naturalnumbers}\) be a \sequence of \function[functions].
    The sequence \emph{converges} at a point \(x_0\) if
    \[
        \lim_{n\to\infty} f_n(x_0) < \infty
    \]
\end{snippetdefinition}

\begin{snippetdefinition}{pointwise-convergence-definition}{Pointwise convergence}
    Let \(\{f_n\}_{n\in\naturalnumbers}\) be a \sequence of \function[functions].
    The sequence \emph{converges pointwise} to a \function \(f\colon D \to \realnumbers\) if
    \[
        \forall x\in D, \lim_{n\to\infty} f_n(x) = f(x)
    \]
\end{snippetdefinition}

\plain{So the sequence converges at every point, but the speed of convergence may depend on the point.}

\begin{snippetdefinition}{function-sequence-uniform-convergence-definition}{Uniform convergence}
    Let \(\{f_n\}_{n\in\naturalnumbers}\) be a \sequence of \function[functions].
    The sequence \emph{converges uniformly} to a \function \(f\colon D \to \realnumbers\) if
    \[
        \sup_{x\in D} \left| f_n(x) - f(x) \right| \to 0
    \]
    as \(n\to\infty\).
\end{snippetdefinition}

\begin{snippet}{uniform-convergence-alternative}
    Alternatively, we can say that the condition is
    \[
        \forall \varepsilon > 0, \exists N \suchthat \forall n>N, ||f_n-f||_{\infty, E} < \varepsilon
    \]
    
    We should say that the difference
    \[
        |f_n(x) - f| \leq \varepsilon
    \]
    but since this must hold for all \(x\), we can use the supremum.
    
    So, the speed of convergence is the same at every point.
    Everything that converges uniformly also converges pointwise.
\end{snippet}

\begin{snippetdefinition}{function-sequence-uniform-cauchy-convergence-definition}{Uniform Cauchy convergence}
    Let \(\{f_n\}_{n\in\naturalnumbers}\) be a \sequence of \function[functions].
    The \sequence is \emph{uniformly Cauchy} if
    \[
        \forall \varepsilon > 0,
        \exists N \in \naturalnumbers \suchthat
        \forall n,m > N,
        \sup_{x\in D} \left| f_n(x) - f_m(x) \right| < \varepsilon
    \]
\end{snippetdefinition}

\plain{Starting from a certain index, all functions in the sequence are very close to each other uniformly over the entire domain,
regardless of the limit function.}

\end{document}
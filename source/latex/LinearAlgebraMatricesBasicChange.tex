\documentclass[preview]{standalone}

\usepackage{amsmath}
\usepackage{amssymb}
\usepackage{stellar}
\usepackage{definitions}

\begin{document}

\id{matrices-basis-change}
\genpage

\section{Change of basis}

\begin{snippettheorem}{change-of-basis-theorem}{Change of basis}
    Given a \basis \(\mathcal{B}_1\) and \(\mathcal{B}_2\),
    there exists a \matrix \(M\) such that \(M\vec{v}\) translates
    the vector from \(\mathcal{B}_1\) to \(\mathcal{B}_2\).
    
    The \matrix \(M\) is formed by the columns of \(\mathcal{B}_1\) but written in
    \(\mathcal{B}_2\).
    
    The operation \(M^{-1}\vec{v}\) translates
    the vector from \(\mathcal{B}_2\) to \(\mathcal{B}_1\).
\end{snippettheorem}

\begin{snippetproposition}{change-of-basis-linear-transformation}{Change of basis of a linear transformation}
    If \(T_1\) is a transformation in \(\mathcal{B}_1\), we can find another
    transformation \(T_2\) that is similar to \(T_1\) but works for the \basis \(\mathcal{B}_2\).
    
    Instead of computng \(T_1\vec{v}\), we first multiply \(\vec{v}\) by the
    change of basis matrix \(M\), then apply \(T_1\) and finally apply the inverse change of basis matrix \(M^{-1}\).
    \[
        T_2 = M^{-1}T_1M
    \]
\end{snippetproposition}

\end{document}
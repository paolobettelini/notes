\documentclass[preview]{standalone}

\usepackage{amsmath}
\usepackage{amssymb}
\usepackage{stellar}
\usepackage{definitions}
\usepackage{bettelini}

\begin{document}

\id{triple-integrals-exercises}
\genpage

\section{Exercises}

\plain{TODO: exercise sheet.}

\begin{snippetexercise}{triple-integrals-ex1}{Volume of tetrahedron}
    Compute the volume of
    \[
        D = \{(x,y,z) \in \realnumbers^3 \suchthat x\geq 0 \land y \geq 0 \land z \geq 0 \land x + y + z \leq 1\}
    \]
    which is the standard \(3\)-standard, or the standard tetrahedron
    of the positive octant.
\end{snippetexercise}

\begin{snippetsolution}{triple-integrals-ex1-so1}{Volume of tetrahedron - Solution 1}
    We want to compute
    \[
        \iiint_D f(x,y,z) \,\text{d}x\,\text{d}y\,\text{d}z
    \]
    where \(f(x,y,z)=1\).
    We can dissect it into layers \(z \in [0,1]\)
    \begin{align*}
        \iiint_D f(x,y,z) \,\text{d}x\,\text{d}y\,\text{d}z
        &=
        \integral[0][1][
            \left(
                \iint_{D_z} f(x,y,z) \,\text{d}x\,\text{d}y
            \right)
        ][z] \\
        &= \integral[0][1][
            \integral[0][1-z][
                \integral[0][1-z-y][1][x]
            ][y]
        ][z] \\
        &= \integral[0][1][
            \integral[0][1-z][
                1-z-y
            ][y]
        ][z] = \frac16
    \end{align*}
\end{snippetsolution}

\begin{snippetsolution}{triple-integrals-ex2-so1}{Volume of tetrahedron - Solution 2}
    We want to compute
    \[
        \iiint_D f(x,y,z) \,\text{d}x\,\text{d}y\,\text{d}z
    \]
    where \(f(x,y,z)=1\).
    We can dissect it by threads. Consider the projection of the domain
    on the \(xy\) plane.
    Consider vertical threads reaching the edge of the domain \(z = 1 - x - y\).
    So a single thread goes from \(0\) a \(1-x-y\). We thus have
    \begin{align*}
        \iiint_D f(x,y,z) \,\text{d}x\,\text{d}y\,\text{d}z
        &=
        \iint_T \left(\integral[0][1-x-y][f(x,y,z)][z]\right) \,\text{d}x\,\text{d}y
    \end{align*}
    where \(T\) is the base \(\{(x,y) \suchthat x \geq 0 \land y \geq 0 \land x+y \leq 1\}\).
    Thus,
    \begin{align*}
        \iiint_D 1 \,\text{d}x\,\text{d}y\,\text{d}z
        &= \iint_T \integral[0][1-x-y][1][z] \,\text{d}x\,\text{d}y \\
        &= \iint_T 1-x-y \,\text{d}x\,\text{d}y \\
        &= \integral[0][1][
            \integral[0][1-x][1][y]
        ][x] = \frac16
    \end{align*}
\end{snippetsolution}

\end{document}
\documentclass[preview]{standalone}

\usepackage{amsmath}
\usepackage{amssymb}
\usepackage{stellar}
\usepackage{definitions}

\begin{document}

\id{infimum-and-supremum}
\genpage

\section{Definitions}

\begin{snippetdefinition}{supremum-definition}{Supremum}
    Let \(E\) and \(X\) be \set[sets] where \(E\subseteq X \land E \neq \emptyset\)
    and \(\leq\) be a \partialorder on \(E\).
    Then, the \emph{supremum} of \(E\) is the \upperbound of \((E,\leq)\) that is less than every other 
    \upperbound[upper bounds] of \((E,\leq)\), if it exists.
    \[
        \mu = \origsup E
    \]
\end{snippetdefinition}

\begin{snippetdefinition}{infimum-definition}{Infimum}
    Let \(E\) and \(X\) be \set[sets] where \(E\subseteq X \land E \neq \emptyset\)
    and \(\leq\) be a \partialorder on \(E\).
    Then, the \emph{infimum} of \(E\) is the \lowerbound of \((E,\leq)\) that is greater than every other 
    \lowerbound[lower bounds] of \((E,\leq)\), if it exists.
    \[
        \mu = \originf E
    \]
\end{snippetdefinition}

\section{Basic results}

\begin{snippetproposition}{supremum-of-union-of-sets}{Supremum of unions of sets}
    Let \(A\) and \(B\) be \bounded[bounded sets] under some \partialorder where \(A \neq \emptyset \land B\neq \emptyset\).
    Then,
    \[ \sup A \union B = \max\{\sup A, \sup B\} \] 
\end{snippetproposition}

\section{Limit inferior and superior}

\begin{snippetdefinition}{limit-superior-definition}{Limit superior}
    Let \(E\) and \(X\) be \set[sets] where \(E\subseteq X \land E \neq \emptyset\)
    and \(\leq\) be a \partialorder on \(E\) such that the \set is also a \topologicalspace.
    Then, the \emph{limit superior} is defined as
    \[
        \origlimsup E \triangleq \sup \left\{
            x \in X \suchthat x \text{ is a \accumulationpoint[limit point] of } E
        \right\}
    \]
\end{snippetdefinition}

\begin{snippetdefinition}{limit-inferior-definition}{Limit inferior}
    Let \(E\) and \(X\) be \set[sets] where \(E\subseteq X \land E \neq \emptyset\)
    and \(\leq\) be a \partialorder on \(E\) such that the \set is also a \topologicalspace.
    Then, the \emph{limit inferior} is defined as
    \[
        \origliminf E \triangleq \inf \left\{
            x \in X \suchthat x \text{ is a \accumulationpoint[limit point] of } E
        \right\}
    \]
\end{snippetdefinition}

\end{document}
\documentclass[preview]{standalone}

\usepackage{amsmath}
\usepackage{amssymb}
\usepackage{stellar}
\usepackage{definitions}
\usepackage{bettelini}

\begin{document}

\id{topology-metric-spaces}
\genpage

\newcommand\ts{{(X, \mathcal{T})}}

\section{Metric topology}

\begin{snippetdefinition}{metric-topology-definition}{Metric topology}
    Let \((X, d)\) be a \metricspace. The \textit{topology induced by} \((X, d)\),
    denoted \(\mathcal{T}_d\), is defined as \((X, \mathcal{T})\)
    where \(\mathcal{T}\) is the set of all sets that are \msopenset on \((X, d)\).
\end{snippetdefinition}

\begin{snippetproposition}{metric-topology-is-topology}{Metric topology is a topology}
    Let \((X, d)\) be a \metricspace. Then, \(\metrictopology_d\) is a \topologicalspace[topology][Topology]
    on X.
\end{snippetproposition}

\begin{snippetproof}{metric-topology-is-topology-proof}{metric-topology-is-topology}{Metric topology is a topology}
    \begin{enumerate}
        \item the empty set is always open, \(\emptyset \in \mathcal{T}\);
        \item the set itself is always open, \(X \in \mathcal{T}\);
        \item let \(x \in A \intersection B\).
        Using the metric we have \(r,s > 0\) such that \(\ball_r(x) \subseteq A\)
        and \(\ball_s(x) \subseteq B\).
        Let \(t = \min\{r,s\}\). We thus have
        \begin{align*}
            \ball_t(x) &\subseteq \ball_r(x) \subseteq A \\
                   &\subseteq \ball_s(x) \subseteq B \\
                   &\subseteq A \intersection B
        \end{align*}
        \item consider the union of an arbitrary family and let
        \[
            x \in \bigcup_{i\in I} A_i
        \]
        this is equivalent to saying that there exist an \(i\in I\)
        such that \(x \in A_i\), and thus there exist an \(r>0\) such that
        \(\ball_r(x) \subseteq A_i\). But
        \[ A_i \subseteq \bigcup_{i\in I} A_i \]
        and thus
        \[
            \ball_r(x) \subseteq \bigcup_{i\in I} A_i
        \]
    \end{enumerate}
\end{snippetproof}

\begin{snippetdefinition}{standard-real-topology-definition}{Standard real topology}
    The \emph{standard topology} on \(\realnumbers\) is the topology induced by the standard \metricspace
    \((\realnumbers, d)\) where \(d(x,y) = |x - y|\).
    The open sets of the standard topology are called \textit{open intervals}.
\end{snippetdefinition}

\begin{snippetproposition}{metric-topology-open-classification}{}
    Let \((X, d)\) be a \metricspace and \(\mathcal{T}_d\) the topology induced by \(d\). Then,
    \begin{enumerate}
        \item every \openball is open in \(\mathcal{T}_d\);
        \item the \topologicalspace[open sets][Open set] of \(\mathcal{T}_d\)
        are exactly the subsets of \(X\) which can be written as open unions of \openball[open balls];
        \item for every \(x\in X\), a subset \(U \subseteq X\)
        is a \neighborhood of \(x\) (for the \(\mathcal{T}_d\) topology) \ifandonlyif
        \(\exists \ball_r(x)\) entirely contained in \(U\).
    \end{enumerate}
\end{snippetproposition}

\begin{snippetproof}{metric-topology-open-classification-proof}{metric-topology-open-classification}{}
    \begin{enumerate}
        \item Let \(x\in X\) and \(y \in \ball_r(x)\).
        Since \(y\) is in the ball, we have the inequality
        \(d(x,y) < r\). Let \(s = r - d(x,y)\).
        We want to show that the ball is completely included \(B_s(y) \subseteq B_r(x)\).
        To prove this, let \(z \in B_s(y)\). We know that \(d(z, y) < s\).
        Using the triangular inequality, we can estimate
        \[
            d(z,x) \leq d(z,y) + d(x, y) < d(x,y) + s
            = r- d(x,y) + d(x,y) = r
        \]
        This proves the inclusion.
        \item We know that every \openball is \topologicalspace[open][Open set]
        in \(\mathcal{T}_d\). Viceversa, let \(A\) be an \topologicalspace[open set][Open set]
        of \(\mathcal{T}_d\). We want to express this as a union of \openball[open balls].
        For all \(x \in A\), there exist \(r(x) > 0\) such that \(\ball_{r(x)}(x) \subseteq A\).
        We can thus just consider
        \[
            A = \bigcup_{x\in A} B_{r(x)}(x)
        \]
        \item \iffproof{
            Since \(A\) is \topologicalspace[open][Open set]
            in \(\mathcal{T}_d\), by its definition, \(A \supseteq \ball_r(x)\).
            This means that \(U \supseteq A \supseteq \ball_r(x)\).
        }{
            Let \(U \supseteq \ball_r(x) \ni x\). The \openball[ball] is open
            by the first point. \(U\) is a \neighborhood of \(x\).
        }
    \end{enumerate}
\end{snippetproof}

\subsection{Different metrics, same topology}

\begin{snippetproposition}{equivalent-topology-from-different-metrics}{}
    Let \((X, d), (X, d')\) be \metricspace[metric spaces] on the same \set.
    Suppose that there exist constants \(c_1, c_2>0\) such that
    \[
        d(x,y) \leq c_1 d'(x,y), \quad \land \quad
        d'(x,y) \leq c_2d(x,y)
    \]
    Then, the induced topologies are the same \(\mathcal{T}_d = \mathcal{T}_{d'}\).
\end{snippetproposition}

\begin{snippetproof}{equivalent-topology-from-different-metrics-proof}{equivalent-topology-from-different-metrics}{}
    We compare the \topologicalspace[open balls][Open sets]
    of \(\mathcal{T}_d\) and \(\mathcal{T}_{d'}\).
    Consider
    \begin{align*}
        \ball^{d}_r(x) &= \{y \in X \suchthat d(x,y) < r\} \\
        &\supseteq \left\{y \in X \suchthat \frac{d'(x,y)}{c_2} < r\right\} \\
        &= \ball^{d'}_{r c_2}(x)
    \end{align*}
    since \(d'(x,y) < rc_2\). Likewise, we get the other inclusion.
\end{snippetproof}

\plain{This clearly shows that we lose the quantitative aspects of a metric space when inducing a topological space,
but the qualitative aspects are preserved.}

\plain{All the metrics induced by the p-norms induce the same topology.}

\end{document}
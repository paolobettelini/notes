\documentclass[preview]{standalone}

\usepackage{amsmath}
\usepackage{amssymb}
\usepackage{stellar}
\usepackage{definitions}
\usepackage{bettelini}

\begin{document}

\id{mechanics-ex-1}
\genpage

\section{Exercises - Batch 1}

\begin{snippetexercise}{mechanics-ex-1.0}{\underline{1.0} Free fall}
    A body is dropped from a certain height with zero initial velocity: how much time \(t\) must pass before,
    starting from the instant \(t=0\), the body travels a distance \(s=20\) m at time \(\tau=1\) s?
\end{snippetexercise}

\begin{snippetsolution}{mechanics-ex-1.0-sol}{\underline{1.0} Free fall}
    The falling function is given by \(s = \frac{1}{2}at^2 + v_0t + s_0\).
    We thus have
    \begin{align*}
        20 &= \integral[t][t+\tau][v_0 + ax][x] \\
        20 &= \frac{1}{2}a{(t+1)}^2 - \frac{1}{2}at^2 \\
        40 &= a{(t+1)}^2 - at^2 \\
        40 &= a(t^2 + 2t + 1) - at^2 \\
        40 &= at^2 + 2at + a - at^2 \\
        40 &= 2at + a \\
        40 - a &= 2at \\
        t &= \frac{40 - a}{2a}
    \end{align*}
\end{snippetsolution}

\begin{snippetexercise}{mechanics-ex-1.1}{\underline{1.1}}
    The motion in the \(x, y\) plane of a particle is defined by the equations
    \begin{align*}
        x &= \alpha t^2 + \beta t \\
        y &= \alpha t^2 - \beta t
    \end{align*}
    Calculate the magnitudes of velocity and acceleration at time \(t=\tau\) s.
\end{snippetexercise}

\begin{snippetsolution}{mechanics-ex-1.1-sol}{\underline{1.1}}
    The velocity is given by
    \begin{align*}
        \frac{dx}{dt} &= 2\alpha t + \beta \\
        \frac{dy}{dt} &= 2\alpha t - \beta
    \end{align*}
    and the magnitude at \(t=\tau\) is
    \begin{align*}
        \sqrt{{(2\alpha\tau + \beta)}^2 + {(2\alpha\tau - \beta)}^2}
        = \sqrt{8\alpha^2\tau^2 + 2\beta^2}
    \end{align*}
    The acceleration is given by
    \begin{align*}
        \frac{d^2x}{dt^2} &= 2\alpha \\
        \frac{d^2y}{dt^2} &= 2\alpha
    \end{align*}
    and the magnitude at \(t=\tau\) is
    \begin{align*}
        \sqrt{{(2\alpha)}^2 + {(2\alpha)}^2} = 2\alpha\sqrt{2}
    \end{align*}
\end{snippetsolution}

\begin{snippetexercise}{mechanics-ex-1.2}{\underline{1.2}}
    A car moving with velocity \(v_0\) begins to brake and, moving in rectilinear motion, stops after traveling a distance \(1\).
    Determine the scalar braking acceleration \(a_m\) in the following three cases:
    \begin{itemize}
        \item the scalar acceleration has a constant value \(A\) over time;
        \item the acceleration depends on the scalar velocity according to the law \(a = b(v + v_0)\);
        \item the acceleration varies linearly with time, \(a = \gamma t\).
    \end{itemize}
\end{snippetexercise}

\begin{snippetsolution}{mechanics-ex-1.2-sol}{\underline{1.2}}
    \begin{itemize}
        \item we have \(a(t) = A\) meaning that \(v(t) = v_0 + At\).
            We want to find \(t^*\) such that \(v(t^*) = 0\), meaning
            \[
                t^* = -\frac{v_0}{A}
            \]
            We integrate the velocity to find \(x(t) = v_0t + \frac{1}{2}At^2\).
            Since we know that \(x(t^*) = l\), we get
            \[
                l = \frac{-v_0^2}{A}
            \]
            Finally,
            \[
                a_m = \frac{\Delta v}{\Delta t}
                = \frac{-v_0}{t^*} = A = -\frac{v_0^2}{2l}
            \]
        \item we have \(a(t) = bv(t) + bv_0\). In order to find the velocity, we need to solve
            \begin{align*}
                \frac{dv}{dt} &= b(v(t) + v_0) \\
                \int \frac{dv}{v + v_0} &= \integral[b][t] \\
                \log|v(t) + v_0| &= bt + C \\
                v(t) &= Ce^{bt} - v_0
            \end{align*}
            We know that \(v(0) = 0\), and thus \(C = 2v_0\).
            We now want to find \(t^*\) such that \(v(t^*) = 0\), meaning
            \[t^* = -\frac{1}{b} \log 2\]
            With this we can find
            \[ a_m = \frac{\Delta v}{\Delta t} = \frac{bv_0}{\log 2} \]
            However, since \(x(t^*)=l\), we can find
            \[
                b = \frac{v_0}{l}(\log 2 - 1)
            \]
            which leads to the final solution
            \[
                a_m = \frac{v_0^2}{l} \frac{\log2 - 1}{\log 2}
            \]
        \item we have \(a(t) = \gamma t\) meaning \(v(t) = \frac{\gamma}{2}t^2 + v_0\).
        We want to find \(t^*\) such that \(v(t^*) = 0\), meaning
        \[
            t^* = \sqrt{-\frac{2}{\gamma} v_0}
        \]
        We also have \(x(t) = \frac{\gamma}{6}t^3 + v_0t\), and since
        \(x(t^*) = l\) we can find
        \[
            \gamma = -\frac{8}{9}\frac{v_0^3}{l^2}
        \]
        Finally,
        \[
            a_m = \frac{\Delta v}{\Delta t}
            = \frac{-v_0}{\sqrt{-\frac{2}{\gamma} v_0}} = - \frac{2v_0^2}{3l}
        \]
    \end{itemize}
\end{snippetsolution}

\begin{snippetexercise}{mechanics-ex-1.3}{\underline{1.3}}
    A small-sized body is launched vertically upwards at time \(t = 0\).
    During the ascent and subsequent descent, the body passes through the altitude \( h \),
    relative to the launch position, at times \( t_1 \) and \( t_2 \), respectively.
    Prove that the relation \( t_1 t_2 = 2h/g \) holds.
    (Neglect the effect of air resistance on the body's motion.)
\end{snippetexercise}

\begin{snippetsolution}{mechanics-ex-1.3-sol}{\underline{1.3}}
    We have \(a(t) = -g\), \(v(t) = -gt + v_0\) and \(x(t) = -\frac{g}{2}t^2 + v_0t\).
    We now find \(t_1\) and \(t_2\),
    \[
        t_{1,2} = \frac{v_0}{g} \pm \sqrt{\frac{v_0^2}{g^2} - \frac{2h}{g}}
    \]
    Finally,
    \[
        t_1t_2 = \frac{v_0^2}{g^2} - \frac{v_0^2}{g^2} + \frac{2h}{g} = \frac{2h}{g}
    \]
\end{snippetsolution}

\begin{snippetexercise}{mechanics-ex-1.4}{\underline{1.4}}
    A body is launched horizontally from a height \( h \) relative to the ground, with velocity \( v_0 \). Neglecting air resistance, calculate:
    \begin{itemize}
        \item the tangential component \( a_T \) and the normal component \( a_N \) of the acceleration of the body relative to its trajectory at a generic point at height \( h \);
        \item the distance traveled by the body from the launch instant (\( t = 0 \)) to the moment it touches the ground.
    \end{itemize}
\end{snippetexercise}

\begin{snippetsolution}{mechanics-ex-1.4-sol}{\underline{1.4}}
    We have the following
    \[
        \begin{cases}
            a_x = 0 \\
            a_z = -g
        \end{cases} \quad
        \begin{cases}
            v_x = v_0 \\
            v_z = -gt
        \end{cases} \quad
        \begin{cases}
            x = v_0t \\
            z = h_0 - \frac{1}{2}gt^2
        \end{cases}
    \]
    Since \(z = h_0 - \frac{1}{2}gt^2\), we have \(t = \sqrt{\frac{2}{g}(h_0-z)}\).
    Furthermore, since \(v_z = -gt\), we have
    \[
        \vec{v} = v_0 \hat{x} - \sqrt{\frac{2}{g}(h_0-z)} \hat{z}
    \]
    and
    \[
        |\vec{v}| = \sqrt{v_0^2 + 2g(h_0 - z)}
    \]
    Let \(\alpha\) be the angle between the gravitational vector and the normal vector
    \[
        \cos\alpha = \frac{\vec{v} \cdot \hat{x}}{|\vec{v}|}
        = \frac{v_0}{\sqrt{v_0^2 + 2g(h_0 - z)}}
    \]
    Thus, \(a_T = g\sin\alpha\) and \(a_N = g\cos\alpha\).
    The ball touches the the ground at \(t^* = \sqrt{\frac{2h_0}{g}}\).
    The range is \(x(t^*) = v_0 \sqrt{\frac{2h_0}{g}}\).
    In order to compute the arc length we need to compute
    \begin{align*}
        \integral[0][x(t^*)][\sqrt{1 + {\left(\frac{dz}{dx}\right)}^2}][x]
        &= \integral[0][x(t^*)][\sqrt{1 + {\left(\frac{-gt}{v_0}\right)}^2}][x] \\
        &= \integral[0][x(t^*)][\sqrt{1 + {\left(\frac{g}{v_0^2}\right)}^2x^2}][x]
    \end{align*}
\end{snippetsolution}

\begin{snippetexercise}{mechanics-ex-1.5}{\underline{1.5}}
    A person climbs a spiral staircase starting from the ground floor at time \( t = 0 \). The person always remains at a constant distance \( r = 2 \) m from the central axis of the staircase and climbs one step every second. Each step has a height of \( h = 20 \) cm and a depth of \( d = 20 \) cm. To study the person's motion, we can use:
    \begin{itemize}
        \item a system of cartesian coordinates;
        \item a system of cylindrical coordinates.
    \end{itemize}
    Derive in both cases the trajectory equation, the motion equation, and the components of velocity as a function of time.
\end{snippetexercise}

\begin{snippetsolution}{mechanics-ex-1.5-sol}{\underline{1.5}}
    Since the person is moving at one step per second,
    we have a velocity of \(0.20\) m/s, and an angular increment per step
    \[
        \Delta \theta = \frac dr = 0.10 \text{ rad}
    \]
    \begin{enumerate}
        \item \emph{cartesian coordinates:}
        the person moves on a helix of radius \(2\) with angle \(\theta(t) =\omega t = 0.1t\)
        and height \(z(t) = 0.20t\)
        \[
            \vec{r}(t) = \begin{pmatrix}
                x(t) \\ y(t) \\ z(t)
            \end{pmatrix} = \begin{pmatrix}
                2\cos(0.1t) \\ 2\sin(0.1t) \\ 0.2t
            \end{pmatrix}
        \]
        for the velocity we differentiate componentwise:
        \[
            \vec{v}(t) = \begin{pmatrix}
                \dot{x}(t) \\ \dot{y}(t) \\ \dot{z}(t)
            \end{pmatrix} = \begin{pmatrix}
                -0.20\sin(0.1t) \\ 0.20\cos(0.1t) \\ 0.2
            \end{pmatrix}
        \]
        for the acceleration we derive again
        \[
            \vec{a}(t) = \begin{pmatrix}
                \ddot{x}(t) \\ \ddot{y}(t) \\ \ddot{z}(t)
            \end{pmatrix} = \begin{pmatrix}
                -0.02\cos(0.1t) \\ -0.02\cos(0.1t) \\ 0
            \end{pmatrix}
        \]
        \item \emph{cylindrical coordinates:}
        the parametric trajectory in cylindrical form is
        \[
            (\rho, \theta, z)(t) = (2, 0.1t, 0.20t)
        \]
        the velocity is given by
        \[
            \vec{v} = \dot{\rho}\hat{e}_\rho + \rho\dot{\theta}\hat{e}_\theta + \dot{z}\hat{e}_z
        \]
        in this case we have
        \[
            \vec{v} = 0 \cdot \hat{e}_\rho + 0.20 \hat{e}_\theta + 0.20 \hat{e}_z
        \]
        For the acceleration we have
        \[
            \vec{a} = (\ddot{\rho} - \rho \dot\theta^2)\hat{e}_\rho +(\rho\ddot\theta + 2\dot\rho\dot\theta)\hat{e}_\theta + \ddot{z}\hat{e}_z
        \]
        and in this case \(\ddot{r} = 0\), \(\ddot\theta = 0\), \(\dot{r} = 0\) and \(\ddot{z} = 0\). Thus,
        \[
            \vec{a} = -0.02\hat{e}_\rho + 0\cdot \hat{e}_\theta + 0\cdot \hat{e}_z
        \]
    \end{enumerate}
\end{snippetsolution}

\begin{snippetexercise}{mechanics-ex-1.6}{\underline{1.6}}
    A point moves along an elliptical trajectory with constant speed \( V \) over time. With respect to an orthogonal cartesian coordinate system, the equation of the ellipse is
    \[
        \frac{x^2}{a^2} + \frac{y^2}{b^2} = 1,
    \]
    where \( a \) and \( b \) are the semi-axes. Compute the \( x \) and \( y \) components of the acceleration at the point with position \( P = (x, y) \).
\end{snippetexercise}

\begin{snippetsolution}{mechanics-ex-1.6-sol}{\underline{1.6}}
    \todo
\end{snippetsolution}

\begin{snippetexercise}{mechanics-ex-1.7}{\underline{1.7}}
    Consider a plane motion in which the instantaneous velocity of the point always maintains the same angle \( \alpha \in (0, \frac{\pi}{2}) \) with the line connecting the point to the origin of the axes. Derive the trajectory equation. [This exercise schematically illustrates the nocturnal flight of moths.]
\end{snippetexercise}

\begin{snippetsolution}{mechanics-ex-1.7-sol}{\underline{1.7}}
    \todo
\end{snippetsolution}

\end{document}
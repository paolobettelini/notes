\documentclass[preview]{standalone}

\usepackage{amsmath}
\usepackage{amssymb}
\usepackage{stellar}
\usepackage{definitions}
\usepackage{bettelini}

\begin{document}

\id{mechanics-ex-1}
\genpage

\section{Exercises - Batch 1}

\begin{snippetexercise}{mechanics-ex-1.0}{\underline{1.0} Free fall}
    A body is dropped from a certain height with zero initial velocity: how much time \(t\) must pass before,
    starting from the instant \(t=0\), the body travels a distance \(s=20\) m at time \(\tau=1\) s?
\end{snippetexercise}

\begin{snippetsolution}{mechanics-ex-1.0-sol}{\underline{1.0} Free fall}
    The falling function is given by \(s = \frac{1}{2}at^2 + v_0t + s_0\).
    We thus have
    \begin{align*}
        20 &= \integral[t][t+\tau][v_0 + ax][x] \\
        20 &= \frac{1}{2}a{(t+1)}^2 - \frac{1}{2}at^2 \\
        40 &= a{(t+1)}^2 - at^2 \\
        40 &= a(t^2 + 2t + 1) - at^2 \\
        40 &= at^2 + 2at + a - at^2 \\
        40 &= 2at + a \\
        40 - a &= 2at \\
        t &= \frac{40 - a}{2a}
    \end{align*}
\end{snippetsolution}

\begin{snippetexercise}{mechanics-ex-1.1}{\underline{1.1}}
    The motion in the \(x, y\) plane of a particle is defined by the equations
    \begin{align*}
        x = \alpha t^2 + \beta t \\
        y = \alpha t^2 - \beta t
    \end{align*}
    with \(\alpha = 0.1\) m/s\(^2\) and \(\beta = 1\) m/s.
    Calculate the magnitudes of velocity and acceleration at time \(t = 10\) s.
\end{snippetexercise}

\begin{snippetsolution}{mechanics-ex-1.1-sol}{\underline{1.1}}
    \todo
\end{snippetsolution}

\begin{snippetexercise}{mechanics-ex-1.2}{\underline{1.2}}
    A car moving with velocity \(v_0\) begins to brake and, moving in rectilinear motion, stops after traveling a distance \(1\).
    Determine the scalar braking acceleration \(a_m\) in the following three cases:
    \begin{itemize}
        \item the scalar acceleration has a constant value \(A\) over time;
        \item the acceleration depends on the scalar velocity according to the law \(a = b(v + v_0)\);
        \item the acceleration varies linearly with time, \(a = \gamma t\).
    \end{itemize}
\end{snippetexercise}

\begin{snippetsolution}{mechanics-ex-1.2-sol}{\underline{1.2}}
    \todo
\end{snippetsolution}

\begin{snippetexercise}{mechanics-ex-1.3}{\underline{1.3}}
    A small-sized body is launched vertically upwards at time \(t = 0\).
    During the ascent and subsequent descent, the body passes through the altitude \( h \),
    relative to the launch position, at times \( t_1 \) and \( t_2 \), respectively.
    Prove that the relation \( t_1 t_2 = 2h/g \) holds.
    (Neglect the effect of air resistance on the body's motion.)
\end{snippetexercise}

\begin{snippetsolution}{mechanics-ex-1.3-sol}{\underline{1.3}}
    \todo
\end{snippetsolution}

\begin{snippetexercise}{mechanics-ex-1.4}{\underline{1.4}}
    A body is launched horizontally from a height \( h \) relative to the ground, with velocity \( v_0 \). Neglecting air resistance, calculate:
    \begin{itemize}
        \item the tangential component \( a_T \) and the normal component \( a_N \) of the acceleration of the body relative to its trajectory at a generic point at height \( h \);
        \item the distance traveled by the body from the launch instant (\( t = 0 \)) to the moment it touches the ground.
    \end{itemize}
\end{snippetexercise}

\begin{snippetsolution}{mechanics-ex-1.4-sol}{\underline{1.4}}
    \todo
\end{snippetsolution}

\begin{snippetexercise}{mechanics-ex-1.5}{\underline{1.5}}
    A person climbs a spiral staircase starting from the ground floor at time \( t = 0 \). The person always remains at a constant distance \( r = 2 \) m from the central axis of the staircase and climbs one step every second. Each step has a height of \( h = 20 \) cm and a depth of \( d = 20 \) cm. To study the person's motion, we can use:
    \begin{itemize}
        \item a system of Cartesian coordinates;
        \item a system of cylindrical coordinates.
    \end{itemize}
    Derive in both cases the trajectory equation, the motion equation, and the components of velocity as a function of time.
\end{snippetexercise}

\begin{snippetsolution}{mechanics-ex-1.5-sol}{\underline{1.5}}
    \todo
\end{snippetsolution}

\begin{snippetexercise}{mechanics-ex-1.6}{\underline{1.6}}
    A point moves along an elliptical trajectory with constant speed \( V \) over time. With respect to an orthogonal Cartesian coordinate system, the equation of the ellipse is
    \[
    \frac{x^2}{a^2} + \frac{y^2}{b^2} = 1,
    \]
    where \( a \) and \( b \) are the semi-axes. Compute the \( x \) and \( y \) components of the acceleration at the point with position \( P = (x, y) \).
\end{snippetexercise}

\begin{snippetsolution}{mechanics-ex-1.6-sol}{\underline{1.6}}
    \todo
\end{snippetsolution}

\begin{snippetexercise}{mechanics-ex-1.7}{\underline{1.7}}
    Consider a plane motion in which the instantaneous velocity of the point always maintains the same angle \( \alpha \in (0, \frac{\pi}{2}) \) with the line connecting the point to the origin of the axes. Derive the trajectory equation. [This exercise schematically illustrates the nocturnal flight of moths.]
\end{snippetexercise}

\begin{snippetsolution}{mechanics-ex-1.7-sol}{\underline{1.7}}
    \todo
\end{snippetsolution}

\end{document}
\documentclass[preview]{standalone}

\usepackage{amsmath}
\usepackage{amssymb}
\usepackage{stellar}

\hypersetup{
    colorlinks=true,
    linkcolor=black,
    urlcolor=blue,
    pdftitle={Stellar},
    pdfpagemode=FullScreen,
}

\begin{document}

\title{Stellar}
\id{orlando-furioso-xix-analisi}
\genpage

\section{Analisi}

\begin{snippet}{orlando-furioso-xix-analisi}
    Tutto l'episodio è diviso in due,
    abbiamo la storia di Cloridano e Medoro (episodio epico e di guerra)
    ed un episodio amoroso (Medoro e Angelica).
    \\\\
    All'ottava 17 si nota il cambiamento e passaggio tra i due episodi.
    Si passa in modo armonioso tra l'uno e l'altro.
    Cambia da un argomento all'altro per mantenere vivo l'interesse.
    \\\\
    Viene richiamata una delel storie d'amore più famose di tutta la lettera mondiale,
    ossia quella dell'Eneide (35-37).
    La vicenda di Angelica viene descritta come l'amore di Petrarca, riprendendo sintagmi
    e le rime in maniera verbatim.
    L'elemento anormale è che i ruoli sono invertiti (rovesciamento); la donna
    fa ciò che farebbe l'uomo. Inoltre, il punto di prospettiva è quello della donna,
    lei è la protagonista.
\end{snippet}

\end{document}
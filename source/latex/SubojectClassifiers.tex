\documentclass[preview]{standalone}

\usepackage{amsmath}
\usepackage{amssymb}
\usepackage{stellar}
\usepackage{definitions}
\usepackage{tikz}

\usetikzlibrary{cd}

\begin{document}

\id{subobject-classifiers}
\genpage

\section{Subojects}

\plain{The following is a generalization of subsets.}

\begin{snippetdefinition}{subobject-definition}{Subobject}
    Let \(\mathcal{C}\) a \category and \(A \in \catob(\mathcal{C})\).
    The \emph{subobjects} of \(A\) in \(\mathcal{C}\) are
    \[
        \text{Sub}_\mathcal{C}(A)
        \triangleq
        \catob((\mathcal{C}\slicecat A)_\text{mono}) / \sim
    \]
    where given \(m_1 \colon B_1 \fromto A\) and \(m_2 \colon B_2 \fromto A\),
    we have \(m_1 \sim m_2\) \ifandonlyif there exists an \catisomorphism
    \(i \colon B_1 \fromto B_2\) in \(\mathcal{C}\) such that \(m_2 \circ i = m_1\).
\end{snippetdefinition}

\plain{Subojects are classes of monomorphisms in the slice category if there is an isomorphism between them that makes the triangle commute.}

\subsection{Subojects in the category of sets}

\begin{snippetproposition}{subojects-in-set}{Subobjects in \(\mathbf{Set}\)}
    Let \(A \in \catset\). Then,
    \[
        \subojects_{\catset}(A) \cong \powerset(A) %TODOURGENT link isomorphism
    \]
\end{snippetproposition}

\begin{snippetproof}{subojects-in-set-proof}{subojects-in-set}{Subobjects in \(\mathbf{Set}\)}
    Objects in \(\catset \slicecat A\) are \function[functions]
    \(f \colon X \fromto A\). \monomorphism[Monomorphisms]
    in \(\catset\) are \injective \function[functions].
    Thus, \monomorphism[monomorphisms] in \(\catset \slicecat A\)
    are \injective[injections] \(m \colon B \hookrightarrow A\).
    Two such maps are isomorphic in \(\catset \slicecat A\)
    \ifandonlyif they have the same image subset.
\end{snippetproof}

\subsection{Classifying objects}

\begin{snippet}{subobject-classification-in-set}
    In the \category \(\catset\), subsets \(S\)
    of some \set \(X\) can be identified with their characteristic
    \function[functions] \(\chi_S \colon X \fromto \{0,1\}\).
    Let \(\text{true}: \{\ast\} \fromto \{0,1\}\) where \(\{\ast\}\)
    is the singleton (terminal object) \(1_{\catset}\).
    The \function sends \(\ast\) to \(1\) and we get the following
    pullback square
    \begin{center}
        % https://tikzcd.yichuanshen.de/#N4Igdg9gJgpgziAXAbVABwnAlgFyxMJZABgBpiBdUkANwEMAbAVxiRAGUQBfU9TXfIRRkAjFVqMWbABrdeIDNjwEiI8uPrNWiEAB1dwfXTg59XOXyWDVpMdU1Sd+4KLPdxMKAHN4RUADMAJwgAWyQyEBwIJDVIuiwGNgALCAgAaxB7SW0QLEyQBjoAIxgGAAV+ZSEQQKwvJJwLECDQ8OoopAAmLK02AEImlrDEbsjoxABmHsc9XRwYAA8cYBxAlnMeAODh2I7J6Zz9AGMkrAB9Ti4KLiA
        \begin{tikzcd}
        S \arrow[d, "i"', hook] \arrow[r, "!"] & \{\ast\} \arrow[d, "\text{true}"] \\
        X \arrow[r, "\chi_S"]                  & {\{0,1\}}                        
        \end{tikzcd}
    \end{center}
    where \(i\colon S \fromto X\) is the inclusion
    and \(! \colon S \fromto \{\ast\}\) is the unique arrow
    in \(\catset\) to the terminal object \(1_{\catset}\).
\end{snippet}

% TODO link "1" to terminal object

\begin{snippetdefinition}{suboject-classifier-definition}{Suboject classifier}
    Let \(\mathcal{C}\) be a \category with finite limits, a \emph{subobject classifier}
    is a \monomorphism \(\text{true}\colon 1_\mathcal{C} \fromto \Omega\),
    such that for every \monomorphism \(m \colon a' \fromto a\) there is a unique arrow
    \(\chi_m \colon a \fromto \Omega\) called the \emph{classifying arrow} of \(m\),
    such that we have a pullback square
    \begin{center}
        % https://tikzcd.yichuanshen.de/#N4Igdg9gJgpgziAXAbVABwnAlgFyxMJZABgBpiBdUkANwEMAbAVxiRDoHIQBfU9TXPkIoyARiq1GLNnR58QGbHgJFR5CfWatEIUQH0AOgYC2dHAAsAxo2ABhbnP5Khq0uOqbpOowHljMAHNZbgkYKAD4IlAAMwAnCGMkMhAcCCQAJg8pbRAAQkcQOISM6lSkAGYsrTYjHBgADxxgHFiWB14Y+MTENRS0xErJau8DS3MsPUTqBjoAIxgGAAUBZWEQWKwA8xwCou7ksp6qrxApkBn5pZWXHQ2tnZDuIA
        \begin{tikzcd}
        a' \arrow[r, "!"] \arrow[d, "m"'] & 1_\mathcal{C} \arrow[d, "\text{true}"] \\
        a \arrow[r, "\chi_m"']            & \Omega                                
        \end{tikzcd}
    \end{center}
\end{snippetdefinition}

\end{document}
\documentclass[preview]{standalone}

\usepackage{amsmath}
\usepackage{amssymb}
\usepackage{stellar}
\usepackage{bettelini}

\hypersetup{
    colorlinks=true,
    linkcolor=black,
    urlcolor=blue,
    pdftitle={Stellar},
    pdfpagemode=FullScreen,
}

\begin{document}

\title{Geografia economica}
\id{geoeconomica-ascesa-ordine-economico-mondiale}
\genpage

\section{L'ascesa dell'ordine economico mondiale}

\begin{snippetdefinition}{trenta-gloriosi-definition}{Trenta gloriosi}
    I \textit{trenta gloriosi} si riferiscono a un periodo di notevole prosperità economica e di crescita in Francia
    che va approssimativamente dal 1945 al 1975. Questo periodo è stato caratterizzato da un rapido
    sviluppo economico, una forte crescita industriale e un miglioramento generale delle condizioni di
    vita per la popolazione francese.
\end{snippetdefinition}

\begin{snippet}{trenta-gloriosi-caratteristiche}
    Le caratteristiche sono
    \begin{itemize}
        \item Welfare state, stato sociale esteso;
        \item meno disoccupazione più posti lavoro;
        \item forte crescita economica.
    \end{itemize}
\end{snippet}

\begin{snippetdefinition}{modello-fordista-definition}{Modello fordista}
    Il \textit{modello fordista} intraprende l'utilizzo di tecnologie e catene di montaggio
    per la produzione lavorativa.
\end{snippetdefinition}

\begin{snippet}{ragioni-cause-modello-fordista}
    Le ragioni e le cause sono le seguenti:
    \begin{itemize}
        \item Piano Marshall
        \item I nuovi accordi sul commercio mondiale
        \item La stabilità del sistema economico
        \item Progressi nella ricerca scientifica e tecnologica
        \item Imposizione / diffusione del modello fordista
        \item Disponibilità di fonti energetiche (petrolio e gas naturale) a basso prezzo
    \end{itemize}
\end{snippet}

\end{document}
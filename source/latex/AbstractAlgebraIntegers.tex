\documentclass[preview]{standalone}

\usepackage{amsmath}
\usepackage{amssymb}
\usepackage{parskip}
\usepackage{fullpage}
\usepackage{hyperref}
\usepackage{bettelini}
\usepackage{stellar}
\usepackage{definitions}

\newcommand{\divides}{\,|\,}

\begin{document}

\id{integers-definitions}
\genpage

\section{Divides operator}

\begin{snippetdefinition}{divide-operator-definition}{Divide Operator}
    Given two integers \(a\) and \(b\),
    we say that \(a \divides b\) if \(a\) divides \(b\),
    meaning that
    \[
        \exists x \,|\, ax = b
    \]
\end{snippetdefinition}

\begin{snippet}{divide-operator-properties}
Given the integers \(a\), \(b\) and \(c\)
\begin{align*}
    a \divides b \iff -a \divides b \iff a \divides -b \\
    |a| \leq |b|, \quad b \neq 0 \\
    a \divides b \implies a \divides bc \\
    a \divides b \land b \divides c \implies a \divides c
\end{align*}
\end{snippet}

\section{Division with remainder}

\begin{snippetproposition}{division-with-remainder}{Division with remainder}
    Given two integers \(a\) and \(b\) with \(b > 0\),
    \[
        \exists_{=1} q,r \,|\, a=bq+r, \quad 0 \leq r < b
    \]
\end{snippetproposition}

% TODO proof

% Lemma 5.3.13
\begin{snippetlemma}{common-dividors-of-quotient-and-remainder}{Quotient and remainder common divisors}
Let \(a \in \mathbb{Z}\) and \(b \in {\mathbb{Z}}^+\).
Let \(q\) and \(r\) be the quotient and remainder of the division of \(b\)
by \(a\).
The common divisors of \(a\) and \(b\) are equivalent to the common divisors of \(r\) and \(q\).
\end{snippetlemma}

% TODO proof

\section{Euclidean algorithm}

\begin{snippet}{euclidean-algorithm}
Euclid's algorithm, is an efficient method for computing the greatest common divisor of two integers
\(a\) and \(b\) where \(b > 0\).

Consider
\[
    a = bq + r
\]
The process is iterative.
For each iteration take the coefficient of the quotient (\(b\)) and divide it by the remainder.

\begin{align*}
    &a = bq + r, &0 \leq r < b \\
    &b = rq_1 + r_1, &0 \leq r_1 < r \\
    &r = r_1q_2 + r_2, &0 \leq r_2 < r_1 \\
    \phantom{ } &\qquad \vdots & \\
    &r_n = r_{n+1}q_{n+2} + r_{n+1}, &0 \leq r_{n+2} < r_{n+1} \\
    &r_{n+1} = r_{n+2}q_{n+3} + 0&
\end{align*}
This sequence is stricly decreasing and will terminate with a null remainder.
The last remainder \(r_{n+2}\) is then the greatest common divisor between \(a\) and \(b\).
\end{snippet}

\section{Bézout's identity}

\begin{snippettheorem}{bezout-identity}{Bézout's identity}
    Let \(a\) and \(b\) be integers with greatest common divisor \(d\).
    Then, there exist integers \(x\) and \(y\) such that
    \[
        ax+by=d
    \]
    Furthermore, the integers \(az+bt\) are multiples of \(d\).
\end{snippettheorem}

\section{Greatest common divisor of multiple integers}

\begin{snippetdefinition}{greatest-common-divisor-definition}{Greater common divisor}
The greatest common divisors of \(a_0, a_1, \cdots, a_n\), denoted \(\gcd(a_0, a_1, \cdots, a_n)\),
is the greatest integer \(n\) such that \(n \divides a_k\).
\end{snippetdefinition}

\begin{snippetproposition}{greatest-common-divisor-exists-and-unique}{Uniqueness of GCD}
    The greatest common divisors of \(a_0, a_1, \cdots, a_n\) exists and is unique.
\end{snippetproposition}

\begin{snippetproposition}{greatest-common-divisor-multipliers}{}
Given integers \(a_0, a_1, \cdots, a_n\), there exists integers \(u_k\) such that
\[
    a_0u_0 + \cdots a_n u_n = \gcd(a_0, a_1, \cdots, a_n)
\]
\end{snippetproposition}

\begin{snippetproposition}{composite-greatest-common-divisor}{Composite GCD}
Given integers \(a_0, a_1, \cdots, a_n\)
for \(n \geq 2\), \[\gcd(\gcd(a_0, \cdots, a_{n-1}), a_n) = \gcd(a_0, \cdots, a_n)\]
\end{snippetproposition}

\begin{snippetproposition}{greatest-common-divisor-scalar-multiplication}{}
    Given integers \(a_0, a_1, \cdots, a_n\) and  an integer \(c\),
    \[\gcd(ca_0, ca_1, \cdots, ca_n) = c \cdot \gcd(a_0, a_1, \cdots, a_n)\]
\end{snippetproposition}
% TODO proof

% TODO proof

\subsection{Coprime numbers}

\begin{snippetdefinition}{coprime-definition}{Coprime numbers}
    Two integers \(a\) and \(b\) are said to be \textit{coprime}
    if they have no common divisor other than \(1\), meaning that \(\gcd(a,b)=1\).
\end{snippetdefinition}

\begin{snippetproposition}{coprimes-given-by-gcd}{Coprimes given by GCD}
    Let \(d = \gcd(a, b) \neq 0\). Then, the integers \(a'\) and \(b'\) where \(a = da'\) and \(b = db'\)
    are coprime because.
\end{snippetproposition}

\begin{snippetproof}{coprimes-given-by-gcd-proof}{Coprimes given by GCD}
    Let \(d = \gcd(a, b) \neq 0\). Then \(d = \gcd(da', db') = d\cdot \gcd(a', b') \implies \gcd(a', b') = 1\).
\end{snippetproof}

% TODO pag 35 fundamental theorem of arithmetic
\end{document}

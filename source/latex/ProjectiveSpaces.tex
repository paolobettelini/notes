\documentclass[preview]{standalone}

\usepackage{amsmath}
\usepackage{amssymb}
\usepackage{stellar}
\usepackage{definitions}
\usepackage{tikz}
\usepackage{bettelini}

\usetikzlibrary{cd}

\begin{document}

\id{projective-spaces}
\genpage

\section{Projective spaces}

\subsection{Real spaces}

\begin{snippetdefinition}{real-projective-space-definition}{Real projective space}
    The \emph{real projective space of dimension \(n\)} is defined as
    \[
        \mathbb{P}^n(\realnumbers) \triangleq (\realnumbers^{n+1} \difference \{0\}) /_\sim
    \]
    where \(x \sim y\) if and only if \(\exists y \in \realnumbers \difference \{0\} \suchthat x = \lambda x\).
\end{snippetdefinition}

\plain{This is the quotient by the action of the group of homotheties.}

\begin{snippetproposition}{real-projective-space-standard-affine-cover}{Standard affine cover}
    For each \(i = 0, \dots, n\), define
    \[
        A_i \triangleq
        \bigl\{ [ (x_0,\dots,x_n) ] \in \mathbb{P}^n(\realnumbers)
        \suchthat x_i \neq 0 \bigr\}.
    \]
    Then,
    \begin{enumerate}
        \item each \(A_i\) is \openset[open] in \(\mathbb{P}^n(\realnumbers)\);
        \item the family \(\{A_i\}_{i=0}^n\) covers \(\mathbb{P}^n(\realnumbers)\);
        \item each \(A_i\) is homeomorphic to \(\realnumbers^n\).
    \end{enumerate}
\end{snippetproposition}

\begin{snippetproof}{real-projective-space-standard-affine-cover-proof}{real-projective-space-standard-affine-cover}{Standard affine cover}
    Let \[
        \pi \colon \realnumbers^{n+1} \setminus \{0\}
        \longrightarrow
        \mathbb{P}^n(\realnumbers),
        \qquad
        \pi(x) = [x]
    \]
    denote the canonical projection.
    \begin{enumerate}
        \item We have
            \[
                \pi^{-1}(A_i) =
                \bigl\{ (x_0,\dots,x_n) \in \realnumbers^{n+1} \setminus \{0\}
                \suchthat x_i \neq 0 \bigr\},
            \]
            which is open in \(\realnumbers^{n+1} \setminus \{0\}\).
            Since \(\pi\) is a quotient map, \(A_i\) is open.
        \item Every equivalence class \([x]\) contains a nonzero vector, hence at least
            one coordinate \(x_i\) is nonzero.
            Therefore \([x] \in A_i\) for some \(i\).
        \item Fix \(i\).
            Each class \([x] \in A_i\) admits a representative with \(x_i = 1\).
            The map
            \[
                [x_0:\dots:x_n]
                \longmapsto
                \left(
                    \frac{x_0}{x_i}, \dots,
                    \widehat{\frac{x_i}{x_i}}, \dots,
                    \frac{x_n}{x_i}
                \right)
            \]
            defines a homeomorphism between \(A_i\) and \(\realnumbers^n\).
    \end{enumerate}
\end{snippetproof}

\begin{snippetproposition}{real-projective-space-sphere-description}{Description via sphere}
    The real projective space \(\mathbb{P}^n(\realnumbers)\) is homeomorphic to the quotient
    \[
        S^n / (x \sim -x),
    \]
\end{snippetproposition}

\begin{snippetproof}{real-projective-space-sphere-description-proof}{real-projective-space-sphere-description}{Description via sphere}
    Let \[
        \pi \colon \realnumbers^{n+1} \setminus \{0\}
        \longrightarrow
        \mathbb{P}^n(\realnumbers),
        \qquad
        \pi(x) = [x]
    \]
    denote the canonical projection. \\
    Every projective class \([x]\) contains a unit vector, since
    \[
        x \sim \frac{x}{\|x\|}.
    \]
    If \(x,y \in S^n\) satisfy \(y = \lambda x\) for some \(\lambda \neq 0\), then
    \[
        1 = \|y\| = \|\lambda x\| = |\lambda| \cdot \|x\| = |\lambda|,
    \]
    hence \(\lambda = \pm 1\).
    Thus the restriction of \(\sim\) to \(S^n\) is generated by the action of the
    finite group
    \[
        \mathbb{Z}_2 = \{ \mathrm{id}, -\mathrm{id} \}.
    \]
    Consider the commutative diagram
    \begin{center}
        % https://tikzcd.yichuanshen.de/#N4Igdg9gJgpgziAXAbVABwnAlgFyxMJZABgBpiBdUkANwEMAbAVxiRAGUA9QgX1PUy58hFAEZyVWoxZsAOrIBOMRmCYBbAEYwFcTsDABqUT3lQsAM3PaYYAMYx5wYvJ4g+A7HgJEATBOr0zKyIHNxu-CAYnsK+pKKSgTIhXGAA9AD68thq4R5C3ihk8QHSwaFpmbLZuZGCXiLI4sVSQXKyanQ4ABYaGsAACjzcABTySirqWjoAlG6SMFAA5vBEoOYKEDmIZCA4EEgALCWtIWPwOArA8mhYPMAprtQMdFoM-XUxIQpYi104NetNkg-Lt9ogAMzHJIgM5wC5XWQ3O4PAEbLY7PZIcQgZ6vd7RAo4mDmf5QspYEDUHB0LAMNhdCAQADWqKBiGxmMQIMSZQUrK2HLBAFYyW0bvzDlThaKQuYJYgRaCkJCWtDzHoALTGOY8IA
        \begin{tikzcd}
            S^n \arrow[d, "\restr{\pi}{S^n}"'] \arrow[r, "i", hook] & \realnumbers^{n+1}\difference\{0\} \arrow[r, "r"] \arrow[d, "\pi"] & S^n \arrow[d, "\restr{\pi}{S^n}"] \\
            S^n/_\sim \arrow[r, "f"]                                & \mathbb{P}^n(\realnumbers) \arrow[r, "f^{-1}"]                     & S^n/_\sim                        
        \end{tikzcd}
    \end{center}
    where \(r(x) = x / \|x\|\).
    The map \(f\) is a well-defined homeomorphism.
\end{snippetproof}

\begin{snippetcorollary}{real-projective-space-hausdorff}{Real projective space is Hausdorff}
    The real projective space \(\mathbb{P}^n(\realnumbers)\) is a Hausdorff space.
\end{snippetcorollary}

\begin{snippetproof}{real-projective-space-hausdorff-proof}{real-projective-space-hausdorff}{Real projective space is Hausdorff}
    The sphere \(S^n\) is Hausdorff, and the action of \(\mathbb{Z}_2\) on \(S^n\) is
    proper and discontinuous.
    Therefore the quotient \(S^n / \mathbb{Z}_2\), which is homeomorphic to
    \(\mathbb{P}^n(\realnumbers)\), is Hausdorff.
\end{snippetproof}

\subsection{Complex spaces}

\begin{snippetdefinition}{complex-projective-space-definition}{Complex projective space}
    The \emph{complex projective space of dimension \(n\)} is defined as
    \[
        \mathbb{P}^n(\complexnumbers) \triangleq (\complexnumbers^{n+1} \difference \{0\}) /_\sim
    \]
    where \(x \sim y\) if and only if \(\exists \lambda \in \complexnumbers \difference \{0\} \suchthat y = \lambda x\).
\end{snippetdefinition}

\end{document}
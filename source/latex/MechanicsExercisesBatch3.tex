\documentclass[preview]{standalone}

\usepackage{amsmath}
\usepackage{amssymb}
\usepackage{stellar}
\usepackage{definitions}
\usepackage{bettelini}

\begin{document}

\id{mechanics-ex-3}
\genpage

\section{Exercises - Batch 3}

\begin{snippetexercise}{mechanics-ex-3.1}{\underline{3.1}}
    \todo
\end{snippetexercise}

\begin{snippetsolution}{mechanics-ex-3.1-sol}{\underline{3.1}}
    \todo
\end{snippetsolution}

\begin{snippetexercise}{mechanics-ex-3.2}{\underline{3.2} Rotating platform}
    A platform rotates with angular velocity
    \(\omega\) around a vertical central axis.
    At the instant \(t = 0\), a ball is launched horizontally with velocity \(v_0\)
    from the centre of the platform; the friction the ball encounters is negligible,
    so that it moves with respect to the earth in uniform rectilinear motion with velocity \(v_0\).
    Determine the acceleration of the ball, at a generic instant, with respect to a reference system
    integral with the platform.
\end{snippetexercise}

\begin{snippetsolution}{mechanics-ex-3.2-sol}{\underline{3.2} Rotating platform}
    The position is given by
    \[
        \vec{R}(t) = \begin{pmatrix}
            v_0t\cos(\omega t) \\
            v_0t\sin(\omega t)
        \end{pmatrix}
    \]
    The velocity is given by
    \[
        \vec{v}(t) = v_0 \begin{pmatrix}
            -\omega t \sin(\omega t) + \cos(\omega t) \\
            \omega t \cos(\omega t) + \sin(\omega t)
        \end{pmatrix}
    \]
    The acceleration is given by
    \[
        \vec{a}(t) = v_0 \omega \begin{pmatrix}
            -\omega t \cos(\omega t) - 2\sin(\omega t) \\
            -\omega t \sin(\omega t) + 2\cos(\omega t)
        \end{pmatrix}
    \]
    and its module
    \begin{align*}
        |\vec{a}(t)| &= v_0 \omega \sqrt{
            {(-\omega t \cos(\omega t) - 2\sin(\omega t))}^2 + {(-\omega t \sin(\omega t) + 2\cos(\omega t))}^2
        } \\
        &= v_0 \omega \sqrt{4 + \omega^2t^2}
    \end{align*}
\end{snippetsolution}

\begin{snippetexercise}{mechanics-ex-3.3}{\underline{3.3}}
    \todo
\end{snippetexercise}

\begin{snippetsolution}{mechanics-ex-3.3-sol}{\underline{3.3}}
    \todo
\end{snippetsolution}

\begin{snippetexercise}{mechanics-ex-3.4}{\underline{3.4}}
    \todo
\end{snippetexercise}

\begin{snippetsolution}{mechanics-ex-3.4-sol}{\underline{3.4}}
    \todo
\end{snippetsolution}

\begin{snippetexercise}{mechanics-ex-3.5}{\underline{3.5}}
    \todo
\end{snippetexercise}

\begin{snippetsolution}{mechanics-ex-3.5-sol}{\underline{3.5}}
    \todo
\end{snippetsolution}

\begin{snippetexercise}{mechanics-ex-3.6}{\underline{3.6}}
    \todo
\end{snippetexercise}

\begin{snippetsolution}{mechanics-ex-3.6-sol}{\underline{3.6}}
    \todo
\end{snippetsolution}

\end{document}
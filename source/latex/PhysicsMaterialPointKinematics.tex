\documentclass[preview]{standalone}

\usepackage{amsmath}
\usepackage{amssymb}
\usepackage{stellar}
\usepackage{definitions}
\usepackage{bettelini}

\begin{document}

\title{Stellar}
\id{material-point-kinematics}
\genpage

\section{Kinematics of a material point}


\begin{snippetdefinition}{position-definition}{Position of a material point in a 3D space}
    The position of a particle is defined by a \snippetref[parametric-curve-definition][parametric curve] $\mathbf{r} \colon I \fromto \realnumbers^3 $, where $I \subseteq \realnumbers$ is a non-empty interval.
    \[
        \mathbf{r}(t) = \begin{bmatrix}
            r_x(t) \\ r_y(t) \\ r_z(t)
        \end{bmatrix} 
    \] 
\end{snippetdefinition}


\begin{snippetdefinition}{velocity-definition}{Velocity of a material point in a 3D space}
    The velocity of a particle is defined by a parametric curve $\mathbf{v} \colon I \fromto \realnumbers^3 $, where $I \subseteq \realnumbers$ is a non-empty interval. \\
    The velocity is the first derivative with respect to the time $t$ of the position $\mathbf{r}(t)$ 
    \begin{align*}
        \mathbf{v}(t) &= \lim_{h \to 0} \frac{\mathbf{r}(t + ht) - \mathbf{r}(t)}{h} = 0 \\
        \mathbf{v}(t) &= \frac{d\mathbf{r}}{dt} \\
        \mathbf{v}(t) &= \dot{\mathbf{r}} \\
        \mathbf{v}(t) &= \begin{bmatrix}
            v_x(t) \\ v_y(t) \\ v_z(t)
        \end{bmatrix}
    \end{align*}
\end{snippetdefinition}

\includesnpt[width=75\%|src=/snippet/static/speed-definition-visualization.png]{centered-img}


\begin{snippetdefinition}{acceleration-definition}{Acceleration of a material point in a 3D space}
    The acceleration of a particle is defined by a parametric curve $\mathbf{a} \colon I \fromto \realnumbers^3 $, where $I \subseteq \realnumbers$ is a non-empty interval. \\
    The acceleration is the first derivative with respect to the time $t$ of the velocity $\mathbf{v}(t)$, and the second derivative with respect to $t$ of the position $\mathbf{x}(t)$.
    \begin{align*}
        \mathbf{a}(t) &= \lim_{h \to 0} \frac{\mathbf{v}(t + ht) - \mathbf{v}(t)}{h} = 0 \\
        \mathbf{a}(t) &= \frac{d\mathbf{v}}{dt} = \frac{d^2 \mathbf{r}}{dt^2} \\
        \mathbf{a}(t) &= \dot{\mathbf{v}} = \ddot{\mathbf{r}} \\
        \mathbf{a}(t) &= \begin{bmatrix}
            a_x(t) \\ a_y(t) \\ a_z(t)
        \end{bmatrix}
    \end{align*} 
\end{snippetdefinition}

\includesnpt[width=75\%|src=/snippet/static/acceleration-definition-visualization.png]{centered-img}



\begin{snippettheorem}{newtwon-theorem}{Newton's laws}
    Newton's laws are three and they put a relationship between forces and the motion of a body. \\
    1. A body has acceleration if and only if there's a force exerted on it.  \\
    2. $F = ma = \frac{d}{dt}\left(m \mathbf{v} \right)$ \\
    3. When two bodies apply forces on each other, these forces are equal in magnitude but opposite in direction.
\end{snippettheorem}


\begin{snippetproposition}{linear-motion-definition}{Linear Motion}
    The linear motion can be either uniform or non-uniform. A uniform linear motion is characterised by a constant velocity, therefore no acceleration, while a non-uniform linear motion has variable velocity. \\
    $$ \begin{array}{|l|l|}
        \hline \text {  } & \text { Equation } \\
        \hline \text { Velocity } & v=v_i+a\left(t-t_i\right) \\
        \hline \text { Equation of motion } & r=\frac{1}{2} a\left(t-t_i\right)^2+v_i\left(t-t_i\right)+r_i \\
        \hline \text { Velocity at } t_0=0 & v=v_0+a t \\
        \hline \text { Equation of motion at } t_0=0 & r=\frac{1}{2} a t^2+v_0 t+r_0 \\
        \hline
    \end{array} $$
\end{snippetproposition}

\begin{snippetproof}{linear-motion-proof}{linear-motion-definition}{Linear Motion Proof }
    considering $a$ as a constant. \\
    For the velocity:
     $$ F = ma = m\dot{v} $$
     $$ a = \dot{v} $$
     $$ \integral[t_i][t][a][t] = \integral[t_i][t][\dot{v}][t] $$
     $$ a(t - t_i) = v - v_i \Longleftrightarrow v = a(t - t_i) + v_i $$
    For the position
    $$ \frac{F}{m} = a = \ddot{r} $$
    $$ \integral[t_i][t][a][t] = \integral[t_i][t][\dot{v}][t] $$
    $$ a(t - t_i) = v - v_i \Longleftrightarrow v = a(t - t_i) + v_i $$ 
    $$ \integral[t_i][t][a(t - t_i)][t] = \integral[t_i][t][\left( v - v_i \right)][t] = \integral[t_i][t][\left( \dot{x} - v_i \right)][t] \Longleftrightarrow r=\frac{1}{2} a\left(t-t_i\right)^2+v_i\left(t-t_i\right)+r_i $$
\end{snippetproof}






\end{document}

\documentclass[preview]{standalone}

\usepackage{amsmath}
\usepackage{amssymb}
\usepackage{stellar}
\usepackage{definitions}
\usepackage{bettelini}
\usepackage{wrapfig}

\begin{document}

\id{italiano-biografia-dante}
\genpage

\section{Biografia}

\begin{snippet}{biografia-dante}
    La biografia di Dante è molto offuscata e nessuno scritto originale è rimasto.
    Questi fattori rendono difficile datare le sue varie opere e contestualizzarle. Inoltre,
    è anche difficile validare in maniera precisa le parole esatte scritte dall'autore, siccome i testi che possediamo
    sono frutto di trascrizioni.
    
    Dante nasce a Firenze nel 1265 in una piccola nobiltà cittadina.
    A 12 anni diventa promesso sposo di a Gemma Donati.
    
    A 18 anni incontra Beatrice, dopo averla vista per la prima volta a 9 anni.
    
    Verso l'anno 1295 Dante, si avvicina alla politica.
    Si iscrive all'Arte (Arte dei medici e degli speziali), questo è dato dal fatto che essere iscritti ad un Arte
    fosse un requisito necessario per esercitare un'attività politica. Dante diventa Priore, per cui a capo della cittadina. %%
    
    Prima della nascita di Dante, i Ghibellini sostenevano il potere dell'imperatore, mentre i Guelfi sostenevano quello papa.
    Le due parti erano in forte conflitto, e nella battaglia del 1266, muore il figlio dell'imperatore.
    I Ghibellini escono quindi di scena quando Dante è appena nato.
    
    Successivamente, i Guelfi si separano in Bianchi e Neri, con un conflitto ancora più forte di quello precedente.
    
    La scena politica fiorentina era dominata dallo scontro fra i Bianchi e i Neri.
    Dante, durante il suo priorato, manda in esilio i più violenti dei Neri, fino alla scaduta del suo priorato.
    Papa Bonificio VIII manda le truppe di Carlo di Valois, le quali permettono ai Neri di prendere carica al governo.
    I bianchi vengono quindi esiliati, fra cui Dante.
\end{snippet}

\end{document}
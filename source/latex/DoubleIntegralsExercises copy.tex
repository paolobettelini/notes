\documentclass[preview]{standalone}

\usepackage{amsmath}
\usepackage{amssymb}
\usepackage{stellar}
\usepackage{definitions}
\usepackage{bettelini}

\begin{document}

\id{double-integrals-exercises}
\genpage

\section{Exercises}

\plain{TODO: exercise sheet.}

\begin{snippetexercise}{double-integrals-ex1}{}
    Let \(Q = [0,1] \cartesianprod [0,3]\) and \[
        f(x,y) = \begin{cases}
            2xy & y \leq x^2 \\
            x^3 + 2x^2y & y > x^2
        \end{cases}
    \]
    Compute
    \[
        \iint\nolimits_Q f(x,y) \,\text{d}x\,\text{d}y
    \]
\end{snippetexercise}

\begin{snippetsolution}{double-integrals-ex1-sol}{}
    This boundary is an integrable function and therefore negligible,
    since it is a set of measure zero; hence \(f\) is integrable.
    The function \(f(x, -)\) is integrable since it is continuous except possibly at a single point.
    Therefore, Fubini's theorem applies. We thus have
    \begin{align*}
        \iint\nolimits_Q f(x,y) \,\text{d}x\,\text{d}y &= \integral[0][1][
            \integral[0][3][
                f(x,y)
            ][y]
        ][x] \\
        &= \integral[0][1][
            \left(
            \integral[0][x^2][
                2xy
            ][y] + 
            \integral[x^3][3][
                (x^3 + 2x^2y)
            ][y]
            \right)
        ][x] \\
        &= \integral[0][1][
            3x^3 + 9x^2 - x^6
        ][x]
    \end{align*}
\end{snippetsolution}

\begin{snippetexercise}{double-integrals-ex2}{}
    Let \(Q = [1,2] \cartesianprod [0,1]\) and \(f(x,y) = x^{-3}e^{\frac{y}{x}}\).
    Compute
    \[
        \iint\nolimits_Q f(x,y) \,\text{d}x\,\text{d}y
    \]
\end{snippetexercise}

\begin{snippetsolution}{double-integrals-ex2-sol}{}
    We have no issues and the function is continuous on the domain, so it is integrable.
    It is also bounded since it is defined on a compact set. We first integrate with respect to \(y\):
    \begin{align*}
        \iint\nolimits_Q f(x,y) \,\text{d}x\,\text{d}y &= \integral[1][2][
            \integral[0][1][
                x^{-3}e^{\frac{y}{x}}
            ][y]
        ][x] \\
        &= \integral[1][2][
            x^{-2}(e^{\frac{1}{x}} - 1)
        ][x] \\
        &= \integral[1][2][x^{-2}e^{\frac{1}{x}}][x]
        - \integral[1][2][x^{-2}][x] \\
        &= -\frac12 + \sqrt{e} + e
    \end{align*}
\end{snippetsolution}

\begin{snippetexercise}{double-integrals-ex3}{}
    Compute the volume of the solid bounded by
    \(z = xy\) over the domain
    \[
        D = \{
            (x,y) \in \realnumbers^2 \suchthat 0 \leq y \leq \frac{3}4 \land x^2 + y^2 - 25 \leq 0    
        \}
    \]
\end{snippetexercise}

\begin{snippetsolution}{double-integrals-ex3-sol}{}
    We need to take the absolute value to ensure a positive result
    \[
        \iint\nolimits_D |f(x,y)| \,\text{d}x\,\text{d}y 
    \]
    The domain is a circle of radius \(5\) centered at the origin,
    cut by a line (and by the x-axis).
    We are therefore in the first quadrant.
    We note that the point \((4, 3)\) lies on the circumference.
    We slice this area horizontally:
    \begin{align*}
        \iint\nolimits_D |f(x,y)| \,\text{d}x\,\text{d}y 
        &= \integral[0][3][
            y\integral[\frac43y][\sqrt{25 - y^2}][
                x
            ][x]
        ][y] \\
        &= \integral[0][3][
            \frac{25}{2} \left(1 - \frac19 y^2\right)
        ][y] \\
        &= \frac{225}{8}
    \end{align*}
\end{snippetsolution}

\begin{snippetexercise}{double-integrals-ex4}{}
    Compute
    \[
        \iint\nolimits_D f(x,y) \,\text{d}x\,\text{d}y 
    \]
    where
    \[
        D = \{
            (x,y) \suchthat 0 \leq x \leq \pi \land \sqrt{x} < y \sqrt{\pi}    
        \}
    \]
    and \(f(x,y) = \frac{\sin(y^2)}{y}\).
\end{snippetexercise}

\begin{snippetsolution}{double-integrals-ex4-sol}{}
    We note that the denominator never vanishes, but the origin is an accumulation point,
    so we can approach it. We need to verify that \(f\) is bounded.
    Certainly \(f > 0\), moreover
    \[
        \frac{y^2}{y} \leq \frac{y^2}{y} = y \to 0
    \]
    so it is bounded, and at the origin it can be extended by continuity.
    \begin{align*}
        \iint\nolimits_D |f(x,y)| \,\text{d}x\,\text{d}y 
        &= \integral[0][\sqrt\pi][
            \integral[0][y^2][
                \frac{\sin(y^2)}{y}
            ][x]
        ][y] \\
        &= \integral[0][\sqrt\pi][
            \sin(y^2)y
        ][y] \\
        &= \frac12(-\cos \pi + \cos 0) = 1
    \end{align*}
\end{snippetsolution}

\begin{snippetexercise}{double-integrals-ex5}{}
    Compute, for \(\alpha > 1\),
    \[
        \iint\nolimits_D \frac{x^2}{y} \,\text{d}x\,\text{d}y 
    \]
    where 
    \[
        D = \{
            (x,y) \in \realnumbers^2 \suchthat  x> 0 \land x^\alpha < y < x
        \}
    \]
\end{snippetexercise}

\begin{snippetsolution}{double-integrals-ex5-sol}{}
    Since \(x^\alpha < x\), we must have \(x < 1\).
    The domain is therefore the intersection in the first quadrant between \(y=x\)
    and \(y=x^\alpha\).
    In general, the limit of this function at the origin does not exist, but we need to consider only
    the directions that lie within the domain.
    We can note that
    \[
        f(x,y) = \frac{x^2}{y} \leq \frac{x^2}{x^\alpha} = x^{2-\alpha}
    \]
    so if \(\alpha \leq 2\) everything is fine.
    If \(\alpha > 2\), the function is unbounded.
    Indeed, if \(\varepsilon > 0\) is such that
    \(2 - 2 + \varepsilon < \alpha\), then \(f(t, t^{2 + \varepsilon}) = 1 / t^\varepsilon \to \infty\).
    Thus the function is bounded only for \(1 < \alpha \leq 2\).
    We then have 
    \begin{align*}
        \iint\nolimits_D |f(x,y)| \,\text{d}x\,\text{d}y 
        &= \integral[0][1][
            \integral[x^\alpha][x][
                \frac{x^2}{y}
            ][y]
        ][x] \\
        &= (1-\alpha) \integral[0][1][
            x^2\ln(x)
        ][x] \\
        &= \frac{\alpha - 1}{9}
    \end{align*}
\end{snippetsolution}

\begin{snippetexercise}{double-integrals-ex6}{}
    Exchange the order of integration in
    \[
        \integral[-1][1][
            \integral[|x|][\sqrt{2-x^2}][f(x,y)][y]
        ][x]
    \]
\end{snippetexercise}

\begin{snippetsolution}{double-integrals-ex6-sol}{}
    We have that \(|x| = \sqrt{2-x^2}\) if and only if \(x = \pm 1\).
    The integration domain is therefore an upward circular slice.
    To slice it horizontally we need to split it into two parts:
    \begin{align*}
        \iint\nolimits_D |f(x,y)| \,\text{d}x\,\text{d}y 
        &= \integral[0][1][
            \integral[-y][y][
                f(x,y)
            ][x]
        ][y] + \integral[1][\sqrt 2][
            \integral[-\sqrt{2-y^2}][\sqrt{2 - y^2}][
                f(x,y)
            ][x]
        ][y]
    \end{align*}
\end{snippetsolution}

\begin{snippetexercise}{double-integrals-ex7}{}
    Compute
    \[
        \iint\nolimits_{\{x^2 + y^2 < 1\}} (x-y)\cos(x^2 + y^2) \,\text{d}x\,\text{d}y 
    \]
\end{snippetexercise}

\begin{snippetsolution}{double-integrals-ex7-sol}{}
    Note that \(f(x,y) = -f(-x, -y)\) and the domain is symmetric, and thus the integral is zero.
\end{snippetsolution}

\end{document}
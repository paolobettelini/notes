\documentclass[preview]{standalone}

\usepackage{amsmath}
\usepackage{amssymb}
\usepackage{parskip}
\usepackage{fullpage}
\usepackage{hyperref}
\usepackage{wrapfig}
\usepackage{pgfplots}
\usepackage{bettelini}
\usepackage{makecell}
\usepackage{stellar}
\usepackage{definitions}

\makeatletter 
\tikzset{ 
reuse path/.code={\pgfsyssoftpath@setcurrentpath{#1}} 
} 
\tikzset{even odd clip/.code={\pgfseteorule}, 
protect/.code={ 
\clip[overlay,even odd clip,reuse path=#1] 
(current bounding box.south west) rectangle (current bounding box.north east)
; 
}} 
\makeatother

\usetikzlibrary{3d,arrows.meta,decorations.markings,perspective}
\tikzset{->-/.style={decoration={% https://tex.stackexchange.com/a/39282/194703
  markings,
  mark=at position #1 with {\arrow{>}}},postaction={decorate}},
  ->-/.default=0.55}

\begin{document}
  
\id{complexanalysis-definitions}
\genpage

\section{Riemann Spheres}

\begin{snippetdefinition}{riemann-sphere-definition}{Riemann Sphere}
    A \textit{Riemann Sphere} is a unit sphere used to represent the complex plane
    using stereographic projection.
    The Riemann sphere lays on the complex plane. A complex number
    is represented by the intersection between the sphere
    and a ray starting from the topmost point of the sphere
    and intersecting with the given complex number on the complex plane.
\end{snippetdefinition}

\begin{snippet}{riemann-sphere-illustration}
\begin{center}
\pgfmathsetmacro{\myaz}{15}
\begin{tikzpicture}[
        declare function={%
            stereox(\x,\y)=2*\x/(1+\x*\x+\y*\y);
            stereoy(\x,\y)=2*(\y+1)/(1+\x*\x+(\y+1)*(\y+1)) + 1;
            stereoz(\x,\y)=(-1+\x*\x+\y*\y)/(1+\x*\x+\y*\y);
            Px=1.75;
            Py=-1.5;
            Qx=-1.0;
            Qy=-1.0;
            amax=2.5;
        },
        scale=2.5,
        line join=round,line cap=round,
        dot/.style={circle,fill,inner sep=1pt},>={Stealth[length=1.2ex]}]
    \pgfdeclarelayer{background}
    \pgfdeclarelayer{foreground} 
    \pgfsetlayers{background,main,foreground}
    \path[save path=\pathSphere,ball color=gray,fill opacity=0.6] 
        (0, 1, 0) circle[radius=1];
    \begin{scope}[3d view={\myaz}{15}]
        \draw (-amax,amax) -- (-amax,-amax) coordinate (bl) -- (amax,-amax) 
            coordinate (br)-- (amax,amax);
        \begin{scope}
            \tikzset{protect=\pathSphere}
            \draw (-amax,amax) -- (amax,amax);
        \end{scope}
        \begin{scope}
            \clip[reuse path=\pathSphere];
            \draw[dashed] (-amax,amax) -- (amax,amax);
        \end{scope}
        \begin{scope}[canvas is xy plane at z=0]
            \path[save path=\pathPlane] (\myaz:amax) -- (\myaz+180:amax) --(bl) -- (br) -- cycle;
            \begin{pgfonlayer}{background}
                \fill[blue!30,fill opacity=0.6]
                    (\myaz:1) arc[start angle=\myaz,end angle=\myaz-180,radius=1]
                    -- (-amax,0) -- (-amax,amax) -- (amax,amax) -- (amax,0) -- cycle;
            \end{pgfonlayer}
            \fill[blue!30,fill opacity=0.6]
                (\myaz:1) arc[start angle=\myaz,end angle=\myaz-180,radius=1]
                -- (-amax,0) -- (-amax,-amax) -- (amax,-amax) -- (amax,0) -- cycle;
        \end{scope}

        %\draw[-] (0,-2,0) node[dot,label=below:{$z$}](z){}
        %    -- node[auto,pos=0.5]{\(p\)} (0, -1, 1)
        %    node[dot,label=below right:{\(s\)}](z*){};

        \draw[-] (-2,0,0) node[dot,label=below:{$z$}](z){}
            -- node[auto,pos=0.5]{} (-1, 0, 1)
            node[dot,label=below right:{\(s\)}](z*){};
  
        \begin{pgfonlayer}{background} 
            \draw[dashed] (z*) -- (0,0,2) node[dot,label=above:{$\zeta$}](zeta){};
        \end{pgfonlayer}
    \end{scope}
\end{tikzpicture}
\end{center}
\end{snippet}

\end{document}
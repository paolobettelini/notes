\documentclass[preview]{standalone}

\usepackage{amsmath}
\usepackage{amssymb}
\usepackage{parskip}
\usepackage{fullpage}
\usepackage{hyperref}
\usepackage{tikz}
\usepackage{wrapfig}
\usepackage{bettelini}
\usepackage{stellar}

\hypersetup{
    colorlinks=true,
    linkcolor=black,
    urlcolor=blue,
    pdftitle={Theory of Computation},
    pdfpagemode=FullScreen,
}

% =======
\usetikzlibrary{ % tikz packages
    automata,positioning,
    arrows.meta,bending
}
\tikzset{every state/.style={
    inner sep=2pt,
    minimum size=4pt
}}
\tikzset{>=stealth}  %latex, to, stealth
% Empty string symbol.
\newcommand{\emptyString}{\lambda}
% =======

\begin{document}

\id{theoryofcomputation-regular-expressions}
\genpage

\section{Regular Expressions}

\begin{snippetdefinition}{regular-expression-definition}{Regular Expression}
    A \textit{regular expression} is a mean to express a language.
    Let \(\Sigma\) be an alphabet.
    The following constants are defined as regular expressions:
    \begin{enumerate}
        \item \textbf{empty set:} \(\emptyset\), denoting the set \(\emptyset\);
        \item \textbf{empty string:} \(\emptyString\), denoting the set \(\{\emptyString\}\);
        \item \textbf{literal character:} \(a\), where \(a \in \Sigma\), denoting the set \(\{a\}\).
    \end{enumerate}
    Let \(R_1\) and \(R_2\) be regular expressions.
    Then \(R_1^*\), \(R_1 R_2\) and \(R_1 \cap R_2\) are also defined as regular expressions.
    % TODO need to explicit state what those things are since we haven't defined
    % these operators for regular expressions
    % https://en.wikipedia.org/wiki/Regular_expression#Formal_language_theory
\end{snippetdefinition}


\section{Properties}

\begin{snippet}{regular-expressions-properties}
    Let \(R_1\) be a regular expression describing \(L_1\) and \(R_2\) a regular expression
    describing \(L_2\).

    \begin{itemize}
        \item \(\emptyString\) is a regular expression describing  \(\{\emptyString\}\)
        \item \(\emptyset\) is a regular expression describing \(\emptyset\)
        \item \(\emptyset^*\) is a regular expression describing \(\{\emptyString\}\)
        \item Let \(\Sigma\) be a non-empty alphabet, \(\forall a \in \Sigma, a\) is a regular expression describing \(\{a\}\)
        \item \(R_1R_2\) is a regular expression describing \(L_1L_2\)
        \item \(R_1\union R_2\) is a regular expression describing \(L_1\union L_2\)
        \item \(R_1\intersection R_2\) is a regular expression describing \(L_1\intersection L_2\)
        \item \(R_1^*\) is a regular expression describing \(L_1^*\)
        \item \(\bar{R_1}\) is a regular expression describing \(\bar{L_1}\)
    \end{itemize}
    If \(L_1 = L_2\), then we say \(R_1 = R_2\) (e.g. \(\emptyString = \emptyset^*\)).

    Let \(R_1, R_2\) and \(R_3\) be regular expressions
    \begin{itemize}
        \item \(R_1 \emptyset = \emptyset R_1 = \emptyset\)
        \item \(R_1 \emptyString = \emptyString R_1 = R_1\)
        \item \(R_1 \union R_2 = R_2 \union R_1\)
        \item \(R_1 \union \emptyset = R_1\)
        \item \(R_1 \union R_1 = R_1\)
        \item \(R_1(R_2 \union R_3) = R_1R_2 \union R_1R_3\)
        \item \((R_1 \union R_2)R_3 = R_1R_3 \union R_2R_3\)
        \item \(R_1(R_2R_3) = (R_1R_2)R_3\)
        \item \(\emptyset^*=\emptyString\)
        \item \(\emptyString^*=\emptyString\)
        \item \((\emptyString \union R_1)^* = R_1^*\)
        \item \((\emptyString \union R_1)(\emptyString \union R_1)^* = R_1^*\)
        \item \(R_1^*(\emptyString \union R_1)=(\emptyString \union R_1)R_1^* = R_1^*\)
        \item \(R_1^*R_2 \union R_2 = R_1^*R_2\)
        \item \(R_1(R_2R_1)^*=(R_1R_2)^*R_1\)
        \item \((R_1 \union R_2)^* = (R_1^*R_2)^*R_1^* = (R_2^*R_1)^*R_2^*\)
    \end{itemize}
\end{snippet}

\end{document}
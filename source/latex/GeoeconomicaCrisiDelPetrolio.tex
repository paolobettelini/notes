\documentclass[preview]{standalone}

\usepackage{amsmath}
\usepackage{amssymb}
\usepackage{stellar}
\usepackage{bettelini}

\hypersetup{
    colorlinks=true,
    linkcolor=black,
    urlcolor=blue,
    pdftitle={Stellar},
    pdfpagemode=FullScreen,
}

\begin{document}

\title{Geografia economica}
\id{geoeconomica-crisi-del-petrolio}
\genpage

\begin{snippet}{fa934d02-aa82-43c4-9f95-cbbc31b7a99a}
    Le principali crisi del petrolio si sono verificate nel 1973 e nel 1979.

    La crisi del petrolio si riferisce a diversi eventi di instabilità e aumento dei prezzi del petrolio greggio che hanno avuto un impatto significativo sull'economia mondiale. Le principali crisi del petrolio si sono verificate nel 1973 e nel 1979.
    
    \begin{enumerate}
        \item La crisi del 1973 è stata scatenata da una serie di eventi, tra cui la guerra del Kippur tra Israele e i paesi arabi, in particolare l'OPEC, e la decisione dell'OPEC di imporre un embargo petrolifero contro gli Stati che sostenevano Israele. Questo ha portato a un repentino aumento dei prezzi del petrolio e a una grave crisi energetica, con carenze e aumenti dei prezzi dei carburanti, influenzando l'economia mondiale.
        \item La crisi del 1979 è stata causata principalmente dalla rivoluzione iraniana e dalla conseguente interruzione delle esportazioni petrolifere dall'Iran. Ciò ha ridotto l'offerta globale di petrolio e ha portato a un altro aumento dei prezzi e a ulteriori tensioni sul mercato energetico mondiale.
    \end{enumerate}
    
    Entrambe le crisi petrolifere hanno avuto impatti significativi sull'economia globale, portando a recessioni, inflazione e cambiamenti nelle politiche energetiche dei paesi, con un maggiore interesse per fonti energetiche alternative e per la diversificazione delle fonti di energia.
    La crisi del petrolio contribuisce a terminare il periodo dei 30 gloriosi.
\end{snippet}

\end{document}
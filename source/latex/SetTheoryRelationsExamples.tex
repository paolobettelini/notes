\documentclass[preview]{standalone}

\usepackage{amsmath}
\usepackage{amssymb}
\usepackage{stellar}
\usepackage{definitions}

\begin{document}

\id{settheory-relations-examples}
\genpage

\section{Equivalence classes}

\begin{snippetexample}{equivalence-relation-example-1}{Equivalence relation}
    Given a \set \(A\), the equality relation is an \equivrelation \(\sim\) on \(A\)
    \[ \{ (a,a) \suchthat a \in A \} \]
\end{snippetexample}

\begin{snippetexample}{equivalence-relation-example-2}{Equivalence relation}
    Given a \set \(A\), the \binrelation where every element is in relation with every other element
    is an \equivrelation \(\sim\) on \(A\)
    \[ \{ (a,b) \suchthat a,b \in A \} \]
\end{snippetexample}

\begin{snippetexample}{equivalence-relation-example-3}{Equivalence relation}
    The \binrelation \(\sim\) on \(\naturalnumbers\) defined as
    \[ \{ (a,b) \in \naturalnumbers^2 \suchthat a+b = 2k, k \in \naturalnumbers \} \]
    is an \equivrelation.
    The only non-trivial property is transitivity; if \(a + b\) and \(b+c\) are even, then so is \(a+c+2b\).
    But since \(2b\) is even, then \(a+c\) must also be even.
\end{snippetexample}

\plain{The same would not work if the sum needed to be odd, as the relation would not be reflexive and transitive.}

\end{document}
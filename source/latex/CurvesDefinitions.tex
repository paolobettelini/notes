\documentclass[preview]{standalone}

\usepackage{amsmath}
\usepackage{amssymb}
\usepackage{stellar}
\usepackage{definitions}
\usepackage{bettelini}
\usepackage{tikz}

\begin{document}

\id{curves-definitions}
\genpage

\section{Definitions}

\begin{snippetdefinition}{curve-definition}{Curve}
    A \emph{curve} is a continuous \function
    \(\varphi \colon [a,b] \subseteq \realnumbers \fromto \realnumbers^n\).
    The image \(\varphi([a,b])\) is said to be the
    \emph{support} of the curve. 
\end{snippetdefinition}

\begin{snippetdefinition}{closed-curve-definition}{Closed curve}
    A \curve \(\varphi\colon [a,b] \subseteq \realnumbers \fromto \realnumbers^n\) is said to
    be \emph{closed} if \(\varphi(a) = \varphi(b)\).
\end{snippetdefinition}

\begin{snippetdefinition}{simple-curve-definition}{Simple curve}
    A \curve \(\varphi\colon [a,b] \subseteq \realnumbers \fromto \realnumbers^n\)
    is said to be \emph{simple} if
    \[
        a \leq t_1 < t_2 \leq b \land \varphi(t_1) = \varphi(t_2)
        \implies t_1 = a \land t_2 = b
    \]
\end{snippetdefinition}

\plain{This means that the curve can only intersect at the ends.}

\begin{snippetdefinition}{regular-curve-definition}{Regular curve}
    A \curve \(\varphi\colon I \subseteq \realnumbers \fromto \realnumbers^n\)
    is said to be \emph{regular} if
    \(\varphi \in continuityclass^1(I)\)
    and \(||\varphi'(t)|| \neq 0\) for every \(t\in I\).
\end{snippetdefinition}

\plain{This means that all the components are diferentiable and with a continuous derivative,
and the derivative (or the norm) is never null.}

\begin{snippetdefinition}{curve-tangent-definition}{Tangent of a curve}
    A line \(r(t)\) is \emph{tangent} to a \curve \(\varphi\) at \(x_0 = \varphi(t_0)\)
    if
    \[
        \varphi(t) - r(t) = \littleO(t-t_0)
    \]
\end{snippetdefinition}

\plain{This is an approximation of the curve of the first order.}

\begin{snippet}{curve-c1-class-expl}
    If the \curve is in \(\continuityclass^1(I)\) then \[
        \varphi(t) = \varphi(t_0) + \varphi'(t_0)(t-t_0) + \littleO(t-t_0)
    \]
\end{snippet}

\begin{snippetdefinition}{piecewise regular curve-definition}{Piecewise regular curve}
    A \curve \(\varphi \colon [a,b] \fromto \realnumbers^n\) is said to be \emph{piecewise regular}
    if there exist a \partition \(P\) of the interval \([a,b]\) such that
    \(\varphi\) is \regularcurve on every interval of \(P\).
\end{snippetdefinition}

\begin{snippetdefinition}{curve-equivalence-definition}{Curve equivalence}
    Let \(\varphi_1 \colon I_1 \fromto \realnumbers^n\)
    and \(\varphi_2 \colon I_2 \fromto \realnumbers^n\) be two \curve[curves].
    We say that \(\varphi_1\)
    is \emph{equivalent} to \(\varphi_2\) (\(\varphi_1 \sim \varphi_2\)) if there exist
    a \function \(\sigma \colon I_2 \fromto I_1\) such that:
    \begin{enumerate}
        \item \(\sigma\) is \surjective;
        \item \(\sigma \in \continuityclass^1(I_2)\);
        \item \(\forall t \in I_2, \sigma'(t) \neq 0\);
        \item \(\varphi_2(t) = \varphi_1(\sigma(t))\).
    \end{enumerate}
\end{snippetdefinition}

\begin{snippetdefinition}{curve-orientation-definition}{Curve orientation}
    let \(\varphi_1\), \(\varphi_2\) be two \curve[curves] such that \(\varphi_1 \curveequivalent \varphi_2\)
    with \(\sigma(t)\).
    \begin{enumerate}
        \item if \(\sigma'(t) > 0\) then \(\varphi_1\) and \(\varphi_2\) have the same \emph{orientation}.
        \item if \(\sigma'(t) < 0\) then \(\varphi_1\) and \(\varphi_2\) have the opposite \emph{orientation}.
    \end{enumerate}
\end{snippetdefinition}

\begin{snippetdefinition}{curve-length-definition}{Length of a curve}
    let \(\varphi \colon [a,b] \fromto \realnumbers^n\) and let \(P\) be a \partition of \([a,b]\)
    and let
    \[
        L(\varphi, P) = \sum_{i=1}^n ||\varphi(t_i) - \varphi(t_{i-2})||
    \]
    with the euclidean norm.
    Then, \(\varphi\) is said to be \emph{rectifiable} if
    \[
        \xi = \sup_P L(\varphi, P) < \infty
    \]
    where \(P\) is the \set of all \partition[partitions] of \([a,b]\).
    In such case, we will call \(\xi\)
    the \emph{length of the curve \(\varphi\)}, denoted \(L(\varphi)\).
\end{snippetdefinition}

\section{Examples}

\begin{snippetexample}{tangent-curve-example1}{Tangent curve}
    let \(\varphi(t) = (t^2, t^3)\) from \([0,1] \fromto \realnumbers^2\) with \(t_0 = \frac{1}{2}\).
    The \curve is \regularcurve meaning that the tangent is
    \begin{align*}
        r(t) &= \varphi(t_0) + \varphi'(t_0)(t-t_0) \\
        &= \begin{pmatrix} \frac{1}{4} \\ \frac{1}{8} \end{pmatrix}
        + \begin{pmatrix} 2t_0 \\ 3t_0^2 \end{pmatrix}\left(t-\frac{1}{2}\right) \\
        &= \begin{pmatrix} \frac{1}{4} \\ \frac{1}{8} \end{pmatrix}
        + \begin{pmatrix} t - \frac{1}{4} \\ \frac{3}{4}t - \frac{1}{4} \end{pmatrix}
    \end{align*}
\end{snippetexample}

\begin{snippetexample}{curve-example1-example}{}
    Let \(\sigma \colon [0,1] \fromto [a,b]\) such that \(\sigma(t) = a + (b-a)t\)
    which is a line that maps \([0,1]\) into \([a,b]\).
    Let now \(\varphi \colon [a,b] \fromto \realnumbers^n\)
    and
    \[
        \hat{\varphi} = \varphi(\sigma(t)) = \varphi(a + (b-a)t)
    \]
    Clearly, \(\varphi \curveequivalent \hat{\varphi}\).
    Thus, the second curve is equivalent to the first but defined on \([0,1]\) rather than \([a,b]\).
\end{snippetexample}

\begin{snippetexample}{curve-example2-example}{}
    Consider \(\varphi(t) = (\cos t, \sin t)\) defined in
    \begin{enumerate}
        \item \([0, \pi]\) semplice, non chiusa;
        \item \([0, 2\pi]\) semplice, chiusa;
        \item \([0, 3\pi]\) non semplice, non chiusa;
        \item \([0, 4\pi]\) non semplice, chiusa;
    \end{enumerate}
    the last three have the same \curve[support].
\end{snippetexample}

\begin{snippetexample}{curve-example3-example}{}
    Let \(\varphi_1(t) = (t, \sqrt{1-t^2})\) in \([1/2, 1]\)
    and \(\varphi_2(t) = (\cos t, \sin t)\) in \([0, \pi/3]\).
    Clearly, \(\varphi_1 \curveequivalent \varphi_2\) with \(\sigma(t) \colon [0, \pi/3] \fromto [1/2, 1]\)
    given by \(\sigma(t) = \cos t\), thus
    \[
        \varphi_2(t) = \varphi_1(\sigma(t))
        = \varphi_2(\cos t)
    \]
\end{snippetexample}

\end{document}
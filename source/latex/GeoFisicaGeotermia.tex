\documentclass[preview]{standalone}

\usepackage{amsmath}
\usepackage{amssymb}
\usepackage{tikz}
\usepackage{stellar}
\usepackage{bettelini}

\hypersetup{
    colorlinks=true,
    linkcolor=black,
    urlcolor=blue,
    pdftitle={Assets},
    pdfpagemode=FullScreen,
}

\begin{document}

\title{Geotermia}
\id{geofisica-geotermia}
\genpage

\section{Geotermia}

\begin{snippetdefinition}{geotermia}{Geotermia}
    La \textit{geotermia} è la disciplina delle scienze della
    Terra che studia l'insieme dei fenomeni naturali coinvolti nella
    produzione e nel trasferimento di calore proveniente dall'interno della di un pianeta.
\end{snippetdefinition}

\plain{Già a 15 metri di profondità la temperatura del suolo
è costante durante tutto l'anno.
Generalmente in Svizzera la temperatura del suolo aumenta di circa 30°C
per chilometro di profondità.
Questa energia geotermica può essere utilizzata con l'ausilio di diversi metodi.
}

\end{document}
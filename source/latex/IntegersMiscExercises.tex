\documentclass[preview]{standalone}

\usepackage{amsmath}
\usepackage{amssymb}
\usepackage{stellar}
\usepackage{definitions}
\usepackage{bettelini}

\begin{document}

\id{integers-misc-exercises}
\genpage

\section{Exercises}

\begin{snippetexercise}{algebra-misc-ex1}{}
    Find the last digit in base \(10\) of
    \[
        \sum_{n=1}^{100} n\factorial
    \]
\end{snippetexercise}

\begin{snippetsolution}{algebra-misc-ex1-sol}{}
    Note that for \(n\geq 5\), \(n\factorial\) is a multiple of \(10\).
    So, \(10 + 4\factorial + 3\factorial + 2\factorial + 1\factorial = 43\) and the last digit is \(3\).
\end{snippetsolution}

\begin{snippetexercise}{algebra-misc-ex2}{}
    Prove that for \(n \in \integers\),
    \[
        n^3 + {(n+1)}^3 + {(n+2)}^3
    \]
    is a multiple of \(3\).
\end{snippetexercise}

\begin{snippetsolution}{algebra-misc-ex2-sol}{}
    \begin{align*}
        n^3 + {(n+1)}^3 + {(n+2)}^3 &= 3n^3 + 9n^2 + 15n + 9 \\
        &= 3(n^3 + 3n^2 + 5n + 3)
    \end{align*}
\end{snippetsolution}

\begin{snippetexercise}{algebra-misc-ex6}{}
    Find the remainder of \(17^{5\factorial} + 4^{134}\) divided by \(35\).
\end{snippetexercise}

\begin{snippetsolution}{algebra-misc-ex6-sol}{}
    Since \(17\) and \(35\) are \coprime,
    \[ 17^{25} \equiv 1 \pmod{35} \]
    given that \(\eulertotient(35)=\eulertotient(5)\eulertotient(7)=24\).
    We note that \(5\factorial\) is a multiple of \(24\) and thus
    \(17^{5\factorial} \equiv 1^5 \pmod{35} \equiv 1 \pmod{35}\).
    On the other hand, \(4^{134} = 2^{268}\).
    Again, \(2\) and \(35\) are \coprime so
    \(2^{24} \equiv 1 \pmod{35}\). Since \(268 \equiv 4 \pmod{24}\),
    we have \(2^{268} \equiv 2^4 \pmod{35}\) and finally,
    \(16+1=17 \equiv \pmod{35}\).
\end{snippetsolution}

\begin{snippetexercise}{algebra-misc-ex7}{}
    Prove that \(4^n+5^n\)
    is a multiple of \(9\) \ifandonlyif \(n\) is odd.
\end{snippetexercise}

\begin{snippetsolution}{algebra-misc-ex7-sol}{}
    We note that \(5\equiv -4 \pmod{9}\),
    so we need to prove \(4^n+{(-4)}^n \equiv 4^n + {(-1)}^n4^n \equiv 0 \pmod{9}\)
    \ifandonlyif \(n\) is odd. \\
    \iffproof{
        \begin{align*}
            4^n + {(-1)}^n4^n &\equiv 0 \pmod{9} \\
            4^n &\equiv -{(-1)}^n4^n \pmod{9} \\
            -1 &\equiv {(-1)}^n \pmod{9}
        \end{align*}
        which is only possible if \(n\) is odd. Thus, \(n\) is odd.
    }{
        If \(n\) is odd, then \({(-1)}^n=-1\) and so
        \[4^n - 4^n = 0 \equiv 0 \pmod{9}\]
    }
\end{snippetsolution}

\begin{snippetexercise}{algebra-misc-ex8}{Odd square division by 8}
    Let \(n\) be odd. Show that the remainder of \(a^2\) divided by \(8\) is \(1\).
\end{snippetexercise}

\begin{snippetsolution}{algebra-misc-ex8-sol1}{Odd square division by 8 - Induction}
    We write \(a = 2k+1\) for some \(k\in\integers\). Thus,
    \({(2k+1)}^2 = 4k^2 + 4k + 1 = 8l + 1\) for some \(l\in\integers\).
    We proceed by induction:
    \begin{itemize}
        \item the base case is \(4\cdot0^2 + 4\cdot 0 + 1 = 8l+1\) for \(l=0\);
        \item the induction case is
            \begin{align*}
                &\phantom{=}4{(k+1)}^2 + 4(k+1) + 1 \\
                &=4(k^2 + 2k + 1) + 4k + 4 + 1 \\
                &= 4k^2 + 8k + 4 + 4k + 4 + 1\\
                &= 4k^2 + 4k + 1 + 8k + 8 \\
                &= 8l+1 + 8k + 8 \\
                &= 8(l+k+1) + 1 \\
                &= 8j + 1
            \end{align*}
            for \(j = l+k+1\).
    \end{itemize}
\end{snippetsolution}

\begin{snippetsolution}{algebra-misc-ex8-sol2}{Odd square division by 8 - Product of consecutive numbers}
    We write \(a = 2k+1\) for some \(k\in\integers\).
    We note that \({(2k+1)}^2 = 4k^2 + 4k + 1 = 4k(k+1) + 1\).
    The term \(k(k+1)\) is the product of a number with its successor.
    Either \(k\) or \(k+1\) is a multiple of \(2\), so \(k(k+1)\) is a multiple
    of \(2\) and \(4k(k+1)\) is a multiple of \(8\), which leaves the remainder of \(1\).
\end{snippetsolution}

\end{document}
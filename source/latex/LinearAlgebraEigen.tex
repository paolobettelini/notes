\documentclass[preview]{standalone}

\usepackage{amsmath}
\usepackage{amssymb}
\usepackage{stellar}
\usepackage{definitions}
\usepackage{bettelini}

\begin{document}

\id{eigen-values-and-vectors}
\genpage

\section{Eigenvalues and eigenvectors}

\begin{snippetdefinition}{eigenvalues-eigenvectors-definition}{Eigenvalues and eigenvectors}
    The \textit{eigenvectors} \(\vec{v}\) and \textit{eigenvalues} \(\lambda\) of a matrix \(M\)
    are values such that
    \[
        M\vec{v} = \lambda \vec{v}
    \]
    for \(\vec{v}\neq \vec{0}\).
\end{snippetdefinition}

\plain{The null vectors are excluded as they are always a trivial solution.}

\begin{snippettheorem}{computing-eigenvectors-and-eigenvalues-theorem}{Computing eigenvectors and eigenvalues}
    Let \(M\) be a \matrix of size \(n\).
    The \eigenvalue[eigenvalues] of \(M\) can be found by computing the values \(\lambda\) for which
    \[ \det(M-\lambda \identmatrix{n}) = 0 \]
    Given an \eigenvalue \(\lambda\), its corresponding \eigenvector[eigenvectors] are
    obtained by solving the equation
    \[ (M - \lambda \identmatrix{n})\vec{v} = \vec{0} \]
\end{snippettheorem}

\begin{snippetproof}{computing-eigenvectors-and-eigenvalues-proof}{computing-eigenvectors-and-eigenvalues-theorem}{Computing eigenvectors and eigenvalues}
    Let \(M\) be a \matrix of size \(n\).
    We start with
    \[
        M\vec{v} = \lambda \vec{v}
    \]
    On the left-hand side, we can replace the scalar multiplier \(\lambda\) with
    a scaled identity matrix
    \begin{align*}
        M\vec{v} &= \lambda \identmatrix{n} \vec{v} \\
        M\vec{v} - \lambda \identmatrix{n} \vec{v} &= \vec{0} \\
        (M - \lambda \identmatrix{n})\vec{v} &= \vec{0}
    \end{align*}
    Since \(\vec{v} \neq \vec{0}\), the equation is only satisfied
    when \(M - \lambda \identmatrix{n}\) squishes the space into a low dimension. That is,
    \[ \det(M-\lambda \identmatrix{n}) = 0 \]
    Given a \(\lambda\) solution, we can plug it back into
    \[ (M - \lambda \identmatrix{n})\vec{v} = \vec{0} \] to find the values for \(\vec{v}\).
\end{snippetproof}

\end{document}
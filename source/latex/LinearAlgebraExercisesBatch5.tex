\documentclass[preview]{standalone}

\usepackage{amsmath}
\usepackage{amssymb}
\usepackage{stellar}
\usepackage{definitions}

\begin{document}

\id{linearalgebra-exercises-batch-5}
\genpage

\section{Exercises}

\begin{snippetexercise}{linear-algebra-batch-5-ex-1}{}
    Let \(U \subset \realnumbers^5\) be the linear subpace defined as
    \[
        U = \{(x_1, x_2, x_3, x_4, x_5) \in \realnumbers^5 \suchthat x = 3x2 \land x_3 = 7x_4\}
    \]
    \begin{enumerate}
        \item Find a \basis for \(U\);
        \item Expand the \basis of \(U\) to a \basis of \(\realnumbers^5\);
        \item Find a linear subspace \(W \subset \realnumbers^5\) such that \(U \circ  W= \realnumbers^5\);
    \end{enumerate}
\end{snippetexercise}

\begin{snippetsolution}{linear-algebra-batch-5-ex-1-sol}{}
    \begin{enumerate}
        \item We can rewrite \(U\) as \[U=\{(3x_2, x_2, 7x_4, x_4, x_5)\}\]
        a \basis for it is thus \[\mathbb{B}=\left\{
            \begin{pmatrix}3 \\ 1 \\ 0 \\ 0 \\ 0\end{pmatrix},
            \begin{pmatrix}0 \\ 0 \\ 7 \\ 1 \\ 0\end{pmatrix},
            \begin{pmatrix}0 \\ 0 \\ 0 \\ 0 \\ 1\end{pmatrix}
        \right\}\]
        \item in order to make this a \basis of \(\realnumbers^5\) we can add the vectors
        \[
            \hat{\mathbb{B}} = \left\{
                \begin{pmatrix}0 \\ 1 \\ 0 \\ 0 \\ 0\end{pmatrix},
                \begin{pmatrix}0 \\ 0 \\ 1 \\ 0 \\ 0\end{pmatrix}
            \right\}
        \]
        \item since the \vector[vectors] in the \basis are \linearlyindependent, we can take \(W = \hat{\mathbb{B}} \difference \mathbb{B}\).
    \end{enumerate}
\end{snippetsolution}

\begin{snippetexercise}{linear-algebra-batch-5-ex-2}{}
    Let \(\{v_1, v_2, v_3, v_4\}\) be a \basis of a \vectorspace over \(\complexnumbers\).
    Find \(a\in \complexnumbers\) such that \(\{v_1 + av_2, v_2 + av_3, v_3 + av_4, v_4 + av_1\}\).
\end{snippetexercise}

\begin{snippetsolution}{linear-algebra-batch-5-ex-2-sol}{}
    We need to find \(a\) such that  \(\{v_1 + av_2, v_2 + av_3, v_3 + av_4, v_4 + av_1\}\)
    is \linearlyindependent. We thus need
    \begin{align*}
        \alpha_1(v_1 + av_2) + \alpha_2(v_2 + av_3) + \alpha_3(v_3 + av_4) + \alpha_4(v_4 + av_1) &= 0 \\
        (\alpha_1 + a\alpha_4)v_1 + (\alpha_2 + a\alpha_1)v_2 + (\alpha_3 + a\alpha_2)v_3 + (\alpha_4 + a\alpha_3)v_4 &= 0
    \end{align*}
    which leads to the system
    \[
        \begin{pmatrix}
            1 & 0 & 0 & a \\
            a & 1 & 0 & 0 \\
            0 & a & 1 & 0 \\
            0 & 0 & a & 1
        \end{pmatrix}
        \begin{pmatrix}
            \alpha_1 \\ \alpha_2 \\ \alpha_3 \\ \alpha_4
        \end{pmatrix}
        =
        \begin{pmatrix}
            0 \\ 0 \\ 0 \\ 0
        \end{pmatrix}
    \]
    which is reduced to
    \[
        \begin{pmatrix}
            1 & 0 & 0 & a \\
            0 & 1 & 0 & -a^2 \\
            0 & 0 & 1 & a^3 \\
            0 & 0 & 0 & 1-a^2
        \end{pmatrix}
        \begin{pmatrix}
            \alpha_1 \\ \alpha_2 \\ \alpha_3 \\ \alpha_4
        \end{pmatrix}
        =
        \begin{pmatrix}
            0 \\ 0 \\ 0 \\ 0
        \end{pmatrix}
    \]
    which is \linearlyindependent \ifandonlyif \(1-a^2 \neq 0\) meaning \(a \in \complexnumbers \difference \{\pm 1, \pm i\}\).
\end{snippetsolution}

%\begin{snippetexercise}{linear-algebra-batch-5-ex-3}{}
%    \todo
%\end{snippetexercise}
%
%\begin{snippetsolution}{linear-algebra-batch-5-ex-3-sol}{}
%    \todo
%\end{snippetsolution}

\begin{snippetexercise}{linear-algebra-batch-5-ex-4}{}
    Consider the \set[sets]
    \[
        U = \{p \in \mathbb{P}_2(\mathbb{F}) \suchthat p'(0) = 0\},
        V = \{p \in \mathbb{P}_2(\mathbb{F}) \suchthat p'(1) = 0\}
    \]
    \begin{enumerate}
        \item Prove that \(U, V\) are linear subspaces of \(\mathbb{P}_2(\mathbb{F})\);
        \item Verify that \(U+V = \mathbb{P}_2(\mathbb{F})\);
        \item Compute \(\lineardim U \intersection V\) and find a \basis for it;
        \item Verify whether \(U+V\) is a direct sum;
    \end{enumerate}
\end{snippetexercise}

\begin{snippetsolution}{linear-algebra-batch-5-ex-4-sol}{}
    \begin{enumerate}
        \item let \(u,v \in U\) and \(\alpha, \beta \in \mathbb{F}\)
        \begin{enumerate}
            \item \(0 \in U\) as \(0'(0) =0\);
            \item \begin{align*}
                (\alpha u + \beta v)'(0) = \alpha u'(0) + \beta v'(0) = \alpha 0 + \beta 0 = 0
            \end{align*}
            meaning \(\alpha u + \beta v \in U\);
        \end{enumerate}
        Let \(u,v \in V\) and \(\alpha, \beta \in \mathbb{F}\)
        \begin{enumerate}
            \item \(0 \in U\) as \(0'(1) = 0\);
            \item \begin{align*}
                (\alpha u + \beta v)'(1) = \alpha u'(1) + \beta v'(1) = \alpha 0 + \beta 0 = 0
            \end{align*}
            meaning \(\alpha u + \beta v \in V\);
        \end{enumerate}
        \item we first explicit the elements of \(U\) and \(V\).
        The elements of \(U\) are of the form
        \[
            p(x) = a_0 + a_1 x + a_2 x^2
        \]
        such that \(p'(0) = a_1 = 0\) meaning \(U = \{a_0 + a_2 x^2\}\).
        The elements of \(V\) are of the form
        \[
            p(x) = b_0 + b_1 x + b_2 x^2
        \]
        such that \(p'(1) = b_1 + 2b_2 = 0\) meaning \(V = \{b_0 - 2a_2x + b_2 x^2\}\).
        We also have
        \[
            \lineardim U \intersection V = \lineardim \{a_0 \suchthat a_0 \in \mathbb{F}\} = 1
        \]
        Then,
        \[
            \lineardim U+V = \lineardim U + \lineardim V - \lineardim U \intersection V = 2 + 2 - 1 = 3
        \]
        and thus \(U+V = \mathbb{P}_2(\mathbb{F})\) which has also dimension \(3\).
        \item Since \(\lineardim U \intersection V = 1\) the sum is not direct.
    \end{enumerate}
\end{snippetsolution}

\begin{snippetexercise}{linear-algebra-batch-5-ex-5}{}
    Show that the the linear subspaces of \(\realnumbers^2\) are exactly \(\{0\}, \realnumbers^2\) and the lines through the origin.
\end{snippetexercise}

\begin{snippetsolution}{linear-algebra-batch-5-ex-5-sol}{}
    Let \(V\) be a linear subspace of \(\realnumbers^2\).
    We have \(\lineardim V = 0\) \ifandonlyif \(V = \{0\}\) and \(\lineardim V = 2\) \ifandonlyif \(V = \realnumbers^2\).
    The remaining case is \(\lineardim V = 1\), meaning that \(V = \linearspan\{v\}\) for some \(v \in \realnumbers^2\).
    We can write \(v = (x,y)\) and thus
    \[
        V = \{(tx, ty) \suchthat t \in \realnumbers\}
    \]
    which is a line through the origin.
    Conversely, let \(V\) be a line through the origin, meaning that \(V = \{(tx, ty) \suchthat t \in \realnumbers\}\) for some \(v = (x,y)\).
    We can write \(V = \linearspan\{v\}\) and thus \(\lineardim V = 1\).
\end{snippetsolution}

\begin{snippetexercise}{linear-algebra-batch-5-ex-6}{}
    Let \(U\) and \(W\) be linear subspaces of \(\realnumbers^9\) such that \(\lineardim U = \lineardim W = 5\).
    Show that \(U\intersection W \neq \{0\}\).
\end{snippetexercise}

\begin{snippetsolution}{linear-algebra-batch-5-ex-6-sol}{}
    We have
    \begin{align*}
        \lineardim U \intersection W &= \lineardim U + \lineardim W - \lineardim U + W \\
        &= 10 - \lineardim U + W \\
    \end{align*}
    but since \(\lineardim U + W \leq 9\) we have
    \[
        \lineardim U \intersection W \geq 1
    \]
    but \(U \intersection W = \{0\}\) \ifandonlyif \(\lineardim U \intersection W = 0\).
\end{snippetsolution}

\end{document}
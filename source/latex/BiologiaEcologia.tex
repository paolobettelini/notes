\documentclass[preview]{standalone}

\usepackage{amsmath}
\usepackage{amssymb}
\usepackage{stellar}
\usepackage{bettelini}

\hypersetup{
    colorlinks=true,
    linkcolor=black,
    urlcolor=blue,
    pdftitle={Biologia},
    pdfpagemode=FullScreen,
}

\begin{document}

\title{Biologia}
\id{biologia-ecologia}
\genpage

\section{Ecologia}

\begin{snippetdefinition}{ecologia-definition}{Ecologia}
    L'\textit{ecologia} è lo studio scientifico dei fattori che determinano la
    distribuzione e l'abbondanza degli organismi sulla Terra.
\end{snippetdefinition}

\begin{snippetdefinition}{ecosistema-definition}{Ecosistema}
    L'\textit{ecosistema} consiste nell'unità funzionale fondamentale in ecologia: è
    l'insieme degli organismi viventi e delle sostanze non viventi
    con le quali i primi stabiliscono uno scambio di materiali e di
    energia, in un'area delimitata, per es. un lago, un prato, un
    bosco ecc.
\end{snippetdefinition}

\plain{L'ecologo studia quindi componenti (abiotiche e biotiche (organismi e detriti)) e fattori
(biotici e abiotici).}

\begin{snippetexample}{studio-ecologia-example}{Studio di ecologia}
    Considerando i pesci e le alghe, l'ecologo può studiare la relazioni fra
    quest'ultime misurando l'ossigeno prodotto dalla fotosintesi - e relazionandone
    la produzione ed il consumo (fattori biotici).
\end{snippetexample}

\begin{snippetdefinition}{popolazione-definition}{Popolazione}
    Una \textit{popolazione} è un insieme di organismi della stessa
    specie che vivono in una determinato posto.
\end{snippetdefinition}

\begin{snippetdefinition}{comunita-definition}{Comunità}
    Una \textit{comunità} è un insieme di molteplici popolazioni.
\end{snippetdefinition}

\includesnpt{atmosfera-definition}
\includesnpt{idrosfera-definition}
\includesnpt{litosfera-definition}
\includesnpt{biosfera-definition}

\begin{snippet}{suddivisione-biosfera}
    La biosfera si suddivide in \textbf{biotopo} (o \textbf{habitat}) e \textbf{biocenosi} (o \textbf{comunità}).
\end{snippet}

\begin{snippet}{683f4bfc-a683-494a-8536-0a0596da67d7}
    L'ecologia studia spesso comunità, e per cui le relazioni fra popolazioni e fra popolazioni
    e ambienti. Un habitat ed una comunità inducono un ecosistema.
    Un ecosistema può essere suddiviso in \textbf{atmosfera}, \textbf{idrosfera}, \textbf{litosfera}
    e \textbf{biosfera}.
\end{snippet}

\end{document}
\documentclass[preview]{standalone}

\usepackage{amsmath}
\usepackage{amssymb}
\usepackage{stellar}
\usepackage{definitions}
\usepackage{wrapfig}
\usepackage{tikz}
\usepackage{enumitem}
\usepackage{circuitikz}
\usepackage{siunitx}

\usetikzlibrary{arrows}
\usetikzlibrary{decorations.pathreplacing}
\usetikzlibrary{cd}

\begin{document}

\id{electromagnetism-exercises}
\genpage

\section{Exercises}

\begin{snippetexercise}{electromagnetic-induction-coil-ex}{Electromagnetic Induction}
    A coil consisting of \(N=4\) turns has an area of \(A=200~\mathrm{cm}^{2}\) (\(0.02~\mathrm{m}^2\)). It is placed in a uniform magnetic field perpendicular to the surface. The field varies from \(25~\mathrm{mT}\) to \(10~\mathrm{mT}\) in a time interval \(\Delta t=5~\mathrm{s}\). Given that the resistance of the coil is \(R=5.0~\Omega\), calculate the induced current intensity.
\end{snippetexercise}

\begin{snippetsolution}{electromagnetic-induction-coil-ex-sol}{Electromagnetic Induction}
    We apply the Faraday-Lenz law combined with Ohm's law:
    \[
    I = \frac{\text{emf}}{R} = \frac{1}{R} \cdot N \cdot \frac{\Delta\Phi}{\Delta t}
    \]
    Considering the variation of the field \(\Delta B = 15 \cdot 10^{-3}~\mathrm{T}\):
    \[
    I = \frac{4 \cdot 0.02 \cdot 15 \cdot 10^{-3}}{5 \cdot 5} = 0.000048~\mathrm{A} = 48~\mathrm{mA}
    \]
\end{snippetsolution}

\begin{snippetexercise}{lorentz-force-direction-ex}{Direction of the Lorentz Force}
    A positive charge moves with velocity along the positive x-axis in the presence of a magnetic field \(B\) directed along the positive z-axis. Determine the direction of the force acting on the charge.
\end{snippetexercise}

\begin{snippetsolution}{lorentz-force-direction-ex-sol}{Direction of the Lorentz Force}
    Using the right-hand rule for the cross product \(\vec{F} = q\vec{v} \times \vec{B}\):
    \begin{itemize}
        \item Velocity vector \(\vec{v}\) towards \(+x\).
        \item Field vector \(\vec{B}\) towards \(+z\).
        \item The resulting force \(\vec{F}\) is directed along the negative \(y\) axis.
    \end{itemize}
\end{snippetsolution}

\begin{snippetexercise}{electrostatic-force-superposition-ex}{Superposition of Electrostatic Forces}
    Three charges are arranged on a line: \(Q=30\cdot10^{-6}~\mathrm{C}\), \(q=5\cdot10^{-6}~\mathrm{C}\), and a charge \(-2Q\) (from calculation). The charge \(q\) is placed centrally at a distance \(d=0.30~\mathrm{m}\) from the other two. Calculate the total force acting on \(q\).
\end{snippetexercise}

\begin{snippetsolution}{electrostatic-force-superposition-ex-sol}{Superposition of Electrostatic Forces}
    The total force is the vector sum of the forces exerted by \(Q\) and \(-2Q\):
    \[
    F_{tot} = \frac{1}{4\pi\epsilon_{0}}\cdot\frac{q\cdot Q}{d^{2}} + \frac{1}{4\pi\epsilon_{0}}\cdot\frac{q\cdot 2Q}{d^{2}}
    \]
    Inserting the values (with \(k=9\cdot10^{9}~\mathrm{Nm^{2}/C^{2}}\)):
    \[
    F_{tot} = 9\cdot10^{9} \cdot 5\cdot10^{-6} \cdot \left(\frac{30\cdot10^{-6}}{0.09} + \frac{60\cdot10^{-6}}{0.09}\right) = 7.5~\mathrm{N}
    \]
\end{snippetsolution}

\begin{snippetexercise}{inductor-energy-ex}{Energy of an Inductor}
    Calculate the energy stored in an inductor with inductance \(L=0.5\cdot10^{-3}~\mathrm{H}\) when a current \(i=4.0~\mathrm{A}\) flows through it.
\end{snippetexercise}

\begin{snippetsolution}{inductor-energy-ex-sol}{Energy of an Inductor}
    The energy \(U\) is given by the formula:
    \[
    U = \frac{1}{2}Li^{2} = 0.5 \cdot 0.5\cdot10^{-3} \cdot 4^{2} = 4\cdot10^{-3}~\mathrm{J} \quad (4~\mathrm{mJ})
    \]
\end{snippetsolution}

\begin{snippetexercise}{capacitors-series-ex}{Capacitors in Series}
    Two capacitors \(C_{1}=15~\mu\mathrm{F}\) and \(C_{2}=30~\mu\mathrm{F}\) are connected in series to a source of \(\Delta V=50~\mathrm{V}\). Calculate the charge on \(C_{2}\).
\end{snippetexercise}

\begin{snippetsolution}{capacitors-series-ex-sol}{Capacitors in Series}
    In series, the charge \(q\) is identical on both capacitors and equal to the charge of the equivalent capacitance:
    \[
    1/C_{eq} = 1/15 + 1/30 \Rightarrow C_{eq} = 10~\mu\mathrm{F}
    \]
    \[
    q = C_{eq} \cdot \Delta V = 10\cdot10^{-6} \cdot 50 = 5\cdot10^{-4}~\mathrm{C}
    \]
\end{snippetsolution}

\begin{snippetexercise}{electric-field-flux-gauss-ex}{Electric Field Flux (Gauss's Law)}
    A sphere has a surface charge density \(\sigma=4.0~\mathrm{nC/m^{2}}\) and radius \(r=0.02~\mathrm{m}\). Calculate the electric flux through a spherical Gaussian surface of radius \(0.04~\mathrm{m}\).
\end{snippetexercise}

\begin{snippetsolution}{electric-field-flux-gauss-ex-sol}{Electric Field Flux (Gauss's Law)}
    The flux depends only on the total enclosed charge:
    \[
    Q_{tot} = \sigma \cdot 4\pi r^{2} = 4\cdot10^{-9} \cdot 4\pi \cdot (0.02)^{2}
    \]
    \[
    \Phi = Q_{tot}/\epsilon_{0} \approx 2.27~\mathrm{Vm}
    \]
\end{snippetsolution}

\begin{snippetexercise}{em-waves-parameters-ex}{Electromagnetic Wave Parameters}
    Given a solar radiation intensity \(I=1340~\mathrm{W/m}^{2}\), determine the amplitude of the electric field \(E_{0}\) and the magnetic field \(B_{0}\).
\end{snippetexercise}

\begin{snippetsolution}{em-waves-parameters-ex-sol}{Electromagnetic Wave Parameters}
    \[
    E_{0} = \sqrt{2I/(c\epsilon_{0})} \approx 1000~\mathrm{V/m}
    \]
    \[
    B_{0} = E_{0}/c \approx 3.33\cdot10^{-6}~\mathrm{T}
    \]
\end{snippetsolution}

\begin{snippetexercise}{magnetic-field-infinite-wires-ex}{Magnetic Field of Infinite Wires}
    Two parallel wires separated by a distance \(d=5~\mathrm{mm}\) carry opposite currents \(i=60~\mathrm{A}\). Calculate the magnetic field \(B\) at an internal point located at \(r_{1}=2~\mathrm{mm}\) from the first wire.
\end{snippetexercise}

\begin{snippetsolution}{magnetic-field-infinite-wires-ex-sol}{Magnetic Field of Infinite Wires}
    Since the currents are opposite, the fields between the wires sum up. With \(r_{1}=0.002~\mathrm{m}\) and \(r_{2}=0.003~\mathrm{m}\):
    \[
    B_{tot} = \frac{\mu_{0} \cdot i}{2\pi} \cdot \left(\frac{1}{r_{1}} + \frac{1}{r_{2}}\right)
    \]
    \[
    B_{tot} = 2\cdot10^{-7} \cdot 60 \cdot (500 + 333.3) \approx 2\cdot10^{-2}~\mathrm{T} \quad (20~\mathrm{mT})
    \]
\end{snippetsolution}

\begin{snippetexercise}{electrostatic-potential-difference-ex}{Electrostatic Potential Difference}
    Calculate the potential difference between two points A and B generated by a system of two point charges \(q\) and \(Q\).
\end{snippetexercise}

\begin{snippetsolution}{electrostatic-potential-difference-ex-sol}{Electrostatic Potential Difference}
    Calculate the potential at each point as the sum of the potentials of the individual charges (\(V=k\sum q_{i}/r_{i}\)). Following the numerical steps from the sheet:
    \[
    V_{A} - V_{B} = +60~\mathrm{V}
    \]
\end{snippetsolution}

\begin{snippetexercise}{resistance-series-circuit-ex}{Resistance in a Series Circuit}
    A circuit is powered by an \(\text{emf}=20~\mathrm{V}\). When two resistors \(R_{1}\) and \(R_{2}\) are in series, a current \(i=1.0~\mathrm{A}\) flows. Knowing that \(R_{2}=15~\Omega\), determine \(R_{1}\).
\end{snippetexercise}

\begin{snippetsolution}{resistance-series-circuit-ex-sol}{Resistance in a Series Circuit}
    From Ohm's law for the series circuit:
    \[
    R_{tot} = R_{1} + R_{2} = \frac{V}{i}
    \]
    \[
    R_{1} + 15 = \frac{20}{1.0} = 20~\Omega
    \]
    \[
    R_{1} = 20 - 15 = 5~\Omega
    \]
\end{snippetsolution}

\begin{snippetexercise}{capacitor-circuit-ex}{Capacitor Circuit}
    Given a circuit powered by a voltage \(V_{0}=18~\mathrm{V}\) composed of three capacitors \(C_{1}=20\mu \mathrm{F}\), \(C_{2}=10\mu \mathrm{F}\), and \(C_{3}=30\mu \mathrm{F}\), determine the equivalent capacitance and the charge \(q_{1}\) on the first capacitor.
\end{snippetexercise}

\begin{snippetsolution}{capacitor-circuit-ex-sol}{Capacitor Circuit}
    \textbf{Capacitance in parallel:} \(C_{2}\) and \(C_{3}\) are in parallel:
    \[
    C_{23} = C_{2} + C_{3} = 10\mu \mathrm{F} + 30\mu \mathrm{F} = 40\mu \mathrm{F}
    \]
    \textbf{Equivalent capacitance:} \(C_{1}\) is in series with \(C_{23}\):
    \[
    \frac{1}{C_{eq}} = \frac{1}{C_{1}} + \frac{1}{C_{23}} = \frac{1}{20} + \frac{1}{40} = \frac{3}{40}\mu \mathrm{F}^{-1}
    \]
    \[
    \Rightarrow C_{eq} = \frac{40}{3}\mu \mathrm{F} \approx 13.33~\mu \mathrm{F}
    \]
    
    \textbf{Charge \(q_{1}\):} In a series circuit, the charge on each component is equal to the total charge supplied by the source:
    \[
    q_{1} = q_{tot} = C_{eq} \cdot V_{0} = \frac{40}{3} \cdot 18 = 240~\mu \mathrm{C}
    \]
\end{snippetsolution}

\begin{snippetexercise}{charged-particle-acceleration-ex}{Charged Particle Acceleration}
    A proton (\(m_{p}=1.67\cdot10^{-27}\) kg, \(q=1.6\cdot10^{-19}\mathrm{C}\)) starts from rest and is accelerated by a potential difference \(\Delta V=4\cdot10^{3}\) V. Calculate the final velocity \(v\).
\end{snippetexercise}

\begin{snippetsolution}{charged-particle-acceleration-ex-sol}{Charged Particle Acceleration}
    Applying the principle of conservation of energy (\(K=W\)):
    \[
    \frac{1}{2}m_{p}v^{2} = q\Delta V \Rightarrow v = \sqrt{\frac{2q\Delta V}{m_{p}}}
    \]
    \[
    v = \sqrt{\frac{2\cdot1.6\cdot10^{-19}\cdot4000}{1.67\cdot10^{-27}}} \approx 8.76\cdot10^{5}~\mathrm{m/s}
    \]
\end{snippetsolution}

\begin{snippetexercise}{em-waves-amplitude-ex}{Electromagnetic Waves}
    In a plane electromagnetic wave in a vacuum, the maximum electric field is \(E_{max}=600~\mathrm{V/m}\). Calculate the amplitude of the magnetic field \(B_{max}\).
\end{snippetexercise}

\begin{snippetsolution}{em-waves-amplitude-ex-sol}{Electromagnetic Waves}
    From the fundamental relationship between the field magnitudes in a plane wave (\(E=cB\)):
    \[
    B_{max} = \frac{E_{max}}{c} = \frac{600}{3\cdot10^{8}} = 2\cdot10^{-6}~\mathrm{T} \quad (\text{i.e. } 2~\mu \mathrm{T})
    \]
\end{snippetsolution}

\begin{snippetexercise}{point-charges-force-ex}{Force Between Point Charges}
    Three charges are aligned on the x-axis: \(q_{1}=40\mu \mathrm{C}\) at \(x_{1}=-20\) cm, \(q_{2}=50\mu \mathrm{C}\) at \(x_{2}=30\) cm, and \(q_{3}=4\mu \mathrm{C}\) at the origin (\(x=0\)). Calculate the resultant force on \(q_{3}\).
\end{snippetexercise}

\begin{snippetsolution}{point-charges-force-ex-sol}{Force Between Point Charges}
    Assuming all charges are positive, \(q_{3}\) experiences a push to the right from \(q_{1}\) (\(F_{13}\)) and a push to the left from \(q_{2}\) (\(F_{23}\)):
    \[
    F_{tot} = |F_{13}| - |F_{23}| = \frac{q_{3}}{4\pi\epsilon_{0}}\left(\frac{q_{1}}{r_{1}^{2}} - \frac{q_{2}}{r_{2}^{2}}\right)
    \]
    \[
    F_{tot} = (9\cdot10^{9})\cdot(4\cdot10^{-6})\cdot\left(\frac{40\cdot10^{-6}}{0.2^{2}} - \frac{50\cdot10^{-6}}{0.3^{2}}\right) \approx 16~\mathrm{N} \quad (\text{direction } +x)
    \]
\end{snippetsolution}

\begin{snippetexercise}{faraday-lenz-law-loop-ex}{Faraday-Lenz Law}
    A circular loop is immersed in a uniform magnetic field \(B=1.5~\mathrm{T}\) perpendicular to the plane of the loop. The radius of the loop increases linearly with time according to the law \(r(t) = r_{0}+vt\) with \(r_{0}=0.12~\mathrm{m}\) and \(v=0.03~\mathrm{m/s}\). Calculate the instantaneous induced emf.
\end{snippetexercise}

\begin{snippetsolution}{faraday-lenz-law-loop-ex-sol}{Faraday-Lenz Law}
    The magnetic flux is \(\Phi(B) = B \cdot A = B \cdot \pi r^{2}\).
    \[
    E = -\frac{d\Phi}{dt} = -B\pi\frac{d}{dt}(r^{2}) = -B\pi\left(2r\cdot\frac{dr}{dt}\right)
    \]
    Substituting the values at the initial instant:
    \[
    E = -1.5 \cdot \pi \cdot 2 \cdot 0.12 \cdot 0.03 \approx -33.9~\mathrm{mV}
    \]
\end{snippetsolution}

\begin{snippetexercise}{joule-effect-ex}{Joule Effect}
    Calculate the thermal energy dissipated in a time \(\Delta t=120\) s by a resistor \(R=150~\Omega\) connected to a voltage \(\Delta V=20~\mathrm{V}\).
\end{snippetexercise}

\begin{snippetsolution}{joule-effect-ex-sol}{Joule Effect}
    \[
    E = P \cdot \Delta t = \frac{\Delta V^{2}}{R} \cdot \Delta t = \frac{20^{2}}{150} \cdot 120 = 320~\mathrm{J}
    \]
\end{snippetsolution}

\begin{snippetexercise}{rlc-circuit-sinusoidal-ex}{RLC Circuit in Sinusoidal Regime}
    A series RLC circuit has \(R=100~\Omega\), \(L=1~\mathrm{H}\), \(C=2\mu \mathrm{F}\). The source supplies \(V(t) = 100 \sin(500t)\). Calculate the effective current \(I_{eff}\).
\end{snippetexercise}

\begin{snippetsolution}{rlc-circuit-sinusoidal-ex-sol}{RLC Circuit in Sinusoidal Regime}
    \begin{enumerate}
        \item \textbf{Angular frequency:} \(\omega = 500~\mathrm{rad/s}\)
        \item \textbf{Reactances:}
        \[
        X_{L} = \omega L = 500 \cdot 1 = 500~\Omega
        \]
        \[
        X_{C} = \frac{1}{\omega C} = \frac{1}{500 \cdot 2 \cdot 10^{-6}} = 1000~\Omega
        \]
        \item \textbf{Impedance:}
        \[
        Z = \sqrt{R^{2} + (X_{L} - X_{C})^{2}} = \sqrt{100^{2} + (500 - 1000)^{2}} \approx 509.9~\Omega
        \]
        \item \textbf{Effective current:}
        \[
        V_{eff} = \frac{V_{max}}{\sqrt{2}} = \frac{100}{\sqrt{2}} \approx 70.71~\mathrm{V}
        \]
        \[
        I_{eff} = \frac{V_{eff}}{Z} = \frac{70.71}{509.9} \approx 0.139~\mathrm{A}
        \]
    \end{enumerate}
\end{snippetsolution}

\begin{snippetexercise}{wire-resistance-calculation-ex}{}
    Calculate the resistance in \(\Omega\) of a wire with resistivity \(3.2 \cdot 10^{-8}\)
    \(\Omega \text{m}\), with length \(2.5\) meters and diameter \(0.5 \text{mm}\).
\end{snippetexercise}

\begin{snippetsolution}{wire-resistance-calculation-ex-sol}{}
    We use the second law of Ohm: \(R = \rho \frac{L}{A}\). \\
    Convert diameter to radius and to meters: \(r = \frac{d}{2} = 0.25 \, \text{mm} = 2.5 \cdot 10^{-4} \, \text{m}\). \\
    The cross-sectional area is \(A = \pi r^2 = \pi (2.5 \cdot 10^{-4})^2 \approx 1.96 \cdot 10^{-7} \, \text{m}^2\). \\
    Calculation:
    \[
    R = (3.2 \cdot 10^{-8}) \frac{2.5}{1.96 \cdot 10^{-7}} \approx \frac{8 \cdot 10^{-8}}{1.96 \cdot 10^{-7}} \approx 0.408 \, \Omega
    \]
\end{snippetsolution}

\begin{snippetexercise}{em-wave-B-max-ex}{}
    If the maximum value of the \(E\) component of an electromagnetic wave is \(600 \text{V/m}\), what is the maximum of the \(B\) component?
\end{snippetexercise}

\begin{snippetsolution}{em-wave-B-max-ex-sol}{}
    In an electromagnetic wave in a vacuum, the relationship between the field amplitudes is \(E = cB\), where \(c \approx 3 \cdot 10^8 \, \text{m/s}\).
    \[
    B = \frac{E}{c} = \frac{600}{3 \cdot 10^8} = 200 \cdot 10^{-8} = 2.0 \cdot 10^{-6} \, \text{T} = 2.0 \, \mu\text{T}
    \]
\end{snippetsolution}

\begin{snippetexercise}{square-charges-electric-field-ex}{}
    Two charges are placed on a square of side \(1.5\) meters, one of \(2 \text{nC}\)
    and one of \(3 \text{nC}\) on two opposite vertices. What is the intensity of the electric
    field on one of the other two vertices?
\end{snippetexercise}

\begin{snippetsolution}{square-charges-electric-field-ex-sol}{}
    Let the charges be \(q_1 = 2\,\text{nC}\) and \(q_2 = 3\,\text{nC}\). The vertex considered is adjacent to both, so the distance from each charge is \(r = 1.5\,\text{m}\). The generated electric fields are perpendicular to each other.
    \[ E_1 = k \frac{q_1}{r^2} = (8.99 \cdot 10^9) \frac{2 \cdot 10^{-9}}{1.5^2} \approx 8.0 \, \text{N/C} \]
    \[ E_2 = k \frac{q_2}{r^2} = (8.99 \cdot 10^9) \frac{3 \cdot 10^{-9}}{1.5^2} \approx 12.0 \, \text{N/C} \]
    The total field is the vector sum (Pythagoras):
    \[ E_{tot} = \sqrt{E_1^2 + E_2^2} = \sqrt{8^2 + 12^2} = \sqrt{64 + 144} \approx 14.4 \, \text{N/C} \]
\end{snippetsolution}

\begin{snippetexercise}{charged-sphere-electric-field-ex}{}
    A sphere of volume \(12 \text{cm}^3\) is filled with a non-conducting material
    with a uniformly distributed charge of \(30 \text{pC}\) throughout the volume.
    What is the electric field intensity at \(1.0 \text{cm}\) from the center?
\end{snippetexercise}

\begin{snippetsolution}{charged-sphere-electric-field-ex-sol}{}
    Let's find the radius of the sphere \(R\). \(V = \frac{4}{3}\pi R^3 \Rightarrow R = \sqrt[3]{\frac{3V}{4\pi}}\).
    With \(V=12\), \(R \approx 1.42\,\text{cm}\). Since the requested distance \(r=1.0\,\text{cm}\) is less than \(R\), we are inside the distribution.
    The internal field of a uniformly charged insulating sphere is:
    \[ E = \frac{Q_{tot} r}{4\pi \epsilon_0 R^3} = \frac{\rho r}{3\epsilon_0} \]
    More simply, using the proportion of enclosed charge \(Q_{enc} = Q_{tot} \frac{r^3}{R^3}\):
    \[ E = k \frac{Q_{enc}}{r^2} = k \frac{Q_{tot} r}{R^3} \]
    \[ E = (8.99 \cdot 10^9) \frac{30 \cdot 10^{-12} \cdot 0.01}{(0.0142)^3} \approx 941 \, \text{N/C} \]
\end{snippetsolution}

\begin{snippetexercise}{rectangular-loop-fem-ex}{}
    Consider a rectangular loop of \(0.2 \text{m}\) with a uniform magnetic field perpendicular
    to the plane of the loop. The magnetic field intensity is \(B = 0.4 T \cdot e^{t/J}\) seconds and
    \(J = 4.0\). Calculate induced emf at \(t = 2.0\) seconds.
\end{snippetexercise}

\begin{snippetsolution}{rectangular-loop-fem-ex-sol}{}
    Assume the loop is square with side \(l=0.2\,\text{m}\), so Area \(A = 0.04\,\text{m}^2\).
    Faraday-Neumann-Lenz Law: \(\mathcal{E} = - \frac{d\Phi_B}{dt} = -A \frac{dB}{dt}\).
    Given \(B(t) = 0.4 e^{t/4}\), the derivative is \(\frac{dB}{dt} = 0.4 \cdot \frac{1}{4} e^{t/4} = 0.1 e^{t/4}\).
    At \(t=2.0\): \(\frac{dB}{dt} = 0.1 e^{0.5} \approx 0.165 \, \text{T/s}\).
    \[ |\mathcal{E}| = 0.04 \cdot 0.165 \approx 0.0066 \, \text{V} = 6.6 \, \text{mV} \]
\end{snippetsolution}

\begin{snippetexercise}{wire-force-parallel-field-ex}{}
    Consider a \(2\) meter cable suspended parallel to a uniform magnetic field of \(0.5 T\),
    carried by a current of \(0.6 A\). Find the force in Newtons applied to the cable.
\end{snippetexercise}

\begin{snippetsolution}{wire-force-parallel-field-ex-sol}{}
    The force on a current-carrying wire is \(F = I L B \sin(\theta)\).
    The text specifies that the cable is \textbf{parallel} to the magnetic field, so \(\theta = 0^\circ\) (or \(180^\circ\)).
    Since \(\sin(0) = 0\), the magnetic force is:
    \[ F = 0 \, \text{N} \]
\end{snippetsolution}

\begin{snippetexercise}{resistor-potential-difference-electrons-ex}{}
    If \(5 \times 10^{21}\) electrons enter a \(20 \Omega\) resistor in \(10 \text{min}\), what is the potential
    difference in \(V\) across the resistor?
\end{snippetexercise}

\begin{snippetsolution}{resistor-potential-difference-electrons-ex-sol}{}
    Let's calculate the total charge first and then the current.
    \(Q = N \cdot e = 5 \cdot 10^{21} \cdot 1.6 \cdot 10^{-19} = 800 \, \text{C}\).
    The time in seconds is \(t = 10 \cdot 60 = 600 \, \text{s}\).
    Current \(I = \frac{Q}{t} = \frac{800}{600} = \frac{4}{3} \, \text{A} \approx 1.33 \, \text{A}\).
    Ohm's Law:
    \[ V = R \cdot I = 20 \cdot \frac{4}{3} = \frac{80}{3} \approx 26.7 \, \text{V} \]
\end{snippetsolution}

\begin{snippetexercise}{capacitor-energy-circuit-ex}{}
    If \(V_a - V_b = 22 \text V\), what is the energy stored in the \(50 \mu F\) capacitor?
    \begin{center}
        \begin{circuitikz}
            \draw (0,2) node[left] {a} 
                to[short, o-] (0.5, 2) % small piece of initial wire
                to[C, l=\(25\,\mu\text{F}\)] (3,2) % Horizontal capacitor
                -- (3,2); % Arrival at the corner
            \draw (3,2) 
                to[C, l=\(50\,\mu\text{F}\)] (3,0); 
            \draw (3,0) 
                to[C, l=\(25\,\mu\text{F}\)] (0.5, 0)
                to[short, -o] (0,0) node[left] {b}; % close on B
        \end{circuitikz}
    \end{center}
\end{snippetexercise}

\begin{snippetsolution}{capacitor-energy-circuit-ex-sol}{}
    The three capacitors (\(25\mu\text{F}, 50\mu\text{F}, 25\mu\text{F}\)) are in series.
    Let's calculate the equivalent capacitance:
    \[ \frac{1}{C_{eq}} = \frac{1}{25} + \frac{1}{50} + \frac{1}{25} = \frac{2+1+2}{50} = \frac{5}{50} = \frac{1}{10} \]
    So \(C_{eq} = 10 \, \mu\text{F}\).
    The total charge supplied by the generator is \(Q_{tot} = C_{eq} V = 10\mu\text{F} \cdot 22\text{V} = 220 \, \mu\text{C}\).
    In series, the charge is identical on each capacitor: \(Q_{50} = 220 \, \mu\text{C}\).
    The energy stored in the \(50\mu\text{F}\) capacitor is:
    \[ U = \frac{Q^2}{2C} = \frac{(220 \cdot 10^{-6})^2}{2 \cdot 50 \cdot 10^{-6}} = \frac{48400 \cdot 10^{-12}}{100 \cdot 10^{-6}} = 484 \cdot 10^{-6} \, \text{J} = 484 \, \mu\text{J} \]
\end{snippetsolution}

\end{document}
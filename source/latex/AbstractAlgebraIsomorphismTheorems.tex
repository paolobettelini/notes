\documentclass[preview]{standalone}

\usepackage{amsmath}
\usepackage{amssymb}
\usepackage{stellar}
\usepackage{definitions}

\begin{document}

\id{isomorphism-theorems}
\genpage

\section{Isomorphism theorems}

\plain{The same theorems apply to rings, vector spaces and such.}

\begin{snippettheorem}{first-isomorphism-theorem}{First isomorphism theorem}
    Let \(\varphi\colon G \fromto H\) be a \grouphomomorphism.
    Then, \[G / \grpker_\varphi \groupisomorphic \image \varphi\]
\end{snippettheorem}

\begin{snippettheorem}{second-isomorphism-theorem}{Second isomorphism theorem}
    Let \(G\) be a \group and let
    \(H \subgroupleq G, K \normalsubgrp G\).
    Then,
    \[
        H \intersection K \normalsubgrp H
        \land \frac{H}{H \intersection K} \groupisomorphic \frac{HK}{K}
    \]
\end{snippettheorem}

\begin{snippettheorem}{third-isomorphism-theorem}{Third isomorphism theorem}
    Let \(G\) be a \group and \(N \normalsubgrp G\).
    The \subgroup[subgroups] of the quotient are all and only the \subgroup[subgroups] of form
    \(H/N\) with \(H \subgroupleq G\) and \(G\) containing \(N\).
    Furthermore, \(H/N \normalsubgrp G / N\) \ifandonlyif \(H \normalsubgrp G\).
    In such case,
    \[
        \frac{G/N}{H/N} \groupisomorphic \frac{G}{H}
    \]
\end{snippettheorem}

\begin{snippet}{isomorphism-theorems-exp11}
    Ricordiamo che se \(\varphi \colon G \fromto H\) è un omomorfismo:
    \begin{enumerate}
        \item if \(K \subgroupleq G\) then \(K\varphi\varphi^\inversefunction \geq K\);
        \item if \(L \subgroupleq H\) then \(L \varphi^\inversefunction \varphi \subgroupleq L\).
    \end{enumerate}
    L'uguaglianza vale:
    \begin{enumerate}
        \item \(K\varphi\varphi^\inversefunction = K\) \ifandonlyif \(K \geq \grpker_\varphi\);
        \item \(L \varphi^\inversefunction \varphi = L\) \ifandonlyif \(L \subgroupleq \image \varphi\).
    \end{enumerate}

    Dunque, abbiamo una biiezione tra i sottogruppi di \(G\)
    che contengono il nucleo e i sottogruppi di \(H\)
    contenuti nell'immagine.
\end{snippet}

\subsection{Proofs}

\begin{snippetproof}{first-isomorphism-theorem-proof}{first-isomorphism-theorem}{First isomorphism theorem}
    Sappiamo che se \(x\) e \(y\) sono elementi
    di \(G\), si ha che \(x\varphi = y\varphi\) \ifandonlyif
    \(x\grpker_\varphi = y\grpker_\varphi\). Quindi tutti gli elementi dello stesso laterale
    (sinistro in questo caso) hanno la stessa immagine.
    Possiamo allora definire \(\psi\colon G / \grpker_\varphi \fromto \image \varphi\)
    ponendo \[(x\grpker_\varphi) \psi \triangleq x\varphi\]
    Questa funzione è ben definita, siccome
    \[
        x \grpker_\varphi = y\grpker_\varphi \implies x\varphi = y\varphi
    \]
    L'implicazione nella direzione opposta, ci dice che \(\psi\)
    è iniettiva. La funzione è anche banalmente suriettiva,
    se \(z\in \image \varphi\), allora \(z=x\varphi\)
    per qualche \(x\in G\), e quindi
    \(z = (x \grpker_\varphi) \psi\).
    Dimostriamo ora che l'omomorfismo rispetta l'operazione:
    \begin{align*}
        (x\grpker_\varphi)\psi (y \grpker_\varphi) \psi
        &= x\varphi \cdot y \varphi = (xy) \varphi
    \end{align*}
    e
    \begin{align*}
        (x\grpker_\varphi \cdot y\grpker_\varphi) \psi &=
        (xy \grpker_\varphi) \psi = (xy) \psi
    \end{align*}
\end{snippetproof}

\begin{snippetproof}{second-isomorphism-theorem-proof}{second-isomorphism-theorem}{Second isomorphism theorem}
    Poiché \(K \normalsubgrp G\), abbiamo che
    \(HK \leq G\).
    Consideriamo questi due omomorfismi: l'inclusione
    di \(H\) in \(HK\), % manda l'elemento in sè ma non è l'identità in quanto HK è più grande
    e l'omomorfismo canonico da \(HK \in HK/K\).
    Notiamo che \(K \normalsubgrp G\) e a maggior ragione \(K \normalsubgrp HK\).
    Componiamo questi due omomorfismi 
    \[
        \varphi \colon H \fromto \frac{HK}{K}
    \]
    Quindi \(\varphi(h) = hK\).
    \begin{itemize}
        \item Applichiamo il teorema precedente, \(\image\varphi \cong H/\grpker_\varphi\).
        Ora \(\image\varphi = HK/K\) cioè \(\varphi\) è suriettiva.
        Infatti l'elemento generico di \(HK/K\) è del tipo
        \(hkK\) con \(h\in H\) e \(k\in K\).
        Ma \(hk \in hK\). Ma se un elemento sta in certo laterale, il laterale che lo contiene
        è lo stesso, cioè \(hkK = hK\) e quindi
        \(hkK = hK = h\varphi\).
        \item Studiamo ora il nucleo: esso contiene le identità del quoziente,
        ma esso è \(\{h\in H \,|\, hK=K \}\), il che succede
        per \(\{h \in H \,|\, h\in K\}\), quindi \(h \in H \cap K\).
        In particolare, \(H \cap K \normalsubgrp H\). Dunque,
        \[
            \frac{HK}{K} = \image\varphi
            \cong H/\grpker_\varphi = \frac{H}{H \cap K}
        \]
    \end{itemize}
\end{snippetproof}

\begin{snippetproof}{third-isomorphism-theorem-proof}{third-isomorphism-theorem}{Third isomorphism theorem}
    Consideriamo l'omomorfismo canonico
    \[
        \varphi\colon G \fromto G/N
    \]
    Sappiamo che \(\grpker_\varphi = N\) e che \(\varphi\)
    è suriettivo. Per quanto detto prima,
    abbiamo una biiezione tra i sottogruppi
    di \(G\) contenenti \(\grpker_\varphi = N\), e i sottogruppi
    di \(G/N\) contenuti nell'immagine \(\image\varphi\).
    Dunque, un sottogruppo di \(G/N\) è immagine di un sottogruppo
    \(H\) di \(G\) contenente \(N\), ed è quindi del tipo \(H/N\).
    Sappiamo poi che se \(H \normalsubgrp G\), allora
    \(H\varphi \normalsubgrp \image\varphi\),
    cioè \(H/N \normalsubgrp G/N\) e, viceversa, se \(H/N \normalsubgrp G/N\),
    allora \({(H/N)}\varphi^\inversefunction \normalsubgrp G\),
    cioè \(H \normalsubgrp G\). \\
    Sia allora \(N \leq H \normalsubgrp G\). Mostriamo
    \[
        \frac{G/N}{H/N} \cong \frac{G}{H}
    \]
    Definiamo un omomorfismo
    \[
        \theta\colon G/N \fromto G/H
    \]
    ponendo \((Nx)\theta \triangleq Hx\).
    La funzione è ben definita, infatti se \(Nx = Ny\), cioè \(x \in Ny\),
    allora \(x \in Ny \subseteq Hy\), cioè \(Hx = Hy\). Si verifica facilmente
    che \(\theta\) è un omomorfismo (suriettivo)
    di nucleo \(H/N\). Dunque, \(\image\theta \cong \frac{G/N}{\grpker_\theta}\),
    cioè \(G/H \cong (G/N) / (H/N)\).
\end{snippetproof}

\end{document}
\documentclass[preview]{standalone}

\usepackage{amsmath}
\usepackage{amssymb}
\usepackage{stellar}
\usepackage{definitions}

\begin{document}

\id{euclidean-domains}
\genpage

\section{Euclidean domains}

\begin{snippetdefinition}{euclidean-domain-definition}{Euclidean domain}
    Let \(A\) be a domain of integrity
    and \(\delta \colon A \difference \{0_A\} \fromto \naturalnumbers\)
    such that \(\forall a,b \in A\), \(\exists q, r \in A\)
    with \(\delta(r) < \delta(b)\) or \(r = 0_A\) and \(a = b\cdot q + r\).
    Then, \(A\) is said to be a \emph{domain of integrity}.
\end{snippetdefinition}

\begin{snippetproposition}{euclidea-domain-is-principal-ideal-domain}{}
    An euclidean domain is a principal ideal domain.
\end{snippetproposition}

\plain{This is the same proof as the one over the integers or over polynomials over a field.}

\begin{snippetproof}{euclidea-domain-is-principal-ideal-domain-proof}{euclidea-domain-is-principal-ideal-domain}{}
    Let \(A\) be an euclidean domain.
    Take \(I \idealin A\) such that \(I \neq A, \{0_A\}\)
    and take \(x\in I\) such that \(\delta(x) \min \{ \delta(y) \suchthat y \in I\}\).
    We show that \(I = (x)\).
    If we take \(y \in I\), then \(\exists q, r \in A\) with \(\delta(r) < \delta(x)\)
    or \(r = 0_A\) such that \(y = qx + r\).
    Thus, \(r = y - qx \in I\) which means that \(\delta(r)\)
    is not less than \(\delta(x)\) and thus \(r = 0_A\).
    This implies that \(y = qx\) and \(y \in (x)\).
\end{snippetproof}

\plain{In an euclidean domain we can define the greatest common divisor.}

\end{document}
\documentclass[preview]{standalone}

\usepackage{amsmath}
\usepackage{amssymb}
\usepackage{stellar}
\usepackage{definitions}

\begin{document}

\id{taylor-series-well-known}
\genpage

\section{Well-known Taylor series}

\begin{snippetproposition}{exponentia-function-maclaurin-series}{Exponential function MacLaurin series}
    \[
        \eulernumber^x = \sum_{k=0}^\infty \frac{x^k}{k!}
    \]
\end{snippetproposition}

\begin{snippetproposition}{sine-maclaurin-series}{Sine MacLaurin series}
    \[
        \sin(x) = \sum_{k=0}^\infty X
    \]
\end{snippetproposition}

\begin{snippetproposition}{cosine-maclaurin-series}{Cosine MacLaurin series}
    \[
        \cos(x) = \sum_{k=0}^\infty X
    \]
\end{snippetproposition}

\begin{snippetproposition}{tangent-maclaurin-series}{Tangent MacLaurin series}
    \[
        \tan(x) = \sum_{k=0}^\infty X
    \]
\end{snippetproposition}

\begin{snippetproposition}{hyperbolic-sine-maclaurin-series}{Hyperbolic sine MacLaurin series}
    \[
        \sinh(x) = \sum_{k=0}^\infty X
    \]
\end{snippetproposition}

\begin{snippetproposition}{hyperbolic-cosine-maclaurin-series}{Hyperbolic cosine MacLaurin series}
    \[
        \cosh(x) = \sum_{k=0}^\infty X
    \]
\end{snippetproposition}

% log(1+x) log(1-x) (1+x)^a

\end{document}
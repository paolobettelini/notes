\documentclass[preview]{standalone}

\usepackage{amsmath}
\usepackage{amssymb}
\usepackage{stellar}
\usepackage{bettelini}

\hypersetup{
    colorlinks=true,
    linkcolor=black,
    urlcolor=blue,
    pdftitle={Stellar},
    pdfpagemode=FullScreen,
}

\begin{document}

\title{Stellar}
\id{italiano-decameron-analisi}
\genpage

\section{Analisi del Decameron}

\plain{TODO: link testo}

\begin{snippetdefinition}{locus-amoenus-definition}{Locus amoenus}
    Un \textit{locus amoenus} è un termine usato in letteratura
    che fa riferimento a un luogo idealizzato e piacevole.
    È un posto immerso tra piante ed alberi,
    spesso situato nelle vicinanze di una fonte o di un ruscello,
    ricco di ombra ed in qualche modo simile al Paradiso terrestre. 
\end{snippetdefinition}

\begin{snippet}{analisi-decameron}
    I primi 7 paragrafi [1-7] sono una scusa rivolte alle donne. Viene delineata la necessità di 
    avere una salita di sofferenza prima del piacere.
    Boccaccio spiega che l'introduzione (sofferente) è necessaria per capire come si arriva alle novelle. \\
    La storia vera e propria comincia all'ottavo paragrafo [8] mediante l'arrivo della peste.
    Questa peste, deriva geograficamente dall'Oriente prima di giungere a Firenze
    (è stata trasportata dai topi sulle barche mercantili).
    Secondo Boccaccio, è possibile che la pesta sia una punizione divina inflitta sugli uomini.
    Viene descritto come la peste influenzasse il corpo, come lo facesse gonfiare e come i sintomi evolvessero per poi giungere alla morte.
    Successivamente, [13] si parla dello scopo dei medici; nessuna medicina riusciva a contrastare questa piaga, la quale è contagiosa [13]
    anche fra uomini e animali [17].
    Dopo aver descritto in dettaglio le dinamiche della malattia, vengono descritte
    le reazioni delle persone [19]: 1. Forse il modo migliore è quello di vivere con parsimonia ma con cibi e vini di qualità
    2. Fare festa e concedersi gli eccessi.
    Le leggi divine e leggi umane sono finite, in quanto tutti sono affetti dal caos [23].
    \\\\
    Nei paragrafi [53-65] vi è l'inizio del lungo discorso di Pampinea, la quale arriva a proporre alle altre 6 conoscenti
    di uscire da Firenze. Essa insiste molto che ciò non è una scelta di disimpegno,
    al massimo sono loro ad essere il frutto dell'abbandono.
    La proposta è di uscita, che tuttavia non è fuga nel divertimento, bensì di vita onesta nella campagna (contado),
    onestà che non c'è più nella città. Lo scopo è quello di fare festa, provare piacere ed essere allegri,
    ma \quotes{senza trapassare in alcun atto il segno della ragione}.
    \\\\
    Nei seguenti paragrafi [-67] viene descritto un locus amoenus.
    Questa descrizione è data come opposto al luogo di Firenze.
    Questa non è una proposta di abbandono o fuga definitiva della città [71],
    ma vedranno che cosa gli risparmia il cielo se rimarranno vivi.
    Ma Filomena [74], la quale era molto discreta, ricorda che un gruppo di donne non può gestirsi
    senza un uomo che le guidi, quanto le donne sono litigiose, sospettose, pusillanimi e paurose.
    Elissa le da ragione, e fa notare che il progetto dell'onestà cadrebbe dall'inizio se
    prendessero degli uomini non legati a loro (vivere sotto il medesimo tetto sarebbe uno scandalo).
    Per pura coincidenza [78] 3 uomini entrano nella chiesa. Gli uomini sono chiamati Panfilo, Filostrato e Dioneo
    e sono belli, di buone maniere e nobili, come lo sono le donne (per quanto ne sappiamo quest'ultimi potrebbero essere solo dei soprannomi).
    Questi 3 non sono completi estranei perché sono innamorati di alcune delle ragazze.
    Neifile [81], l'amata di una dei ragazzi, parla bene dei ragazzi, ma teme
    promiscuità siccome i ragazzi sono innamorati di loro e non sono i loro mariti.
    La risposta a questa preoccupazione viene da Filomena, la quale dice che ciò non importa,
    essendo il piano onesto. \\
    Pampinea approccia i ragazzi con un sorriso e cominciano ad organizzare il trasferimento [88].
    Usciti dalla città si misero in via. Il luogo scelto è solo a due miglia di distanza dalla città [89].
    Nei successivi paragrafi viene descritto il luogo: un palazzo con sale e camere affrescate, cortile, logge,
    giardini meravigliosi con pozzi d'acqua freschissimi e cantine di vini raffinati.
    Le prime parole sono di Dioneo, il quale dice si aver lasciato i propri pensieri luttuosi e tristi nella città.
    Il suo programma consiste in una dicotomia: o le donne cantano e si divertono con lui
    (sempre nei limiti dell'onestà), o lui se ne torna in città.
    Pampinea propone di avere un campo per tenere la comunità entro i propri limiti e in armonia,
    che si preoccupi di far star bene gli altri 9.
    Affinché tutti abbiano ambo i ruoli, Pampinea indica una carica a rotazione giornaliera,
    dove il primo capo è scelto collettivamente e i successivi vengono scelti dai predecessori.
\end{snippet}

\end{document}
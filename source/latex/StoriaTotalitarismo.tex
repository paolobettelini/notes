\documentclass[preview]{standalone}

\usepackage{amsmath}
\usepackage{amssymb}
\usepackage{stellar}
\usepackage{bettelini}
\usepackage{adjustbox}

\hypersetup{
    colorlinks=true,
    linkcolor=black,
    urlcolor=blue,
    pdftitle={Stellar},
    pdfpagemode=FullScreen,
}

\begin{document}

\title{Stellar}
\id{storia-totalitarismo}
\genpage

\section{Il totalitarismo}

\begin{snippetdefinition}{totalitarismo-definition}{Totalitarismo}
    Sistema politico autoritario, in cui tutti i poteri sono concentrati in un partito unico, nel suo
    capo o in un ristretto gruppo dirigente, che tende a dominare l'intera società grazie al controllo
    centralizzato dell'economia, della politica, della cultura, e alla repressione poliziesca.
\end{snippetdefinition}

\begin{snippet}{totalitarismo-expl1}
    Storicamente, il concetto di t. nasce con riferimento alle esperienze del fascismo italiano: in un
    articolo scritto da G. Amendola per Il Mondo, nel 1923, si parla del fascismo come «sistema
    totalitario» in quanto «promessa del dominio assoluto e dello spadroneggiamento completo e
    incontrollato nel campo politico e amministrativo» mentre l'uso del sostantivo si fa risalire a L.
    Basso nel 1925. Fu peraltro lo stesso Mussolini a rivendicare per il fascismo una precisa
    «volontà totalitaria», capovolgendo il senso dispregiativo del termine. Estendendosi in seguito
    a connotare sia il regime nazista, sia i vecchi e nuovi sistemi comunisti, il t. è entrato nel
    linguaggio comune per descrivere una forma politica caratterizzata da assenza di strutture e
    controlli parlamentari, dalla presenza di un partito unico, dalla soppressione delle garanzie di
    libertà e pluralismo proprie dello Stato di diritto. Il modello totalitario prevede la preminenza
    del partito unico sullo Stato; un radicale antipluralismo politico e sociale; l'ideologia della
    «rivoluzione permanente» e del «nemico oggettivo» per tenere alta la mobilitazione del
    consenso di massa; l'impiego massiccio delle tecniche di comunicazione come strumenti
    di propaganda; l'uso sistematico del terrore come strumento di governo. In questo senso i
    regimi moderni di t. si differenziano non solo dalla democrazia, ma anche dall'autoritarismo,
    nel quale sono presenti alcuni di questi elementi ma non tutti assieme e con lo stesso grado di
    intensità. In particolare, i regimi autoritari sono diversi dai regimi totalitari per il fatto di
    ammettere limitate forme di pluralismo, sia sociale sia politico, nella misura in cui risultino
    funzionali alle strategie di mantenimento delle riserve di sostegno e di controllo sociale.
    \\\\
    Il totalitarismo è caratterizzato dalle seguenti proprietà:
    \begin{itemize}
        \item culto della personalità;
        \item assenza di strutture e controlli parlamentari;
        \item monopartitismo;
        \item uso estensivo della propaganda;
        \item soppressione della libertà fondamentale o pluralismo;
        \item preminenza del partito unico sullo Stato;
        \item controllo della popolazione;
        \item soppressione della libertà;
        \item dottrina e obiettivi ben determinati;
        \item idealizazzione del nemico comune.
    \end{itemize}
\end{snippet}

% CONTROLLARE CHE QUESTA TABELLA SIA GIUSTA
\begin{snippet}{tabella-totalitarismo}
    \begin{table}[htbp]
    \begin{adjustbox}{width=\columnwidth,center}
    \begin{tabular}{|c|c|c|c|}
        \hline \begin{tabular}{l} 
        Regimi \\
        totalitari
        \end{tabular} & Fascismo & Nazismo & Stalinismo \\
        \hline \begin{tabular}{l} 
        Fondato \\
        \end{tabular} & 1919 & 1920 & 1927 \\
        \hline \begin{tabular}{l} 
        Inizio \\
        \end{tabular} & 1922 & 1933 & 1929 \\
        \hline Fine & $1943(1945)$ & 1945 & 1953 \\
        \hline \begin{tabular}{c} 
        Fondatore \\
        \end{tabular} & B. Mussolini & A. Hitler & J. Stalin \\
        \hline Ideologia & \begin{tabular}{l} 
        Supremazia culturale \\
        e nazionalismo \\
        \end{tabular} & \begin{tabular}{l}
        \begin{tabular}{l} 
        Razzismo biologico e \\
        nazionalismo
        \end{tabular} \\
        \end{tabular} & \begin{tabular}{l}
        \begin{tabular}{l} 
        Socialismo in un solo \\
        paese (U.R.S.S.)
        \end{tabular} \\
        \end{tabular} \\
        \hline Caratteristica & \begin{tabular}{l} 
        Compromesso tra \\
        monarchia e paese\\
        su criteri razziali e \\
        di prestazione a favore \\
        della collettività e
        \\non su distinzione \\
        sociale.
        \end{tabular} & \begin{tabular}{l} 
        Stato-Partito \\
        controllo totale su\\
        tutti gli aspetti \\
        della vita del paese \\
        (politica, economia, \\
        società, cultura) \\
        \end{tabular} & \begin{tabular}{l} 
        Stato-Partito \\
        controllo totale su tutti gli\\
        aspetti della vita del paese\\
        (politica, economia, società,\\
        cultura)
        \end{tabular} \\
        \hline \begin{tabular}{l} 
        Forma di \\
        governo
        \end{tabular} & \begin{tabular}{l} 
        Dittatura del \\
        partito unico. Mussolini \\
        “duce" e capo \\
        del governo. \\
        Potere legiferante del \\
        governo
        \end{tabular} & \begin{tabular}{l} 
        Dittatura del partito \\
        unico potere legiferante \\
        del Führer
        \end{tabular} & Dittatura del partito unico \\
        \hline \begin{tabular}{l} 
        Sistema di \\
        governo
        \end{tabular} & \begin{tabular}{l} 
        Apparato burocratico in \\
        parte espresso dal partito \\
        ed in parte preesistente
        \end{tabular} & \begin{tabular}{l} 
        Apparato burocratico  \\
        espresso dal partito
        \end{tabular} & \begin{tabular}{l} 
        Apparato burocratico \\
        espresso dal partito
        \end{tabular} \\
        \hline Politica & \begin{tabular}{l} 
        Blocco di potere tra i \\
        maggiori gruppi industriali \\
        e finanziari e partito \\
        \end{tabular} & \begin{tabular}{l} 
        Compromesso tra nazisti ed i \\
        gruppi dirigenti tradizionali
        \end{tabular} & \begin{tabular}{l} 
        Intervento diretto dello \\
        stato
        \end{tabular} \\
        \hline Politica estera & \begin{tabular}{l} 
        Imperialismo \\
        alleanza con Germania \\
        (1937) e Giappone; \\
        influenza sui partiti \\
        fascisti europei.
        \end{tabular} & \begin{tabular}{l} 
        "Diritto" allo spazio vitale \\
        alleanza con Italia e Giappone - \\
        patto con U.R.S.S. (1939-41)
        \end{tabular} & \begin{tabular}{l} 
        Guida dei partiti comunisti \\
        mondiali (Comintern) \\
        Rapporti economici con \\
        diversi stati - patto con \\
        Germania (1939-41) \\
        alleanza con G. Bretagna, \\
        Francia, U.S.A (1941-45) \\
        \end{tabular} \\
        \hline Classe dirigente & \begin{tabular}{l} 
        Classe di amministratori \\
        politici di nomina \\
        governativa selezionata \\
        nel partito. \\
        Dirigenti politici dotati di \\
        scarsa autonomia politica.
        \end{tabular} & \begin{tabular}{l} 
        Classe di amministratori \\
        politici Dirigenti \\
        politici dotati di una \\
        certa autonomia \\ 
        decisionale
        \end{tabular} & \begin{tabular}{l} 
        Classe di amministratori \\
        economici \\
        Dirigenti politici \\ 
        privi di \\
        autonomia decisionale
        \end{tabular} \\
        \hline \begin{tabular}{l} 
        Tipo di \\
        economia
        \end{tabular} & \begin{tabular}{l} 
        Incentivi statali \\
        all'industria privata; \\
        Industrie ed aziende di \\
        Stato; \\
        Bonifica agraria; \\
        Organizzazione \\
        corporativa del lavoro; \\
        Legislazione sociale. \\
        \end{tabular} & \begin{tabular}{l} 
        Forti incentivi statali \\
        all'industria privata; \\
        Organizzazione del lavoro; \\
        Lavoro obbligatorio; \\
        Controllo dell'industria \\ bellica; \\
        Controllo delle risorse; \\
        Legislazione sociale.
        \end{tabular} & \begin{tabular}{l} 
        Pianificazione statale; \\
        Collettivizzazione dei \\
        sistemi di produzione; \\
        Industrializzazione forzata; \\
        Scomparsa dello spirito \\
        imprenditoriale; \\
        Creazione del "mito" del \\
        lavoro collettivo.
        \end{tabular} \\
        \hline \begin{tabular}{l} 
        Rapporto tra \\
        partito e società
        \end{tabular} & \begin{tabular}{l} 
        Fascismo come “religione \\
        civile”;\\
        riti politici collettivi;\\
        organizzazione della\\
        società attraverso\\
        strutture ricreative\\
        dipendenti dal partito
        \end{tabular} & \begin{tabular}{l} 
        Culto del capo carismatico; \\
        riti politici collettivi; \\
        organizzazione \\ capillare delle \\
        masse.
        \end{tabular} & \begin{tabular}{l} 
        Culto di Stalin; \\
        idolatria dei capi;\\
        riti politici collettivi;\\
        acculturazione di massa;\\
        propaganda;
        \end{tabular}\\
        \hline \begin{tabular}{l} 
        Libertà di \\
        espressione
        \end{tabular} & \begin{tabular}{l} 
        Censura; \\
        Creazione del consenso; \\
        Limiti alla libertà \\
        individuale; \\
        Controllo della scuola; \\
        Assimilazione delle \\
        minoranze linguistiche; \\
        Orientamento per mezzo \\
        della propaganda. \\
        Compromesso con la \\
        Chiesa Cattolica sulla \\
        religione di stato. \\
        \end{tabular} & \begin{tabular}{l} 
        Censura; \\
        Controllo dell'attività  \\
        culturale ed artistica; \\
        limiti alla libertà personale; \\
        Controllo della scuola; \\
        Completo disprezzo dei diritti \\
        umani; \\
        manipolazione per mezzo della \\
        propaganda. \\
        Controllo della Chiesa \\
        Protestante.
        \end{tabular} & \begin{tabular}{l} 
        Censura; \\
        Controllo dell'attività \\
        culturale ed artistica; \\
        mancanza di libertà \\
        personale; \\
        Controllo della scuola; \\
        Dopo il 1945, persecuzione \\
        delle minoranze sospettate \\
        di tradimento. \\
        Persecuzione religiosa - poi \\
        compromesso con gli \\
        Ortodossi. \\
        \end{tabular} \\
        \hline \begin{tabular}{l} 
        Avversari \\
        politici
        \end{tabular} & \begin{tabular}{l} 
        Scioglimento di partiti e \\
        sindacati avversari. \\
        Persecuzione degli \\
        avversari. \\
        Demonizzazione del \\
        complotto "delle \\
        democrazie plutocratico- \\
        massoniche" \\
        \end{tabular} & \begin{tabular}{l} 
        Persecuzione ed eliminazione \\
        degli avversari \\
        Demonizzazione del "complotto \\
        internazionale"
        \end{tabular} & \begin{tabular}{l} 
        Caccia ai nemici del popolo \\
        Demonizzazione delle \\
        "oscure forze del male"
        \end{tabular} \\
        \hline Stato di polizia & \begin{tabular}{l} 
        Polizia segreta agli ordini \\
        di Mussolini (Ovra); \\
        Tribunale Speciale per i \\
        reati contro lo Stato; \\
        carcere politico e confino; \\
        Milizia Volontaria per la \\
        Sicurezza Nazionale: corpo \\
        militare alle dipendenze \\
        del partito; \\
        ripristino della condanna a \\
        morte; \\
        leggi razziali dal 1938.
        \end{tabular} & \begin{tabular}{l} 
        Polizia segreta \\ agli ordini di \\
        Himmler (Gestapo); \\
        Repressione poliziesca; \\
        1934: notte dei lunghi \\
        coltelli - eliminati \\
        avversari interni; \\
        SS: corpo autonomo \\ all'interno \\
        dello Stato tedesco, \\
        con proprie leggi e codici. \\
        Lager: campi di \\ concentramento \\
        (dal 1941 di sterminio) per \\
        avversari, malati mentali, \\
        asociali, omosessuali, \\ ebrei, slavi,
        \end{tabular} & \begin{tabular}{l} 
        Polizia segreta agli ordini di \\
        Stalin (Nkvd); \\
        1936-38: purghe di Mosca \\
        eliminati avversari interni \\
        nel partito e \\nell'esercito; \\
        Gulag: campi di \\
        concentramento per gli \\
        avversari.
        \end{tabular} \\
        \hline
    \end{tabular}
    \end{adjustbox}
    \end{table}
    \vspace{0.25cm}
\end{snippet}

\section{Fascismo}

\begin{snippet}{fasi-fascismo}
    La storia del fascismo può essere divisa in quattro fasi:
    \begin{enumerate}
        \item \textbf{Fascismo delle origini:} dal 1919 al 1922, dalla fondazione dei Fasci di combattimento a Milano (San
        Sepolcro) fino all'affidamento a Mussolini dell'incarico di presidente del Consiglio del Regno d'Italia.
        \item \textbf{Fascismo legale:} tra il 1922 al 1924, periodo durante il quale il Partito nazionale fascista (PNF) guida l'Italia
        con un governo di coalizione rispettando la Costituzione (Statuto albertino), anche qui però i fascisti
        condussero un'occupazione sistematica delle posizioni di potere, con un contorno di violenza che lo stesso
        Mussolino non volle o non seppe limitare.
        \item \textbf{Regime fascista:} dal 1925 al 1943, è la fase più lunga e importante, Inizia con il delitto Matteotti e pone
        fine all'ambiguità legalitaria del primo governo Mussolini. Dal 1925 il PNF si trasformò in un vero regime
        autoritario, basato sulla soppressione delle libertà politiche e delle principali libertà civili.
        \item \textbf{Ultima fase:} 25 luglio 1943 - 25 aprile 1945, dall'esautorazione di Mussolini, alla repubblica di Salò fino alla
        liberazione nazionale dall'occupazione nazifascista.
    \end{enumerate}
\end{snippet}

\begin{snippetdefinition}{partito-popolare-italiano-definition}{Partito Italiano Popolare}
    Il \textit{Partito Popolare Italiano} (PPI) è stato un partito politico italiano
    fondato nel 1919 da Don Luigi Sturzo. 
\end{snippetdefinition}

\begin{snippet}{ppi-expl}
    Il PPI è un partito di cattolici ma non cattolico.
    Esso unisce diverse ideologia politiche.
\end{snippet}

\includesnpt{biennio-rosso-definizione}

% Paura che ciò possa portare ciò che è successo in russia.

\begin{snippetdefinition}{manifesto-fasci-italiani-definition}{Manifesto dei Fasci italiani di combattimento}
    Il \textit{Manifesto dei Fasci italiani di combattimento},
    anche detto \textit{Programma di San Sepolcro},
    fu pubblicato nel 1919.
\end{snippetdefinition}

\begin{snippetdefinition}{squadrismo-definition}{Squadrismo}
    Lo \textit{squadrismo} fu un fenomeno politico-sociale verificatosi in Italia
    a partire dal 1919 che consistette nell'uso di squadre d'azione paramilitari armate
    che avevano lo scopo d'intimidire e reprimere violentemente gli
    avversari politici, specialmente quelli appartenenti al movimento
    operaio; fu in breve tempo assorbito dal fascismo che lo usò come
    strumento della propria affermazione.
    Questo movimento prova quindi ad arginare il movimento socialista.
    Gli industriali temono che il socialismo (data la rivoluzione appena avvenuta)
    perché porebbe succedere qualcosa di analogo.
    Anche gli agrari supportano il fascismo.
\end{snippetdefinition}

\begin{snippet}{squadrismo-expl}
    Lo sguadrismo usa la violenza per andare contro le leghe rosse
    (organizzazioni paramilitari di ispirazione socialista o comunista),
    leghe bianche (di ispirazione cattoliche), sindacati e socialisti.
    La violenza aveva anche lo scopo di intimidire l'opinione pubblica, suscitare
    sfiducia verso l'inefficienza del governo, provocando la richiesta di interventi autoritari.
\end{snippet}

\begin{snippetdefinition}{partito-nazionale-fascista-definition}{Partito Nazionale Fascista}
    Nel 1921 Mussolini convoca un congresso a Roma e decise di dare al movimento dei
    Fasci di combattimento una struttura organizzata con la fondazione del
    \textit{Partito nazionale fascista} (PNF).
\end{snippetdefinition}

\begin{snippetdefinition}{marcia-su-roma-definition}{Marcia su Roma}
    La \textit{marcia su Roma} fu una manifestazione armata eversiva
    organizzata dal Partito Nazionale Fascista, volta al colpo di Stato con
    l'obiettivo di favorire l'ascesa di Benito Mussolini
    alla guida del governo in Italia.
    Il 28 ottobre 1922 migliaia di fascisti si diressero verso la
    capitale minacciando la presa del potere con la violenza. 
\end{snippetdefinition}

\begin{snippetdefinition}{discorso-bivacco-definition}{Il discorso del bivacco}
    Dopo la marcia su Roma e l'incarico di formare un nuovo governo, Mussolini parla alla Camera
    dei deputati il 16 novembre 1922, per ottenere la fiducia. Dal discorso pronunciato in quella
    sede emerge tutto il disprezzo mussoliniano per le istituzioni dello Stato liberale, cui si
    accompagna la rivendicazione del carattere rivoluzionario del fascismo e della legittimità del
    ricorso alla forza, a sostegno dell'azione di governo. Sottolinea che con lui è cambiato non solo
    il governo, ma tutto il modo di fare politica: il suo ministero rappresenta la rivoluzione fascista
    che verrà difesa anche a costo di rovesciare il regime parlamentare.
\end{snippetdefinition}

\begin{snippet}{discorso-del-bivacco-expl}
    Il discorso è autoritario ed è una minaccia nei confronti di parlamento.
    XXX
\end{snippet}

% il delitto matteotti

% leggi fascistissime
\begin{snippetdefinition}{leggi-fascistissime-definition}{Leggi fascistissime}
    Le \textit{leggi fascistissime}, o \textit{leggi eccezionali del fascismo},
    furono una serie di atti normativi, emanati tra il 1925 e il 1926,
    che iniziarono la trasformazione dell'ordinamento giuridico del Regno d'Italia
    nel regime fascista.
\end{snippetdefinition}

\begin{snippet}{leggi-fascistissime-expl1}
    Le leggi fascistissime implicano i seguenti punti:
    \begin{itemize}
        \item chiusura di circoli e associazioni politiche antifascisti;
        \item capo del governo investito di ampi poteri, diventata responsabile solo davanti al (e non più al parlamento);
        \item rafforzare le funzioni di foverno può emanare leggi senza riferire al parlamento;
        \item soppressi tutti i partiti politici (salvo il PNF);
        \item vengono soppressi i sindacati (socialisti e cattolici), così come lo sciopero e la serrata;
        \item soppressa la libertà di stampa e di associazione;
        \item soppressa la possibilità del popolo di eleggere i municipi; viene inserita la figura del podestà;
        \item viene creata una polizia politica, l'OVRA (Opera per la Vigilanza e la Repressione dell'Antifascismo)
            e la milizia volontaria per la sicurezza nazionale (MVSN);
        \item viene reintrodotta la pena di morte;
        \item viene istituito un tribunale speciale per la difesa dello stato (giudica i reati di natura politica).
    \end{itemize}
\end{snippet}

\begin{snippetdefinition}{ovra-definition}{OVRA}
    L'\textit{OVRA}, sigla di \textit{Opera Vigilanza (o Volontaria) Repressione Antifascismo},
    è stata la denominazione non ufficiale della polizia politica dell'Italia fascista dal 1927 al 1943 e
    nella Repubblica Sociale Italiana dal 1943 al 1945, costituita dopo l'emanazione delle leggi fascistissime nel 1926. 
\end{snippetdefinition}

\begin{snippet}{leggi-fascistissime-expl2}
    Alcuni di questi elementi sono estremamente analoghi a quelli del socialismo, come la pena di morte,
    le squadre speciali di polizia politica etc.
\end{snippet}

% Uccisione di Matteoti

% Quello dell'italia monarchica è un parlamento bicamerale

% Il capo dell'amministrazione comunale, durante il regime fascista. (scelti da mussolini)

\subsubsection{Il plebiscito del 1929}

\begin{snippetdefinition}{plebiscito-definition}{Plebiscito}
    Il \textit{plebiscito} indica \quotes{ciò che è stabilito dalla plebe}.
    Nell'antica Roma, infatti, riguardava quelle
    norme votate dalla plebe su proposta dei tribuni, vincolanti solo per la prima e non per i
    secondi.
\end{snippetdefinition}

\begin{snippet}{plebiscito-expl1}
    Negli anni delle dittature totalitarie il plebiscito perse i caratteri di interrogazione della
    volontà collettiva in merito alle grandi questioni della vita politica (come la sovranità
    territoriale o la forma dello Stato) per assumere il significato di manifestazione plateale di
    un consenso unanime, che concede poco o nulla al dissenso, certificando piuttosto la volontà
    non della collettività ma dei gruppi dirigenti. Così accade nel caso delle consultazioni del
    1929, dove agli elettori non venne data la possibilità di scegliere i singoli candidati su liste
    contrapposte, ma solo quella di apporre una croce di assenso al quesito contenuto nella
    scheda.
    Nei sistemi il plebiscito divenne quindi strumento per consolidare la volontà del regime.
    Un plebiscito è molto più facile da controllare rispetto ad altri tipi di votazioni.
\end{snippet}

% FOTO con busta sì e no
% FOTO campagna elettorale per il plebiscito

\begin{snippetdefinition}{patti-lateranensi-definition}{Patti Lateranensi}
    I \textit{Patti Lateranensi} sono gli accordi sottoscritti tra
    il Regno d'Italia e la Santa Sede l'11 febbraio 1929
    contenenti un trattato, una convenzione e un concordato.
    Ai Patti si devono l'istituzione della Città del Vaticano
    come Stato indipendente e la piena riapertura formale dei
    rapporti fra Italia e Santa Sede, interrotti nel 1870 ma
    gradualmente riallacciati nei decenni successivi fino alla loro
    definitiva sistemazione con la stipula di tali accordi.
\end{snippetdefinition}

\begin{snippet}{patti-lateranensi-expl}
    Il trattato internazione pone fine alla Quoestione romana, che affonda
    le sue radici nel momento dell'unificazione italiana (1860-1861).
    Roma sarà conquistata e annessa all'italia solo nel 1870.
    \\\\
    Il \textit{trattato internazionale}:
    \begin{itemize}
        \item Santa sede riconosce la sovranità del regno d'Italia con capitale Roma;
        \item a sua volta lo stato italiano riconosce Città del Vaticano come autonoma.
    \end{itemize}
    Il \textit{concordato}:
    \begin{itemize}
        \item definisce il ruolo della religione cattolica in italia (es. la religione cattolica
        viene riconosciiuta come religione di Stato e viene insegnata nelle scuole).
        Ciò si oppone alla laicità dello Stato.
    \end{itemize}
\end{snippet}

\subsubsection{Monopolio dei mezzi di comunicazione di massa}

\begin{snippet}{mezzi-comunicazione-fascismo-expl}
    Rivoluzione \(\rightarrow\) Mutare la società italiana \(\rightarrow\) fascismo garante
    dell'ordine e della modernizzazione del paese.
    oltre a mutare la società, mira a mutare il singolo, un uomo nuovo, che deve
    essere pronto a battersi per la causa nazionale.
    Il fascismo mira a mutare l'educazione scolastica: i testi scolastici
    vengono controllati dal regime, gli esercizi di matematica sono incentrati
    su valori fascisti. Oggetti che non potevano mancare nelle aule erano il crocificco,
    il ritratto del re e del duce.
    Esso controlla quindi capillarmente l'informazione di massa (giornali, radio,
    cinema).
\end{snippet}

\begin{snippetdefinition}{ministero-cultura-popolare-definition}{Ministero della cultura popolare}
    Il \textit{Ministero della cultura popolare} (MinCulPop) era un ministero
    del governo italiano con compiti riguardando la cultura popolare
    e la propaganda fascista fra il 1937 e il 1944.
\end{snippetdefinition}

\plain{I comunicati ufficiali dell'MinCulPop erano chiamati le veline. Essi fornivano direttive precise per i media}

\begin{snippetdefinition}{ente-italiano-audizioni-radiofoniche-definition}{Ente italiano audizioni radiofoniche}
    L'\textit{Ente italiano audizioni radiofoniche} (EIAR) fu la società anonima titolare
    della concessione in esclusiva delle trasmissioni radiofoniche circolari in Italia
    dal 1927 al 1944.
\end{snippetdefinition}

\plain{L'azienda venne riaperta con la denominazione di Rai (Radio Audizioni Italiane) e nel 1954 come Radiotelevisione Italiana.}

\begin{snippetdefinition}{istituto-luce-definition}{Istituto LUCE}
    L'\textit{istituto luce} (L'Unione Cinematografica Educativa)
    è stata una società fascista creata nel 1924, celebre per essere divenuta un grande
    strumento di propaganda.
\end{snippetdefinition}

\begin{snippet}{etiopia-fascismo-expl}
    Il fascismo italiano cercò di espandere il proprio impero coloniale.
    In particolare, l'Abissinia (oggi Etiopia) fu uno degli obiettivi principali.
    L'Etiopia faceva parte della società delle nazioni.
    \begin{enumerate}
        \item Il 3 ottobre 1935 le truppe italiane penetrano in Etiopia, muovendo dai possedimenti italiani in Eritrea e Somalia;
        \item il 3 maggio 1936 le truppe italiane raggiungono Addis Abeba costringendo alla fuga il negus (re) Haile Salassie I;
        \item le basi di rifornimento sono piuttosto distanti: ogni aiuti in uomini e mezzi deve infatti partire dall'Italia, cioè da 4000 chilometri di distanza da Massaua e 80000 da Mogadiscio, i due soli porti dei quali è possibile usufruire.
    \end{enumerate}
\end{snippet}

\subsubsection{Leggi per la difesa della razza}

\begin{snippet}{leggi-difese-razza-fascismo}
    Le leggi razziali fasciste furono un insieme di provvedimenti legislativi e amministrativi
    emanati e applicati in Italia fra il 1938 e il primo lustro degli anni quaranta,
    dapprima del regime fascista del Regno d'Italia e poi dalla Repubblica Sociale Italiana,
    rivolti prevalentemente verso le persone ebree.
\end{snippet}

\subsubsection{Morte di Mussolini}

\includesnpt[src=/snippet/static/mussolini-death.webp|width=50\%]{centered-img}

\end{document}
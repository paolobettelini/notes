\documentclass[preview]{standalone}

\usepackage{amsmath}
\usepackage{amssymb}
\usepackage{stellar}
\usepackage{definitions}
\usepackage{tikz}
\usepackage{adjustbox}

\usetikzlibrary{cd}

\begin{document}

\id{categorytheory-monoids-groupoids}
\genpage

\section{Definitions}

\begin{snippetdefinition}{categorical-monoid-definition}{Monoid}
    A \emph{monoid} is a \category with only \(1\) object.
\end{snippetdefinition}

\begin{snippetdefinition}{groupoid-definition}{Grupoid}
    A \emph{grupoid} is a \category
    where every \category[morphism][Morphism] is an \catisomorphism.
\end{snippetdefinition}

\plain{A grupoid with only one element is equivalent to a group, where every morphism is an element
of the group and the composition is equivalent to the group binary operation.}

\begin{snippetproposition}{groupoid-identity-transformation-center}{}
    Let \(G\) be a \groupoid.
    Then, the natural transformations from the identity functor to itself
    correspond exactly to \(\groupcenter(G)\).
\end{snippetproposition}

\begin{snippetproof}{groupoid-identity-transformation-center-proof}{groupoid-identity-transformation-center}{}
    Let \(\ast\) be the only object of \(G\).
    We want to consider a natural transformation between
    these two functors \(\alpha_\ast: 1_G \fromto 1_G\) \\
    \adjustbox{scale=1.25,center}{%
        % https://tikzcd.yichuanshen.de/#N4Igdg9gJgpgziAXAbVABwnAlgFyxMJZABgBpiBdUkANwEMAbAVxiRAHEQBfU9TXfIRQBGclVqMWbTl3EwoAc3hFQAMwBOEALZIyIHBCSiJzVohDCA+p2pwAFllU4kAWmE81mnYj0Gjthyd-EylzKxsQBjoAIxgGAAV+PAI2dSwFO2dZLiA
        \begin{tikzcd}
        G \arrow[r, "1_G", shift left] \arrow[r, "1_G"', shift right] & G
        \end{tikzcd}
    }
    We need the following diagram to commute: \\
    \adjustbox{scale=1.25, center}{%
        % https://tikzcd.yichuanshen.de/#N4Igdg9gJgpgziAXAbVABwnAlgFyxMJZABgBpiBdUkANwEMAbAVxiRAEYB9AcQAoAdfnTg4AlCAC+pdJlz5CKMuyq1GLNlz6DhYydJAZseAkXall1es1aIOPAUJHipMo-NPkVV9bc0OdziowUADm8ESgAGYAThAAtkhkIDgQSADMlmo2dnwhzvox8UhmyamIAEyZ1hr2eSDUDHQARjAMAAqyxgog0VghABY4elGxCYgZpUiVqtW22gxo-XSc2iLDIIVjSSnFVT4g84vLq0MNza0dbia2vQNDEhQSQA
        \begin{tikzcd}
        1_G(\ast) \arrow[r, "1_G(g)"] \arrow[d, "\alpha_\ast"'] & 1_G(\ast) \arrow[d, "\alpha_\ast"] \\
        1_G(\ast) \arrow[r, "1_G(g)"']                          & 1_G(\ast)                         
        \end{tikzcd}
    }
    The morphisms on \(\ast\) are the elements of the \group, meaning that
    \(\alpha_\ast\) is an element of the \group.
    The composition corresponds to the group multiplication.
    In order for the diagram to commute, we need \(\alpha_\ast\)
    to commute with every other element in the \group.
    This is equivalent to saying that \(\alpha_\ast \in \groupcenter(G)\).
\end{snippetproof}

\end{document}
\documentclass[preview]{standalone}

\usepackage{amsmath}
\usepackage{amssymb}
\usepackage{stellar}
\usepackage{definitions}
\usepackage{bettelini}
\usepackage{tikz}

\usetikzlibrary{decorations.markings, arrows.meta}

\begin{document}

\id{differential-forms-exercises}
\genpage

\section{Exercises}

\begin{snippetexercise}{differential-forms-ex1}{}
    Let
    \[
        \omega = (x^4 - y)\,dx + (x + y^2)\,dy
    \]
    and compute its integral on
    \begin{center}
        \tikzset{
            mid arrow/.style={
                postaction={
                    decorate,
                    decoration={
                        markings,
                        mark=at position 0.6 with {\arrow[scale=1]{Latex}}
                    }
                }
            }
        }
        \tikzset{
            quarter arrow/.style={
                postaction={
                    decorate,
                    decoration={
                        markings,
                        mark=at position 0.3 with {\arrow[scale=1]{Latex}}
                    }
                }
            }
        }
        \begin{tikzpicture}
            \draw[->, gray, thick] (-1.5, 0) -- (1.5, 0) node[right] {$x$};
            \draw[->, gray, thick] (0, -1.5) -- (0, 1.5) node[above] {$y$};

            \def\R{1} 
            \coordinate (StartSemicircle) at (\R, 0);   % (1, 0)
            \coordinate (EndSemicircle) at (-\R, 0);    % (-1, 0)
            \coordinate (Origin) at (0, 0);             % (0, 0)
            \coordinate (PointMinus1) at (0, -1);       % (0, -1)
            \coordinate (PointPlus1) at (0, 1);         % (0, 1)

            \begin{scope}[color=red!80!black, very thick, line join=round, line cap=round]
                \draw[quarter arrow] (StartSemicircle) arc (0:180:\R);
                \draw[mid arrow] (EndSemicircle) -- (Origin);
                \draw[mid arrow] (Origin) -- (PointMinus1);
                \draw[mid arrow] (PointMinus1) -- (StartSemicircle);
            \end{scope}

            \fill (PointMinus1) circle (1pt) node[right] {$-1$};
            \fill (PointPlus1) circle (1pt) node[above right] {$1$};
            \fill (StartSemicircle) circle (1pt) node[below right] {$1$};
            \fill (EndSemicircle) circle (1pt) node[below left] {$-1$};
        \end{tikzpicture}
    \end{center}
\end{snippetexercise}

\begin{snippetsolution}{differential-forms-ex1-sol}{}
    We observe that the curve is piecewise smooth and closed.
    Thus, we split the integral into 4 obvious paths.
    Since it is closed, if the differential form is exact then the integral is zero.
    Since we are in \(\realnumbers^2\) (star-shaped), we check if it is closed. We have
    \begin{align*}
        \frac{\partial X}{\partial y} &= -1 \quad \frac{\partial Y}{\partial x} = 1
    \end{align*}
    Clearly it is not closed, but we can rewrite it as
    \[
        \omega = (x^4 - y)\,dx + (x +y^2)\,dy = (x^4 + y)\,dx + (x + y^2)\,dy - 2x\,dx
    \]
    So \(\omega_1\) is an exact form, and we are left with the integral over \(\omega_2\).
    We can parametrize (possibly over the same interval)
    \begin{align*}
        &\varphi_1(t) = (\cos(\pi t), \sin(\pi t)) \\
        &\varphi_2(t) = (t-1, 0) \\
        &\varphi_3(t) = (0, -1 -t) \\
        &\varphi_4(t) = (t, t-1)
    \end{align*}
    with \(t\in[0,1]\).
    We are left with
    \begin{align*}
        \int_{\varphi} \omega &= \int_\varphi \omega_2 = -2\int_\varphi y\,dx \\
        &= \integral[0][1][
            -\pi \sin^2 (\pi t) + 0 + 0 + (t-1)
        ][t]=\pi+1
    \end{align*}
\end{snippetsolution}

\end{document}
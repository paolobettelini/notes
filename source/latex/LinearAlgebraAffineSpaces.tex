\documentclass[preview]{standalone}

\usepackage{amsmath}
\usepackage{amssymb}
\usepackage{stellar}
\usepackage{definitions}

\begin{document}

\id{affine-spaces}
\genpage

\section{Affine spaces}

\plain{We want to be able to describe geometric spaces where points,
rather than vectors, are the fundamental objects, allowing us to formalize
translations and work in settings where there is no natural origin.}

\begin{snippetdefinition}{affine-space-definition}{Affine space}
    An \emph{affine space} on a \field \(\mathbb{F}\) is a \set
    \(\mathcal{V}\) such that there exist 
    \begin{enumerate}
        \item a \vectorspace \(V\) over \(\mathbb{F}\);
        \item a map \(\vec{pq}\colon \mathcal{V} \cartesianprod \mathcal{V} \to V\)
    \end{enumerate}
    such that
    \begin{enumerate}
        \item \(\forall p \in \mathcal{V}\) fixed, the map \(\mathcal{V} \to V\) given by \(q \mapsto \vec{pq}\) is \bijective;
        \item \emph{Chasles relation:} \(\forall p, q, r \in \mathcal{V}\), \(\vec{pq} = \vec{pq} + \vec{rq}\).
    \end{enumerate}
    The elements of \(\mathcal{V}\) are called \emph{points} and \(V\) is called the \emph{direction space} of \(\mathcal{V}\).
\end{snippetdefinition}

\end{document}
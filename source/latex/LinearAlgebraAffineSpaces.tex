\documentclass[preview]{standalone}

\usepackage{amsmath}
\usepackage{amssymb}
\usepackage{stellar}
\usepackage{definitions}

\begin{document}

\id{affine-spaces}
\genpage

\section{Affine spaces}

\plain{We want to be able to describe geometric spaces where points,
rather than vectors, are the fundamental objects, allowing us to formalize
translations and work in settings where there is no natural origin.}

\begin{snippetdefinition}{affine-space-definition}{Affine space}
    An \emph{affine space} on a \field \(\mathbb{F}\) is a \set
    \(A\) such that there exist 
    \begin{enumerate}
        \item a \vectorspace \(V\) over \(\mathbb{F}\);
        \item a \function \(\vec{pq}\colon A \cartesianprod A \to V\)
    \end{enumerate}
    such that
    \begin{enumerate}
        \item \(\forall p \in A\) fixed, the map \(A \to V\) given by \(q \mapsto \vec{pq}\) is \bijective;
        \item \emph{Chasles relation:} \(\forall p, q, r \in A\), \(\vec{pq} = \vec{pr} + \vec{rq}\).
    \end{enumerate}
    The elements of \(A\) are called \emph{points} and \(V\) is called the \emph{direction space} or 
    \emph{associated space} of \(A\).
\end{snippetdefinition}

\begin{snippetdefinition}{affine-space-dimensiona-definition}{Affine space dimension}
    Let \(A\) be an \affinespace over a \field \(\mathbb{F}\) with associated \vectorspace \(V\).
    Then, the \emph{dimension} of \(A\) is defined as the dimension of \(V\) over \(\mathbb{F}\).
\end{snippetdefinition}

\begin{snippetproposition}{vector-space-canonincal-affine-form}{Canonincal affine form}
    Let \(V\) be a \vectorspace over a \field \(\mathbb{F}\).
    Then, the \set \(A = V\) is an \affinespace over \(\mathbb{F}\) with associated \vectorspace \(V\) and the map
    \[
        \vec{pq} = q - p
    \]
\end{snippetproposition}

\begin{snippetproof}{vector-space-canonincal-affine-form-proof}{vector-space-canonincal-affine-form}{Canonical affine form}
    \begin{enumerate}
        \item Let \(p\) be fixed. The map \(q \mapsto q - p\) is a \bijective[bijection] since it is a linear map with inverse \(q \mapsto q + p\).
        \item \emph{Chasles relation:} Let \(p, q, r \in A\). Then,
        \begin{align*}
            \vec{pq} = -p + q = r-p+q-r = \vec{pr} + \vec{rq}
        \end{align*}
    \end{enumerate}
\end{snippetproof}

\begin{snippetdefinition}{canonincal-nth-dimensional-affine-space-definition}{Canonincal nth-dimensional affine space}
    Let \(V=\mathbb{F}^n\) be a \vectorspace over a \field \(\mathbb{F}\).
    Then, the \snippetref[vector-space-canonincal-affine-form][canonincal] \affinespace is denoted
    \[
        \mathbb{A}^n(\mathbb{F})
    \]
\end{snippetdefinition}

\begin{snippetproposition}{linear-system-solutions-affine-space}{}
    Let \(A \in \matrices_{m \times n}(\mathbb{F})\)
    and \(b \in \mathbb{F}^m\).
    The \emph{solution set} of the linear system \(Ax = b\) is the \set
    \[
        \text{Sol}(Ax = b) = \{x \in \mathbb{F}^n \mid Ax = b\}
    \]
    is an \affinespace over \(\mathbb{F}\) with associated \vectorspace
    \(\grpker_A\) where
    \[
        \vec{x_1x_2} = x_2 - x_1
    \]
\end{snippetproposition}

\begin{snippetproof}{linear-system-solutions-affine-space-proof}{linear-system-solutions-affine-space}{}
    We have
    \[
        A(x_2 - x_1) = Ax_2 - Ax_1 = b - b = 0 \in \grpker A
    \]
    and
    \begin{enumerate}
        \item Let \(x_1\) be fixed. The map \(x_2 \mapsto x_2 - x_1\) is a \bijective[bijection] since it is a linear map with inverse \(x_2 \mapsto x_2 + x_1\).
        \item \emph{Chasles relation:} Let \(x_1, x_2, x_3 \in A\). Then,
        \begin{align*}
            \vec{x_1x_2} = -x_1 + x_2 = -x_1 + x_3 - x_3 + x_2 = \vec{x_1x_3} + \vec{x_3x_2}
        \end{align*}
    \end{enumerate}
\end{snippetproof}

\begin{snippetproposition}{affine-spaces-properties}{Properties of affine spaces}
    Let \(A\) be an \affinespace over a \field \(\mathbb{F}\) with associated \vectorspace \(V\).
    Then,
    \begin{enumerate}
        \item \(\forall p \in A, \vec{pp} = 0\);
        \item \(\forall p,q \in A, \vec{pq} = -\vec{qp}\);
        \item \(\forall p,p',q',q \in A, \vec{pp'} = \vec{qq'} \iff \vec{pq} = \vec{p'q'}\);
    \end{enumerate}
\end{snippetproposition}

%%%%%%

\begin{snippetproof}{affine-spaces-properties-proof}{affine-spaces-properties}{Properties of affine spaces}
    \begin{enumerate}
        \item Let \(p \in A\). Then, \(\vec{pp} = \vec{pp} + \vec{pp}\).
        \item Let \(p,q \in A\). Then, \(\vec{pp} = \vec{qp} + \vec{qp}\).
        \item Let \(p,p',q',q \in A\). Then,
        \begin{align*}
            0 = \vec{pp'} + \vec{q'q} = \vec{pq} + \vec{qp'} + \vec{q'p'} + \vec{p'q} = \vec{pq} + \vec{q'p'}
        \end{align*}
    \end{enumerate}
\end{snippetproof}

\begin{snippetdefinition}{affine-independence-definition}{Affine independence}
    Let \(A\) be an \affinespace over a \field \(\mathbb{F}\).
    Then, a \set of \affinepoint[points] \(p_0, \ldots, p_k \in A\) is said to be
    \emph{affinely independent} if
    \[
        \lineardim \linearspan(\{\vec{p_0p_1}, \ldots, \vec{p_0p_k}\}) = k
    \]
\end{snippetdefinition}

\begin{snippetdefinition}{affine-frame-definition}{Affine frame}
    Let \(A\) be an \affinespace over a \field \(\mathbb{F}\) with associated \vectorspace \(V\).
    An \emph{affine frame} of \(A\) is a tuple \((p, \{v_1, \cdots, v_n\})\)
    where
    \begin{enumerate}
        \item \(p \in A\) is a \emph{reference point} called the \emph{origin of the frame};
        \item \(\{v_1, \cdots, v_2\}\) is a \basis of \(V\).
    \end{enumerate}
\end{snippetdefinition}

\plain{We can thuse have coordinates in affine spaces given a frame.}

\begin{snippetdefinition}{parallel-affine-spaces-definition}{Parallel affine spaces}
    Let \(A\) and \(B\) be \affinespace over a \field \(\mathbb{F}\) with associated \vectorspace \(V\).
    Then, \(A\) and \(B\) are said to be \emph{parallel} if they have the same direction
    \[
        A \,||\, B
    \]
\end{snippetdefinition}

\plain{Note that using this definition, we cannot have, for instance, a plane being parallel to a line.}

\begin{snippetproposition}{parallel-affine-spaces-either-disjoint-or-same}{}
    Let \(A\) and \(B\) be \affinespace over a \field \(\mathbb{F}\) with associated \vectorspace \(V\).
    Then, \(A\) and \(B\) are either \disjoint or the same.
    \begin{align*}
        A \intersection B = \emptyset \lxor A = B
    \end{align*}
\end{snippetproposition}

\begin{snippetproof}{parallel-affine-spaces-either-disjoint-or-same-proof}{parallel-affine-spaces-either-disjoint-or-same}{}
    If they are not \disjoint, there exist \(p\in A \intersection B\).
    Then, we have
    \[
        A = \{\vec{pa} \suchthat a \in A \}, \quad
        B = \{\vec{pb} \suchthat b \in B \}
    \]
    but since they are parallel, given \(\vec{pa} \in A\) we have \(\vec{pa} \in B\) meaning
    \(a\in W\) and likewise given \(\vec{pb} \in B\) we have \(\vec{pb} \in A\) meaning \(b\in W\).
    Thus, \(A = B\).
\end{snippetproof}

\begin{snippetproposition}{affine-spaces-fifth-postulate}{}
    Let \(A\) be an \affinespace over a \field \(\mathbb{F}\) with associated \vectorspace \(V\).
    Then, the \emph{fifth postulate} of Euclid holds in \(A\)
\end{snippetproposition}

\begin{snippetproof}{affine-spaces-fifth-postulate-proof}{affine-spaces-fifth-postulate}{}
    \todo
\end{snippetproof}

\begin{snippettheorem}{talete-theorem}{Talete's theorem}
    Let \(A\) be an \affinespace over a \field \(\mathbb{F}\) with associated \vectorspace \(V\).
    Let \(l, l', l''\) be parallel lines in \(A\) and let \(r_1, r_2\) be lines that are not parallel to \(l\).
    Also let
    \[
        p_i = r_i \intersection l, \quad
        p'_i = r_i \intersection l', \quad
        p''_i = r_i \intersection l''
    \]
    Then, given \(k_1, k_2 \in \mathbb{F}\) such that \(\vec{p_ip_i''} = k_i\vec{p_ip_i'}\) it follows
    \[
        k_1 = k_2
    \]
\end{snippettheorem}

\begin{snippetproof}{talete-theorem-proof}{talete-theorem}{}
    If \(r_1 = r_2\) we are done. Otherwise,
    by swapping \(l\) with \(l'\) or \(l''\) we can assume \(p_1 \neq p_2\).
    Let \(v = \vec{p_1p_2} \neq 0\). By the Chasles relation, we have
    \[
        \vec{p_2p_2'} = \vec{p_2p_1} + \vec{p_1p_1'} + \vec{p_1'p_2'}
    \]
    and
    \[
        \vec{p_2p_2'} - \vec{p_1p_1'} = \vec{p_2p_1} + \vec{p_1'p_2'} = \alpha v
    \]
    since \(l \,||\, l''\). Likewise for \(\beta v\).
\end{snippetproof}

\end{document}
\documentclass[preview]{standalone}

\usepackage{amsmath}
\usepackage{amssymb}
\usepackage{stellar}
\usepackage{definitions}

\begin{document}

\id{floating-points}
\genpage

\section{Floating points}

\begin{snippetdefinition}{floating-point-definition}{Floating point numbers}
    The \set of \emph{floating point numbers} is
    \begin{align*}
        f(\beta, t, m, M) = \{0, \text{NaN}, \pm\infty\} \cup
        \left\{x = \text{sign}(x) \cdot \beta^e \sum_{i=1}^t y_i \beta^{-i} \suchthat
        t,y_i,m,M \in \naturalnumbers, y_1 \neq 0, -m\leq e \leq M \right\}
    \end{align*}
\end{snippetdefinition}

\begin{snippet}{machine-precision-explanation}
    We can estimate the relative error
    \[
        \frac{|x-\tilde x|}{|x|}
    \]
    where \(x\in\realnumbers\) and \(\tilde x \in f(\beta, t, m, M)\) is its best representation in a computer.
    Consider \(x > 0\).
    Clearly, if \(x \in f(\beta, t, m, M)\), then \(|x-\tilde x| = 0\).
    Otherwise, \(x\in [a,b]\) where \(a,b\in f(\beta, t, m, M)\) and are consecutive in the \set.
    Thus,
    \[
        |x-\tilde x| \leq \frac{b-a}{2}
    \]
    We then have
    \[
        a = \beta^e \sum_{i=1}^t y_i \beta^{-i}
    \]
    and
    \[
        b = \beta^e \left( \sum_{i=1}^t y_i \beta^{-i} + \beta^{-t} \right)
        = a + \beta^{e - t}
    \]
    Thus, the difference is given by
    \[
        |x - \tilde x| \leq \frac{1}{2}\beta^{e-t}
    \]
    We now need to lower bound the normalizing element
    \[
        |x| = \beta^e \sum_{i=1}^\infty y_i\beta^{-i} \geq \beta^e \cdot y_1 \beta^{-1} \geq \beta^{e-1}
    \]
    We then have
    \[
        \frac{1}{|x|} \leq \beta^{1-e}
    \]
    By combining the two results we get
    \[
        \frac{|x-\tilde x|}{|x|} \leq \frac{1}{2}\beta^{e-t} \beta^{1-e}
        = \frac{1}{2} \beta^{1-t} \triangleq u
    \]
    Then, \(u\) is the machine precision.
\end{snippet}

\begin{snippetdefinition}{machine-precision-definition}{Machine precision}
    The \emph{machine precision} is an upper bound on the relative error due to rounding in floating point arithmetic.
    For a floating point system \(f(\beta, t, m, M)\), it is defined as
    \[
        u \triangleq \frac{1}{2} \beta^{1-t}
    \]
\end{snippetdefinition}

\end{document}
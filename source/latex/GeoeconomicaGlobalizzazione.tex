\documentclass[preview]{standalone}

\usepackage{amsmath}
\usepackage{amssymb}
\usepackage{stellar}
\usepackage{bettelini}

\hypersetup{
    colorlinks=true,
    linkcolor=black,
    urlcolor=blue,
    pdftitle={Stellar},
    pdfpagemode=FullScreen,
}

\begin{document}

\title{Geografia economica}
\id{geoeconomica-globalizzazione}
\genpage

\begin{snippetdefinition}{globalizzazione-definizione}{Globalizzazione}
    La \textit{globalizzazione} è un fenomeno che coinvolge l'interconnessione e l'interdipendenza crescente tra le nazioni e le persone in tutto il mondo.
\end{snippetdefinition}

\begin{snippet}{globalizzazione-expl}
    La globalizzazione porta su scala mondiale un incentramento di aspetti economici, sociali, culturali e politici.

    I terminini \textbf{multinazionale} e \textbf{transnazionale} hanno spesso
    un'accezione comune ma possono essere distinti nella seguente maniera
    \begin{itemize}
        \item \textbf{Multinazionale:} si riferisce a un'azienda o un'impresa che
            ha operazioni o filiali in più di un paese.
            Queste aziende hanno una presenza globale e conducono attività in diverse nazioni,
            ma il termine non necessariamente implica che l'azienda sia completamente interconnessa
            o integrata in tutte le operazioni.
            Le multinazionali possono avere sedi in diversi paesi,
            ma le decisioni strategiche e operative possono rimanere decentralizzate.
            \item \textbf{Transnazionale:} si riferisce a un'entità o un'organizzazione che opera
            al di là dei confini nazionali.
            Questo concetto si concentra sull'idea di superare le frontiere nazionali
            e lavorare in un contesto globale,
            con un'accentuata integrazione delle operazioni e delle decisioni su scala internazionale.
            Le aziende transnazionali tendono a essere più interconnesse e
            integrate rispetto alle multinazionali e spesso cercano di operare
            in un modo che superi le limitazioni geografiche e politiche.
    \end{itemize}
\end{snippet}

\end{document}
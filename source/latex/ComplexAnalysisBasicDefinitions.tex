\documentclass[preview]{standalone}

\usepackage{amsmath}
\usepackage{amssymb}
\usepackage{stellar}
\usepackage{definitions}

\begin{document}

\id{complexnumbers-basic-definitions}
\genpage

\section{Basic Definitions}

\begin{snippetdefinition}{complex-real-part-definition}{Real Part}
    The \textit{real part} of a complex number \(z = a+bi \in \complexnumbers\)
    and is defined as
    \[
        \origRe(z) = a
    \]
\end{snippetdefinition}

\begin{snippetdefinition}{complex-imaginary-part-definition}{Imaginary Part}
    The \textit{real part} of a complex number \(z = a+bi \in \complexnumbers\)
    and is defined as
    \[
        \origIm(z) = b
    \]
\end{snippetdefinition}

\begin{snippetproposition}{real-numbers-are-complex}{Real numbers are complex}
    \[\realnumbers \subset \complexnumbers\]
\end{snippetproposition}

\begin{snippetproof}{real-numbers-are-complex-proof}{real-numbers-are-complex}{Real numbers are complex}
    For any number \(r \in \realnumbers\), \(r\)
    is equivalent to \(r + 0i\) and thus \(\realnumbers \subseteq \complexnumbers\).
    However, there are numbers in \(\complexnumbers\) that are not in \(\realnumbers\),
    meaning \[\realnumbers \subset \complexnumbers\]
\end{snippetproof}

\begin{snippetdefinition}{complex-numbers-abs}{Absolute value of complex number}
    The \textit{absolute value} (or \textit{module}) of a complex number \(z = a+bi\) defined as its distance from the origin.
    \[
        |z| = |a+bi| = \sqrt{\Re^2(z) + \Im^2(z)}
    \]
\end{snippetdefinition}

\begin{snippetdefinition}{complex-conjugate-definition}{Complex Conjugate}
    The \textit{complex conjugate} of a complex number \(z=a+bi\) is denoted as \(z^*\) or \(\overline{z}\).
    It is defined as
    \[
        z^* = a-bi
    \]
\end{snippetdefinition}

\begin{snippet}{complex-numbers-conj-geometrical-expl}
    Geometrically, \(z^\complexconj\) is the reflection about the real axis in the complex plane.
\end{snippet}

\begin{snippet}{complex-numbers-conj-properties}
    The conjugate has the following trivial properties
    \begin{align*}
        {(z^\complexconj)}^\complexconj &= z
        \\
        \Re(z^\complexconj) &= \Re(z)
        \\
        \Im(z^\complexconj) &= -\Im(z)
    \end{align*}
\end{snippet}

\begin{snippetdefinition}{complex-numbers-arg}{Complex argument}
    The \textit{argument} of a complex number is the angle formed with the x-axis in
    the complex plane
    \[
        \arg(a+bi)= \arctan \left(\frac{b}{a}\right)
    \]
\end{snippetdefinition}

\end{document}
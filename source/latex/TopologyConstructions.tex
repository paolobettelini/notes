\documentclass[preview]{standalone}

\usepackage{amsmath}
\usepackage{amssymb}
\usepackage{stellar}
\usepackage{definitions}
\usepackage{graphicx}
\usepackage{tikz}
\usepackage{pgfplots}

\usetikzlibrary{3d, decorations.markings, calc, perspective, shadings, calc, arrows.meta}
\pgfplotsset{compat=1.18}

\tikzset{
    labelstyle/.style={align=center, font=\Large, gray!80!black},
    mathstyle/.style={font=\sffamily\Large\bfseries},
    arrowstyle/.style={->, -Latex, thick, gray!60, shorten >= 3pt, shorten <= 3pt},
    boundarythick/.style={line width=1.5pt}
}

% Identification arrow for the squares to be glued
\tikzset{
    identification arrow/.style={
        postaction={
            decorate,
            decoration={
                markings,
                mark=at position 0.525 with {\arrow[black, scale=1.2, -{Triangle}]{>}}
            }
        }
    }
}

\begin{document}

\id{topology-constructions}
\genpage

\section{Spheres and disks}

\begin{snippetdefinition}{disk-definition}{Disk}
    The \emph{disk} is defined as
    \[
        D^n = \{
            x \in \realnumbers^n
            \suchthat ||x|| \leq 1
        \}
    \]
\end{snippetdefinition}

\begin{snippetdefinition}{sphere-definition}{Sphere}
    The \emph{sphere} is defined as
    \[
        S^n = \{
            x \in \realnumbers^{n+1}
            \suchthat ||x|| = 1
        \}
    \]
\end{snippetdefinition}

\begin{snippettheorem}{sphere-disk-relation-theorem}{}
    \[ % TODOURGENT homeomorphic
        \boundary[\realnumbers^n][D^n] \cong S^{n-1}
    \]
\end{snippettheorem}

\includesnpt{sphere-disk-illustration}

\section{Gluing rectangles}

\begin{snippetdefinition}{cylinder-definition}{Cylinder}
    Let \(I = [0,1]\) be the closed unit interval.
    Define the equivalence relation \(\sim\) on \(I\) generated
    by \((0,t) \sim (1,t)\).
    The \emph{cylinder} is defined as \(C = (I \times I)/\sim\).
\end{snippetdefinition}

\includesnpt{cylinder-illustration-gluing}

\begin{snippetdefinition}{torus-surface-definition}{Torus surface}
    Let \(I = [0,1]\) be the closed unit interval.
    Define the equivalence relation \(\sim\) on \(I\) generated
    by \((x,0) \sim (x,1)\) and \((0,y) \sim (1,y)\).
    The \emph{surface of the torus} is defined as \(T = (I \times I)/\sim\).
\end{snippetdefinition}

\includesnpt{torus-surface-illustration-gluing}

\begin{snippetdefinition}{mobius-strip-definition}{Möbius strip}
    Let \(I = [0,1]\) be the closed unit interval.
    Define the equivalence relation \(\sim\) on \(I\) generated
    by \((0,t) \sim (1,1-t)\).
    The \emph{Möbius strip} is defined as \(M = (I \times I)/\sim\).
\end{snippetdefinition}

\includesnpt{mobius-strip-illustration-gluing}

\begin{snippetdefinition}{klein-bottle-definition}{Klein's bottle}
    Let \(I = [0,1]\) be the closed unit interval.
    Define the equivalence relation \(\sim\) on \(I\) generated
    by \((x,0) \sim (x,1)\) and \((0,y) \sim (1,1-y)\).
    The \emph{Klein's bottle} is defined as \(K = (I \times I)/\sim\).
\end{snippetdefinition}

\plain{We first glue two sides and make a cylinder, but then because of the inverted orientation, the cylinder must pass through itself to be glued to the other side.}

\includesnpt{klein-bottle-illustration-gluing}

\section{Higher genus surfaces}

\begin{snippet}{genus-two-torus-expl}
    We can analogously create a torus with two holes.
    Geometrically, this operation is performed by removing a small
    open disk from two separate tori and gluing them together along
    the resulting circular boundaries.
    This joining process also transforms the fundamental polygon.
    Since a single torus is constructed from a 4-sided polygon
    (a square) with the boundary word $aba^{-1}b^{-1}$,
    joining two tori effectively concatenates their identification schemes.
    Consequently, the double torus is identified from an \textbf{octagon}
    (8 sides). The edges are glued in the sequence:
    \[
        a_1 b_1 a_1^{-1} b_1^{-1} a_2 b_2 a_2^{-1} b_2^{-1}
    \]
    where the first four edges form the first handle and the subsequent
    four form the second.
\end{snippet}

\includesnpt{genus-two-torus-illustration}

\begin{snippet}{multiple-genus-torus-expl}
    In general, the construction for a two-holed torus holds for any orientable
    surface of genus $g$.
    To construct a surface with $g$ holes, we employ a
    fundamental polygon with \textbf{$4g$ sides}.
    The standard canonical identification follows the product of $g$ commutators:
    \[
        \prod_{i=1}^{g} a_i b_i a_i^{-1} b_i^{-1} = a_1 b_1 a_1^{-1} b_1^{-1} \dots a_g b_g a_g^{-1} b_g^{-1}.
    \]
\end{snippet}

\end{document}
\documentclass[preview]{standalone}

\usepackage{amsmath}
\usepackage{amssymb}
\usepackage{stellar}
\usepackage{definitions}

\begin{document}

\id{submonoids}
\genpage

\section{Introduction}

\begin{snippet}{reducing-set-of-binary-operation}
    When given an \algebraicstructure, it is possible to reduce
    the size of the \set and maintain the same \binoperation.
    If the operation is associative or commutative, those properties
    will be preserved. However, an identity element is not necessarily preserved.
    Indeed, it could happen that a new identity element arises, which is different than the previous one.
\end{snippet}

% TODOURGENT
%\sexample{}{
%    Considera il monoide delle matrici quadrate di ordine \(2\)
%    a coefficienti interi rispetto alla moltiplicazione usuale.
%    L'elemento neutro è la matrice identica.
%    Consideriamo il sottoinsieme
%    \[
%        S = \left\{
%            \begin{bmatrix}
%                a & 0 \\
%                0 & 0
%            \end{bmatrix}
%        \right\}
%    \]
%    con \(a \in \mathbb{Z}\).
%    Allora l'elemento neutro è
%    \[
%        \begin{bmatrix}
%            1 & 0 \\
%            0 & 0
%        \end{bmatrix}
%    \]
%    che non è la matrice identica.
%}

\section{Submonoids}

\begin{snippetdefinition}{submonoid-definition}{Submonoid}
    A \textit{submonoid} of a \monoid \((S, \circ)\)
    is a \monoid \((T, \circ)\) such that \(T \subseteq S\)
    where the identity element is the same, written
    \[
        (S, \circ) \origsubseteq (T, \circ)
    \]
\end{snippetdefinition}

\plain{It suffices for the operation to be closed under the operation and for the identity element to be preserved.}

\begin{snippetproposition}{submonoid-commutativity-preserved}{Submonoid commutativity}
    Let \((M, \circ)\) be a commutative \monoid
    and \((S, \circ)\) be a \submonoid of \((M, \circ)\).
    Then, \((M, \circ)\) is commutative.
\end{snippetproposition}

\begin{snippetproposition}{submonoid-inverse}{Submonoid inverse}
    Let \((M, \circ)\) be a \monoid
    and \((S, \circ)\) be a \submonoid of \((M, \circ)\).
    Then, if \(x\in S\) is invertible in \((S, \circ)\),
    then it is invertible in \((M, \circ)\).
\end{snippetproposition}

\begin{snippetproposition}{trivial-submonoids}{Trivial submonoids}
    Given a \monoid \((M, \circ)\):
    \begin{enumerate}
        \item \((\{1\}, \circ) \submonoidsubseteq (M, \circ)\);
        \item \((M, \circ) \submonoidsubseteq (M, \circ)\);
    \end{enumerate}
\end{snippetproposition}

\end{document}
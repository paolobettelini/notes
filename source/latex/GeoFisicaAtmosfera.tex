\documentclass[preview]{standalone}

\usepackage{amsmath}
\usepackage{amssymb}
\usepackage{tikz}
\usepackage{stellar}
\usepackage{bettelini}

\hypersetup{
    colorlinks=true,
    linkcolor=black,
    urlcolor=blue,
    pdftitle={Assets},
    pdfpagemode=FullScreen,
}

\begin{document}

\title{Stellar}
\id{geofisica-atmosfera}
\genpage

\section{Atmosfera}

\begin{snippetdefinition}{autora-polare}{Aurora polare}
    L'\textit{aurora polare} è un fenomeno
    caratterizzato visivamente da bande luminose che assumono un'ampia gamma di forme e colori, rapidamente mutevoli nel tempo e nello spazio,
    causato dall'interazione di particelle cariche con la ionosfera (contenente le radiazioni del Sole).
\end{snippetdefinition}

\includesnpt[width=90\%|src=/snippet/static/aurora.avif]{centered-img}

\plain{
    L'aurora boreale si vede solo di notte perché è buio.
Viene denominata <b>aurora boreale</b> qualora si verifichi nell'emisfero nord (boreale),
mentre il nome <b>aurora australe</b> è riferito all'analogo dell'emisfero sud (australe). 
}
\plain{Il criterio per definire lo strato dell'atmosfera è principalmente la temperatura.}

\begin{snippetdefinition}{nebbia}{Nebbia}
    La \textit{nebbia} è un fenomeno dato dalla saturazione di acqua nell'aria,
    come le nuvole, il valore acque satura l'aria e crea un deposito visibile di goccioline.
\end{snippetdefinition}

\plain{Quando l'aria è satura di vapore acqueo essa si deposita su oggetti oppure piccole particelle nell'aria.}

\plain{
    La quantità di vapore acqueo, in grammi, contenuta in un metro cubo d'aria di chiama <b>umidità assoluta</b>.
L'<b>umidità relativa</b> è il rapporto fra quella assoluta e il valore massimo che può raggiungere.
}

\plain{
    La condensa che si forma quando apro una finestra è data dal fatto che la temperatura si abbassa, e quindi
il limite di saturazione si abbassa e l'aria diventa satura.
}

\begin{snippet}{salita-aria-motivi}
    L'aria può salire verso l'alto per tre motivi:
    \begin{enumerate}
        \item perché è calda e umida, e quindi leggera;
        \item perché incontra una montagna che la ostacola;
        \item perché si scontra con una massa d'aria più fredda ed è costretta a salirvi sopra;
        \item perché a terra, si scontra con un'altra massa d'aria ed entrambe vanno verso l'alto.
    \end{enumerate}
\end{snippet}

\plain{Il vento è dato dalle differenze di pressione (come l'acqua nei bicchieri)
genera spostamenti di masse di aria.
Questa differenza di pressione è data dalla differenza di temperatura e vapore dell'aria,
la quale sale o scende creando delle correnti.
Quando l'aria si muove verso l'alto, la pressione a terra diminuisce,
mentre quando l'aria fredda scende verso il basso, la pressione a terra aumenta
(in altitudine si avrebbe l'opposto).}

\plain{All'equatore (dove è più caldo), l'aria sale maggiormente, causando una pressione minore.}

\plain{Il vento non scorre tuttavia dai poli all'equatore a bassa quota e dall'equatore ai poli in alta quota,
per via della forza di Coriolis.}

\end{document}
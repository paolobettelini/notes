\documentclass[preview]{standalone}

\usepackage{amsmath}
\usepackage{amssymb}
\usepackage{stellar}
\usepackage{definitions}

\begin{document}

\id{linearalgebra-exercises-batch-3}
\genpage

\section{Exercises}

\begin{snippetexercise}{linear-algebra-batch-3-ex-1}{}
    \todo
\end{snippetexercise}

\begin{snippetsolution}{linear-algebra-batch-3-ex-1-proof}{}
    \todo
\end{snippetsolution}

\begin{snippetexercise}{linear-algebra-batch-3-ex-2}{}
    \todo
\end{snippetexercise}

\begin{snippetsolution}{linear-algebra-batch-3-ex-2-proof}{}
    \todo
\end{snippetsolution}

\begin{snippetexercise}{linear-algebra-batch-3-ex-3}{}
    \todo
\end{snippetexercise}

\begin{snippetsolution}{linear-algebra-batch-3-ex-3-proof}{}
    \todo
\end{snippetsolution}

\begin{snippetexercise}{linear-algebra-batch-3-ex-4}{}
    Let \(A \in \matrices_{m\times n}(\mathbb{R})\) and \(b\in \mathbb{F}^m\),
    study when
    \[
        \text{Sol}(Ax = b) = \{
            x \in \mathbb{F}^n \suchthat Ax = b
        \}
    \]
    is a linear subspace of \(\mathbb{F}^n\).
\end{snippetexercise}

\begin{snippetsolution}{linear-algebra-batch-3-ex-4-proof}{}
    We need
    \[
        0 \in \text{Sol}(Ax = b)
    \]
    but this can only happen if \(b = 0\). So the \set
    is a linear subspace of \(\mathbb{F}^n\) \ifandonlyif
    \(b=0\).
\end{snippetsolution}

\begin{snippetexercise}{linear-algebra-batch-3-ex-5}{}
    \todo
\end{snippetexercise}

\begin{snippetsolution}{linear-algebra-batch-3-ex-5-proof}{}
    \todo
\end{snippetsolution}

\begin{snippetexercise}{linear-algebra-batch-3-ex-6}{}
    \todo
\end{snippetexercise}

\begin{snippetsolution}{linear-algebra-batch-3-ex-6-proof}{}
    \todo
\end{snippetsolution}

\begin{snippetexercise}{linear-algebra-batch-3-ex-7}{}
    \todo
\end{snippetexercise}

\begin{snippetsolution}{linear-algebra-batch-3-ex-7-proof}{}
    \todo
\end{snippetsolution}

\begin{snippetexercise}{linear-algebra-batch-3-ex-8}{}
    \todo
\end{snippetexercise}

\begin{snippetsolution}{linear-algebra-batch-3-ex-8-proof}{}
    \todo
\end{snippetsolution}

\end{document}
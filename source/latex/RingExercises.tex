\documentclass[preview]{standalone}

\usepackage{amsmath}
\usepackage{amssymb}
\usepackage{stellar}
\usepackage{definitions}

\begin{document}

\id{rings-exercises}
\genpage

\section{Exercises}

\begin{snippetexercise}{rings-ex1}{}
    Let \(A\) be a \ring such that \(\forall a \in A, a^2 = a\) (boolean ring).
    Show that:
    \begin{enumerate}
        \item \(A\) is commutative;
        \item \(\forall a \in A, a+a = 0_A\);
    \end{enumerate}
\end{snippetexercise}

\begin{snippetsolution}{rings-ex1-sol}{}
    \begin{enumerate}
        \item We need to show that \(ab - ba = 0_A\)
        for all \(a,b \in A\). We have
        \begin{align*}
            a+b &= (a+b)(a+b) =a + b + 2b + b^2 \\
            ab + ba &= 0_A \\
            &\implies ab = -ba
        \end{align*}
        But \(-ba = ab\) by the next proposition.
        \item We have
        \begin{align*}
            (a+1)(a+1) &= a+1 \\
            a\cdot a + 2a + 1 &= a + 1 \\
            2a &= 0
        \end{align*}
    \end{enumerate}
\end{snippetsolution}

\begin{snippetexercise}{rings-ex2}{}
    Let \(U\) be a \set.
    Given \(A,B \subseteq \powerset(U)\) define
    \begin{itemize}
        \item \[
            A + B \triangleq (A \difference B) \union (B \difference A)
        \]
        \item \[
            A \cdot B \triangleq A \intersection B
        \]
    \end{itemize}
    Show that \(S = (\powerset(U), +, \cdot)\) is a commutative \ring.
\end{snippetexercise}

\begin{snippetsolution}{rings-ex2-sol}{}
    \begin{enumerate}
        \item \(0_S = \emptyset\) as
        \begin{enumerate}
            \item \(A \cdot \emptyset = A \intersection \emptyset = \emptyset\);
            \item \[A + \emptyset = (A \difference \emptyset) \union (\emptyset \difference A) = A \union \emptyset = A \]
        \end{enumerate}
        \item \(1_S = U\) as \(A \cdot U = A \intersection U = A\);
        \item the product is associative since the intersection is associative;
        \item \((\powerset(U), +)\) is an \abeliangroup:
        \begin{enumerate}
            \item it is clearly abelian by the symmetric of the addition operation;
            \item \(-A = A\) as
            \[
                A + A = (A \difference A) \union (A \difference A) = \emptyset = 0_S
            \]
            \item the addition is clearly closed;
            \item the associativity is given by the symmetry of the diagram.
        \end{enumerate}
        \item the commutativity is given by the commutativity of the intersection.
        \item \emph{distributivity:}
        \begin{align*}
            (A + B) \cdot C &= AC + BC = A \intersection C + B \intersection C
            = ((A \intersection C) \difference (B \intersection C)) \union
            ((B \intersection C) \difference (A \intersection C)) \\
            &= ((A \difference B) \union (B \difference A)) \intersection C
        \end{align*}
        which can be verified with the two diagrams.
    \end{enumerate}
\end{snippetsolution}


\begin{snippetexercise}{rings-ex3}{}
    Consider the \set
    \[
        A = \left\{
            \begin{pmatrix}
                z & w \\
                -w^* & z^*
            \end{pmatrix}
            \suchthat
            z,w \in \complexnumbers
        \right\} \subseteq \matrices_2(\complexnumbers)
    \]
    Show that \(A \cong \mathbb{H}\).
\end{snippetexercise}

\begin{snippetsolution}{rings-ex3-sol}{}
    Consider the basis
    \[
        \left\{
            \begin{pmatrix} 1 & 0 \\ 0 & 1 \end{pmatrix},
            \begin{pmatrix} 0 & 1 \\ -1 & 0 \end{pmatrix},
            \begin{pmatrix} i & 0 \\ 0 & -i \end{pmatrix},
            \begin{pmatrix} 0 & i \\ i & 0 \end{pmatrix}
        \right\}
    \]
\end{snippetsolution}

\end{document}
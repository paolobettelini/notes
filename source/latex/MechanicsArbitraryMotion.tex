\documentclass[preview]{standalone}

\usepackage{amsmath}
\usepackage{amssymb}
\usepackage{stellar}
\usepackage{definitions}

\begin{document}

\id{arbitrary-particle-motion}
\genpage

\section{Arbitrary motion}

\begin{snippettheorem}{arbitrary-particle-motion-theorem}{Arbitrary particle motion}
    Let \(\vec{r}(t)\) be an arbitrary vector representing the motion of a particle.
    Then, its acceleration has tangential component and a normal component (\emph{centripetal} component)
    \[
        \vec{a}(t) = \frac{dv(t)}{dt}\vec{T} + \frac{v^2}{R}\vec{N}
    \]
    where \(v\) is the velocity and \(R\) is the \emph{radius of curvature}.
\end{snippettheorem}

\begin{snippetproof}{arbitrary-particle-motion-theorem-proof}{arbitrary-particle-motion-theorem}{Arbitrary particle motion}
    The velocity is given by \(\vec{v}(t) = \frac{d\vec{r}(t)}{dt}\) which is a vector pointing in the direction of the trajectory.
    The tangent vector is defined such that
    \[
        \hat{T}(t) v(t) = \vec{v}(t)
    \]
    And thus we have
    \[
        \vec{a}(t) = \frac{d}{dt} \left(v(t)\hat{T}(t)\right) = \frac{dv(t)}{dt}\hat{T}(t) + v(t)\frac{d\hat{T}(t)}{dt} = a_t(t)\hat{T}(t) + v(t)\frac{d\hat{T}(t)}{dt}
    \]
    In order to study the meaning of such terms, we start by the identity \(\hat{T}(t) \cdot \hat{T}(t) = 1\).
    Then,
    \begin{align*}
        \frac{d}{dt} \left( \hat{T}(t) \cdot \hat{T}(t) \right) &= 0 \\
        \frac{d\hat{T}(t)}{dt} \cdot \hat{T}(t) + \hat{T}(t)\frac{d\hat{T}(t)}{dt} &= 0 \\
        \hat{T}(t)\frac{d\hat{T}(t)}{dt} &= 0 
    \end{align*}
    By the differential analysis we find
    \[
        \left|\frac{\hat{T}(t)}{dt}\right| = \lim_{\Delta t \to 0} \frac{dl}{\Delta t}
    \]
    and the circular arc
    \[
        s = Rd\theta
    \]
    where \(R\) is the length of the line to the rotation point (radius of curvature, meaning the radius of the osculating circle).
    By putting together these two information we find
    \[
        \left|\frac{d\hat{T}(t)}{dt}\right| = \lim_{\Delta t \to 0} \left[\frac{S}{\Delta t} \frac{1}{R}\right] = \frac{v}{R}
    \]
    and now we can write
    \[
        \vec{a}(t) = \frac{dv}{dt}\hat{T} + v\frac{d\hat{T}}{dt} = \frac{dv}{dt}\hat{T} + \frac{v^2}{R} \hat{N}
    \]
\end{snippetproof}

\plain{Note that the centripetal acceleration is smaller the bigger the circle is, and thus null when the particle is
moving on a straight line.}

\subsection{Circular motion}

\begin{snippetproposition}{circular-motion}{Circular motion}
    A circular motion only has a centripetal acceleration
    \[
        \vec{a} = -\omega^2\vec{r} = \frac{v^2}{R} \hat{N}
    \]
    where \(\omega = \frac{v}{R}\).
\end{snippetproposition}

\end{document}
\documentclass[preview]{standalone}

\usepackage{amsmath}
\usepackage{amssymb}
\usepackage{stellar}
\usepackage{bettelini}

\hypersetup{
    colorlinks=true,
    linkcolor=black,
    urlcolor=blue,
    pdftitle={Stellar},
    pdfpagemode=FullScreen,
}

\begin{document}

\title{Stellar}
\id{italiano-definizioni}
\genpage

\section{Definizioni}

\begin{snippetdefinition}{locus-amoenus-definizione}{Locus amoenus}
    Un \textit{locus amoenus} è un termine usato in letteratura
    che fa riferimento a un luogo idealizzato e piacevole.
    È un posto immerso tra piante ed alberi,
    spesso situato nelle vicinanze di una fonte o di un ruscello,
    ricco di ombra ed in qualche modo simile al Paradiso terrestre. 
\end{snippetdefinition}

\section{Figure retoriche}

\begin{snippetdefinition}{allitterazione-definition}{Allitterazione}
    L'allitterazione consiste nella ripetizione di un suono
    (o di una serie di suoni simili) all'inizio di due o più parole.
\end{snippetdefinition}

\begin{snippetexample}{allitterazione-example}{Allitterazione}
    \quotes{E \textbf{c}addi \textbf{c}ome \textbf{c}orpo morto \textbf{c}adde} (Dante)
\end{snippetexample}

\begin{snippetdefinition}{onomatopea-definition}{Onomatopea}
    L'onomatopea è una delle figure retoriche più note e riproduce il suono di qualcosa o il verso di un animale.
\end{snippetdefinition}

\begin{snippetdefinition}{chiasmo-definition}{Chiasmo}
    Il chiasmo è la disposizione incrociata di due parole o di due gruppi di parole secondo lo schema AB - BA.
\end{snippetdefinition}

\begin{snippetexample}{chiasmo-example}{Chiasmo}
    \quotes{Odi greggi \textbf{belar} / \textbf{muggire} armenti} (Leopardi)
\end{snippetexample}

% anafora

% https://it.babbel.com/it/magazine/elenco-figure-retoriche


\end{document}
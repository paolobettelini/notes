\documentclass[preview]{standalone}

\usepackage{amsmath}
\usepackage{amssymb}
\usepackage{stellar}
\usepackage{definitions}

\begin{document}

\id{mechanics-central-forces}
\genpage

\section{Central forces}

\begin{snippetdefinition}{central-force-definition}{Central force}
    \emph{Central forces} are positional forces whose module
    \(\vec{F}(\vec{r}) = f(r)\) only depends by the distance,
    and are directed like \(\vec{r}\) (radial direction).
\end{snippetdefinition}

\begin{snippettheorem}{central-forces-are-conservative-theorem}{Central forces are conservative}
    Every central force is conservative.
\end{snippettheorem}

\begin{snippetproof}{central-forces-are-conservative-theorem-proof}{central-forces-are-conservative-theorem}{Central forces are conservative}
    Consider \(\vec{F}(\vec{r}) = -\gradient U\).
    We want to find
    \[
        \frac{\partial U}{\partial x}
    \]
    the component \(x\) is embedded in \(\vec{r}\).
    Thus, \(r=\sqrt{x^2 + y^2 + z^2}\) and
    \begin{align*}
        \frac{\partial U}{\partial x} &= \frac{dU}{dr} \frac{\partial r}{\partial x} \\
        &= \frac{dU}{dr} \frac{x}{r}
    \end{align*}
    The other variables are analogous.
    The gradient of \(U\) is thus given by
    \begin{align*}
        \gradient U(r) &=
        \frac{dU}{dr} \frac{x}{r} \hat{x} +
        \frac{dU}{dr} \frac{y}{r} \hat{y} +
        \frac{dU}{dr} \frac{z}{r} \hat{z} \\
        &= \frac{dU}{dt} \hat{r}
    \end{align*}
    The central force is thus a function
    \[
        f(r) = - \frac{dU}{dr}
    \]
    and thus
    \[
        \vec{F} = f(r) \hat{r} = -\gradient U
    \]
\end{snippetproof}

\end{document}
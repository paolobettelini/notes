\documentclass[preview]{standalone}

\usepackage{amsmath}
\usepackage{amssymb}
\usepackage{stellar}
\usepackage{definitions}
\usepackage{boldline}

\begin{document}

\id{integers-euler-totient-function}
\genpage

\section{Euler's totient function}

\plain{We are now interested in counting how many congruence classes are invertible.}

\begin{snippetdefinition}{euler-totient-function-definition}{Euler's totient function}
    The \textit{Euler's totient function} counts the positive integers up
    to a given integer \(n\) that are \coprime to \(n\).
    \[
        \varphi(n)
    \]
\end{snippetdefinition}

\begin{snippet}{euler-totient-function-table-transposed}
    \begin{center}
        \bgroup{}
        \def\arraystretch{1.25}
        \begin{tabular}{|c| c| c| c| c| c| c| c| c|}
            \hline
            \(n\) & 1 & 2 & 3 & 4 & 5 & 6 & 7 & 8 \\
            \hline
            \(\eulertotient(n)\) & 1 & 1 & 2 & 2 & 4 & 2 & 6 & 4 \\
            \hline
        \end{tabular}
        \egroup{}
    \end{center}
    \phantom{}
\end{snippet}

\begin{snippetcorollary}{euler-totient-of-prime}{Euler's totient function of prime}
    Let \(p\) be a \primen. Then,
    \[ \eulertotient(n) = n-1 \]
\end{snippetcorollary}

\begin{snippetproposition}{amount-of-invertible-classes}{Amount of invertible modulo classes}
    The amount of \invertiblecongclass[invertible] classes in \(\integers / n\) is \(\eulertotient(n)\).
\end{snippetproposition}

\begin{snippetproof}{amount-of-invertible-classes-proof}{amount-of-invertible-classes}{Amount of invertible modulo classes}
    The distinct classes in \(\integers / n\) are
    \[
        {[0]}_n, {[1]}_n, \cdots, {[n-1]}_n
    \]
    Since \({[a]}_n\) is \invertiblecongclass[invertible] \ifandonlyif \(\gcd(a,n)=1\),
    the amount of invertible classes is given by \(\eulertotient(n)\).
\end{snippetproof}

\section{Theorems}

\begin{snippettheorem}{euler-theorem}{Euler's theorem}
    Let \(a,n\in\integers\) such that \(a\) and \(n\) are \coprime.
    Then,
    \[
        a^{\eulertotient(n)} \equiv 1 \pmod{n}
    \]
\end{snippettheorem}

\begin{snippetproof}{euler-theorem-proof}{euler-theorem}{Euler's theorem}
    Consider the \invertiblecongclass[invertible] classes in \(\integers / n\)
    \[ {[b_1]}_n, {[b_2]}_n, \cdots, {[b_{\eulertotient(n)}]}_n \]
    Now, the classes \[
        {[a]}_n{[b_1]}_n, {[a]}_n{[b_2]}_n, \cdots, {[a]}_n{[b_{\eulertotient(n)}]}_n 
    \]
    Since \({[a]}_n\) and \({[b_k]}_n\) are \invertiblecongclass[invertible], then
    \({[a]}_n{[b_k]}_n\) is also \invertiblecongclass[invertible]. They are also distinct
    by the cancellation law. Thus, the former and latter list coincide up to ordering.
    Given this information, we consider the product of said classes:
    \begin{align*}
        \prod_{k=1}^{\eulertotient(n)} {[b_k]}_n &= \prod_{k=1}^{\eulertotient(n)} {[a]}_n{[b_k]}_n \\
        &= \left( \prod_{k=1}^{\eulertotient(n)} {[b_k]}_n \right)
        \left( \prod_{k=1}^{\eulertotient(n)} {[a]}_n \right)
    \end{align*}
    Since the classes \({[b_k]}_n\) are \invertiblecongclass[invertible],
    we can simplify them
    \begin{align*}
        {[1]}_n &= \left( \prod_{k=1}^{\eulertotient(n)} {[a]}_n \right) \\
        &= {[a^{\eulertotient(n)}]_n}
    \end{align*}
\end{snippetproof}

\begin{snippettheorem}{fermat-little-theorem}{Fermat's little theorem}
    Let \(a,p\in\integers\) such that \(p\) is \primen. Then,
    \[ a^p \equiv a \pmod{p} \]
\end{snippettheorem}

\begin{snippetproof}{fermat-little-theorem-proof}{fermat-little-theorem}{Fermat's little theorem}
    If \(a\) and \(p\) are \coprime, by \eulertheorem we have
    \[
        a^{\eulertotient(p)} \equiv 1 \pmod{p}
    \]
    However, \(\eulertotient(p) = p-1\) since \(p\) is \primen, and thus
    \[
        a^{p-1} \equiv 1 \pmod{p}
    \]
    from which we get
    \[
        a^p \equiv p \pmod{p}
    \]
\end{snippetproof}

\begin{snippet}{fermat-little-theorem-but-not-coprime}
    If \(a\) is not \coprime with \(p\), we have that \(p\divides a\). That is,
    \(a \equiv 0 \pmod{p}\), and, thus, every power with positive exponent of \(a\) is
    congruent to \(0 \pmod{p}\). In particular, \(a^p \equiv 0 \pmod{p}\).
\end{snippet}

\end{document}
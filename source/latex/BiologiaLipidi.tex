,
    pdfpagemode=FullScreen,
}
    I  sono una classe di composti idrofobi (idrorepellenti)
    costituiti prevalentemente da atomi di carbonio e idrogeno.
    Il  costituisce una riserva energetica della cellula (comunemente grasso).
    Il monogliceride è composto da un glicerolo, attaccato (per condensazione) ad un acido grasso.
    Il trigliceride è attaccato a 3 catene di acido grasso.
    Le catene di acidi grassi possono essere dritti (saturi) oppure piegate (insaturi)
    in quanto posseggono un doppio legame.
    I grassi saturi sono generalmente solidi a temperatura ambiente (es. burro),
    mentre quelli insaturi sono liquidi (es. olio)
[width=50
    Il  sono composti da una testa idrofila e da una code idrofoba.
    Lo  è una molecola con una struttura di 4 anelli.
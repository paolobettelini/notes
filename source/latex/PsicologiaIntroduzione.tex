\documentclass[preview]{standalone}

\usepackage{amsmath}
\usepackage{amssymb}
\usepackage{stellar}
\usepackage{definitions}

\begin{document}

\id{psicologia-introduzione}
\genpage

\section{Cos'è la psicologia}

\begin{snippetdefinition}{psicologia-definition}{Psicologia}
    La \emph{psicologia} è lo studio scientifico del comportamento degli individui e dei loro processi
    mentali, per questo motivo si fonda su un metodo scientifico per cui è possibile giungere a
    conclusioni tramite la raccolta di prove. In quanto scienza della salute, la psicologia cerca di
    migliorare la qualità di vita di ogni individuo.
\end{snippetdefinition}

\begin{snippetdefinition}{psicologia-comportamento-definition}{Comportamento}
    Il comportamento è l'insieme delle azioni attraverso cui gli organismi rispondono agli stimoli ed
    interagiscono con l'ambiente. Gli psicologi esaminano cosa fa l'individuo e come esso si comporta
    nel farlo, considerando ciascun fenomeno comportamentale calato nell'ambiente sociale e culturale
    in cui è prodotto.
\end{snippetdefinition}

\begin{snippet}{psicologia-comportamento-expl}
    Nella maggior parte dei casi l'oggetto dell'analisi psicologica sono gli esseri umani (ad esempio un
    neonato che fa le prime esperienze, un atleta alla ricerca di motivazione). Non è però possibile
    capire le azioni degli esseri umani senza capirne anche i \emph{processi mentali}, i meccanismi della
    mente. Si tratta dell'aspetto più importante dell'indagine psicologica per cui si sono sviluppate
    tecniche per studiare le manifestazioni ed i processi della mente.
\end{snippet}

\begin{snippetexample}{psicologia-comportamento-example}{Comportamento}
    Facciamo un esempio, ci poniamo di indagare cosa influenza la rapidità di lettura, per cui andremo
    a chiedere ai partecipanti dell'osservazione di leggere delle parole; in questo caso la
    \underline{rapidità di lettura} consiste in un \snippetref[psicologia-comportamento-definition][comportamento]
    che vogliamo analizzare, mentre l'azione di leggere le parole
    è il \emph{compito/strumento} attraverso cui verrà condotta l'analisi.
    In questo caso, variabili come
    \begin{itemize}
        \item \emph{lunghezza delle parole:} \texttt{VIA} vs \texttt{RAGIONE},
        \item \emph{frequenza delle parole:} \texttt{TOPO} vs \texttt{CUPO}
    \end{itemize}
    sono potenzialmente informative dei \emph{processi mentali} che guidano il comportamento.
    Parole più frequenti sono meglio discriminabili dalla nostra mente (presenza di un sistema memoria,
    sensibilità al numero di volte in cui una parola compare), mentre parole più lunghe sono associate a
    tempi di lettura più lunghi.
\end{snippetexample}

\begin{snippet}{psicologia-scienze-biologiche-expl}
    Gli psicologi condividono con i ricercatori in scienze biologiche l'interesse per i \emph{processi cerebrali}
    e le \emph{basi biochimiche} del comportamento. L'informatica, la filosofia, la linguistica e le
    neuroscienze offrono, invece, punti di incontro fondamentali per quanto riguarda la ricerca nelle
    scienze cognitive.
\end{snippet}

\begin{snippet}{psicologia-metodo-scientifico-galileo-galilei}
    Le tematiche psicologiche sono al centro dell'interesse dell'uomo sin dall'antichità; già Platone ed
    Aristotele si interessarono a fenomeni di cui oggi si occupa la psicologia: memoria, apprendimento,
    sogni, comportamenti inusuali, ecc (Platone ha teorizzato la netta separazione tra anima e corpo).
    Galileo Galilei è considerato un riferimento
    fondamentale per questa disciplina, indicando le
    linee guida del metodo scientifico che è per lui un
    intreccio di ``sensate esperienze'' e di ``certe
    dimostrazioni''.
    A distanza di circa due secoli dagli esperimenti di Galileo, si avverte l'esigenza di utilizzare il
    metodo sperimentale anche nella psicologia; tale nuovo interesse fu alla base della fondazione dei
    \emph{primi laboratori di psicologia sperimentale}, a partire da quello di \emph{Wundt} fondato a Lipsia nel
    1879.
\end{snippet}

\end{document}
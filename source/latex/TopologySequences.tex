\documentclass[preview]{standalone}

\usepackage{amsmath}
\usepackage{amssymb}
\usepackage{stellar}
\usepackage{definitions}
\usepackage{bettelini}
\usepackage{tikz}

\usetikzlibrary{cd}

\newcommand\ts{{(X, \mathcal{T})}}

\begin{document}

\id{topology-hausdorff-spaces}
\genpage

\section{Hausdorff spaces}

\begin{snippetdefinition}{hausdorff-space-definition}{Hausdorff space}
    A \topologicalspace \(\ts\) is an \textit{Hausdorff space} if for all \(x,y\in X\)
    where \(x\neq y\), there is a \neighborhood of \(x\), \(U_x\) and a \neighborhood of \(y\), \(U_y\)
    such that \(U_x\) and \(U_y\) are \disjoint.
\end{snippetdefinition}

\begin{snippetproposition}{metric-spaces-are-hausdorf}{}
    Let \((X, d)\) be a \metricspace and \(\mathcal{T}_d\) the topology induced by \(d\). Then,
    \((X, d)\) is Hausdorf.
\end{snippetproposition}

\begin{snippetproof}{metric-spaces-are-hausdorf-proof}{metric-spaces-are-hausdorf}{}
    Let \(x,y \in X\) such that \(x \neq y\).
    Thus, \(d(x,y) > 0\). Consider \(r\) such that
    \[
        0 < r < \frac{d(x,y)}{2}
    \]
    We want to show that \(\ball_r(x) \intersection B_r(y) = \emptyset\).
    Suppose that it is not and let \(z \leq \ball_r(x) \intersection B_r(y)\).
    We have that \(d(x,z) < r\)
    and \(d(y,z) < r\).
    Using the triangular inequality
    \[
        d(x,y) \leq d(x, z) + d(z, y) < 2r < d(x,y)
    \]
    which is absurd \lightning.
\end{snippetproof}

\begin{snippetproposition}{topology-t2-implies-t1}{}
    Let \(\ts\) be a \topologicalspace.
    If \(\ts\) is \(T_2\), it is also \(T_1\).
\end{snippetproposition}

\begin{snippetproof}{topology-t2-implies-t1-proof}{topology-t2-implies-t1}{}
    Let \(x \in X\).
    We want to show that \(\{x\}\) is \closedset[closed], meaning that
    \(X \difference \{x\}\) is \topologicalspace[open][Open set].
    This is equivalent to saying that \(\forall y \in X \difference \{x\}\),
    there exists a \neighborhood \(V\) of \(y\) such that \(V \subseteq X \difference \{x\}\).
    Since the space is \(T_2\), there exist \neighborhood[neighborhoods]
    \(U\) and \(V\) of \(y\) which are \disjoint.
    This is equivalent to saying that \(V \subseteq X \difference U\),
    which implies that \(V \subseteq X \difference U \subseteq X \difference \{x\}\).
\end{snippetproof}

\begin{snippetproposition}{hausdorff-spaces-derivatives-are-hausdorff}{}
    Let \(\ts\) be a \(T_2\) \topologicalspace. Then,
    \begin{enumerate}
        \item every subspace of \(\ts\) is also Hausdorff;
        \item let \(Y\) be another \(T_2\) \topologicalspace.
        Then, \(X \cartesianprod Y\) is also Hausdorff.
    \end{enumerate}
\end{snippetproposition}

\begin{snippetproof}{hausdorff-spaces-derivatives-are-hausdorff-proof}{hausdorff-spaces-derivatives-are-hausdorff}{}
    \begin{enumerate}
        \item There exist \neighborhood[neighborhoods] \(U\) and \(V\)
        of \(X\) such that \(x\in U\), \(y \in V\) and \(U \intersection V = \emptyset\).
        We thus have
        \[
            (U \intersection Y) \intersection (V \intersection Y)
            = (U \intersection V) \intersection Y = \emptyset \intersection Y = \emptyset.
        \]
        \item Let \((x,y), (x',y') \in X \cartesianprod Y\).
        The two points are distinct if \(x \neq x' \lor y \neq y'\).
        Without loss of generality choose let \(x \neq x'\).
        There exist \neighborhood[neighborhoods]
        \(U\) and \(U'\) of \(x\) and \(x'\) respectively that are \disjoint.
        Then,
        \[
            (U \cartesianprod Y) \intersection (U' \cartesianprod Y)
            = (U \intersection U') \cartesianprod Y = \emptyset \cartesianprod Y = \emptyset
        \]
    \end{enumerate}
\end{snippetproof}

\begin{snippetproposition}{hausdorff-space-diagonal-characterization}{}
    Let \(X\) be a \topologicalspace.
    Then, \(X\) is \(T_2\) \ifandonlyif
    \(\Delta \subseteq X \cartesianprod X\) is a \closedset in \(X \cartesianprod X\)
    where \(\Delta = \{(x,x) \suchthat x \in X\}\).
\end{snippetproposition}

\begin{snippetproof}{hausdorff-space-diagonal-characterization-proof}{hausdorff-space-diagonal-characterization}{}
    Saying that \(x\neq y\) is equivalent to saying that \(\forall (x,y) \in (X \cartesianprod X) \difference \Delta\)
    there exist a \neighborhood of \((x,y) \subseteq (X \cartesianprod X) \difference \Delta\).
    \(\forall x \neq y\) there exist \topologicalspace[open sets][Open set] \(U\) and \(V\)
    of \(X\) such that \((x,y) \in U \cartesianprod V\) and \(U \cartesianprod V \subseteq (X \cartesianprod X) \difference \Delta\).
    The latter is equivalent to saying that \(U \intersection V = \emptyset\).
\end{snippetproof}

\begin{snippetproposition}{hausdorff-space-basis-characterization}{}
    Let \(\mathcal{B}\) be a \topologicalbasis
    for a \topologicalspace \(X\).
    Then, \(X\) is \(T_2\) \ifandonlyif \(\forall x \neq y, \exists B, B' \in \mathcal{B}\)
    such that \(x \in B \land y\in B' \land B\intersection B' = \emptyset\).
\end{snippetproposition}

\begin{snippetproof}{hausdorff-space-basis-characterization-proof}{hausdorff-space-basis-characterization}{}
    \ffiproof{
        This is trivial.
    }{
        Given \topologicalspace[open sets][Open set]
        \(U, V\) such that \(x \in U\) and \(y \in V\)
        and \(U \intersection V = \emptyset\).
        Since \(\mathcal{B}\) is a \topologicalbasis for \(X\),
        we can write
        \[
            U = \bigcup_{i\in I} B_i, \qquad
            V = \bigcup_{j \in J} B_j'
        \]
        for \(B_i, B_j' \in \mathcal{B}\).
        Since \(x \in U\), there exists \(i \in I\) such that \(x \in B_i\).
        Since \(y \in V\), there exists \(j \in J\) such that \(y \in B_j\).
        Thus, \(B_i \intersection B_j' \subseteq U \intersection V = \emptyset\).
    }
\end{snippetproof}

\begin{snippetcorollary}{topological-maps-hausdorff-equalizer-closed}{}
    Let \(X, Y\) be \topologicalspace[topologicalspaces]
    where \(Y\) is \(T_2\) and let \(f,g \colon X \fromto Y\).
    Then, \(\text{Eq}(f,g)\) is a \closedset in \(X\). % TODOURGENT link Eq
\end{snippetcorollary}

\begin{snippetproof}{topological-maps-hausdorff-equalizer-closed-proof}{topological-maps-hausdorff-equalizer-closed}{}
    Consider the diagram
    \begin{center}
        % https://tikzcd.yichuanshen.de/#N4Igdg9gJgpgziAXAbVABwnAlgFyxMJZARgBoAGAXVJADcBDAGwFcYkQANEAX1PU1z5CKMsWp0mrdgE0ABAB15AY3oAnHPCz0waVdFnSefEBmx4CRAEykxNBizaIQh3vzNCi5G+PtSnL8RgoAHN4IlAAMz0AWyQvEBwIJGsJB3Zgo0iYuJpEpABmO0lHEAiQGkZ6ACMYRgAFAXNhEFUsYIALHEzS7MQyBKTEfNceiFi+3MHLEaixnIGkfsYsMBKoejh2oPLUvxBFGAAPLDgcOABCUgV5Y55KbiA
        \begin{tikzcd}
        & X \arrow[rd, "g"] \arrow[ld, "f"'] \arrow[d, "{\exists!, \xi}", dashed] &   \\
        Y & Y \cartesianprod Y \arrow[l] \arrow[r]                                  & Y
        \end{tikzcd}
    \end{center}
    where \(\xi = (f,g)\).
    By the universal property of the product, there exists a unique continuous map \(\xi\)
    such that the diagram commutes.
    Since \(Y\) is \(T_2\), \(\Delta \subseteq Y \cartesianprod Y\) is \closedset[closed]
    in \(Y \cartesianprod Y\) where \(\Delta = \{(y,y) \suchthat y \in Y\}\).
    Since \(\xi\) is continuous,
    \[
        \xi^{-1}(\Delta) = \{x \in X \suchthat \xi(x) \in \Delta\}
    \]
    but this condition is equivalent to sayin that \((f(x), g(x)) \in \Delta\),
    which is the equalizer condition \(f=g\).
\end{snippetproof}

\plain{If the two spaces are equal, the solutions to the fixed point equation is closed.}

\begin{snippetlemma}{topology-hausdorff-finite-distinct-point-separation}{}
    Let \(X\) be a \(T_2\) \topologicalspace
    and let \(p_1, \cdots, p_n\) a finite sequence of distinct points.
    Then, there exist a finite sequence of \disjoint \topologicalspace[opens sets][Open set]
    \(U_1, \cdots, U_n\) such that \(p_i \in U_i\).
\end{snippetlemma}

\begin{snippetproof}{topology-hausdorff-finite-distinct-point-separation-proof}{topology-hausdorff-finite-distinct-point-separation}{}
    Since \(X\) is Hausdorf,
    \(\forall 1 \leq i < j \leq n\)
    there exists \disjoint \topologicalspace[opens sets][Open set] \(U_{i,j}\) and \(U_{j,i}\)
    such that \(p_i \in U_{i,j}\) and \(p_j \in U_{j,i}\).
    Let \[
        U_i = \bigcap_{i \neq j} U_{i,j}
    \]
    which is \topologicalspace[opens][Open set] as it is a finite intersection of \topologicalspace[opens sets][Open set].
    Thus, if \(i \neq j\), \(U_i \intersection U_j = \emptyset\).
\end{snippetproof}

\subsection{Sequences} %%%%%%%%%%%%%%%%%%%%%%

\plain{Defining convergence on topological spaces is a bit tricky because there is
no notion of distance (since there is no metric). We can define convergence as
the members of the sequence (eventually) all lie in each open neighborhood of the convergence point.}

\begin{snippetdefinition}{topology-convergence-definition}{Convergence in topological space}
    Let \(\ts\) be a \topologicalspace.
    A \sequence \({\{x_n\}}_{n \in \naturalnumbers}\) in \(X\) converges to a point \(\alpha\in X\)
    (written \(x_n \to \alpha\)) if 
    \[ \forall U \in \mathcal{T}, \exists N \in \naturalnumbers \suchthat a_n \in U, \quad \forall n \geq N \]
\end{snippetdefinition}

\plain{This definition of convergence admits sequences that converge to multiple values.
This is because we might not have enough open sets to require the sequence to be arbitrarily "close"
(like in a metric space) to only one convergence point.}

\begin{snippetexample}{topology-multiple-convergence-example}{Topological space multiple convergence}
    Let \(\ts\) be the \topologicalspace where \(X=\realnumbers\) and
    \[ \mathcal{T} = \{\emptyset, \realnumbers\} \union \{(b, \infty) \suchthat b \in \realnumbers\}\]
    The sequence \({\{a_n\}}_{n\in \naturalnumbers} = {\{\frac{1}{n}\}}_{n\in \naturalnumbers}\).
    This sequences clearly converges to \(0\) because for \(n>0\) the value of the sequence is
    contained in every open set that contains \(0\). However, the same goes for \(-1\), \(-2\) and so on.
    Thus, \(a_n\topologyconverges 0 \land a_n\topologyconverges -1 \land a_n \topologyconverges -2 \land \cdots\).
\end{snippetexample}

\plain{In order to achieve uniqueness, we need to require disjoint open sets for every pair of distinct elements.
This is precisely what the Hausdorff property does.}

\end{document}
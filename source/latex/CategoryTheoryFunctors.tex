\documentclass[preview]{standalone}

\usepackage{amsmath}
\usepackage{amssymb}
\usepackage{stellar}
\usepackage{definitions}
\usepackage{tikz}
\usepackage{makecell}
\usepackage{adjustbox}
\usepackage{bettelini}

\usetikzlibrary{cd}

\begin{document}

\id{categorytheory-functors}
\genpage

\section{Functors}

\subsection{Definitions}

\begin{snippetdefinition}{functor-definition}{Functor}
    Let \(\mathcal{C}, \mathcal{C'}\) be two \category[categories].
    A \emph{functor} from \(\mathcal{C}\) to \(\mathcal{C'}\)
    consists of a map
    \[
        F \colon \catob(\mathcal{C}) \fromto \catob(\mathcal{C'})
    \]
    and of maps
    \[
        F \colon \cathom_{\mathcal{C}}(a,b) \fromto \cathom_{\mathcal{C'}}(F(a), F(b))
    \]
    for all \(a,b \in \catob(\mathcal{C})\) such that:
    \begin{enumerate}
        \item \(F(\text{id}_a) = \text{id}_{F(a)}\) for all \(a \in \catob(\mathcal{C})\);
        \item \(F(g \circ f) = F(g) \circ F(f)\) for all \(f\colon a \fromto b, g\colon b \to c\).
    \end{enumerate}
\end{snippetdefinition}

\begin{snippetdefinition}{presheave-definition-definition}{Presheave}
    Let \(\mathcal{C}\) be a \category.
    A \emph{presheave} on \(\mathcal{C}\)
    % todo link dualcat
    is a \functor from \({\mathcal{C}}^{\text{op}}\) to \(\mathcal{C}\).
\end{snippetdefinition}

\subsection{Natural transformations}

\begin{snippetdefinition}{natural-transformation-definition}{Natural transformation}
    Let \(\mathcal{C}, \mathcal{C'}\) be two \category[categories]
    and let \(F_1, F_2 \colon \mathcal{C} \fromto \mathcal{C'}\) be two \functor[functors].
    A \emph{natural transformation} \(\alpha\colon F_1 \fromto F_2\)
    is a \function assigning to each object \(a \in \catob(\mathcal{C})\)
    a morphism \(\alpha(a) \colon F_1(a) \fromto F_2(a)\) in \(\mathcal{C'}\)
    such that for all morphisms \(f \colon a \fromto b\) in \(\mathcal{C}\)
    the following diagram commutes:
    \begin{center}
        % https://tikzcd.yichuanshen.de/#N4Igdg9gJgpgziAXAbVABwnAlgFyxMJZABgBpiBdUkANwEMAbAVxiRADEB9ARgAo6AlCAC+pdJlz5CKMtyq1GLNlz4AjIaPHY8BIt1Jzq9Zq0QdOAJl7qRYkBm1S95ecaVmuVwSPkwoAc3giUAAzACcIAFskMhAcCCQAZiNFUxAAHXTGNAALOn4NO3CopOp4pAsUk2VLXhDC0IjoxFjyxH0Fao8eOqFqBjpVGAYABQkdaRAwrH8cnFtGkvayhMRKzvcMrIZc-Jt+weGxx10zadn54QphIA
        \begin{tikzcd}
        F_1(a) \arrow[r, "\alpha(a)"] \arrow[d, "F_1(f)"'] & F_2(a) \arrow[d, "F_2(f)"] \\
        F_1(b) \arrow[r, "\alpha(b)"']                     & F_2(b)                    
        \end{tikzcd}
    \end{center}
\end{snippetdefinition}

\begin{snippetdefinition}{natural-isomorphism-definition}{Natural isomorphism}
    A \naturaltransformation \(\alpha\colon F_1 \fromto F_2\) is called a \emph{natural isomorphism} if
    for every object \(a \in \catob(\mathcal{C})\) the morphism \(\alpha(a)\) is an isomorphism.
\end{snippetdefinition}

\subsection{Representable functor}

\begin{snippetdefinition}{representable-functor-definition}{Representable functor}
    Let \(\mathcal{C}\) be a localy small \category and let \(c\in\catob(\mathcal{C})\).
    We define a \functor \(\cathom_{\mathcal{C}}(c, -) \colon \mathcal{C} \fromto \textbf{Set}\)
    defined by:
    \begin{itemize}
        \item \(\cathom_{\mathcal{C}}(c, -)(a) = \cathom_{\mathcal{C}}(c, a)\) for all \(a \in \catob(\mathcal{C})\);
        \item \(\cathom_{\mathcal{C}}(c, -)(f) \colon \cathom_{\mathcal{C}}(c, a) \fromto \cathom_{\mathcal{C}}(c, b)\)
        given by \(g \to f \circ g\) for all \(f \colon a \fromto b\) and \(g \colon c \fromto a\).
    \end{itemize}
    A \functor that is naturally isomorphic to \(\cathom_{\mathcal{C}}(c, -)\)
    for some \(c \in \catob(\mathcal{C})\) is said to be \emph{representable}.
\end{snippetdefinition}

\plain{The construction satisfies the properties of a functor (using function composition associativity).}

\plain{The dual can also be considered, where the morphisms go into the object instead of out of it.}

\subsection{Types of functors}

\begin{snippetdefinition}{faithful-functor-definition}{Faithful functor}
    A \functor \(F \colon \mathcal{C} \fromto \mathcal{D}\) is said to be \emph{faithful} if
    for all \(a,b \in \catob(\mathcal{C})\)
    \[
        F \colon \cathom_{\mathcal{C}}(a,b) \fromto \cathom_{\mathcal{D}}(F(a), F(b))
    \]
    is \injective.
\end{snippetdefinition}

\begin{snippetdefinition}{full-functor-definition}{Full functor}
    A \functor \(F \colon \mathcal{C} \fromto \mathcal{D}\) is said to be \emph{full} if
    for all \(a,b \in \catob(\mathcal{C})\)
    \[
        F \colon \cathom_{\mathcal{C}}(a,b) \fromto \cathom_{\mathcal{D}}(F(a), F(b))
    \]
    is \surjective.
\end{snippetdefinition}

\plain{We can also define a notion of surjectivity for objects.
However, in category theory the equality of objects is not very relevant,
and it is replaced by the notion of isomorphism.}

\begin{snippetdefinition}{essentially-surjective-functor-definition}{Essentially surjective functor}
    A \functor \(F \colon \mathcal{C} \fromto \mathcal{D}\) is said to be \emph{essentially surjective} if
    every object \(d \in \catob(\mathcal{D})\) is isomorphic to an object of the form \(F(c)\)
    for some \(c \in \catob(\mathcal{C})\).
\end{snippetdefinition}

\begin{snippetdefinition}{subcategory-definition}{Subcategory}
    \todo
\end{snippetdefinition}

\end{document}
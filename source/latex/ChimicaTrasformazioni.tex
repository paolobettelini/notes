\documentclass[preview]{standalone}

\usepackage{amsmath}
\usepackage{amssymb}
\usepackage{bettelini}
\usepackage{stellar}

\hypersetup{
    colorlinks=true,
    linkcolor=black,
    urlcolor=blue,
    pdftitle={Chimica},
    pdfpagemode=FullScreen,
}

\begin{document}

\title{Chimica}
\id{chimica-trasformazioni}
\genpage

\plain{Le trasformazioni possono essere classificate come <i>chimiche</i> o <i>fisiche</i>.}

\begin{snippetdefinition}{trasformazione-chimica-definition}{Trasformazione chimica}
    Una \textit{trasformazione chimica}
    modifica la sostanza.
\end{snippetdefinition}

\plain{Nelle trasformazioni chimiche, gli atomi sono gli stessi ma gli elementi sono diversi. Le particelle quindi
mutano.}

\begin{snippetdefinition}{trasformazione-fisica-definition}{Trasformazione fisica}
    Una \textit{trasformazione fisica}
    non modifica la materia ma il suo stato.
\end{snippetdefinition}

\plain{Nelle trasformazioni fisiche, la materia mantiene le sue proprietà e rimane invariata.}

\begin{snippetexample}{trasformazioni-chimiche-esempio}{Trasformazioni chimiche}
    \begin{itemize}
        \item Combustione di una candela (anche fisica).
        \item Cottura di un uovo (le proteine cambiano).
        \item Formazione della ruggina.
    \end{itemize}
\end{snippetexample}

\begin{snippetexample}{trasformazioni-chimiche-esempio}{Trasformazioni chimiche}
    \begin{itemize}
        \item Combustione di una candela (anche chimica).
        \item Sbucciare una mela.
        \item Scaldare il tisolfato di sofio.
        \item Dissoluzione dello zucchero nell'acqua.
    \end{itemize}
\end{snippetexample}

\end{document}

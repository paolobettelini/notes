\documentclass[preview]{standalone}

\usepackage{amsmath}
\usepackage{amssymb}
\usepackage{tikz}
\usepackage{stellar}
\usepackage{bettelini}

\hypersetup{
    colorlinks=true,
    linkcolor=black,
    urlcolor=blue,
    pdftitle={Assets},
    pdfpagemode=FullScreen,
}

\begin{document}

\title{Suolo}
\id{geofisica-suolo}
\genpage

\section{Il suolo}

\begin{snippetdefinition}{il-suolo}{Il Suolo}
    Il \textit{suolo} è lo strato incoerenti di detriti minerali prodotti dal disfacimento delle rocce,
    ricco di materia organica, liquidi, gas e forme di vita, che poggia sulla roccia
    in posto inalterata. Costituisce una superficie di transizione tra litosfera,
    idrosfera, atmosfera e biosfera.
\end{snippetdefinition}

\begin{snippet}{suolo-strati}
    Il suolo è composto da diversi strati (dalla superficie verso il basso):
    \begin{enumerate}
        \item \textbf{orizzonte A:} coperto da un sottile
            straterello di sabia con sostanza organica indecomposta (foglie, radici etc.),
            è costituito da sostanza organica decomposta (humus) e da minerali insolubili;
        \item \textbf{orizzonte B:} (rossiccio) è povero di materia organica e ricco di minerali che provengono
            dall'orizzonte A. Qui si sono depositati i composti chimici che l'acqua ha trasportato in soluzione infiltrandosi nel terreno.
        \item \textbf{orizzonte C:} è costituito da frammenti alterati, di varie dimensioni, della roccia madre sottostante;
        \item \textbf{roccia madre}.
    \end{enumerate}
\end{snippet}

\begin{snippet}{suolo-clima}
    Sopra il suolo, il clima può essere
    \begin{itemize}
        \item \textbf{temperato:} ricco di humic e adatto alla crescita delle piante;
        \item \textbf{caldo umido:} umido e povero di humus, ma con vegetazione rigogliosa;
        \item \textbf{arido:} povero di humus, piante con bassa necessità di acqua.
    \end{itemize}
\end{snippet}

\begin{snippet}{3d0b7d7d-e84c-4e56-9d07-9fd081ba0058}
    La composizione e lo spessore dle suolo sono il risultato di vari fattori strettamente
    collegati tra loro.
    La roccia madre fornisce il detrito e determina la composizione della parte minerale del suolo.
    La pendenza del terreno incide in modo rilevante sullo spessore del suolo.
    Se il terreno è inclinato, i detriti di roccia non si accumulano sul posto, ma scivolano
    verso il basso.
    Se la pendenza è molto forte, il suolo può essere del tutto assente, come
    sulle pareti rocciose di alta montagna.

    Il clima influisce sui processi di formazione, sullo spessore del suolo e sullo sviluppo e sul tipo di vegetazione.
    Nei climi temperati, dove le precipitazioni sono frequenti e la vegetazione ha un discreto sviluppo,
    lo spessore del suolo è consistente.
    Nelle regioni caldo umide, dove le forti precipitazioni favoriscono un grande sviluppo della vegetazione
    e le alte temperature accelerano le reazioni di alterazione delle rocce, il suolo assume spessori massimi.

    I posti con climi tropicali e temperati hanno generalmente una quantità
    abbastanza abbondante di orrizzonti A, B e C, mentre i climi freddi e aridi hanno pricipalmente
    l'orizzonte C.
\end{snippet}

\includesnpt[width=30\%|src=/snippet/static/profili-terra.png]{centered-img}

\end{document}
\documentclass[preview]{standalone}

\usepackage{amsmath}
\usepackage{amssymb}
\usepackage{stellar}
\usepackage{definitions}

\begin{document}

\id{linearalgebra-exercises-batch-2}
\genpage

\section{Exercises}

\begin{snippetexercise}{linear-algebra-batch-2-ex-1}{}
    Solve the following linear systems:
    \[
        \begin{cases}
            x_1 - 3x_2 - x_3 + x_4 = 0 \\
            2x_2 - 4x_3 + x_4 = 1 \\
            x_1 - x_2 + 4x_3 - 2x_4 = -1 \\
            2x_1 - 2x_2 - x_3 + 2x_4 = 0
        \end{cases}
        \quad
        \begin{cases}
            x + y + z + w = 1 \\
            8x + 4y + 2z + w = 5 \\
            27x + 9y + 3z + w = 14 \\
            64x + 16y + 4z + w = 30
        \end{cases}
    \]
\end{snippetexercise}

\begin{snippetsolution}{linear-algebra-batch-2-ex-1-proof}{}
    For the first one we have
    \[
        \begin{bmatrix}
            1 & -3 & -1 & 1 \\
            0 & 2 & -4 & 1 \\
            1 & -1 & 4 & -2 \\
            2 & -2 & -1 & 2
        \end{bmatrix}
        \begin{pmatrix}
            x_1 \\
            x_2 \\
            x_3 \\
            x_4
        \end{pmatrix}
        = \begin{pmatrix}
            0 \\
            1 \\
            -1 \\
            0
        \end{pmatrix}
    \]
    using Gaussian elimination we get
    \[
        \begin{bmatrix}
            1 & -3 & -1 & 1 \\
            0 & 2 & -4 & 1 \\
            0 & 0 & 9 & -4 \\
            0 & 0 & 0 & 2
        \end{bmatrix}
        \begin{pmatrix}
            x_1 \\
            x_2 \\
            x_3 \\
            x_4
        \end{pmatrix}
        = \begin{pmatrix}
            0 \\
            1 \\
            -2 \\
            0
        \end{pmatrix}
    \]
    fron which we get
    \[
        (x_1, x_2, x_3, x_4) = \left(
            -\frac{1}{18}, \frac{1}{18}, -\frac{2}{9}, 0
        \right)
    \]
    For the second one we have
    \[
        \begin{bmatrix}
            1 & 1 & 1 & 1 \\
            8 & 4 & 2 & 1 \\
            27 & 9 & 3 & 1 \\
            64 & 16 & 4 & 1
        \end{bmatrix}
        \begin{pmatrix}
            x \\
            y \\
            z \\
            w
        \end{pmatrix}
        = \begin{pmatrix}
            1 \\
            5 \\
            14 \\
            30
        \end{pmatrix}
    \]
    using Gaussian elimination we get
    \[
        \begin{bmatrix}
            1 & 1 & 1 & 1 \\
            0 & -4 & -6 & -7 \\
            0 & 0 & 3 & \frac{11}{2} \\
            0 & 0 & 0 & -1
        \end{bmatrix}
        \begin{pmatrix}
            x_1 \\
            x_2 \\
            x_3 \\
            x_4
        \end{pmatrix}
        = \begin{pmatrix}
            1 \\
            -3 \\
            \frac{1}{2} \\
            0
        \end{pmatrix}
    \]
    fron which we get
    \[
        (x,y,z,w) = \left(
            \frac{1}{3}, \frac{1}{2}, \frac{1}{6}, 0
        \right)
    \]
\end{snippetsolution}

\begin{snippetexercise}{linear-algebra-batch-2-ex-2}{}
    Compute every possible product between the following matrices:
    \[
        A = \begin{pmatrix}
            1 & 2 & 4
        \end{pmatrix},
        \quad
        B = \begin{pmatrix}
            3 \\ 2 \\ 1
        \end{pmatrix},
        \quad
        C = \begin{pmatrix}
            1 & 2 & 4 \\
            2 & 0 & -1
        \end{pmatrix},
        \quad
        D = \begin{pmatrix}
            1 & 2 \\
            0 & 1 \\
            -1 & 4
        \end{pmatrix}
    \]
\end{snippetexercise}

\begin{snippetsolution}{linear-algebra-batch-2-ex-2-proof}{}
    The possible products are:
    \begin{align*}
        AB &= 11 \\
        AD &= \begin{pmatrix}-3 & 20\end{pmatrix} \\
        CB &= \begin{pmatrix}11 \\ 5\end{pmatrix} \\
        CD &= \begin{pmatrix}-3 & 20 \\ 3 & 0\end{pmatrix} \\
        BA &= \begin{pmatrix}3 & 6 & 12 \\ 2 & 4 & 8 \\ 1 & 2 & 4\end{pmatrix} \\
        DC &= \begin{pmatrix}5 & 2 & 2 \\ 2 & 0 & -1 \\ 7 & -2 & -8\end{pmatrix}
    \end{align*}
\end{snippetsolution}

\begin{snippetexercise}{linear-algebra-batch-2-ex-5}{}
    \todo
\end{snippetexercise}

\begin{snippetsolution}{linear-algebra-batch-2-ex-5-proof}{}
    \todo
\end{snippetsolution}

\begin{snippetexercise}{linear-algebra-batch-2-ex-6}{}
    \todo
\end{snippetexercise}

\begin{snippetsolution}{linear-algebra-batch-2-ex-6-proof}{}
    \todo
\end{snippetsolution}

\begin{snippetexercise}{linear-algebra-batch-2-ex-7}{}
    \todo
\end{snippetexercise}

\begin{snippetsolution}{linear-algebra-batch-2-ex-7-proof}{}
    \todo
\end{snippetsolution}

\begin{snippetexercise}{linear-algebra-batch-2-ex-8}{}
    \todo
\end{snippetexercise}

\begin{snippetsolution}{linear-algebra-batch-2-ex-8-proof}{}
    \todo
\end{snippetsolution}

\end{document}
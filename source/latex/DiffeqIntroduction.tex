\documentclass[preview]{standalone}

\usepackage{amsmath}
\usepackage{amssymb}
\usepackage{stellar}
\usepackage{definitions}

\begin{document}

\id{diffeq-introduction}
\genpage

\section{Introduction}

\begin{snippetdefinition}{nth-order-differential-equation-normal-form-definition}{\(N\)-th order differential equation in normal form}
    Let \(n\in\naturalnumbers^+\).
    An \emph{\(n\)-th order differential equation in normal form} is an equation of the form
    \[
        \frac{d^n y}{dx^n} = f\left(x, y, \frac{dy}{dx}, \ldots, \frac{d^{n-1}y}{dx^{n-1}}\right)
    \]
    where \(f\colon A \subseteq \realnumbers^{n+1} \fromto \realnumbers\) with \(A\) \msopenset[open]
    and \(f\) is continuous.
\end{snippetdefinition}

\begin{snippetproposition}{nth-order-differential-equation-normal-form-necessary-condition}{}
    Consider an \(n\)-th order differential equation in normal form
    \[
        \frac{d^n y}{dx^n} = f\left(x, y, \frac{dy}{dx}, \ldots, \frac{d^{n-1}y}{dx^{n-1}}\right)
    \]
    A \emph{solution} to it is written \((\varphi, I)\) where \(\varphi\colon I \subseteq \realnumbers \fromto \realnumbers\) is a \function
    and \(I\) is an interval such that:
    \begin{enumerate}
        \item \(\forall t\in I, (t, \varphi(t), \varphi'(t), \cdots, \varphi^{(n-1)}(t)) \in A\);
        \item \(\varphi \in \continuityclass^n(I)\);
        \item \(\forall t\in I\),
        \[
            \frac{d^n \varphi}{dt^n} = f\left(t, \varphi, \frac{d\varphi}{dt}, \ldots, \frac{d^{n-1}\varphi}{dt^{n-1}}\right).
        \]
    \end{enumerate}
\end{snippetproposition}

\begin{snippetdefinition}{nth-order-cauchy-problem-definition}{Cauchy problem}
    Let \(n\in\naturalnumbers^+\).
    A \emph{Cauchy problem} or \emph{initial value problem} is a differential equation
    with a constraint on the initial conditions
    \[
        \begin{cases}
            \frac{d^n y}{dx^n} = f\left(x, y, \frac{dy}{dx}, \ldots, \frac{d^{n-1}y}{dx^{n-1}}\right) \\
            y(x_0) = y_0 \\
            \frac{dy}{dx}(x_0) = y_1 \\
            \vdots \\
            \frac{d^{n-1}y}{dx^{n-1}}(x_0) = y_{n-1}
        \end{cases}
    \]
    where \((x_0, y_0, y_1, \ldots, y_{n-1}) \in A\).
\end{snippetdefinition}

\begin{snippetproposition}{nth-order-cauchy-problem-necessary-condition}{}
    Consider a Cauchy problem
    \[
        \begin{cases}
            \frac{d^n y}{dx^n} = f\left(x, y, \frac{dy}{dx}, \ldots, \frac{d^{n-1}y}{dx^{n-1}}\right) \\
            y(x_0) = y_0 \\
            \frac{dy}{dx}(x_0) = y_1 \\
            \vdots \\
            \frac{d^{n-1}y}{dx^{n-1}}(x_0) = y_{n-1}
        \end{cases}
    \]
    A solution \((\varphi, I)\) to the differential equation is also a solution to the Cauchy problem
    if
    \[
        \varphi(x_0) = y_0, \quad \varphi'(x_0) = y_1, \quad \cdots, \quad \varphi^{(n-1)}(x_0) = y_{n-1}
    \]
\end{snippetproposition}

\begin{snippetdefinition}{diffeq-linear-definition}{Linear Differential Equation}
    A differential equation is said to be \textit{linear} if it can be written as
    \[
        \sum_n a_n(t) \frac{d^n}{dt^n}y(t)=g(t)
    \]
    where there are no products of the function \(y(t)\) and its derivatives,
    \(y(t)\) or its derivative do not occur to any power other than the first power
    and \(y(t)\) or any of its derivative are composed with another function.
\end{snippetdefinition}

\end{document}
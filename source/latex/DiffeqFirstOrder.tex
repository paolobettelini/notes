\documentclass[preview]{standalone}

\usepackage{amsmath}
\usepackage{amssymb}
\usepackage{fullpage}
\usepackage{bettelini}
\usepackage{stellar}
\usepackage{definitions}
%\usepackage{pgfplots}
%\usepackage{wrapfig}

%\usetikzlibrary{calc}

\begin{document}

\id{diffeq-first-order}
\genpage

\section{Linear equations}

\begin{snippettheorem}{linear-first-order-differential-equation-solution}{Solution to a linear first-order differential equation}
    The general solution to the differential equation
    \[
        y' = a(t)y(t) + b(t)
    \]
    is given by
    \[
        y(t)=Ce^{A(t)} + e^{A(t)} \int e^{-A(t)} b(t)\, dt
    \]
    where \(A(t)=\int a\,dt\), \(c\in\realnumbers\) and \(a\) and \(b\) are
    continuous \function[functions].
\end{snippettheorem}
    
\begin{snippetproof}{linear-first-order-differential-equation-solution-proof}{linear-first-order-differential-equation-solution}{Solution to a linear first-order differential equation}
    Let's start by letting \(b(t)=0\)
    \[
        y' = ay
        \implies
        \frac{y'}{y}=a
        \implies
        \ln|y|'=a
        \implies
        \ln|y|=\int a\,dt
    \]
    concluding that
    \[
        y = \pm e^{A+c_0}=\pm e^{c_0} \cdot e^{A} = Ce^A
    \]
    where \(A=\int a\, dt\).
    \\
    We choose an integrating factor \(\mu\) such that
    \[
        -a\mu = \mu'
    \]
    By solving this differential equation we get
    \[
        \frac{\mu'}{\mu} = -a
        \implies
        \ln|\mu| = -A+C
        \implies
        \mu(t) = Ce^{-A}
    \]
    And by choosing \(C=1\) we have
    \[
        \mu(t) = e^{-A(t)}
    \]
    Now we multiply our equation by the integrating factor
    \begin{align*}
        y' - ay &= b \\
        y'\mu - a\mu y &= \mu b \\
        y' \mu + \mu' y &= \mu b \\
        \left(y \mu\right)' &= \mu b \\
        \left(e^{-A(t)}y\right)' &= e^{-A(t)}b \\
        e^{-A(t)}y &= \int e^{-A(t)}b \,dt + C \\
        y(t) &= Ce^{A(t)} + e^{A(t)} \int e^{-A(t)}b \,dt
    \end{align*}
\end{snippetproof}

\section{Bernoulli Equation}

\begin{snippetdefinition}{bernoulli-equation-definition}{Bernoulli Equation}
    The Bernoulli equation has the form
    \[
        y' = p(t)y + q(t)y^n
    \]
\end{snippetdefinition}

\begin{snippettheorem}{bernoulli-equation-sol}{Theorem}
    The general solution of the Bernoulli equation is the general
    solution of the linear equation
    \[
        v' = -(n-1)p(t)v-(n-1)q(t)
    \]
    where
    \[
        v = \frac{1}{y^{n-1}}
    \]
\end{snippettheorem}

\begin{snippetproof}{bernoulli-equation-sol-proof}{bernoulli-equation-sol}{Theorem}
    They idea is to transform this equation into a simplier
    linear first-order equation. \\
    Start by dividing both sides by \(y^n\)
    \[
        \frac{y'}{y^n} = \frac{p(t)}{y^{n-1}} + q(t)
    \]
    Let
    \[
        v = y^{-(n-1)}, 
        \quad
        v' = -(n-1)y^{-n}y'
    \]
    Thus
    \[
        -\frac{v'}{n-1} = \frac{y'(t)}{y^n(t)}
    \]
    By substituting we get
    \begin{align*}
        -\frac{v'}{n-1} &= p(t)v + q(t) \\
        v' &= -(n-1)p(t)v-(n-1)g(t)
    \end{align*}
\end{snippetproof}

\section{Separable equations}

\plain{Separable equations are equations that can be solved by integrating both sides.}

\begin{snippetdefinition}{separale-equation-definition}{Separable Equation}
    A \emph{separable equation} has the form
    \[
        h(y)y'=g(t)
    \]
\end{snippetdefinition}

\begin{snippettheorem}{separale-equation-sol}{Solution to a separable equation}
    A separable differential equation has an implicit solution
    \[
        H(y(t)) = G(t) + C
    \]
    where
    \begin{align*}
        H(y) = \int h(s)\,ds
        ,\quad
        G(t) = \int g(t)\,dt
    \end{align*}
\end{snippettheorem}

\begin{snippetproof}{separale-equation-sol-proof}{separale-equation-sol}{Solution to a separable equation}
    Start by integrating both sides of the equation
    \[
        \int h(y(t))y'(t)\,dt =
        \int g(t)\,dt + C
    \]
    Now substitute for
    \[
        s=y(t),
        \quad
        ds=y'(t)\,dt
    \]
    meaning
    \[
        \int h(s)\, ds = 
        \int g(t)\,dt
    \]
    which could be written as
    \[
        H(y) = G(t) + C
    \]
\end{snippetproof}

\section{Exact equations}

\begin{snippetdefinition}{exact-equation-definition}{Exact Equation}
    Consider a differential equation with the form
    \[
        M(x,y)+N(x,y)\frac{dy}{dx}=0
    \]
    Then, the equation is \textit{exact} if there exists a continuously
    differentiable function \(\Psi(x,y)\) such that
    \[
        \frac{\partial \Psi}{\partial x} = M(x,y)
        \quad \text{and} \quad
        \frac{\partial \Psi}{\partial y} = N(x,y)
    \]
\end{snippetdefinition}

\begin{snippet}{exact-equation-sol}
    We can then rewrite the differential equation as
    \[
        \frac{\partial \Psi}{\partial x}+\frac{\partial \Psi}{\partial y} \frac{dy}{dx}=0
    \]
    Using the multi variable chain rule it can be reduced to
    \[
        \frac{d}{dx}\left( \Psi(x,y(x)) \right) = 0
    \]
    We can clearly see that here the derivative is equal to \(0\), meaning that
    the function must be a constant. This gives us an implicit solution
    \[
        \Psi(x,y)=C
    \]
\end{snippet}

\section{Homogeneous equations}

\begin{snippetdefinition}{homogenous-equation-definition}{Homogeneous Equation}
    A \textit{homogenous} equation has the form
    \[
        \frac{dy}{dx} = F\left(\frac{y}{x}\right)
    \]
\end{snippetdefinition}

\begin{snippet}{homogenous-equation-sol}
    We use the substitution
    \[
        v=\frac{y}{x}
    \]
    Note that
    \begin{align*}
        y' &= (xv)' = v+yv' \\
        &= F(v)
    \end{align*}
    We then have
    \begin{align*}
        v+xv'&=F(v) \\
        xv'&=F(v)-v \\
        \frac{v'}{F(v)-v}&=\frac{1}{x}
    \end{align*}

    This is a separable equation. Thus, an implicit solution is given by
    \[
        \integral[\frac{1}{F(v)-v}][v]=\ln|x|+C
    \]
\end{snippet}

\section{Slope Field}

\begin{snippetdefinition}{slope-field-definition}{Slope Field}
    A slope field or directional field is a field to visualize
    solutions to a first-order differential equation.
\end{snippetdefinition}

% TODO
%\begin{snippet}{slope-field-example}
%\begin{wrapfigure}{l}{7.5cm}
%    \begin{tikzpicture}[declare function={f(\x,\y)=\x+\y;}]
%        \def\xmax{3} \def\xmin{-3}
%        \def\ymax{3} \def\ymin{-3}
%        \def\nx{15}
%        \def\ny{15}
%        
%        \pgfmathsetmacro{\hx}{(\xmax-\xmin)/\nx}
%        \pgfmathsetmacro{\hy}{(\ymax-\ymin)/\ny}
%        \foreach \i in {0,...,\nx}
%            \foreach \j in {0,...,\ny}{
%                \pgfmathsetmacro{\yprime}{f({\xmin+\i*\hx},{\ymin+\j*\hy})}
%                \draw[blue,shift={({\xmin+\i*\hx},{\ymin+\j*\hy})}] 
%                (0,0)--($(0,0)!2mm!(.1,.1*\yprime)$);
%            }
%        
%        % a solution y=(yo+1)e^x-x-1
%        \def\yo{1}
%        \draw[magenta] plot[domain=\xmin:.9] (\x,{(\yo+1)*exp(\x)-\x-1});
%        
%        \draw[->] (\xmin-.5,0)--(\xmax+.5,0) node[below right] {\(x\)};
%        \draw[->] (0,\ymin-.5)--(0,\ymax+.5) node[above left] {\(y\)};
%        
%        \draw (current bounding box.south) node[below]
%        {Slope field of \quad \(\frac{dy}{dx}=x+y\).};
%    \end{tikzpicture}
%\end{wrapfigure}
%
%This field is obtained by picking points on the plane. \\
%For each point \((x,y)\) we know that the slope \((\frac{dy}{dx})\)
%is \(x + y\). \\
%This means that if a solution passes through \((x,y)\), then its slope is \(x+y\). \\
%The red curve shows a solution.
%
%\wrapfill
%\end{snippet}

\subsection{Euler's Method}

\begin{snippetdefinition}{euler-method}{Euler's Method}
    Euler's method is a technique for solving a 
    first-order differential equation numerically given a point of the solution.
    \\
    Starting at the known solution point \(A_0\), we take small steps the direction
    of the slope field. As the length of the steps \(s \to 0\)
    we approach the solution to the equation. \\
\end{snippetdefinition}

\end{document}
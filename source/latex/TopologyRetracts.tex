\documentclass[preview]{standalone}

\usepackage{amsmath}
\usepackage{amssymb}
\usepackage{stellar}
\usepackage{definitions}
\usepackage{bettelini}

\begin{document}

\id{topology-retracts}
\genpage

\section{Retracts and deformation retracts}

\begin{snippetdefinition}{retract-definition}{Retract}
    Let \(X\) be a \topologicalspace and \(Y \subseteq X\) a subspace.
    We say that \(Y\) is a \emph{retract} of \(X\)
    if there exists a \topologycontinuous[continuous] map \(r \colon X \fromto Y\),
    called a \emph{retraction}, such that \(r(y) = y\) for all \(y \in Y\).
\end{snippetdefinition}

\begin{snippetdefinition}{deformation-retract-definition}{Deformation retract}
    Let \(X\) be a \topologicalspace.
    A subspace \(Y \subseteq X\) is called a \emph{deformation retract} of \(X\)
    if there exists a \topologycontinuous[continuous] map
    \[
        R \colon X \cartesianprod [0,1] \fromto X
    \]
    called a \emph{deformation of \(X\) onto \(Y\)}, such that:
    \begin{enumerate}
        \item \(R(x, 0) = x\) for all \(x \in X\);
        \item \(R(x, 1) \in Y\) for all \(x \in X\);
        \item \(R(y, t) = y\) for all \(y \in Y\) and \(t \in [0,1]\).
    \end{enumerate}
\end{snippetdefinition}

\begin{snippetexample}{star-shaped-deformation-retract}{Star-shaped set}
    Let \(A \subseteq \realnumbers^n\) be a star-shaped set with respect to a point \(p \in A\).
    Then \(\{p\}\) is a deformation retract of \(A\) via the map
    \[
        R(x, t) = (1-t)x + tp
    \]
\end{snippetexample}

\begin{snippetproposition}{deformation-retract-is-homotopy-equivalence}{Deformation retract is homotopy equivalence}
    Let \(X\) be a \topologicalspace and \(Y \subseteq X\) a deformation retract.
    Then \(Y\) is a retract of \(X\) and the inclusion \(i \colon Y \inclusion X\)
    is a \snippetref[homotopy-equivalence-definition][homotopy equivalence].
\end{snippetproposition}

\begin{snippetproof}{deformation-retract-is-homotopy-equivalence-proof}{deformation-retract-is-homotopy-equivalence}{Deformation retract is homotopy equivalence}
    Let \(R \colon X \cartesianprod [0,1] \fromto X\) be the deformation.
    Define \(r \colon X \fromto Y\) by \(r(x) = R(x, 1) \in Y\).
    
    For \(y \in Y\), we have \(r(y) = R(y, 1) = y\), so \(r\) is a retraction.
    Thus \(r \circ i = \identity_Y\).
    
    On the other hand, \(R\) provides a homotopy between \(i \circ r\) and \(\identity_X\):
    \begin{itemize}
        \item \(R(x, 0) = x = \identity_X(x)\);
        \item \(R(x, 1) = r(x) = (i \circ r)(x)\).
    \end{itemize}
    Hence \(i \circ r \simeq \identity_X\).
\end{snippetproof}

\begin{snippetexample}{sphere-deformation-retract}{Sphere as deformation retract}
    Consider the inclusion \(S^n \inclusion \realnumbers^{n+1} \difference \{0\}\)
    and the map \(r \colon \realnumbers^{n+1} \difference \{0\} \fromto S^n\)
    given by \(r(x) = x / \|x\|\).
    
    The map \(R(x, t) = (1-t)x + t \cdot r(x)\) gives a deformation retraction,
    so \(S^n\) is a deformation retract of \(\realnumbers^{n+1} \difference \{0\}\).
\end{snippetexample}

\end{document}

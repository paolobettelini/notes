\documentclass[preview]{standalone}

\usepackage{amsmath}
\usepackage{amssymb}
\usepackage{stellar}
\usepackage{definitions}

\begin{document}

\id{multivariable-optimization-exercises}
\genpage

\section{Exercies}

\begin{snippetexercise}{multivariable-optimization-ex-1}{}
    Determine the local extrema of the \function
    \[
        f(x,y) = x^2y - xy^2 + xy
    \]
\end{snippetexercise}

\begin{snippetsolution}{multivariable-optimization-ex-1-sol}{}
    The gradient is given by
    \[
        \gradient f = \begin{pmatrix}
            2xy - y^2 + y \\ x^2 - 2xy + x
        \end{pmatrix}
    \]
    the components are null when
    \begin{align*}
        \begin{cases}
            y(2x - y + 1) = 0 \\
            x(x-2y + 1) = 0
        \end{cases}
    \end{align*}
    meaning that the point \(P_1 = (0,0)\),
    \(P_2 = (-1,0)\), \(P_3 = (0,1)\) and \(P_4 = (-1/3, 1/3)\) are critical.
    The Hessian matrix is given by
    \[
        Hf = \begin{pmatrix}
            2y & 2x - 2y + 1 \\ 2x - 2y + 1 & -2x
        \end{pmatrix}
    \]
    At the four points we have
    \[
        Hf(P_1) = \begin{pmatrix}
            0 & 1 \\ 1 & 0
        \end{pmatrix}, \quad
        Hf(P_2) = \begin{pmatrix}
            0 & -1 \\ -1 & 2
        \end{pmatrix}, \quad
        Hf(P_3) = \begin{pmatrix}
            2 & -1 \\ -1 & 0
        \end{pmatrix}, \quad
        Hf(P_4) = \begin{pmatrix}
            \frac23 & -\frac13 \\ -\frac13 & \frac23
        \end{pmatrix}
    \]
    which yields the following characterizations:
    \begin{enumerate}
        \item[\(P_1\):] \(\det Hf(P_1) = -1\). One positive and one negative eigenvalue, meaning this is a saddle point;
        \item[\(P_2\):] \(\det Hf(P_2) = -1\). One positive and one negative eigenvalue, meaning this is a saddle point;
        \item[\(P_3\):] \(\det Hf(P_3) = -1\). One positive and one negative eigenvalue, meaning this is a saddle point;
        \item[\(P_4\):] \(\det Hf(P_4) = 3/9\). We have two positive eigenvalues,
        and \(\trace(Hf(P_4)) = 4/3>0\) meaning that it is a minimum.
    \end{enumerate}
\end{snippetsolution}

\begin{snippetexercise}{multivariable-optimization-ex-2}{}
    Determine the local extrema of the \function
    \[
        f(x,y) = \frac{1}{2}x^2y^2 - 2y^2 + \frac{1}{3}x^3
    \]
\end{snippetexercise}

\begin{snippetexercise}{multivariable-optimization-ex-3}{}
    Determine the local extrema of the \function
    \[
        f(x,y) = {(x-1)}^2 (x^2 - y^2)
    \]
\end{snippetexercise}

\begin{snippetexercise}{multivariable-optimization-ex-4}{}
    Determine the local extrema of the \function
    \[
        f(x,y) = \frac{x^3y}{3} + \frac{1}{2}x^2y + \frac{1}{2}y^2
    \]
\end{snippetexercise}

\begin{snippetexercise}{multivariable-optimization-ex-5}{}
    Determine the local extrema of the \function
    \[
        f(x,y) = x^2y(x-y+1)
    \]
\end{snippetexercise}

\begin{snippetexercise}{multivariable-optimization-ex-6}{}
    Determine the local extrema of the \function
    \[
        f(x,y) = xy{(1-x^2-y^2)}^2
    \]
\end{snippetexercise}

\begin{snippetexercise}{multivariable-optimization-ex-7}{}
    Determine the local extrema of the \function
    \[
        f(x,y) = \frac{x^2 + 2y}{x^2 + y^2 + 1}
    \]
\end{snippetexercise}

\begin{snippetexercise}{multivariable-optimization-ex-8}{}
    Verify that the origin is a critical point for the \function[functions]
    \[
        f(x,y) = \sin^2(x-z) + y^2 - xyz
    \]
    and
    \[
        g(x,y) = \sin^2(x-z) + y^2 + y^2z
    \]
    and determine their nature.
\end{snippetexercise}

\end{document}
\documentclass[preview]{standalone}

\usepackage{amsmath}
\usepackage{amssymb}
\usepackage{stellar}
\usepackage{bettelini}

\hypersetup{
    colorlinks=true,
    linkcolor=black,
    urlcolor=blue,
    pdftitle={Stellar},
    pdfpagemode=FullScreen,
}

\begin{document}

\title{Geografia economica}
\id{geoeconomica-crisi-1929}
\genpage

\section{La crisi del 1929}

\begin{snippetdefinition}{crisi-1929-definizione}{Crisi 1929}
    La \textit{crisi del 1929} fu causata da diverse cause quali:
    \begin{itemize}
        \item speculazione eccessiva sul mercato azionario negli Stati Uniti;
        \item capacità di acquisto della popolazione minore della sovrvrapproduzione Stati Uniti;
        \item disuguaglianze economiche eccessive;
        \item prestiti delle banche sconsiderate e spesso con capitale prestato;
        \item risposta del governo degli Stati Uniti non sufficientemente rapida o efficace per affrontare la crisi.
    \end{itemize}
\end{snippetdefinition}

\begin{snippet}{crisi-1929-expl}
    La crisi del 1929 inizia con il crollo della Borsa di New York, che fu la conseguenza degli
    squilibri post bellici della Prima guerra mondiale.
    \\
    Gli Stati Uniti e il Giappone, sfruttando il loro arricchimento durante il conflitto mondiale,
    divennero le principali potenze, dominando l'economia mondiale.
    \\
    L'Europa, imporevita dalla guerra, affrontò una grande instabilità economica che portò a un
    costo della materia prima e della vita estremamente alto.
    \\
    Dal grande arricchimento, gli Stati Uniti d'America specularono nel finanziamento delle aziende
    coinvolte nei conflitti interni europei, le quali non resero abbastanza guadagni e portatono a
    un'imminente crollo della Borsa americana. Giorno ricordato come \textit{Giovedì nero}.
    \\
    Con il crollo della Borsa, susseguì una grave crisi economica globale.
    Le aziende di tutto il mondo fallirono, la disoccupazione aumentò e la domanda interna crollò.
    \\
    La depressione dovuta alla carenza economica toccò il suo apice nel 1932, devastando l'economia
    mondiale e causando la più grande crisi mondiale.
\end{snippet}

\begin{snippet}{434f0b7b-595b-408c-98c7-b87246ef4a50}
    Con \textbf{Grande Depressione} si intende il periodo 1929-1939 al seguite della crisi del '29.

    Questa crisi economica devastante ebbe origine negli Stati Uniti a causa del crollo del mercato azionario,
    che si diffuse a livello globale.
    Negli Stati Uniti, portò a una maggiore disoccupazione, povertà diffusa e un
    declino drammatico dell'economia.
    In Europa, la crisi portò a una serie di instabilità politiche, sociale ed economica,
    facilitando l'ascesa di regimi autoritari e alimentando il malcontento sociale,
    che in alcuni casi condusse a tensioni prebelliche.
    
    La Grande Depressione rappresentò anche un elemento catalizzatore per il protezionismo economico,
    con molti paesi adottando politiche di autarchia e barriere commerciali per proteggere le proprie economie,
    portando a un ulteriore isolazionismo economico tra le nazioni.
    Questo contesto geopolitico contribuì indirettamente allo scoppio della Seconda Guerra Mondiale.
    L'instabilità economica e politica causata dalla Grande Depressione ha reso il terreno
    fertile per la crescita di regimi totalitari in Europa, come il nazismo in Germania e il fascismo in Italia,
    che alla fine hanno portato allo scoppio del conflitto mondiale nel 1939.
\end{snippet}

\end{document}
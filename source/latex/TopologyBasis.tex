\documentclass[preview]{standalone}

\usepackage{amsmath}
\usepackage{amssymb}
\usepackage{stellar}
\usepackage{definitions}
\usepackage{bettelini}

\def\setX{\blue X \clear}
\def\setT{\teal \mathcal{T} \clear}
\def\ts{(\setX, \setT)}

\begin{document}

\id{topology-basis}
\genpage

\section{Basis}

\begin{snippetdefinition}{topological-space-basis-definition}{Basis of topological space}
    \def\tbasis{\scolor[orange] \mathcal{B} \clear}
    \def\Ai{\scolor[orange!75] A_{\gray i} \clear}
    \def\setU{\scolor[teal!75] U \clear}
    \def\indexes{\gray i\in I \clear}
    Let \(\ts\) be a \topologicalspace.
    A collection of \topologicalspace[open sets] \(\tbasis \subseteq \setT\)
    is called a \textit{basis} (\textit{base}) of \(\setT\) if for all \(\setU\in \setT\)
    there is a \sequence \({\{\Ai\}}_{\indexes}\) with \(\Ai \in \tbasis\) and
    \[ \bigcup_{\indexes} \Ai=\setU \]
\end{snippetdefinition}

\begin{snippetproposition}{topology-is-basis}{Topology is always a basis}
    Let \((X, \mathcal{T})\) be a \topologicalspace.
    Then, \(\mathcal{B} = \mathcal{T}\) is a \topologicalbasis of \(\mathcal{T}\).
\end{snippetproposition}

\begin{snippetproposition}{basis-of-discrete-topology}{Basis of discrete topology}
    Let \((X, \powerset(X))\) be a \topologicalspace.
    Then, \(\mathcal{B} = \left\{\{x\} \suchthat x\in X\right\}\) is a \topologicalbasis of \(\mathcal{T}\).
\end{snippetproposition}

\begin{snippetproposition}{topology-induced-by-metric-basis}{Basis of topology induced by metric space}
    Let \((X,d)\) be a \metricspace and \((X, \metrictopology_d)\)
    be the \topologicalspace induced by \((X,d)\).
    Then, \[\mathcal{B} = \left\{\ball_\varepsilon(x) \suchthat x\in X, \varepsilon>0\right\}\]
    is a \topologicalbasis of \(\mathcal{T}\).
\end{snippetproposition}

\begin{snippettheorem}{topology-from-arbitrary-basis-theorem}{}
    Let \(X\) be a \set and \(\mathcal{B} \subseteq \powerset(X)\).
    There exist a unique \topologicalspace[topology][Topology]
    over \(X\) where \(\mathbb{B}\) is a \topologicalbasis
    \ifandonlyif:
    \begin{enumerate}
        \item \[
            X = \bigcup_{B\in\mathcal{B}} B
        \]
        \item \(\forall A, B \in \mathcal{B}\), \(A \intersection B\)
        can be written as the union of subsets in \(\mathbb{B}\).
    \end{enumerate}
\end{snippettheorem}

\begin{snippetproof}{topology-from-arbitrary-basis-theorem-proof}{topology-from-arbitrary-basis-theorem}{}
    \iffproof{
        The first condition is satisfied since \(X\) is open and \(\mathcal{B}\) is a \topologicalbasis.
        The second condition is satisfied since \(A,B\) are open and thus \(A \cap B\)
        is open in \(X\) and thus, by definition of \topologicalbasis, it can be written as the union of subsets
        of \(\mathcal{B}\).
    }{
        We define \(\mathcal{T}\) as the collection of all the subsets
        that can be written as the union of subsets in \(\mathcal{B}\).
        This definition is forced if we want the collection \(\mathcal{B}\)
        to induce a \topologicalbasis for \(\mathcal{T}\).
        This is the reason why the construction it's unique - because I must choose these sets.
        \begin{enumerate}
            \item We have \(\emptyset \in \mathcal{T}\) because
            \[
                \emptyset = \bigcup \mathcal{B}', \quad \mathcal{B}' \subseteq \mathcal{B}
            \]
            by (trivially) taking \(\mathcal{B}' = \emptyset\).
            \item We have \(X \in \mathcal{T}\) because
            \[
                X = \bigcup \mathcal{B}', \quad \mathcal{B}' \subseteq \mathcal{B}
            \]
            by (trivially) taking \(\mathcal{B}' = X\).
            \item Let \(S, T \in \mathcal{T}\). We have
            \[
                S = \bigcup_{i \in I} B_i, \quad 
                T = \bigcup_{j \in J} B_j'
            \]
            for \(B_i, B_j' \in \mathcal{B}\). By taking the intersection
            \begin{align*}
                S \intersection T &=
                \left(
                    \bigcup_{i \in I} B_i
                \right)
                \intersection
                \left(
                    \bigcup_{j \in J} B_j'
                \right) \\
                &= \bigcup_{\substack{i \in I \\ j \in J}} (B_i \intersection B_j')
            \end{align*}
            But \(B_i \intersection B_j'\) can be expressed as the union of subsets in \(\mathcal{B}\).
            \item Consider the family \(\{A_i\}_{i\in I}\) with \(A_i \in \mathcal{T}\).
            We have
            \[
                \bigcup_{i \in I} A_i \in \mathcal{T}
            \]
        \end{enumerate}
    }
\end{snippetproof}

\end{document}
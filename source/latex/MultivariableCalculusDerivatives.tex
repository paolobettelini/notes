\documentclass[preview]{standalone}

\usepackage{amsmath}
\usepackage{amssymb}
\usepackage{stellar}
\usepackage{definitions}
\usepackage{bettelini}

\begin{document}

\id{multivariablecalculus-derivatives}
\genpage

\section{Derivatives}

\begin{snippettheorem}{multivariable-chainrule}{Multivariable chainrule}
    Let \(x=g(t)\) and \(y=h(t)\) be differentiable \function[functions] of \(t\)
    and \(z=f(x,y)\) be a differentiable \function of \(x\) and \(y\).
    Then \(z=f(x(t), y(t))\) is a differentiable \function of \(t\) and
    \[
        \frac{dz}{dt} =
        \frac{\partial z}{\partial x} \cdot \frac{dx}{dt} +
        \frac{\partial z}{\partial y} \cdot \frac{dy}{dt}
    \]
    where the ordinary derivatives are evaluated at \(t\) and the partial
    derivatives are evaluated at \((x,y)\).
\end{snippettheorem}

\begin{snippetdefinition}{gradient-definition}{Gradient}
    Let \(E \subseteq {\realnumbers}^n\) be an open set and
    let \(f\colon E \fromto \realnumbers\) be a \function where
    all partial derivatives \(\frac{\partial f}{\partial x_1}\),
    \(\cdots\), \(\frac{\partial f}{\partial x_n}\) of \(f\)
    at a point \(a\in E\) exist.
    The \textit{gradient} of \(f\) at the point \(a\)
    is defined as
    \[
        \nabla f(\vec{a}) \triangleq
        \left(\begin{array}{c}
        \frac{\partial f}{\partial x_1}(\vec{a}) \\
        \vdots \\
        \frac{\partial f}{\partial x_n}(\vec{a})
        \end{array}\right)
    \]
\end{snippetdefinition}

\begin{snippetproposition}{tangent-plane-formula}{Tangent plane formula}
    The equation of the plane that is tangent to \(f(x,y)\)
    at the point \((x_0, y_0)\) is given by
    \[
        z = f(x_0, y_0) + f_x(x_0, y_0)(x-x_0) + f_y(x_0, y_0)(y-y_0)
    \]
\end{snippetproposition}

% Implicit function theorem

\plain{Lagrange multipliers
are technique for finding maximum or minimum 
values of a function subject to some constraint,
like finding the highest point on a mountain subject
to the fact you can only walk along a trail.}

\begin{snippettheorem}{lagrange-multiplier-theorem}{Lagrange Multiplier Theorem}
    Consider an open set $E \subseteq {\realnumbers}^n$, two functions $f, g: E \rightarrow \realnumbers$
    of class $C^1(E)$ and let $c \in \realnumbers$ be a constant.
    If the function $f$ restricted to the level set $\{\vec{x} \in E: g(\mathbf{x})=c\}$
    achieves a local extreme value at a point a and additionally
    $\nabla g(\vec{a}) \neq \vec{0}$ then there must be a scalar
    number $\lambda \in \realnumbers$ such that
    $\nabla f(\vec{a})=\lambda \nabla g(\vec{a})$. 
    The number $\lambda$ is called the \textit{Lagrange multiplier}.
\end{snippettheorem}


% https://www.youtube.com/watch?v=bk9IKHS5KbY

\section{Other}

\begin{snippetproposition}{partial-derivatives-determinant}{}
    Let \(A \in \matrices_{n\times n}(\realnumbers)\) such that \(A_{i,j} = a_{i,j}\).
    Then,
    \[
        \frac{\partial}{\partial a_{i,j}} \det A = \cofactor_{i,j}
    \]
\end{snippetproposition}

\begin{snippetproof}{partial-derivatives-determinant-proof}{partial-derivatives-determinant}{}
    We can use the determinant expansion
    \begin{align*}
        \det A &= \sum_{k=1}^n a_{i,k} \cofactor_{i,k} \\
        \frac{\partial}{\partial a_{i,j}} \det A
        &= \frac{\partial}{\partial a_{i,j}} \sum_{k=1}^n a_{i,k} \cofactor_{i,k} \\
        &= \sum_{k=1}^n \frac{\partial}{\partial a_{i,j}} a_{i,k} \cofactor_{i,k} \\
        &= \sum_{k=1}^n \cofactor_{i,k} \frac{\partial a_{i,k}}{\partial a_{i,j}} \\
        &= \sum_{k=1}^n \cofactor_{i,k} \delta_{k,j} \\
        &= \cofactor_{i,j}
    \end{align*}
\end{snippetproof}

\end{document}
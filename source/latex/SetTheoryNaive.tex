\documentclass[preview]{standalone}

\usepackage{amsmath}
\usepackage{amssymb}
\usepackage{stellar}
\usepackage{definitions}

\begin{document}

\id{settheory-naive}
\genpage

\section{Naive set theory}

\begin{snippetdefinition}{naive-set-definition}{Naive set}
    A \textit{set} is a collection of well-defined objects, called \textit{elements},
    such that given an object \(x\) and a set \(A\), we must be able to tell whether \(x\) is in \(A\),
    denoted \(x\in A\), or \(x\notin A\) otherwise.
    Given two sets \(A\) and \(B\), we say that \(A=B\) if every element of \(A\) is in \(B\)
    and every element of \(B\) is in \(A\).
    \[
        A=B \iff \forall a\in A, a\in B \land \forall b\in B, b\in A
    \]
\end{snippetdefinition}

\plain{Determining whether an object is in a set or not is often a difficult task. For example,
if we consider the set of prime numbers it is often computationally hard to tell whether the element is in the set.}

\plain{To describe a finite set it is possible to list every element inside it.}

\begin{snippet}{set-description-by-list}
    \[
        A = \left\{ 3, 2, 5, \sqrt{3}, -\frac{1}{3}, \picircle \right\}
    \]
\end{snippet}

\plain{Note that order and repetitions are not relevant.}

\begin{snippet}{set-repetition-irrelevant}
    \[
        \{ 3, 3, 3, 2 \} = \{ 2, 3 \}
    \]
\end{snippet}

\begin{snippet}{set-description-by-property}
    A set can be described by a common property \(P(a)\) for some object \(a\)
    that all its object share
    \[
        A = \left\{ n \suchthat P(n) \right\}
    \]
\end{snippet}

\section{Russel's paradox}

\begin{snippettheorem}{russel-paradox-theorem}{Russel's paradox}
    An axiomatic system where a \set can contain itself is not consistent. 
\end{snippettheorem}

\begin{snippetproof}{russel-paradox-theorem-proof}{russel-paradox-theorem}{Russel's paradox}
    Let \(R\) be the \set defined as:
    \[
    R = \{ x \suchthat x \notin x \},
    \]
    that is, the \set of all \set[sets] that do not contain themselves as a member.\\
    Now, we ask the question: does \(R\) contain itself as a member? 
    \begin{itemize}
        \item If \( R \in R \), then by the definition of \( R \), it must be true that \( R \notin R \). This is a contradiction \lightning.
        \item If \( R \notin R \), then by the definition of \( R \), it must be true that \( R \in R \). This is also a contradiction \lightning.
    \end{itemize}
    Thus, \( R \) cannot consistently exist, leading to the paradox.
\end{snippetproof}

\end{document}
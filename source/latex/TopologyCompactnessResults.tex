\documentclass[preview]{standalone}

\usepackage{amsmath}
\usepackage{amssymb}
\usepackage{stellar}
\usepackage{definitions}
\usepackage{bettelini}

\begin{document}

\id{topology-compactness-results}
\genpage

\section{Compactness results}

\begin{snippettheorem}{unit-interval-is-compact}{[0,1] is compact}
    The unit interval \([0,1]\) is compact
    with respect to the Euclidean topology on \(\realnumbers\).
\end{snippettheorem}

\begin{snippetproof}{unit-interval-is-compact-proof}{unit-interval-is-compact}{[0,1] is compact}
    Let \(\mathcal{A}\) be an open cover of \([0,1]\).
    Define the subset \(S \subseteq [0, 1]\) as:
    \[
        S = \{t \in [0, 1] \suchthat [0,t] \text{ is covered by finitely many sets in } \mathcal{A}\}
    \]
    Note that \(S \neq \emptyset\) since \(0 \in S\).
    Indeed, \(0 \in [0,1]\), so there exists \(A \in \mathcal{A}\) with \(0 \in A\),
    thus \([0,0] \subseteq A\).
    
    Since \(S\) is bounded above by \(1\), let \(b = \sup S\). Clearly \(b \leq 1\).
    
    \textbf{Step 1:} We show that \(b \in S\).
    Since \(b \in [0,1]\), there exists \(A \in \mathcal{A}\) such that \(b \in A\).
    Being \(A\) open, there exists \(\delta > 0\) such that \((b-\delta, b] \subseteq A\).
    By definition of supremum, there exists \(t \in S\) with \(t > b - \delta\).
    Since \(t \in S\), there exists a finite family \(\mathcal{F} \subseteq \mathcal{A}\)
    that covers \([0,t]\).
    Then \(\mathcal{F} \cup \{A\}\) is a finite family that covers \([0,b]\),
    since \([0,b] \subseteq [0,t] \cup (b-\delta, b]\). Thus \(b \in S\).
    
    \textbf{Step 2:} We show that \(b = 1\).
    Suppose for contradiction that \(b < 1\).
    Since \(b \in A\), there exists \(\eta > 0\) such that \([b, b+\eta] \subseteq A\)
    with \(b+\eta \le 1\).
    Combining the finite cover of \([0,b]\) with \(A\), we cover \([0, b+\eta]\).
    Hence \(b+\eta \in S\), contradicting \(b = \sup S\) \lightning.
    
    We conclude that \(b=1\). Since \(b \in S\), then \(1 \in S\),
    meaning \([0,1]\) admits a finite subcover.
\end{snippetproof}

\begin{snippetcorollary}{real-compact-closed-bounded}{Compactness in the reals}
    A subspace of \(\realnumbers\) is compact if and only if
    it is closed and bounded.
\end{snippetcorollary}

\begin{snippetproof}{real-compact-closed-bounded-proof}{real-compact-closed-bounded}{Compactness in the reals}
    \iffproof{
        Let \(A \subseteq \realnumbers\) be compact.
        
        \emph{Bounded:} Consider the open cover \(\{(-n, n) \suchthat n \in \naturalnumbers\}\).
        Since \(A\) is compact, there exists a finite subcover,
        so \(A \subseteq [-N, N]\) for some \(N > 0\).
        
        \emph{Closed:} We show \(\closure[]{A} \subseteq A\).
        If \(p \notin A\), the function \(f(x) = 1/(x-p)\) is continuous on \(A\).
        Since \(A\) is compact, \(f(A)\) is compact, hence bounded.
        This implies \(p \notin \closure[]{A}\).
    }{
        If \(A\) is closed and bounded,
        then \(A \subseteq [-a, a]\) for some \(a \geq 0\).
        Since \([-a, a]\) is homeomorphic to \([0,1]\), it is compact.
        As a closed subspace of a compact space, \(A\) is compact.
    }
\end{snippetproof}

\begin{snippetcorollary}{weierstrass-theorem}{Weierstrass extreme value theorem}
    Let \(X\) be a compact \topologicalspace and \(f \colon X \fromto \realnumbers\)
    a \topologycontinuous[continuous] map. Then \(f\) attains its maximum and minimum.
\end{snippetcorollary}

\begin{snippetproof}{weierstrass-theorem-proof}{weierstrass-theorem}{Weierstrass theorem}
    Since \(f\) is continuous and \(X\) is compact. \(f(X)\) is compact in \(\realnumbers\).
    Therefore \(f(X)\) is closed and bounded.
    
    Being bounded, \(\inf f(X)\) and \(\sup f(X)\) are real numbers.
    By definition, infimum and supremum always belong to the closure.
    Since \(f(X)\) is closed, it equals its closure,
    so \(\inf f(X)\) and \(\sup f(X)\) belong to \(f(X)\) itself,
    i.e., they are maximum and minimum.
\end{snippetproof}

\begin{snippettheorem}{compact-in-hausdorff-is-closed}{Compact in Hausdorff is closed}
    Let \(X\) be a hausdorff metric space and \(K \subseteq X\) a compact subspace.
    Then \(K\) is \closedset[closed] in \(X\).
\end{snippettheorem}

\begin{snippetproof}{compact-in-hausdorff-is-closed-proof}{compact-in-hausdorff-is-closed}{Compact in Hausdorff is closed}
    We show that \(X \difference K\) is \openset[open].
    Let \(p \in X \difference K\).
    For each \(k \in K\), since \(X\) is Hausdorff,
    there exist disjoint open neighborhoods \(U_k\) of \(p\) and \(V_k\) of \(k\).
    The family \(\{V_k\}_{k \in K}\) is an open cover of \(K\).
    Since \(K\) is compact, there exist \(k_1, \ldots, k_n\) such that
    \(K \subseteq V_{k_1} \union \cdots \union V_{k_n}\).
    Let \(U = U_{k_1} \intersection \cdots \intersection U_{k_n}\).
    Then \(U\) is an open neighborhood of \(p\) disjoint from \(K\).
    Thus \(X \difference K\) is open.
\end{snippetproof}

\end{document}

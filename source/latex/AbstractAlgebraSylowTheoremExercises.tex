\documentclass[preview]{standalone}

\usepackage{amsmath}
\usepackage{amssymb}
\usepackage{stellar}
\usepackage{definitions}

\begin{document}

\id{sylow-theorems-exercises}
\genpage

\section{Exercises}

\begin{snippetexercise}{sylow-ex1}{Group is not simple}
    Show that a \group of order \(105\) is not simple. % TODOURGENT \simplegrp
\end{snippetexercise}

\begin{snippetsolution}{sylow-ex1-sol}{Group is not simple}
    Let \(G\) be a \group where \(|G| = 3\cdot5\cdot 7\).
    The \sylowpsubgroup[Sylow \(p\)-subgroups] of order \(p=3,5,7\)
    have \primen order, thus are \cyclicgroup[cyclic].
    We want to find \normalsubgrptext[normal subgroups] that are not trivial.
    Let \(n_p\) be the amount of \sylowpsubgroup[Sylow \(p\)-subgroups] of \(G\).
    We have that \(n_3 \equiv 1 \pmod{3}\) and \(n_3 \divides 5\cdot 7\).
    The divisors of \(5\cdot 7\) are \(1,5,7,35\). Among them, both \(1\)
    and \(7\) are congruent to \(1\) modulo \(3\).
    Then, \(n_5 \equiv 1 \pmod{5}\) and \(n_5 \divides 3\cdot 7\).
    The divisors of \(3\cdot 7\) are \(1,3,7,21\).
    Among them, both \(1\) and \(21\) are congruent to \(1\) modulo \(5\).
    For \(n_7\), the divisors are \(1,3,5,15\) and \(1\) and \(15\)
    are congruent to \(1\) modulo \(7\).
    No luck thrice. If at least one among \(n_3, n_5, n_7 = 1\)
    we would be done as we would have found a \normalsubgrptext[normal] \sylowpsubgroup.
    If none of them is \(1\), we would have \(n_3 =7\), \(n_5 = 21\)
    and \(n_7 = 15\). Consider the \sylowpsubgroup[\(3\)-Sylows]: they have trivial intersection
    as their order is \primen. Each one of them contains \(3-1\) elements of period \(3\)
    that are not contained in the other \sylowpsubgroup[\(3\)-Sylows].
    In \(G\) we find \(7\cdot 2 = 14\) elements of period \(3\). Likewise,
    for the other \primen[primes] we find \(21 \cdot (5-1) = 84\) elements of period \(5\)
    and \(15\cdot (7-1) = 90\) elements of period \(7\).
    However, \(14+84+90>105\) and thus at least one among \(n_3, n_5, n_7\)
    is \(1\). We can also note that only \(n_5 + n_7 > 105\)
    and thus at least one between \(n_5\) and \(n_7\) is \(1\).
    Furthermore, if \(n_7 = 15\), meaning the \sylowpsubgroup[\(7\)-Sylows]
    are not \normalsubgrptext[normal], then \(n_5 \neq 21\) (which we had already noted)
    and thus \(n_5 = 1\) (with \(4\) elements of period \(5\)), and we would have
    \(14+4+90 > 105\), meaning \(n_3 = 1\) \lightning. In conclusion,
    if the \sylowpsubgroup[\(7\)-Sylows] are not \normalsubgrptext[normal],
    then the \sylowpsubgroup[\(3\)-Sylows] and the \sylowpsubgroup[\(5\)-Sylows]
    are not \normalsubgrptext[normal].
\end{snippetsolution}

\begin{snippetexercise}{sylow-ex2}{Group is not simple}
    Show that a \group of order \(56\) is not simple. % TODOURGENT \simplegrp
\end{snippetexercise}

\begin{snippetsolution}{sylow-ex2-sol}{Group is not simple}
    We have \(|G| = 56 = 2^3 \cdot 7\).
    Let \(n_p\) be the amount of \sylowpsubgroup[Sylow \(p\)-subgroups] of \(G\).
    We have \(n_2 \equiv 1 \pmod{2}\) and \(n_2 \divides 7\),
    \(n_7 \equiv 1 \pmod{7}\) and \(n_7 \divides 2^3\).
    The counting argument seems to not be effective.
    Tne \sylowpsubgroup[\(7\)-Sylows] are \cyclicgroup[cyclic] of \primen order,
    meaning they have trivial intersection.
    However, the \sylowpsubgroup[\(2\)-Sylows] have order \(8\),
    and we cannot say much about their intersection.
    If \(n_7 \neq 1\) we have \(8\) \sylowpsubgroup[\(7\)-Sylows] and each of them contains
    \(6\) elements of period \(7\) not contained in the other
    \sylowpsubgroup[\(7\)-Sylows]. Thus, we have \(8\cdot 6 = 48\) elements of period
    \(7\). Since the space left is \(56 - 48 = 8\) elements to form
    the \sylowpsubgroup[\(2\)-Sylows]. However, each \sylowpsubgroup[\(2\)-Sylow]
    has \(8\) elements and thus we have room only for \(1\) of them (which is thus \normalsubgrptext[normal]).
\end{snippetsolution}

\begin{snippetexercise}{sylow-ex3}{Show that group is cyclic}
    Let \(G\) be a \group of order \(3\cdot5\cdot17\).
    Show that \(G\) is \cyclicgroup[cyclic].
\end{snippetexercise}

\begin{snippetsolution}{sylow-ex3-sol}{Show that group is cyclic}
    The \sylowpsubgroup[Sylow \(p\)-subgroups] for \(p=3,5,17\)
    have prime order and are thus \cyclicgroup[cyclic].
    If those are normal \normalsubgrptext, then \(G\) is a direct product
    of them (of finite \cyclicgroup[cyclic groups] with \coprime orders) and thus
    \(G\) is \cyclicgroup[cyclic].
    Let \(n_p\) be the amount of \sylowpsubgroup[Sylow \(p\)-subgroups] of \(G\).
    We have \(n_3 \equiv 1 \pmod{3}\) and \(n_3 \divides 5\cdot 17\).
    The divisors of \(5\cdot 17\) are \(1,5,17,5\cdot 17\). The ones that are congruent to one
    modulo three are \(1,5\cdot 17\).
    Likewise, for \(n_5\) we have \(1, 3\cdot 17\)
    and for \(n_{17}\) we only have \(1\).
    Thus, we have only one \sylowpsubgroup[\(17\)-Sylow] \(N\) which is thus \normalsubgrptext
    in \(G\).
    We now want to show that at least one of the other two is \normalsubgrptext using a counting argument.
    If \(n_p \neq 1\) then \(n_3 = 5\cdot 17\) and each of the \sylowpsubgroup[\(3\)-Sylows]
    contains \(3-1=2\) elements of period \(3\) which are not contained in the other \sylowpsubgroup[\(3\)-Sylows].
    We have \(2\cdot 5 \cdot 17\) elements of period \(3\). Likewise, if \(n_5 \neq 1\),
    we have \((5-1) \cdot 3 \cdot 17\) elements of period \(5\). We would thus have
    at least \(2\cdot5\cdot 17+ 4\cdot3\cdot17 > |G|\) elements in \(G\).
    Thus, at least one in \(n_5\) and \(n_5\) is \(1\).
    If \(H\) is a \sylowpsubgroup[\(3\)-Sylow] and \(K\)
    is a \sylowpsubgroup[\(5\)-Sylow], at least one of them is \normalsubgrptext in \(G\).
    This, \(HK\) is a \subgroup of \(G\) with order
    \[
        \frac{|H|\cdot|K|}{|H \intersection K|} = \frac{3\cdot5}{1} = 15
    \]
    But then \(HK\) is \cyclicgroup[cyclic] as it is the product
    of two distinct \primen[primes] of which the biggest is not congruent to \(1\)
    modulo the smaller one. Let \(L = HK\). Since \(L\) is \cyclicgroup[cyclic] and thus \abeliangroup[abelian],
    \(H \normalsubgrp L\), and thus \(\groupnormalizer_G(H) \supseteq L\).
    But then \(n_3 = |G \,:\, \groupnormalizer_G(H)|\) divides \(|G \,:\, L| = 17\)
    as \(L\) contains the normalizer. Thus, \(n_5\) can only be \(1, 17\).
    Only \(1\) is congruent to \(1\) modulo \(3\).
    Likewise for \(K\), \(n_5 = 1\).
\end{snippetsolution}

\begin{snippetexercise}{sylow-ex4}{Classification of groups of order \(30\)}
    Classify the \group[groups] of order \(30\).
\end{snippetexercise}

\begin{snippetsolution}{sylow-ex4-sol}{Classification of groups of order \(30\)}
    We have \(30 = 2\cdot3\cdot5\). The \sylowpsubgroup[Sylow \(p\)-subgroups] for \(p=2,3,5\)
    have prime order and are thus \cyclicgroup[cyclic].
    Let \(n_p\) be the amount of \sylowpsubgroup[Sylow \(p\)-subgroups] of \(G\).
    We have \(n_2 \equiv 1 \pmod{2}\) and \(n_2 \divides 3\cdot 5\).
    The possible divisors are \(1,3,5,15\).
    For \(n_3\) we have \(n_2 \equiv 1 \pmod{3}\) and \(n_3 \divides 2\cdot 5\),
    the candidates are \(1,6\). 
    If \(n_2=n_3=n_5 = 1\), then the \sylowpsubgroup[Sylow \(p\)-subgroups]
    are \normalsubgrptext and \(G\) is \cyclicgroup[cyclic].
    If \(n_3 \neq 1\) we have \((3-1)\cdot 10 = 20\) elements of period \(3\).
    If \(n_5 \neq 1\) we have \((5-1)\cdot 6 = 24\) elements of period \(5\).
    Since \(20+24 > 30\) at least one of \(n_3\) and \(n_5\) is \(1\).
    If \(H\) is a \sylowpsubgroup[\(3\)-Sylow] and \(K\) is a \sylowpsubgroup[\(5\)-Sylow],
    at least one of them is \normalsubgrptext in \(G\).
    Thus, \(HK\) is a \subgroup of \(G\) of order \(2\)
    \[
        15 = \frac{|H| \cdot |K|}{|H \intersection K|} = 15
    \]
    which is \cyclicgroup[cyclic] since it has order \(15\). Furthermore,
    \(|G \,:\, HK| = 2\) and thus \(HK \normalsubgrp G\).
    Let \(N = HK\) and let \(L\) be a \sylowpsubgroup[\(2\)-Sylow].
    \(L\) is \cyclicgroup[cyclic]. Now,
    \[
        |LN| = \frac{|L| \cdot |N|}{|L \intersection N|} = \frac{2\cdot15}{1} = 30
    \]
    meaning \(G = LN\) with \(N \normalsubgrp G\) and \(L \intersection N = 1\).
    Thus, \(G\) is the semidirect product \(L \ltimes N\).
    In order to study the rare cases we would need to study the homomorphisms \(\varphi\)
    from \(L\) into \(\text{Aut}(N)\), meaning the homomorphisms
    \[
        \varphi \colon C_2 \fromto \text{Aut}(C_{15})
    \]
    Such \(\varphi\) is found by assigning the image of a generator \(C_2\).
    There is only one choice as \(eulertotient(2) = 1\) and thus \(C_2 = \gengrp{y}\).
    We thus choose the image among the elements of \(\text{Aut}(C_{15})\) that have period
    dividing \(2\). If \(C_{15} = \gengrp{x}\), the endomorphisms of \(C_{15}\)
    are given by assigning the image of \(x\) among the elements of \(C_{15}\)
    of period which divides \(15\) (all of them).
    In order to make it an automorphism, the image needs to have period \(15\),
    meaning of the form \(x^t\) where \(t\) and \(15\) are \coprime.
    We thus have \(\varphi(15) = 8\) automorphisms of \(C_{15}\):
    \begin{itemize}
        \item \(\theta_1 \colon x \fromto x^1\): (identity). Has period \(1\) which divides \(2\);
        \item \(\theta_2 \colon x \fromto x^2\): Its period does not divide \(2\) as \(\theta_2^2\colon x\fromto x^2 \fromto x^4 \neq x\);
        \item \(\theta_4 \colon x \fromto x^4\): Its period does divide \(2\) as \(\theta_4^2\colon x\fromto x^4 \fromto x^{16} = x\);
        \item \(\theta_7 \colon x \fromto x^7\): Its period does not divide \(2\) as \(\theta_7^2\colon x\fromto x^7 \fromto x^{49} \neq x\);
        \item \(\theta_8 \colon x \fromto x^8\): Its period does not divide \(2\) as \(\theta_8^2\colon x\fromto x^8 \fromto x^{64} \neq x\);
        \item \(\theta_{11} \colon x \fromto x^{11}\): Its period does divide \(2\) as \(\theta_{11}^2\colon x\fromto x^{11} \fromto x^{121} = x\);
        \item \(\theta_{13} \colon x \fromto x^{13}\): Its period does not divide \(2\) as \(\theta_{13}^2\colon x\fromto x^{-2} \fromto x^{4} \neq x\);
        \item \(\theta_{13} \colon x \fromto x^{13}\): Its period does not divide \(2\) as \(\theta_{13}^2\colon x\fromto x^{-2} \fromto x^{4} \neq x\);
        \item \(\theta_{14} \colon x \fromto x^{14}\): Its period does divide \(2\) as \(\theta_{14}^2\colon x\fromto x^{-1} \fromto x = x\);
    \end{itemize}
    We found \(4\) automorphisms of \(C_{15}\) of period which divides \(2\) and, thus,
    we can define \(4\) homomorphisms from \(C_2\) into \(\text{Aut}(C_{15})\) which will
    induce \(4\) possible semidirect products \(C_2 \ltimes_\varphi C_{15}\).
    If \(\varphi \colon y \fromto \theta_1\), we have in reality the direct product \(C_{2} \times C_{15}\)
    which is the \cyclicgroup of order \(30\).
    In the other cases we have semidirect products which are not direct, thus not \abeliangroup[abelian].
    However, we do not know whether the latter are isomorphic.
    The homomorphisms \(\varphi\) determined how \(C_2\) acts on \(C_{15}\) by conjugation.
    We really only need to know how it acts on the generator. We have the various cases
    \[
        x^y = x^4 \qquad x^y = x^{11} \qquad x^y = x^{-1}
    \]
    The idea is to try and count the amount of \sylowpsubgroup[\(2\)-Sylows], meaning
    \(|G \,:\, \groupnormalizer_G(C_2)|\). If \(z \in \groupnormalizer_G(C_2)\), then
    \(C_2^z = C_2\). But
    \[
        C_2 = \{1,y\} \qquad \text{and} \qquad C_2^z = \{1^z, y^z\} = \{1, y^z\}
    \]
    and this in order for it to be \(C_2 = C_2^z\) it must that that \(y^z = y\),
    meaning \(z \in C_G(y)\). Now, the generic element of \(G = C_2 \ltimes C_{15}\)
    is writen as \(y^r x^s\). This elements centralized \(y\) when
    \(y (y^r x^s) = (y^r x^s) y\), meaning \ifandonlyif \(y^{r+1}x^s = y^rx^sy\)
    which simplifies to \(yx^s = x^sy\) (the powers of \(x\) which commute with \(y\)).
    This happens \ifandonlyif \(x^s = {(x^s)}^y\).
    Consider the various cases:
    \begin{itemize}
        \item \(x^y = x^4\), from which \({(x^s)}^y = x^{4s}\) and \(x^{4s} = x^5\)
        \ifandonlyif \(4s \equiv 5 \pmod{15}\) meaning \(35\) is multiple of \(15\),
        meaning \(s\) is multiple of \(5\). Thus, \(C_G(y) = \{y^r x^s \suchthat 5 \divides s\}\).
        We have two choices for \(r\) (0, and 1) and three choices for \(s\) (0, 5, 10).
        Thus,
        \[
            |C_G(y)| = 2\cdot 3 = 6
        \]
        and \[
            |G \,:\, \groupnormalizer_G(C_2)| = 30/6 = 5 = n_2
        \]
        \item \(x^y = x^11\), from which \(11s \equiv s \pmod{15}\) mening \(10s\) is a multiple of \(15\)
        and thus \(s\) is a multiple of \(3\). We have two choices for \(r\) and five for \(s\).
        Thus, \(n_2 = \frac{30}{2\cdot 5} = 3\).
        \item \(x^y = x^{-1}\), from which \(s\) is multiple of \(15\). We have two choices for \(r\)
        and one choice for \(s\). Thus, \(n_2 = \frac{30}{2\cdot 1} = 15\)
    \end{itemize}
    These \(3\) \group[groups] are thus not isomorphic.
\end{snippetsolution}

\end{document}
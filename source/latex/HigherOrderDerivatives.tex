\documentclass[preview]{standalone}

\usepackage{amsmath}
\usepackage{amssymb}
\usepackage{stellar}
\usepackage{definitions}
\usepackage{bettelini}

\begin{document}

\id{higher-order-derivatives}
\genpage

\section{Higher-order derivatives}

\begin{snippet}{second-directional-derivative-construction}
    Let \(\Omega \subseteq \realnumbers^n\) be open
    and consider \(f \colon \Omega \fromto \realnumbers\).
    Fix a unit \vector \(v\) and let \(a \in \Omega\).
    Suppose the directional derivative \(D_v f\) exists in a \neighborhood \(B_r(a)\).
    Then \(D_v f\) is a \function defined on an open \set containing \(a\),
    so we can take its directional derivative in turn.
    Fix another unit \vector \(w \in \realnumbers^n\) and construct
    \[
        D_w(D_v f)(a) \triangleq D^2_{w,v} f(a)
    \]
    if it exists.
\end{snippet}

\begin{snippetdefinition}{second-partial-derivative-definition}{Second partial derivative}
    When \(w\) and \(v\) are coordinate axis directions,
    these directional derivatives are called \emph{second partial derivatives}, denoted
    \[
        \frac{\partial^2 f}{\partial x_i \partial x_j}(a)
        \triangleq \frac{\partial}{\partial x_j} \left(\frac{\partial f}{\partial x_i}(a)\right)
        \triangleq f_{x_j x_i}(a)
    \]
    If \(i \neq j\), these are called \emph{mixed partial derivatives}.
    If \(i = j\), we use the notation \(\frac{\partial^2 f}{\partial x_i^2}(a)\).
\end{snippetdefinition}

\plain{We can construct n² partial derivatives and form a square matrix.}

\begin{snippetdefinition}{hessian-matrix-definition}{Hessian matrix}
    If all second partial derivatives of \(f\) exist at \(a\),
    we define the \emph{Hessian matrix}:
    \[
        \hessian f(a) \triangleq \begin{pmatrix}
            \frac{\partial^2 f}{\partial x_1^2}(a) &
            \frac{\partial^2 f}{\partial x_1 \partial x_2}(a) & \cdots &
            \frac{\partial^2 f}{\partial x_1 \partial x_n}(a) \\
            \vdots & \ddots & & \vdots \\
            \frac{\partial^2 f}{\partial x_n \partial x_1}(a) &
            \frac{\partial^2 f}{\partial x_n \partial x_2}(a) & \cdots &
            \frac{\partial^2 f}{\partial x_n^2}(a)
        \end{pmatrix}
    \]
\end{snippetdefinition}

\begin{snippetexample}{hessian-polynomial-example}{}
    Let \(f(x, y, z) = 3x^2 y + xz^3\). Then
    \[
        \hessian f(x, y, z) = \begin{pmatrix}
            6y & 6x & 3z^2 \\
            6x & 0 & 0 \\
            3z^2 & 0 & 6xz
        \end{pmatrix}
    \]
\end{snippetexample}

\begin{snippetexample}{mixed-partial-nonexistence-example}{}
    Let \(f(x, y) = x^2 \sqrt[3]{y}\).
    For \(y \neq 0\), the \function is \differentiable by the total differential theorem.
    At a point \((x_0, 0)\), \(f\) is identically zero, so \(\frac{\partial f}{\partial x}(x_0, 0) = 0\).
    For the other partial:
    \[
        \frac{\partial f}{\partial y}(x_0, 0)
        = \lim_{y \to 0} \frac{x_0^2}{y^{2/3}}
        = \begin{cases}
            \infty & x_0 \neq 0 \\
            0 & x_0 = 0
        \end{cases}
    \]
    At the origin, \(f\) is \differentiable (verified directly).
    However, \(f_{xy}(0, 0)\) does not exist because
    \(\lim_{x \to 0} \frac{f_y(x, 0) - f_y(0, 0)}{x}\) does not exist.
    Meanwhile, \(f_{yx}(0, 0) = 0\).
\end{snippetexample}

\plain{The existence of one mixed partial derivative does not imply
the existence of the other with swapped order.}

\begin{snippetexample}{mixed-partials-different-example}{}
    Let
    \[
        f(x, y) = \begin{cases}
            \frac{xy^3}{x^2 + y^2} & (x, y) \neq (0, 0) \\
            0 & (x, y) = (0, 0)
        \end{cases}
    \]
    The \function is \realcontinuous at the origin (verified in polar coordinates).
    For \((x, y) \neq (0, 0)\):
    \[
        f_x(x, y) = \frac{y^3(y^2 - x^2)}{(x^2 + y^2)^2}, \quad
        f_y(x, y) = \frac{xy^2(3x^2 + y^2)}{(x^2 + y^2)^2}
    \]
    Also \(\gradient f(0, 0) = (0, 0)\) since \(f\) is constant along axes.
    Computing:
    \[
        f_{xy}(0, 0) = \lim_{x \to 0} \frac{f_y(x, 0) - f_y(0, 0)}{x} = 0
    \]
    But
    \[
        f_{yx}(0, 0) = \lim_{y \to 0} \frac{f_x(0, y) - f_x(0, 0)}{y}
        = \lim_{y \to 0} \frac{y^5/y^4}{y} = 1
    \]
    So both mixed partials exist but \(f_{xy}(0, 0) \neq f_{yx}(0, 0)\).
\end{snippetexample}

\begin{snippettheorem}{schwarz-theorem}{Schwarz's theorem}
    Let \(f \colon \Omega \subseteq \realnumbers^n \fromto \realnumbers\),
    with \(\Omega\) open, \(a \in \Omega\), and \(v, w\) two unit \vector[vectors] in \(\realnumbers^n\).
    Suppose there exists \(B_r(a)\) where \(D^2_{v,w} f\) and \(D^2_{w,v} f\) exist
    and are \realcontinuous at \(a\).
    Then \(D^2_{v,w} f(a) = D^2_{w,v} f(a)\).
\end{snippettheorem}

\begin{snippetcorollary}{hessian-symmetry-corollary}{}
    If the partial derivatives exist and are \realcontinuous in \(\Omega\), then
    \[
        \frac{\partial^2 f}{\partial x_i \partial x_j} = \frac{\partial^2 f}{\partial x_j \partial x_i}
    \]
    i.e., \(\hessian f(x)\) is symmetric.
\end{snippetcorollary}

\begin{snippetdefinition}{c2-class-definition}{Class \(\continuityclass^2\)}
    Let \(f \colon \Omega \subseteq \realnumbers^n \fromto \realnumbers\).
    Then \(f \in \continuityclass^2(\Omega)\) if all first and second partial derivatives of \(f\)
    exist and are \realcontinuous in \(\Omega\).
\end{snippetdefinition}

\section{Twice differentiable functions}

\begin{snippetdefinition}{twice-differentiable-definition}{Twice differentiable}
    Let \(f \colon \Omega \subseteq \realnumbers^n \fromto \realnumbers^m\).
    We say \(f\) is \emph{twice differentiable} at \(a \in \Omega\) if
    \(f\) is \differentiable in \(\Omega\) and \(\text{d}f \colon \Omega \fromto \mathcal{L}(\realnumbers^n, \realnumbers^m)\)
    is \differentiable at \(a\).
\end{snippetdefinition}

\begin{snippetproposition}{twice-differentiable-properties}{}
    \begin{enumerate}
        \item \(f\) is twice \differentiable at \(a\) \ifandonlyif \(\partial f / \partial x_i\) is \differentiable at \(a\) for all \(i\).
        \item If \(f\) is twice \differentiable at \(a\), then all second partial derivatives exist and we can swap the order of differentiation.
        \item If \(f\) is twice \differentiable at \(a\), then:
        \begin{enumerate}
            \item \(D_v f\) is \differentiable at \(a\) for every \(v\);
            \item \(D^2_{v,w} f(a) = D^2_{w,v} f(a)\);
            \item \(D^2_{v,w} f(a) = \sum_{i,j=1}^n \frac{\partial^2 f}{\partial x_i \partial x_j}(a) v_i w_j\).
        \end{enumerate}
        \item If all first partial derivatives exist and are \realcontinuous in a \neighborhood of \(a\),
        and all second partial derivatives exist and are \realcontinuous at \(a\),
        then \(f\) is twice \differentiable at \(a\).
        \item If \(f \in \continuityclass^2(\Omega)\), then \(f\) is twice \differentiable in \(\Omega\).
    \end{enumerate}
\end{snippetproposition}

\begin{snippet}{second-differential-bilinear-form}
    If \(f \colon \Omega \fromto \realnumbers^m\) is twice \differentiable at \(a\),
    then \(\text{d}^2 f(a) \in \mathcal{L}(\realnumbers^n, \mathcal{L}(\realnumbers^n, \realnumbers^m))\).
    This means \(\text{d}^2 f(a)\) is bilinear; we view it as \(\text{d}^2 f(a) \colon \realnumbers^n \times \realnumbers^n \fromto \realnumbers^m\):
    \[
        \text{d}^2 f(a)(h, k) = \sum_{i,j=1}^n \frac{\partial^2 f}{\partial x_i \partial x_j}(a) h_i k_j
        = h^\transpose \hessian f(a) \, k
    \]
    For real-valued \(f\):
    \[
        \text{d}f(a)(h) = \gradient f(a) \cdot h, \quad
        \text{d}^2 f(a)(h, k) = h^\transpose \hessian f(a) \, k
    \]
    When \(h = k\), we get a quadratic form:
    \[
        \text{d}^2 f(a)(h) = \sum_{i,j=1}^n \frac{\partial^2 f}{\partial x_i \partial x_j}(a) h_i h_j
    \]
    and \(\frac{\text{d}^2 f(a)(h)}{\|h\|^2}\) is bounded as \(h \to 0\).
\end{snippet}

\section{\(k\)-th order derivatives}

\begin{snippetdefinition}{ck-class-definition}{Class \(\continuityclass^k\)}
    We say \(f \in \continuityclass^k(\Omega)\) if every partial derivative of order \(\leq k\)
    exists and is \realcontinuous in \(\Omega\).
\end{snippetdefinition}

\plain{These classes are nested: \(\continuityclass^{k+1}(\Omega) \subseteq \continuityclass^k(\Omega)\).}

\begin{snippetdefinition}{k-th-differential-definition}{\(k\)-th differential}
    The \(k\)-th differential
    \[
        \text{d}^k f(a) \colon \underbrace{\realnumbers^n \times \cdots \times \realnumbers^n}_{k \text{ times}} \fromto \realnumbers^m
    \]
    is a multilinear map defined by
    \[
        \text{d}^k f(a)(h^{(1)}, \ldots, h^{(k)})
        = \sum_{i_1, \ldots, i_k = 1}^n \frac{\partial^k f}{\partial x_{i_1} \cdots \partial x_{i_k}}(a)
        h_{i_1}^{(1)} \cdots h_{i_k}^{(k)}
    \]
\end{snippetdefinition}

\begin{snippet}{k-th-differential-same-direction}
    When all \(h^{(i)} = h\), by Schwarz's theorem:
    \[
        \text{d}^k f(a)(h) = \sum_{\substack{j_1, \ldots, j_n \geq 0 \\ j_1 + \cdots + j_n = k}}
        \binom{k}{j_1, \ldots, j_n} \frac{\partial^k f}{\partial x_n^{j_n} \cdots \partial x_1^{j_1}}(a)
        h_1^{j_1} \cdots h_n^{j_n}
    \]
    where the multinomial coefficient is \(\binom{k}{j_1, \ldots, j_n} = \frac{k!}{j_1! \cdots j_n!}\).
    The expression \(\text{d}^k f(a)(h)\) is a homogeneous \polynomial of degree \(k\)
    in \(h_1, \ldots, h_n\), and \(\frac{\text{d}^k f(a)(h)}{\|h\|^k}\) is bounded as \(h \to 0\).
\end{snippet}

\section{Taylor's theorem}

\begin{snippettheorem}{multivariable-taylor-theorem}{Taylor's theorem (Multivariable)}
    Let \(\Omega \subseteq \realnumbers^n\) be open,
    \(a, x \in \Omega\) such that \([a, x] \subseteq \Omega\),
    and \(f \colon \Omega \fromto \realnumbers\) with \(f \in \continuityclass^k(\Omega)\).
    Then
    \[
        f(x) = \sum_{i=0}^{k-1} \frac{\text{d}^i f(a)(x - a)}{i!} + R_k(x - a)
    \]
    is the Taylor \polynomial of order \(k - 1\) of \(f\) at \(a\).
\end{snippettheorem}

\begin{snippetdefinition}{taylor-remainder-peano-definition}{Peano remainder}
    The remainder in Peano form is
    \[
        R_k(x - a) = \frac{\text{d}^k f(a)(x - a)}{k!} + o(\|x - a\|^k)
    \]
    as \(x \to a\).
\end{snippetdefinition}

\begin{snippetdefinition}{taylor-remainder-lagrange-definition}{Lagrange remainder}
    The remainder in Lagrange form is
    \[
        R_k(x - a) = \frac{\text{d}^k f(y)(x - a)}{k!}
    \]
    for some \(y \in (a, x)\).
\end{snippetdefinition}

\begin{snippet}{taylor-polynomial-uniqueness}
    The Taylor \polynomial is the unique \polynomial of degree at most \(k - 1\)
    such that \(f(x) - P(x) = o(\|x - a\|^{k-1})\) as \(x \to a\).
    
    Note: we say "order" of the Taylor \polynomial rather than "degree"
    because the \polynomial has degree at most the given value, but could be smaller.
\end{snippet}

\end{document}
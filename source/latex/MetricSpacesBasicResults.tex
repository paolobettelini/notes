\documentclass[preview]{standalone}

\usepackage{amsmath}
\usepackage{amssymb}
\usepackage{bettelini}
\usepackage{stellar}
\usepackage{definitions}

\begin{document}

\id{metric-spaces-basic-results}
\genpage

\section{Reverse triangle inequality}

\begin{snippetlemma}{metricspaces-reverse-triangle-inequality}{Reverse triangle inequality}
    Given a \metricspace \((X,d)\). Let \(x,y,z \in X\).
    Then, \[|d(x,y) - d(x,z)| \leq d(y,z)\]
\end{snippetlemma}

\begin{snippetproof}{metricspaces-reverse-triangle-inequality-proof}{metricspaces-reverse-triangle-inequality}{Reverse triangle inequality}
    The absolute values can be removed and the lemma expressed as
    \[
        d(x,y) - d(x,z) \leq d(y,z)
    \]
    and
    \[
        d(x,z) - d(x,y) \leq d(y,z)
    \]
    Both follow from the triangle inequality.
\end{snippetproof}

\section{Bounded sets}

\begin{snippetlemma}{metricspaces-boundness-equivalence}{}
    Let \((X, d)\) be a \metricspace and \(Y \subseteq X\).
    The following statements are equivalence:
    \begin{itemize}
        \item \(Y\) is contained in some open ball (bounded).
        \item \(Y\) is contained in some closed ball.
        \item The set \(\{d(y_1,y_2) \suchthat y_1, y_2 \in Y\}\)
            is a bounded subset of \(\realnumbers\).
    \end{itemize}
\end{snippetlemma}

\begin{snippetproof}{metricspaces-boundness-equivalence-proof}{metricspaces-boundness-equivalence}{}
    TODO
\end{snippetproof}

\section{Open and closed sets}

\begin{snippetproposition}{open-sets-union-and-intersection}{Open sets union and intersection}
    \msopenset[Open sets] have the following properties in any \metricspace:
    \begin{enumerate}
        \item given a family of \set[sets] \({\{A_i\}}_{i\in I}\) where \(\forall i \in I, A_i\)
        is \msopenset[open],
        \[
            \bigcup_{i \in I} A_i
        \]
        is \msopenset[open];
        \item given the  \set[sets] \(A_0, A_1, \cdots, A_n\) where \(A_i\)
        is \msopenset[open],
        \[
            \bigcap_{k=0}^n A_k
        \]
        is \msopenset[open].
    \end{enumerate}
\end{snippetproposition}

\begin{snippetproof}{open-sets-union-and-intersection-proof}{open-sets-union-and-intersection}{Open sets union and intersection}
    \begin{enumerate}
        % TODOURGENT
        \item Every element of the union is in at least one \msopenset, quindi se è interno
        ad un insieme aperto la loro union è anch'essa aperta.
        \item In the case where
        \[
            \bigcup_{i \in I} A_i = \emptyset
        \]
        the proposition is obviously true.
        Otherwise, \(\forall k, x \in \mathbb{R}\) che è aperto quindi
        esiste un raggio \(r > 0\) tale che la bolla di raggio attorno all'elemento è un sottinsieme di
        tutti gli elementi di \(A_k\). Per trovare questo raggio, prendiamo il minimo di tutti
        \[
            r = \min\{ r_1, r_2, \cdots, r_n \}
        \]
        dove \(r_n\) è un raggio tale che la bolla di raggio \(r_n\) attorno all'elemento
        è contenuta nell'insieme.
        Se il numero degli insiemi fosse infinitio, dovremmo prendere
        \[
            r = \inf\{ r_1, r_2, \cdots, r_n \}
        \]
        che potrebbe essere anche \(0\) o non esistere.
    \end{enumerate}
\end{snippetproof}

\begin{snippetproposition}{closed-sets-union-and-intersection}{Closed sets union and intersection}
    \msclosedset[Closed sets] have the following properties in any \metricspace:
    \begin{enumerate}
        \item given the \set[sets] \(A_0, A_1, \cdots, A_n\) where \(A_i\)
        is \msclosedset[closed],
        \[
            \bigcup_{k=0}^n A_k
        \]
        is \msclosedset[closed].
        \item given a family of \set[sets] \({\{A_i\}}_{i\in I}\) where \(\forall i \in I, A_i\)
        is \msclosedset[closed],
        \[
            \bigcap_{i \in I} A_i
        \]
        is \msclosedset[closed];
    \end{enumerate}
\end{snippetproposition}

\begin{snippetproposition}{set-is-open-iff-it-is-union-of-open-sets}{}
    A \set \(A\) is \msopenset[open] \ifandonlyif it is a union of \msopenset[open sets].
\end{snippetproposition}

\begin{snippetproposition}{both-open-and-closed-set-empty-reals}{Both open and closed set}
    The \set[sets] \(\emptyset\) and \(\realnumbers\) are both \msopenset[open]
    and \msopenset[closed].
\end{snippetproposition}

\end{document}

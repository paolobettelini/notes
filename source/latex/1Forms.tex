\documentclass[preview]{standalone}

\usepackage{amsmath}
\usepackage{amssymb}
\usepackage{stellar}
\usepackage{definitions}

\begin{document}

\id{1-forms}
\genpage

\section{Differential forms}

\begin{snippet}{1-form-intuition}
    Imagine a series of parallel surfaces, much like the contour
    lines on a topographic map.
    A \(1\)-form associated a measurement to every vector of a tangent space.
    This measurement is the amount of surfaces it pierces.
    Consequently, a 1-form defines a density:
    the more closely packed the surfaces,
    the higher the measurement.
    The tangent space of \(\realnumbers^n\) is
    \(\realnumbers^n\) itself, so a \(1\)-form can measure
    any vector in \(\realnumbers^n\).
\end{snippet}

\begin{snippetdefinition}{1-form-rn-definition}{\(1\)-form}
    [\{
        "generalizations": ["1-form-definition"]
    \}]
    Let \(\Omega \subseteq \realnumbers^n\) be open.
    Then, a \(1\)-form is a smooth \function \(\omega \colon \Omega \fromto (\realnumbers^n)^*\).
\end{snippetdefinition}

\begin{snippetdefinition}{1-form-definition}{\(1\)-form}
    Let \(M\) be a differentialbe manifold.
    A \(1\)-form \(\omega\) is a \function that associated to a point \(p\)
    a linear form from the tanget space into a scalar value \(\omega_p \colon T_pM \fromto \realnumbers\).
\end{snippetdefinition}

\begin{snippet}{dual-vector-space-basis}
    Consider \(f\in (\realnumbers^n)^*\).
    Any such linear form is a particular case of a linear transformation (i.e. vector viewed as a matrix).
    This means that we can write \(f(x) = \langle x,y \rangle\) for some \(y\in\realnumbers^n\)
    \[
        f(x) = \sum_i x_iy_i
    \]
    In particular, the differentials \(\text{d}x_i \in (\realnumbers^n)^*\)
    defined as \(\text dx_i \triangleq \langle x, e_i \rangle\) where \(e_i\) is the standard
    canonical vector in \(\realnumbers^n\). That is, \(\text dx_i\) is the projection of the
    \(i\)-th component of \(x\). Since \((\realnumbers^n)^*\) is itself a \vectorspace,
    the set of \(\{\text dx_1, \text dx_2, \cdots, \text dx_n\}\) is
    the standard basis of \((\realnumbers^n)^*\), and we can thus write
    \[
        f = \sum_{i=1}^n \alpha_i \text dx_i
    \] 
\end{snippet}

\begin{snippetproposition}{1-form-expansion}{}
    A \(1\)-form \(\omega\) can be expanded as
    \[
        \omega(x) = \sum_{i=1}^n \alpha_i(x) \text dx_i
    \]
    where \(\alpha_i \colon \Omega \to \realnumbers\) are smooth \function[functions]
    defined by \(\alpha_i(x) = \omega(x)(e_i)\),
    with \(\{e_i\}\) being the standard basis of \(\realnumbers^n\).
\end{snippetproposition}

\begin{snippetexample}{differential-form-example}{Differential form}
    The \function
    \[
        \omega(x,y,z) = x^2\text{d}x - e^z \text{d}y + (xyz)\text{d}z
    \]
    is a \(1\)-form from \(\realnumbers^3\) into \((\realnumbers^3)'\).
    The differential form computed at a point is a linear form
    \[
        \omega(1,1,0) = \text{d}x - \text{d}y
    \]
    This \function, applied to a vector
    \[
        \omega(1,1,0)(\alpha, \beta, \gamma) = \alpha - \beta
    \]
    This is equivalent to the object \((x^2, -e^z, xyz)\),
    which is also a vector field.
    Indeed, at the point \((1,1,0)\) the vector field has value \((1, -1, 0)\).
\end{snippetexample}

\begin{snippetproposition}{differential-form-from-differentiable-function}{Differential form from differentialble function}
    Let
    \(F \colon \Omega \to \realnumbers\) be a differentiable \function.
    Then, the differential \(\text dF\) is a linear form represented by the gradient
    \[
        \text dF =
        \left(
            \frac{\partial F}{\partial x_1},
            \cdots,
            \frac{\partial F}{\partial x_n}
        \right)
    \]
    We can thus costruct
    \begin{align*}
        \omega(x) = \sum_{i=1}^n
        \frac{\partial F}{\partial x_i}(x) \,\text{d}x_i
    \end{align*}
\end{snippetproposition}

\begin{snippetexample}{differential-form-from-differentiable-function-example}{}
    Let \(F(x,y) = x^2y + \ln x\)
    for \(\Omega = \{ x>0\}\).
    Then, \(\omega(x,y) = (2xy + 1/x)\text dx +x^2\text dy\)
    is a differential form.
\end{snippetexample}

\subsection{Exact forms}

\begin{snippetdefinition}{exact-differential-form-definition}{Exact differential form}
    A differential form \(\omega\) is said to be \emph{exact} if
    \(w = \text{d}f\) for some \(f \in \mathcal{C}^1(\Omega)\),
    called \emph{primitive} or \emph{potential}.
\end{snippetdefinition}

\plain{Analogously, we say that a vector field is conservative if it is the gradient of some function.}

\begin{snippet}{exact-differential-form-expl}
    In general, A differential \(k\)-form \(\omega\) is said to be \emph{exact}
    if it is the external derivative of a \(k-1\)-form.
    In the case of a \(1\)-form, it is exact if there exists a \(0\)-form
    (a real smooth \function) such that \(\omega = \text d f\).
    In this case, the external derivative \(d\) is represented by the gradient
    \[
        \text df = \sum \frac{\partial f}{\partial x_i} \text dx_i
    \]
\end{snippet}

\begin{snippetexample}{non-exact-differential-form-example}{Non-exact differential form}
    Consider the vector field \(F(x,y) = (x^2y, x)\). This is not a conservative field.
    Suppose that it is, and thus there exists \(f\) such that \(F(x,y) = \nabla f\).
    Then,
    \[
        \frac{\partial f}{\partial x} = x^2y, \quad \frac{\partial f}{\partial y} = x
    \]
    but its derivatives are differentiable, thus \(f \in \mathcal{C}^2\).
    But then, by Schwarz's theorem, the following must be equal
    \[
        \frac{\partial^2 f}{\partial y\partial x} = x^2, \quad
        \frac{\partial^2 f}{\partial x\partial y} = 1
    \]
    which is absurd \lightning.
\end{snippetexample}

\end{document}
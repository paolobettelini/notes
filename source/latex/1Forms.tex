\documentclass[preview]{standalone}

\usepackage{amsmath}
\usepackage{amssymb}
\usepackage{stellar}
\usepackage{definitions}

\begin{document}

\id{1-forms}
\genpage

\section{Differential forms}

\begin{snippet}{1-form-intuition}
    Imagine a series of parallel surfaces, much like the contour
    lines on a topographic map.
    A \(1\)-form associated a measurement to every vector of a tangent space.
    This measurement is the amount of surfaces it pierces.
    Consequently, a 1-form defines a density:
    the more closely packed the surfaces,
    the higher the measurement.
    The tangent space of \(\realnumbers^n\) is
    \(\realnumbers^n\) itself, so a \(1\)-form can measure
    any vector in \(\realnumbers^n\).
\end{snippet}

\end{document}
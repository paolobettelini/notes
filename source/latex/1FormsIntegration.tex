\documentclass[preview]{standalone}

\usepackage{amsmath}
\usepackage{amssymb}
\usepackage{stellar}
\usepackage{definitions}
\usepackage{bettelini}

\begin{document}

\id{1-forms-integration}
\genpage

\section{Integration over 1-forms}

\plain{We can integrate a 1-form over a curve independently of how fast we traverse it.}

\begin{snippetdefinition}{integral-over-1-form-definition}{Integral over \(1\)-form}
    Let \(\omega \colon \Omega \fromto (\realnumbers^n)^*\) be a \(1\)-form
    and \(\varphi \colon [a,b] \fromto \Omega\) a piecewise regular curve. We then define
    \[
        \int_\varphi \omega \triangleq
        \integral[a][b][\omega(\varphi(t))\cdot \varphi'(t)][t]
    \]
\end{snippetdefinition}

\begin{snippet}{integral-over-1-form-expansion}
    The integral can be written as
    \[
        \int_\varphi \omega \triangleq \integral[a][b][
            \sum_{i=1}^n \alpha_i(\varphi(t)) \varphi_i'(t)
        ][t]
    \]
    Clearly \(\alpha_i(\varphi(t))\) is an inner product.
    It's as if the differential form was acting on the derivative
    (on the vector of derivatives).
\end{snippet}

\begin{snippetexample}{1-form-integration-example1}{}
    Consider the curve
    \[
        \varphi(t) = (t,t^2), \quad t\in[0,1]
    \]
    and the \(1\)-form
    \[
        \omega(x,y) = y\text{d}x - xy\text{d}y
    \]
    We have
    \begin{align*}
        \int_\varphi \omega
        &= \integral[0][1][
            (t^2, -t^3)
            \cdot (1,2t)
        ][t]  \\
        &= \integral[0][1][
            (t^2 - 2t^4)
        ][t] = -\frac{1}{15}
    \end{align*}
\end{snippetexample}

\end{document}
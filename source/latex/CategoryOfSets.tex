\documentclass[preview]{standalone}

\usepackage{amsmath}
\usepackage{amssymb}
\usepackage{stellar}
\usepackage{definitions}
\usepackage{bettelini}

\begin{document}

\id{categorytheory-sets}
\genpage

\section{Category of sets}

\begin{snippetdefinition}{set-category-definition}{\(\mathbf{Set}\)}
    The \emph{\(\mathbf{Set}\) category} is defined as the
    \category where:
    \begin{enumerate}
        \item \(\catob(\mathbf{Set})\) is the class of all \set[sets];
        \item \(\forall A,B \in \catob(\mathbf{Set}), \cathom_\mathbf{Set}(A,B)\)
        is the \set of all \function[functions] from \(A\) to \(B\);
        \item composition is the \function composition.
    \end{enumerate}
\end{snippetdefinition}


\begin{snippetproposition}{category-set-monomorphism}{Monomorphism in \textbf{Set}}
    In the \category \textbf{Set}, a morphism is a
    \monomorphism \ifandonlyif it is an \injective \function.
\end{snippetproposition}

\begin{snippetproof}{category-set-monomorphism-proof}{category-set-monomorphism}{Monomorphisms in \textbf{Set}}
    \iffproof{
        Suppose that \(f \colon A \fromto B\) is a \monomorphism.
        and assume that \(f \colon A \fromto B\) is not \injective.
        Then, there exist \(a_1, a_2 \in A\) such that \(a_1 \neq a_2\) and \(f(a_1) = f(a_2)\).
        Let \(X = \{\ast\}\) be a singleton and define \(g_1, g_2 \colon X \fromto A\)
        by \(g_1(\ast) = a_1\) and \(g_2(\ast) = a_2\).
        Then,
        \[ f \circ g_1(\ast) = f(a_1) = f(a_2) = f \circ g_2(\ast) \]
        so \(f \circ g_1 = f \circ g_2\).
        However, since \(f\) is a \monomorphism, we must have \(g_1 = g_2\),
        which is a contradiction since \(a_1 \neq a_2\) \lightning.
        Therefore, \(f\) must be \injective.
    }{
        Assume that \(f \colon A \fromto B\) is \injective.
        Let \(X\) be any \set and let \(g_1, g_2 \colon X \fromto A\) be \function[functions] such that
        \(f \circ g_1 = f \circ g_2\).
        Then, for all \(x \in X\),
        \[ f(g_1(x)) = (f \circ g_1)(x) = (f \circ g_2)(x) = f(g_2(x)) \]
        Since \(f\) is \injective, we must have \(g_1(x) = g_2(x)\) for all \(x \in X\).
        Therefore, \(g_1 = g_2\), and \(f\) is a \monomorphism.
    }
\end{snippetproof}

\begin{snippetproposition}{category-set-epimorphism}{Epimorphism in \textbf{Set}}
    In the \category \textbf{Set}, a morphism is an
    \epimorphism \ifandonlyif it is a \surjective \function.
\end{snippetproposition}

\begin{snippetproof}{category-set-epimorphism-proof}{category-set-epimorphism}{Epimorphisms in \textbf{Set}}
    \iffproof{
        Suppose that \(f \colon A \fromto B\) is an \epimorphism,
        and assume that \(f \colon A \fromto B\) is not \surjective.
        Pick some \(b_0 \in B \difference f(A)\) and define \(g_1, g_2 \colon B \fromto \{0,1\}\)
        by
        \[ g_1(b) = 0 \quad\text{and}\quad g_2(b) = \begin{cases}
            0 & b \in f(A) \\
            1 & b = b_0 \\
            0 & \text{(or any other value) otherwise}
        \end{cases} \]
        Then, for all \(a \in A\),
        \[ g_1(f(a)) = 0 = g_2(f(a)) \]
        so \(g_1 \circ f = g_2 \circ f\).
        However, since \(f\) is an \epimorphism, we must have \(g_1 = g_2\),
        which is a contradiction since \(g_1(b_0) = 0\) and \(g_2(b_0) = 1\) \lightning.
        Therefore, \(f\) must be \surjective.
    }{
        Let \(f\colon A \fromto B\) be \surjective.
        Let \(Y\) be any \set and let \(g_1, g_2 \colon B \fromto Y\) be \function[functions] such that
        \(g_1 \circ f = g_2 \circ f\).
        Then, for all \(b \in B\),
        pick some \(a \in A\) such that \(f(a) = b\) (this is possible since \(f\) is \surjective).
        Then,
        \[ g_1(b) = g_1(f(a)) = (g_1 \circ f)(a) = (g_2 \circ f)(a) = g_2(f(a)) = g_2(b) \]
        Therefore, \(g_1 = g_2\), and \(f\) is an \epimorphism.
    }
\end{snippetproof}

\begin{snippetproposition}{category-set-isomorphism}{Isomorphism in \textbf{Set}}
    In the \category \textbf{Set}, a morphism is an
    \catisomorphism \ifandonlyif it is a \bijective \function.
\end{snippetproposition}

\begin{snippetproof}{category-set-isomorphism-proof}{category-set-isomorphism}{Isomorphisms in \textbf{Set}}
    Since \epimorphism[epimorphisms]
    and \monomorphism[monomorphisms] in \textbf{Set} are exactly
    \surjective[surjective] and \injective[injective] \function[functions] respectively,
    and a \function is \bijective[bijective] \ifandonlyif it is both
    \surjective and \injective, we have that a \function is an
    \catisomorphism \ifandonlyif it is \bijective.
\end{snippetproof}

\end{document}
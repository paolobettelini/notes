\documentclass[preview]{standalone}

\usepackage{amsmath}
\usepackage{amssymb}
\usepackage{parskip}
\usepackage{fullpage}
\usepackage{hyperref}
\usepackage{tikz}
\usepackage{stellar}
\usepackage{definitions}
\usepackage{bettelini}

\begin{document}

\id{sequences}
\genpage

\section{Definition}

\begin{snippetdefinition}{sequence-definition}{Sequence}
    A \textit{sequence} is a \function \(\varphi\colon \naturalnumbers \fromto X\)
    for some \set X.
    A sequence may be written as
    \[
        \{a_n\} \quad \quad {\{a_n\}}_{n=k}^\infty
    \]
    for some \(k\in\naturalnumbers\).
\end{snippetdefinition}

\begin{snippetdefinition}{sequence-limit-definition}{Limit of a sequence}
    Let \({\{a_n\}}\) be a \sequence and let \(\xi \in \extendedrealnumbers\).
    Then, the \emph{limit} of \({\{a_n\}}\) is defnied as
    \[
        \lim a_n = \xi \iff
        \exists N \in \naturalnumbers \suchthat
        \forall n>N, x_n \in I(\xi, \varepsilon, M)
    \]
    where
    \[
        I(\xi, \varepsilon, M) = \begin{cases}
            (\xi-\varepsilon, \xi+\varepsilon) & \xi \in \realnumbers \\
            (M, +\infty) & \xi = +\infty \\
            (-\infty, -M) & \xi = -\infty
        \end{cases}
    \]
    for all \(M>0\) and \(\varepsilon>0\).
    \begin{enumerate}
        \item the limit \emph{converges} to \(\xi\) if \(\xi\in\realnumbers\) and \(\lim a_n = \xi\);
        \item the limit \emph{diverges} to \(\xi\) if \(\xi=\pm\infty\) and \(\lim a_n = \xi\);
        \item the limit \emph{exists} if \(\exists \xi \in \extendedrealnumbers\) such that
            \(\lim a_n = \xi\);
    \end{enumerate}
    \emph{Syntax:} we write \(a_n \to \xi\) if the limit exists.
\end{snippetdefinition}

\begin{snippetlemma}{extended-real-topology-disjoint-neighborhoods}{}
    Let \(\lambda, \mu \in \extendedrealnumbers\) such that \(\lambda \neq \mu\).
    Then, there exist \neighborhood[neighborhoods] \(I\) and \(J\) of \(\lambda\) and \(\mu\) respectively such that
    \[
        I \cap J = \emptyset
    \]
\end{snippetlemma}

\begin{snippetproof}{extended-real-topology-disjoint-neighborhoods-proof}{extended-real-topology-disjoint-neighborhoods}{}
    Let \(\lambda, \mu \in \realnumbers\) such that \(\lambda < \mu\).
    Then,
    \[
        I = (\lambda - r, \lambda + r) \quad \text{and} \quad
        J = (\mu - r, \mu + r)
    \]
    are \disjoint for
    \[
        0 < r \leq \frac{\mu - \lambda}{2}
    \]
    Let \(\mu = \pm\infty\). Then,
    \[
        I = (\lambda - r, \lambda + r) \quad \text{and} \quad
        J = \begin{cases}
            (-\infty, -M) & \text{if } \mu = -\infty \\
            (M, +\infty) & \text{if } \mu = +\infty
        \end{cases}
    \]
    are \disjoint for \(r>0\) and \(M < \lambda + r\).
\end{snippetproof}

%\begin{snippetcorollary}{function-sequence-limit-equivalence}{}
%    If \(f(x)\) is a \function such that \(f(n)=a_n\)
%    \[
%        \lim_{x\to\infty}f(x)=L \implies
%        \lim_{n\to\infty}a_n=L
%    \]
%\end{snippetcorollary}

\subsection{Properties}

\begin{snippetproposition}{limit-sequence-uniqueness}{Uniqueness of sequence limit}
    Let \(\{x_n\}\) be a \sequence such that \(x_n \seqconverges \xi\)
    and \(x_n \seqconverges \mu\).
    Then, \(\xi = \mu\).
\end{snippetproposition}

\begin{snippetproof}{limit-sequence-uniqueness-proof}{limit-sequence-uniqueness}{Uniquenes of sequence limit}
    By \snippetref[this lemma][extended-real-topology-disjoint-neighborhoods] there exist
    \neighborhood[neighborhoods] \(I\) and \(J\) of \(\xi\) and \(\mu\) respectively such that
    \[
        I \cap J = \emptyset
    \]
    Thus, there exist \(N_1\) and \(N_2\) such that
    \[
        \forall n>N_1, x_n \in I \quad \text{and} \quad
        \forall n>N_2, x_n \in J
    \]
    Let \(N = \max\{N_1, N_2\}\). Then, for all \(n>N\) we have
    \(x_n \in I \cap J\) \lightning.
\end{snippetproof}

\begin{snippetproposition}{convergent-sequence-is-bounded}{}
    Let \(\{x_n\}\) be a \sequence such that \(x_n \seqconverges \xi \in \realnumbers\).
    Then, there exists \(m, M \in \realnumbers\) such that \(m \leq x_n \leq M\) for all \(n\in\naturalnumbers\).
\end{snippetproposition}

\begin{snippetproof}{convergent-sequence-is-bounded-proof}{convergent-sequence-is-bounded}{}
    Let \(\varepsilon > 0\). By the definition of limit, there exists \(N\) such that
    \[
        \forall n>N, x_n \in (\xi - \varepsilon, \xi + \varepsilon)
    \]
    This means that \(-\varepsilon \leq x_n \leq \varepsilon\) for \(n>N\). We still need to consider
    the finite number of terms \(x_1, x_2, \ldots, x_N\).
    Let \(m = \min\{x_1, x_2, \ldots, x_{N-1}\}\) and \(M = \max\{x_1, x_2, \ldots, x_{N-1}\}\).
    Then, for all \(n\) we have
    \[
        \min\{x_1, x_2, \cdots, x_{N-1}, \xi - \varepsilon\}
        \leq x_n \leq
        \max\{x_1, x_2, \cdots, x_{N-1}, \xi + \varepsilon\}
    \]
\end{snippetproof}

%%%%%%%%%%%%%%%%%%%%%%%%%%%%%%%%%%%%%%

\begin{snippet}{sequences-properties}
    if \(\{a_n\}\) and \(\{b_n\}\) are convergent sequences, then

    \[
        \lim_{n\to\infty} (a_n \pm b_n) = \lim_{n\to\infty} a_n \pm
        \lim_{n\to\infty} b_n
    \]
    \[
        \lim_{n\to\infty} ca_n = c \lim_{n\to\infty} a_n
    \]
    \[
        \lim_{n\to\infty} (a_n b_n) =
        \left(\lim_{n\to\infty} a_n\right)
        \left(\lim_{n\to\infty} b_n\right)
    \]
    % 2 omitted
\end{snippet}

\section{Squeeze Theorem}

\begin{snippettheorem}{sequence-squeeze-theorem}{Sequence Squeeze Theorem}
    If \(a_n \leq c_n \leq b_n\) for sufficiently large \(n>N\) for some \(N\)
    and \(\lim_{n\to\infty}a_n =\lim_{n\to\infty}b_n=L\)
    then \(\lim_{n\to\infty} c_n =L\)
\end{snippettheorem}

\section{Absolute Value}

\begin{snippet}{abs-value-sequence-limit}
Note the following
\[
    -|a_n| \leq a_n \leq |a_n|
\]
Then, if we assume
\[
    \lim_{n\to\infty} (-|a_n|) = - \lim_{n\to\infty} |a_n| =0 
\]
by the Squeeze Theorem we get
\[
    \lim_{n\to\infty} a_n =0
\]

We conclude that if \(\lim_{n\to\infty} |a_n|=0\) then
\(\lim_{n\to\infty} a_n=0\).
\end{snippet}

\section{Exponential sequence} % right name?

\begin{snippettheorem}{exponential-sequence-convergence}{Exponential Sequence Convergence}
    The sequence \({\{a^n\}}_{n=0}^\infty\) converges iff \(-1<r\leq 1\)
    \[
        \lim_{n\to\infty} a^n = \begin{cases}
            0 & \text{if } -1 < a < 1 \\
            1 & \text{if } a=1
        \end{cases}
    \]
\end{snippettheorem}

\section{Convergence of even and odd indexes}

\begin{snippettheorem}{convergence-of-even-and-odd-indexes-theorem}{}
    If \(\lim_{n\to\infty}a_{2n}=L\) and \(\lim_{n\to\infty}a_{2n+1}=L\)
    then \(\{a_n\}\) is convergent and \(\lim_{n\to\infty}a_n=L\).
\end{snippettheorem}

\begin{snippetproof}{convergence-of-even-and-odd-indexes-proof}{convergence-of-even-and-odd-indexes}{}
    Let \(\varepsilon>0\). \\
    Since \(\lim_{n\to\infty}a_{2n}=L\) there exists an \(N_1\) such that
    if \(n>N_1\) then
    \[
        |a_{2n}-L|<\varepsilon
    \]
    Also, since \(\lim_{n\to\infty}a_{2n+1}=L\) there exists an \(N_2\) such that
    if \(n>N_2\) then
    \[
        |a_{2n+1}-L|<\varepsilon
    \]
    Let \(N=\max(2N_1, 2N_2+1)\) and let \(n>N\).
    Then either \(a_n=a_{2k}\) for some \(k>N_1\) or \(a_n=a_{2k+1}\)
    for some \(k>N_2\). Either way we have
    \[
        |a_n-L|<\varepsilon
    \]
    which satisfies the convergence of \(\{a_n\}\).
\end{snippetproof}

\section{Properties of a sequence}

% TODOURGENT: exist upper and lower bound in the IMAGE, is bounded
\begin{snippetdefinition}{increasing-sequence-definition}{Increasing Sequence}
    A \sequence is \textit{increasing} if \(\forall n, a_n<a_{n+1}\).
\end{snippetdefinition}

\begin{snippetdefinition}{decreasing-sequence-definition}{Decreasing Sequence}
    A \sequence is \textit{decreasing} if \(\forall n, a_n>a_{n+1}\).
\end{snippetdefinition}

\begin{snippetdefinition}{monotonic-sequence-definition}{Monotonic Sequence}
    If \(\{a_n\}\) is increasing or decreasing it is also called \textit{monotonic}.
\end{snippetdefinition}

\begin{snippetdefinition}{bounded-below-sequence-definition}{Bounded Below Sequence}
    If there exists a number \(m\) such that \(\forall n, m \leq a_n\)
    the \sequence is \textit{bounded below} by a lower bound.
\end{snippetdefinition}

\begin{snippetdefinition}{bounded-above-sequence-definition}{Bounded Above Sequence}
    If there exists a number \(m\) such that \(\forall n, m \geq a_n\)
    the \sequence is \textit{bounded above} by an upper bound.
\end{snippetdefinition}

\subsection{Bounded}

\begin{snippetdefinition}{bounded-sequence-definition}{Bounded Sequence}
    If the \sequence is both bounded above and bounded below it is
    said \textit{bounded}.
\end{snippetdefinition}

\section{Convergence of bounded and monotonic sequences}

\begin{snippettheorem}{bounded-and-monotonic-sequences-convergence}{Convergence of bounded and monotonic sequences}
    If \(a_n\) is bounded and monotonic then it is convergent.
\end{snippettheorem}

\begin{snippettheorem}{sequence-convergence-of-abs}{Convergence of absolute value}
    Le \(a_n\) be a \sequence.
    \[ \lim_{n\to\infty} |a_n| = 0 \implies \lim_{n\to\infty} a_n = 0 \]
\end{snippettheorem}

% Calculus I ok
% Calculus II
    % Series and Sequences

\end{document}
\documentclass[preview]{standalone}

\usepackage{amsmath}
\usepackage{amssymb}
\usepackage{stellar}

\hypersetup{
    colorlinks=true,
    linkcolor=black,
    urlcolor=blue,
    pdftitle={Stellar},
    pdfpagemode=FullScreen,
}

\begin{document}

\title{Stellar}
\id{cesare-beccaria-origine-pene}
\genpage

\section{Origine delle pene}

\begin{snippet}{origini-delle-pene-parte1}
    In assenza di leggi, ogni individuo ha una libertà infinita. Ma lo sforzo che uno deve fare
    per proteggergli dagli altri è troppo, e quindi le leggi sono un compromesso per potersi godere una libertà
    limitata. Le leggi sono dei vincoli, delle limitazioni di libertà, che cercano di massimizzare
    il rapproto fra la libertà garantita e la libertà persa.
    Per esempio, è più vantaggio perdere il diritto di uccidere, in cambio della garanzia
    di non essere ucciso, piuttosto che poter uccidere e avere la possibilità di essere uccisi.
    Questo è il motivo della nascita della società, ossia un fattore di convenienza.
    Il teorico fondatore del \textit{contratto sociale} è Russeau.
    
    Una pena per essere deterrente, deve essere sempre più svantaggiosa dell'azione commessa.
    La pena è quindi sensibili (sensismo).
\end{snippet}

\begin{snippetdefinition}{sensismo-definition}{Sensismo}
    Il \textit{sensismo} è un termine che designa quelle
    dottrine filosofiche che riportano ogni contenuto e la stessa azione del
    conoscere al sentire, ossia al processo di trasformazione delle sensazioni,
    escludendo in tal modo dalla conoscenza tutto quello che non sia riportabile ai sensi.
    A volte viene usato come suo sinonimo sensualismo, che però trova definizione diversa.
\end{snippetdefinition}

\begin{snippet}{origini-delle-pene-parte2}
    La prospettiva di una pena non ci deve abbandonare mai, come una piccola forza costante;
    nella nostra testa ci deve essere sempre quell'idea dell'associazione fra azione (delitto) e pena.
    \\\\
    Non è sufficiente insegnare all'uomo i valori morali, è necessaria la pena deterrente.
    Questa è una critica alla Chiesa e in particolare ai gesuiti. 
\end{snippet}

\end{document}
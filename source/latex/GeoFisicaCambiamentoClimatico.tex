\documentclass[preview]{standalone}

\usepackage{amsmath}
\usepackage{amssymb}
\usepackage{tikz}
\usepackage{stellar}
\usepackage{bettelini}

\hypersetup{
    colorlinks=true,
    linkcolor=black,
    urlcolor=blue,
    pdftitle={Assets},
    pdfpagemode=FullScreen,
}

\begin{document}

\title{Cambiamento climatoco}
\id{geofisica-cambiamento-climatico}
\genpage

\section{Effetto serra}

\begin{snippetdefinition}{effetto-serra}{Effetto serra}
    L'\textit{effetto serra} è un fenomeno per il quale alcuni gas
    atmosferici, detti appunti \textit{gas serra},
    permettono l'ingresso della radiazione solare proveniente dalla stella, 
    mentre ostacolano l'uscita della radiazione infrarossa riemessa dalla
    superficie del corpo celeste.
\end{snippetdefinition}

\includesnpt{illustrazione-effetto-serra}

\begin{snippet}{effetto-serra-expl1}
    L'effetto serra è un fenomeno per il quale alcuni gas atmosferici, detti appunto gas serra,
    permettono l'ingresso della radiazione solare proveniente dal sole, mentre ostacolano
    l'uscita della radiazione infrarossa riemessa dalla superficie terrestre.
    I raggi del sole attraversano l'atmosfera e riscaldano la superficie terrestre; tuttavia, una
    parte significativa della radiazione infrarossa risultante viene dispersa nello spazio (circa il
    30\%). In condizioni naturali, circa il 70\% della radiazione infrarossa viene assorbito dal
    vapore acqueo e dagli altri gas serra presenti nell'atmosfera. Tali gas agiscono come
    pannelli di vetro in una serra, intrappolando il calore e riflettendo di nuovo sulla superficie
    terrestre. Quando aumenta la concentrazione dei gas serra in atmosfera, cresce la quantità
    di calore intrappolato e riflesso. Gli oceani si riscaldano e liberano più vapore acqueo, che
    a sua volta può incrementare l'effetto serra.
    È da notare che l'effetto serra svolge un ruolo cruciale nel mantenimento delle condizioni
    climatiche idonee per la vita sulla Terra. In presenza di tale fenomeno, la temperatura media
    globale risulta approssimativamente di +15°C. Senza l'azione dell'effetto serra, la
    temperatura media sulla Terra si ridurrebbe drasticamente, attestandosi a -18°C, rendendo
    le condizioni ambientali del nostro pianeta incompatibili con la vita.
    Alcuni gas serra naturali presenti nell'atmosfera sono: CO2 (anidride carbonica), CH\({}_4\)
    (metano), H\({}_2\)O (vapore acqueo) e O\({}_3\) (ozono). La diversa influenza dei gas serra sul
    surriscaldamento del pianeta dipende dal loro potenziale di riscaldamento e dalla loro
    permanenza nell'atmosfera.
\end{snippet}

\section{Emissioni di CO2}

\begin{snippet}{emissioni-co2-cause}
    La formazione di CO\({}_2\) e altri gas serra avviene attraverso:
    \begin{itemize}
        \item \textbf{decomposizione delle sostanze organiche};
        \item \textbf{combustione (ossidazione) di carbone, dei combustili fossili, petrolio e
        gas:} produce anidride carbonica e ossido di azoto;
        \item \textbf{deforestazione:} gli alberi aiutano a regolare il clima
        assorbendo CO\({}_2\) dall'atmosfera.
        \item \textbf{allevamento di bestiame:} bovini e ovini producono grandi quantità
        di metano (principale responsabile del riscaldamento globale) durante il processo di
        digestione;
        \item \textbf{fertilizzanti e pesticidi in agricoltura}.
    \end{itemize}
    Invece, la sua concentrazione può diminuire con:
    \begin{itemize}
        \item \textbf{temperatura:} che influenza la solubilità dell'aria;
        \item \textbf{fotosintesi:} causa un considerevole abbassamento di CO\({}_2\).
    \end{itemize} 
\end{snippet}

\section{Aumento della temperatura}

\begin{snippet}{aumento-temperatura-expl1}
    La rivoluzione industriale ha segnato in
    modo significativo la nostra vita e il
    pianeta stesso. Per gli esseri umani è
    migliorato il tenore di vita grazie ad
    evoluzioni tecnologiche mentre con l'avanzare di essa è cominciato il
    degrado della
    natura. Questo è causato dal sempre in aumento bisogno di materie prime sia per
    consumo giornaliero (petrolio → benzina) o per la creazione di beni più durevoli.
\end{snippet}

\includesnpt{nasa-temperature}

\section{Fonti}

\includesnpt{cambiamento-climatico-fonti}

% https://www.europarl.europa.eu/topics/it/article/20230316STO77629/cambiamento-climatico-gas-a-effetto-serra-che-causano-il-riscaldamento-globale
% https://climate.nasa.gov/vital-signs/global-temperature/?intent=121

\end{document}
\documentclass[preview]{standalone}

\usepackage{amsmath}
\usepackage{amssymb}
\usepackage{stellar}
\usepackage{definitions}

\begin{document}

\id{multivariable-constrained-extrema-exercises}
\genpage

\section{Exercises}

\begin{snippetexercise}{contrained-extrema-ex1}{}
    Determine the points of absolute maximum and minimum of
    \[
        f(x,y) = xe^{-(x^2 + y^2)}
    \]
    constrained to
    \[
        g(x,y) = x^2 + (y-1)^2 - 1 = 0
    \]
\end{snippetexercise}

\begin{snippetsolution}{contrained-extrema-ex1-sol}{}
    We can parametrize the constraint \(g\) as
    \((\cos \theta, \sin \theta + 1)\) for \(\theta \in [0, 2\pi)\).
    By substituting we get
    \begin{align*}
        f(\theta) &= \cos\theta e^{-(\cos^2 \theta + \sin^2 \theta + 1 + 2\sin \theta)} \\
        &= \cos\theta e^{-2-2\sin\theta} \\
        &= e^{-2} \cos\theta e^{-2\sin\theta}
    \end{align*}
    We can thus optimize \(h(\theta) = \cos\theta e^{-2\sin\theta}\).
    The derivate is given by
    \begin{align*}
        h'(\theta) &= -\sin\theta e^{-2\sin\theta} - 2\cos^2 \theta e^{-2\sin\theta} \\
        &= -e^{-2\sin\theta} \left(
            2\cos^2\theta + \sin\theta
        \right)
    \end{align*}
    The derivative is null when
    \begin{align*}
        \sin\theta + 2(1 - \sin^2\theta) &= 0 \\
        2 \sin^2\theta - \sin\theta - 2 &= 0 \\
        \sin\theta &= \frac{1-\sqrt{17}}{4}
    \end{align*}
    and
    \[
        \cos\theta = \pm \sqrt{\frac{\sqrt{17} -1}{8}}
    \]
    Hence, the critical points are
    \[
        \left(
            \sqrt{\frac{\sqrt{17} -1}{8}},
            1 + \frac{1-\sqrt{17}}{4}
        \right), \quad
        \left(
            -\sqrt{\frac{\sqrt{17} -1}{8}},
            1 + \frac{1-\sqrt{17}}{4}
        \right)
    \]
    By computing \(f\) at these points we get
    \[
        f\left(
            \sqrt{\frac{\sqrt{17} -1}{8}},
            1 + \frac{1-\sqrt{17}}{4}
        \right) >
        f\left(
            -\sqrt{\frac{\sqrt{17} -1}{8}},
            1 + \frac{1-\sqrt{17}}{4}
        \right)
    \]
    and thus one is the absolute maximum and the other the absolute minimum
    by the Weierstrass theorem.
\end{snippetsolution}

\includesnpt[pattern=7;colors=mixed]{juggling}

\begin{snippetexercise}{contrained-extrema-ex2}{}
    Determine the points of absolute maximum and minimum of
    \[
        f(x,y) = x^3 - y^3
    \]
    constrained to
    \[
        g(x,y) = x^4 + y^4 - 1 = 0
    \]
\end{snippetexercise}

\begin{snippetsolution}{contrained-extrema-ex2-sol}{}
    \todo
\end{snippetsolution}

\begin{snippetexercise}{contrained-extrema-ex3}{}
    Determine the points of absolute maximum and minimum of
    \[
        f(x,y) = xe^{-x^2 - y^2}
    \]
    constrained to
    \[
        g(x,y) = x^2 - y^2 - 1 = 0
    \]
\end{snippetexercise}

\begin{snippetsolution}{contrained-extrema-ex3-sol}{}
    \todo
\end{snippetsolution}

\begin{snippetexercise}{contrained-extrema-ex4}{}
    Let \(A \in \matrices_{n \times n}(\realnumbers)\) be symmetric
    and \(f \colon \realnumbers^n \fromto \realnumbers\) defined by
    \[
        f(x) = x^t Ax
    \]
    Determine the points of absolute maximum and minimum of \(f\)
    constrained to
    \[
        g(x,y) = ||x||^2 - 1 = 0
    \]
    Determine the nature of all the critical poins.
\end{snippetexercise}

\begin{snippetsolution}{contrained-extrema-ex4-sol}{}
    \todo
\end{snippetsolution}

\begin{snippetexercise}{contrained-extrema-ex5}{}
    Let \(A \in \matrices_{n \times n}(\realnumbers)\) be symmetric
    and \(f \colon \realnumbers^n \fromto \realnumbers\) defined by
    \[
        f(x) = ||x||^2
    \]
    Determine the points of absolute maximum and minimum of \(f\)
    constrained to
    \[
        g(x,y) = x^t A x - 1 = 0
    \]
    Determine the nature of all the critical poins.
\end{snippetexercise}

\begin{snippetsolution}{contrained-extrema-ex5-sol}{}
    \todo
\end{snippetsolution}

\begin{snippetexercise}{contrained-extrema-ex6}{}
    Determine the minimum of the \function
    \[
        f(x_1, \cdots, x_n) = \sum_{i=1}^n x_i, \quad n \geq 2, x_i \geq 0
    \]
    restricted to
    \[
        g(x_1, \cdots, x_n) = \left(\prod_{i=1}^n x_i\right) - 1 = 0
    \]
    Deduce the following relationship
    between the geometric mean and the arithmetic mean
    \[
        \sqrt[n]{\prod_{i=1}^n a_i} \leq \frac1n \sum_{i=1}^n a_i, \quad
        a_i \geq 0
    \]
\end{snippetexercise}

\begin{snippetsolution}{contrained-extrema-ex6-sol}{}
    \todo
\end{snippetsolution}

\begin{snippetexercise}{contrained-extrema-ex7}{}
    Determine the minimum of the \function
    \[
        f(x_1, \cdots, x_n) = \sum_{i=1}^n \frac{1}{x_i},
        \quad n \geq 2, x_i > 0
    \]
    restricted to
    \[
        g(x_1, \cdots, x_n) = \left(\prod_{i=1}^n x_i\right) - 1 = 0
    \]
    Deduce the following relationship
    between the geometric mean and the arithmetic mean
    \[
        \frac{n}{\sum_{i=1}^n \frac{1}{a_i}} \leq \sqrt[n]{\prod_{i=1}^n a_i}, \quad
        a_i \geq 0
    \]
\end{snippetexercise}

\begin{snippetsolution}{contrained-extrema-ex7-sol}{}
    \todo
\end{snippetsolution}

\begin{snippetexercise}{contrained-extrema-ex8}{}
    Prove that among all triangles of a fixed perimeter,
    the equilateral triangle has maximum area.
\end{snippetexercise}

\begin{snippetsolution}{contrained-extrema-ex8-sol}{}
    \todo
\end{snippetsolution}

\begin{snippetexercise}{contrained-extrema-ex9}{}
    Determine the extrema of the \function
    \[
        f(x,y) = |y-1|(2-y-x^2)
    \]
    in the region
    \[
        D = \{(x,y) \suchthat 0 \leq y \leq 2 - x^2 - y^2\}
    \]
\end{snippetexercise}

\begin{snippetsolution}{contrained-extrema-ex9-sol}{}
    \todo
\end{snippetsolution}

\begin{snippetexercise}{contrained-extrema-ex10}{}
    Determine the extrema of the \function
    \[
        f(x,y,z) = z
    \]
    constrained to
    \[
        \begin{cases}
            x^2 + y^2 - z^2 = 1 \\
            x + y + 2z = 0
        \end{cases}
    \]
\end{snippetexercise}

\begin{snippetsolution}{contrained-extrema-ex10-sol}{}
    \todo
\end{snippetsolution}

\end{document}
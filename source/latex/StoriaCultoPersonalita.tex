\documentclass[preview]{standalone}

\usepackage{amsmath}
\usepackage{amssymb}
\usepackage{bettelini}
\usepackage{stellar}

\hypersetup{
    colorlinks=true,
    linkcolor=black,
    urlcolor=blue,
    pdftitle={Stellar},
    pdfpagemode=FullScreen,
}

\begin{document}

\title{Stellar}
\id{storia-culto-personalita}
\genpage

\section{Culto della personalità}

\begin{snippetdefinition}{realismo-socialista-definizione}{Realismo Socialista}
    Il \textit{realismo socialista} è un movimento artistico e culturale nato nell'Unione Sovietica nel 1934 e poi allargatosi a tutti i paesi socialisti del centro ed est Europa. La funzione principale era quella di avvicinare l'espressione artistica alla cultura delle classi proletarie e celebrare il progresso socialista.
\end{snippetdefinition}

\begin{snippetdefinition}{culto-personalita-definizione}{Culto della personalità}
    Il \textit{culto della personalità} è una forma di idolatria sociale che
    generalmente si configura nell'assoluta devozione a un leader,
    solitamente politico o religioso, attraverso l'esaltazione del pensiero e 
    delle capacità, tanto da attribuirgli doti di infallibilità.
\end{snippetdefinition}

\begin{snippet}{0c6f8469-2ce4-4624-beef-ff605b7bf56b}
    Questa espressione indica la tendenza dei regimi totalitari
    a celebrare in modo acritico la persona, l'azione e il pensiero del proprio leader.

    \paragraph*{Il significato dell'espressione}
    L'espressione \quotes{culto della personalita} fu coniata in URSS negli anni '50 del Novecento, poco
    dopo la morte di Stalin, per descrivere in modo fortemente critico la totale sottomissione del
    partito comunista e del governo del paese alla volonta del dittatore sovietico. In un significato
    piu generale, l'espressione indica la tendenza dei regimi totalitari a celebrare in modo acritico
    la persona, l'azione e il pensiero dei propri leader.

    \paragraph*{Venerare il capo per legittimare il suo potere assoluto}
    L'esaltazione operata dalla propaganda nei confronti di Mussolini, Hitler e, appunto, Stalin, non
    riguardava solo le loro qualita di uomini politici e di governo, ma anche la loro abilita in ogni
    settore, la loro benevolenza nei confronti del popolo, la loro prestanza fisica. Questi dittatori
    venivano quindi proposti come modelli di uomini perfetti, che le masse dovevano venerare.
    Questa esaltazione era finalizzata, soprattutto, a legittimare il potere assoluto del leader. Nei
    regimi totalitari il ruolo del capo non derivava infatti da una investitura ereditaria (come quella
    dei re delle regine di oggi), o addirittura divina (come nelle monarchie antiche), ma era fondato
    solo sulle sue formidabili abilita e qualita, che dovevano quindi essere costantemente celebrate.
    L'esaltazione serviva anche a contrastare la credibilita di eventuali critiche denunce.

    \paragraph*{L'efficacia della propaganda}
    In ogni epoca re e imperatori sono stati celebrati in vita con sculture, ritratti, composizioni
    musicali o poetiche. Cio che differenzia il culto della personalita da altre forme di adulazione
    del potere e il fatto che, nei totalitarismi del Novecento (e anche nelle attuali dittature), lo
    sviluppo dei mezzi di comunicazione di massa e di controllo governativo sulla cultura, sul tempo
    libero e l'istruzione hanno permesso alla propaganda di raggiungere e influenzare le persone
    in ogni ambito della vita quotidiana.
\end{snippet}

\includesnpt[src=/snippet/static/stalin.jpg|width=50\%]{centered-img}

\end{document}
\documentclass[preview]{standalone}

\usepackage{amsmath}
\usepackage{amssymb}
\usepackage{parskip}
\usepackage{fullpage}
\usepackage{hyperref}
\usepackage{stellar}
\usepackage{bettelini}

\hypersetup{
    colorlinks=true,
    linkcolor=black,
    urlcolor=blue,
    pdftitle={Differential Equations},
    pdfpagemode=FullScreen,
}

\begin{document}

\title{Differential Equations Definitions}
\id{diffeq-definitions}
\genpage

\section{Definition}

\begin{snippet}{diffeq-definition}
Differential equations are equations where the solution is a function
or a set of functions.
\end{snippet}

\subsection{Order}

\begin{snippet}{diffeq-order-definition}
The \textit{order} of a differential equation is the largest derivative present in the
differential equation
\end{snippet}

\subsection{Types}

\begin{snippet}{diffeq-types}
\textit{Ordinary differential equations} are equations with only
ordinary derivatives in them, whilst \textit{partial differential equations}
have partial derivatives in them.
\end{snippet}

\subsection{Linearity}

\begin{snippetdefinition}{diffeq-linear-definition}{Linear Differential Equation}
    A differential equation is said to be \textit{linear} if it can be written as
    \[
        \sum_n a_n(t) \frac{d^n}{dt^n}y(t)=g(t)
    \]
    where there are no products of the function \(y(t)\) and its derivatives,
    \(y(t)\) or its derivative do not occur to any power other than the first power
    and \(y(t)\) or any of its derivative are composed with another function.
\end{snippetdefinition}

\end{document}
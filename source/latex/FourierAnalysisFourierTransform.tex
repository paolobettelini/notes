\documentclass[preview]{standalone}

\usepackage{amsmath}
\usepackage{amssymb}
\usepackage{parskip}
\usepackage{fullpage}
\usepackage{hyperref}
\usepackage{bettelini}
\usepackage{stellar}
\usepackage{definitions}

\begin{document}

\id{fourieranalysis-fourier-transform}
\genpage

\section{Fourier Transform}

\begin{snippet}{fourier-analysis-fourier-transform-expl}
    The next animation is distance of the center of mass from the origin as the frequency changes.
    The key concept is that when the frequency matches the period, the center of mass is unusually further from the origin.
    Go back to the last section and look at the blue dot moving far away from the origin when the frequency is the right one
    (reset the sine or cosine wave so that it is clearer). When the frequency is a component of the function, this distance peaks.
    If the function is composed of multiple frequencies, the same distance will peak multiple times, and more it peaks,
    the more the frequency is present in the original function.
    The function that represents the center of mass as the frequency changes is called Fourier transform.
    Actually, not quite, the Fourier transform is defined with the integral over \(\realnumbers\)
    and is not divided by the time span. The absolute value (distance from the origin) of this function
    is the amount of all continuous frequencies present in the original signal.
    \[
        \hat{f}(\xi)=\integral[-\infty][\infty][f(t)\,e^{-2\pi it\xi}][t]
    \]
\end{snippet}

\includesnpt{fourier-lib}
\includesnpt{fourier-fourier-transform}

\end{document}

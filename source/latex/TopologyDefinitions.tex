\documentclass[preview]{standalone}

\usepackage{amsmath}
\usepackage{amssymb}
\usepackage{bettelini}
\usepackage{stellar}
\usepackage{definitions}

\begin{document}

\id{topology-definitions}
\genpage

\section{Definitions}

\newcommand\ts{{(X, \mathcal{T})}}

\begin{snippetdefinition}{topological-space-definition}{Topological space}
    A \textit{topological space} is a tuple \(\ts\)
    where \(X\) is a non-empty \set and \(\mathcal{T} \subseteq \powerset(X)\)
    is called the \textit{topology on \(X\)}
    with the following conditions:
    \begin{enumerate}
        \item \(\emptyset, X \in \mathcal{T}\);
        \item \(\forall U, V \in \mathcal{T}, U \intersection V \in \mathcal{T}\);
        \item \({\{U_i\}}_{i \in I} \subseteq \mathcal{T} \implies \bigcup_{i \in I} U_i \in \mathcal{T}\).
    \end{enumerate}
    The \set[sets] in \(\mathcal{T}\) are called the \textit{open sets} of \(X\).
\end{snippetdefinition}

\plain{Note that the intersection of infinitely many open sets need not be open.}

\plain{The elements of the topology are precisely what defines which sets are open or not.
We usually say that a set is open if it does not contain its boundary, but a topological
space does not have any inherent meaning of ordering.
If we consider the topological space induced by (0;1),
the reason why 0 and 1 are boundary points is given by the classical ordering on the reals.
The elements of the topology precisely define what the boundaries of the topological space are
by defining the open sets.}

\begin{snippetdefinition}{coarse-topology-definition}{Coarse topology}
    Let \((X, \tau_1)\) and \((X, \tau_2)\) be two \topologicalspace
    on the same \set.
    We say that \(\tau_1\) is \emph{coarser} \(\tau_2\)
    (or that \(\tau_2\) is \emph{finer} than \(\tau_1\))
    if \(\tau_1 \subseteq \tau_2\).
\end{snippetdefinition}

\subsection{Classification of points}

\begin{snippetdefinition}{neighbourhood-definition}{Neighbourhood}
    Let \(\ts\) be a \topologicalspace
    and \(p\) be a point in \(X\). A \textit{neighborhood} of \(p\) is a subset \(V\) of \(X\)
    that includes an \openset \(U\) containing \(p\)
    \[ p\in U \subseteq V \subseteq X \]
\end{snippetdefinition}

\begin{snippetdefinition}{interior-point-definition}{Interior point}
    Let \(\ts\) be a \topologicalspace
    and \(S \subseteq X\). A point \(p \in X\) is an \textit{interior point} if \(S\) is a
    \neighborhood of \(p\).
\end{snippetdefinition}

\begin{snippetdefinition}{interior-set-definition}{Interior set}
    Let \(\ts\) be a \topologicalspace
    and \(S \subseteq X\). The \textit{interior} of \(S\), denoted \({\text{int}}_{\ts} S\),
    can be defined equivalently as
    \begin{enumerate}
        \item the largest open subset of \(X\) contained in \(S\);
        \item the union of all open sets of \(X\) contained in \(S\);
        \item the \set of all interior points of \(S\).
    \end{enumerate}
\end{snippetdefinition}

\begin{snippetdefinition}{connected-set-definition}{Connected set}
    Let \(\ts\) be a \topologicalspace
    and \(\Omega \subseteq X\). \(\Omega\) is \textit{connected} if it cannot be written as
    \(\Omega=\Omega_1 \union \Omega_2\) where \(\Omega_1 \intersection \Omega_2 = \emptyset\).
\end{snippetdefinition}

\begin{snippetdefinition}{closed-set-definition}{Closed set}
    Let \(\ts\) be a \topologicalspace.
    A \set \(S \subseteq X\) is \textit{closed} if \(X \difference S\) is \openset[open].
\end{snippetdefinition}

\begin{snippetdefinition}{closure-definition}{Closure}
    Let \(\ts\) be a \topologicalspace.
    The \textit{closure} of a \set \(S\in X\), denoted \({\text{cl}}_{\ts} S\), is defined as
    the smallest \closedset in \(X\) containing \(S\).
\end{snippetdefinition}

\plain{Interior and closure and dual via the complement (by De Morgan's laws).}

\begin{snippetdefinition}{boundary-definition}{Boundary}
    Let \(\ts\) be a \topologicalspace
    and \(S \subseteq X\). The \textit{boundary} of \(S\), denoted \({\partial}_\ts S\),
    is defined as
    \[ {\partial}_\ts S = \closure[\ts][S] \difference \interior[\ts][S] \]
\end{snippetdefinition}

\begin{snippetdefinition}{boundary-point-definition}{Boundary point}
    Let \(\ts\) be a \topologicalspace.
    A point \(p \subseteq X\) is a \emph{boundary point} if \(p \in \boundary[\ts][S]\).
\end{snippetdefinition}

% Equivalently, every neighbourhood contains points both in the exterior and interior.

\begin{snippetdefinition}{exterior-point-definition}{Exterior point}
    Let \(\ts\) be a \topologicalspace.
    A point \(p \in X\) is an exterior point of \(X\) if it is not a \boundarypoint
    or an \interiorpoint.
\end{snippetdefinition}

\begin{snippetdefinition}{accumulation-point-definition}{Accumulation point}
    Let \(\ts\) be a \topologicalspace and \(S \subseteq X\).
    A point \(p \in X\) is a \emph{limit point} or \emph{accumulation point} of \(S\)
    if for every \neighborhood \(J\) of \(p\), \(J \intersection (S \difference \{p\}) \neq \emptyset\).
\end{snippetdefinition}

\plain{This means that every neighbourhood of the point contains points of in the subset other than the central point itself.}

\begin{snippetdefinition}{derived-point-definition}{Derived set}
    Let \(\ts\) be a \topologicalspace and \(S \subseteq X\).
    The \emph{derived set} \(S'\) is the \set of all points \(x\in X\) that are
    \accumulationpoint[accumulation points] of \(S\).
\end{snippetdefinition}

\begin{snippetdefinition}{dense-set-definition}{Dense set}
    Let \(\ts\) be a \topologicalspace and \(S, B \subseteq X\).
    Then, \(S\) is \emph{dense} in \(B\)
    if \(B \subseteq \closure[\ts][S]\).
\end{snippetdefinition}

\begin{snippetproposition}{dense-set-iff-condition}{}
    Let \(\ts\) be a \topologicalspace and \(S \subseteq X\).
    Then, \(S\) is dense \ifandonlyif
    for all non-empty open \(V\) of \(X\), \(V \intersection S \neq 0\).
\end{snippetproposition}

\begin{snippetproof}{dense-set-iff-condition-proof}{dense-set-iff-condition}{}
    If \(S\) is dense we have \(\closure[\ts][S] = X\)
    which is equivalent to saying that
    \[
        X \difference \closure[\ts][S] = X \difference X = \emptyset
    \]
    we can compute the closure using the interior
    \(\interior[\ts][X \difference \closure[\ts][S]]\).
    \(V \subseteq X \difference S\) is equivalent to \(V\intersection S = \emptyset\).
    In turn, this is equivalent to saying that \(V \intersection S = \emptyset\)
    implies \(V = \emptyset\). By the contrapositive
    \(V \neq \emptyset\) implies \(V \intersection S \neq \emptyset\).
\end{snippetproof}

\end{document}

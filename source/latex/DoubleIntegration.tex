\documentclass[preview]{standalone}

\usepackage{amsmath}
\usepackage{amssymb}
\usepackage{stellar}
\usepackage{definitions}
\usepackage{bettelini}

\begin{document}

\id{double-integration}
\genpage

\section{Double integration}

\subsection{Rectangular domains}

\begin{snippetdefinition}{riemann-integral-rectangular-domain-definition}{Riemann integral on a rectangular domain}
    Let \(Q = [a,b] \times [c,d]\) be a rectangle
    and let \(f \colon Q \fromto \realnumbers\) be bounded
    on \(Q\).
    Let \(\Pi_X = \{a = x_0, < x_1 < \cdots < x_n = b\}\)
    and \(\Pi_Y = \{c = y_0, < y_1 < \cdots < y_m = d\}\) be two partitions.
    Let then \(\Pi = \Pi_X \times \Pi_Y\) be a partition of \(Q\).
    Denote by \(x_h\) and \(y_k\) the coordinates of the rectangles for the respective variables, so that
    we have \(n \times m\) small rectangles
    \[
        Q_{hk} = [x_{h-1}, x_h] \times [y_{k-1}, y_k]
    \]
    We set
    \[
        m_{hk} \triangleq \inf_{Q_{hk}} f, \quad
        M_{hk} \triangleq \sup_{Q_{hk}} f
    \]
    and
    \[
        |Q_{hk}| = (x_h - x_{h-1}) (y_k - y_{k-1})
    \]
    the area of a small rectangle.
    We then define
    \[
        s(\Pi) \triangleq \sum_{h,k} |Q_{hk}| m_{hk}, \quad
        S(\Pi) \triangleq \sum_{h,k} |Q_{hk}| M_{hk}
    \]
    We define the lower and upper integrals
    \[
        \lriint_Q f(x,y)\,\text{d}x\,\text{dy} \triangleq
        \sup_\Pi s(\Pi), \quad \uriint_Q f(x,y)\,\text{d}x\,\text{dy} \triangleq
        \inf_\Pi S(\Pi)
    \]
    If the two coincide, then we say that \(f\) is Riemann-integrable on \(Q\)
    and
    \[
        \iint_Q f(x,y)\,\text{d}x\,\text{dy} \triangleq \lriint_Q f(x,y)\,\text{d}x\,\text{dy}
        = \uriint_Q f(x,y)\,\text{d}x\,\text{dy}
    \]
\end{snippetdefinition}

\begin{snippettheorem}{bidimensional-dirichlet-function-not-riemann-integrable-theorem}{}
    The bidimensional Dirichlet function \(\indicator_{\rationalnumbers^2}\)
    is not Riemann integrable on \([0,1]^2\).
\end{snippettheorem}

\section{Arbitrary domains}

\plain{The idea is to consider a rectangle that contains the domain and extend the function to be zero outside the domain.}

\end{document}
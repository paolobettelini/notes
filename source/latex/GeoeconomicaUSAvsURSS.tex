\documentclass[preview]{standalone}

\usepackage{amsmath}
\usepackage{amssymb}
\usepackage{stellar}
\usepackage{bettelini}
\usepackage{tikz}
\usepackage{fancybox}
\usepackage{array}
\usepackage{makecell}

\usetikzlibrary{cd}

\hypersetup{
    colorlinks=true,
    linkcolor=black,
    urlcolor=blue,
    pdftitle={Stellar},
    pdfpagemode=FullScreen,
}

\begin{document}

\title{Geografia economica}
\id{geoeconomica-usa-vs-urss}
\genpage

\begin{snippetdefinition}{nato-definizione}{Nato}
    La \textit{NATO} (North Atlantic Treaty Organization) è un associazione di sicurezza composta dsa diverse nazioni.
\end{snippetdefinition}

\begin{snippet}{usa-vs-urss-illustration}
    \begin{center}
        \begin{tikzcd}
            \fbox{\textbf{USA}}               & \textit{differenze}                       & \fbox{\textbf{URSS}}               \\
            \text{Democrazia}          & \ovalbox{\textbf{Politiche}} \arrow[r] \arrow[l]  & \text{Autocrazia}           \\
            \text{Economia di mercato} & \ovalbox{\textbf{Economiche}} \arrow[r] \arrow[l] & \text{Economia pianificata} \\
            \text{NATO}                & \ovalbox{\textbf{Militari}} \arrow[r] \arrow[l]   & \text{Patto di Varsavia}   
        \end{tikzcd}
    \end{center}
    \phantom{}
\end{snippet}

% espandere commercio internazinoale
% regole vincolanti
% sistema stabile dei cambi
% 3 organi internazinoali: banca mondiale, FMI, GATT (-> WTF 1995)

\end{document}
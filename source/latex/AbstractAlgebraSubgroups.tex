\documentclass[preview]{standalone}

\usepackage{amsmath}
\usepackage{amssymb}
\usepackage{stellar}
\usepackage{definitions}
\usepackage{bettelini}

\begin{document}

\id{subgroups}
\genpage

\section{Subgroups}

\subsection{Definition}

\begin{snippetdefinition}{subgroup-definition}{Subgroups}
    Given an \algebraicstructure \(h=(H, \circ)\) and a \group \(g=(G, \circ)\), \(h\)
    is a \textit{subgroup} of \(g\)
    \[g \leq h\]
    if \(H \subseteq G\) and \(h\) is a \group where the identity element is the same.
\end{snippetdefinition}

\begin{snippetproposition}{trivial-subgroups}{Trivial subgroups}
    Given a \group \((G, \circ)\):
    \begin{enumerate}
        \item \((\{1\}, \circ) \subgroupsleq (G, \circ)\);
        \item \((G, \circ) \subgroupsleq (G, \circ)\);
    \end{enumerate}
\end{snippetproposition}

\subsection{One-Step Subgroup Test}

\begin{snippettheorem}{one-step-subgroup-test}{One-Step Subgroup Test}
    Let \((G, \circ)\) be a \group and let \(H \subseteq G\) where \(\emptyset \neq H\).\\
    Then \((H, \circ) \subgroupleq (G, \circ) \iff
    \forall a,b \in H, a \circ b^{-1} \in H\).
\end{snippettheorem}

\begin{snippetproof}{one-step-subgroup-test-proof}{one-step-subgroup-test}{One-Step Subgroup Test}
    \iffproof{
        Assume \((H, \circ) \subgroupleq (G, \circ)\).
        The properties of a \group directly infer \(\forall a,b \in H, a \circ b^{-1} \in H\).
    }{
        Assume \(\forall a,b \in H, a \circ b^{-1} \in H\):
        \begin{itemize}
            \item \textbf{identity}: let \(a=b\), then \(a\circ a^{-1} H \implies e \in H\).
            \item \textbf{inverse}: let \(k\in H\), \(a=e\) and \(b=k\).
            \(a\circ b^{-1} = e \circ k^{-1} \implies k^{-1} \in H\).
            \item \textbf{closure}: let \(m, n \in H \implies n^{-1} \in H\) and let \(a=m\) and \(b=n^{-1}\).
            \(a\circ b^{-1} = a \circ (b^{-1})^{-1}=a\circ b\). This implies \(a, b \in H\).
        \end{itemize}
    }
\end{snippetproof}

\subsection{The centralizer subgroup}

\begin{snippetdefinition}{centralizer-group}{The centralizer subgroup}
    Let \(H \subgroupleq G\) be \group[groups] and define
    \[
        \text{C}_G(H) = \{
            g \in G \suchthat \forall h \in H, gh=hg
        \}
    \]
    as the centralizer of \(H\).
    This is the set of all elements of \(G\) such that they commute with every element of \(H\).
\end{snippetdefinition}

\begin{snippettheorem}{centralizer-of-subgroup-is-subgroup}{}
    Let \(H \subgroupleq G\), then \(\text{C}_G(H) \subgroupleq G\).
\end{snippettheorem}

\begin{snippetproof}{centralizer-of-subgroup-is-subgroup-proof}{centralizer-of-subgroup-is-subgroup}{Centralizer of subgroup is subgroup}
    Suppose \(a,b \in \text{C}_G(H)\).
    We want to show \(ab^{-1} \in \text{C}_G(H)\).\\
    Note that the condition \(gh=hg \iff hg^{-1}=g^{-1}h\).\\
    Consider the expression \((ab^{-1})h = a(b^{-1}h) = ahb^{-1} = h(ab^{-1})\).
    This means that \(ab^{-1} \in \text{C}_G(H)\) and thus in \(H\).
\end{snippetproof}

\subsection{The conjugate subgroup}

\begin{snippetdefinition}{conjugate-subgroup-definition}{The conjugate subgroup}
    Let \(H \subgroupleq G\) be \group[groups]. Then, the \textit{conjugate subgroup} is defined as
    \[
        g^{-1}Hg = \{
            g^{-1}hg \suchthat h \in H
        \}
    \]
\end{snippetdefinition}

\begin{snippettheorem}{conjugate-subgroup-is-subgroup}{}
    Let \(H \subgroupleq G\), then \(g^{-1}Hg \subgroupleq G\).
\end{snippettheorem}

\begin{snippetproof}{conjugate-subgroup-is-subgroup-proof}{conjugate-subgroup-is-subgroup}{}
    Suppose \(a,b \in g^{-1}Hg\).
    We want to show \(ab^{-1} \in g^{-1}Hg\).\\
    Note that \(a = g^{-1}h_1g\) and \(b = g^{-1}h_2g\)
    for some \(h_1, h_2 \in H\). \\
    This means that \(ab^{-1}=a{(g^{-1}h_2g)}^{-1} = a(g^{-1}h_2^{-1}g)
    =g^{-1}h_1gg^{-1}h_2^{-1}g = g^{-1} (h_1h_2) g \in g^{-1}Hg \)
    because \(h_1h_2 \in H\).
\end{snippetproof}

\section{Center of a group}

\begin{snippetdefinition}{center-of-group-definition}{Center of a group}
    Let \(G\) be a \group. The center of the \group \(G\) is defined as
    \[
        \text{Z}(G) = \{
            g \in G \suchthat \forall x \in G, gx = xg
        \}
    \]

    This is the set of all elements that commute with every other element.
    The condition \(gx=xg\) is also sometimes expressed as \(gxg^{-1} = x\).
\end{snippetdefinition}

\begin{snippettheorem}{center-of-group-is-subgroup}{}
    Let \(G\) be a \group, then \(\groupcenter(G) \subgroupleq G\).
\end{snippettheorem}

\begin{snippetproof}{center-of-group-is-subgroup-proof}{center-of-group-is-subgroup}{}
    Assume \(a, b \in \groupcenter(G)\) meaning \(a = gag^{-1}\) and \(b = gag^{-1}\) for any \(g \in G\). \\
    We want to show \(ab^{-1} \in \groupcenter(G)\).
    \(ab^{-1} = (gag^{-1}){(gbg^{-1})}^{-1} = gag^{-1}gb^{-1}g^{-1}
    = g ab^{-1} g^{-1}\) which is precisely the requirement to be in \(\groupcenter(G)\).
\end{snippetproof}

% https://www.youtube.com/watch?v=m4yYeTGe-ic&list=PL22w63XsKjqwN7sHsEiy0yqkcjQfXAuVb

\end{document}
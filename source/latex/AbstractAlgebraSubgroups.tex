\documentclass[preview]{standalone}

\usepackage{amsmath}
\usepackage{amssymb}
\usepackage{stellar}
\usepackage{definitions}
\usepackage{bettelini}
\usepackage{makecell}

\begin{document}

\id{subgroups}
\genpage

\section{Subgroups}

\subsection{Definition}

\begin{snippetdefinition}{subgroup-definition}{Subgroups}
    Given an \algebraicstructure \(h=(H, \circ)\) and a \group \(g=(G, \circ)\), \(h\)
    is a \textit{subgroup} of \(g\)
    \[g \leq h\]
    if \(H \subseteq G\) and \(h\) is a \group where the identity element is the same.
\end{snippetdefinition}

\begin{snippetproposition}{trivial-subgroups}{Trivial subgroups}
    Given a \group \((G, \circ)\):
    \begin{enumerate}
        \item \((\{1\}, \circ) \subgroupleq (G, \circ)\);
        \item \((G, \circ) \subgroupleq (G, \circ)\);
    \end{enumerate}
\end{snippetproposition}

\subsection{Subgroup tests}

%TODOURGENT
%\begin{snippettheorem}{identity-in-set-equivalent-non-empty}{}
%    Let \((G, \circ) \subgroupleq (H, \circ)\) be a \group[groups].
%    Then,
%    \[
%        1 \in H \iff H \neq H
%    \]
%\end{snippettheorem}
%
%\begin{snippetproof}{identity-in-set-equivalent-non-empty-proof}{identity-in-set-equivalent-non-empty}{}
%    \todo
%\end{snippetproof}

\plain{The condition of containing the identity in a group can be replaced
by the fact that the set must be non-empty. The same does not work in a monoid.}

\begin{snippettheorem}{one-step-subgroup-test}{One-Step Subgroup Test}
    Let \((G, \circ)\) be a \group and let \(H \subseteq G\) where \(\emptyset \neq H\).\\
    Then, \((H, \circ) \subgroupleq (G, \circ) \iff
    \forall a,b \in H, a \circ b^{-1} \in H\).
\end{snippettheorem}

\begin{snippetproof}{one-step-subgroup-test-proof}{one-step-subgroup-test}{One-Step Subgroup Test}
    \iffproof{
        Assume \((H, \circ) \subgroupleq (G, \circ)\).
        The properties of a \group directly infer \(\forall a,b \in H, a \circ b^{-1} \in H\).
    }{
        Assume \(\forall a,b \in H, a \circ b^{-1} \in H\) and \(H \neq \emptyset\):
        \begin{itemize}
            \item \textit{identity}: let \(a=b\), then \(a\circ a^{-1} \in H \implies 1 \in H\).
            \item \textit{inverse}: let \(k\in H\), \(a=1\) and \(b=k\). Then,
            \(a\circ b^{-1} = 1 \circ k^{-1} \implies k^{-1} \in H\).
            \item \textit{closure}: let \(m, n \in H \implies n^{-1} \in H\) and let \(a=m\) and \(b=n^{-1}\).
            \(a\circ b^{-1} = a \circ (b^{-1})^{-1}=a\circ b\). This implies \(a, b \in H\).
        \end{itemize}
    }
\end{snippetproof}

\begin{snippetproposition}{finite-subgroup-test}{Finite subgroup test}
    Let \((G, \circ)\) be a \group and let \(H \subseteq G\) where \(\emptyset \neq H\)
    and \(\cardinality{H} < \infty\). \\
    Then, \((H, \circ) \subgroupleq (G, \circ) \iff
    \forall a,b \in H, a \circ b \in H\)
\end{snippetproposition}

\begin{snippetproof}{finite-subgroup-test-proof}{finite-subgroup-test}{Finite subgroup test}
    \iffproof{
        Assume \((H, \circ) \subgroupleq (G, \circ)\).
        The properties of a \group directly infer \(\forall a,b \in H, ab \in H\).
    }{
        Assume \(\forall a,b \in H, ab \in H\) and \(H \neq \emptyset\).
        We need to show that \(\forall a \in H, a^{-1} \in H\).
        By \principleofinduction[induction] on \(n\) we show that \(x^n \in H\) for \(n \in \naturalnumbers^\exceptzero\):
        \begin{itemize}
            \item the base case is given by \(a^1 = a \in H\);
            \item for the induction case let \(k \in \naturalnumbers^\exceptzero\)
                such that \(x^k \in H\). Then, \(x^{k+1} = x^k \circ x \in H\)
                since \(\circ\) is closed.
        \end{itemize}
        Since \(\cardinality{H} < \infty\) and \(\forall a, a^k \in H\)
        for \(n\in \naturalnumbers^\exceptzero\), infinite powers of \(a\)
        will give a finite amount of results, meaning that there are \(n,m \in \naturalnumbers^\exceptzero\)
        where \(m \neq n\) such that the exponentiation is equal \(x^n = x^m\).
        Without loss of generality, assume \(n>m\).
        Then, \(x^m = x^n \circ x^{m-n} = x^n\). By the cancellation laws, we find \(x^{m-n} = 1\).
        We distinguish two cases:
        \begin{enumerate}
            \item if \(m-n=1\), we get \(x^1 = 1\) which means \(x=1 \land x^{-1} = 1 \in H\);
            \item if \(m-n > 1\), we get \(x \circ x^{m-n-1} = 1\) which means
            \(x^{m-n-1} = x^{-1}\) and \(x^{m-n-1} \in H\) since \(m-n-1 > 0\).
        \end{enumerate}
    }
\end{snippetproof}

\subsection{Intersection and unions of sets}

\begin{snippetproposition}{subgroup-intersection-is-subgroup}{Subgroup intersection is subgroup}
    Let \(\{H_i\}_{i\in I}\) be \set[sets] such that
    \((H_i, \circ) \subgroupleq (G, \circ)\) for some \group \((G, \circ)\).
    Then,
    \[
        \left(\,\bigcap_{i\in I} H_i, \circ \right) \subgroupleq (G, \circ)
    \]
\end{snippetproposition}

\plain{The intersection of a submonoid is a monoid (if in the definition we require the two identity elements
to be the same).}

\begin{snippetproof}{subgroup-intersection-is-subgroup-proof}{subgroup-intersection-is-subgroup}{Subgroup intersection is subgroup}
    Without loss of generality, assume \(\{H_i\}_{i\in I} = \{H, K\}\).
    Since \(1\in H\) and \(1\in K\), we have \(1 \in H \intersection K\).
    Let \(a,b \in H\intersection K\), meaning \(a \in H\), \(a\in K\), \(b\in K\), \(b\in K\).
    Then, \(a\in H, b\in H \implies a\circ b^{-1} \in H\)
    and \(a\in K, b\in K \implies a\circ b^{-1} \in K\),
    which together imply \(a \circ b^{-1} \in H \intersection K\).
\end{snippetproof}

\plain{The same does not work for the union.}

\begin{snippetexample}{union-of-subgroups-not-subgroups-example}{union of subgroups is not subgroup}
    Consider in \((\integers, +)\) the \subgroup[subgroups]
    \(3\integers\) and \(4\integers\).
    Now, \(3\integers \union 4\integers\)
    contains the numbers that are multiples at least one of \(3\) and \(4\).
    The union contains \(3\) but not \(4+3=7\). Thus, it is not a \subgroup.
\end{snippetexample}

\end{document}
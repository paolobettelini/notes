\documentclass[preview]{standalone}

\usepackage{amsmath}
\usepackage{amssymb}
\usepackage{stellar}
\usepackage{definitions}
\usepackage{bettelini}

\begin{document}

\id{geoeconomica-diplomazia-panda}
\genpage

\section{Incontro Nixon 1972}

\begin{snippet}{incontro-nixon-1972-expl}
    Nel 1972, il presidente degli Stati Uniti Richard Nixon compì
    una storica visita in Cina, un evento che segnò un punto di svolta nelle
    relazioni tra Stati Uniti e Cina e contribuì a cambiare il panorama geopolitico mondiale.

    Questa visita portò ad una normalizzazione delle relazioni diplomatiche,
    bilancio geopolitico (che indebolì l'URSS) e l'apertura della Cina verso l'estero.
\end{snippet}

\section{Diplomazia del panda}

\begin{snippetdefinition}{diplomazia-del-panda-definition}{Diplomazia del panda}
    La \textit{diplomazia del panda} è una strategia utilizzata dalla Cina per sviluppare relazioni internazionali attraverso la diplomazia culturale, economica e politica, utilizzando la simpatia, la concessione di aiuti e l'uso della sua influenza economica e culturale.
\end{snippetdefinition}

\end{document}
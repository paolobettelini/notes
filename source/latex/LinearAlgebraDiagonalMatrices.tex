\documentclass[preview]{standalone}

\usepackage{amsmath}
\usepackage{amssymb}
\usepackage{stellar}
\usepackage{definitions}
\usepackage{bettelini}

\begin{document}

\id{diagonal-matrices}
\genpage

\section{Diagonal matrices}

\begin{snippettheorem}{first-diagonalizable-matrix-criterion-theorem}{First diagonalizable matrix criterion}
    \todo
    % sia V F-spazio vettoriale di dimensione n e sia f un endomorfismo
    % se il polinomio caratteristico di f ha n radidi distinte allora f
    % è diagonalizzabile
\end{snippettheorem}

\begin{snippetproof}{first-diagonalizable-matrix-criterion-theorem-proof}{first-diagonalizable-matrix-criterion-theorem}{First diagonalizable matrix criterion}
    Let \(\lambda_1, \lambda_2, \cdots, \lambda_n \in \mathbb{F}\) be the
    roots of \(p_f(t)\). Then, they are \eigenvalue[eigenvalues] of \(f\).
    There exist \(n\) non-null \vector[vectors] \(v_1, v_2, \cdots, v_n\)
    such that \(f(v_i) = \lambda v_i\). We want to show that the \eigenvector[eigenvectors]
    are \linearlyindependent, so that they form a \basis for \(V\), meaning
    that \(f\) is diagonalizable. %TODOURGENT
    To show that, we proceed by \principleofinduction[induction]:
    \begin{enumerate}
        \item the first \vector,
        \(v_1\), is not null and thus \(\{v_1\}\) is obviously \linearlyindependent.
        \item assume that \(\{v_1, v_2, \cdots, v_k\}\) is \linearlyindependent. We need to show that
        \(v_{k+1} \notin \linearspan\{\{v_1, v_2, \cdots, v_k\}\}\).
        If that were not to be the case, we would have scalars \(\alpha_1, \cdots, \alpha_k\)
        \[
            v_{k+1} = \sum_{i=1}^k \alpha_i v_i
        \]
        By applying those values to \(f\) we get
        \begin{align*}
            \lambda_{k+1} v_{k+1} &= f(v_{k+1}) = \sum_{i=1}^k \alpha_i f(v_i) \\
            &= \sum_{i=1}^k \alpha_i \lambda_i = v_i \\
            0 &= \sum_{i=1}^k \alpha_i(\lambda_i - \lambda_{k+1})
        \end{align*}
        by \linearlyindependent[linear independence],
        \[
            \alpha_i(\lambda_i - \lambda_{k+1}) = 0
        \]
        but since \(\lambda_i\) are all distinct and non-null, it must be
        \(\alpha_i = 0\), which would mean that \(v_{k+1} = 0\) \lightning.
    \end{enumerate}
\end{snippetproof}

% TODOURGENTE, magari da questa dimostrazione estrarre come lemma il fatto che
% autovettori relativi ad autovalori distinti formano un insieme linearmente indipendente.

% per esempio le omotetia non sodisfano il criterio ma sono diagonalizzabili

% non ammette autovalori => non è diagonalizzabile

\begin{snippetdefinition}{matrix-geometric-multiplicity-definition}{Geometric multiplicity}
    \todo
\end{snippetdefinition}

\begin{snippetdefinition}{matrix-algebraic-multiplicity-definition}{Algebraic multiplicity}
    \todo
\end{snippetdefinition}

% nel caso del primo criterio sono ambo uguali a 1, altrimenti sono diversi fra loro

\begin{snippetlemma}{geometric-multiplicity-leq-algebraic-multiplicity}{}
    \todo % se lambda è un autovalore di f allora \mu_g(\lambda) \leq \mu_a(\lambda)
\end{snippetlemma}

\begin{snippetproof}{geometric-multiplicity-leq-algebraic-multiplicity-proof}{geometric-multiplicity-leq-algebraic-multiplicity}{}
    By definition, \(\mu_g(\lambda) \geq 1\). Let
    \(\mu_g(\lambda) = k\) and consider a \basis \(\{v_1, v_2, \cdots, v_k\}\)
    for \(V_\lambda = \grpker (\lambda\identmatrix{n} - f)\) where \(n = \lineardim V\).
    We want to expand this \basis into a \basis \(\mathcal{B} = \{v_1, v_2, \cdots, v_n\}\)
    of \(V\). We have the form
    \[
        \mathcal{M}(f,\mathcal{B}) =
        \left[\,
        \begin{array}{@{}c|c@{}}
            \lambda I_{k} & A' \\ \hline
            0              & A''
            \end{array}
        \,\right].
    \]
    so that
    \[
        p_f(t) = \det (\lambda \identmatrix{n} - f)
        = \det(\lambda \identmatrix{n} - \mathcal{M}(f,\mathcal{B}))
    \]
    which is a block \matrix and has form
    \[
        \det \left[
            \begin{array}{c|c}
              (t - \lambda)\identmatrix{k} & -A' \\ \hline
              0 & t\identmatrix{n-k} - A''
            \end{array}
            \right]
            = (t - \lambda)^k \det(t\identmatrix{n-k} - A'')
            = (t - \lambda)^k p_{A''}(t)
    \]
    meaning that \(\mu_a(\lambda) \geq \mu_g(\lambda)\).
\end{snippetproof}

\begin{snippettheorem}{second-diagonalizable-matrix-criterion-theorem}{Second diagonalizable matrix criterion}
    % todo
    % f è diagonalizzabile se e solo se
    % i) tutti gli autovalori di f sono in f
    % ii) per tutti gli autovalori, le due molteplictità coincidono.
\end{snippettheorem}

\begin{snippetproof}{second-diagonalizable-matrix-criterion-theorem-proof}{second-diagonalizable-matrix-criterion-theorem}{First diagonalizable matrix criterion}
    \iffproof{
        We can find a \basis \(\mathcal{B}\) for \(V\) such that
        \(\mathcal{f, \mathcal{B}}\) is diagonal, and \(\mathcal{B}\)
        is formed by \eigenvector[eigenvectors] of \(f\).
        Let \(\lambda_1, \lambda_2, \cdots, \lambda_r\) be the \eigenvalue[eigenvalues]
        that are distinct and let \(\mu_i = \mu_g(\lambda_i)\).
        \[
            \mathcal{M}(f, \mathcal{B})
            = \text{diag}(
                \underbrace{\lambda_1, \cdots, \lambda_1}_{\mu_1 \text{ times}},
                \underbrace{\lambda_2, \cdots, \lambda_2}_{\mu_2 \text{ times}},
                \cdots,
                \underbrace{\lambda_r, \cdots, \lambda_r}_{\mu_r \text{ times}},
            )
        \]
        Obviously, \(\lambda_i \in \mathbb{F}\).
        \begin{align*}
            p_f(t) = \det(t\identmatrix{n} - \mathcal{M}(t, \mathcal{B}))
            = \prod_{i=1}^r {(t-\lambda_i)}^{\mu_i}
        \end{align*}
        but since \(\lambda_j\) are distinct, we have that \(\mu_i = \mu_a(\lambda_i)\).
    }{
        Let \(\lambda_1, \lambda_2, \cdots, \lambda_r \in \mathbb{F}\)
        be the \eigenvalue[eigenvalues] of \(f\).
        Let \(\{v_1^i, \cdots, v_{\mu_i}^i\}\) be a \basis of \(V_{\lambda_i}\)
        where \(\mu_i = \mu_g(\lambda_i)\).
        We now have
        \begin{align*}
            \sum_{i=1}^r \mu_g(\lambda_i)
            = \sum_{i=1}^r \mu_a(\lambda_i) = n
        \end{align*}
        since all the roots are in \(\mathbb{F}\).
        Since the \eigenvector[eigenvectors] relative to the distinct \eigenvalue[eigenvalues]
        are \linearlyindependent,
        \[
            \mathcal{B} = \{
                v_1^1, \cdots, V_{\mu_1}^1,
                v_1^2, \cdots, V_{\mu_2}^2,
                \cdots,
                v_1^r, \cdots, V_{\mu_r}^r
            \}
        \]
        is a \basis for \(V\).
        \[
            \mathcal{M}(f, \mathcal{B}) = \text{diag}(
                \underbrace{\lambda_1, \cdots, \lambda_1}_{\mu_1 \text{ times}},
                \underbrace{\lambda_2, \cdots, \lambda_2}_{\mu_2 \text{ times}},
                \cdots,
                \underbrace{\lambda_r, \cdots, \lambda_r}_{\mu_r \text{ times}},
            )
        \]
        meaning that \(f\) is diagonalizable.
    }
\end{snippetproof}

\end{document}
\documentclass[preview]{standalone}

\usepackage{amsmath,stackengine}
\usepackage{amssymb}
\usepackage{stellar}
\usepackage{definitions}

\stackMath{}

\begin{document}

\id{determinants}
\genpage

\section{Determinants}

\begin{snippetdefinition}{determinant-definition}{Determinant}
    The \textit{determinant} of a square \snippetref[matrix-definition][matrix] \(M\),
    denoted \(\origdet(M)\),
    is a scalar which represents the factor by which the area of any shape (or volume and so on) changes
    after the linear transformation representes by the matrix is applied.
    If the transformation inverts the orientation of the space, the determinant is negative.
\end{snippetdefinition}

\begin{snippetdefinition}{matrix-minor-definition}{Matrix minor}
    Let \(M\) be a square matrix.
    The \textit{minor} of the entry in the \(i\)th row and \(j\)th column
    is the determinant of the submatrix formed by deleting the \(i\)th row and \(j\)th column.
    \[
        \text{minor of }b=
        \stackinset{c}{}{c}{1\baselineskip}{\rule{4.25\baselineskip}{0.5pt}}{
        \stackinset{c}{0\baselineskip}{c}{}{\rule{0.5pt}{3.5\baselineskip}}{
        \begin{vmatrix}
            a & b & c \\
            d & e & f \\
            g & h & i
        \end{vmatrix}}}
        =
        \begin{vmatrix}
            d & f \\
            g & i
        \end{vmatrix}
    \]
\end{snippetdefinition}

\begin{snippettheorem}{laplace-expansion-theorem}{Laplace expansion}
    The Laplace expansion is a formula that allows us to express the determinant of
    a matrix as a linear combination of the minors.
    \[
        \det(A)=\sum_{j=1}^{n}{(-1)}^{i+j}A_{ij}\det(\text{minor of }A_{ij})
    \]

    where \(i\) is any row.

    Instead of expanding the series along a row we could expand it along any column,
    the result is always the same.
\end{snippettheorem}

\begin{snippetexample}{laplace-expansion-2x2-example}{Determinant of \(2\times2\) matrix}
    Given a \(2 \times 2\) matrix \(A\)
    \[
        A=
        \begin{bmatrix}
            a & b \\
            c & d
        \end{bmatrix}
    \]
    its determinant is given by
    \[
        \det(A)=
        \begin{vmatrix}
            a & b \\
            c & d
        \end{vmatrix}
        \equiv ad-bc
    \]
    \phantom{}
\end{snippetexample}

\begin{snippetexample}{laplace-expansion-3x3-example}{Determinant of \(3\times3\) matrix}
    Given a \(3 \times 3\) matrix \(A\)
    \[
        A=
        \begin{bmatrix}
            a & b & c \\
            d & e & f \\
            g & h & i
        \end{bmatrix}
    \]
    its determinant is given by
    \[
        \det(A)=
        \begin{vmatrix}
            a & b & c \\
            d & e & f \\
            g & h & i
        \end{vmatrix}
        =
        a \begin{vmatrix}
            e & f \\
            h & i
        \end{vmatrix}
        -b \begin{vmatrix}
            d & f \\
            g & i
        \end{vmatrix}
        +c \begin{vmatrix}
            d & e \\
            g & h
        \end{vmatrix}
    \]
    \phantom{}
\end{snippetexample}

\begin{snippetproposition}{determinant-properties}{Properties of the determinant}
    Let \(A\) and \(B\) be two square \matrix[matrices] of the same size.
    The determinant has some properties, for instance

    \begin{align*}
        \det(A^\transpose)&=\det(A) \\
        \det(AB)&=\det(A)\det(B)
    \end{align*}
\end{snippetproposition}

\begin{snippetdefinition}{cofactor-matrix-definition}{Cofactor matrix}
    Let \(A\) be a \(n \times n\) \matrix. Then,
    the \emph{cofactor matrix of \(A\)} is defined as the matrix
    \[
        C_{i,j} = {(-1)}^{i+j} \det (\text{minor of } A_{i,j})
    \]
\end{snippetdefinition}

\end{document}
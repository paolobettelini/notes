\documentclass[preview]{standalone}

\usepackage{amsmath}
\usepackage{amssymb}
\usepackage{stellar}
\usepackage{definitions}
\usepackage{bettelini}

\begin{document}

\id{biologia-filogenesi}
\genpage

\section{La filogenesi e l'albero della vita}

\plain{Per ricostruire la filogenesi delle specie è
importante distinguere le omologie dalle analogie}

\begin{snippetdefinition}{omologia-bio-definition}{Omologia}
    Si dicono \textit{omologhi} tra loro organi di specie diverse derivanti dalla stessa porzione dell'embrione.
    Più in generale, si dicono omologhi tra loro organi che hanno la stessa origine ma svolgono funzioni diverse, come la pinna di un pesce e la zampa di un gatto, o l'ala di un uccello e il braccio di un uomo.
\end{snippetdefinition}

\begin{snippetdefinition}{analogia-bio-definition}{Analogia}
    In biologia si dicono \textit{analoghi} gli organi che svolgono la stessa funzione, ma provengono da origine diversa, come ad esempio l'ala di un uccello e di un insetto.
\end{snippetdefinition}

\begin{snippetdefinition}{evoluzione-convergente-definition}{Evoluzione convergente}
    Si definisce convergenza evolutiva il fenomeno per cui specie diverse che vivono nello stesso tipo di ambiente, o in nicchie ecologiche simili, sulla spinta delle stesse pressioni ambientali, si evolvono sviluppando, per selezione naturale, determinate strutture o adattamenti che li portano ad assomigliarsi moltissimo. Tali specie sono dette convergenti. 
\end{snippetdefinition}

\begin{snippetdefinition}{evoluzione-divergente-definition}{Evoluzione divergente}
    L'\textit{evoluzione divergente} è il fenomeno per mezzo del quale alcune caratteristiche fenotipiche, di comune origine, si sono differenziate nel corso della storia evolutiva. 
\end{snippetdefinition}

\begin{snippet}{50dabe0c-76a2-437e-a6a6-703bf2b907f5}
    Le strutture omologhe sono frutta di evoluzione divergente, mentre le analogie da
    evoluzione convergente.
\end{snippet}

\begin{snippetdefinition}{anatomia-comparata-definition}{Anatomia comparata}
    L'\textit{anatomia comparata} è una disciplina di sintesi che 
    opera mediante la \quotes{comparazione} fra le strutture anatomiche
    dei diversi gruppi di vertebrati e si pone l'obiettivo di individuare ed analizzare
    le cause della loro forma, della loro organizzazione strutturale e dei loro adattamenti.
\end{snippetdefinition}

\begin{snippetdefinition}{omologia-molecolare-definition}{Omologia molecolare}
    L'\textit{omologia molecolare} confronta le sequenze di DNA
    tra più specie per stabilire se condividono o meno un antenato.
\end{snippetdefinition}

\plain{L'omologia molecolare è più affidabile dell'anatomia comparata.}

\begin{snippetdefinition}{speciazione-definition}{Speciazione}
    La speciazione è un processo evolutivo grazie al quale si formano nuove specie da quelle preesistenti. 
\end{snippetdefinition}

\begin{snippet}{0509fecc-8c3d-4b71-8f81-e2272472f87a}
    A volte, una specie si evolve semplicemente in una direzione, ma è possibile che si formino
    uno o più rami, creando così specie diverse (speciazione).
\end{snippet}

% specie vs popolazione

\section{Regni}

\begin{snippetdefinition}{regno-funghi-definition}{Regno Funghi}
    Organismi eucarioti, unicellulari o pluricellulari, eterotrofi
    che assorbono sostanza organica.
\end{snippetdefinition}

\begin{snippetdefinition}{regno-protista-definition}{Regno Protista}
    Organismi eucarioti unicellulari, sia autotrofi
    che eterotrofi
\end{snippetdefinition}

\begin{snippetdefinition}{regno-monera-definition}{Regno Monera}
    Organismi procarioti
    unicellulari, sia autotrofi che eterotrofi
\end{snippetdefinition}

\begin{snippetdefinition}{regno-animalia-definition}{Regno Animalia}
    Organismi eucarioti, pluricellulari, eterotrofi
    che ingeriscono gli alimenti
\end{snippetdefinition}

\begin{snippetdefinition}{regno-platae-definition}{Regno Platae}
    Organismi eucarioti pluricellulari autotrofi.
\end{snippetdefinition}

\end{document}
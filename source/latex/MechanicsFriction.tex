\documentclass[preview]{standalone}

\usepackage{amsmath}
\usepackage{amssymb}
\usepackage{stellar}
\usepackage{definitions}

\begin{document}

\id{mechanics-fricion}
\genpage

\section{Friction}

\begin{snippet}{friction-simple-expl}
    If the velocity of an object is null, the friction force exerted on it is also null.
    Otherwise, for small velocities, the force is given by
    \[
        \vec{F}(t) = -\gamma v(t)
    \]
\end{snippet}

\begin{snippetexample}{friction-sphere-in-liquid-example}{Coefficient of fricion of a sphere in a fluid}
    Consider a fluid with viscosity \(\nu\). From Stokes' law, a sphere of radius \(a\) (with a perfectly
    rigid surface) has a coefficient of friction of
    \[
        \gamma = 6\pi\nu a
    \]
\end{snippetexample}

\begin{snippetexample}{motion-of-body-affected-by-friction-example}{Motion of body affected only by friction}
    Consider a body only affected by a friction force. Then \(-\gamma v = ma\).
    The differential equation for the motion is
    \[
        \frac{dv}{dt} = -\frac{\gamma}{m}v
    \]
    and thus
    \[
        v(t) = v_0e^{\lambda t}, \quad \lambda = -\frac{\gamma}{m}
    \]
    the motion is given by
    \[
        x(t) = \frac{v_0}{\lambda}e^{\lambda t} + x_0, \quad \lambda = -\frac{\gamma}{m}
    \]
    Since \(x(0)=0\) we have \(x_0 = -\frac{v_0}{\gamma}\) and thus
    \[
        x(t) = v_0 \frac{m}{\gamma}\left[1 - e^{-\frac{\gamma}{m}t}\right]
    \]
    The velocity tends to zero without reaching it, but the position is finite.
\end{snippetexample}

\begin{snippetexample}{motion-of-falling-body-affected-by-friction-example-example}{Motion of falling body affected by friction}
    A falling object has force
    \[
        F_z = -mg - \gamma v_z = ma_z
    \]
    The differential equation is
    \[
        \frac{dv_z}{dt} = -g-\frac{\gamma}{m}v_z
    \]
    and thus
    \[
        v_z(t) = \frac{gm}{\gamma}\left[
            e^{-\frac{\gamma}{m}t} - 1
        \right]
    \]
\end{snippetexample}

\begin{snippet}{frictions-expl}
    An object on a plane to which a force is applied will experience the opposing force  
    of friction, where \(|\vec{F_A}| = \mu R\),  
    where \(\mu\) is a dimensionless coefficient that depends on the nature of the two surfaces,  
    and \(R\) is the normal reaction force. \textbf{It is NOT} proportional to the mass,  
    but the normal reaction is certainly related to the mass.  
    The normal reaction must exactly counteract the weight force.  
    Therefore, the normal reaction is \(mg\).  
    It is important to note that there may be a case where \(g=0\),  
    in which there is no friction, and thus it is not proportional to mass.  
    
    The coefficients of friction are distinguished into static and kinetic friction.  
    Static friction is related to the minimum force required to set an object in motion,  
    while kinetic friction is associated with the movement of the object itself.  
    In general, \(\mu_S > \mu_D\).  
    
    The coefficient of friction arises from the fact that the contact surfaces are rough;  
    only a small portion of the two macroscopic surfaces are actually in contact.  
    This is also the reason why the static friction coefficient is generally  
    greater than the kinetic one, as sufficient energy is required  
    to break the interlocking points between the two surfaces.  
    The friction coefficients do not depend on the contact area (in fact,  
    given the same mass but a different surface area,  
    the friction coefficients remain the same).  
\end{snippet}

\subsection{Inclined plane}

\begin{snippettheorem}{friction-on-inclined-plane-theorem}{Friction on inclined plane}
    On an inclined plane, the object moves only if \(mg\sin\alpha > \mu_S mg\cos\alpha\),  
    that is, \(\tan \alpha > \mu_S\).  
    Once in motion, the acceleration is given by \(g\cos\alpha\left[\tan\alpha-\mu_D\right]>0\).  
    This type of friction does not depend on the velocity.  
\end{snippettheorem}

\end{document}
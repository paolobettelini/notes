\documentclass[preview]{standalone}

\usepackage{amsmath}
\usepackage{amssymb}
\usepackage{stellar}
\usepackage{definitions}

\begin{document}

\id{implicit-function-theorem-exercises}
\genpage

\section{Exercises}

\begin{snippetexercise}{implicit-function-theorem-ex1}{}
    Show that the equation
    \[ xe^y - ye^x - 1 = 0 \]
    implicitly defines a unique function \(y = \varphi(x)\) in a \neighborhood of \((1,0)\).
    Determine \(\varphi'(1)\).
\end{snippetexercise}

\begin{snippetsolution}{implicit-function-theorem-ex1-sol}{}
    \todo
\end{snippetsolution}

\begin{snippetexercise}{implicit-function-theorem-ex2}{}
    Show that the equation
    \[ \sqrt[3]{x^2 - y}-2x + y = 0 \]
    implicitly defines a unique function \(y = \varphi(x)\) in a \neighborhood of \((2,3)\).
    Determine \(\varphi'(2)\).
\end{snippetexercise}

\begin{snippetsolution}{implicit-function-theorem-ex2-sol}{}
    \todo
\end{snippetsolution}

\begin{snippetexercise}{implicit-function-theorem-ex3}{}
    Show that the equation
    \[ x^3 - x\sqrt{y} + x + y = 0 \]
    implicitly defines a unique function \(y = \varphi(x)\) in a \neighborhood of \((-1,1)\).
    Determine \[
        \lim_{x\to -1} \frac{\varphi(x) + 1 + 2x}{{(x+1)}^2}
    \]
\end{snippetexercise}

\begin{snippetsolution}{implicit-function-theorem-ex3-sol}{}
    Let \(f(x,y) = x^3 - x\sqrt{y} + x + y\).
    We have \(f(-1,1) = 0\)
    and \[
        f_y(x,y) = -\frac{x}{2\sqrt{y}} + 1
    \]
    which is continuous in a \neighborhood of \((-1,1)\)
    and \(f_y(-1,1) = \frac32 \neq 0\). Thus, there exists a unique \function
    \(y = \varphi(x)\) near such a point.
    We will now compute \(\varphi'\). We have
    \begin{align*}
        \left[f(x,\varphi(x))\right]'
        &= [x^3 - x\sqrt{\varphi(x)} + \varphi(x) + x]' \\
        &= 3x^2 - \sqrt{\varphi(x)} - \frac{x\varphi'(x)}{2\sqrt{\varphi(x)}} + \varphi'(x) + 1 = 0 \\
        \varphi'(x) &= \frac{-1-3x^2 + \sqrt{\varphi(x)}}{1-\frac{x}{2\sqrt{\varphi(x)}}} \\
        &= -\frac{f_x(x, \varphi(x))}{f_y(x,\varphi(x))}
    \end{align*}
    For the limit we have
    \begin{align*}
        \lim_{x\to -1} \frac{\varphi'(x) + 2}{2(x+1)}
        &\overset{H}{=} \frac{\varphi'(-1) + 1}{2} = \frac00
    \end{align*}
    We need \(\varphi''(-1)\) to do a Taylor expansion.
    \begin{align*}
        \left\{\left[f(x,\varphi(x))\right]'\right\}'
        &= \left\{3x^2 - \sqrt{\varphi(x)} - \frac{x\varphi'(x)}{2\sqrt{\varphi(x)}} + \varphi'(x) + 1\right\}' \\
        \varphi''(x) \left(- \frac{x}{2 \varphi(x)} + 1\right) &=
        -6x + \frac{\varphi'(x)}{2\sqrt{\varphi(x)}} + \frac{1}{2}
        \frac{\varphi'(x)}{\sqrt{\varphi(x)}} + \frac{1}{4} \frac{x [\varphi'(x)]^2}{\sqrt{\varphi(x)}\varphi(x)} \\
        \varphi''(-1) &= 2
    \end{align*}
    We now have \(\varphi(x) = 1 + \varphi'(-1)(x + 1) + \littleO(x+1)\). Then,
    \begin{align*}
        \lim_{x\to -1} \frac{\varphi''(x)}{2} = 1
    \end{align*}
\end{snippetsolution}

\begin{snippetexercise}{implicit-function-theorem-ex4}{}
    Show that the equation
    \[ x^3 - xy^2 - 2y^3 = 0 \]
    implicitly defines a unique function \(y = \varphi(x)\) in a \neighborhood of \((1,1)\).
    Determine the equation of the line tangent to the graph of \(\varphi\) at \(x=1\).
    Determine \(\varphi''(1)\).
\end{snippetexercise}

\begin{snippetsolution}{implicit-function-theorem-ex4-sol}{}
    \todo
\end{snippetsolution}

\begin{snippetexercise}{implicit-function-theorem-ex5}{}
    Show that the equation
    \[ e^{xy} - e^{-2y}\cos(x) - 1=0 \]
    implicitly defines a unique function \(y = \varphi(x) \in \mathcal{C}^1\) in a \neighborhood of
    \(x_0 = \pi/2\).
    Determine \(\varphi'(\pi/2)\).
\end{snippetexercise}

\begin{snippetsolution}{implicit-function-theorem-ex5-sol}{}
    Let \(F(x,y) = e^{xy} - e^{-2y}\cos(x) - 1\).
    We must have \(F(x_0, y) = e^{\frac{\pi y}{2}}-1 = 0\)
    and thus \(y=0\). Hence the point is \((\pi/2, 0)\).
    We have the derivatives
    \begin{align*}
        \frac{\partial F}{\partial x} &= ye^{xy} + e^{-2y}\sin(x) \\
        \frac{\partial F}{\partial y} &= xe^{xy} + 2x^{-2y}\cos(x)
    \end{align*}
    At the point we have \(F_y(\pi/2, 0) = \pi/2 \neq 0\) meaning that
    there exists an implicit function. We finally have
    \begin{align*}
        \varphi'\left(\frac{\pi}{2}\right) &=
        -\frac{
            \varphi\left(\frac{\pi}{2}\right) e^{\varphi\left(\frac{\pi}{2}\right)\frac{\pi}{2}}
            + e^{-2\varphi\left(\frac{\pi}{2}\right)}\sin\left(\frac{\pi}{2}\right)
        }{
            \frac{\pi}{2} e^{\varphi\left(\frac{\pi}{2}\right)\frac{\pi}{2}}
            + 2e^{-2\varphi\left(\frac{\pi}{2}\right)}\cos\left(\frac{\pi}{2}\right)
        } = -\frac{2}{\pi}
    \end{align*}
\end{snippetsolution}

\begin{snippetexercise}{implicit-function-theorem-ex6}{}
    Show that the equation
    \[ \sinh(z-1) - e^x + e^y + xz - y = 0 \]
    implicitly defines a unique function \(z = \varphi(x,y)\) in a \neighborhood of \((0,0,1)\).
    Determine the nature of the point \((0,0)\) for \(\varphi\).
\end{snippetexercise}

\begin{snippetsolution}{implicit-function-theorem-ex6-sol}{}
    \todo
\end{snippetsolution}

\begin{snippetexercise}{implicit-function-theorem-ex7}{}
    Show that the equation
    \[ x^2 + 2x + e^y + y - z^3 = 0 \]
    implicitly defines a unique function \(y = \varphi(x,z)\) in a \neighborhood of \((-1,0,0)\).
    Determine the equation of the plane tangent to the graph of \(\varphi\) at \((-1,0,0)\).
    Determine \(\varphi''(1)\).
\end{snippetexercise}

\begin{snippetsolution}{implicit-function-theorem-ex7-sol}{}
    Let \(F(x,y,z) = x^2 + 2x + e^y + y - z^3\).
    We have \(f(-1,0,0) = 0\) and \(f_y = e^y + 1\)
    meaning \(f_y(-1,0,0) = 2 \neq 0\). Hence there exists a unique function
    \(y = \varphi(x,z)\).
    We also have \(\varphi(-1,0) = 0\) and \(f(x,0,z) > 0\).
    The tangent plane has equation \(y = \varphi(-1,0) + \gradient \varphi(-1,0) \cdot (x + 1, z)\).
    We need \(\varphi_x(-1,0)\) and \(\varphi_z(-1,0)\).
    We know that \(F(x, \varphi(x,z), z) = 0\) and thus
    \[
        \frac{\partial}{\partial x} [F(x,\varphi(x,z), z)] = 0
    \]
    From this we get \(\varphi_x(-1,0) = 0\) and \(\varphi_x(-1,0) = 0\).
    Thus, \((0-1)\) is a critical point for \(\varphi\) and the tangent plane is
    \(y=0\).
\end{snippetsolution}

\begin{snippetexercise}{implicit-function-theorem-ex8}{}
    Verify that \((0,0,0)\) is a critical point for the function \(z = z(x,y,u)\) implicitly
    determined by the equation
    \[
        x^2 + xu^2 + y^2 + e^{xu} - z + y^2e^z = 0
    \]
    Determine the nature of this point.
\end{snippetexercise}

\begin{snippetsolution}{implicit-function-theorem-ex8-sol}{}
    \todo
\end{snippetsolution}

\begin{snippetexercise}{implicit-function-theorem-ex9}{}
    Show that the equation
    \[ xy^2 + y + \sin(xy) + 3(e^x -1) = 0 \]
    implicitly defines a function \(y = y(x)\) in a \neighborhood of \((0,0)\).
    Determine
    \[
        \lim_{x\to 0} \frac{y(x) + 3x}{x}
    \]
\end{snippetexercise}

\begin{snippetsolution}{implicit-function-theorem-ex9-sol}{}
    \todo
\end{snippetsolution}

\begin{snippetexercise}{implicit-function-theorem-ex10}{}
    Consider the \function
    \[
        F(x,y,z) = x^2e^z + y^2 + z = 0
    \]
    \begin{enumerate}
        \item Show that the equation \(F=0\) implicitly defines a function
        \(z = z(x,y)\) in a \neighborhood of \((0,0,0)\);
        \item determine the nature of the point \((0,0)\) for the \function \(z\);
        \item determine whether \((0,0)\) is an absolute maximum;
        \item show that the maximal domain for \(z\) is \(\realnumbers^2\);
        \item compute
        \[\lim_{(x,y) \fromto \infty} z(x,y)\]
    \end{enumerate}
\end{snippetexercise}

\begin{snippetsolution}{implicit-function-theorem-ex10-sol}{}
    \todo
\end{snippetsolution}

\begin{snippetexercise}{implicit-function-theorem-ex11}{}
    Show that the equation
    \[ e^z + (x^2 - y^2)z - (1+xy)e^{\sin(x^2 + y^2)} = 0 \]
    implicitly defines a function \(z = z(x,y)\) in a \neighborhood of \((0,0,0)\)
    and determine the nature of the point \((0,0)\) for \(z\).
\end{snippetexercise}

\begin{snippetsolution}{implicit-function-theorem-ex11-sol}{}
    \todo
\end{snippetsolution}

\begin{snippetexercise}{implicit-function-theorem-ex12}{}
    Given the system
    \[
        \begin{cases}
            x + \ln y + 5z - 10 = 0\\
            2x + y^2 + 3z^3 - 25 = 0
        \end{cases}
    \]
    determine which pair of variables can be explicable with respect to the other
    in a \neighborhood of \((0,1,2)\).
\end{snippetexercise}

\begin{snippetsolution}{implicit-function-theorem-ex12-sol}{}
    \todo
\end{snippetsolution}

\begin{snippetexercise}{implicit-function-theorem-ex13}{}
    Let \(F \colon \realnumbers^4 \fromto \realnumbers^2\) defined by
    \[
        F(x,y,s,t) = \begin{pmatrix}
            (y + 3)s - \arctan(s + t) + 2x \\
            \sin(s + t) + 3y - x(t+3)
        \end{pmatrix}
    \]
    Show that \(F = 0\) implicitly defines \(s=s(x,y)\) and \(t=t(x,y)\)
    in a \neighborhood of \((0,0,0,0)\). Determine
    \[
        \frac{\partial s}{\partial y}(0,0)
    \]
\end{snippetexercise}

\begin{snippetsolution}{implicit-function-theorem-ex13-sol}{}
Let \(F = (F_1, F_2) \colon \realnumbers^4 \fromto \realnumbers^2\).
    Compute \(\jacobian F(x, y, s, t)\):
    \[
        \jacobian F(x, y, s, t) = \begin{pmatrix}
            2 & s & y + 3 - \frac{1}{1 + (s+t)^2} & -\frac{1}{1 + (s+t)^2} \\[0.5em]
            -(t + 3) & 3 & \cos(s + t) & \cos(s + t) - x
        \end{pmatrix}
    \]
    At the origin:
    \[
        \jacobian F(0, 0, 0, 0) = \begin{pmatrix}
            2 & 0 & 2 & -1 \\
            -3 & 3 & 1 & 1
        \end{pmatrix}
    \]
    We can thus choose arbitrarily two variables as \function[functions] of the other two,
    for example \(s = s(x, y)\) and \(t = t(x, y)\).
    Compute the differential at \((0, 0)\):
    \[
        \text{d}\varphi(x_0) = -F_y(x_0, y_0)^{-1} \circ F_x(x_0, y_0)
    \]
    Since \(F \in \continuityclass^\infty(\realnumbers^4)\), we have \(\varphi \in \continuityclass^\infty\)
    in a \neighborhood of \((0, 0)\).
    \[
        \jacobian_{(x,y)} F(0, 0, 0, 0) = \begin{pmatrix}
            2 & 0 \\
            -3 & 3
        \end{pmatrix}, \quad
        \jacobian_{(s,t)} F(0, 0, 0, 0) = \begin{pmatrix}
            2 & -1 \\
            1 & 1
        \end{pmatrix}
    \]
    And the inverse is given by
    \[
        \jacobian_{(s,t)} F(0, 0, 0, 0)^{-1} = \frac{1}{3} \begin{pmatrix}
            1 & 1 \\
            -1 & 2
        \end{pmatrix}
    \]
    We thus have the differential
    \[
        \text{d}\varphi(0, 0) = -\frac{1}{3} \begin{pmatrix}
            1 & 1 \\
            -1 & 2
        \end{pmatrix}
        \begin{pmatrix}
            2 & 0 \\
            -3 & 3
        \end{pmatrix}
        = \begin{pmatrix}
            1/3 & -1 \\
            1/3 & -2
        \end{pmatrix}
    \]
\end{snippetsolution}

\begin{snippetexercise}{implicit-function-theorem-ex14}{}
    Determine for which \(\lambda \in \realnumbers\) the \function
    \[
        f(x,y) = (x + \lambda y, y - (\lambda + 1)x^2)
    \]
    is a diffeomorphism of \(\realnumbers^2\) on \(\realnumbers^2\).
\end{snippetexercise}

\begin{snippetsolution}{implicit-function-theorem-ex14-sol}{}
    The Jacobian is given by
    \[
        \jacobian_f(x,y) = \begin{pmatrix}
            1 & \lambda \\ -2(\lambda + 1)x & 1
        \end{pmatrix}
    \]
    Hence the determinant is \(\det \jacobian_f(x,y) = 1 + 2\lambda(\lambda + 1)x\).
    We must have \(\lambda = 0\) or \(\lambda = -1\).
    \begin{enumerate}
        \item if \(\lambda = 0\), \(f(x,y) = (x, y-x^2)\). The inverse is
        \((u,v) \fromto (u, v + u^2)\), smooth and defined on all \(\realnumbers^n2\);
        \item if \(\lambda = -1\), \(f(x,y) = (x-y,y)\). The inverse is
        \((u,v)\fromto (u + v, v)\), smooth and global.
    \end{enumerate}
    Hence, \(f\) is such a diffeomorphism for \(\lambda \in \{0,-1\}\).
\end{snippetsolution}

\begin{snippetexercise}{implicit-function-theorem-ex15}{}
    Let \(F(x, y) = -xe^y + 2y - 1\).
    Show that in a \neighborhood of \((0, \frac{1}{2})\), the solutions of
    \(F(x, y) = 0\) implicitly define a \function \(y = y(x)\).
    Then find the MacLaurin expansion to second order of \(y = y(x)\) at \(x = 0\).
\end{snippetexercise}

\begin{snippetsolution}{implicit-function-theorem-ex15-sol}{implicit-function-theorem-ex15}{}
    Note that the problem is well-posed since \(F(0, \frac{1}{2}) = 0\).
    The condition guaranteeing existence of this \function
    is that \(F_y \neq 0\).
    We have \(F_y = -xe^y + 2\) and \(F_y(0, \frac{1}{2}) = 2 \neq 0\).
    Thus, by the implicit \function theorem, there exists a unique \function \(y = y(x)\)
    such that \(F(x, y(x)) \equiv 0\).
    For the MacLaurin expansion, we have
    \(y(0) = \frac{1}{2}\). Computing the derivative:
    \begin{align*}
        \frac{d}{dx} \left[F(x, y(x))\right]
        &= -e^y - xe^y y' + 2y' = 0
    \end{align*}
    In particular, for \(x = 0\), we have \(-e^{y(0)} - 0 + 2y'(0) = 0\),
    thus \(y'(0) = \frac{e^{1/2}}{2}\).
    To find the next term, differentiate again:
    \begin{align*}
        \frac{d^2}{dx^2}\left[F(x, y(x))\right] = -e^y y' - e^y y' - x(\cdots) + 2y'' = 0
    \end{align*}
    From which we get \(0 = -2e^{y(0)} y'(0) + 2y''(0)\)
    and thus \(y''(0) = \frac{e}{2}\).
    Therefore
    \[
        y(x) = \frac{1}{2} + \frac{1}{2}e^{1/2}x + \frac{1}{4}ex^2 + o(x^2)
    \]
\end{snippetsolution}

\end{document}
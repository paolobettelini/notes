\documentclass[preview]{standalone}

\usepackage{amsmath}
\usepackage{amssymb}
\usepackage{tikz}
\usepackage{wrapfig}
\usepackage{bettelini}
\usepackage{stellar}
\usepackage{definitions}

\begin{document}

\id{theoryofcomputation-introduction}
\genpage

\section{Fields of Study}

\begin{snippet}{theory-of-computation-fields-of-stufy}
    \vspace{-0.25cm}
    \begin{itemize}
        \item \textbf{Complexity Theory:} studies the resources (such as time and space) required to solve computational problems. It categorizes problems into complexity classes (e.g., P, NP, PSPACE) and explores the limits of efficient computation.
        \item \textbf{Computability Theory:} examines what problems can be solved by a computer, regardless of efficiency. It defines models like Turing machines and classifies problems as computable, undecidable, or semi-decidable.
        \item \textbf{Automata Theory:} focuses on abstract machines (e.g., finite automata, pushdown automata, Turing machines) and formal languages. It investigates the computational power of these models and their relationship with language classes in the Chomsky hierarchy.
    \end{itemize}
\end{snippet}

\end{document}
\documentclass[preview]{standalone}

\usepackage{amsmath}
\usepackage{amssymb}
\usepackage{stellar}
\usepackage{definitions}

\begin{document}

\id{algorithms-introduction}
\genpage

\section{Algorithms}

\begin{snippetdefinition}{problem-definition}{Problem}
    A \emph{problem} is a \function
    \[
        f\colon D_I \fromto D_S
    \]
    from a \set \(D_I\) of \emph{instances} to a \set of \emph{solutions} \(D_S\).
\end{snippetdefinition}

\begin{snippetdefinition}{algorithm-definition}{Algorithm}
    An \emph{algorithm} is a finite, well-defined sequence of steps or rules designed to solve a problem.
    It takes an \emph{input}, processes it according to a set of operations, and produces an \emph{output}.
    An algorithm \(A\) defines a \function \(f_A\) and given an input \(x\), \(f_A(x)\) denotes the solution.
    We say that \(A\) solves a problem \(f\) if \(f(x) = f_A(x)\) for all instances \(x\).
\end{snippetdefinition}

\end{document}
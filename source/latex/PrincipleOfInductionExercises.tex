\documentclass[preview]{standalone}

\usepackage{amsmath}
\usepackage{amssymb}
\usepackage{stellar}
\usepackage{definitions}

\begin{document}

\id{principle-of-induction-exercises}
\genpage


\section{Exercises}

\begin{snippetexercise}{induction-ex-1}{Weak induction}
    Prove that for each \(n \geq 1\), the sum \[ \sum_{k=1}^n k = \frac{n(n+1)}{2} \].
    \begin{itemize}
        \item The base case is given by \(n=1\) where \(1 = \frac{2}{2} = 1\).
        \item The inductive case is given by \(\xi = n+1\)
        \begin{align*}
            \frac{n(n+1)}{2} + \xi &= \frac{n(n+1)}{2} + \frac{2n}{2} + \frac{2}{2} \\
            &= \frac{n^2 + 3n + 2}{2} \\
            &= \frac{(n+1)(n+2)}{2} \\
            &= \frac{\xi(\xi+1)}{2}
        \end{align*}
    \end{itemize}
\end{snippetexercise}

\begin{snippetexercise}{induction-ex-2}{Weak induction}
    Prove \(n\factorial > n^2\) for \(n \geq 4\).
    The base case is \(4\factorial=24 > 4^2 = 16\).

    The induction step is to prove \(n\factorial > n^2 \implies (n +1)\factorial > {(n+1)}^2\).
    Note that \((n+1)\factorial=(n+1)n\factorial\).
    Since \(n\factorial > n^2\), then
    \begin{align*}
        n\factorial(n+1) &> n^2(n+1) \\
        n\factorial(n+1) &> n^3 + n^2
    \end{align*}
    Since \(n \geq 4\), \(n^3 + n^2 > {(n+1)}^2=n^2+2n+1\).
    Thus, by the transitive property, \((n+1)\factorial > {(n+1)}^2\).
\end{snippetexercise}

\begin{snippetexercise}{induction-ex-3}{Weak induction}
    Prove that
    \[
        \sum_{k=1}^n k^2 = \frac{n(n+1)(2n+2)}{6}
    \]
    \begin{itemize}
        \item The base case is \(\frac{6}{6}=1\).
        \item The induction case is given by
        \begin{align*}
            \frac{n(n+1)(2n+2)}{6} + {(n+1)}^2  &= \frac{n(n+1)(2n+2)}{6} + n^2 + 2b + 1 \\
            &= \frac{(n+1)(n+2)(2(n+1)+1)}{6}
        \end{align*}
    \end{itemize}
\end{snippetexercise}

\begin{snippetexercise}{induction-ex-4}{Weak induction}
    Prove that for every \(n \geq 0\) and for every \(x > -1\),
    \[
        {(1+x)}^n \geq 1 + nx
    \]
    \begin{itemize}
        \item The base case is \({(1+x)}^0 = 1 \geq 1\).
        \item The induction case is \({(1+x)}^n \geq 1 + nx \implies {(1+x)}^{n+1} \geq 1 + (n+1)x\).
            We split the term \({(1+x)}^{n+1}\) into \({(1+x)}^n(1+x)\).
            Using the induction hypothesis we have \[{(1+x)}^n(1+x) \geq (1+nx)(1+x)\]
            Now, expand the right-hand side \[(1+nx)(1+x) = 1 + nx + x + nx^2 = 1 + (n+1)x + nx^2\]
            Now, we need to show \[1+(n+1)x + nx^2 \geq 1 + (n+1)x\]
            We subtract \(1+(n+1)x\) from both sides to get \(nx^2 \geq 0\).
            Since \(n \geq 0\) and \(x > -1\), \(x^2 \geq 0\) for all \(x\), and thus \(nx^2 \geq 0\).
            Therefore, the inequality holds for \(n+1\).
    \end{itemize}
\end{snippetexercise}

\begin{snippetexercise}{induction-ex-5}{Strong induction}
    Prove that every integer is written as a product of primes.
    \begin{itemize}
        \item The base case is \(n=2\), which is a prime number.
        \item The induction case is given by assuming \(\forall k \leq m\),
        the number \(k\) is a product of primes. The next number \(n+1\)
        is either prime or not prime. In the first case it is prime and we can thus
        write it as a product of prime. In the latter case it is not prime, and thus can be written
        as \(n+1 = kh\) for \(2\leq k,h< n+1\).
        By the inductive hypothesis \(h\) and \(k\) can be written as a product of primes.
        The same goes for \(n+1=hk\). 
    \end{itemize}
\end{snippetexercise}

\begin{snippetexercise}{inductionn-ex-6}{Weak induction}
    Prove \(m \geq \ln(m)\) for \(m \geq 1\).
    \begin{itemize}
        \item The base case is \(1 \geq \ln(1)=0\).
        \item The induction case is \(m + 1\geq \ln(m + 1) \implies m \geq \ln(m)\).
        Note that \(m + 1 - 1 \geq \ln(m-1) + 1 = \ln(m-1) + \ln(e)\)
        and thus \(m + 1 - 1 \geq \ln(e\cdot(m-1)) = \ln(em-e) \geq \ln(m + e - e) = \ln(m)\).
    \end{itemize}
\end{snippetexercise}

\begin{snippetexercise}{inductionn-ex-7}{Weak induction}
    Prove \(n\factorial \geq 2^{n-1}\) for \(n \geq 1\).
    \begin{itemize}
        \item The base case is \(1 \geq 2^0\).
        \item The induction case is \(n\factorial \geq 2^{n-1} \implies (n+1)\factorial \geq 2^n\).
        We have that \[m\factorial(n+1) \geq (n+1)2^{n-1} = n2^{n-1} + 2\cdot2^{n-1}\]
        The last term is greater or equal than \(2^n\) since \(n\neq 1\).
    \end{itemize}
\end{snippetexercise}

\begin{snippetexercise}{induction-ex-8}{Weak induction}
    Consider the sequence
    \[
        \begin{cases}
            x_0 = 0 \\
            x_{n+1} = 2x_n + 1
        \end{cases}        
    \]
    Prove that \(x_n = 2^n - 1\).
    \begin{itemize}
        \item The base case is \(x_0 = 0\).
        \item The induction case is \(x_{n} = 2^n - 1 \implies x_{n+1} = 2^{n+1}-1\).
        We have that \(x_{n+1} = 2x_n + 1\) which is by the induction step
        \[2(2^n - 1) + 1 = 2^{n+1} - 1\]
    \end{itemize}
\end{snippetexercise}

\begin{snippetexercise}{inductionn-ex-9}{Weak induction}
    Prove \[
        \sum_{k=0}^n q^k = \frac{q^{n+1} - 1}{q-1}
    \]
    for \(0<q<1\).
    \begin{itemize}
        \item The base case is \(1 = 1\).
        \item The induction case is
        \begin{align*}
            \sum_{k=0}^{n + 1} q^k = q^{n+1} + \sum_{k=0}^n q^k &= \frac{q^{n+1} - 1}{q-1} + q^{n+1} \\
            &= \frac{q^{n+2}-1}{q-1}
        \end{align*}
    \end{itemize}
\end{snippetexercise}

\begin{snippetexercise}{inductionn-ex-10}{Weak induction}
    Prove \(\frac{1}{m} \geq \frac{1}{m+1}\)
    for \(m \geq 1\).
    \begin{itemize}
        \item The base case is \(1 \leq \frac{1}{2}\).
        \item The induction case is
        \begin{align*}
            \frac{1}{m+1} = \frac{m+1-1}{m+1} &= 1- \frac{m}{m+1} = \frac{1}{\displaystyle \frac{1}{1 + \frac{1}{m}}} \\
            & \leq 1 + \frac{1}{m} \leq 1 + \frac{1}{m-1}
        \end{align*}
        which implies
        \[
            \frac{1}{m+1} \leq 1 - \frac{1}{\displaystyle \frac{1}{1 + \frac{1}{m}}} = \frac{1}{m}
        \]
    \end{itemize}
\end{snippetexercise}

\begin{snippetexercise}{inductionn-ex-11}{Weak induction}
    Prove that
    \[
        \sum_{i=0}^n i(i+1) = \frac{n(n+1)(n+2)}{3}
    \]
    \begin{itemize}
        \item The base case is \(0 = 0\).
        \item The induction case is
        \begin{align*}
            \sum_{i=0}^{n+1} i(i+1) &= (k+1)(k+2) + \sum_{i=0}^n i(i+1) \\ 
            &= (k+1)(k+2) + \frac{n(n+1)(n+2)}{3} \\
            &= (k+1)(k+2)\left(\frac{k}{3} + 1\right) \\
            &= \frac{n(n+1)(n+2)}{3}
        \end{align*}
    \end{itemize}
\end{snippetexercise}

% We can start at n+1 and go to the n

\end{document}
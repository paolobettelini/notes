\documentclass[preview]{standalone}

\usepackage{amsmath}
\usepackage{amssymb}
\usepackage{stellar}
\usepackage{definitions}

\begin{document}

\id{subsequences}
\genpage

\section{Subsequences}

\begin{snippetdefinition}{subsequence-definition}{Subsequence}
    Given a \sequence \(\varphi\colon \naturalnumbers \fromto X\), a \emph{subsequence} of \(\varphi\) is a sequence of the form
    \[
        \varphi' = \varphi \circ \psi
    \]
    where \(\psi\colon \naturalnumbers \fromto \naturalnumbers\) is a strictly increasing \function.  
    In terms of notation, a subsequence of \(\{a_n\}\) is often written as  
    \[
        \{a_{n_k}\} \quad \quad {\{a_{n_k}\}}_{k=l}^\infty
    \]
    for some \(l\in\naturalnumbers\),
    where \(\{n_k\}\) is an increasing sequence.
\end{snippetdefinition}

\begin{snippettheorem}{sequence-subsequence-limit-equivalence-theorem}{}
    Let \(\{x_n\}\) be a \sequence and let \(\{n_k\}\) be a strictly increasing \sequence in \(\naturalnumbers\).
    Then, the following statements are equivalent:
    \begin{enumerate}
        \item \(\{x_n\} \seqconverges \xi\);
        \item \(\{x_{n_k}\} \seqconverges \xi\);
        \item from every \subsequence \(\{x_{n_k}\}\) of \(\{x_n\}\) we can extract a further \subsequence \(\{x_{n_{k_j}}\}\) such that \(\{x_{n_{k_j}}\} \seqconverges \xi\);
    \end{enumerate}
    where \(\xi \in \extendedrealnumbers\).
\end{snippettheorem}

\begin{snippetproof}{sequence-subsequence-limit-equivalence-theorem-proof}{sequence-subsequence-limit-equivalence-theorem}{}
    \begin{itemize}
        \item \emph{\((1) \implies (2)\)}: there exists \(N \in \naturalnumbers\) such that
            \[
                \forall n > N, x_n \in I(\xi, \varepsilon, M)
            \]
            for all \(\varepsilon,M>0\).
            Since \(\{n_k\}\) is strictly increasing, we have \(n_k > N\) \eventually.
            Thus,
            \[
                \forall k > K, x_{n_k} \in I(\xi, \varepsilon, M)
            \]
            for some \(K\in\naturalnumbers\), which implies
            \(\{x_{n_k}\} \seqconverges \xi\).
        \item \emph{\((2) \implies (1)\)}: since \(\{n_k \seqconverges \xi\}\), every \subsequence of it
        also converges to \(\xi\).
        \item \emph{\((3) \implies (1)\)}: we will prove that if \(\{x_n\}\) does not converge to \(\xi\),
            then there exist a \subsequence \(\{x_{n_k}\}\) such that no \subsequence of it converges to \(\xi\).
            This means that there exist.
    \end{itemize}
\end{snippetproof}

\end{document}
\documentclass[preview]{standalone}

\usepackage{amsmath}
\usepackage{amssymb}
\usepackage{stellar}
\usepackage{definitions}

\begin{document}

\id{limits-exercises}
\genpage

\section{Exercises}

% TODOURGENT: put this in a separate thing for the limit of sinx/x
% If we consider the trigonometric circle with an angle \(\alpha\),
% the cord between \(\alpha\) and \(-\alpha\) has length \(2\sin\alpha\).
% The arc associated with it has angle \(2\alpha\).
% So, their ratio is \(\frac{\sin \alpha}{\alpha}\).
% As \(\alpha\) tends to \(0\), the ratio goes to \(1\).
% TODOURGENT: illustration

% TODOURGENT: same thing for cosx/x = 0
% use cos2x = 1-2sin^2 x

\begin{snippetproposition}{sine-over-linear}{}
    \[
        \lim_{x \to 0} \frac{\sin(x)}{x} = 1
    \]
\end{snippetproposition}

\begin{snippetproof}{sine-over-linear-proof}{sine-over-linear}{}
    \todo
\end{snippetproof}

\begin{snippetproposition}{cosine-over-linear}{}
    \[
        \lim_{x \to 0} \frac{\cos(x)}{x} = 0
    \]
\end{snippetproposition}

\begin{snippetproof}{cosine-over-linear-proof}{cosine-over-linear}{}
    \todo
\end{snippetproof}

\end{document}
\documentclass[preview]{standalone}

\usepackage{amsmath}
\usepackage{amssymb}
\usepackage{stellar}
\usepackage{definitions}

\begin{document}

\id{limits-exercises}
\genpage

\section{Exercises}

\begin{snippetproposition}{fundamental-trig-limit}{Fundamental trigonometric limit}
    \[
        \lim_{x \to 0} \frac{\sin(x)}{x} = 1
    \]
\end{snippetproposition}

\begin{snippetproof}{fundamental-trig-limit-hopital-proof}{fundamental-trig-limit}{Fundamental trigonometric limit - L'Hôpital Rule}
    Using \lhopitalrule we get
    \[
        \lim_{x \to 0} \frac{\sin(x)}{x} = \frac{\cos(x)}{1} = 1
    \]
\end{snippetproof}

\begin{snippetproof}{fundamental-trig-limit-squeeze-proof}{fundamental-trig-limit}{Fundamental trigonometric limit - Squeeze theorem}
    We start by considering the following inequalities
    \[ \cos(x) \leq \frac{\sin(x)}{x} \leq 1, \quad x>0 \]
    and
    \[ \cos(x) \geq \frac{\sin(x)}{x} \geq 1, \quad x<0 \]
    As \(x\) tends to \(0\), both \(\cos(x)\) and \(1\) tend to \(1\)
    \[ \lim_{x\to 0} \cos(x)=1 \]
    Thus, by the \squeezetheorem we get that
    \[
        \lim_{x \to 0} \frac{\sin(x)}{x} = \frac{\cos(x)}{1} = 1
    \]
\end{snippetproof}

\begin{snippetproposition}{cosine-over-linear}{}
    \[
        \lim_{x \to 0} \frac{\cos(x)}{x} = 0
    \]
\end{snippetproposition}

\begin{snippetproof}{cosine-over-linear-proof}{cosine-over-linear}{}
    \todo
    % TODOURGENT
    % use cos2x = 1-2sin^2 x
\end{snippetproof}

\begin{snippetexercise}{limits-ex1}{}
    Prove that
    \[
        \lim_{x\to 1} \sqrt{x^2 + 3} = 2
    \]
\end{snippetexercise}

\begin{snippetsolution}{limits-ex1-sol}{}
    We must verify that \(\forall \varepsilon > 0\) there exists \(\delta>0\)
    such that \(\forall x \in \mathbb{R}\) we have \(0<|x-1|<\delta\)
    implies \(|f(x) - 2| < \varepsilon\).
    So we take \(\varepsilon > 0\).
    We study the inequality
    \[
        |\sqrt{x^2 + 3} - 2| < \varepsilon
    \]
    and determine a \(\delta > 0\) such that the inequality holds for every \(x \in (1-\delta, 1 + \delta)\)
    and possibly \(x \neq 1\).
    We therefore want
    \[
        2 - \varepsilon < \sqrt{x^2 + 3} < 2 + \varepsilon
    \]
    Suppose \(\varepsilon < 1\) and thus \(2-\varepsilon>0\).
    Since everything is positive, we square both sides and obtain
    \[
        \begin{cases}
            (x^2 + 3) < {(2 + \varepsilon)}^2 \\
            x^2 + 3 > {(2 - \varepsilon)}^2
        \end{cases}
        \rightarrow
        \begin{cases}
            x^2 < 1 + 4\varepsilon + \varepsilon^2 \\
            x^2 > 1 - 4\varepsilon + \varepsilon^2
        \end{cases}
    \]
    So, the first becomes \(|x| < \sqrt{1 + 4\varepsilon + \varepsilon^2}\),
    while the second is always true if \(1-4\varepsilon+\varepsilon^2 < 0\),
    and in the case in which it is greater than or equal to zero, then \(|x|>\sqrt{1-4\varepsilon+\varepsilon^2}\).
    We choose \(\varepsilon<\frac{1}{4}\) and then \(1 - 4\varepsilon + \varepsilon^2 > 0\)
    and the inequality \(|\sqrt{x^2 +3} - 2| < \varepsilon\)
    is equivalent to
    \[
        \begin{cases}
            |x| < \sqrt{1 + 4\varepsilon + \varepsilon^2} \\
            |x| > \sqrt{1 - 4\varepsilon + \varepsilon^2}
        \end{cases}
    \]
    Now suppose that \(x>0\), given the nature of the limit, so
    \[
        \sqrt{1 - 4\varepsilon + \varepsilon^2} < x < \sqrt{1 + 4\varepsilon + \varepsilon^2}
    \]
    We then choose
    \[
        \delta = \min\{\sqrt{1 - 4\varepsilon + \varepsilon^2}, \sqrt{1 + 4\varepsilon + \varepsilon^2}\}
    \]
\end{snippetsolution}

\end{document}
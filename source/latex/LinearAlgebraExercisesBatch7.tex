\documentclass[preview]{standalone}

\usepackage{amsmath}
\usepackage{amssymb}
\usepackage{stellar}
\usepackage{definitions}

\begin{document}

\id{linearalgebra-exercises-batch-7}
\genpage

\section{Exercises}

\begin{snippetexercise}{linear-algebra-batch-7-ex-1}{}
    \todo
\end{snippetexercise}

\begin{snippetsolution}{linear-algebra-batch-7-ex-1-proof}{}
    \todo
\end{snippetsolution}

\begin{snippetexercise}{linear-algebra-batch-7-ex-2}{}
    \todo
\end{snippetexercise}

\begin{snippetsolution}{linear-algebra-batch-7-ex-2-proof}{}
    \todo
\end{snippetsolution}

\begin{snippetexercise}{linear-algebra-batch-7-ex-3}{}
    \todo
\end{snippetexercise}

\begin{snippetsolution}{linear-algebra-batch-7-ex-3-proof}{}
    \todo
\end{snippetsolution}

\begin{snippetexercise}{linear-algebra-batch-7-ex-4}{}
    \todo
\end{snippetexercise}

\begin{snippetsolution}{linear-algebra-batch-7-ex-4-proof}{}
    \todo
\end{snippetsolution}

\begin{snippetexercise}{linear-algebra-batch-7-ex-5}{}
    Let \(A\in \matrices_n(\mathbb{F})\) and consider the \function
    \(\text{tr}\colon \matrices_n(\mathbb{F})\to \mathbb{F}\) defined by
    \[
        \text{tr}(A) \triangleq \sum_{i=1}^n A_{ii}
    \]
    \begin{enumerate}
        \item Show that \(\text{tr}\) is a \lineartransformation;
        \item Compute \(\lineardim \grpker \text{tr}\).
    \end{enumerate}
\end{snippetexercise}

\begin{snippetsolution}{linear-algebra-batch-7-ex-5-proof}{}
    \begin{enumerate}
        \item \begin{enumerate}
            \item \emph{omogeneity:} \[
                \text{tr}(\lambda A) = \sum_{i=1}^n (\lambda A)_{ii} = \sum_{i=1}^n \lambda A_{ii} = \lambda \sum_{i=1}^n A_{ii} = \lambda \text{tr}(A)
            \]
            \item \emph{additivity:} \[
                \text{tr}(A+B) = \sum_{i=1}^n (A+B)_{ii} = \sum_{i=1}^n A_{ii} + \sum_{i=1}^n B_{ii} = \text{tr}(A) + \text{tr}(B)
            \]
        \end{enumerate}
        \item we have
        \[
            \lineardim \grpker \text{tr} = \lineardim \matrices_n(\mathbb{F}) - \lineardim \image \text{tr}
            = n^2 - 1
        \]
    \end{enumerate}
\end{snippetsolution}

\begin{snippetexercise}{linear-algebra-batch-7-ex-6}{}
    \todo
\end{snippetexercise}

\begin{snippetsolution}{linear-algebra-batch-7-ex-6-proof}{}
    \todo
\end{snippetsolution}

\begin{snippetexercise}{linear-algebra-batch-7-ex-7}{}
    \todo
\end{snippetexercise}

\begin{snippetsolution}{linear-algebra-batch-7-ex-7-proof}{}
    \todo
\end{snippetsolution}

\begin{snippetexercise}{linear-algebra-batch-7-ex-8}{}
    \todo
\end{snippetexercise}

\begin{snippetsolution}{linear-algebra-batch-7-ex-8-proof}{}
    \todo
\end{snippetsolution}

\begin{snippetexercise}{linear-algebra-batch-7-ex-13}{}
    Consider the \function \(\varphi\colon \matrices_n(\mathbb{F})\to \matrices_n(\mathbb{F})\) defined by
    \[
        \varphi(A) = \frac{1}{2}\left(A + A^\transpose\right)
    \]
    \begin{enumerate}
        \item Show that \(\varphi\) is a \lineartransformation;
        \item Show that \(\varphi\) is a \linearprojection;
        \item Find \(\image \varphi\) and \(\grpker \varphi\);
    \end{enumerate}
\end{snippetexercise}

\begin{snippetsolution}{linear-algebra-batch-7-ex-13-sol}{}
    \begin{enumerate}
        \item \begin{enumerate}
            \item \emph{omogeneity:} \[
                \varphi(\lambda A) =
                \frac{1}{2}\left(\lambda A + \lambda A^\transpose\right) = \frac{\lambda}{2}\left(A + A^\transpose\right) = \lambda \varphi(A)
            \]
            \item \emph{additivity:} \[
                \varphi(A+B) = \frac{1}{2}\left(A+B + (A+B)^\transpose\right) = \frac{1}{2}\left(A + A^\transpose + B + B^\transpose\right) = \varphi(A) + \varphi(B)
            \]
        \end{enumerate}
        \item \begin{align*}
            \varphi(\varphi(A)) &= \frac{1}{2} \left(
                \frac{1}{2}\left(A + A^\transpose\right)
                +
                {\left(\frac{1}{2}\left(A + A^\transpose\right)\right)}^\transpose
            \right) \\
            &= \frac{1}{4} \left(
                A + A^\transpose + A\transpose + A
            \right) \\
            &= \frac{1}{2} \left(A + A^\transpose\right) = \varphi(A)
        \end{align*}
        \item To have an idea of the transformation, consider \[
        \varphi\left(
            \begin{bmatrix}
                a & b & c \\
                d & e & f \\
                g & h & i
            \end{bmatrix}
        \right)
        =
        \begin{bmatrix}
            a & \frac{b+d}{2} & \frac{g+c}{2} \\
            \frac{b+d}{2} & e & \frac{f+h}{2} \\
            \frac{g+c}{2} & \frac{f+h}{2} & i
        \end{bmatrix}
    \]
    We can clearly see that
    \[
        \image \varphi = \{
            A\in \matrices_n(\mathbb{F}) \suchthat A = A^\transpose
        \}
    \]
    and
    \[
        \grpker \varphi = \{
            A\in \matrices_n(\mathbb{F}) \suchthat A = -A^\transpose
        \}
    \]
    \end{enumerate}
\end{snippetsolution}

\end{document}
\documentclass[preview]{standalone}

\usepackage{amsmath}
\usepackage{amssymb}
\usepackage{stellar}
\usepackage{bettelini}

\hypersetup{
    colorlinks=true,
    linkcolor=black,
    urlcolor=blue,
    pdftitle={Biologia},
    pdfpagemode=FullScreen,
}

\begin{document}

\title{Biologia}
\id{biologia-acidi-nucleici}
\genpage

\section{Acidi nucleici}

\plain{I monomeri degli aicid nucleici si chiamano <b>nucleotidi</b>.}

\begin{snippetdefinition}{acido-nucleico-definition}{Acido nucleico}
    L'\textit{acido nucleico} è composto da un gruppo fosfato, zucchero e base azotata.
\end{snippetdefinition}

\begin{snippetdefinition}{dna-definition}{DNA}
    Il \textit{DNA} è composto da due filamenti di nucleotidi.
\end{snippetdefinition}

\plain{I nucleotidi del DNA sono 4 (A, C, G, T).}

\end{document}
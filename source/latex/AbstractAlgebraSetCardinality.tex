\documentclass[preview]{standalone}

\usepackage{amsmath}
\usepackage{amssymb}
\usepackage{stellar}
\usepackage{definitions}
\usepackage{bettelini}
\usepackage{tikz}

\usetikzlibrary{cd}

\begin{document}

\id{settheory-cardinality}
\genpage

\section{Cardinality}

\begin{snippetaxiom}{axiom-of-choice}{Axiom of choice}
    Let \(A\) be a \set where \(A \neq \emptyset\).
    There exist a \function
    \[
        f\colon \powerset(A) \fromto A
    \]
    such that
    \[
        \forall X \subset A \suchthat A \neq\emptyset, f(x) \in x
    \]
\end{snippetaxiom}

\begin{snippetdefinition}{finite-set-definition}{Finite set}
    Let \(A\) be a \set.
    Then, \(A\) is \emph{finite} if
    \[
        \exists n\in\naturalnumbers, \exists \varphi \colon \naturalnumbers^\exceptzero_{\leq n} \fromto A
    \]
    such that \(f\) is \bijective.
\end{snippetdefinition}

% TODOURGENT F_n è univoco e esiste

\begin{snippetdefinition}{cardinality-definition}{Cardinality}
    The \textit{cardinality} of a \set \(A\) is a measure of the amount of its elements.
    \begin{itemize}
        \item if \(A\) is \setfinite, \(|A| = n\) where
        \(n\in\naturalnumbers\) such that \(\exists \varphi \colon \naturalnumbers^\exceptzero_{\leq n} \fromto A\)
        where \(\varphi\) is \bijective.
        \item if \(A\) is not \setfinite, let \(B\) be another \set. \\
        We say
        \[
            |A| = |B| \iff \exists f \colon A \fromto B \suchthat f \text{ is \bijective}
        \]
        We say
        \[
            |A| \leq |B| \iff \exists f \colon A \fromto B \suchthat f \text{ is \injective}
        \]
    \end{itemize}
\end{snippetdefinition}

\begin{snippetdefinition}{set-equivalence-definition}{Set equivalence}
    Let \(A\) and \(B\) be \set[sets].
    Then, the \emph{set equivalence} is defined as
    \[
        A \sim B \iff \cardinality{A} = \cardinality{B}
    \]
\end{snippetdefinition}

\begin{snippetproposition}{set-cardinality-transitivity}{Cardinality transitivity}
    Let \(X,Y,Z\) be \set[sets]. Then,
    \[
        \cardinality{X} \leq \cardinality{Y} \leq \cardinality{Z}
        \implies \cardinality{X} \leq \cardinality{Z}
    \]
\end{snippetproposition}

\begin{snippetproof}{set-cardinality-transitivity-proof}{set-cardinality-transitivity}{Cardinality transitivity}
    By definition, \(\cardinality{X} \leq \cardinality{Y}\) means that there exist an \injective
    \function \(f\colon X \fromto Y\). \\
    Likewise, \(\cardinality{Y} \leq \cardinality{Z}\) means that there exist an \injective
    \function \(f\colon Y \fromto Z\). \\
    Thus, there exist an \injective \function \(h\colon X \fromto Z\), and thus \(\cardinality{X} \leq \cardinality{Z}\).
\end{snippetproof}

\begin{snippettheorem}{cantor-schroeder-bernstein-theorem}{Cantor-Schroeder-Bernstein Theorem}
    Let \(X,Y\) be \set[sets]. Then, assuming the \axiomofchoice
    \[
        \cardinality{A} \leq \cardinality{B}
        \land
        \cardinality{B} \leq \cardinality{A}
        \implies A \setequivalent B
    \]
\end{snippettheorem}

\section{Countability}

\begin{snippetdefinition}{countable-set-definition}{Countable set}
    A \set \(S\) is \textit{countable} if there exists
    a \function \(f\colon S \fromto \naturalnumbers\) that is \injective.
\end{snippetdefinition}

\begin{snippettheorem}{infinite-sets-characterization-theorem}{Characterization of infinite sets}
    \begin{enumerate}
        \item \emph{Smallest cardinality:} let \(X\) be an infinite \set. Then,
        \[
            \exists X_0 \subseteq X \suchthat X_0 \setequivalent \naturalnumbers
        \]
        \item A \set \(X\) is infinite \ifandonlyif
        \[
            \exists X_1 \subset X \suchthat X_1 \setequivalent X
        \]
    \end{enumerate}
\end{snippettheorem}

\begin{snippetproof}{infinite-sets-characterization-theorem-proof}{infinite-sets-characterization-theorem}{Characterization of infinite sets}
    \begin{enumerate}
        \item Let \(X\) be an infinite \set, meaning that for every \(n\in\naturalnumbers\),
        there is no \bijective \function from \(X\) to \(\naturalnumbers^\exceptzero_{\leq n}\).
        Let \(x_1 \in X\). We have that \(X \neq \{x_1\}\) since otherwise it would be \setfinite.
        Thus, \(\exists x_2 \neq x_1 \suchthat x_2 \in X\).
        Now, \(X \neq \{x_1, x_2\}\) since otherwise it would be \setfinite.
        By \principleofinduction[induction], this iterated process does not end.
        We now find a \sequence
        \[
            X_0 = \{x_1, x_2, \cdots, x_n, \cdots\} \subseteq X
        \]
        such that \(\forall n > m, x_n \neq x_m\). The elements are all distinct.
        By construction we have a \bijective \function from \(X_0\) to \(\naturalnumbers\).
        \item \iffproof{
            If \(X\) is \setfinite, there cannot be a \bijective \function
            from \(X\) to a proper subset of \(X\).
        }{
            Let \(X\) be an infinite \set. By the first point, there exist
            \[
                X_0 = \{x_1, x_2, \cdots, x_n, \cdots\} \setequivalent \naturalnumbers
            \]
            We now consider \(X_1 = \{x_2, x_4, \cdots, x_{2k}\} \subseteq X_0\) \\
            and define \(\varphi\colon (X \difference X_0) \union X_1 \fromto X\)
            such that
            \[
                \varphi(x) = \begin{cases}
                    x & x \in X \difference X_0 \\
                    x_k & x = x_{2k}
                \end{cases}
            \]
        }
    \end{enumerate}
\end{snippetproof}

\begin{snippettheorem}{countable-union-is-countable-theorem}{Countable union is countable}
    Let \(\{X_n\}\) be \countable \set[sets]. Then,
    \[
        \bigcup_{n} X_n
    \]
    is \countable.
\end{snippettheorem}

\begin{snippetproof}{countable-union-is-countable-theorem-proof}{countable-union-is-countable-theorem}{Countable union is countable}
    Since \(X_n\) is \countable, consider
    \[
        X_n = \{x_{n,1}, x_{n,2}, \cdots\}
    \]
    We display every \set \(X_n\) in a matrix
    \begin{center}
        % https://tikzcd.yichuanshen.de/#N4Igdg9gJgpgziAXAbVABwnAlgFyxMJZABgBpiBdUkANwEMAbAVxiRAA8B9YARlJ4C+IAaXSZc+Qij6Vq9Zq0QdufAExCRY7HgJFV5KrUYs2AHVMBjKBBwJNIDNslEAzAbnHFy3qQC2G0QdxHSkSfkN5EyUuYH1BYUDHCV1pcI8FNhj9dQStZNC4iM8zS2tbXKCnFOQ3HiKM6O59fwqkkKIyVXqo71qAvPbUrvSemLcc+zbnFGzur3MoMrtE4Oma0mGjBt6-fsr8jtIXOZKaJdbV6r5jkfnTM5tlgbX9G62e8wfyycvQtzfIncvk99oN1gAWE6NYBgXYXKoFUiQ24lKyPeEHFBkZHvLwxWHxH4Ioh8HGAzLcWETQwwKAAc3gRFAADMAE4QXxIMggHAQJCCQJsjn86i8pDg+xCzmIcGivmIAAckvZ0oVcqQAFZlcLEBr1Yh1IKVUh9Dz5S5tdK3GakAA2S12-UATgdiCd+p46goAiAA
        \begin{tikzcd}
            {x_{1,1}} \arrow[r]  & {x_{1,2}} \arrow[ld] & \cdots \arrow[r]  & {x_{1,m}} \arrow[ld] \\
            {x_{2,1}} \arrow[d]  & {x_{2,2}} \arrow[ru] & \cdots \arrow[ld] & {x_{2,m}}            \\
            {x_{3,1}} \arrow[ru] & {x_{3,2}} \arrow[ld] & \ddots            & {x_{3,m}}            \\
            \vdots               & \vdots               & \vdots            & \vdots               \\
            {x_{n,1}}            & {x_{n,2}}            & \cdots            & {x_{n,m}}           
        \end{tikzcd}
    \end{center}
    we consider the \sequence induced by the arrow and thus the union is \countable.
\end{snippetproof}

\begin{snippetcorollary}{cartesian-prod-is-countable}{Cartesian product is countable}
    Let \(X\) and \(Y\) be \countable \set[sets].
    Then, \(X \cartesianprod Y\) is \countable.
\end{snippetcorollary}

\begin{snippetproof}{cartesian-prod-is-countable-proof}{cartesian-prod-is-countable}{Cartesian product is countable}
    \todo
\end{snippetproof}

\begin{snippetcorollary}{unitary-interval-reals-bijection}{}
    The \function \(f\colon (-1;1) \fromto \realnumbers\)
    such that
    \[
        f(x) = \frac{1}{1-|x|}
    \]
    is \bijective.
\end{snippetcorollary}

\begin{snippettheorem}{reals-are-uncountable-theorem}{Reals are uncountable}
    Since there is a \bijective \function between \((0;1)\) with \(\realnumbers\),
    we will prove that \((0;1)\) is not \countable.
    Assume that \((0;1)\) is \countable. Then,
    \[
        (0;1) = \{r_1, r_2, \cdots\}
    \]
    where each number has a numeric expansion \(r_n = 0.d_{n,1}d_{n,2}\cdots\)
    in any base \(B\). Note that we only want to consider unique numbers, so we discard any (period)
    representation of values which are already present.
    We now construct a real number \(r = g_1g_2\cdots\)
    where \(g_i = d_{i,i} + 1 \bmod{B} \in (0;1)\).
    Now, for every \(r_i\) in the \sequence, \(g_i \neq r_i\) because they differ
    at at the \(i\)-th decimal \lightning.
\end{snippettheorem}

\begin{snippetaxiom}{continuum-hypothesis-theorem}{Continuum Hypothesis}
    There is not \set \(X\)
    such that
    \[
        \cardinality{\mathbb{N}} < \cardinality{X} < \cardinality{\mathbb{R}}
    \]
\end{snippetaxiom}

\begin{snippetproposition}{power-set-naturals-and-reals}{}
    \[
        \powerset(\naturalnumbers)
        \setequivalent \realnumbers
    \]
\end{snippetproposition}

\begin{snippetproof}{power-set-naturals-and-reals-proof}{power-set-naturals-and-reals}{}
    We only consider \((0,1) \subset \realnumbers\).
    Any \(x\in (0,1)\) can be written asa binary expansion.
    In general, given a \set \(X\), \[
        \powerset(X) \setequivalent \{\chi \suchthat \chi \colon X \fromto \{0,1\}\}
    \]
    where given \(E \subseteq X\),
    \[
        \chi_E(x) \triangleq \begin{cases}
            1 & x \in E \\
            0 & x \notin E
        \end{cases}
    \]
    Meaning that \[\powerset(\naturalnumbers) \setequivalent \{\chi \suchthat \chi \colon X \fromto \{0,1\}\}\]
    The \set \((0,1)\) has a \bijective mapping with the binary expansions that are not eventually \(1\),
    meaning with the non-empty subsets of \(\naturalnumbers\)
    that are not of the form \(E \union F\), with \(F = \{n \suchthat n \geq n_0\}\)
    and \(E\) is \setfinite.
    The \set of \(\chi\) \function[functions] that are eventually \(1\)
    have a \bijective mapping with the \set of \function[functions] that are eventually null.
    The mapping is constructed by associating the finite representation of a number to the
    infinite expansion which is eventually \(1\). Furthermore, the \set
    of fiinte expansions is countable, as it is the union of finite expansions
    \[
        \{\chi \colon \naturalnumbers \fromto \{0,1\} \suchthat \exists n_0 \suchthat \forall n \geq n_0, \chi(0) =0\}
        = \bigcup_{n_0} \{\chi \colon \naturalnumbers \fromto \{0,1\} \suchthat \forall n \geq n_0, \chi(n) = 0\}
    \]
    We now construct a mapping from the finite expansions to those with period \(1\)
    \[
        \{\chi\colon \naturalnumbers \fromto \{0,1\} \suchthat \chi \text{ \eventually }\; 0 \}
        \setequivalent
        \{\chi\colon \naturalnumbers \fromto \{0,1\} \suchthat \chi \text{ \eventually }\; 0 \text{ or } 1\}
    \]
    Thus, \((0,1) \setequivalent \powerset(\naturalnumbers)\).
\end{snippetproof}

\begin{snippettheorem}{cardinality-strictly-less-than-power-set-theorem}{}
    For every \set \(X\),
    \[
        \cardinality{X} < \cardinality{\powerset(X)}
    \]
\end{snippettheorem}

\begin{snippetproof}{cardinality-strictly-less-than-power-set-theorem-proof}{cardinality-strictly-less-than-power-set-theorem}{}
    We note that the \function \(f \colon X \fromto \powerset(X)\)
    defined as \(f(x) = \{x\}\) is \injective and thus
    \[
        \cardinality{X} \leq \powerset(X)
    \]
    We now need to show that there is no mapping
    \[
        \varphi\colon X \fromto \powerset(X)
    \]
    that is \surjective.
    Assume that such \function exists and define
    \(A \subseteq X\) as
    \[
        A = \{a\in X \suchthat a \notin \varphi(a)\}
    \]
    Since \(\varphi\) is \surjective, \(\exists a_0 \in X \suchthat \varphi(a_0) = A\).
    We now have two cases:
    \begin{itemize}
        \item \(a_0\in A\), meaning \(a_0 \notin A = \varphi(a_0)\) \lightning;
        \item \(a_0\notin A\), meaning \(a_0 \in A\) \lightning.
    \end{itemize}
    Thus, \(\cardinality{X} < \cardinality{\powerset(X)}\).
\end{snippetproof}

\begin{snippetproposition}{square-side-to-square-bijection}{}
    \[
        \{(x,y) \suchthat x,y \in [0;1)\} \setequivalent [0;1)
    \]
\end{snippetproposition}

\begin{snippetproof}{square-side-to-square-bijection-proof}{square-side-to-square-bijection}{}
    The proof is done by considering the decimal expansions of \(x\) and \(y\)
    and constructing a number that alternates their digits.
\end{snippetproof}

% TODOURGENT |X cartesian Y| = max(|X|, |Y|)

\end{document}
\documentclass[preview]{standalone}

\usepackage{amsmath}
\usepackage{amssymb}
\usepackage{stellar}
\usepackage{definitions}
\usepackage{bettelini}

\begin{document}

\id{geofisica-svizzera-aspetti-economici}
\genpage

\section{Aspetti economici}

\begin{snippet}{composizione-strutturale-economica}
    La \quotes{composizione strutturale di un'economia} si riferisce alla distribuzione delle diverse attività
    economiche tra i vari settori. Questa composizione varia nel tempo e tra le regioni, influenzando la
    struttura economica e sociale di un paese.
\end{snippet}

\subsection{Modelli di Sviluppo Economico}

\begin{snippetdefinition}{modello-dei-tre-settori-definition}{Modello dei tre settori}
    Il \textit{modello dei tre settori}, sviluppato da Allan Fisher, Colin Clark e Jean Fourastié,
    divide l'economia in tre settori di attività (primario, secondario e terziario).
\end{snippetdefinition}

\includesnpt[width=80\%|src=/snippet/static/fourastie.png]{centered-img}

\subsection{Svizzera e Ticino}

\begin{snippet}{settori-economici-svizzera-ticino}
    L'analisi economica della Svizzera e del Ticino evidenzia come i diversi settori
    contribuiscano al PIL e all'occupazione. Il settore terziario ha acquisito un ruolo
    predominante, mentre il settore primario ha visto una riduzione significativa nel corso degli
    anni.
    L'economia svizzera si caratterizza per una forte predominanza del settore terziario, un'importante
    presenza del settore secondario con industrie ad alta tecnologia e una ridotta ma significativa
    attività nel settore primario. L'evoluzione economica segue il modello di sviluppo di Fourastié /
    Clark-Fisher, evidenziando una transizione verso un'economia basata sui servizi e sulle attività
    tecnologiche avanzate.
\end{snippet}

\section{Caratteristiche dell'Agricoltura in Svizzera}

\begin{snippet}{caratteristiche-agricolatura-svizzera}
    L'agricoltura in Svizzera è influenzata da vari fattori geografici, economici e sociali. Ecco alcune
    delle principali caratteristiche e dati salienti:
    \begin{enumerate}
        \item \textbf{Regioni con Maggiore Percentuale di Impiegati nel Settore Primario:}
            Le regioni alpine e prealpine hanno una maggiore percentuale di impiegati nel
            settore primario. Queste aree sono tradizionalmente dedicate all'agricoltura e
            all'allevamento a causa della conformazione del territorio e delle tradizioni locali.
        \item \textbf{Percentuale Complessiva di Persone Attive nell'Agricoltura:}
            La percentuale di persone attive nell'agricoltura in Svizzera è relativamente bassa
            rispetto ad altri settori economici, riflettendo una tendenza verso
            l'industrializzazione e i servizi. Tuttavia, l'agricoltura rimane un settore importante
            per la sostenibilità e l'autosufficienza alimentare del paese.
        \item \textbf{Superfici Coltivate e Unità di Bestiame:}
            Le regioni con la maggior superficie coltivata sono le pianure, come il Mittelland e la
            valle del Rodano. Queste aree sono caratterizzate da terreni fertili e un clima
            favorevole per la coltivazione di una varietà di colture.
            Il maggior numero di unità di bestiame si trova nelle regioni alpine e prealpine, dove
            l'allevamento di bovini è una pratica comune a causa della disponibilità di pascoli.
        \item \textbf{Autosufficienza Agricola:}
            La Svizzera raggiunge o quasi l'autosufficienza per alcuni prodotti agricoli, con una
            produzione interna che supera il 75\% del consumo totale. Questi prodotti includono
            latte e derivati, carne di manzo, pollame, patate e cereali.
        \item \textbf{Prodotti Vegetali e Superfici Coltivate:}
            La maggior parte delle superfici coltivate in Svizzera è dedicata a cereali (come
            frumento e mais), patate e ortaggi. La coltivazione di vigneti e frutteti è anche
            significativa, specialmente nelle regioni più calde come il Canton Ticino e il Vallese.
    \end{enumerate}

    L'agricoltura svizzera affronta diverse sfide, tra cui la pressione urbanistica, il cambiamento
    climatico, la globalizzazione dei mercati e le esigenze di sostenibilità ambientale. Tuttavia, le
    politiche agricole del paese mirano a supportare gli agricoltori attraverso sussidi, formazione e
    incentivi per pratiche agricole sostenibili.
\end{snippet}

\end{document}
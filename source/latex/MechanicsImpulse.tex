\documentclass[preview]{standalone}

\usepackage{amsmath}
\usepackage{amssymb}
\usepackage{stellar}
\usepackage{definitions}
\usepackage{bettelini}

\begin{document}

\id{mechanics-impulse}
\genpage

\section{Impulse}

\begin{snippetdefinition}{impulse-definition}{Impulse}
    \emph{Impulse} is defined as the vector integral
    \[
        \vec{I} \triangleq \integral[t_0][t_1][\vec{F}(t)][t]
    \]
\end{snippetdefinition}

\begin{snippettheorem}{impulse-theorem}{Impulse theorem}
    The impulse is the change in momentum
    \[
        \vec{I} = \vec{p}(t_2) - \vec{p}(t_1) = m(\vec{v}(t_2) - \vec{v}(t_1))
    \]
\end{snippettheorem}

\begin{snippetproof}{impulse-theorem-proof}{impulse-theorem}{Impulse theorem}
    We substitute Newton's second law
    \[
        \vec{F}(t) = \frac{d\vec{p}}{dt}
    \]
    into the definition of impulse
    \begin{align*}
        \vec{I} &= \integral[t_1][t_2][\vec{F}(t)][t]
        = \integral[t_1][t_2][\frac{d\vec{p}}{dt}][t]
        = \vec{p}(t_2) - \vec{p}(t_1)
    \end{align*}
\end{snippetproof}

\end{document}
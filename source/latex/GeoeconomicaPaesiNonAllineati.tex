\documentclass[preview]{standalone}

\usepackage{amsmath}
\usepackage{amssymb}
\usepackage{stellar}
\usepackage{definitions}
\usepackage{bettelini}

\begin{document}

\id{geoeconomica-paesi-non-allineati}
\genpage

\section{Paesi non allineati}

\begin{snippetdefinition}{paesi-non-allineati-definition}{I Paesi Non Allineati}
    Con \textit{I Paesi Non Allineati} si intende una coalizione di nazioni che durante la Guerra Fredda
    si sono distinte per la loro posizione neutrale e indipendente rispetto alle potenze mondiali (USA e URSS).
\end{snippetdefinition}

\plain{Fondato nel 1961, il Movimento dei Paesi Non Allineati (MPNA) era composto da nazioni principalmente dell'Africa, dell'Asia e dell'America Latina, che condividevano l'obiettivo di preservare la loro sovranità nazionale, promuovere la cooperazione internazionale e difendere il diritto all'autodeterminazione.}



\end{document}
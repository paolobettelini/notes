\documentclass[preview]{standalone}

\usepackage{amsmath}
\usepackage{amssymb}
\usepackage{parskip}
\usepackage{fullpage}
\usepackage{hyperref}
\usepackage{stellar}
\usepackage{definitions}
\usepackage{bettelini}

\begin{document}

\id{taylor-series}
\genpage

\section{Prerequisites}

\begin{snippetproposition}{power-rule-nth-derivative}{\(n\)-th derivative of \(x^n\)}
    The \(n\)-the derivative of \(x^n\) for \(n \in {\naturalnumbers}^*\)
    is given by
    \[
        \derivativeD^n x^n = n\factorial
    \]
\end{snippetproposition}

\begin{snippetproof}{power-rule-nth-derivative-proof}{power-rule-nth-derivative}{\(n\)-th derivative of \(x^n\)}
    \begin{align*}
        \frac{d^n}{dx^n}\left(x^n\right)
        &=\frac{d^{(n-1)}}{dx^{(n-1)}}\left(nx^{(n-1)}\right)\\
        &=\frac{d^{(n-2)}}{dx^{(n-2)}}\left(n(n-1)x^{(n-2)}\right)\\
        &=\cdots\\
        &=\prod_i^n i=n\factorial
    \end{align*}
\end{snippetproof}

\begin{snippetlemma}{centered-polynomial-derivative}{Centered polynomial derivative}
    Let
    \[
        P(x) = \sum_{k=0}^n a_k{(x-x_0)}^k
    \]
    be a \polynomial where \(\polynomialdeg P(x) \leq n\) in \(x-x_0\).
    Then,
    \[
        \derivativeD^m P(x_0) = \sum_{k=0}^n a_k \derivativeD^m {(x-x_0)}^k
        = \begin{cases}
            0 & m \neq k \\
            m\factorial = k\factorial & m = k
        \end{cases}
    \]
\end{snippetlemma}

\section{Introduction}

\begin{snippet}{taylor-series-introduction-derivation}
    We want to construct a power series centered around the point \(x=a\),
    so the variable will be \(x-a\).
    \begin{align*}
        \sum_{n=0}^{\infty}c_n{(x-a)}^n
    \end{align*}
    The goal is to find the coefficients \(c_n\).
    We first notice that \(f(a)=c_0\), which is the only coefficient that does not multiply a variable.
    If we take the derivative, the coefficient \(c_1\) will lose its variable
    \begin{align*}
        c_1+2c_2(x-a)+3c_3{(x-a)}^2+\cdots
    \end{align*}
    Now we have \(f'(a)=c_1\).
    \\
    We take the derivative of the polynomial again
    \begin{align*}
        2c_2+6c_3(x-a)+\cdots
    \end{align*}
    This time we have
    \begin{align*}
        f''(a)&=2c_2\\
        c_2&=\frac{f''(a)}{2}
    \end{align*}
    And again
    \begin{align*}
        f'''(a)&=6c_3\\
        c_3&=\frac{f'''(a)}{6}
    \end{align*}

    More generally, to extract the coefficient \(c_n\) we take the n-th derivative of the function. \\
    This brings us to
    \begin{align*}
        c_n=\frac{f^{(n)}(a)}{n\factorial}
    \end{align*}
    and finally to
    \begin{align*}
        \sum_{n=0}^{\infty}\frac{{(x-a)}^n f^{(n)}(a)}{n\factorial}
    \end{align*}
    which might not converge and not coincide with the function.
\end{snippet}

\section{Taylor Polynomial}

\begin{snippetdefinition}{taylor-polynomial-definition}{Taylor polynomial}
    Let \(f\colon I \fromto \realnumbers\) where \(f\)
    is differentiable \(n\) times in \(x_0\) where \(x_0\) is an internal point of \(I\).
    The \emph{Taylor polynomial of order \(n\) centered in \(x_0\)}
    is defined as
    \[
        P_n(f, x_0, x) \triangleq \sum_{k=0}^n
        \frac{f^{(k)}(x_0) \cdot {(x-x_0)}^k}{k!}
    \]
\end{snippetdefinition}

\begin{snippetdefinition}{maclaurin-polynomial-definition}{MacLaurin polynomial}
    A \emph{MacLaurin polynomial} is a Taylor polynomial centered around \(x_0 = 0\).
\end{snippetdefinition}

\begin{snippetproposition}{taylor-polynomial-unique-property}{}
    The Taylor polynomial of \(f\) centered around \(x_0\)
    up to the order \(n\) is the only \polynomial such that
    \(\forall k \leq n\),
    \[
        \derivativeD^k P_n(x_0) = \derivativeD^kf(x_0)
    \]
\end{snippetproposition}

\plain{The next theorem extends the properties of differentiable functions, allowing them to be approximated with
higher degree polynomial}

\begin{snippettheorem}{taylor-theorem}{Taylor's theorem}
    Let \(f\colon I \to \realnumbers\) be differentiable
    \(n\) times in \(x_0\) whre \(x_0\) is an internal point of \(I\) and let
    \[
        P_n(x) = \sum_{k=0}^n \frac{f^{(k)}(x_0){(x-x_0)}^k}{k!}
    \]
    be its Taylor polynomial of order \(n\)
    centered around \(x_0\).
    Then,
    \begin{enumerate}
        \item \emph{Taylor formula with Peano form of the remainder:} \[
            f(x) = P_n(x) + r_n(x), \quad \littleO({(x-x_0)}^n) \to 0 \text{ for } x\to \infty
        \]
        \item \emph{Taylor formula with Lagrange form of the remainder:} assume that \(f\) is differentiable \(n+1\) times
        in \(I\), then
        \[
            f(x) = P_n(x) + r_n(x)
        \]
        where \(r_n\) is represented as follows:
        \(\forall x\in I\), there exist \(c_x\) between \(x_0\) and \(x\)
        (which generally both depends on \(x\) and \(n\)), such that
        \[
            r_n(x) = \frac{f^{(n+1)}(c_x)}{(n+1)!} (x-x_0)^{n+1}
        \]
    \end{enumerate}
\end{snippettheorem}

\begin{snippetproof}{taylor-theorem-proof}{taylor-theorem}{Taylor's theorem}
    \begin{enumerate}
        \item Abbiamo
        \begin{align*}
            \lim_{x\to x_0} \frac{r_n(x)}{{(x-x_0)}^n} &=
            \lim_{x\to x_0} \frac{f(x) - P_n(x)}{{(x-x_0)}^n} \\
            &\overset{H}{=} \lim_{x\to x_0} \frac{f'(x) - P_n'(x)}{n{(x-x_0)}^{n-1}} \\
        \end{align*}
        Siccome \(\forall k \leq n, \derivativeD^k P(x_0) = \derivativeD^k f(x_0)\) possiamo applicare nuovamente
        il teormea de l'Hôpital iterativamente.
        \begin{align*}
            &\overset{H}{=}\lim_{x\to x_0} \frac{
                \derivativeD^2f(x) - \derivativeD^2P_n(x)
            }{
                n(n-1){(x-x_0)}^{n-1}
            } \\
            &\overset{H}{=}\lim_{x\to x_0} \frac{
                \derivativeD^{n-1}f(x) - \derivativeD^{n-1}P_n(x)
            }{
                n(n-1) \cdots 2(x-x_0)
            }
        \end{align*}
        A questo punto non possiamo più applicare il teorema in quanto
        si necessiti che la funzione sia derivabiel in un intorno, ma
        possiamo necessariamente derivarla solo in \(x_0\).
        Tuttavia, \(\derivativeD^n(f(x_0)) = \derivativeD^n P_n(x_0)\). Quindi,
        \begin{align*}
            &= \frac{1}{n!} \lim_{x\to x_0}
            \frac{\derivativeD^{n-1}f(x) - \derivativeD^{n-1} f(x_0)}{x-x_0}
            - \frac{\derivativeD^{n-1}P_n(x) - \derivativeD^{n-1}P_n(x_0)}{x-x_0} \\
            &= \frac{1}{n!} \left\{
                \derivativeD^n f(x_0) - \derivativeD^n P_n(x_0)
            \right\} \\
            &= 0
        \end{align*}
        \item Mostriamo che per tutte le \(x\) esiste \(c\)
        compreso tra \(x_0\) e \(x\) tale che
        \[
            r_n(x) = f(x) - P_n(x) = \frac{1}{(n+1)!} \derivativeD^{n+1}f(c)(x-x_0)^{n+1}
        \]
        equivalentemente
        \[
            \frac{f(x) - P_n(x)}{{(x-x_0)}^{n+1}} = \frac{1}{(n+1)!} \derivativeD^{n+1}f(c)
        \]
        Vogliamo applicare il teorema di Cauchy. Poniamo \(F(x) = f(x) - P_n(x)\)
        e \(G(x) = {(x-x_0)}^{n+1}\) e notiamo che
        \(F(0) = 0\) e infatti \(\forall k \leq n, \derivativeD^k F(x_0) = 0\),
        e analogamente \(G(x_0) = 0\) e infatti \[\forall k \leq n, D^k {(x-x_0)}^{n+1} = (n+1)n\cdots(n-k){(x-x_0)}^{n-k+1} = 0\]
        in \(x=x_0\).
        Allora,
        \[
            \frac{f(x) - P_n(x)}{{(x-x_0)}^{n+1}} = \frac{F(x) - F(x_0)}{G(x) - G(x_0)}
        \]
        notando che \(F,G\) sono derivabili \(n+1\) volte in \(I\)
        e \(G(x) - G(x_0) \neq 0\), e \(\derivativeD^k G(x) \neq 0\) per \(x\neq x_0\)
        per ogni \(t = 1 \cdots n+1\).
        Esiste quindi \(x_1\) compreso fra \(x\) e \(x_0\) tale che
        \[
            \frac{F(x) - F(x_0)}{x-x_0} = \frac{F'(x_1)}{G'(x_1)}
            = \frac{F'(x_1) - F'(x_0)}{G'(x_1) - G'(x_0)}
        \]
        Usiamo allora il teorema di Cauchy nuovamente: esiste un \(x_2\)
        fra \(x_1\) e \(x\)
        \begin{align*}
            = \frac{\derivativeD^2 F(x_2)}{\derivativeD^2 G(x_2)} = \frac{\derivativeD^2 F(x_0) - \derivativeD^2 F(x_0)}{\derivativeD^2 G(x_2) - \derivativeD^2 G(x_0)}
        \end{align*}
        e ripetiamo iterativamente quanto possiamo: esiste \(c\)
        compreso tra \(x_n\) e \(x_0\)
        \begin{align*}
            = \frac{\derivativeD^{n+1} F(c)}{\derivativeD^{n+1} G(c)}
            = \frac{1}{(n+1)} \derivativeD^{n+1}F(c) = (n+1)!
        \end{align*}
        Abbiamo quindi
        \[
            \frac{1}{(n+1)!} \left\{
                \derivativeD^{n+1}f(c) - \derivativeD^{n+1}P_n(c)
            \right\} = \frac{1}{(n+1)!} \derivativeD^{n+1}f(c)
        \]
    \end{enumerate}
\end{snippetproof}

\begin{snippettheorem}{taylor-uniqueness-theorem}{Taylor uniqueness theorem}
    Sia \(f\colon I \to \realnumbers\) derivabile \(n\) volte in \(x_0\)
    interno ad \(I\).
    Supponiamo che
    \[
        P(x) = \sum_{k=0}^n a_k {(x-x_0)}^k
    \]
    sia un polinomio di grado minore o uguale a \(n\)
    tale che \(f(x) = P(x) + \littleO({(x-x_0)}^n)\).
    Allora, \(P(x) = P_n(x)\).
    In particolare,
    \[
        \forall k \leq n, a_k = \frac{\derivativeD^k f(x_0)}{k!}
    \]
\end{snippettheorem}

\begin{snippetproof}{taylor-uniqueness-proof}{taylor-uniqueness}{Taylor uniqueness theorem}
    Poiché \(f\) è derivabile \(n\) volte in \(x_0\)
    dalla Fromula di Taylor conr esto Peano risulta che
    \[
        f(x) = P_n(x) +\littleO({(x-x_0)}^n)
    \]
    risulta quindi
    \[
        P(x) - P_n(x) = [f(x) -P_n(x)]
        - [f(x) - P(x)] = \littleO({(x-x_0)}^n)
    \]
    Posto \(Q(x) = P(x) - P_n(x)\) la dimostrazione si riduce a mostrare che se \(Q(x)\)
    è un polinomio in \((x-x_0)\) di grado minore o uguale a \(n\)
    e \(Q(x) = \littleO({(x-x_0)}^n)\) allora \(Q(x) = 0\)
    cioè tutti i coefficienti di \(Q\) sono nulli.
    Scriviamo
    \[
        Q(x) = \sum_{k=0}^n b_k{(x-x_0)}^k
    \]
    e supponiamo per assurdo che
    \[
        \frac{Q(x)}{{(x-x_0)}^n} \to 0
    \]
    ma \(Q(x) \neq 0\).
    Tra tutti i coefficienti non nulli, ce n'è uno con indice più piccolo \(k_0\),
    quindi tutti quelli prima sono nulli.
    Abbiamo quindi che \[
        Q(x) = \sum_{k=k_0}^n b_k {(x-x_0)}^k
    \]
    Siccome il termine tende a zero, raccogliamo la potenza più piccola
    \[
        Q(x) = {(x-x_0)}^{k_0}\sum_{k=k_0}^n b_k {(x-x_0)}^{t-t_0}
    \]
    da cui
    \[
        \frac{Q(x)}{{(x-x_0)}^n} = {(x-x_0)}^{k_0 - n}
        \sum_{k=k_0}^n b_k {(x-x_0)}^{k-k_0}
    \]
    la sommatoria tende a \(b_{k_0}\) e il termine esterno
    \(\pm \infty\) per \(k_0 < n\), altrimenti \(1\) (o magari non esiste il limite).
    Quindi, \(Q(x)\) non tende a zero \lightning.
\end{snippetproof}

\end{document}
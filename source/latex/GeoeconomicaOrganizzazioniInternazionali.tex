\documentclass[preview]{standalone}

\usepackage{amsmath}
\usepackage{amssymb}
\usepackage{stellar}
\usepackage{bettelini}

\hypersetup{
    colorlinks=true,
    linkcolor=black,
    urlcolor=blue,
    pdftitle={Stellar},
    pdfpagemode=FullScreen,
}

\begin{document}

\title{Stellar}
\id{geoeconomica-organizzazioni-internazionali}
\genpage

\section{Forum esclusivi e organizzazioni internazionali}

\begin{snippet}{classificazion-forum-orgainzzazioni-internazionali}
    Un'organizzazione viene detta un forum informale se non è uno statuto giuridico.
    In caso affermativo, se l'organizzazione è statale viene detta
    organizzazione internazionale intergovernative (OIG), altrimenti viene detta
    organizzazione non governative (ONG).
\end{snippet}

\begin{snippetdefinition}{organizzazione-internazionale-intergovernativa-definizione}{Organizzazione internazionale intergovernativa}
    Un'\textit{organizzazione internazionale intergovernativa} (OIG) è un'associazione di Stati costituita da un
    trattato (contrariamente alle organizzazioni non governative create da persone private).
    Essa consi ste in una persona giuridica distinta da quella degli Stati e funziona grazie a degli organi comuni
    previsti dai suoi statuti.
\end{snippetdefinition}

\begin{snippet}{2627a7a1-bd3c-4131-beb1-201e8624724c}
    È dunque una struttura di cooperazione tra Stati che persegue degli scopi di
    interesse comune definiti nel suo atto costitutivo.
    L'organizzazione internazionale è il simbolo dell'evoluzione delle relazioni internazionali. Gli Stati
    sono maestri del gioco poiché essi hanno il potere di creare delle organizzazioni internazionali, ma
    le loro creature rischiano di sfuggire al loro controllo. Se gli Stati, specialmente quelli più potenti,
    controllano le organizzazioni con l'arma efficace del contributo finanziario, essi non hanno più il monopolio delle relazioni internazionali.
    Si distinguono due forme di organizzazioni governative: quelle \quotes{universali} e quelle \quotes{regionali}.
\end{snippet}

% https://moodle.edu.ti.ch/libe/pluginfile.php/122248/mod_resource/content/0/Pass_06f_OrganizzazioniEForum_23-24_schede.pdf

\end{document}
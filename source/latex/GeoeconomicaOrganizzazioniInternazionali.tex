\documentclass[preview]{standalone}

\usepackage{amsmath}
\usepackage{amssymb}
\usepackage{stellar}
\usepackage{bettelini}

\hypersetup{
    colorlinks=true,
    linkcolor=black,
    urlcolor=blue,
    pdftitle={Stellar},
    pdfpagemode=FullScreen,
}

\begin{document}

\title{Stellar}
\id{geoeconomica-organizzazioni-internazionali}
\genpage

\section{Forum esclusivi e organizzazioni internazionali}

\begin{snippet}{classificazion-forum-orgainzzazioni-internazionali}
    Un'organizzazione viene detta un forum informale se non è uno statuto giuridico.
    In caso affermativo, se l'organizzazione è statale viene detta
    organizzazione internazionale intergovernative (OIG), altrimenti viene detta
    organizzazione non governative (ONG).
\end{snippet}

\begin{snippetdefinition}{organizzazione-internazionale-intergovernativa-definizione}{Organizzazione internazionale intergovernativa}
    Un'\textit{organizzazione internazionale intergovernativa} (OIG) è un'associazione di Stati costituita da un
    trattato (contrariamente alle organizzazioni non governative create da persone private).
    Essa consi ste in una persona giuridica distinta da quella degli Stati e funziona grazie a degli organi comuni
    previsti dai suoi statuti.
\end{snippetdefinition}

\begin{snippet}{2627a7a1-bd3c-4131-beb1-201e8624724c}
    È dunque una struttura di cooperazione tra Stati che persegue degli scopi di
    interesse comune definiti nel suo atto costitutivo.
    L'organizzazione internazionale è il simbolo dell'evoluzione delle relazioni internazionali. Gli Stati
    sono maestri del gioco poiché essi hanno il potere di creare delle organizzazioni internazionali, ma
    le loro creature rischiano di sfuggire al loro controllo. Se gli Stati, specialmente quelli più potenti,
    controllano le organizzazioni con l'arma efficace del contributo finanziario, essi non hanno più il monopolio delle relazioni internazionali.
    Si distinguono due forme di organizzazioni governative: quelle \quotes{universali} e quelle \quotes{regionali}.
\end{snippet}

% https://moodle.edu.ti.ch/libe/pluginfile.php/122248/mod_resource/content/0/Pass_06f_OrganizzazioniEForum_23-24_schede.pdf

\section{ONU}

\begin{snippetdefinition}{onu-definition}{Organizzazione delle Nazioni Unite}
    L'\textit{Organizzazione delle Nazioni Unite}
    è un'organizzazione intergovernativa a carattere mondiale.
    Tra i suoi obiettivi principali vi sono il mantenimento della pace e
    della sicurezza mondiale, lo sviluppo di relazioni amichevoli tra le nazioni,
    il perseguimento di una cooperazione internazionale e il favorire
    l'armonizzazione delle varie azioni compiute a questi scopi dai suoi membri.
\end{snippetdefinition}

% multilateralismo

\begin{snippet}{onu-componenti}
    L'ONU è composto da diversi componenti:
    \begin{itemize}
        \item \textbf{Assemblea generale:}
            È il principale organo deliberativo dell'ONU, dove sono rappresentati
            tutti gli Stati membri. Ogni Stato ha un voto.
            L'Assemblea Generale discute e adotta risoluzioni su questioni rilevanti per
            la pace e la sicurezza, lo sviluppo internazionale, i diritti umani e
            altri temi globali.
        \item \textbf{Consiglio di Sicurezza:}
            è responsabile del mantenimento della pace e della
            sicurezza internazionali.
            È composto da 15 membri, di cui 5 permanenti con diritto di veto
            (Stati Uniti, Regno Unito, Francia, Russia, Cina) e 10 non permanenti eletti
            per periodi di due anni. Il Consiglio di Sicurezza può autorizzare
            operazioni di pace, sanzioni e azioni militari.
        \item \textbf{Segretariato:}
            è l'organo amministrativo dell'ONU. Il Segretario Generale
            è il capo del Segretariato. Questo organo esegue il lavoro quotidiano dell'ONU e implementa le decisioni degli altri organi.
    \end{itemize}
    e altri ancora.
\end{snippet}

\plain{La Svizzera è entrata nell'ONU nel 2002.}

\begin{snippetdefinition}{diritto-di-veto-definition}{Diritto di veto}
    Con \textit{diritto di veto} si intende
    potere riconosciuto al membro di \underline{un} organo deliberante di bloccarne una decisione. 
\end{snippetdefinition}

\begin{snippet}{problema-onu-veto}
    L'ONU si ritrova spesso con la mani legate in quanto
    molte decisioni vengono bloccate dai diritti di veto, che incrociandosi
    bloccano molte iniziative.
    Il diritto di veto è esso stesso soggetto al diritto di veto, in quanto
    la sua rimozione dovrebbe essere presa dal consiglio di sicurezza.
\end{snippet}

\end{document}
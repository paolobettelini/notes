\documentclass[preview]{standalone}

\usepackage{amsmath}
\usepackage{amssymb}
\usepackage{stellar}
\usepackage{definitions}

\begin{document}

\id{rings-quotient}
\genpage

\section{Quotient rings}

\begin{snippetdefinition}{quotient-ring-definition}{Quotient ring}
    Let \(A\) be a \ring
    and \(I \idealin A\) a \ideal[bilateral ideal].
    The \emph{quotient ring} is defined as
    \[
        A/I \triangleq \{
            a + I \suchthat a \in A   
        \}
    \]
    with addition
    \[
        (a+I) + (b+I) \triangleq (a+b) + I
    \]
    and multiplication
    \[
        (a+I) \cdot (b+I) \triangleq (a \cdot b) + I
    \]
\end{snippetdefinition}

\plain{This is the equivalence class of the equivalence where two elements are equivalent
if their difference is in the lateral.}

\begin{snippetproposition}{quotient-ring-is-ring}{Quotient ring is a ring}
    Let \(A\) be a \ring
    and \(I \idealin A\) a \ideal[bilateral ideal].
    Then, \(A/I\) is a \ring.
\end{snippetproposition}

\begin{snippetproof}{quotient-ring-is-ring-proof}{quotient-ring-is-ring}{Quotient ring is a ring}
    \begin{enumerate}
        \item the neutral element is
        \[
            0_{A/I} = 0_A + I
        \]
        \item the identity element is
        \[
            0_{A/I} = 1_A + I
        \]
        \item the addition is well-defined: TODO.
        \item the multiplication is well-defined:
        let \(a' \in a + I, b' \in b+I\),
        which means that \(a' = a+x, b' =b+y\) for some \(x,y \in I\).
        \begin{align*}
            (a'+I) \cdot (b' + I)
            &= (ab+ay+xb + xy) + I
            = ab + I
        \end{align*}
        since the \ideal is bilateral we can hide all the other terms.
        \item the product is associative and the distributive properties are satisfied
        since they are satisfied in the \ring.
    \end{enumerate}
\end{snippetproof}



\end{document}
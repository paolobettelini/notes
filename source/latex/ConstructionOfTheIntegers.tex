\documentclass[preview]{standalone}

\usepackage{amsmath}
\usepackage{amssymb}
\usepackage{stellar}
\usepackage{definitions}

\begin{document}

\id{construction-of-the-integers}
\genpage

\section{Construction of the naturals}

\includesnpt{axiom-of-induction}

\begin{snippetdefinition}{natural-numbers-definition}{Peano axioms}
    \newcommand{\successor}{\labelref["Successor function"][\scolorweak[black]S]}
    The \textit{Peano axioms} are axiom for the natural numbers \(\mathbb{N}\):
    \begin{enumerate}
        \item an initial value \(0\) is a number in \(\mathbb{N}\);
        \item every number \(n\in\mathbb{N}\) has a \textit{successor} \(\successor(n)\in\mathbb{N}\);
        \item if \(m\neq n\), then \(\successor(m) \neq \successor(n)\);
        \item the number \(0\) is not the successor of any number;
        \item the \axiomofinduction.
    \end{enumerate}
\end{snippetdefinition}

\section{Construction of the integers}

\plain{Starting from the naturals, we construct the integers.}

\begin{snippetdefinition}{integers-definition}{Integers}
    We consider the pairs of naturals \(A \subseteq \naturalnumbers \cartesianprod \naturalnumbers\)
    and construct the \equivrelation \((a,b) \sim (c,d) \iff a+d=b+c\).
    Is idea is to complete \(\naturalnumbers\) with the opposites.
    We define the operations in the \quotset \(A /_\sim\) as
    \[
        {[(a,b)]}_\sim + {[(c,d)]}_\sim = {[(a+c, b+d)]}_\sim
    \]
    We need to show that these operations are well-defined (i.e. they don't depend of the representative
    chosen for a given equivalence class).
    We also define the multiplication is an analogous way. \\
    We then define an \injective[injective function] \[f\colon \naturalnumbers \fromto A /_\sim\]
    which we will denote as \(\mathbb{Z}\), such that \(n \to {[(n,0)]}_\sim\).
    % We use the zero to constraint the various possibilities
    We need to verify that the operations and the ordering are respected.
    \[n \to {[(n,0)]}_\sim\]
    \[m \to {[(m,0)]}_\sim\]
    \[n+n \to {[(m+n,0)]}_\sim = {[(n,0)]}_\sim + {[(m,0)]}_\sim \]
    Finally, we identify in such a way \(\naturalnumbers\) with a subset of \(\mathbb{Z}\).
    In other words, the operations in \(\mathbb{Z}\) extend the ones in \(\naturalnumbers\). \\
    We can then prove that \(\mathbb{Z}\) is a commutative \ring.
\end{snippetdefinition}


% \section{Set-theoretic definition of natural numbers}

\end{document}
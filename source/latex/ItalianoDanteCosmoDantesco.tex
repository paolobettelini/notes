\documentclass[preview]{standalone}

\usepackage{amsmath}
\usepackage{amssymb}
\usepackage{stellar}

\hypersetup{
    colorlinks=true,
    linkcolor=black,
    urlcolor=blue,
    pdftitle={Stellar},
    pdfpagemode=FullScreen,
}

\begin{document}

\title{Stellar}
\id{italiano-cosmo-dantesco}
\genpage

\section{Cosmo Dantesco}

\begin{snippetdefinition}{sistema-tolemaico-definition}{Sistema tolemaico}
    Data la credenza creazionista, Dio ha creato l'uomo e l'ha collocato al centro.
    Per cui, la Terra risiede al centro del sistema solare, dove gli altri pianeti gli ruotano attorno.
\end{snippetdefinition}

\begin{snippet}{cosmo-dantesco-expl}
    Le colonne d'Ercole (Stretto di Gibilterra) e La foce del Gange
    sono i due estremi della Terra. All'uomo non è concesso conoscere oltre questi confini
    (fare ciò implicherebbe peccare di superbia).
    
    Gerusalemme si trova al centro dell'emisfero. Sotto di esso, risiede l'inferno.
    
    Lucifero era l'angelo prediletto di Dio.
    Lucifero si ribella a Dio, e per punizione viene scagliato sulla Terra, la quale,
    prova ribrezzo e si ritira formando la forma conica dell'inferno. Lucifero si trova nel punto
    più profondo dell'inferno, ossia il centro della Terra, nonché il punto più lontano da Dio.
    
    La creazione dell'inferno crea una montagna dall'altra parte del mondo, dove in cima ad esso
    vi è il Giardino dell'Eden. Ciò marca anche la creazione del purgatorio.
\end{snippet}

\includesnpt[src=/snippet/static/sistema-tolemaico-dante.jpg|width=75\%]{centered-img}

\end{document}
\documentclass[preview]{standalone}

\usepackage{amsmath}
\usepackage{amssymb}
\usepackage{bettelini}
\usepackage{stellar}

\hypersetup{
    colorlinks=true,
    linkcolor=black,
    urlcolor=blue,
    pdftitle={Chimica},
    pdfpagemode=FullScreen,
}

\begin{document}

\title{Chimica}
\id{chimica-tecniche-separazione}
\genpage

\begin{snippetdefinition}{decantazione-definizione}{Decantazione}
    La \textit{decantazione} si usa di solito per separare due liquidi di densità diversa
    sfruttando la gravità.
\end{snippetdefinition}

\begin{snippetexample}{decantazione-example}{Decantazione}
    la separazione dell'olio e l'acqua.
\end{snippetexample}

\begin{snippetdefinition}{distillazione-definizione}{Distillazione}
    La \textit{distillazione} sfrutta i diversi punti di ebollizione di due liquidi per separarli.
    La miscela viene riscaldata fino a quando solo uno delle due componenti diventa vapore, per poi
    spostarla e riaffreddarla.
\end{snippetdefinition}

\begin{snippetdefinition}{cromatografia-definizione}{Cromatografia}
    La \textit{cromatografia} sfrutta la tendenza delle sostanze a sciogliersi o interagire
    con diverse specie chimiche.
\end{snippetdefinition}

\begin{snippetdefinition}{estrazione-definizione}{Estrazione}
    L'\textit{estrazione} si basa sulla maggiore o minore solubilità di un componente di un miscuglio in una certa miscela.
\end{snippetdefinition}

\begin{snippetdefinition}{filtrazione-definizione}{Filtrazione}
    TODO
\end{snippetdefinition}

\begin{snippetdefinition}{centrifugazione-definizione}{Centrifugazione}
    TODO
\end{snippetdefinition}

\end{document}

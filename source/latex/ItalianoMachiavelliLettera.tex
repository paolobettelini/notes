\documentclass[preview]{standalone}

\usepackage{amsmath}
\usepackage{amssymb}
\usepackage{stellar}
\usepackage{epigraph}
\usepackage{bettelini}

\hypersetup{
    colorlinks=true,
    linkcolor=black,
    urlcolor=blue,
    pdftitle={Stellar},
    pdfpagemode=FullScreen,
}

\begin{document}

\title{Stellar}
\id{italiano-machiavelli-lettera}
\genpage

\section{Lettera a Francesco Vettori}

\begin{snippet}{machiavelli-lettera-francesco-vettori}
    % Machiavelli segretario delfiglio di Papa Leone Xla repubblica, cade la repubblica e salgono i medici
    % cacciano machiavelli e scrive da questo albergaccio dove si trova.
    % Riesci a farsi reintegrare dai medici, e poi cambia di nuovo il potere e viene cacciato di nuovo
    Niccolò Machiavelli scrisse diverse lettere a vari destinatari durante la sua vita. La lettera a Francesco Vettori, datata 10 dicembre 1513, è una delle sue più famose. In questa lettera, Machiavelli discute degli eventi politici dell'epoca, in particolare il suo recente licenziamento da parte dei Medici dopo la caduta della Repubblica di Firenze.
    Nella lettera, Machiavelli esprime la sua delusione per essere stato escluso dalla vita politica e riflette sulle difficoltà e le instabilità della politica italiana. Egli condivide le sue preoccupazioni sulla situazione politica del tempo e sulla necessità di un principe forte e capace per riportare l'ordine e la stabilità nella regione.
    Machiavelli discute anche della natura umana, sottolineando la necessità per un governante di adattarsi alle circostanze e di essere disposto a prendere decisioni impopolari per il bene comune. La lettera riflette il suo pensiero politico, caratterizzato dalla realpolitik.
    \\\\
    L'esilio del 1512 denota uno spartiacque.
    L'anno prima incontrava il re di Francia, mentre ora
    le sue giornate sono monotone e non sa cosa fare.
    Tuttavia, da un certo punto Machiavelli comincia a tenere il proprio cervello vivo
    studiando (leggendo): questo è il senso della sua vita,
    rialzando la sua scrittura.

    \epigraph{``Mi pasco di quel cibo che solum è mio e ch'io nacqui per lui.''}{\textit{Niccolò Machiavelli}}

    I luoghi della sua giornata sono essenzialmente tre:
    \begin{itemize}
        \item \textbf{naturali} (luoghi aperti, bosco, etc.).
            Questi luoghi sono rappresentanti delle occupazioni pratiche (caccia di uccelli, commercio, taglio legna etc.);
        \item \textbf{la strada} (luogo di riflessione e di incontro con le persone dei paesi vicini);
        \item \textbf{osteria, scrittoio} (luoghi chiusi). Questi due posti sono quasi antitetici per la loro natura.
    \end{itemize}
    \phantom{}\\
    Un ragionamento analogo può essere svolgo circa i tempi della sua giornata:
    \begin{itemize}
        \item \textbf{mattino} (ocupazioni pratiche);
        \item \textbf{pomeriggio} (vita sociale);
        \item \textbf{sera e di notte} (tempo per sè, tempo dello studio).
    \end{itemize}
    \phantom{}\\
    La lettera è attraversata dal dialogo come tema dominante.
    In primis, tutta la lettera è un dialogo con Vettori.
    Nei contenuti, vi sono i dialoghi legati alle merci, legati al gioco/osteria,
    quando chiede notizie e, metaforicamente, dialoga con gli scrittori che legge.
    \\\\
    Attorno a questa lettera vi sono diversi problemi di inconsistenza.
    Per cominciare, la frase
    \begin{center}
        \textit{e} [ho] \textit{composto uno opuscolo De principatibus;}
    \end{center}
    sembra implicare che il trattato sia già finito alla scrittura della lettera.
    Tuttavia, quando viene indicato di cosa parla, vengono indicati solamente
    i principati come argomento, quando il trattato finale
    tratta:
    \begin{enumerate}
        \item Dedica;
        \item (Cap. 1-11) Principati;
        \item (Cap. 12-14) Armi;
        \item (Cap. 19-23) Il principe;
        \item (Cap. 24-26) \quotes{Italia contemporanea}.
    \end{enumerate}
    Il libro è quindi finito o no?
    \\\\
    Nella lettera Machiavelli indica che il dedicatario sarà Giuliano de' Medici,
    mentre dopo essere uscito la dedicata sarà a Lorenzo il giovane. % Non il magnifico (?)
    \\\\
    Machiavelli prima descrive il libro come un ghiribizzo, quasi come un piccolo gioco,
    ma successivamente dice che può essere usato da un principe nuovo.
    \\\\
    Il dubbio dilemmatico è quello di consegnare il libro.
    E se darlo, darglielo personalmente o per altri mezzi?
    La lettera è quindi piena di dissirio e molto problematica.
\end{snippet}

\begin{snippetdefinition}{stile-dilemmatco-definition}{Stile dilemmatico}
    Lo \textit{stile dilemmatico} consiste nel creare un albero decisionale.
\end{snippetdefinition}

\end{document}
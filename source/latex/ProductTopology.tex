\documentclass[preview]{standalone}

\usepackage{amsmath}
\usepackage{amssymb}
\usepackage{stellar}
\usepackage{definitions}
\usepackage{bettelini}
\usepackage{tikz}

\usetikzlibrary{cd}

\begin{document}

\id{product-topology}
\genpage

\section{Product topology}

\plain{We want to naturally induce the least coarse topology for the cartesian product of spaces.}

\begin{snippetdefinition}{product-topology-definition}{Product topology}
    Let \(\{X_i\}_{i \in I}\) be a family of \topologicalspace[topological spaces]
    indexed by some \(I \neq \emptyset\). Then, the \emph{product topology}
    is defined as the least coarse topology for
    \[
        X \triangleq \prod_{i\in I} X_i
    \]
    such that the canonical projections
    \[
        \pi_i \colon \prod_{j \in I} X_j \to X_i
    \]
    given by \(\{x_j\}_{j\in I} \fromto x_i\), are continuous.
\end{snippetdefinition}

\plain{We want to find the least coarse topology such that the two projections are continuous.}

\begin{snippetproposition}{product-topology-is-topology}{Product topology is topology}
    The product topology is a topology for the cartesian product of a family of spaces.
\end{snippetproposition}

% TODO la parte generica
\begin{snippetproof}{product-topology-is-topology-proof}{product-topology-is-topology}{Product topology is a topology}
    Consider the product topology for spaces \(X\) and \(Y\).
    The collection of subsets \(\{U \cartesianprod V \suchthat U \text{ \topologicalspace[open][Open set] in } X \land V \text{ \topologicalspace[open][Open set] in } Y\}\)
    is a \topologicalbasis for the space as:
    \begin{enumerate}
        \item \(X \cartesianprod Y = U \cartesianprod V\) for \(U=X\) and \(V=Y\);
        \item \((U_1 \cartesianprod V_1) \intersection (U_2 \cartesianprod V_2) = (U_1 \intersection U_2) \times (V_1 \intersection V_2)\).
    \end{enumerate}
\end{snippetproof}

\plain{The product topology between spaces is associative up to homeomorphism.}

\begin{snippettheorem}{product-topology-projections-theorem}{}
    Let \(P, Q\) be \topologicalspace[topological spaces]
    and \(p \colon P \cartesianprod Q \fromto P, q \colon P \cartesianprod Q \fromto Q\)
    be the two canonical projections. Then,
    \begin{enumerate}
        \item \(p, q\) are open maps;
        \item for each \((x,y) \in P \cartesianprod Q\), the restrictions
        \(\restr{p}{P \cartesianprod \{y\}}\) and \(\restr{q}{\{x\} \cartesianprod Q}\),
        which are the canonical projections to the subspaces \(P \cartesianprod \{y\}\)
        and \(\{x\} \cartesianprod Q\), are \homeomorphism[homeomorphisms];
        \item a map \(f \colon X \fromto P \cartesianprod Q\) is continuous
        \ifandonlyif \(p \circ f\) and \(q \circ f\) are continuous.
    \end{enumerate}
\end{snippettheorem}

\begin{snippetproof}{product-topology-projections-theorem-proof}{product-topology-projections-theorem}{}
    \begin{enumerate}
        \item[2.] The restrictions are obtained via composition of continuous maps.
            \begin{center}
                % https://tikzcd.yichuanshen.de/#N4Igdg9gJgpgziAXAbVABwnAlgFyxMJZABgBpiBdUkANwEMAbAVxiRAAUACAHW4GM6AJxzwsdMGkHQe3YAE9eAXxCLS6TLnyEUARnJVajFmzQq1IDNjwEiZHQfrNWiDrwHDR4ydICKKgzBQAObwRKAAZlIAtkhkIDgQSHqGTmy8gvA4gsBoisBcbkIi2F5SUDLySsqqEdGx1AlIAEwNdFgMbAAWEBAA1ma1EDGILfGJiMmOxi6mihSKQA
                \begin{tikzcd}
                P \cartesianprod \{y\} \arrow[r, "\restr{p}{P \cartesianprod \{y\}}"] \arrow[d, hook] & p \\
                P\cartesianprod Q \arrow[ru, "p"]                                                     &  
                \end{tikzcd}
            \end{center}
            To study whether the maps are open, we characterize the \topologicalspace[open sets][Open set]
            of \(P \cartesianprod \{y\}\) and \(\{x\} \cartesianprod Q\).
            Consider a generic \topologicalspace[open set][Open set] of the standard \topologicalbasis
            of \(P \cartesianprod Q\), \(U \cartesianprod V\). We have
            \[
                (U \cartesianprod V) \intersection (P \cartesianprod \{y\})
                = \begin{cases}
                    U \cartesianprod \{y\} & y \in V \\
                    \emptyset & y \notin V
                \end{cases}
            \]
            This means that the \topologicalspace[open sets][Open set] are precisely the subsets
            of the form \(U \cartesianprod \{y\}\) for some \topologicalspace[open set][Open set]
            of \(X\) \[ \restr{p}{P \cartesianprod \{y\}}(U \cartesianprod \{y\}) = U \]
            The same goes for \(q\).
        \item[1.] We want to show that \(p(Q)\) is an \topologicalspace[open set][Open set] of \(P\).
        Since \[ Q = \bigcup_{y\in Q} \{y\} \]
        we have that \[ P \cartesianprod Q = \bigcup_{y \in Q} P \cartesianprod \{y\} \]
        Now consider
        \[
            A = \bigcup_{y\in Q} A \intersection (P \cartesianprod \{y\})
        \]
        and thus
        \[
            p(A) = p\left(
                \bigcup_{y\in Q} A \intersection (P \cartesianprod \{y\})
            \right)
            = \bigcup_{y\in Q} \restr{p}{P \cartesianprod \{y\}} (A \intersection (P \cartesianprod \{y\}))
        \]
        which is \topologicalspace[open][Open set] in \(P\) as union of \topologicalspace[open set][Open set].
        The same goes for \(q\). 
        \item[3.] We have the following diagram
        \begin{center}
            % https://tikzcd.yichuanshen.de/#N4Igdg9gJgpgziAXAbVABwnAlgFyxMJZABgBoBGAXVJADcBDAGwFcYkQANEAX1PU1z5CKchWp0mrdgAUABAB15AY3oAnHPCz0waVdFkBFHnxAZseAkVEAmcQxZtEII737mhV0sTuTHIaTziMFAA5vBEoABmegC2SKIgOBBIAMw09lJOaCA0jPQARjCM0gIWwiCqWCEAFjjGUbHxNElI1um+7ACO9SDREHGIZInJiGkghWBQqUMZfmgKyliqSrKRPX0DQy2IbeMwk0gAtCkzHU6dC0pLK2uuvY2DzSMJs+y3lNxAA
            \begin{tikzcd}
                                                                                                    & P                                                 \\
            X \arrow[ru, "p \circ f", bend left] \arrow[rd, "q \circ f", bend right] \arrow[r, "f"] & P \cartesianprod Q \arrow[u, "p"'] \arrow[d, "q"] \\
                                                                                                    & Q                                                
            \end{tikzcd}
        \end{center}
        If \(f\) is continuous, then \(p \circ f\) and \(q \circ f\)
        are continuous as composition of continuous maps. \\
        We know that \(f\) is continuois \ifandonlyif for all \topologicalspace[open sets][Open set]
        of the standard \topologicalbasis of \(P \cartesianprod Q\),
        \(f^{-1}(U \cartesianprod V)\) is \topologicalspace[open][Open set] in \(X\)- We have
        \begin{align*}
            f^{-1}(U \cartesianprod V) &= f^{-1}(p^{-1}(U) \intersection q^{-1}(V)) \\
            &= f^{-1}(p^{-1})(U) \intersection f^{-1}(q^{-1}(V)) \\
            &= {(p \circ f)}^{-1}(U) \intersection {(q \circ f)}^{-1}(V)
        \end{align*}
        which is \topologicalspace[open][Open set] as intersection of
        \topologicalspace[open sets][Open set] by the continuity of the composition.
    \end{enumerate}
\end{snippetproof}

\begin{snippetexample}{topology-non-closed-canonical-projection-example}{}
    The canonical projections are not closed in general.
    Consider \(p \colon \realnumbers^2 \fromto \realnumbers\)
    as the projection onthe first factor.
    Let \(C = g^{-1}(0)\) where \(g\) is the continuous map
    given by \(g \colon \realnumbers^2 \fromto \realnumbers\)
    defined as \((x,y) \fromto xy-1\). We have
    \begin{align*}
        C = g^{-1}(\{0\}) = \{(x,y) \in \realnumbers \suchthat xy=1\}
    \end{align*}
    and thus \(p(C)\) is the \set of \(x\in \realnumbers\)
    such that \(\exists y\) with \((x,y) \in C\).
    But \(\realnumbers \difference \{0\}\) is not a \closedset in \(\realnumbers\)
    and thus \(p\) is not a closed map.
\end{snippetexample}

\end{document}
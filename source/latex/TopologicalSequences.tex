\documentclass[preview]{standalone}

\usepackage{amsmath}
\usepackage{amssymb}
\usepackage{stellar}
\usepackage{definitions}
\usepackage{bettelini}

\newcommand\ts{{(X, \mathcal{T})}}

\begin{document}

\id{topological-sequences}
\genpage

\section{Sequences}

\plain{Defining convergence on topological spaces is a bit tricky because there is
no notion of distance (since there is no metric). We can define convergence as
the members of the sequence (eventually) all lie in each open neighborhood of the convergence point.}

\begin{snippetdefinition}{topology-convergence-definition}{Convergence in topological space}
    Let \(\ts\) be a \topologicalspace.
    A \sequence \({\{x_n\}}_{n \in \naturalnumbers}\) in \(X\) converges to a point \(\alpha\in X\)
    (written \(x_n \to \alpha\)) if 
    \[ \forall U \in \mathcal{T}, \exists N \in \naturalnumbers \suchthat a_n \in U, \quad \forall n \geq N \]
\end{snippetdefinition}

\plain{This definition of convergence admits sequences that converge to multiple values.
This is because we might not have enough open sets to require the sequence to be arbitrarily "close"
(like in a metric space) to only one convergence point.}

\begin{snippetexample}{topology-multiple-convergence-example}{Topological space multiple convergence}
    Let \(\ts\) be the \topologicalspace where \(X=\realnumbers\) and
    \[ \mathcal{T} = \{\emptyset, \realnumbers\} \union \{(b, \infty) \suchthat b \in \realnumbers\}\]
    The sequence \({\{a_n\}}_{n\in \naturalnumbers} = {\{\frac{1}{n}\}}_{n\in \naturalnumbers}\).
    This sequences clearly converges to \(0\) because for \(n>0\) the value of the sequence is
    contained in every open set that contains \(0\). However, the same goes for \(-1\), \(-2\) and so on.
    Thus, \(a_n\topologyconverges 0 \land a_n\topologyconverges -1 \land a_n \topologyconverges -2 \land \cdots\).
\end{snippetexample}

\plain{In order to achieve uniqueness, we need to require disjoint open sets for every pair of distinct elements.
This is precisely what the Hausdorff property does.}

\begin{snippetproposition}{continuous-preserves-convergence}{Continuous maps preserve convergence}
    Every continuous function \(f \colon X \fromto Y\) is sequentially continuous,
    i.e., it sends convergent \sequence[sequences] to convergent \sequence[sequences].
    If \(x_n \to x\) then \(f(x_n) \to f(x)\).
\end{snippetproposition}

\begin{snippetproof}{continuous-preserves-convergence-proof}{continuous-preserves-convergence}{Continuous preserves convergence}
    By hypothesis, for every neighborhood \(U\) of \(x\) in \(X\),
    there exists \(N\) such that if \(n > N\), then \(x_n \in U\).
    Since \(f\) is continuous at \(x\), for every neighborhood \(V\)
    of \(f(x)\) there exists a neighborhood \(U\) of \(x\) such that \(f(U) \subseteq V\).
    It follows that for every neighborhood \(V\) of \(f(x)\), there exists
    \(N\) such that \(\forall n > N\), \(f(x_n) \in V\):
    taking \(U\) such that \(f(U) \subseteq V\) and \(N\) such that \(\forall n > N\),
    \(x_n \in U\), we have \(f(x_n) \in V\).
\end{snippetproof}

\begin{snippetproposition}{hausdorff-unique-limit}{Unique limits in Hausdorff spaces}
    A \sequence \({\{x_n\}}_{n \in \naturalnumbers}\) in a \(T_2\) space has a unique limit.
\end{snippetproposition}

\begin{snippetproof}{hausdorff-unique-limit-proof}{hausdorff-unique-limit}{Unique limits in Hausdorff}
    Suppose for contradiction that \(x_n \to x\) and \(x_n \to x'\) with \(x \neq x'\).
    Since \(X\) is \(T_2\), there exist disjoint open neighborhoods
    \(U\) of \(x\) and \(V\) of \(x'\) such that \(U \intersection V = \emptyset\).
    There exists \(N\) such that for \(n > N\), \(x_n \in U\).
    There exists \(M\) such that for \(n > M\), \(x_n \in V\).
    But then for \(n > \max\{N, M\}\), we have \(x_n \in U\) and \(x_n \in V\),
    so \(x_n \in U \intersection V = \emptyset\) \lightning.
\end{snippetproof}

\end{document}
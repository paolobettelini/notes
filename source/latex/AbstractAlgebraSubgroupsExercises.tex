\documentclass[preview]{standalone}

\usepackage{amsmath}
\usepackage{amssymb}
\usepackage{stellar}
\usepackage{definitions}

\begin{document}

\id{subgroups-exercises}
\genpage

\section{Exercises}

\begin{snippetexercise}{subgroups-ex1}{}
    Consider \(\matrices_n(R)\) and the subset \(S\)
    of symmetric matrices over addition and multiplication.
    Establish whether \(S\) induces a \submonoid and/or \subgroup.
\end{snippetexercise}

\begin{snippetsolution}{subgroups-ex1-sol}{}
    \((\matrices_n(R), +)\) is a \group and \((\matrices_n(R), \cdot)\) is a \monoid.
    We know that \({(A+B)}^\transpose = A^\transpose + B^\transpose\)
    and \({(AB)}^\transpose = B^\transpose A^\transpose\).
    \begin{enumerate}
        \item \(0^\transpose = 0\), so \(0 \in S\);
        \item if \(A \in S\) and \(B \in S\), then \({(A+B)}^\transpose = A^\transpose + B^\transpose = A + B\),
        meaning \(A+B \in S\).
    \end{enumerate}
    Thus, \((S, +)\) is a \submonoid of \((\matrices_n(R), +)\).
    We also have that if \(A \in S\), \(-A \in S\) as it is still symmetric,
    so \((S, +)\) is a \subgroup of \((\matrices_n(R), +)\).
    \begin{enumerate} %TODOURGENT link identtiy
        \item \(I^\transpose = I\), so \(I \in S\);
        \item If \(A \in S\) and \(B \in S\), then \({AB}^\transpose = B^\transpose A ^\transpose = BA\).
        However, \(BA\) is not necessarily symmetric.
    \end{enumerate}
    Thus, \((S, \cdot)\) is not a \submonoid of \((\matrices_n(R), \cdot)\).
\end{snippetsolution}

\begin{snippetexercise}{subgroups-ex2}{}
    Consider \(\integers\) over addition and multiplication.
    Given \(n\in\integers\), let \(n\integers\) denote the multiples of \(n\), which is a subset of \(\integers\).
    Establish whether \(n\integers\) induces a \submonoid and/or \subgroup.
\end{snippetexercise}

\begin{snippetsolution}{subgroups-ex2-sol}{}
    The set \(n\integers\) has elements \(nh\) for \(h \in\integers\).
    Let \(a,b \in n\integers\), then \(a=nh\) and \(b=nk\) for some \(h,k \in \integers\).
    \begin{enumerate}
        \item We have that \(nh+nk = n(h+k) \in n\integers\). Thus, \(n\integers\) is closed under addition.
            Clearly, \(0 \in n\integers\).
            Finally, if \(nh \in n\integers\), then \(-nh = n(-h)\in n\integers\).
            Thus, \((n\integers, +)\) is a \subgroup of \((\integers, +)\).
        \item We have that \(nh\cdot nk = n(hk) \in n\integers\).
            The identity is in \(n\integers\) \ifandonlyif \(n=1\) or \(n=-1\),
            in which case \(n\integers = \integers\).
            Thus, \((n\integers, \cdot)\) is not a \submonoid of \((\integers, \cdot)\)
            unless \(|n|=1\).
    \end{enumerate}
\end{snippetsolution}

\end{document}
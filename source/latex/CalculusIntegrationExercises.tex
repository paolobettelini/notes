\documentclass[preview]{standalone}

\usepackage{amsmath}
\usepackage{amssymb}
\usepackage{stellar}
\usepackage{definitions}
\usepackage{bettelini}
\usepackage{booktabs}
\usepackage{xparse}
\usepackage{tikz}
\usetikzlibrary{calc}

% https://tex.stackexchange.com/questions/97613/tabular-integration-by-parts
\tikzset{Arrow Style/.style={text=black, font=\boldmath}}
\newcommand{\tikzmark}[1]{%
    \tikz[overlay, remember picture, baseline] \node (#1) {};%
}
\newcommand*{\XShift}{0.5em}
\newcommand*{\YShift}{0.5ex}
\NewDocumentCommand{\DrawArrow}{s O{} m m m}{%
    \begin{tikzpicture}[overlay,remember picture]
        \draw[->, thick, Arrow Style, #2] 
                ($(#3.west)+(\XShift,\YShift)$) -- 
                ($(#4.east)+(-\XShift,\YShift)$)
        node [midway,above] {#5};
    \end{tikzpicture}%
}
\NewDocumentCommand{\DrawIntegralLine}{s O{} m m}{%
    \begin{tikzpicture}[overlay,remember picture]
        \draw[->, thick, Arrow Style, #2] 
                ($(#3.west)-(\XShift,\YShift)$) to 
                [bend right=45] ($(#4.east)-(-\XShift,\YShift)$)
        node [midway,above] {};
    \end{tikzpicture}%
}

\begin{document}

\id{integration-exercises}
\genpage

\section{Integrals by substitution}

\begin{snippetexercise}{integral-sub-ex-1}{}
    Solve \[\integral[\sin(x)e^{\cos(x)}][x]\]
\end{snippetexercise}

\begin{snippetsolution}{integral-sub-ex-1-sol}{}
    We substitute
    \begin{align*}
        u&=\cos(x) \\
        du&=\sin(x)dx \\
        dx &= \frac{du}{\sin(x)}
    \end{align*}
    and by substitution
    \begin{align*}
        &\integral[\sin(x)e^u\frac{du}{\sin(x)}][u] \\
        =&\integral[e^u][u]=e^u+C=e^{\cos(x)}+C
    \end{align*}
\end{snippetsolution}

\begin{snippetexercise}{integral-sub-ex-2}{}
    Solve \[\integral[\frac{5}{{(2x+3)}^2}][x]\]
\end{snippetexercise}

\begin{snippetsolution}{integral-sub-ex-2-sol}{}
    We substitute
    \begin{align*}
        u&=2x+3 \\
        du&= 2dx \\
        dx &= \frac{du}{2}
    \end{align*}
    and by substitution
    \begin{align*}
        &\integral[\frac{1}{2}\cdot\frac{5}{u^2}][u] \\
        =&\frac{5}{2}\left(\frac{u^{-1}}{-1}+C\right) \\
        =&\frac{-5}{2(2x+3)}+C
    \end{align*}
\end{snippetsolution}

\begin{snippetexercise}{integral-sub-ex-3}{}
    Solve \[\integral[\frac{x}{\sqrt{{(x^2+5)}^3}}][x]\]
\end{snippetexercise}

\begin{snippetsolution}{integral-sub-ex-3-sol}{}
    We substitute
    \begin{align*}
        u&=x^2+5 \\
        du&= 2xdx \\
        dx &= \frac{du}{2x}
    \end{align*}
    and by substitution
    \begin{align*}
        &\integral[\frac{x}{u^{\frac{3}{2}}}\cdot\frac{1}{2x}][u] \\
        =&\frac{1}{2}\integral[u^{-\frac{3}{2}}][u] \\
        =&-\frac{1}{\sqrt{x^2+5}}+C
    \end{align*}
\end{snippetsolution}

\section{Integrals by parts}

\begin{snippetexercise}{integral-parts-ex-1}{}
    Solve \[\integral[x^3e^x][x]\]
\end{snippetexercise}

\begin{snippetsolution}{integral-parts-ex-1-sol}{}
    We differentiate \(x^3\) and integrate \(e^x\):
    \[
        \renewcommand{\arraystretch}{1.5}
        \begin{array}{c @{\hspace*{1.0cm}} c}\toprule
           D & I \\\cmidrule{1-2}
          x^3   \tikzmark{Left 1} & \tikzmark{Right 1}e^x \\
          3x^2  \tikzmark{Left 2} & \tikzmark{Right 2}e^{x} \\      
          6x    \tikzmark{Left 3} & \tikzmark{Right 3}e^{x} \\      
          6     \tikzmark{Left 4} & \tikzmark{Right 4}e^{x} \\      
          0     \tikzmark{Left 5} & \tikzmark{Right 5}e^{x} \\\bottomrule
        \end{array}
    \]
    %-----------------------------------------
    \DrawArrow[draw=red]{Left 1}{Right 2}{$+$}%
    \DrawArrow[draw=red]{Left 2}{Right 3}{$-$}%
    \DrawArrow[draw=red]{Left 3}{Right 4}{$+$}%
    \DrawArrow[draw=red]{Left 4}{Right 5}{$-$}%

    The result is \(e^x(x^3-3x^2+6x-6)+C\).
\end{snippetsolution}

\section{Miscellaneous}

\begin{snippetexercise}{integral-parts-mixed-1}{}
    Solve \[\integral[\cos(\ln(x))][x]\]
\end{snippetexercise}

\begin{snippetsolution}{integral-parts-mixed-1-sol}{}
    We substitute
    \begin{align*}
        u&=\ln(x) \\
        du&= \frac{dx}{x} \\
        dx &= udx
    \end{align*}
    and by substitution
    \begin{align*}
        \integral[\cos(u)e^u][u]
    \end{align*}
    Now we can differentiate \(e^u\) and integrate \(\cos(u)\):

    \[
        \renewcommand{\arraystretch}{1.5}
        \begin{array}{c @{\hspace*{1.0cm}} c}\toprule
           D & I \\\cmidrule{1-2}
          {\color{blue}+} e^u   \tikzmark{Left 1} & \tikzmark{Right 1}+\cos(u) \\
          {\color{blue}-} e^u  \tikzmark{Left 2} & \tikzmark{Right 2}+\sin(u) \\
          {\color{blue}+} e^u    \tikzmark{Left 3} & \tikzmark{Right 3}-\cos(u) \\      
        \end{array}
    \]
    %-----------------------------------------
    \DrawArrow[draw=red]{Left 1}{Right 2}{}%
    \DrawArrow[draw=red]{Left 2}{Right 3}{}%
    \DrawIntegralLine[draw=red, -]{Left 3}{Right 3}%

    \begin{align*}
        &\integral[\cos(u)e^u][u] = e^u\sin(u)+e^u\cos(u) -\integral[\cos(u)e^u][u] \\
        &2\integral[\cos(u)e^u][u] = e^u(\sin(u)+\cos(u)) \\
        &\integral[\cos(u)e^u][u] = \frac{e^u(\sin(u)+\cos(u))}{2} \\
        &= \frac{x(\sin(\ln(x))+\cos(\ln(x)))}{2} + C
    \end{align*}
\end{snippetsolution}

\begin{snippetexercise}{integral-parts-mixed-2}{}
    Solve \[\integral[1][\infty][t^{-2} \arctan\left(\frac{1}{t}\right)][t]\]
\end{snippetexercise}

\begin{snippetproof}{integral-parts-mixed-2-proof}{integral-parts-mixed-2}{}
    Let \(u = 1/t\) so that \(du = -t^{-2}\,dt\). We have
    \begin{align*}
        -\integral[1][0][u^2\arctan(u)x^2][u]
        &= \integral[0][1][\frac{u^2}{u^2}\arctan(u)][u] \\
        &= \integral[0][1][\arctan(u)][u]
    \end{align*}
    We can integrate \(\arctan(u)\) using integration by parts
    \begin{align*}
        \integral[\arctan(t)][t]
        &= t\arctan(t) - \integral[\frac{t}{t^2 + 1}][t] \\
        &= t\arctan(t) - \frac{1}{2}\log|t^2 + 1|
    \end{align*}
    The integral becomes
    \begin{align*}
        \integral[0][1][\arctan(u)][u] &=
        \frac{\pi}{4} - \ln\sqrt{2}
    \end{align*}
\end{snippetproof}

\begin{snippetexercise}{integral-misc-ex1}{}
    Compute
    \[
        \integral[0][+\infty][\frac{t}{(t+1)(t^2+2t+2)}][t]
    \]
\end{snippetexercise}

\begin{snippetproof}{integral-misc-ex1-proof}{integral-misc-ex1}{}
    To find the primitive, we want to split
    \[
        \frac{A}{t+1} + \frac{Bt+C}{t^2 + 2t + 2} = \frac{t}{(t+1)(t^2 + 2t + 2)}
    \]
    so that \(t^2(A+B) + t(2A + B + C) + 2A+C\), meaning that
    \(A = -1\), \(B=1\) and \(C = 2\).
    We now have
    \begin{align*}
        \integral[\frac{t}{(t+1)(t^2+2t+2)}][t]
        &= -\integral[\frac{1}{t+1}][t] + \integral[\frac{t+2}{t^2+2t+2}][t] \\
        &= -\ln|t+1| + \frac{1}{2} \integral[\frac{2t+2+2}{t^2+2t+2}][t] \\
        &= -\ln|t+1| + \frac{1}{2} \ln|t^2+2t+2| + \integral[\frac{1}{t^2+2t+2}][t] \\
        &= \ln\frac{\sqrt{t^2+2t+2}}{|t+1|} + \integral[\frac{1}{{(t+1)}^2 + 1}][t] \\
        &=  \ln\frac{\sqrt{t^2+2t+2}}{|t+1|} + \arctan(t+1) + C
    \end{align*}
    We now take the limit
    \begin{align*}
        &\lim_{R\to+\infty}
        \integral[0][R][\frac{t}{(t+1)(t^2+2t+2)}][t] \\
        &= \lim_{R\to+\infty}
        \ln\frac{\sqrt{R^2+2R+2}}{|R+1|} + \arctan(R+1)
        - \ln\sqrt{2} - \arctan(1) \\
        &= \frac{\pi}{4} - \ln\sqrt{2}
    \end{align*}
\end{snippetproof}

\begin{snippetproposition}{integral-reciprocal-products}{}
    \[
        \integral[\prod_{k=0}^m \frac{1}{t+k}][t]
        = \sum_{k=0}^m \frac{{(-1)}^k \ln|t+k|}{k!(m-k)!} + C
    \]
\end{snippetproposition}

\begin{snippetproof}{integral-reciprocal-products-proof}{integral-reciprocal-products}{}
    We want to split the integrand into
    \[
        \prod_{k=0}^m \frac{1}{t+k} = \sum_{k=0}^m \frac{a_k}{t+k}
    \]
    so that
    \[
        a_0(t+1)(t+2)\cdots(t+m) + a_1t(t+2)(t+3)\cdots(t+m)
        + \cdots = 1
    \]
    \begin{itemize}
        \item if \(t=0\), we have \(1 = a_0m!\) meaning \[a_0 = \frac{1}{m!}\]
        \item if \(t=-1\), we have \(1 = a_1(-1)(1)(2)\cdots(m-1)\), meaning
            \[ a_1 = \frac{-1}{(m-1)!} \]
        \item if \(t=-2\), we have \(1 = a_2(-2)(-11)(1)(2)\cdots(m-2)\), meaning
            \[ a_2 = \frac{{(-1)}^2}{2!(m-2)!} \]
    \end{itemize}
    In general, we can clearly see that
    \[
        a_k = \frac{{(-1)}^k}{k!(m-k)!}
    \]
    The integral becomes
    \begin{align*}
        \integral[\prod_{k=0}^m \frac{1}{t+k}][t]
        &= \sum_{k=0}^m \integral[\frac{{(-1)}^k}{k!(m-k)!(t+k)}][t] \\
        &= \sum_{k=0}^m \frac{{(-1)}^k}{k!(m-k)!} \int\frac{d\,t}{t+k} \\
        &= \sum_{k=0}^m \frac{{(-1)}^k \ln|t+k|}{k!(m-k)!} + C
    \end{align*}
\end{snippetproof}

\end{document}
\documentclass[preview]{standalone}

\usepackage{amsmath}
\usepackage{amssymb}
\usepackage{stellar}

\hypersetup{
    colorlinks=true,
    linkcolor=black,
    urlcolor=blue,
    pdftitle={Stellar},
    pdfpagemode=FullScreen,
}

\begin{document}

\title{Stellar}
\id{italiano-decameron-gianni-lotteringhi}
\genpage

\section{Analisi}

\begin{snippet}{gianni-lotteringhi-analisi}
    \textbf{Rubrica:} Gianni Lotteringhi ode di notte toccar l'uscio suo; desta la moglie, ed ella gli fa accredere che egli è la fantasima; vanno ad incantare con una orazione, ed il picchiare si rimane.

    % Gianni ha successo nel suo lavoro ma è scarso altrove
    % I frati di un convento nominano Capo Gianni (per sfruttarlo un po' perché ricco)
    % Lui fornisce tessuti per i loro abiti in cambio di insegnamenti (preghiere)
    
    % personaggi
    La novella è caratterizzata da un triangolo amoroso fra Tessa, Gianni e Federigo.
    Gianni ha avuto molta fortuna con il suo lavoro, ma altrove è piuttosto scarso.
    Esso è una persona molto ingenua, e per quanto riguarda la religione è molto
    superstizioso e bigotto (§§4-5).
    Federigo è bello, giovane e fresco (§6).
    Rispetto a Gianni, Federgio è più intelligente e allegro.
    I due sono infatti molto distanti.
    % differenze
    Il marito Gianni si mangia la carna salata, mentre l'amante si gode la cena completa (§§12-13).
    La moglie con il marito dorme e basta, mentre con l'amante si sfoga in effusioni sessuali (§8 e §20).
    Inoltre, un altro elemento di differenza è quello delle preghiere. Infatti, Gianni intende le preghiere in maniera letteraria,
    mentre Federigo riesce ad interderne il significato (§8 e §20).
    %
    Tessa è bella, intelligente e saggia (§6). Rispetto è marito è molto più astuta, ma è innamorata di Federigo.
    Conosce bene il marito e sa di poterlo inganare (§7).
    Il marito viene ingannato dicendogli che la preghiera può ora essere detto dal momento che sono
    entrambi presenti, facendolo sentire importante.
    
    
    % non è importante che ci siano "2 versioni"
    % Il consiglio di Emilia alle altre novellatrici è letterale; la beffa è utile per il tradimento
\end{snippet}

\end{document}
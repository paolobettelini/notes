\documentclass[preview]{standalone}

\usepackage{amsmath}
\usepackage{amssymb}
\usepackage{stellar}
\usepackage{definitions}
\usepackage{bettelini}

\begin{document}

\id{keplers-laws}
\genpage

\section{Kepler's laws of planetary motion}

\begin{snippettheorem}{keplers-laws-theorem}{Kepler's laws of planetary motion}
    Kepler's laws of planetary motion state:
    \begin{enumerate}
        \item The orbit of a planet is an ellipse with the Sun at one of the two foci.
        \item A line segment joining a planet and the Sun sweeps out equal areas during equal intervals of time.
        \item The square of a planet's orbital period is proportional to the cube of the length of the semi-major axis of its orbit.
    \end{enumerate}
\end{snippettheorem}

\begin{snippetproof}{keplers-laws-theorem-proof}{keplers-laws-theorem}{Kepler's laws of planetary motion}
    Consideriamo il sistema Terra-Sole.
    Possiamo scrivere le equazioni di Newton per la Terra, sul sistema di riferimento
    non inerziale del sole, se alla terra associamo la massa ridotta
    \[
        \mu = \frac{M_S M_T}{M_S + M_T}
    \]
    Quindi
    \begin{align*}
        \mu \frac{d^2\vec{r}}{dt^2} &= - G \frac{M_S M_T}{r^2} \hat{r} \\
        \frac{M_S M_T}{M_S + M_T} \frac{d^2\vec{r}}{dt^2} &= - G \frac{M_S M_T}{r^2} \hat{r} \\
        \frac{d^2\vec{r}}{dt^2} &= - G \frac{M_S + M_T}{r^2} \hat{r}
    \end{align*}
    che è l'equazione che avrei scritto se avessi considerato un sistema
    con un oggetto di massa \(M_S + M_T\).

    Ricordiamo che vi sono delle proprietà che vengono conservate, come
    \[
        \vec{L} = \vec{r} \wedge \vec{v}
    \]
    e allora
    \[
        \frac{d\vec{L}}{dt} = 0
    \]
    implica che l'orbita sia piana.
    Poniamo l'origine nel centro delle forze: \(\vec{L} = L\vec{z}\)
    e quindi l'orbita giace sul piano \(xy\).

    Ricordiamo anche che le forse sono centrali e quindi conservative.
    Abbiamo quindi l'energia potenziale
    \[
        \frac{dU}{dt} = -f(r)
    \]
    e quindi
    \[
        U(r) = - \frac{GM}{r}
    \]
    L'energia meccanica per unità di massa è quindi
    \[
        E = \frac{1}{2} v^2 - \frac{GM}{r}
    \]
    che viene conservata come il momento angolare per unità di massa.
    \[
        L_z = x v_y - y v_x
    \]
    Usiamo le coordinate polari piane
    e troviamo \(v_x = \dot{r}\cos\theta - r\dot{\theta}\sin\theta\)
    e \(v_y = \dot{r}\sin<theta + r\dot{\theta}\cos\theta\).
    Otteniamo quindi
    \[
        \begin{cases}
            L_z = r^2 \dot{\theta} \\
            v^2 = \dot{r}^2 + r^2 \dot{\theta}^2
        \end{cases}
    \]
    Otteniamo allora l'energia
    \begin{align*}
        E &= \frac{1}{2} \dot{r}^2 + \frac{1}{2}r^2 \frac{L_z^2}{r^4} - \frac{GM}{r} \\
        &= \frac{1}{2} \dot{r}^2 + \frac{L_z^2}{2r^2} - \frac{GM}{r}
    \end{align*}

    Possiamo identficare il termine \(\frac{L_z^2}{2r^2} - \frac{GM}{r}\)
    come un potenziale efficace \(U_\text{eff}(r)\).
    Notiamo che vi è un asintoto verticale a destra di \(r=0\) verso \(+\infty\),
    e che il limite tende a \(0^+\).
    Siccome l'altro addendo è positivo, quando l'energia è negativa
    non vi sono soluzioni per valori \(E < E_\text{min}\)
    in quanto l'energia minima è appunto il minimo di \(U_\text{eff}(r)\).
    Più in generale, vi è soluzione solo per un certo
    intervallo. Ciò succede quando \(E_\text{min} < E < 0\). % disegnigno

    Per risolvere l'equazione ci chiediamo quale sia la traiettoria della particella.
    Ricordiamo l'altra legge di conservazione
    \[
        L_z = r^2 \dot\theta \to \begin{cases}
            r(t) \\
            \theta(t) \to r(\theta)
        \end{cases}
    \]
    Se prendiamo la derivata otteniamo
    \[
        \frac{d}{dt}r(\theta(t))
        = \frac{dr}{d\theta} \cdot \frac{d\theta}{dt}
        = \frac{dr}{d\theta} \dot\theta = \frac{dr}{d\theta} \cdot \frac{L_z}{r^2}
    \]
    Allora sostituiamo nella conservazione dell'energia e otteniamo
    \[
        E = \frac{1}{2}
        {\left(\frac{dr}{d\theta}\right)}^2 \frac{L^2}{r^4} + \frac{L^2}{2r^2} - \frac{GM}{r}
    \]
    che è una funzione per la traiettoria.
    Definiamo ora una variabile
    \(u = r^{-1}\).
    Allora \(\frac{du}{d\theta} = -\frac{1}{r^2} \cdot \frac{dr}{d\theta}\).
    Quindi
    \[
        \frac{dr}{d\theta} = -r^2 \frac{du}{d\theta}
    \]
    Risostituendo troviamo
    \begin{align*}
        E &= \frac{1}{2} L_z^2 {\left(\frac{du}{d\theta}\right)}^2 + \frac{L^2}{2}u^2 - GMu \\
        {\left(\frac{du}{d\theta}\right)}^2 &= A + Bu - u^2,
        \quad A = \frac{2E}{L_z^2}, B = \frac{2GM}{L_z^2}
    \end{align*}
    La seguente equazione soddisfa l'equazione differenziale
    \[
        \mu(\theta) = a + b\cos(\theta - \theta_0)
    \]
    per opportuni \(a,b\).
    Sostituendo troviamo
    \[
        b^2 = [A-a^2 + aB] + (bB - 2ab)\cos(\theta - \theta_0)
    \]
    Affinché l'equazione sia vera per ogni \(\theta\), abbiamo le condizioni
    \[
        \begin{cases}
            bB = 2ab \\
            A - a^2 + aB = b^2
        \end{cases}
        \to \begin{cases}
            a = \frac{B}{2} \\
            b = \pm\sqrt{A + \frac{B^2}{4}}
        \end{cases}
    \]
    Tuttavia, il \(\pm\) è ridondante in quanto spostando \(\theta_0\)
    ritroviamo le stesse soluzioni. Scegliamo il segno negativo.
    Abbiamo
    \[
        r(\theta) = \frac{l}{1-e\cos\theta}, \quad l = \frac{2}{B}, e = \frac{2}{B} \sqrt{A + \frac{B^2}{4}}
    \]
    isolando \(E\) troviamo
    \[
        E \geq - \frac{{(GM)}^2}{L_z^2} = E_\text{min}
    \]
    Nel caso dell'energia minima, la traiettoria è circolare in quanto l'intervallo dei raggi
    della traiettoria \(r_\text{min} < r < r_\text{max}\) è un singoletto, in quanto ci troviamo al minimo del potenziale. \\
    Per evitare divisione con zero prendiamo \(0 \leq e < 1\).
    Dobbiamo anche vincolare \(\cos\theta < \frac{1}{e}\), che limita la traiettoria possibile.
    Importiamo ora \(x = r\cos\theta\) e \(y = r\sin\theta\) e moltiplichiamo l'equazione per \(\cos\theta\)
    a destra e sinistra e per \(\sin\theta\), otteniamo
    \begin{align*}
        \cos\theta r(\theta) &= \frac{l\cos\theta}{1-e\cos\theta} \\
        l\cos\theta &= x(1-e\cos\theta) \\
        l\sin\theta &= y(1-e\cos\theta) \\
    \end{align*}
    e quindi troviamo
    \[
        \sin\theta = \frac{y}{l+xe}
    \]
    Partendo dall'identità pitagorica
    \begin{align*}
        \cos^2\theta + \sin^2\theta &= \frac{x^2}{{(l+xe)}^2} + \frac{y^2}{{(l+xe)}^2} \\
        x^2 + y^2 &= l^2 +x^2e^2 + 2lex \\
        x^2{(1-e^2)} + y^2 - 2lex &= l^2
    \end{align*}
    che è una conica.
    Se \(e=0\) è una circonferenza,
    se \(e<1\) è un ellisse, se \(e>1\) è un iperbole. % TODOURGENT: verificare

    Notiamo che
    \[
        r_\text{min} = \frac{l}{1+e}
    \]
    che si ottiene per \(\theta = \pi\) e 
    \[
        r_\text{max} = \frac{l}{1-e}
    \]
    che si ottiene per \(\theta = 0\).
    Il semiasse maggiore è dato da
    \[
        a = \frac{r_\text{max} + r_\text{min}}{2}
        = \frac{2/B}{1 - \left(1 + \frac{4A}{B^2}\right)}
        = -\frac{B}{2A}
    \]
    che dipende solo dall'energia (dal modulo).
    Più grande il modulo, più piccolo è il semiasse maggiore.
    Più si va all'esterno del sistema solare più l'energia diminuisce.

    Abbiamo quindi dimostrato le prime due leggi di Kepler.
    Rimane da dimostrare la terza.

    Per fare ciò scriviamo
    \[
        \dot\theta = \frac{L_z}{r^2} = \frac{L_z}{l^2}{(1-e\cos\theta)}^2
    \]
    allora abbiamo
    \[
        \int\limits_0^{2\pi} \frac{d\theta}{{(1-e\cos\theta)}^2}
        = \frac{L_z}{l^2} \cdot 2\pi
    \]
    con la sostituzione \(v = \tan \frac{\theta}{2}\) troviamo
    \[
        T = \integral[-\infty][\infty][\frac{2}{1 + v^2 {\left(1 - e \frac{1 - v^2}{1 + v^2}\right)}^2}][v]
        = \frac{e^{L_z}}{{L_z}} \cdot \frac{2\pi}{{(1-e)}^{3/2}}
    \]
    e quindi
    \begin{align*}
        T^2 = a^3 \frac{4\pi^2}{GM}
    \end{align*}
\end{snippetproof}

\end{document}
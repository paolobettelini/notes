\documentclass[preview]{standalone}

\usepackage{amsmath}
\usepackage{amssymb}
\usepackage{stellar}
\usepackage{definitions}

\begin{document}

\id{topology-examples}
\genpage

\section{Examples}


\plain{The intersection of infinitely many open sets is not necessarily open, but their union is.}

\begin{snippetexample}{infinite-intersection-open-sets-not-open-example}{Intersection of infinite open sets is not open}
    Consider the \set[sets]
    \[
        A_k = \left(-\frac{1}{k}, \frac{1}{k}\right), \quad k\in\naturalnumbers^\exceptzero
    \]
    which are \msopenset[open].
    Then, their intersection is given by
    \[
        \bigcap_{k=1}^\infty A_k = \{0\}
    \]
    which is not an \msopenset.
\end{snippetexample}

\plain{The union of infinitely many closed sets is not necessarily closed, but their intersection is.}

\begin{snippetexample}{infinite-union-closed-sets-not-closed-example}{Union of infinite closed sets is not closed}
    Consider the \set[sets]
    \[
        A_k = \left[\frac{1}{k}, 1\right], \quad k\in\naturalnumbers^\exceptzero
    \]
    which are \msclosedset[closed].
    Then, their intersection is given by
    \[
        \bigcap_{k=1}^\infty A_k = (0, 1]
    \]
    which is not a \msclosedset.
\end{snippetexample}

\begin{snippetexample}{topology-non-continuous-function-example}{Non-continuous function}
    Consider the \topologicalspace \((X, \tau)\)
    where \(X = \{a, b, c\}\) and \(\tau = \{\emptyset, X, \{a\}\}\).
    Consider the \function \(f\colon X \to X\) defined by
    \(f(a) = f(b) = a\) and \(f(c) = c\).
    We first study the \neighborhood[neighborhoods] of \(a\) and \(b\):
    \begin{enumerate}
        \item The \openset[open sets] containing \(a\)
        are \(\{a\}\) and \(X\). Thus, the \neighborhood[neighborhoods] of \(a\) are
        \(\{a\}\), \(X\), but also \(\{a, b\}\) and \(\{a, c\}\).
        \item The only \openset containing \(b\)
        is \(X\). Thus, the only \neighborhood of \(b\) is
        \(X\).
    \end{enumerate}
    We will now study the \topologycontinuous[continuity] of \(f\) at \(a\) and \(b\):
    \begin{enumerate}
        \item For every \neighborhood \(V\) of \(f(a) = a\),
        we need to find a \neighborhood \(U\) of \(a\) such that
        \(f(U) \subseteq V\). For every \neighborhood \(V\) of \(a\),
        we can choose \(U = \{a\}\) and \(f(U) = \{a\} \subseteq V\)
        for every \neighborhood \(V\) of \(a\).
        \item For every \neighborhood \(V\) of \(f(b) = a\),
        we need to find a \neighborhood \(U\) of \(b\) such that
        \(f(U) \subseteq V\). The only possible choice for \(U\) is \(X\),
        but \(f(X) = \{a, c\}\) is a subset of the \neighborhood[neighborhoods] \(X\)
        and \(\{a, c\}\) of \(a\), but not the other \neighborhood[neighborhoods] \(\{a\}\) and \(\{a, b\}\)
        of \(a\). Thus, \(f\) is not \topologycontinuous at \(b\).
    \end{enumerate}
\end{snippetexample}

\section{Examples of topologies}

\begin{snippetdefinition}{trivial-topology-definition}{Trivial topology}
    Let \(X\) be a \set. Then, the \emph{trivial topology}
    or \emph{indiscrete topology} of \(X\) if \(\{\emptyset, X\}\).
\end{snippetdefinition}

\begin{snippetdefinition}{discrete-topology-definition}{Discrete topology}
    Let \(X\) be a \set. Then, the \emph{discrete topology} of
    \(X\) is \(\powerset(X)\).
\end{snippetdefinition}

\plain{The singletons of every element form a basis for this topology.}

\begin{snippetdefinition}{upper-semicontinuous-topology-definition}{Upper semicontinuous topology}
    The \emph{upper semicontinuous topology} on \(\realnumbers\)
    is a \topologicalspace[topology] on \(\realnumbers\) with \openset[open sets]
    of the form \((-\infty, a)\) for \(a \in \realnumbers \union \{\infty\}\)
    and \(\emptyset\).
\end{snippetdefinition}

\begin{snippetproposition}{upper-semicontinuous-topology-is-topology}{}
    The upper semicontinuous topology is a \topologicalspace[topology].
\end{snippetproposition}

\begin{snippetproof}{upper-semicontinuous-topology-is-topology-proof}{upper-semicontinuous-topology-is-topology}{}
    \begin{enumerate}
        \item \(\emptyset\) is \openset[open] by definition;
        \item \(\realnumbers = (-\infty, \infty)\) is \openset[open];
        \item the intersection is given by
        \[
            (-\infty, a_1) \intersection (-\infty, a_2)
            = (-\infty, \min\{a_1, a_2\})
        \]
        which is \openset[open];
        \item consider the union of \(\{a_i \in \realnumbers \union \{\infty\} \suchthat i \in I\}\)
        for some family of indexes \(I\)
        \[
            \bigcup_{i\in I} (-\infty, a_i)
        \]
        we have two cases:
        \begin{enumerate}
            \item \(\{a_i \in \realnumbers \union \{\infty\} \suchthat i \in I\}\) is \bounded[bounded above]:
                By the supremum property, every subsets has a supremum,
                and by its definition
                \[
                    \bigcup_{i\in I} (-\infty, a_i)
                    = (-\infty, \sup \{a_i \in \realnumbers \union \{\infty\} \suchthat i \in I\})
                \]
                the double inclusion of this equivalence is trivial.
            \item \(\{a_i \in \realnumbers \union \{\infty\} \suchthat i \in I\}\) is not \bounded[bounded above]:
                there is no supremum and thus
                \[
                    \bigcup_{i\in I} (-\infty, a_i) = (-\infty, \infty)
                    = \realnumbers
                \]
        \end{enumerate}
    \end{enumerate}
\end{snippetproof}

\begin{snippetdefinition}{sorgentfrey-topology-definition}{Sorgentfrey topology}
    The \emph{Sorgentfrey topology} is defined as the
    \topologicalspace[topology] with a real \topologicalbasis
    of the form \([a,b)\).
\end{snippetdefinition}

\plain{This topology has the same open sets of the Euclidea topology and more (like the semiopen intervals).}

\begin{snippetdefinition}{cofinite-topology-definition}{Cofinite topology}
    Let \(X\) be a \set.
    The \emph{cofinite topology} on \(X\)
    is the \topologicalspace[topology] defined as
    \[
        \mathcal{T}_{\text{cof}}
        = \{
            U \subseteq X
            \suchthat
            U = \emptyset \lor \, % TODO padding su | di cardinality
            \cardinality{X \difference U} < \infty    
        \}
    \]
\end{snippetdefinition}

\end{document}
\documentclass[preview]{standalone}

\usepackage{amsmath}
\usepackage{amssymb}
\usepackage{stellar}
\usepackage{definitions}

\begin{document}

\id{topology-examples}
\genpage

\section{Examples}


\plain{The intersection of infinitely many open sets is not necessarily open, but their union is.}

\begin{snippetexample}{infinite-intersection-open-sets-not-open-example}{Intersection of infinite open sets is not open}
    Consider the \set[sets]
    \[
        A_k = \left(-\frac{1}{k}, \frac{1}{k}\right), \quad k\in\naturalnumbers^\exceptzero
    \]
    which are \msopenset[open].
    Then, their intersection is given by
    \[
        \bigcap_{k=1}^\infty A_k = \{0\}
    \]
    which is not an \msopenset.
\end{snippetexample}

\plain{The union of infinitely many closed sets is not necessarily closed, but their intersection is.}

\begin{snippetexample}{infinite-union-closed-sets-not-closed-example}{Union of infinite closed sets is not closed}
    Consider the \set[sets]
    \[
        A_k = \left[\frac{1}{k}, 1\right], \quad k\in\naturalnumbers^\exceptzero
    \]
    which are \msclosedset[closed].
    Then, their intersection is given by
    \[
        \bigcap_{k=1}^\infty A_k = (0, 1]
    \]
    which is not a \msclosedset.
\end{snippetexample}

\begin{snippetexample}{topology-non-continuous-function-example}{Non-continuous function}
    Consider the \topologicalspace \((X, \tau)\)
    where \(X = \{a, b, c\}\) and \(\tau = \{\emptyset, X, \{a\}\}\).
    Consider the \function \(f\colon X \to X\) defined by
    \(f(a) = f(b) = a\) and \(f(c) = c\).
    We first study the \neighborhood[neighborhoods] of \(a\) and \(b\):
    \begin{enumerate}
        \item The \topologicalspace[open sets][Open set] containing \(a\)
        are \(\{a\}\) and \(X\). Thus, the \neighborhood[neighborhoods] of \(a\) are
        \(\{a\}\), \(X\), but also \(\{a, b\}\) and \(\{a, c\}\).
        \item The only \topologicalspace[open set][Open set] containing \(b\)
        is \(X\). Thus, the only \neighborhood of \(b\) is
        \(X\).
    \end{enumerate}
    We will now study the \topologycontinuous[continuity] of \(f\) at \(a\) and \(b\):
    \begin{enumerate}
        \item For every \neighborhood \(V\) of \(f(a) = a\),
        we need to find a \neighborhood \(U\) of \(a\) such that
        \(f(U) \subseteq V\). For every \neighborhood \(V\) of \(a\),
        we can choose \(U = \{a\}\) and \(f(U) = \{a\} \subseteq V\)
        for every \neighborhood \(V\) of \(a\).
        \item For every \neighborhood \(V\) of \(f(b) = a\),
        we need to find a \neighborhood \(U\) of \(b\) such that
        \(f(U) \subseteq V\). The only possible choice for \(U\) is \(X\),
        but \(f(X) = \{a, c\}\) is a subset of the \neighborhood[neighborhoods] \(X\)
        and \(\{a, c\}\) of \(a\), but not the other \neighborhood[neighborhoods] \(\{a\}\) and \(\{a, b\}\)
        of \(a\). Thus, \(f\) is not \topologycontinuous at \(b\).
    \end{enumerate}
\end{snippetexample}

\end{document}
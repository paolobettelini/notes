\documentclass[preview]{standalone}

\usepackage{amsmath}
\usepackage{amssymb}
\usepackage{stellar}
\usepackage{definitions}
\usepackage{bettelini}

\begin{document}

\id{topology-covering-spaces}
\genpage

\section{Covering spaces}

\begin{snippetdefinition}{covering-space-definition}{Covering space}
    Let \(X\) be a connected \topologicalspace.
    A continuous map \(p \colon E \fromto X\) is called a \emph{covering}
    (or \emph{covering map}) if every point \(x \in X\)
    has an \openset[open] neighborhood \(V \subseteq X\) such that
    \(p^{-1}(V)\) is a disjoint union of open sets \(\{U_i\}_{i \in I}\)
    with each restriction
    \[
        \restr{p}{U_i} \colon U_i \fromto V
    \]
    being a \homeomorphism.
    
    The space \(E\) is called the \emph{total space}
    and \(X\) is called the \emph{base space}.
    Opens \(V\) with this property are called \emph{evenly covered}
    or \emph{trivializing neighborhoods}.
\end{snippetdefinition}

\begin{snippetdefinition}{local-homeomorphism-definition}{Local homeomorphism}
    A continuous map \(f \colon X \fromto Y\) is a \emph{local homeomorphism}
    if every point \(x \in X\) has an open neighborhood \(U\)
    such that \(f(U)\) is open in \(Y\) and
    \(\restr{f}{U} \colon U \fromto f(U)\) is a homeomorphism.
\end{snippetdefinition}

\begin{snippetproposition}{covering-is-local-homeomorphism}{Covering is local homeomorphism}
    Every covering map is a local homeomorphism.
\end{snippetproposition}

\begin{snippetdefinition}{trivial-covering-definition}{Trivial covering}
    A covering \(p \colon E \fromto X\) is called \emph{trivial}
    if \(X\) itself is an evenly covered neighborhood,
    i.e., \(E \cong X \cartesianprod F\) for some discrete space \(F\).
\end{snippetdefinition}

\subsection{Properly discontinuous actions}

\begin{snippetdefinition}{properly-discontinuous-action-definition}{Properly discontinuous action}
    Let \(G\) be a group acting on a \topologicalspace \(X\) via \(\alpha \colon G \cartesianprod X \fromto X\).
    The action is called \emph{properly discontinuous} if for every \(x \in X\)
    there exists an open neighborhood \(V\) of \(x\) such that
    \[
        gV \intersection g'V = \emptyset
    \]
    for all distinct \(g, g' \in G\).
\end{snippetdefinition}

\begin{snippetproposition}{properly-discontinuous-is-free}{Properly discontinuous implies free}
    Every properly discontinuous action is free (i.e., \(gx = x\) implies \(g = 1\)).
\end{snippetproposition}

\begin{snippettheorem}{properly-discontinuous-gives-covering}{Properly discontinuous gives covering}
    Let \(G\) act properly discontinuously on \(X\).
    Then the quotient map \(p \colon X \fromto X/G\) is a covering.
\end{snippettheorem}

\begin{snippetproof}{properly-discontinuous-gives-covering-proof}{properly-discontinuous-gives-covering}{Properly discontinuous gives covering}
    The quotient map \(p\) is continuous, surjective, and open since
    \(p^{-1}(p(U)) = \bigcup_{g \in G} gU\).
    
    For any \(x \in X\), choose an open \(U\) containing \(x\) with
    \(gU \intersection g'U = \emptyset\) for distinct \(g, g'\).
    Then \(p(U)\) is evenly covered:
    \[
        p^{-1}(p(U)) = \bigsqcup_{g \in G} gU
    \]
    and each \(\restr{p}{gU} \colon gU \fromto p(U)\) is a homeomorphism.
\end{snippetproof}

\begin{snippetexample}{circle-covering-example}{The circle covering}
    The exponential map \(e \colon \realnumbers \fromto S^1\)
    given by \(e(t) = e^{2\pi i t}\) is a covering.
    The action of \(\integers\) on \(\realnumbers\) by translation
    is properly discontinuous, and \(S^1 \cong \realnumbers / \integers\).
\end{snippetexample}

\end{document}

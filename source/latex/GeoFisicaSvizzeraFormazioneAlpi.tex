\documentclass[preview]{standalone}

\usepackage{amsmath}
\usepackage{amssymb}
\usepackage{tikz}
\usepackage{stellar}
\usepackage{definitions}
\usepackage{bettelini}

\begin{document}

\id{geofisica-svizzera-formazione-alpi}
\genpage

\section{La formazione delle alpi}

\begin{snippet}{formazione-alpi-part-1}
    L'inizio della struttura geologica della Svizzera è dato dal formarsi delle Alpi. Tale processo ha avuto
    molteplici fasi. Nel mesozoico, il medioevo geologico, la Svizzera era ancora ricoperta dalle acque.
    Era parte di un esteso mare che i geologi chiamano mare di Tethys (Tetide). Sul fondo di questo si
    depositò il materiale trasportato dai fiumi. Gli strati continuarono a depositarsi uno sull'altro fino a
    raggiungere uno spessore valutato in varie migliaia di metri.
\end{snippet}

\includesnpt[width=75\%|src=/snippet/static/formazione-alpi-svizzere-1.png]{centered-img}

\begin{snippet}{formazione-alpi-part-2}
    Circa 180 mio di anni fa si estendeva sull’Europa centrale un mare nel quale avvenivano
    processi di sedimentazione.
    La formazione delle rocce sedimentarie dipendeva da vari fattori, principalmente dalla qualità del
    materiale sedimentario depositato e dalle caratteristiche delle zone di mare nelle quali queste
    sedimentazioni, questi consolidamenti e queste formazioni di rocce ebbero luogo. Si distinguono
    essenzialmente tre zone di sedimentazione, l'area elvetica, comprendente la costa settentrionale
    del Tethys, la penninica comprendente i fondali centrali, e la alpino-orientale comprendente la
    costa meridionale.
\end{snippet}

\includesnpt[width=75\%|src=/snippet/static/formazione-alpi-svizzere-2.png]{centered-img}

\begin{snippet}{formazione-alpi-part-3}
    Potenti spinte spostarono verso nord la massa continentale che si trovava a meridione. Gli
    strati sedimentari vennero piegati e sollevati. Le rocce sedimentarie emersero dal mare e
    formarono diverse serie di isole allungate e allineate: sui rilievi iniziò l'opera di erosione. \(\rightarrow\) In una
    prima fase, e fino a enormi profondità, la zona alla deriva spinse i sedimenti depositati in fondo al
    mare avanti a sé. La resistenza della massa di terraferma situata a settentrione provocò il
    piegamento in falde di questi sedimenti. Prime a piegarsi furono le sedimentazioni più molli del
    fondo marino situate nella zona penninica. Presumibilmente in seguito a tale movimento il fondale
    marino emerse dall'acqua formando lunghe catene di isole
\end{snippet}

\includesnpt[width=75\%|src=/snippet/static/formazione-alpi-svizzere-3.png]{centered-img}

\begin{snippet}{formazione-alpi-part-4}
    Le spinte verso nord continuarono. Le pieghe si sovrapposero le une sulle altre e
    formarono falde coricate. Dai rilievi scesero i primi fiumi che depositarono nel mare i materiali di
    erosione, che poi si trasformarono in quel tipo di roccia che oggi chiamiamo molassa. \(\rightarrow\) A causa
    della pressione sempre crescente le masse sedimentarie alpinoorientali, che inizialmente si
    trovavano ancora molto a sud, si misero in moto fino a sovrapporsi alle pieghe degli strati penninici
    e invasero l'area di sedimentazione elvetica. Queste pieghe, che si protendono molto più in senso
    orizzontale di quanto non si elevino verticalmente con le loro sinclinali ma che ricoprono anche
    superfici caratterizzate da una massa rocciosa di diverso tipo, vengono chiamate falde di
    ricoprimento o «couches»
\end{snippet}

\includesnpt[width=75\%|src=/snippet/static/formazione-alpi-svizzere-4.png]{centered-img}

\begin{snippet}{formazione-alpi-part-5}
    Le ultime spinte da sud agirono fin oltre il margine settentrionale della molassa e
    causarono la formazione della catena del Giura.
\end{snippet}

\subsection{Fasi della formazione delle alpi}

\begin{snippet}{fasi-formazione-alpi}
    \vspace{-1cm}
    \begin{itemize}
        \item Nella prima fase, i sedimenti più molli della zona penninica si piegarono e formarono
        catene di isole.
        \item Nella seconda fase, le masse sedimentarie alpino-orientali si sovrapposero agli strati
        penninici e invasero l'area elvetica, creando strutture chiamate falde di ricoprimento o
        \quotes{couches}. L'intera regione alpina orientale si è formata dalla sovrapposizione di tali falde.
        L'erosione ha poi esposto strati più profondi in alcune aree, come la Bassa Engadina.
        \item La formazione delle Alpi occidentali e orientali è stata più complessa e influenzata da
        ulteriori movimenti geologici. L'erosione ha rimosso gran parte delle falde alpino-orientali,
        lasciando solo resti isolati. La diminuzione del carico sulle masse di magma sottostanti ha
        causato l'emersione del fondo cristallino delle Alpi centrali. L'orogenesi ha sollevato strati
        sovrapposti nell'area elvetica, portando in superficie le masse di granito e gneiss,
        modellando le più alte vette alpine odierne. Nonostante la conclusione della struttura a
        falde delle Alpi, persistono tensioni interne che causano frequenti terremoti di bassa
        intensità.
        \item Non appena l'azione di movimenti interni determina differenze di altitudine, l'erosione
        comincia a demolire le alture in formazione. Gli agenti atmosferici disgregano,
        decompongono e frantumano la roccia. Le acque correnti trasportano i detriti dell'erosione.
        I fiumi approfondiscono sempre più il loro alveo, erodono le montagne e producono il loro
        abbassamento smantellandole.
        \item Dove le Alpi in via di formazione avevano raggiunto la maggiore altezza, cioè nell'arco
        svizzero e nelle catene francesi immediatamente vicine, l'erosione è stata della massima
        intensità. Ha asportato l'intera falda alpino-orientale, strato per strato, lasciandone pochi
        resti.
    \end{itemize}
\end{snippet}

\end{document}
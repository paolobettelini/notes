\documentclass[preview]{standalone}

\usepackage{amsmath}
\usepackage{amssymb}
\usepackage{stellar}
\usepackage{definitions}

\begin{document}

\id{integers-modular-exercises}
\genpage

\section{Exercises}

\begin{snippetexercise}{invertible-classes-ex-1}{}
    Consider the \congruenceclass \({[12]}_{1731}\).
    Determine whether it is \invertiblecongclass[invertible] and its inverse if any.
\end{snippetexercise}

\begin{snippetsolution}{invertible-classes-ex-1-sol}{}
    We compute \(\gcd(1731, 12)\), which is
    \(3\), so their are not \coprime and thus the class is not \invertiblecongclass[invertible].
\end{snippetsolution}

\begin{snippetexercise}{invertible-classes-ex-2}{}
    Consider the \congruenceclass \({[12]}_{1721}\).
    Determine whether it is \invertiblecongclass[invertible] and its inverse if any.
\end{snippetexercise}

\begin{snippetsolution}{invertible-classes-ex-2-sol}{}
    We compute \(\gcd(1721, 12)\), which is
    \(1\), so their are \coprime and thus the class is \invertiblecongclass[invertible].
    To compute the \invertiblecongclass[inverse], we find a Bezout's identity
    \begin{align*}
        1 &= 5 - 2 \cdot 2 = 5-(12 - 5\cdot 2)\cdot 2 \\
        &= 12 \cdot (-2) + 5 \cdot 5 = 12(-2) + (1721 - 12 \cdot 143) \cdot 5 \\
        &= 1721 \cdot 5 + 12 \cdot (-717)
    \end{align*}
    Moving on to the rest of module \(1721\), we find
    \({[1]}_{1721} = {[1721]}_{1721} \cdot {[5]}_{1721} + {[12]}_{1721} \cdot {[-717]}_{1721}\)
    from which \({[1]}_{1721} = {[12]}_{1721} \cdot {[-717]}_{1721}\).
    Thus, \({[12]}_{1721}^{-1} = {[-747]}_{1721} = {[1004]}_{1721}\).
\end{snippetsolution}

\end{document}
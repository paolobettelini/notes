\documentclass[preview]{standalone}

\usepackage{amsmath}
\usepackage{amssymb}
\usepackage{parskip}
\usepackage{fullpage}
\usepackage{hyperref}
\usepackage{tikz}
\usepackage{makecell}
\usepackage{adjustbox}
\usepackage{stellar}
\usepackage{definitions}

\usetikzlibrary{ % tikz packages
    cd % tikz-cd communitative diagrams
}

\newcommand{\opp}[1]{{#1}^{\text{op}}}

\begin{document}

\id{categorytheory}
\genpage

% after abstraction we notice that everything is the same. Every theory is the same thing.

\section{Category}

\begin{snippetdefinition}{category-definition}{Category Definition}
    A \emph{category} is a triple \(C=(\text{ob}(C), \text{Hom}_C, \circ)\) where
    \begin{enumerate}
        \item \(\text{ob}(C)\) is a class of \emph{objects};
        \item for any \(a,b\in\text{ob}(C)\), \(\text{Hom}_C(a,b)\) is a class of \emph{morphisms} or \emph{arrows}
        from \(a\) to \(b\);
        \item \(\circ\) a \binoperation called \emph{composition}.
        For any \(a,b,c\in \text{ob}(C)\),
        \[
            \circ \colon \text{Hom}_C(a,b) \cartesianprod \text{Hom}_C(b,c) \fromto \text{Hom}_C(a,c)
        \]
        denoted \((f,g) \fromto g \circ f\),
    \end{enumerate}
    satisfying the following properties:
    \begin{enumerate}
        \item \emph{associativity of composition}: for all
        \(a,b,c,d\in \text{ob}(C)\) and
        \(f \in \text{Hom}_C(a,b), g \in \text{Hom}_C(b,c), h \in \text{Hom}_C(c,d)\),
        \[
            (h \circ g) = f = h \circ (g \circ f)
        \]
        \item \emph{identity morphism}: for every object \(a\in \text{ob}(C)\),
        there exists a morphism \(\text{id}_a \in \text{Hom}_C(a,a)\) such that for all \(f \in \text{Hom}_C(a,b)\) and \(g \in \text{Hom}_C(b,a)\),
        \[
            f \circ \text{id}_a = f \land \text{id}_a \circ g = g.
        \]
    \end{enumerate}
\end{snippetdefinition}

\begin{snippetdefinition}{locally-small-category-definition}{Locally small category}
    A \category is said to be \emph{locally small}
    if \(\forall a,b \in \catob_C\) we have a \set of morphisms
    from \(a\) to \(b\).
\end{snippetdefinition}

\begin{snippetexample}{category-example-illustration}{Category illustration}
    \[
        \begin{tikzcd}
            a \arrow[r, bend left] \arrow[r, bend left=49, shift left] \arrow[loop, distance=2em, in=215, out=145] & b \arrow[loop, distance=2em, in=35, out=325] \arrow[l, bend left]
        \end{tikzcd}
        \\
        \makecell[l] {
            \text{An example of}
            \\
            \text{objects and morphisms}
        }
    \]
\end{snippetexample}

\begin{snippetdefinition}{dual-category-definition}{Dual Category}
    Let \(C = (\catob(C), \cathom_C, \circ_C)\) be a \category. The \emph{dual category} or \emph{opposite category} of \(C\)
    is defined as
    \[
        C^\text{op} = (\catob(C), \cathom_C^{\text{op}}, \circ_C^{\text{op}})
    \]
    where \(\forall a,b,c\in \catob(C)\), \[\cathom_C^{\text{op}}(a,b) = \cathom_C(b,a)\]
    and \(\forall f \in \cathom_C^{\text{op}}(a,b), \forall g \in \cathom_C^{\text{op}}(b,c)\),
    \[
        g \circ_C^{\text{op}} f = f \circ_C g.
    \]
\end{snippetdefinition}

\plain{The dual category is the same category but with every morphism inverted.}

\plain{A statement is valid in all categories if and only if its dual is.}

\begin{snippetexample}{dual-category-example}{Dual category}
    \begin{center}
        % https://tikzcd.yichuanshen.de/#N4Igdg9gJgpgziAXAbVABwnAlgFyxMJZABgBpiBdUkANwEMAbAVxiRDpAF9T1Nd9CKMgEYqtRizYAjLjxAZseAkWGlR1es1aIQAY1m9FAogGZyYzZJ0duh-spRn14rdIPy+SwcgAsaixLaelxiMFAA5vBEoABmAE4QALZIZCA4EEiqLlYgMSDUDHRSMAwACp7GOnFY4QAWOO7xSZnU6UgATBqBbOH5IIXFZRUOINV1Dba5CcmIqW2IndlB4QAEADprulhxuit5k00zfmkZiGZLbDEAesAbODAAHjjAEGicnI3TSACsrafnliC11ua3uTxeb046022124Rud0ez1e7wAvMAABSrDZbHZ7ACUnARoKREPefQGJXKRhGY3qfTgtSwMQaswOX0QvxOSGOgJ6xLByMhn2aiCy83anAonCAA
        \begin{tikzcd}
        a \arrow[d, "f"'] \arrow[rd, "g \circ f"] &   &  & a                            &                                                                                                         \\
        b \arrow[r, "g"'] \arrow[r]               & c &  & b \arrow[u, "f^{\text{op}}"] & c \arrow[lu, "f^{\text{op}} \circ g^{\text{op}}={(g \circ f)}^{\text{op}}"'] \arrow[l, "g^{\text{op}}"]
        \end{tikzcd}
    \end{center}
\end{snippetexample}

\section{Types of morphisms}

\begin{snippetdefinition}{epimorphism-definition}{Epimorphism}
    Let \(\mathcal{C}\) be a \category
    and let \(f \colon X \fromto Y\) be a morphism in \(\mathcal{C}\).
    Then, \(f\) is said to be an \emph{epimorphism} if
    for all morphisms \(g_1, g_2 \colon Y \fromto Z\) in \(\mathcal{C}\),
    \[
        g_1 \circ f = g_2 \circ f \implies g_1 = g_2.
    \]
    An epimorphism is denoted \(\twoheadrightarrow\).
\end{snippetdefinition}

\plain{An epimorphism is a morphism that is right-cancellable.
It is the categorical generalization of a surjective function.}

\begin{snippet}{surjectivity-from-morphisms-expl}
    Consider a morphism \(f\colon a \rightarrow b\) which maps elements of \(a\) onto \(b\).
    Let's also define the morphisms \(g_1\) and \(g_2\) which map elements from \(b\) to \(c\).
    The condition for an epimorphism is equivalent to requiring the following
    diagram to commute:
    \begin{center}
        \begin{tikzcd}
            a \arrow[r, "f"] &
            b \arrow[r, "g_1", shift left]
            \arrow[r, "g_2"', shift right] & c
        \end{tikzcd}
    \end{center}
    The domain of \(g_1\) and \(g_2\) is the codomain of \(f\). These two functions act
    as \(f\) for object in the image of \(f\), but may map objects differently
    for objects in the codomain of \(f\) but outside the image of \(f\).
    If the morphism is surjective, hence if the codomain and the image of \(f\) are the same,
    then \(g_1\) and \(g_2\) will always act as \(f\).
\end{snippet}

\begin{snippetdefinition}{monomorphism-definition}{Monomorphism}
    Let \(\mathcal{C}\) be a \category
    and let \(f \colon X \fromto Y\) be a morphism in \(\mathcal{C}\).
    Then, \(f\) is said to be an \emph{monomorphism} if
    for all morphisms \(g_1, g_2 \colon Z \fromto X\) in \(\mathcal{C}\),
    \[
        f \circ g_1 = f \circ g_2 \implies g_1 = g_2.
    \]
    A monomorphism is denoted \(\hookrightarrow\).
\end{snippetdefinition}

\plain{A monomorphism is a morphism that is left-cancellable.
It is the categorical generalization of an injective function.}

\begin{snippet}{injectivity-from-morphisms-expl}
    The condition for a monomorphism is equivalent to requiring the following
    diagram to commute:
    \begin{center}
        \begin{tikzcd}
            c \arrow[r, "g_1", shift left]
            \arrow[r, "g_2"', shift right] &
            a \arrow[r, "f"] & b
        \end{tikzcd}
    \end{center}
\end{snippet}

\begin{snippetdefinition}{cat-theory-isomorphism-definition}{Isomorphism}
    Let \(\mathcal{C}\) be a \category
    and let \(f \colon X \fromto Y\) be a morphism in \(\mathcal{C}\).
    Then, \(f\) is said to be an \emph{isomorphism} if
    there exists a morphism \(g \colon Y \fromto X\) in \(\mathcal{C}\)
    such that
    \[
        g \circ f = \text{id}_X \land f \circ g = \text{id}_Y
    \]
    An isomorphism is labelled with \(\xrightarrow{\sim}\).
\end{snippetdefinition}

\subsection{Split-mono and epimorphisms}

\plain{Note that every isomorphism is both mono and epic, but the converse is not true in general.
We can however require a stronger hypothesis.}

\begin{snippetdefinition}{split-monomorphism-definition}{Split-monomorphism}
    Let \(\mathcal{C}\) be a \category. A morphism \(f\colon A \fromto B\)
    is said to be a \emph{split-monomorphism} if there exists a morphism
    \(s \colon B \fromto A\) (called \emph{retration} or \emph{right inverse}) such that
    \[
        s \circ f = \text{id}_A
    \]
\end{snippetdefinition}

\begin{snippetdefinition}{split-epimorphism-definition}{Split-epimorphism}
    Let \(\mathcal{C}\) be a \category. A morphism \(f\colon A \fromto B\)
    is said to be a \emph{split-monomorphism} if there exists a morphism
    \(s \colon B \fromto A\) (called \emph{section} or \emph{left inverse}) such that
    \[
        f \circ s = \text{id}_B
    \]
\end{snippetdefinition}

\begin{snippetproposition}{split-monomorphisms-are-monomorphisms}{}
    Let \(\mathcal{C}\) be a \category and \(f\)
    a split monomorphism. Then, \(f\) is also a monomorphism.
\end{snippetproposition}

\begin{snippetproposition}{split-epimorphisms-are-epimorphisms}{}
    Let \(\mathcal{C}\) be a \category and \(f\)
    a split epimorphism. Then, \(f\) is also an epimorphism.
\end{snippetproposition}

\begin{snippetproof}{split-monomorphisms-are-monomorphisms-proof}{split-monomorphisms-are-monomorphisms}{}
    \todo
\end{snippetproof}

\begin{snippetproof}{split-epimorphisms-are-epimorphisms-proof}{split-epimorphisms-are-epimorphisms}{}
    \todo
\end{snippetproof}

\begin{snippetproposition}{split-mono-and-epi-is-iso}{}
    Let \(\mathcal{C}\) be a \category and \(f\)
    be a split monomorphism and an epimorphism. Then,
    \(f\) is also a \catisomorphism.
\end{snippetproposition}

\begin{snippetproposition}{split-epi-and-mono-is-iso}{}
    Let \(\mathcal{C}\) be a \category and \(f\)
    be a split epimorphism and a monomorphism. Then,
    \(f\) is also a \catisomorphism.
\end{snippetproposition}

\begin{snippetproof}{split-mono-and-epi-is-iso-proof}{split-mono-and-epi-is-iso}{}
    \todo
\end{snippetproof}

\begin{snippetproof}{split-epi-and-mono-is-iso-proof}{split-epi-and-mono-is-iso}{}
    \todo
\end{snippetproof}

\end{document}
\documentclass[preview]{standalone}

\usepackage{amsmath}
\usepackage{amssymb}
\usepackage{tikz}
\usepackage{float}
\usepackage{xcolor}
\usepackage{stellar}
\usepackage{definitions}

\begin{document}

\id{differentiation-rules}
\genpage

\section{Rules for differentiation}

\subsection{Linear combination}

\begin{snippetproposition}{derivative-of-linear-combination}{Derivative of linear combination}
    Let \(f(x)\) and \(g(x)\) be differentiable \function[functions] and \(a,b\in\realnumbers\).
    Then, \[
        \frac{d}{dx}\left(af(x) + bg(x)\right) = a\frac{d}{dx}\left(f(x)\right)
        + a\frac{d}{dx}\left(f(x)\right)
    \]
\end{snippetproposition}

\begin{snippetproof}{derivative-of-linear-combination-proof}{derivative-of-linear-combination}{Derivative of linear combination}
    \begin{align*}
        \frac{d}{dx}\left(af(x) + bg(x)\right) &= \lim_{h\to 0}
        \frac{\left[af(x+h) + bg(x+h) \right] - \left[af(x) + bg(x) \right]}{h} \\
        &= a\frac{d}{dx}\left(f(x)\right) + b\frac{d}{dx}\left(g(x)\right)
    \end{align*}
\end{snippetproof}

\subsection{Chain Rule}

\begin{snippettheorem}{real-function-chain-rule}{Chain Rule}
    Let \(f\colon I \subseteq \realnumbers \fromto \realnumbers\) and \(g\colon J \subseteq \realnumbers \fromto \realnumbers\)
    be \function[functions] and let \(x_0\) internal such that
    \(f\) is \differentiable in \(x_0\) and \(g\) is \differentiable in \(y_0 = f(x_0)\).
    Then,
    \[
        (g\circ f)'(x_0) = g'(y_0)f'(x_0)
    \]
\end{snippettheorem}

\begin{snippetproof}{chain-rule-proof}{chain-rule}{Chain Rule}
    We want to show that the following limit exists and is finite
    \[
        \lim_{x\to x_0} \frac{g(f(x)) - g(f(x_0))}{x-x_0}
        = \derivativeD g(f(x_0)) \cdot \derivativeD f(x_0)
    \]
    Let \(\varphi \colon J \fromto \realnumbers\) defined as
    \[
        \varphi(y) \triangleq \begin{cases}
            \frac{g(y) - g(y_0)}{y-y_0} & y \neq y_0 \\
            g'(y_0) & y = y_0
        \end{cases}
    \]
    By definition of derivative, we have
    \[
        \lim_{y\to y_0} \varphi(y) = \varphi(y_0) = g'(y_0)
    \]
    Note that
    \[
        \forall x \in I,
        \frac{g(f(x)) - g(f(x_0))}{x-x_0}
        = \varphi(f(x)) \cdot \frac{f(x) - f(x_0)}{x-x_0}  
    \]
    Since \differentiable[differentiability] implies \realcontinuous[continuity], we have
    \[
        \varphi(f(x)) \to \varphi(y_0) = \derivativeD g(f(x_0))
    \]
    from which
    \[
        \lim_{x\to x_0} \frac{g(f(x)) - g(f(x_0))}{x-x_0}
        = \lim_{x\to x_0} \varphi(f(x)) \cdot \frac{f(x) - f(x_0)}{x-x_0}
    \]
    which is the thesis.
\end{snippetproof}

\subsection{Inverse}

\begin{snippetproposition}{real-function-inverse-derivative}{Derivative of inverse}
    Let \(f\colon I \fromto \realnumbers\) be a \function[function]
    and \(x_0\) internal point of \(I\) such that \(f\) is invertible in \(I\) and \differentiable in \(x_0\).
    Suppose that:
    \begin{enumerate}
        \item \(\derivativeD f(x_0) \neq 0\);
        \item \(f^{-1}\) is \realcontinuous in \(y_0 = f(x_0)\);
    \end{enumerate}
    Then, the inverse function \(f^{-1}\colon f(I) \fromto I\) is \differentiable in \(y_0 = f(x_0)\) and
    \[
        \derivativeD f^{-1}(y_0) = \frac{1}{\derivativeD f(x_0)}
        = \frac{1}{\derivativeD f(f^{-1}(y_0))}
    \]
\end{snippetproposition}

\begin{snippetproof}{real-function-inverse-derivative-proof}{real-function-inverse-derivative}{Derivative of inverse}
    We study the limit
    \begin{align*}
        \lim_{y\to y_0} \frac{f^{-1}(y) - f^{-1}(y_0)}{y-y_0}
    \end{align*}
    We let \(f^{-1}(y) = x\) and \(f^{-1}(y_0) = x_0\),
    we have \(f(x) = y\) and \(f(x_0) = y\) and by \realcontinuous[continuity]
    of \(f^{-1}\) in \(y_0\) if \(y\to y_0\),
    then \(f^{-1}(y) = x \to f^{y_0} = x_0\).
    hence, we can apply the change of variable
    \begin{align*}
        \lim_{x\to x_0} \frac{x-x_0}{f(x) - f(x_0)}
        &= \lim_{x\to x_0} = \frac{1}{\frac{f(x) - f(x_0)}{x-x_0}} \\
        &= \frac{1}{Df(x_0)}
    \end{align*}
\end{snippetproof}

\subsection{Product}

\begin{snippetproposition}{derivative-of-product}{Derivative of product}
    Let \(f(x)\) and \(g(x)\) be differentiable \function[functions].
    Then, \[
        \frac{d}{dx}\left(f(x)g(x)\right) =\frac{df(x)}{dx}g(x) + \frac{dg(x)}{dx}f(x)
    \]
\end{snippetproposition}

\begin{snippetproof}{derivative-of-product-proof}{derivative-of-product}{Derivative of product}
    \begin{align*}
        \frac{d}{dx}\left(f(x)g(x)\right) &= \lim_{h \to 0}
        \frac{f(x+h)g(x+h) - f(x)g(x)}{h} \\
        &= \lim_{h \to 0} \frac{f(x+h)g(x+h) - f(x)g(x) -f(x)g(x+h) + f(x)g(x+h)}{h} \\
        &= \lim_{h \to 0} \frac{[f(x+h)-f(x)]g(x+h) + f(x)[g(x+h)-g(x)]}{h} \\
        &= \lim_{h \to 0} g(x+h)\frac{f(x+h)-f(x)}{h} + \lim_{h \to 0} f(x+h)\frac{g(x+h)-g(x)}{h} \\
        &= g(x) \frac{df(x)}{dx} + f(x) \frac{dg(x)}{dx}
    \end{align*}
\end{snippetproof}

\begin{snippetproposition}{derivative-of-division}{Derivative of division}
    Let \(f(x)\) and \(g(x)\) be differentiable \function[functions].
    Then, \[
        \frac{d}{dx}\frac{f(x)}{g(x)} = \frac{f'(x)g(x) - g'(x)f(x)}{g^2(x)}
    \]
\end{snippetproposition}

\begin{snippetproof}{derivative-of-division-proof}{derivative-of-division}{Derivative of division}
    \begin{align*}
        \frac{d}{dx}\frac{f(x)}{g(x)} &= \frac{d}{dx}f(x)\cdot\frac{1}{g(x)} \\
        &= f(x) \cdot \frac{d}{dx}g^{-1}(x) + g^{-1}(x) \cdot \frac{d}{dx}f(x) \\
        &= -f(x) g^{-2}(x) g'(x) + g^{-1}(x) f'(x) \\
        &= \frac{-f(x) g'(x)}{g^2} + \frac{f'(x)g(x)}{g^2(x)} \\
        &= \frac{f'(x)g(x) - g'(x)f(x)}{g^2(x)}
    \end{align*}
\end{snippetproof}

\subsection{Exponent}

\begin{snippettheorem}{derivative-of-exponent-theorem}{Derivative of exponent}
    Let \(n\in{\realnumbers}^\exceptzero\). Then,
    \[
        \frac{d}{dx}(x^n)=nx^{n-1}
    \]
\end{snippettheorem}

\begin{snippetproof}{derivative-of-exponent-theorem-proof}{derivative-of-exponent-theorem}{Derivative of exponent}
    \todo % TODO dimostrazione con l'asintotico 22 nov

    % qua ho fatto 1/m ma si potrebbe fare m/n
    If we have \(x^\frac{1}{m}\), we can set \(f(x)=x^m\) and \(g(x) = x^\frac{1}{m}\)
    so \(h(x)=f(g(x)) = 1\).
    We thus have
    \begin{align*}
        1 = \frac{dh(x)}{dx} &= \frac{df(y)}{dy} \cdot \frac{dg(x)}{dx} \\
        &= \frac{df(y)}{dy}\cdot mx^{m-1}
    \end{align*}
    and
    \begin{align*}
        \frac{df(y)}{dy} = \frac{1}{mx^{m-1}}
        &= \frac{1}{my^\frac{m-1}{m}} \\
        &= \frac{1}{m} y^{\frac{1}{m}-1}
    \end{align*}
    So the properties holds in \(\rationalnumbers\).
\end{snippetproof}

\subsection{Exponential function}

\begin{snippettheorem}{derivative-of-exponential-function}{Derivative of exponential function}
    \todo
    \[
        \frac{d}{dx} a^x = \ln(a) a^x
    \]
\end{snippettheorem}

\begin{snippetproof}{derivative-of-exponential-function-proof}{derivative-of-exponential-function}{Derivative of exponential function}
    \todo
    \begin{align*}
        \lim_{h \to 0} \frac{e^{x+h}-e^x}{h} &= \lim_{h \to 0} e^x \frac{e^h-1}{h} \\
        &= e^x \lim_{h \to 0} \frac{e^h-1}{h} \\
        &= e^x \lim_{h \to 0} \lim_{n\to\infty} \frac{{\left(1+\frac{1}{n}\right)}^{nh}-1}{h} \\
        &= e^x
    \end{align*}
\end{snippetproof}

\subsection{Logarithm}

\begin{snippettheorem}{derivative-of-Logarithm-theorem}{Derivative of logarithm}
    \todo
    \[
        \frac{d}{dx} \ln(f(x)) = \frac{f'(x)}{f(x)}
    \]
\end{snippettheorem}

\begin{snippetproof}{derivative-of-Logarithm-theorem-proof}{derivative-of-Logarithm-theorem}{Derivative of logarithm}
    \todo
    %Let \(y = g(x) = e^x\) and \(h(x) = f(g(y))\)
    %\begin{align*}
    %    1 = \frac{dh(x)}{dx} &= \frac{df(y)}{dy} \cdot e^x \\
    %    &= \frac{df(y)}{dy} y
    %\end{align*}
    %so we get
    %\[
    %    \frac{df(y)}{dy} = \frac{1}{y}
    %\]
\end{snippetproof}

\subsection{Trigonometric functions}

\begin{snippettheorem}{derivative-of-sine-theorem}{Derivative of sine}
    \[
        \frac{d}{dx} \sin(x) = \cos(x)
    \]
\end{snippettheorem}

\begin{snippetproof}{derivative-of-sine-theorem-proof}{derivative-of-sine-theorem}{Derivative of sine}
    \begin{align*}
        \lim_{h\to 0} \frac{\sin(x+h) - \sin(x)}{h} &=
        \lim_{h\to 0} \frac{\sin(x)\cos(h) + \cos(x)\sin(h) - \sin(x)}{h} \\
        &= \left\{\lim_{h\to 0} \sin(x) \frac{\cos(h) - 1}{h} + \cos(x) \frac{\sin(h) - 1}{h}\right\} \\
        &= \sin(x) \lim_{h\to 0} \frac{\cos(h) - 1}{h} + \cos(x) \lim_{h\to 0} \frac{\sin(h) - 1}{h} \\
        &= \cos(x)
    \end{align*}
\end{snippetproof}

\begin{snippettheorem}{derivative-of-cosine-theorem}{Derivative of cosine}
    \[
        \frac{d}{dx} \cos(x) = -\sin(x)
    \]
\end{snippettheorem}

\begin{snippetproof}{derivative-of-cosine-theorem-proof}{derivative-of-cosine-theorem}{Derivative of cosine}
    \begin{align*}
        \lim_{h\to 0} \frac{\cos(x+h) - \cos(x)}{h} &=
        \lim_{h\to 0} \frac{\cos(x)\cos(h) - \sin(x)\sin(h) - \cos(x)}{h} \\
        &= \left\{\lim_{h\to 0} \cos(x) \frac{\cos(h) - 1}{h} - \sin(x) \frac{ \sin(h)}{h}\right\} \\
        &= \lim_{h\to 0} \cos(x) \frac{\cos(h) - 1}{h} - \lim_{h\to 0} \sin(x) \frac{ \sin(h)}{h} \\
        &= -\sin(x)
    \end{align*}
\end{snippetproof}

\subsection{Miscellaneous}

\begin{snippet}{differentiation-rules}
    \[
        \frac{d}{dx}(f^g)=f^g\left(\frac{f'g}{f}+g'\ln f\right)
    \]
\end{snippet}

\end{document}
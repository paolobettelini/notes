\documentclass[preview]{standalone}

\usepackage{amsmath}
\usepackage{amssymb}
\usepackage{stellar}
\usepackage{definitions}
\usepackage{makecell}
\usepackage{boldline}
\usepackage{amsthm}

\begin{document}

\id{binary-operations}
\genpage

\section{Cayley tables}

\begin{snippet}{cayley-table-illustration}
    A \binoperation \(\circ\) on a finite set \(G\) can be
    visualized using a \textit{Cayley table}.
    Example: \(G=\integers /_4\).
    \vspace{0.4cm}
    \begin{center}
        \bgroup{}
        \def\arraystretch{1.25}
        \begin{tabular}{|c !{\vrule width0.8pt} c|c|c|c|}
            \hline
            \(+\) & \({[0]}_4\) & \({[1]}_4\) & \({[2]}_4\) & \({[3]}_4\) \\
            \Xhline{0.8pt}
            \({[0]}_4\) & \({[0]}_4\) & \({[1]}_4\) & \({[2]}_4\) & \({[3]}_4\) \\
            \hline
            \({[1]}_4\) & \({[1]}_4\) & \({[2]}_4\) & \({[3]}_4\) & \({[0]}_4\) \\
            \hline
            \({[2]}_4\) & \({[2]}_4\) & \({[3]}_4\) & \({[0]}_4\) & \({[1]}_4\) \\
            \hline
            \({[3]}_4\) & \({[3]}_4\) & \({[0]}_4\) & \({[1]}_4\) & \({[2]}_4\) \\
            \hline
        \end{tabular}
        \egroup{}
        \hspace{1cm}
        \bgroup{}
        \def\arraystretch{1.25}
        \begin{tabular}{|c !{\vrule width0.8pt} c|c|c|c|}
            \hline
            \(\cdot\) & \({[0]}_4\) & \({[1]}_4\) & \({[2]}_4\) & \({[3]}_4\) \\
            \Xhline{0.8pt}
            \({[0]}_4\) & \({[0]}_4\) & \({[0]}_4\) & \({[0]}_4\) & \({[0]}_4\) \\
            \hline
            \({[1]}_4\) & \({[0]}_4\) & \({[1]}_4\) & \({[2]}_4\) & \({[3]}_4\) \\
            \hline
            \({[2]}_4\) & \({[0]}_4\) & \({[2]}_4\) & \({[0]}_4\) & \({[2]}_4\) \\
            \hline
            \({[3]}_4\) & \({[0]}_4\) & \({[3]}_4\) & \({[2]}_4\) & \({[1]}_4\) \\
            \hline
        \end{tabular}
        \egroup{}
    %}
    \end{center}
    \phantom{ }
\end{snippet}

\section{Definition}

\begin{snippetdefinition}{binary-operation-definition}{Binary operation}
    Let \(A\) be a \set.
    A \textit{binary operation} on \(A\)
    is a \function \(\circ\colon A \cartesianprod A \fromto A\).
\end{snippetdefinition}

\begin{snippet}{implicit-binary-operation}
    The operation \(\circ\) between \(a\) and \(b\) may be written as
    \(a\circ b\) or just \(ab\).
\end{snippet}

\begin{snippetdefinition}{algebraic-structure-definition}{Algebraic structure}
    An \textit{algebraic structure} \((S, \circ)\) is a tuple containing
    a \set \(S\) and a \binoperation \(\circ\) on \(S\). 
\end{snippetdefinition}

\begin{snippetdefinition}{algebraic-structure-order-definition}{Algebraic structure order}
    Let \((S, \circ)\) be an \algebraicstructure.
    Then, its \textit{order} is given by \(\cardinality{S}\).
\end{snippetdefinition}

\section{Basic results}

\begin{snippettheorem}{uniqueness-of-the-identity-element}{Uniqueness of the identity element}
    Let \((G, \circ)\) be an \algebraicstructure where \(G\) is a \set and \(\circ\)
    is a \binrelation. Then,
    if \(e\) is an identity element, it is unique.
\end{snippettheorem}

\begin{snippetproof}{uniqueness-of-the-identity-element-proof}{uniqueness-of-the-identity-element}{Uniqueness of the identity element}
    Suppose there is more than one identity element, \(e_1\) and \(e_2\).
    \begin{align*}
        e_1 &= e_1 \circ e_2 &\text { since \(e_2\) is an identity} \\
        &= e_2 &\text { since \(e_1\) is an identity}
    \end{align*}
    Thus, \(e_1\) and \(e_2\) must be the same.
\end{snippetproof}

\end{document}
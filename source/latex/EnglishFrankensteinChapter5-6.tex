\documentclass[preview]{standalone}

\usepackage{amsmath}
\usepackage{amssymb}
\usepackage{bettelini}
\usepackage{stellar}

\hypersetup{
    colorlinks=true,
    linkcolor=black,
    urlcolor=blue,
    pdftitle={English},
    pdfpagemode=FullScreen,
}

\begin{document}

\title{English}
\id{english-frankenstein-chapter-5-6}
\genpage

\section{Exercises}

\begin{snippetexercise}{frankenstein-ex-15}
    {The morning after the awakening of the creature, Victor wanders the streets of Ingolstadt struck
    by fear. On p. 73, Shelley inserts into the novel a passage from a poem by romantic poet Samuel
    Taylor Coleridge, \quotes{The Rime of the Ancient Mariner}. What is the passage about? How are these
    lines significant in relation to the novel? Use the glossary in your edition of the book to understand
    difficult words.}
    \begin{center}
        \quotes{\textit{Like one who, on a lonely road,} \\
        \textit{Doth walk in fear and dread,} \\
        \textit{And, having once turned round, walks on,} \\
        \textit{And turns no more his head;} \\
        \textit{Because he knows a frightful fiend} \\
        \textit{Doth close behind him tread.}}
    \end{center}
    
    TODO 
\end{snippetexercise}

\begin{snippetexercise}{frankenstein-ex-16}
    {What do you think of the fact that letters are incorporated into Victor's story? Focus in particular
    on the following aspects: what effects does this have on the readers? What effect does the
    incorporation of letters have on the overall narrative? Remember that Victor tells his story to Robert
    Walton, who then records it in his own letters to his sister.}
    
    The letters allow for another point of view, but make the narrative less plausible
    (how could Victor remember each word and how could Walton write all of them?).
\end{snippetexercise}

\begin{snippetexercise}{frankenstein-ex-17}
    {How does the ending of chapter 6 create a contrast with the atmosphere in the previous chapter?
    What effects does this have on the readers?}
    
    The ending of chapter 6 is happy, light and joyful, whereas the previous
    ended in a monstly crazy manner. The result is that the readers are fullfilled with anxiety
    as if something bad and sad is going to happen, ebcause we don't know what to except.
    This concept is related to the feelings and health of Victor.
\end{snippetexercise}

\end{document}

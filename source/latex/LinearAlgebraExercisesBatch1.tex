\documentclass[preview]{standalone}

\usepackage{amsmath}
\usepackage{amssymb}
\usepackage{stellar}
\usepackage{definitions}

\begin{document}

\id{linearalgebra-exercises-batch-1}
\genpage

\section{Exercises}

\begin{snippetexercise}{linear-algebra-batch-1-ex-1}{}
    \todo
\end{snippetexercise}

\begin{snippetsolution}{linear-algebra-batch-1-ex-1-sol}{}
    \todo
\end{snippetsolution}

\begin{snippetexercise}{linear-algebra-batch-1-ex-2}{}
    Determine the reciprocal positions of the following lines:
    \[
        r_k = \{(x,y) \in \realnumbers^2 \suchthat x+ky = 1\}
        \qquad
        s_k = \{(x,y) \in \realnumbers^2 \suchthat kx + y = k^2\}
    \]
    for \(k\in\realnumbers\).
\end{snippetexercise}

\begin{snippetsolution}{linear-algebra-batch-1-ex-2-sol}{}
    If \(k=0\), we have
    \[
        r_k = \{(x,y) \in \realnumbers^2 \suchthat x= 1\}
        \qquad
        s_k = \{(x,y) \in \realnumbers^2 \suchthat y = 0\}
    \]
    which are orthogonal.
    Otherwise, we have
    \[
        r_k = \left\{(x,y) \in \realnumbers^2 \suchthat y = -\frac{1}{k} + \frac{1}{k}\right\}
        \qquad
        s_k = \left\{(x,y) \in \realnumbers^2 \suchthat y = -kx + k^2\right\}
    \]
    If \(-1/k = -k\), we have \(k = \pm 1\).
    If \(k = 1\), we have \(y = -x + 1\) and \(y = -x + 1\), which are the same lines.
    If \(k = -1\), we have \(y = x-1\) and \(y = x + 1\), which are parallel lines.
    Otherwise, the lines are intersecting.
\end{snippetsolution}

\begin{snippetexercise}{linear-algebra-batch-1-ex-3}{}
    Consider the linear system
    \[
        \begin{cases}
            a x + by = c \\
            \alpha x + \beta y = \gamma
        \end{cases}
    \]
    Show that the following are equivalent:
    \begin{enumerate}
        \item \(\Delta = a\beta - b\alpha = 0\);
        \item there exist \(u,v \in \realnumbers\) such that \((u,v) \neq (0,0)\) and
        \[
            u\begin{pmatrix}
                a \\ \alpha
            \end{pmatrix}
            +
            v\begin{pmatrix}
                b \\ \beta
            \end{pmatrix}
            = \begin{pmatrix}
                0 \\ 0
            \end{pmatrix}
        \]
        \item there exist \(u,v \in \realnumbers\) such that \((u,v) \neq (0,0)\) and
        \[
            u\begin{pmatrix}
                a \\ b
            \end{pmatrix}
            +
            v\begin{pmatrix}
                \alpha \\ \beta
            \end{pmatrix}
            = \begin{pmatrix}
                0 \\ 0
            \end{pmatrix}
        \]
    \end{enumerate}
\end{snippetexercise}

\begin{snippetsolution}{linear-algebra-batch-1-ex-3-sol}{}
    \begin{itemize}
        \item \((1) \implies (2):\) if \((\alpha, \beta) \neq (0,0)\), then
        we can let \(u = \beta\) and \(v = -\alpha\). Otherwise,
        if \(a \neq 0\), we can let \(u = \frac{-vb}{a}\) and \(v\in\realnumbers^\exceptzero\).
        Otherwise, we can let \(u \in\realnumbers^\exceptzero\) and \(v = 0\).
        \item \((2) \implies (1):\) if \(u\neq 0\), we have
        \[
            a = \frac{-vb}{u}
            \qquad
            \alpha = \frac{-v\beta}{u}
        \]
        meaning that \[
            \Delta = a\beta - b\alpha = \frac{-v}{u}(b\beta - a\alpha) = 0
        \]
        If \(v \neq 0\) we have
        \[
            b = \frac{-ua}{v}
            \qquad
            \beta = \frac{-u\alpha}{v}
        \]
        meaning that
        \[
            \Delta = a\beta - b\alpha = \frac{-u}{v}(b\alpha - a\beta) = 0
        \]
        \item \todo
    \end{itemize}
\end{snippetsolution}

%\begin{snippetexercise}{linear-algebra-batch-1-ex-4}{}
%    \todo
%\end{snippetexercise}
%
%\begin{snippetsolution}{linear-algebra-batch-1-ex-4-sol}{}
%    \todo
%\end{snippetsolution}

\end{document}
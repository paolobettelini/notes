\documentclass[preview]{standalone}

\usepackage{amsmath}
\usepackage{amssymb}
\usepackage{stellar}
\usepackage{chemfig}

\hypersetup{
    colorlinks=true,
    linkcolor=black,
    urlcolor=blue,
    pdftitle={Stellar},
    pdfpagemode=FullScreen,
}

\begin{document}

\title{Stellar}
\id{chimica-reazioni-chimiche}
\genpage

\section{Il linguaggio delle formule}

\begin{snippet}{il-linguaggio-delle-formule}
    A pedice di ogni simbolo chimico si indica quanti atomi o ioni di un elemento sono presenti
    nella molecola, a condizione che il numero di essi sia maggiore di 1 unità:
    \vspace{0.4cm}
    \begin{center}
        {\centering \chemfig{\Charge{-135=\|,135=\|}{O}=[:00]\Charge{45=\|,-45=\|}{O}} \textrightarrow\ O$_2$}
    \end{center}
    \vspace{0.4cm}
    Quando una sostanza è formata da \underline{ioni}, nella formula si indica solo la
    carica complessiva della particella:
    \vspace{0.4cm}
    \begin{center}
        {\centering
        \chemfig{H-[:45]{\Charge{135=\|,45=\|}{O}}-[:-45]C(=[:-90]{\Charge{-135=\|,-45=\|}{O}})-[:45]{\Charge{135=\|,45=\|,-45=\|}{O}}^{-}}
        \textrightarrow\ HCO$^-_3$
        }
    \end{center}
    \vspace{0.4cm}
    Se il numero precede la formula, allora esso indica due atomi o molecole separati:
    \vspace{0.4cm}
    \begin{center}
        {\centering \chemfig{\Charge{-135=\|,135=\|}{O}=[:00]\Charge{45=\|,-45=\|}{O}}
        } +\:
        {\centering \chemfig{\Charge{-135=\|,135=\|}{O}=[:00]\Charge{45=\|,-45=\|}{O}}
        }
        \textrightarrow\ 2O$_2$
    \end{center}
    \vspace{0.4cm}
    Le parentesi tonde sono utilizzate nelle formule chimiche per indicare gruppi di atomi
    che si ripetono all'interno della molecola:
    \vspace{0.4cm}
    \begin{center}
        {\centering CO(NH$_2$)$_2$
        }
    \end{center}
\end{snippet}

\section{Le formule dei composti}

\plain{I composti sono costituiti da atomi o ioni di elementi differenti uniti in proporzioni fisse.}

\begin{snippetdefinition}{composto-molecolare-definition}{Composto molecolare}
    I \textit{composti molecolari} sono formati da atomi uniti da legami ovalenti per
    formare molecole vere e proprie.
\end{snippetdefinition}

\includesnpt[width=25\%|src=/snippet/static/molecular-compound.png]{centered-img}

\begin{snippet}{4e0a309d-1f6b-40ed-8aaa-3878ffbf7a1d}
    CH$_3$OH + CO \textrightarrow\ HCOOCH$_3$
\end{snippet}

\begin{snippetdefinition}{composto-ionico-definition}{Composto ionico}
    I \textit{composti ionici} sono costituiti da ioni positivi e negativi
    uniti da una forza di attrazione elettrica, ossia legame ionico.
\end{snippetdefinition}

\includesnpt[width=25\%|src=/snippet/static/ionic-compound.png]{centered-img}

\begin{snippet}{6184c191-2ae9-49d7-bd7d-e36119dfc9f1}
    \chemfig{\Charge{90=\.}{Na} + \Charge{90=\.,0=\|,-90=\|,180=\|}{Cl}} \textrightarrow\ 
    \chemleft[\chemfig{Na}\chemright]$^+$\
    \chemleft[\chemfig{\Charge{90=\|,0=\|,-90=\|,180=\|}{Cl}}\chemright]$^-$
\end{snippet}

\section{Le equazioni chimiche}

\begin{snippet}{equazione-chimica-expl}
    Un'equazione chimica rappresentazione una reazione chimica mostrando come i reagenti si trasformano
    in prodotti. Le formule chimiche dei reagenti e dei prodotti illustrano i rapporti numerici tra
    le particelle coinvolte:
    \begin{itemize}
        \item La freccia segnala la direzione in cui procede la reazione:
            \begin{itemize}
                \item da sinistra verso destra (\textrightarrow) i reagenti diventano prodotti;
                \item da destra verso sinistra (\textleftarrow) i prodotti tornano reagenti;
                \item in entrambe le direzioni ($\rightleftharpoons$) la reazione è reversibile.
            \end{itemize}
        \item Il simbolo presente sopra la freccia indica i processi o contributi al processo che
            avvengono durante la reazione chimica:
            \begin{itemize}
                \item \schemestart \arrow{->[$\Delta$][]} \schemestop
                    indica il riscaldamento dei reagenti;
                \item \schemestart \arrow{->[$\uparrow$][]} \schemestop
                    indica la liberazione di gas;
                \item \schemestart \arrow{->[$\downarrow$][]} \schemestop
                    indica la formazione di precipitato.
            \end{itemize}
        \item Alcune annotazioni poste alla destra dell'atomo o della molecola indicano
            lo stato fisico dei reagenti e dei prodotti:
            \begin{itemize}
                \item X($g$): l'atomo o la molecola si trova in stato gassoso;
                \item X($s$): l'atomo o la molecola si trova in stato solido;
                \item X($l$): l'atomo o la molecola si trova in stato liquido;
                \item X($aq$): l'atomo o la molecola si trova in una soluzione acquosa.
            \end{itemize}
        \item I coefficienti stechiometrici posti davanti alle formule indicano la proporzione minima
            di molecole o ioni di ciascun reagente necessario per la reazione e di prodotti formati.
            Durante questo procedimento viene \textbf{\underline{bilanciata}} l'equazione chimica. 
            \\
            \begin{center}
                \fbox{\setchemfig{scheme debug=false}
                    \schemestart
                    C$_7$H$_{16}$($l$) + 11O$_2$($g$) \arrow{->[$\Delta$][]} 7CO$_2$($g$) + 8H$_2$O($g$)
                    \schemestop
                }
            \end{center}
    \end{itemize}
\end{snippet}

\end{document}
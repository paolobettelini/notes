\documentclass[preview]{standalone}

\usepackage{amsmath}
\usepackage{amssymb}
\usepackage{stellar}
\usepackage{definitions}

\begin{document}

\id{italiano-canzoniere}
\genpage

\section{Rerum vulgarium fragmenta (Il Canzoniere)}

\begin{snippet}{il-canzoniere-introduzione}
    Il testo è composto da 366 poemi e parla del suo amore tormentato per una donna di nome Laura.
    Tuttavia, il protagonista del libro non è Laura, bensì Petrarca che ne parla.
    
    L'opera è separata in due parti: vi è un foglio bianco fra il Rvf 263 e 264,
    ma Laura muore dal 267, lasciando un cuscinetto di 3 testi.
\end{snippet}

\end{document}
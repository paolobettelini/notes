\documentclass[preview]{standalone}

\usepackage{amsmath}
\usepackage{amssymb}
\usepackage{stellar}
\usepackage{definitions}
\usepackage{bettelini}

\begin{document}

\id{implicit-function-theorem}
\genpage

\section{Implicit function theorem}

\begin{snippetdefinition}{implicitly-defined-function-definition}{Implicitly defined function}
    Let \(\Omega \subseteq \realnumbers^2\) open and \(F \colon \Omega \fromto \realnumbers\).
    Consider \(S = \{(x,y) \suchthat F(x,y) = 0\}\).
    A \function \(\varphi \colon I \fromto \realnumbers\) with \(I\) a real interval
    is \emph{implicitly defined} by the equation \(F(x,y) = 0\) if:
    \begin{enumerate}
        \item \((x, \varphi(x)) \in \Omega\) for all \(x \in I\);
        \item \(F(x, \varphi(x)) = 0\) for all \(x \in I\).
    \end{enumerate}
    An analogous definition holds when \(x\) is expressed as a \function of \(y\).
\end{snippetdefinition}

\begin{snippet}{implicitly-defined-function-local-expl}
    In general, we want to know for which conditions
    the equation \(F(x,y) = 0\) implicitly defines a \function \(y(x)\)
    or \(x(y)\).
    
    To do so, we adopt a local approach: under suitable assumptions,
    the equation \(F(x,y) = 0\) implicitly defines locally a unique \function
    of variable \(y\) or variable \(x\).
\end{snippet}

\begin{snippetexample}{implicit-function-bisettrici-example}{}
    Let \(F(x,y) = x^2 - y^2\). The zeros are clearly the bisectors \(y = x\) and \(y = -x\).
    Choosing a point of the form \((5,5)\), everything works:
    there exists a \neighborhood in which the equation \(F = 0\)
    implicitly defines a unique \function \(y = x\) or \(x = y\).
    However, there is a problem at the origin.
    The solution set does not represent the graph of any \function
    as there could be infinitely many.
\end{snippetexample}

\begin{snippetexample}{implicit-function-only-origin-example}{}
    The equation \(F(x,y) = x^2 + y^2 = 0\)
    vanishes only at the origin. Thus, it does not implicitly define
    any \function.
\end{snippetexample}

\begin{snippetexample}{implicit-function-cusp-example}{}
    Let \(F(x,y) = x^3 - y^2\).
    Its zeros form a curve with a cusp at the origin.
\end{snippetexample}

\begin{snippet}{implicit-function-motivation-expl}
    In the unit circle, we can express a local \function for both variables
    except at the 4 intersections with the coordinate axes.
    At two of these points, we can only express the \function in terms of \(x\),
    while at the other two, only in terms of \(y\).
    
    Consider the Taylor expansion.
    Let \((x_0, y_0) \in \Omega\) such that \(F(x_0, y_0) = 0\).
    Expanding \(F\) to first order:
    \begin{align*}
        F(x,y) &= F(x_0, y_0) + F_x(x_0, y_0)(x - x_0)
        + F_y(x_0, y_0)(y - y_0) + o(\|(x - x_0, y - y_0)\|)
    \end{align*}
    In the linear case, where the little-o term is neglected,
    we have equality to zero when
    \[
        F_x(x_0, y_0)(x - x_0) + F_y(x_0, y_0)(y - y_0) = 0
    \]
    We can isolate \(y\) if \(F_y(x_0, y_0) \neq 0\),
    giving
    \[
        y = y_0 - \frac{F_x(x_0, y_0)}{F_y(x_0, y_0)}(x - x_0)
    \]
    We expect this also holds in the non-linear case, adding \(o(\|x - x_0\|)\).
    
    For instance, considering the gradient of \(F = x^2 + y^2 - 1\)
    which vanishes on the unit circle, we have \(\gradient F = (2x, 2y)\).
    Note that this vanishes when \(x = y = 0\),
    and \((0,0) \notin F^{-1}(0)\). At all other points, we can express a local \function.
    For example, at the point \((1, 0)\), we cannot express a local \function
    for \(y\), but we can express one for \(x\).
\end{snippet}

\begin{snippettheorem}{dini-theorem}{Implicit function theorem (Dini)}
    [\{
        "generalizations": ["general-implicit-function-theorem"]
    \}]
    Let \(\Omega \subseteq \realnumbers^2\) be open,
    \(F \colon \Omega \fromto \realnumbers\) \realcontinuous in \(\Omega\),
    and \((x_0, y_0) \in \Omega\) such that \(F(x_0, y_0) = 0\).
    Suppose that the partial derivative \(F_y\) exists in \(\Omega\)
    and is \realcontinuous at the point \((x_0, y_0)\).
    If \(F_y(x_0, y_0) \neq 0\), then there exist \(\delta, \sigma > 0\)
    such that:
    \begin{enumerate}
        \item \(F_y \neq 0\) in \(\overline{B}_\delta(x_0) \cartesianprod \overline{B}_\sigma(y_0)\);
        \item there exists a unique \(\varphi \colon \overline{B}_\delta(x_0) \fromto \overline{B}_\sigma(y_0)\)
        implicitly defined by the equation \(F(x,y) = 0\).
        Moreover, \(\varphi\) is \realcontinuous in \(\overline{B}_\delta(x_0)\).
    \end{enumerate}
\end{snippettheorem}

\begin{snippetproof}{dini-theorem-proof}{dini-theorem}{Implicit function theorem (Dini)}
    We know that \(F_y(x_0, y_0) \neq 0\)
    (assume without loss of generality that it is positive)
    and \(F_y\) is \realcontinuous at \((x_0, y_0)\).
    By the sign permanence theorem, there exists a \neighborhood where this derivative remains
    strictly positive.
    Thus, there exist \(\sigma, \tilde{\delta} > 0\) such that
    \[
        \overline{B}_{\tilde{\delta}}(x_0) \cartesianprod \overline{B}_\sigma(y_0) \subseteq \Omega
    \]
    and
    \[
        F_y(x,y) > 0, \quad \forall (x,y) \in \overline{B}_{\tilde{\delta}}(x_0) \cartesianprod \overline{B}_\sigma(y_0)
    \]
    which is a small rectangle.
    Consider the values the \function takes on the vertical line \(x = x_0\).
    The restriction \(F(x_0, y)\) vanishes at the given point,
    is non-negative above, and non-positive below, since it is monotone increasing.
    That is, since \(F_y > 0\), we have
    \(F(x_0, y_0 - \sigma) < 0\) and \(F(x_0, y_0 + \sigma) > 0\).
    Since \(F\) is \realcontinuous on all of \(\Omega\),
    there exists \(\delta > 0\) such that
    \[
        F(x, y_0 - \sigma) < 0, \quad \forall x \in \overline{B}_\delta(x_0)
    \]
    and
    \[
        F(x, y_0 + \sigma) > 0, \quad \forall x \in \overline{B}_\delta(x_0)
    \]
    Consider \(x \in \overline{B}_\delta(x_0)\)
    and the restriction \(F(x, -)\), i.e., the vertical lines.
    We have \(F(x, y_0 - \sigma) < 0\),
    \(F(x, y_0 + \sigma) > 0\), and \(F_y > 0\)
    in \(\overline{B}_\delta(x_0) \cartesianprod \overline{B}_\sigma(y_0)\).
    Since \(F\) is \realcontinuous and \(F(x, -)\) is monotone increasing, there exists a single point
    \(y\) where \(F(x, y) = 0\).
    Thus, for every \(x \in \overline{B}_\delta(x_0)\),
    there exists a unique \(y \in \overline{B}_\sigma(y_0)\)
    such that \(F(x, y) = 0\).
    We have thus defined the implicit \function.
    It remains to show continuity.
    If \(\{x_n\} \to x\) with \(x\) in the ball,
    we want \(\varphi(x_n) \to \varphi(x)\).
    Consider a generic \subsequence \(\varphi(x_{n_k})\).
    From this \subsequence, we want to extract a \subsequence converging to \(\varphi(x)\).
    Note that \(\varphi(x_{n_k}) \in \overline{B}_\sigma(y_0)\),
    which is compact.
    Since it is compact, we can extract a convergent \subsequence.
    There exists \(n_{k_j}\) such that \(\varphi(x_{n_{k_j}}) \to y\).
    We only know that \(y \in \overline{B}_\sigma(y_0)\).
    We want to show that \(y = \varphi(x)\).
    We know that for all \(j\),
    \[
        0 = F(x_{n_{k_j}}, \varphi(x_{n_{k_j}}))
    \]
    and taking \(j \to \infty\),
    \[
        F(x_{n_{k_j}}, \varphi(x_{n_{k_j}})) \to F(x, y) = 0
    \]
    by continuity of \(F\), this must be \(0\).
    By uniqueness of the implicit \function, we must have \(y = \varphi(x)\).
\end{snippetproof}

\begin{snippet}{dini-theorem-generalization-expl}
    For \(\Omega \subseteq \realnumbers^n \cartesianprod \realnumbers\)
    and \(F \colon \Omega \fromto \realnumbers\), the proof is the same
    since monotonicity is used only on the last variable
    and the \function has real values.

    If instead the \function has values in \(\realnumbers^m\),
    we can no longer use monotonicity. This generalized version
    is no longer called Dini's theorem because it uses a different argument.
\end{snippet}

\begin{snippetexample}{implicit-function-global-zeros-example}{}
    Let \(F(x,y) = x^2 y + e^{x+y}\).
    We want to describe globally the zero set.
    Clearly \(F \in \continuityclass^\infty(\realnumbers^2)\).
    The derivatives are \(F_x = 2xy + e^{x+y}\)
    and \(F_y = x^2 + e^{x+y}\).
    The second derivative never vanishes.
    In fact, \(F_y > 0\) always.
    For fixed \(x\), along vertical lines,
    the \function is monotone increasing.
    To check if there is a zero, we study the two limits
    \[
        \lim_{y \to -\infty} F(x, y), \quad
        \lim_{y \to +\infty} F(x, y)
    \]
    If the first limit is non-positive and the second is non-negative,
    then there exists \(y = \varphi(x)\) such that \(F(x, \varphi(x)) = 0\).
    Computing:
    \begin{align*}
        \lim_{y \to -\infty} F(x, y)
        &= \begin{cases}
            0 & x = 0 \\
            -\infty & x \neq 0
        \end{cases}
    \end{align*}
    while
    \[
        \lim_{y \to +\infty} F(x, y) = +\infty
    \]
    always.
    If \(x \neq 0\), there certainly exists a zero in between.
    If \(x = 0\), since the \function is increasing, there is no zero.
    This tells us that the domain of the implicit \function is
    \(\realnumbers \backslash \{0\}\).
    
    Now study the sign of \(\varphi\).
    Compute \(F(x, 0)\).
    If \(F(x, 0) > 0\), then \(\varphi(x) < 0\);
    if \(F(x, 0) < 0\), then \(\varphi(x) > 0\) by monotonicity.
    And if \(F(x, 0) = 0\), then \(\varphi(x) = 0\).
    Since \(F(x, 0) = e^x > 0\),
    our implicit \function is always negative.
    
    Compute the 4 limits.
    For the limit as \(x \to -\infty\):
    if it exists, it belongs to \([-\infty, 0]\).
    Cut with horizontal lines:
    \[
        \lim_{x \to -\infty} F(x, -c) = \lim_{x \to -\infty} e^{x-c} - x^2 c = -\infty
    \]
    for \(c > 0\). In particular, \(F(x, -c) < 0\) for \(x\) in a \neighborhood of \(-\infty\).
    Thus, \(-c < \varphi(x) < 0\).
    Since \(c\) is arbitrary, the limit is \(0^-\).
    
    Now compute the limit as \(x \to 0\).
    Consider \(c > 0\) and compute
    \[
        \lim_{x \to 0} F(x, -c) = \lim_{x \to 0} e^{x-c} - x^2 c = e^{-c} > 0
    \]
    Thus, the limit is \(-\infty\) from both sides.
    So there is a vertical asymptote.
    
    Now
    \[
        \lim_{x \to +\infty} F(x, -c) = \lim_{x \to +\infty} e^{x-c} - x^2 c = +\infty
    \]
    for every \(c > 0\).
    
    For the derivative, we have \(F(x, \varphi(x)) = 0\),
    and thus
    \begin{align*}
        0 &= \frac{d}{dx} F(x, y(x)) = F_x(x, y(x)) + F_y(x, y(x)) y'(x) \\
        y'(x) &= -\frac{F_x(x, y(x))}{F_y(x, y(x))}
    \end{align*}
    From a computational standpoint, it is convenient to consider directly the expression
    \(x^2 y(x) + e^{x + y(x)}\) and differentiate with respect to \(x\):
    \begin{align*}
        0 &= 2xy + x^2 y' + e^{x+y}(1 + y') = y'(x^2 + e^{x+y}) + 2xy + e^{x+y} \\
        y'(x) &= -\frac{2xy + e^{x+y}}{x^2 + e^{x+y}}
    \end{align*}
    We can use the fact that \(x^2 y + e^{x+y} = 0\) identically on the graph of \(\varphi\).
    Thus, \(e^{x+y} = -x^2 y\) and so
    \[
        y'(x) = -\frac{2xy - x^2 y}{x^2 + e^{x+y}} = -xy \frac{2-x}{x^2 + e^{x+y}}
    \]
    whose only critical point is \(x = 2\), while the other is not
    to be considered since the \function is not defined at zero.
    Verify that it is a maximum.
    The sign of \(y'\) is given by the sign of \(x(2-x)\),
    so the \function decreases on \((-\infty, 0)\),
    increases on \((0, 2]\), and then decreases again.
    
    Evaluate if there is an oblique asymptote on the right.
    We must evaluate the limit of a \function we do not have explicitly,
    but we can use L'Hôpital's rule to obtain a more manageable form:
    \begin{align*}
        \lim_{x \to \infty} \frac{\varphi(x)}{x}
        &\overset{H}{=} \lim_{x \to \infty} \varphi'(x)
        = \lim_{x \to \infty} -xy \frac{2-x}{x^2 + e^{x+y}} \\
        &= \lim_{x \to \infty} \frac{x-2}{x} \cdot \frac{y}{1 - y} = -1
    \end{align*}
    Thus, the slope, if the oblique asymptote exists, must be \(-1\).
    To find the offset of the asymptote, we must compute
    \[
        \lim_{x \to \infty} (\varphi(x) - mx)
    \]
    To compute it, consider
    \begin{align*}
        \lim_{x \to \infty} F(x, mx + q)
        &= \lim_{x \to \infty} F(x, q - x)
        = \lim_{x \to \infty} qx^2 + e^q - x^3 = -\infty
    \end{align*}
    This means that \(\varphi(x) < q - x\) and thus \(\varphi(x) + x > q\) for all \(q\).
    Thus, the limit of \(\varphi(x) + x = \infty\), and so \(\varphi\)
    does not actually have an oblique asymptote.
    The condition that the limit of \(F(x, mx+q) = 0\)
    is not sufficient for \(mx+q\) to be an oblique asymptote of \(\varphi(x)\).
\end{snippetexample}

\section{Generalized implicit function theorem}

\begin{snippet}{generalized-implicit-function-system-expl}
    Consider the system
    \[
        \begin{cases}
            y\cos(xz) - x^2 - 1 = 0 \\
            y\sin(xz) - x = 0
        \end{cases}
    \]
    We want to study the solution set of the system.
    At most, we can expect to find expressions of two variables
    as a \function of the third.
    We have a \function
    \[
        F(x, y, z) = \begin{pmatrix}
            y\cos(xz) - x^2 - 1 \\
            y\sin(xz) - x
        \end{pmatrix}
    \]
    and we want to study \(F^{-1}(\{(0, 0)\})\).
    Expanding the \function to first order:
    \[
        F(x, y) = F(x_0, y_0) + dF(x_0, y_0) \begin{pmatrix} x - x_0 \\ y - y_0 \end{pmatrix}
        + o(\|(x - x_0, y - y_0)\|)
    \]
    The first term is zero.
    Moreover, \(dF(x_0, y_0)\) is representable by \(\jacobian F(x_0, y_0)\),
    which we can write as \([F_x \mid F_y]\) in block form,
    where \(F_x\) has \(n\) columns and \(F_y\) has \(m\) (and obviously \(m\) rows).
    We have
    \[
        F(x, y) = F_x(x_0, y_0)(x - x_0) + F_y(x_0, y_0)(y - y_0)
        + o(\|(x - x_0, y - y_0)\|)
    \]
    We are interested in \(F(x, y) = 0\) (neglecting the little-o term).
    If \(F_y\) is invertible, we can isolate it:
    \[
        y = y_0 - F_y(x_0, y_0)^{-1} \circ F_x(x_0, y_0)(x - x_0)
    \]
    when \(F_y\) is invertible.
    This is clearly the linear case. We expect the same
    to hold in a non-linear context, but only locally.
    
    In the example, we have
    \[
        \jacobian F = \begin{pmatrix}
            -yz\sin(xz) - 2x & \cos(xz) & -xy\sin(xz) \\
            yz\cos(xz) - 1 & \sin(xz) & xy\cos(xz)
        \end{pmatrix}
    \]
    We want to determine if \(F(x, y, z) = 0\) allows expressing two variables in terms of the third
    in a \neighborhood of \((0, 1, 0)\). We have \(F(0, 1, 0) = (0, 0)\).
    The matrix at this point has rank 2:
    \[
        \jacobian F(0, 1, 0) = \begin{pmatrix}
            0 & 1 & 0 \\
            -1 & 0 & 0
        \end{pmatrix}
    \]
    The only submatrix achieving rank 2 is the leftmost \(2 \cartesianprod 2\) submatrix,
    so \(F = (0, 0)\) implicitly defines \(x = x(z)\) and \(y = y(z)\).
\end{snippet}

\begin{snippettheorem}{general-implicit-function-theorem}{Generalized Implicit function theorem}
    Let \(\Omega \subseteq \realnumbers^n \cartesianprod \realnumbers^m\) be open,
    \((x_0, y_0) \in \Omega\), \(F \colon \Omega \fromto \realnumbers^m\).
    Let \(F\) be \realcontinuous in \(\Omega\). Suppose there exists
    \(F_y \in \setoflineartransformations(\realnumbers^m, \realnumbers^m)\)
    in \(\Omega\), \realcontinuous at \((x_0, y_0)\).
    Let \(F(x_0, y_0) = 0\), \(F_y(x_0, y_0) \in \text{GL}(\realnumbers^m)\).
    Then there exist \(r, R > 0\) such that for all \(x \in \overline{B}_r(x_0)\),
    there exists a unique \(y \in \overline{B}_R(y_0)\), \(y = \varphi(x)\),
    such that \(F(x, y) = 0\).
    Moreover, \(\varphi\) is \realcontinuous in \(\overline{B}_r(x_0)\).
\end{snippettheorem}

\plain{This is a local statement since we need to find sufficiently small radii.}

\begin{snippetproof}{general-implicit-function-theorem-proof}{general-implicit-function-theorem}{Generalized Implicit function theorem}
    Introduce the \function \(T(x, y) = y - F_y(x_0, y_0)^{-1} F(x, y)\).
    Thus \(T \colon \Omega \fromto \realnumbers^m\).
    The definition makes sense since \(F_y(x_0, y_0)\) is invertible.
    Consider \(x\) as a parameter.
    We ask whether \(T\) has a fixed point with respect to \(y\).
    Suppose that \(y\) is a fixed point of \(T\).
    Thus \(T(x, y) = y\),
    which is equivalent to saying that \(F_y(x_0, y_0)^{-1} F(x, y) = 0\),
    hence necessarily \(F(x, y) = 0\) since we can multiply by the inverse, which exists.
    If we show that \(T\) has a unique fixed point with respect to \(y\),
    then we have that for all \(x\), there exists a unique \(y\) such that \(F(x, y) = 0\).
    We want to show that \(T\) maps a \metricspace into itself and that it is a \contraction
    (with factor \(\delta < 1\), with \(\delta\) uniform in \(x\)).
    We want \(T(x, -) \colon \overline{B}_R(y_0) \fromto \realnumbers^m\)
    to be a \contraction for all \(x \in \overline{B}_r(x_0)\).
    \begin{enumerate}
        \item We show it is a \contraction.
            Since \(\overline{B}_R(y_0)\) is convex, it suffices to estimate \(T_y\).
            Since \(T(x, y) = y - F_y(x_0, y_0)^{-1}F(x, y)\), we have
            \(T_y = I - F_y(x_0, y_0)^{-1}F_y(x, y)\).
            This is because the differential of the first term becomes the identity matrix,
            while for the second, the multiplying matrix exits the derivative by linearity.
            \begin{align*}
                \|T_y(x, y)\| &= \|I - F_y(x_0, y_0)^{-1} F_y(x, y)\| \\
                &= \|F_y(x_0, y_0)^{-1} (F_y(x_0, y_0) - F_y(x, y))\| \\
                &\leq \|F_y(x_0, y_0)^{-1}\| \cdot \|F_y(x_0, y_0) - F_y(x, y)\| \to 0
            \end{align*}
            as \((x, y) \to (x_0, y_0)\) because \(F_y\) is \realcontinuous at \((x_0, y_0)\).
            Thus, there exist \(r, R > 0\) such that for all \(x \in \overline{B}_r(x_0)\)
            and for all \(y \in \overline{B}_R(y_0)\),
            \[
                \|T_y(x, y)\| \leq \delta < 1
            \]
            with \(\delta\) arbitrary.
            In particular,
            \[
                \sup_{\substack{x \in \overline{B}_r(x_0) \\ y \in \overline{B}_R(y_0)}} \|T_y(x, y)\| \leq \delta
            \]
            and \(\delta\) is uniform in \(x\).
        \item We show that there exist \(r, R > 0\)
            such that for all \(x \in \overline{B}_r(x_0)\)
            and for all \(y \in \overline{B}_R(y_0)\),
            \(T(x, -) \colon \overline{B}_R(y_0) \fromto \overline{B}_R(y_0)\).
            We want to show that \(\|T(x, y) - y_0\| \leq R\).
            Choose \(R\) as in the previous point.
            We want to show that there exists \(r > 0\) such that the condition is satisfied.
            Let \(y \in \overline{B}_R(y_0)\). Split into two pieces:
            \[
                \|T(x, y) - y_0\| \leq \|T(x, y) - T(x, y_0)\| + \|T(x, y_0) - y_0\|
            \]
            The first term can be estimated using the mean value theorem in \(y\):
            \begin{align*}
                \|T(x, y) - T(x, y_0)\| &\leq \|y - y_0\| \cdot \sup_{0 < t < 1} \|T_y(x, y_0 + t(y - y_0))\| \\
                &\leq R \cdot \delta
            \end{align*}
            We want the second to be less than or equal to \((1 - \delta)R\).
            We can no longer modify \(R, \delta\), only \(r\):
            \begin{align*}
                \|T(x, y_0) - y_0\| &= \|F_y(x_0, y_0)^{-1} F(x, y_0)\| \\
                &\leq \|F_y(x_0, y_0)^{-1}\| \cdot \|F(x, y_0)\| \\
                &= \|F_y(x_0, y_0)^{-1}\| \cdot \|F(x, y_0) - F(x_0, y_0)\| \to 0
            \end{align*}
            which tends to zero as \((x, y) \to (x_0, y_0)\) because \(F\) is \realcontinuous.
            We thus have radii such that
            \[
                \|F(x, y_0) - F(x_0, y_0)\| \leq \frac{(1 - \delta)R}{\|F_y(x_0, y_0)^{-1}\|}
            \]
    \end{enumerate}
    We can thus apply the \contraction mapping theorem (variant with parameter);
    moreover, the fixed point depends continuously on \(x\),
    and thus the implicit \function is \realcontinuous.
\end{snippetproof}

\begin{snippet}{generalized-implicit-function-closed-balls-expl}
    Note that closed balls were used because we need a \mscomplete \metricspace.
    Let us see if we can guarantee existence also on open balls.
    In fact, we only needed the closure of the ball in \(y\):
    there exist \(r_0, R_0 > 0\) such that for all \(x \in B_{r_0}(x_0)\),
    there exists a unique \(y \in B_{R_0}(y_0)\) such that \(F(x, y) = 0\).
    We can choose \(R_0 = R\) provided we shrink \(r_0\) appropriately.
    There exists \(r_0 > 0\) such that \(\varphi(B_{r_0}(x_0)) \subseteq B_R(y_0)\)
    by continuity of \(\varphi\) at \(x_0\), and thus
    we have shown existence and uniqueness of the implicit \function
    also with open balls replacing closed balls.
\end{snippet}

\begin{snippet}{implicit-function-uniqueness-expl}
    Suppose we have \(F \colon \Omega \fromto \realnumbers^m\) and \((x_0, y_0) \in \Omega\).
    Is it possible that two distinct maps \(\varphi, \psi\) exist, both implicitly defined
    by \(F = 0\) and both passing through \((x_0, y_0)\)?
    This can happen, for example, for \(F = x^2 - y^2\), but in this case
    the gradient of \(F\) at \((0, 0)\) is the zero vector, so we cannot apply
    the \snippetref[general-implicit-function-theorem][implicit function theorem].
    However, even at \((1, 1)\), more than one implicit \function can be defined, such as
    \[
        \varphi(x) = x, \quad \psi(x) = \begin{cases}
            -x & x \neq 1 \\
            1 & x = 1
        \end{cases}
    \]
    The problem is that they are not both \realcontinuous.
\end{snippet}

\begin{snippetcorollary}{implicit-function-uniqueness-corollary}{}
    Under the hypotheses of the \snippetref[general-implicit-function-theorem][implicit function theorem], let
    \(\varphi, \psi\) be two \function[functions] implicitly defined by \(F(x, y) = 0\)
    in a \neighborhood of \(x_0\) such that \(\varphi(x_0) = y_0 = \psi(x_0)\).
    If \(\varphi, \psi\) are \realcontinuous, then \(\varphi \equiv \psi\) in \(U(x_0)\).
\end{snippetcorollary}

\begin{snippetproof}{implicit-function-uniqueness-corollary-proof}{implicit-function-uniqueness-corollary}{}
    We exploit the continuity of the two implicit maps.
    Take \(r, R\) as in the \snippetref[general-implicit-function-theorem][implicit function theorem].
    Find \(r_0 < r\) such that
    \[
        \varphi(\overline{B}_{r_0}(x_0)) \union \psi(\overline{B}_{r_0}(x_0)) \subseteq \overline{B}_R(y_0)
    \]
    We can find such a radius since we can find one for each map,
    as they are \realcontinuous, and then take the minimum.
    By the \snippetref[general-implicit-function-theorem][implicit function theorem], the zeros of the map \(F(x, y)\) corresponding to the same \(x\)
    and contained in \(\overline{B}_r(x_0) \cartesianprod \overline{B}_R(y_0)\) are unique.
    Thus \(F(x, \varphi(x)) = 0 = F(x, \psi(x))\), but \((x, \varphi(x))\)
    and \((x, \psi(x))\) are in \(\overline{B}_r(x_0) \cartesianprod \overline{B}_R(y_0)\),
    so uniqueness holds for \(x \in \overline{B}_{r_0}(x_0)\).
\end{snippetproof}

\section{Differentiability of the implicit function}

\begin{snippet}{implicit-function-differentiability-expl}
    Let us study the differentiability of the implicit \function.
    For example, for \(F = y - |x|\),
    the hypotheses of the \snippetref[general-implicit-function-theorem][implicit function theorem] hold, but the \function
    \(y = |x|\) is not \differentiable.
    We need to add hypotheses on \(x\) as well.
    Differentiating the condition, we obtain
    \begin{align*}
        0 &= F(x, \varphi(x)) \\
        0 &= F_x(x, \varphi(x)) + F_y(x, \varphi(x)) \circ \text{d}\varphi(x)
    \end{align*}
    Since \(F_y \in \text{GL}(\realnumbers^m)\) and is \realcontinuous at \((x_0, y_0)\),
    and \(\text{GL}(\realnumbers^m)\) is open,
    \(F_y(x, \varphi(x)) \in \text{GL}(\realnumbers^m)\)
    in a \neighborhood of \(x_0\).
    We deduce that
    \[
        \text{d}\varphi(x) = -F_y(x, \varphi(x))^{-1} \circ F_x(x, \varphi(x))
    \]
\end{snippet}

\begin{snippettheorem}{implicit-function-differentiability-theorem}{Differentiability of the implicit function}
    Let \((x_0, y_0) \in \Omega \subseteq \realnumbers^n \cartesianprod \realnumbers^m\) be open,
    let \(F \colon \Omega \fromto \realnumbers^m\), \(F(x_0, y_0) = 0\).
    Let \(\varphi \colon U(x_0) \fromto \realnumbers^m\)
    such that \(\varphi(x_0) = y_0\), \(F(x, \varphi(x)) = 0\)
    for all \(x \in U(x_0)\). Let \(F\) be \differentiable at \((x_0, y_0)\),
    \(F_y(x_0, y_0) \in \text{GL}(\realnumbers^m)\).
    Then \(\varphi\) is \differentiable at \(x_0\) and
    \[
        \text{d}\varphi(x) = -F_y(x, \varphi(x))^{-1} \circ F_x(x, \varphi(x))
    \]
\end{snippettheorem}

\plain{From this formula, it follows that the implicit function has the same regularity as F.}

\begin{snippetproof}{implicit-function-differentiability-theorem-proof}{implicit-function-differentiability-theorem}{Differentiability of the implicit function}
    We must show that
    \[
        \varphi(x_0 + h) - \varphi(x_0) + F_y(x_0, y_0)^{-1} F_x(x_0, y_0)h = o(\|h\|)
    \]
    as \(h \to 0\).
    We can substitute \(\varphi(x_0) = y_0\).
    Set
    \begin{align*}
        \Delta \varphi(x_0, h) &= \varphi(x_0 + h) - \varphi(x_0) + F_y(x_0, y_0)^{-1} F_x(x_0, y_0)h \\
        &= F_y(x_0, y_0)^{-1} \left[
            F_y(x_0, y_0)(\varphi(x_0 + h) - y_0) + F_x(x_0, y_0)h
        \right]
    \end{align*}
    but the bracket equals the differential applied to something:
    \[
        F_y(x_0, y_0)(\varphi(x_0 + h) - y_0) + F_x(x_0, y_0)h
        = \text{d}F(x_0, y_0)(h, \varphi(x_0 + h) - y_0)
    \]
    Now we know that \(F\) is \differentiable at \((x_0, y_0)\),
    so \(F(x_0 + v, y_0 + w) - F(x_0, y_0) - \text{d}F(x_0, y_0)(v, w) = o(\|v - w\|)\).
    Now, in the bracket add \(F(x_0 + h, \varphi(x_0 + h)) - F(x_0, y_0) = 0 - 0\).
    Thus we get
    \begin{align*}
        &F_y(x_0, y_0)(\varphi(x_0 + h) - y_0) + F_x(x_0, y_0)h \\
        &= F(x_0 + h, \varphi(x_0 + h)) - F(x_0, y_0) - \text{d}F(x_0, y_0)(h, \varphi(x_0 + h) - y_0) \\
        &= o(\|(h, \varphi(x_0 + h) - y_0)\|)
    \end{align*}
    Since \(F_y(x_0, y_0)^{-1}\) is a fixed linear operator,
    \[
        \|\Delta \varphi(x_0, h)\| = o(\|(h, \varphi(x_0 + h) - y_0)\|)
    \]
    as \((h, \varphi(x_0 + h) - y_0) \to (0, 0)\).
    But by continuity of \(\varphi\), if \(h \to 0\), then \((h, \varphi(x_0 + h) - y_0) \to (0, 0)\), so
    \[
        \|\Delta \varphi(x_0, h)\| = o(\|(h, \varphi(x_0 + h) - y_0)\|)
    \]
    as \(h \to 0\).
    We have shown that for all \(\varepsilon > 0\), there exists \(\delta > 0\) such that
    \[
        \|h\| \leq \delta \implies \|\Delta \varphi(x_0, h)\| \leq \varepsilon \|(h, \varphi(x_0 + h) - y_0)\|
    \]
    Consider the vector \((h, \varphi(x_0 + h) - y_0)\). Add and subtract \(F_y(x_0, y_0)^{-1}F_x(x_0, y_0)h\)
    from the second component. We get
    \begin{align*}
        (h, \varphi(x_0 + h) - y_0) &= (h, \Delta \varphi(x_0, h) - F_y(x_0, y_0)^{-1}F_x(x_0, y_0)h) \\
        &= (h, \Delta \varphi(x_0, h) - A_0 h) = (h, -A_0 h) + (0, \Delta \varphi(x_0, h))
    \end{align*}
    We can use the \norm of this in the triangle inequality:
    \begin{align*}
        \|(h, \varphi(x_0 + h) - y_0)\| &\leq \|(h, -A_0 h)\| + \|(0, \Delta \varphi(x_0, h))\| \\
        &\leq C\|h\| + \|\Delta \varphi(x_0, h)\|
    \end{align*}
    Thus we deduce that
    \begin{align*}
        \|\Delta \varphi(x_0, h)\| &\leq C\varepsilon\|h\| + \varepsilon\|\Delta \varphi(x_0, h)\| \\
        &\leq \frac{C\varepsilon}{1 - \varepsilon}\|h\|
    \end{align*}
    so we have shown that for \(\varepsilon > 0\), there exists \(\delta > 0\)
    such that if \(\|h\| < \delta\), then
    \[
        \|\Delta \varphi(x_0, h)\| \leq \frac{C\varepsilon}{1 - \varepsilon}\|h\|
    \]
    thus \(\Delta \varphi(x_0, h) = o(\|h\|)\) as \(h \to 0\).
\end{snippetproof}

\end{document}
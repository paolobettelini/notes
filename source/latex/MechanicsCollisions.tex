\documentclass[preview]{standalone}

\usepackage{amsmath}
\usepackage{amssymb}
\usepackage{stellar}
\usepackage{definitions}
\usepackage{bettelini}

\begin{document}

\id{mechanics-collisions}
\genpage

\section{Collisions}

\begin{snippet}{collisioni-temp}
    Consideriamo un urto e quindi
    \[
        \frac{d\vec{Q}}{dt} = m_1\vec{a}_1 + m_2\vec{a}_2 = \vec{f}_1 + \vec{f}_2
    \]
    Integrando troviamo
    \[
        \integral[t_0][t_1][\frac{d\vec{Q}}{dt}][t] = \vec{Q}(t_1) - \vec{Q}(t_0) =
        \integral[t_0][t_1][\vec{f}_1 + \vec{f}_2][t] \approx 0
    \]
    che si conserva (approssimativamente) se consideriamo il tempo dell'urto come istantaneo.

    Ipotizziamo che esista \(U(r_1, r_2)\) tale che
    \[
        \begin{cases}
            \vec{F}_{1,2} = -\nabla_1 U \\
            \vec{F}_{2,1} = -\nabla_2 U
        \end{cases}
    \]
    Quindi \(E = E_{c1} + E_{c2} + U\) la cui derivata è
    \begin{align*}
        \frac{dE}{dt} &= \vec{v}_1 \left[\vec{F}_{1,2} + \vec{f}_1\right]
        + \vec{v}_2 \left[\vec{F}_{2,1} + \vec{f}_2\right]
        + \nabla_1 U\cdot \vec{v}_1 + + \nabla_2 U\cdot \vec{v}_2 \\
        &= \vec{v}_1 \cdot \vec{f}_1 + \vec{v}_2 \cdot \vec{f}_2
    \end{align*}
    Integriamo tale valore
    \[
        \integral[t_0][t_1][\frac{dE}{dt}][t] = E(t_0) - E(t_1) =
        \integral[t_0][t_1][\vec{v}_1 \cdot \vec{f}_1 + \vec{v}_2 \cdot \vec{f}_2][t] \approx 0
    \]
    Quindi \(E(t_1) = E(t_0)\), cioè un momento prima dell'urto e un momento dopo.
    Possiamo togliere le componenti potenziali in quanto le posizioni sono praticamente le medesime
    \begin{align*}
        E_{c1}(t_1) + E_{c2}(t_1) + U(t_1) &= E_{c2}(t_2) + E_{c2}(t_2) + U(t_2) \\
        E_{c1}(t_1) + E_{c2}(t_1) &= E_{c2}(t_2) + E_{c2}(t_2)
    \end{align*}
    Tale è chiamato \emph{urto elastico}.

    %% One-dimensional Newtonian https://en.wikipedia.org/wiki/Elastic_collision
    % 19 dic.
    % infine c'è il caso particolare m1 = m2, e tutta l'energia cinetica viene trasferita.
\end{snippet}

\end{document}
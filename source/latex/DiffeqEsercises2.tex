\documentclass[preview]{standalone}

\usepackage{amsmath}
\usepackage{amssymb}
\usepackage{stellar}
\usepackage{definitions}
\usepackage{bettelini}

\begin{document}

\id{diffeq-exercises-2}
\genpage

\section{Exercises}

\begin{snippetexercise}{diffeq-ex-1}{}
    Solve the following Cauchy problem:
    \[
        \begin{cases}
        y' = y^3 \sin(2x) \\
        y\left(\frac{\pi}{4}\right) = \sqrt{2}
        \end{cases}
    \]
\end{snippetexercise}

\begin{snippetsolution}{diffeq-ex-1-sol}{}
    The given differential equation is of separable variables type. We can rewrite it as:
    \[ \frac{dy}{dx} = y^3 \sin(2x). \]
    Separating the variables (for \( y \neq 0 \)):
    \[ y^{-3} \, dy = \sin(2x) \, dx. \]
    Integrating both sides:
    \[ \int y^{-3} \, dy = \int \sin(2x) \, dx. \]
    \[ \frac{y^{-2}}{-2} = -\frac{1}{2} \cos(2x) + C. \]
    Multiplying the equation by \(-2\):
    \[ \frac{1}{y^2} = \cos(2x) + K, \quad \text{where } K = -2C. \]
    Now we apply the initial condition \( y(\pi/4) = \sqrt{2} \) to determine the constant \( K \):
    \[ \frac{1}{(\sqrt{2})^2} = \cos\left(2 \cdot \frac{\pi}{4}\right) + K \]
    \[ \frac{1}{2} = \cos\left(\frac{\pi}{2}\right) + K \]
    \[ \frac{1}{2} = 0 + K \implies K = \frac{1}{2}. \]
    Substitute \( K = 1/2 \) back into the general solution:
    \[ \frac{1}{y^2} = \cos(2x) + \frac{1}{2} = \frac{2\cos(2x) + 1}{2}. \]
    Inverting both sides:
    \[ y^2 = \frac{2}{2\cos(2x) + 1}. \]
    Taking the square root:
    \[ y(x) = \pm \sqrt{\frac{2}{2\cos(2x) + 1}}. \]
    Since the initial value \( y(\pi/4) = \sqrt{2} \) is positive, we select the positive branch:
    \[ y(x) = \sqrt{\frac{2}{2\cos(2x) + 1}}. \]
    The solution is valid where the denominator is positive, i.e., \( \cos(2x) > -1/2 \). In the interval containing \( \pi/4 \), the domain is \( x \in (-\pi/3, \pi/3) \).
\end{snippetsolution}

\begin{snippetexercise}{diffeq-ex-2}{}
    Solve the following Cauchy problem:
    \[
        \begin{cases}
        y' + y \tan x = 3 \sin 2x \\
        y(0) = 1.
        \end{cases}
    \]
\end{snippetexercise}

\begin{snippetsolution}{diffeq-ex-2-sol}{}
    This is a linear first-order ordinary differential equation of the form \(y' + p(x)y = q(x)\), with \(p(x) = \tan x\) and \(q(x) = 3 \sin 2x\).

    \begin{enumerate}
        \item \emph{Calculate the integrating factor}
        The integrating factor \(I(x)\) is defined as:
        \[ I(x) = e^{\int p(x) \, dx} = e^{\int \tan x \, dx}. \]
        Since \(\int \tan x \, dx = \int \frac{\sin x}{\cos x} \, dx = -\ln|\cos x|\), we have:
        \[ I(x) = e^{-\ln|\cos x|} = \frac{1}{|\cos x|}. \]
        Given the initial condition at \(x=0\) (where \(\cos x > 0\)), we can simplify this to \(I(x) = \frac{1}{\cos x}\) in the interval \((-\pi/2, \pi/2)\). 
        \item \emph{Multiply and integrate}
        Multiply the entire differential equation by \(\frac{1}{\cos x}\):
        \[ \frac{1}{\cos x}y' + \frac{\sin x}{\cos^2 x}y = \frac{3 \sin 2x}{\cos x}. \]
        The left side becomes the derivative of the product \(y \cdot I(x)\). The right side simplifies using the double-angle identity \(\sin 2x = 2 \sin x \cos x\):
        \[ \frac{d}{dx}\left( \frac{y}{\cos x} \right) = \frac{3(2 \sin x \cos x)}{\cos x} = 6 \sin x. \]
        Now, integrate both sides with respect to \(x\):
        \[ \frac{y}{\cos x} = \int 6 \sin x \, dx = -6 \cos x + C. \]
        \item \emph{General Solution}
        Multiply by \(\cos x\) to isolate \(y\):
        \[ y(x) = \cos x (C - 6 \cos x) = C \cos x - 6 \cos^2 x. \]
        \item \emph{Apply Initial Condition}
        Use \(y(0) = 1\) to find the constant \(C\):
        \[ 1 = C \cos(0) - 6 \cos^2(0) \]
        \[ 1 = C(1) - 6(1)^2 \implies 1 = C - 6 \implies C = 7. \]
        \item \emph{Final Solution}
        Substituting \(C=7\) back into the general solution:
        \[ y(x) = 7 \cos x - 6 \cos^2 x. \]
    \end{enumerate}
\end{snippetsolution}

\begin{snippetexercise}{diffeq-ex-3}{}
    Determine the general integral of the differential equation
    \[ xy' = y + xe^{\frac{y}{x}}. \]
\end{snippetexercise}

\begin{snippetsolution}{diffeq-ex-3-sol}{}
    This is a first-order homogeneous differential equation. We can rewrite it by dividing by \(x\) (assuming \(x \neq 0\)):
    \[ y' = \frac{y}{x} + e^{\frac{y}{x}}. \]
    \begin{enumerate}
        \item \emph{Substitution}
        We use the substitution \( u = \frac{y}{x} \), which implies \( y = xu \).
        Differentiating with respect to \( x \), we get:
        \[ y' = u + xu'. \]
        \item \emph{Separation of variables}
        Substituting the expressions for \( y' \) and \( \frac{y}{x} \) into the differential equation:
        \[ u + xu' = u + e^u. \]
        Simplifying by subtracting \( u \) from both sides:
        \[ xu' = e^u. \]
        Now we separate the variables \( x \) and \( u \):
        \[ \frac{du}{dx} = \frac{e^u}{x} \implies e^{-u} \, du = \frac{1}{x} \, dx. \]
        \item \emph{Integration}
        Integrating both sides:
        \[ \int e^{-u} \, du = \int \frac{1}{x} \, dx. \]
        \[ -e^{-u} = \ln|x| + C, \]
        where \( C \) is an arbitrary integration constant.
        \item \emph{Solving for \(y\)}
        Rearranging the equation to solve for \( u \):
        \[ e^{-u} = -\ln|x| - C. \]
        Let \( K = -C \). For the solution to be defined, we need \( K - \ln|x| > 0 \).
        \[ -u = \ln(K - \ln|x|) \implies u = -\ln(K - \ln|x|). \]
        Finally, substitute back \( u = \frac{y}{x} \):
        \[ \frac{y}{x} = -\ln(K - \ln|x|). \]
        \[ y(x) = -x \ln(K - \ln|x|). \]
    \end{enumerate}
\end{snippetsolution}

\begin{snippetexercise}{diffeq-ex-4}{}
    Determine the general integral of the differential equation
    \[ y' = -\frac{1}{x^2} - \frac{y}{x} + y^2. \]
\end{snippetexercise}

\begin{snippetsolution}{diffeq-ex-4-sol}{}
    This is a Riccati differential equation of the form \( y' = P(x) + Q(x)y + R(x)y^2 \), where \( P(x) = -\frac{1}{x^2} \), \( Q(x) = -\frac{1}{x} \), and \( R(x) = 1 \).
    \begin{enumerate}
        \item \emph{Finding a particular solution}
        We guess a particular solution of the form \( y_p(x) = \frac{k}{x} \).
        Differentiating gives \( y_p' = -\frac{k}{x^2} \). Substituting this into the equation:
        \[ -\frac{k}{x^2} = -\frac{1}{x^2} - \frac{1}{x}\left(\frac{k}{x}\right) + \left(\frac{k}{x}\right)^2 \]
        \[ -\frac{k}{x^2} = \frac{-1 - k + k^2}{x^2}. \]
        Multiplying by \( x^2 \) and rearranging:
        \[ -k = -1 - k + k^2 \implies k^2 = 1 \implies k = \pm 1. \]
        We choose the particular solution \( y_p(x) = \frac{1}{x} \).
        \item \emph{Substitution}
        We use the standard substitution for Riccati equations: \( y(x) = y_p(x) + \frac{1}{u(x)} = \frac{1}{x} + \frac{1}{u} \).
        Differentiating with respect to \( x \):
        \[ y' = -\frac{1}{x^2} - \frac{u'}{u^2}. \]
        Substitute \( y \) and \( y' \) back into the original ODE:
        \[ -\frac{1}{x^2} - \frac{u'}{u^2} = -\frac{1}{x^2} - \frac{1}{x}\left(\frac{1}{x} + \frac{1}{u}\right) + \left(\frac{1}{x} + \frac{1}{u}\right)^2. \]
        Simplifying the right hand side:
        \[ -\frac{1}{x^2} - \frac{u'}{u^2} = -\frac{1}{x^2} - \frac{1}{x^2} - \frac{1}{xu} + \frac{1}{x^2} + \frac{2}{xu} + \frac{1}{u^2} \]
        \[ -\frac{u'}{u^2} = \frac{1}{xu} + \frac{1}{u^2}. \]
        Multiplying by \( -u^2 \), we obtain a linear first-order differential equation for \( u \):
        \[ u' = -\frac{1}{x}u - 1 \implies u' + \frac{1}{x}u = -1. \]
        \item \emph{Solving for \( u(x) \)}
        The integrating factor is \( I(x) = e^{\int \frac{1}{x} dx} = e^{\ln|x|} = |x| \). Assuming \( x > 0 \) for simplicity, \( I(x) = x \).
        Multiplying the linear ODE by \( x \):
        \[ xu' + u = -x \implies (xu)' = -x. \]
        Integrating both sides:
        \[ xu = \int -x \, dx = -\frac{x^2}{2} + C. \]
        \[ u(x) = \frac{2C - x^2}{2x}. \]
        \item \emph{General Integral}
        Substitute \( u(x) \) back into the expression for \( y(x) \):
        \[ y(x) = \frac{1}{x} + \frac{1}{u(x)} = \frac{1}{x} + \frac{2x}{2C - x^2}. \]
        Finding a common denominator:
        \[ y(x) = \frac{2C - x^2 + 2x^2}{x(2C - x^2)} = \frac{2C + x^2}{x(2C - x^2)}. \]
        Letting \( K = 2C \), the general solution is:
        \[ y(x) = \frac{K + x^2}{x(K - x^2)}. \]
    \end{enumerate}
\end{snippetsolution}

\begin{snippetexercise}{diffeq-ex-5}{}
    Determine the general integral of the differential equation
    \[ y' + \frac{1}{3}y = e^x y^4. \]
\end{snippetexercise}

\begin{snippetsolution}{diffeq-ex-5-sol}{}
    This is a Bernoulli differential equation of the form \( y' + a(x)y = b(x)y^\alpha \) with \( \alpha = 4 \). To solve it, we perform a substitution to reduce it to a linear equation.
    \begin{enumerate}
        \item \emph{Substitution}
        We introduce the variable \( z = y^{1-\alpha} = y^{1-4} = y^{-3} \).
        Differentiating \( z \) with respect to \( x \):
        \[ z' = \frac{d}{dx}(y^{-3}) = -3y^{-4}y'. \]
        From this, we can write \( y^{-4}y' = -\frac{1}{3}z' \).
        \item \emph{Transformation}
        Divide the original differential equation by \( y^4 \) (assuming \( y \neq 0 \)):
        \[ \frac{y'}{y^4} + \frac{1}{3}\frac{y}{y^4} = e^x \implies y^{-4}y' + \frac{1}{3}y^{-3} = e^x. \]
        Substitute the expressions in terms of \( z \):
        \[ -\frac{1}{3}z' + \frac{1}{3}z = e^x. \]
        Multiply the entire equation by \( -3 \) to put it in standard linear form:
        \[ z' - z = -3e^x. \]
        \item \emph{Solving the linear equation}
        This is a first-order linear ODE. The integrating factor is \( e^{\int -1 \, dx} = e^{-x} \).
        Multiply both sides by \( e^{-x} \):
        \[ e^{-x}z' - e^{-x}z = -3e^x e^{-x} \implies \frac{d}{dx}(z e^{-x}) = -3. \]
        Integrate both sides with respect to \( x \):
        \[ z e^{-x} = \int -3 \, dx = -3x + C. \]
        Solve for \( z \):
        \[ z(x) = e^x(C - 3x). \]
        \item \emph{General Integral}
        Substitute back \( z = y^{-3} \):
        \[ \frac{1}{y^3} = e^x(C - 3x). \]
        Solving for \( y \):
        \[ y(x) = \frac{1}{\sqrt[3]{e^x(C - 3x)}}. \]
        Note: \( y(x) = 0 \) is also a solution to the original equation (a singular solution lost during the division by \( y^4 \)).
    \end{enumerate}
\end{snippetsolution}

\begin{snippetexercise}{diffeq-ex-6}{}
    \todo
\end{snippetexercise}

\begin{snippetsolution}{diffeq-ex-6-sol}{}
    \todo
\end{snippetsolution}

\begin{snippetexercise}{diffeq-ex-7}{}
    Solve the following Cauchy problem, with \( a \in \realnumbers \):
    \[
        \begin{cases}
        xy' = y(1 + \ln y - \ln x) \\
        y(1) = e^a.
        \end{cases}
    \]
\end{snippetexercise}

\begin{snippetsolution}{diffeq-ex-7-sol}{}
    We can simplify the term inside the parenthesis using the property of logarithms \( \ln y - \ln x = \ln(y/x) \). The differential equation becomes:
    \[ xy' = y\left(1 + \ln\left(\frac{y}{x}\right)\right) \implies y' = \frac{y}{x} \left( 1 + \ln\left(\frac{y}{x}\right) \right). \]
    This is a first-order homogeneous differential equation, as the right-hand side depends only on the ratio \( y/x \).
    \begin{enumerate}
        \item \emph{Substitution}
        Let \( u = \frac{y}{x} \). Then \( y = xu \).
        Differentiating \( y \) with respect to \( x \) using the product rule:
        \[ y' = u + xu'. \]
        \item \emph{Separation of variables}
        Substitute \( y' \) and \( y/x \) back into the differential equation:
        \[ u + xu' = u(1 + \ln u). \]
        Expand the right-hand side:
        \[ u + xu' = u + u \ln u. \]
        Subtract \( u \) from both sides:
        \[ xu' = u \ln u. \]
        Now, separate the variables \( u \) and \( x \) (assuming \( u > 0 \) and \( \ln u \neq 0 \)):
        \[ \frac{du}{u \ln u} = \frac{dx}{x}. \]
        \item \emph{Integration}
        Integrate both sides:
        \[ \int \frac{du}{u \ln u} = \int \frac{dx}{x}. \]
        To integrate the left side, substitute \( v = \ln u \), so \( dv = \frac{1}{u} \, du \):
        \[ \int \frac{dv}{v} = \ln|v| = \ln|\ln u|. \]
        The integral on the right is \( \ln|x| \). Thus:
        \[ \ln|\ln u| = \ln|x| + C_1. \]
        Exponentiating both sides:
        \[ |\ln u| = e^{\ln|x| + C_1} = |x| e^{C_1}. \]
        Removing the absolute values introduces an arbitrary constant \( K \) (where \( K = \pm e^{C_1} \) or \( K=0 \) for the trivial solution \( u=1 \)):
        \[ \ln u = Kx. \]
        Solving for \( u \):
        \[ u = e^{Kx}. \]
        \item \emph{Back-substitution}
        Recall that \( u = y/x \):
        \[ \frac{y}{x} = e^{Kx} \implies y(x) = x e^{Kx}. \]
        \item \emph{Apply Initial Condition}
        We use the condition \( y(1) = e^a \):
        \[ e^a = 1 \cdot e^{K(1)} \implies e^a = e^K. \]
        Comparing the exponents, we find:
        \[ K = a. \]
        \item \emph{Final Solution}
        Substitute \( K = a \) into the general solution:
        \[ y(x) = x e^{ax}. \]
    \end{enumerate}
\end{snippetsolution}

\begin{snippetexercise}{diffeq-ex-8}{Peano's Brush}
    Prove that the Cauchy problem
    \[
        \begin{cases}
            y' = 3y^{2/3} \\
            y(0) = 0
        \end{cases}
    \]
    has infinite solutions in every neighborhood of \( x = 0 \).
\end{snippetexercise}

\begin{snippetsolution}{diffeq-ex-8-sol}{}
    The function \( f(x, y) = 3y^{2/3} \) is continuous on \( \realnumbers^2 \), which guarantees the existence of at least one solution by Peano's Existence Theorem. However, the partial derivative \( \frac{\partial f}{\partial y} = 2y^{-1/3} \) is unbounded as \( y \to 0 \). Therefore, the Lipschitz condition is not satisfied in any neighborhood of the origin, and the uniqueness of the solution is not guaranteed.
    To prove the existence of infinite solutions, we construct them explicitly.
    \begin{enumerate}
        \item \emph{Trivial Solution}
        The function \( y(x) = 0 \) for all \( x \in \realnumbers \) is a solution because \( y'(x) = 0 \) and \( 3(0)^{2/3} = 0 \), satisfying the equation and the initial condition \( y(0) = 0 \).
        \item \emph{Non-trivial Solutions}
        Assuming \( y \neq 0 \), we separate the variables:
        \[ \frac{dy}{y^{2/3}} = 3 \, dx \implies y^{-2/3} \, dy = 3 \, dx. \]
        Integrating both sides:
        \[ \int y^{-2/3} \, dy = \int 3 \, dx \implies 3y^{1/3} = 3x + C. \]
        Dividing by 3 and solving for \( y \):
        \[ y^{1/3} = x + k \implies y(x) = (x + k)^3, \]
        where \( k = C/3 \).
        To satisfy \( y(0) = 0 \), we must have \( k = 0 \), which gives the solution \( y(x) = x^3 \).
        \item \emph{Construction of the Solution Family}
        We can construct piecewise functions that stay on the trivial solution \( y=0 \) for a while and then branch off onto the cubic solution. For any parameter \( c \ge 0 \), define:
        \[
            y_c(x) = \begin{cases}
            0 & \text{if } x \le c \\
            (x - c)^3 & \text{if } x > c.
            \end{cases}
        \]
        Let us verify that \( y_c(x) \) is a solution:
        \begin{itemize}
            \item \emph{Initial Condition:} Since \( c \ge 0 \), \( 0 \le c \), so \( y_c(0) = 0 \).
            \item \emph{Differential Equation:}
            For \( x < c \), both sides of the ODE are 0.
            For \( x > c \), \( y_c'(x) = 3(x-c)^2 \) and \( 3(y_c(x))^{2/3} = 3((x-c)^3)^{2/3} = 3(x-c)^2 \), so the equation holds.
            At the junction point \( x = c \), we check differentiability:
            \[ y_c'(c^-) = 0, \quad y_c'(c^+) = \lim_{h \to 0^+} \frac{(h)^3 - 0}{h} = 0. \]
            The derivative exists and is 0, matching the right-hand side of the ODE.
        \end{itemize}
        Since \( c \) can be chosen arbitrarily from the interval \( [0, \infty) \), there are infinitely many distinct solutions to the Cauchy problem defined on \( \realnumbers \).
    \end{enumerate}
\end{snippetsolution}

\begin{snippetexercise}{diffeq-ex-9}{}
    Solve the following Cauchy problem:
    \[
        \begin{cases}
        y'' - y' - 2y = x^2 + 1 \\
        y(0) = 2 \\
        y'(0) = 0.
        \end{cases}
    \]
\end{snippetexercise}

\begin{snippetsolution}{diffeq-ex-9-sol}{}
    This is a second-order linear non-homogeneous differential equation with constant coefficients. The general solution is the sum of the homogeneous solution \( y_h(x) \) and a particular solution \( y_p(x) \).

    \begin{enumerate}
        \item \emph{Homogeneous equation}
        The associated homogeneous equation is \( y'' - y' - 2y = 0 \).
        The characteristic equation is:
        \[ \lambda^2 - \lambda - 2 = 0 \implies (\lambda - 2)(\lambda + 1) = 0. \]
        The roots are \( \lambda_1 = 2 \) and \( \lambda_2 = -1 \).
        Thus, the homogeneous solution is:
        \[ y_h(x) = c_1 e^{2x} + c_2 e^{-x}. \]

        \item \emph{Particular solution}
        The right-hand side is a polynomial of degree 2, \( g(x) = x^2 + 1 \). Since \( 0 \) is not a root of the characteristic equation, we search for a particular solution in the form of a generic polynomial of degree 2:
        \[ y_p(x) = Ax^2 + Bx + C. \]
        Calculate the derivatives:
        \[ y_p'(x) = 2Ax + B, \quad y_p''(x) = 2A. \]
        Substitute these into the original ODE:
        \[ (2A) - (2Ax + B) - 2(Ax^2 + Bx + C) = x^2 + 1. \]
        Group the terms by powers of \( x \):
        \[ -2Ax^2 + (-2A - 2B)x + (2A - B - 2C) = x^2 + 1. \]
        By comparing the coefficients, we obtain the system:
        \[
        \begin{cases}
            -2A = 1 \implies A = -\frac{1}{2} \\
            -2A - 2B = 0 \implies -2(-\frac{1}{2}) - 2B = 0 \implies 1 - 2B = 0 \implies B = \frac{1}{2} \\
            2A - B - 2C = 1 \implies 2(-\frac{1}{2}) - \frac{1}{2} - 2C = 1 \implies -1 - \frac{1}{2} - 1 = 2C \implies 2C = -\frac{5}{2} \implies C = -\frac{5}{4}.
        \end{cases}
        \]
        So, the particular solution is \( y_p(x) = -\frac{1}{2}x^2 + \frac{1}{2}x - \frac{5}{4} \).

        \item \emph{General solution}
        \[ y(x) = c_1 e^{2x} + c_2 e^{-x} - \frac{1}{2}x^2 + \frac{1}{2}x - \frac{5}{4}. \]

        \item \emph{Applying initial conditions}
        First, compute the derivative of the general solution:
        \[ y'(x) = 2c_1 e^{2x} - c_2 e^{-x} - x + \frac{1}{2}. \]
        Now apply the conditions \( y(0) = 2 \) and \( y'(0) = 0 \):
        \[
        \begin{cases}
            y(0) = c_1 + c_2 - \frac{5}{4} = 2 \implies c_1 + c_2 = \frac{13}{4} \\
            y'(0) = 2c_1 - c_2 + \frac{1}{2} = 0 \implies 2c_1 - c_2 = -\frac{1}{2}.
        \end{cases}
        \]
        Add the two equations to eliminate \( c_2 \):
        \[ (c_1 + 2c_1) + (c_2 - c_2) = \frac{13}{4} - \frac{2}{4} \implies 3c_1 = \frac{11}{4} \implies c_1 = \frac{11}{12}. \]
        Substitute \( c_1 \) back into the first equation:
        \[ \frac{11}{12} + c_2 = \frac{13}{4} \implies c_2 = \frac{39}{12} - \frac{11}{12} = \frac{28}{12} = \frac{7}{3}. \]

        \item \emph{Final solution}
        \[ y(x) = \frac{11}{12}e^{2x} + \frac{7}{3}e^{-x} - \frac{1}{2}x^2 + \frac{1}{2}x - \frac{5}{4}. \]
    \end{enumerate}
\end{snippetsolution}

\begin{snippetexercise}{diffeq-ex-10}{}
    Determine the general integral of the differential equation
    \[ y'' - 2y' + y = \frac{e^x}{1+x^2}. \]
\end{snippetexercise}

\begin{snippetsolution}{diffeq-ex-10-sol}{}
    This is a second-order linear non-homogeneous differential equation with constant coefficients. We solve it using the method of variation of parameters.
    \begin{enumerate}
        \item \emph{Homogeneous solution}
        The characteristic equation associated with the homogeneous part \( y'' - 2y' + y = 0 \) is:
        \[ \lambda^2 - 2\lambda + 1 = 0 \implies (\lambda - 1)^2 = 0. \]
        We have a real double root \( \lambda = 1 \).
        Therefore, the general solution of the homogeneous equation is:
        \[ y_h(x) = c_1 e^x + c_2 x e^x. \]
        \item \emph{Variation of parameters}
        Since the non-homogeneous term \( f(x) = \frac{e^x}{1+x^2} \) is not of a special form (polynomial-exponential), we use Lagrange's method. We look for a particular solution of the form:
        \[ y_p(x) = c_1(x)e^x + c_2(x)xe^x. \]
        First, we calculate the Wronskian determinant of the fundamental system \( \{e^x, xe^x\} \):
        \[ W(x) = \det \begin{pmatrix} e^x & xe^x \\ (e^x)' & (xe^x)' \end{pmatrix} = \det \begin{pmatrix} e^x & xe^x \\ e^x & e^x(1+x) \end{pmatrix}. \]
        \[ W(x) = e^x \cdot e^x(1+x) - xe^x \cdot e^x = e^{2x}(1+x-x) = e^{2x}. \]
        \item \emph{Calculating coefficients}
        We find \( c_1(x) \) and \( c_2(x) \) using the formulas involving the Wronskian:
        \[ c_1'(x) = -\frac{y_2(x) f(x)}{W(x)}, \quad c_2'(x) = \frac{y_1(x) f(x)}{W(x)}. \]
        Calculating \( c_1'(x) \):
        \[ c_1'(x) = -\frac{xe^x \cdot \frac{e^x}{1+x^2}}{e^{2x}} = -\frac{xe^{2x}}{e^{2x}(1+x^2)} = -\frac{x}{1+x^2}. \]
        Integrating with respect to \( x \):
        \[ c_1(x) = \int -\frac{x}{1+x^2} \, dx = -\frac{1}{2} \ln(1+x^2). \]
        Calculating \( c_2'(x) \):
        \[ c_2'(x) = \frac{e^x \cdot \frac{e^x}{1+x^2}}{e^{2x}} = \frac{e^{2x}}{e^{2x}(1+x^2)} = \frac{1}{1+x^2}. \]
        Integrating with respect to \( x \):
        \[ c_2(x) = \int \frac{1}{1+x^2} \, dx = \arctan x. \]
        \item \emph{General integral}
        The particular solution is:
        \[ y_p(x) = \left(-\frac{1}{2} \ln(1+x^2)\right)e^x + (\arctan x)xe^x. \]
        Combining this with the homogeneous solution \( y_h(x) \), we get the general integral:
        \[ y(x) = c_1 e^x + c_2 x e^x - \frac{1}{2} e^x \ln(1+x^2) + x e^x \arctan x. \]
        Factoring out \( e^x \):
        \[ y(x) = e^x \left( c_1 + c_2 x - \frac{1}{2} \ln(1+x^2) + x \arctan x \right). \]
    \end{enumerate}
\end{snippetsolution}

\begin{snippetexercise}{diffeq-ex-11}{}
    Solve the following Cauchy problem:
    \[
        \begin{cases}
        y'' - (\sin x)y' + y \ln(1+x) = 0 \\
        y(0) = 0 \\
        y'(0) = 0.
        \end{cases}
    \]
\end{snippetexercise}

\begin{snippetsolution}{diffeq-ex-11-sol}{}
    We are dealing with a second-order linear homogeneous differential equation of the form:
    \[ y'' + a_1(x)y' + a_0(x)y = 0, \]
    where \( a_1(x) = -\sin x \) and \( a_0(x) = \ln(1+x) \).
    \begin{enumerate}
        \item \emph{Check for existence and uniqueness}
        The coefficients \( a_1(x) \) and \( a_0(x) \) are continuous functions on the interval \( (-1, +\infty) \). Since the initial point \( x_0 = 0 \) belongs to this interval, the Existence and Uniqueness Theorem for linear differential equations guarantees that there exists a unique solution defined on this interval.
        \item \emph{Observation of the trivial solution}
        Notice that the constant function \( y(x) = 0 \) is a solution to the differential equation:
        \[ y'(x) = 0, \quad y''(x) = 0 \implies 0 - (\sin x)(0) + (0)\ln(1+x) = 0. \]
        Furthermore, it satisfies the initial conditions:
        \[ y(0) = 0 \quad \text{and} \quad y'(0) = 0. \]
        \item \emph{Conclusion}
        Since the solution is unique and the trivial solution satisfies the problem, it is the only solution.
        \[ y(x) \equiv 0. \]
    \end{enumerate}
\end{snippetsolution}

\begin{snippetexercise}{diffeq-ex-12}{}
    \todo
\end{snippetexercise}

\begin{snippetsolution}{diffeq-ex-12-sol}{}
    \todo
\end{snippetsolution}

\begin{snippetexercise}{diffeq-ex-13}{}
    Determine the general integral of the differential equation
    \[ y'' + 2y' + 3y = 3e^{2x} \sin(3x). \]
\end{snippetexercise}

\begin{snippetsolution}{diffeq-ex-13-sol}{}
    This is a second-order linear non-homogeneous differential equation with constant coefficients. The general solution is given by the sum of the homogeneous solution \(y_h(x)\) and a particular solution \(y_p(x)\).
    \begin{enumerate}
        \item \emph{Homogeneous solution}
        The characteristic equation associated with the homogeneous part \(y'' + 2y' + 3y = 0\) is:
        \[ \lambda^2 + 2\lambda + 3 = 0. \]
        Calculating the roots:
        \[ \Delta = 4 - 12 = -8. \]
        \[ \lambda_{1,2} = \frac{-2 \pm \sqrt{-8}}{2} = -1 \pm i\sqrt{2}. \]
        Since the roots are complex conjugates \(\alpha \pm i\beta\) with \(\alpha = -1\) and \(\beta = \sqrt{2}\), the homogeneous solution is:
        \[ y_h(x) = e^{-x}(c_1 \cos(\sqrt{2}x) + c_2 \sin(\sqrt{2}x)). \]
        \item \emph{Particular solution}
        The non-homogeneous term is \(f(x) = 3e^{2x} \sin(3x)\).
        We look for a particular solution using the method of undetermined coefficients (similarity method). The term corresponds to the complex number \(z = 2 + 3i\). Since \(2 + 3i\) is not a root of the characteristic equation, we guess a solution of the form:
        \[ y_p(x) = e^{2x}(A \cos(3x) + B \sin(3x)). \]
        We compute the derivatives:
        \[ y_p' = e^{2x}[ (2A+3B)\cos(3x) + (2B-3A)\sin(3x) ]. \]
        \[ y_p'' = e^{2x}[ (-5A+12B)\cos(3x) + (-12A-5B)\sin(3x) ]. \]
        Substituting \(y_p, y_p', y_p''\) into the original equation \(y'' + 2y' + 3y\):
        \begin{align*}
            e^{2x} [ &(-5A+12B)\cos(3x) + (-12A-5B)\sin(3x) \\
            &+ 2(2A+3B)\cos(3x) + 2(2B-3A)\sin(3x) \\
            &+ 3A\cos(3x) + 3B\sin(3x) ] = 3e^{2x} \sin(3x).
        \end{align*}
        Grouping the cosine and sine terms:
        \[ e^{2x} [ (2A + 18B)\cos(3x) + (-18A + 2B)\sin(3x) ] = 3e^{2x} \sin(3x). \]
        Comparing the coefficients, we obtain the system:
        \[
        \begin{cases}
            2A + 18B = 0 \implies A = -9B \\
            -18A + 2B = 3
        \end{cases}
        \]
        Substituting \(A\) into the second equation:
        \[ -18(-9B) + 2B = 3 \implies 162B + 2B = 3 \implies 164B = 3 \implies B = \frac{3}{164}. \]
        Consequently:
        \[ A = -9 \left(\frac{3}{164}\right) = -\frac{27}{164}. \]
        Thus, the particular solution is:
        \[ y_p(x) = \frac{3}{164} e^{2x} (-9 \cos(3x) + \sin(3x)). \]
        \item \emph{General Integral}
        Combining the results:
        \[ y(x) = e^{-x}(c_1 \cos(\sqrt{2}x) + c_2 \sin(\sqrt{2}x)) + \frac{3}{164} e^{2x} (\sin(3x) - 9 \cos(3x)). \]
    \end{enumerate}
\end{snippetsolution}

\begin{snippetexercise}{diffeq-ex-14}{}
    Determine the general integral of the differential equation
    \[ y''' + 6y'' + 11y' + 6y = 0. \]
\end{snippetexercise}

\begin{snippetsolution}{diffeq-ex-14-sol}{}
    This is a third-order linear homogeneous differential equation with constant coefficients.
    \begin{enumerate}
        \item \emph{Characteristic equation}
        The associated characteristic equation is:
        \[ \lambda^3 + 6\lambda^2 + 11\lambda + 6 = 0. \]
        \item \emph{Finding the roots}
        We look for integer roots among the divisors of the constant term 6 (i.e., \(\pm 1, \pm 2, \pm 3, \pm 6\)).
        Let's test \(\lambda = -1\):
        \[ (-1)^3 + 6(-1)^2 + 11(-1) + 6 = -1 + 6 - 11 + 6 = 0. \]
        Since \(\lambda = -1\) is a root, we can divide the polynomial by \((\lambda + 1)\).
        \[ \lambda^3 + 6\lambda^2 + 11\lambda + 6 = (\lambda + 1)(\lambda^2 + 5\lambda + 6). \]
        Now we solve the quadratic equation \(\lambda^2 + 5\lambda + 6 = 0\) by factoring:
        \[ (\lambda + 2)(\lambda + 3) = 0. \]
        The roots are distinct and real: \(\lambda_1 = -1\), \(\lambda_2 = -2\), \(\lambda_3 = -3\).
        \item \emph{General Integral}
        Since the roots are real and distinct, the general solution is a linear combination of the corresponding exponentials:
        \[ y(x) = c_1 e^{-x} + c_2 e^{-2x} + c_3 e^{-3x}, \]
        where \(c_1, c_2, c_3\) are arbitrary real constants.
    \end{enumerate}
\end{snippetsolution}

\begin{snippetexercise}{diffeq-ex-15}{}
    \todo
\end{snippetexercise}

\begin{snippetsolution}{diffeq-ex-15-sol}{}
    \todo
\end{snippetsolution}

\begin{snippetexercise}{diffeq-ex-16}{}
    \todo
\end{snippetexercise}

\begin{snippetsolution}{diffeq-ex-16-sol}{}
    \todo
\end{snippetsolution}

\begin{snippetexercise}{diffeq-ex-17}{}
    \todo
\end{snippetexercise}

\begin{snippetsolution}{diffeq-ex-17-sol}{}
    \todo
\end{snippetsolution}

\begin{snippetexercise}{diffeq-ex-18}{}
    \todo
\end{snippetexercise}

\begin{snippetsolution}{diffeq-ex-18-sol}{}
    \todo
\end{snippetsolution}

\begin{snippetexercise}{diffeq-ex-19}{}
    \todo
\end{snippetexercise}

\begin{snippetsolution}{diffeq-ex-19-sol}{}
    \todo
\end{snippetsolution}

\begin{snippetexercise}{diffeq-ex-20}{}
    \todo
\end{snippetexercise}

\begin{snippetsolution}{diffeq-ex-20-sol}{}
    \todo
\end{snippetsolution}

\begin{snippetexercise}{diffeq-ex-21}{}
    \todo
\end{snippetexercise}

\begin{snippetsolution}{diffeq-ex-21-sol}{}
    \todo
\end{snippetsolution}

\end{document}
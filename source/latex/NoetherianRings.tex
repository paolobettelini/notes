\documentclass[preview]{standalone}

\usepackage{amsmath}
\usepackage{amssymb}
\usepackage{stellar}
\usepackage{definitions}

\begin{document}

\id{noetherian-rings}
\genpage

\section{Noetherian ring}

\begin{snippetdefinition}{noetherian-ring-definition}{Noetherian ring}
    A \emph{left/right Noetherian ring} \(A\) is a \ring
    that satisfies the ascending chain condition:
    for every increasing sequence \(I_1 \subseteq I_2 \cdots\)
    of left/right ideals has a largest element, meaning that
    there exists \(n\) such that \(I_n = I_{n+1} = \cdots\).
\end{snippetdefinition}

\begin{snippettheorem}{noetherian-ring-equivalence-theorem}{}
    Let \(A\) be a commutative \ring. The following are equivalent:
    \begin{enumerate}
        \item \(A\) is Noetherian;
        \item every ideal \(I \idealin A\) is finitely generated, meaning
        \(\exists x_1, \cdots, x_r \in I\) such that
        \[
            I = (x_1) + (x_2) \cdots + (x_r)
        \]
        that is \(\forall y \in I\), \(\exists a_1, \cdots, a_r\)
        such that \(y = a_1x_1 + a_2 x_2 + \cdots, a_rx_r\);
        \item every subset \(\mathcal{J} \neq \emptyset\) of \ideal[ideals] of \(A\)
        has a maximal element, meaning \(\exists I \in \mathcal{J}\)
        such that \(I \notsubseteq J\) for all \(J \in \mathcal{J}\). 
    \end{enumerate}
\end{snippettheorem}

\begin{snippetproof}{noetherian-ring-equivalence-theorem-proof}{noetherian-ring-equivalence-theorem}{}
    \begin{enumerate}
        \item \((1) \implies (3)\): Let \(\mathcal{J} \neq \emptyset\)
        a \set of \ideal[ideals] of \(A\).
        Choose \(I_1 \in \mathcal{J}\).
        If \(I_1\) is \maximalideal[maximal] in \(\mathcal{J}\), we are done.
        Otherwise, \(\exists I_2 \in \mathcal{J}\) such that \(I_1 \subset I_2 \cdots\).
        We iterate this process until we find a \maximalideal.
        Since \(A\) is Noetherian, the extraction will end in a finite amount of picks.
        \item \((3) \implies (2)\): Let \(I \idealin A\).
        Let us check the trivial cases.
        If \(I = \{0_A\} = (0_A)\) we are done.
        If \(I = A = (1_A)\) we are also done.
        Suppose that \(I\) is not trivial.
        Consider the \set of \ideal[ideals] \(\mathcal{J}_I \neq 0\) containing the \ideal[ideals] \(J\)
        finitely generated such that \(J \subseteq I\).
        We know that \(0_A \in I = (0_A) \subseteq I\)
        which means \((0_A) \in \mathcal{J}_I\).
        Let \(J \in \mathcal{J}\) be maximal in \(\mathcal{J}_I\).
        This means that \(J = I\).
        Indeed, if \(J \subset I\) then \(\exists y \in I \difference J\).
        Furthermore, \(J\) is finitely generated and thus \(\exists x_1, \cdots, x_r \in J\)
        which generate
        \[
            \tilde{J} \triangleq J + (y) \subseteq I
        \]
        and
        \[
            \tilde{J} = (x_1) + \cdots + (x_r)
        \]
        which is finitely generated.
        Thus, \(\tilde{J} \in \mathcal{J}_I\) but \(\tilde{J} \supset J\) which was maximal
        in \(\mathcal{J}_I\) \lightning.
        Thus, \(J = I\) implies that \(I\) is finitely generated.
    \end{enumerate}
\end{snippetproof}

\end{document}
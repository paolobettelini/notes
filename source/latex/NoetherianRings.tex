\documentclass[preview]{standalone}

\usepackage{amsmath}
\usepackage{amssymb}
\usepackage{stellar}
\usepackage{definitions}

\begin{document}

\id{noetherian-rings}
\genpage

\section{Noetherian ring}

\begin{snippetdefinition}{noetherian-ring-definition}{Noetherian ring}
    A \emph{left/right Noetherian ring} \(A\) is a \ring
    that satisfies the ascending chain condition:
    for every increasing sequence \(I_1 \subseteq I_2 \cdots\)
    of left/right ideals has a largest element, meaning that
    there exists \(n\) such that \(I_n = I_{n+1} = \cdots\).
\end{snippetdefinition}

\begin{snippettheorem}{noetherian-ring-equivalence-theorem}{}
    Let \(A\) be a \ring. The following are equivalent:
    \begin{enumerate}
        \item \(A\) is Noetherian (commutative);
        \item every ideal \(I \idealin A\) is finitely generated, meaning
        \(\exists x_1, \cdots, x_r \in I\) such that
        \[
            I = (x_1) + (x_2) \cdots + (x_r)
        \]
        that is \(\forall y \in I\), \(\exists a_1, \cdots, a_r\)
        such that \(y = a_1x_1 + a_2 x_2 + \cdots, a_rx_r\);
        \item every subset \(\mathcal{J} \neq \emptyset\) of \ideal[ideals] of \(A\)
        has a maximal element, meaning \(\exists I \in \mathcal{J}\)
        such that \(I \notsubseteq J\) for all \(J \in \mathcal{J}\). 
    \end{enumerate}
\end{snippettheorem}

\begin{snippetproof}{noetherian-ring-equivalence-theorem-proof}{noetherian-ring-equivalence-theorem}{}
    \begin{enumerate}
        \item \((1) \implies (3)\): Let \(\mathcal{J} \neq \emptyset\)
        a \set of \ideal[ideals] of \(A\).
        Choose \(I_1 \in \mathcal{J}\).
        If \(I_1\) is \maximalideal[maximal] in \(\mathcal{J}\), we are done.
        Otherwise, \(\exists I_2 \in \mathcal{J}\) such that \(I_1 \subset I_2 \cdots\).
        We iterate this process until we find a \maximalideal.
        Since \(A\) is Noetherian, the extraction will end in a finite amount of picks.
        \item \((3) \implies (2)\): Let \(I \idealin A\).
        Let us check the trivial cases.
        If \(I = \{0_A\} = (0_A)\) we are done.
        If \(I = A = (1_A)\) we are also done.
        Suppose that \(I\) is not trivial.
        Consider the \set of \ideal[ideals] \(\mathcal{J}_I \neq 0\) containing the \ideal[ideals] \(J\)
        finitely generated such that \(J \subseteq I\).
        We know that \(0_A \in I = (0_A) \subseteq I\)
        which means \((0_A) \in \mathcal{J}_I\).
        Let \(J \in \mathcal{J}\) be maximal in \(\mathcal{J}_I\).
        This means that \(J = I\).
        Indeed, if \(J \subset I\) then \(\exists y \in I \difference J\).
        Furthermore, \(J\) is finitely generated and thus \(\exists x_1, \cdots, x_r \in J\)
        which generate
        \[
            \tilde{J} \triangleq J + (y) \subseteq I
        \]
        and
        \[
            \tilde{J} = (x_1) + \cdots + (x_r)
        \]
        which is finitely generated.
        Thus, \(\tilde{J} \in \mathcal{J}_I\) but \(\tilde{J} \supset J\) which was maximal
        in \(\mathcal{J}_I\) \lightning.
        Thus, \(J = I\) implies that \(I\) is finitely generated.
        \item \((2) \implies (1)\):
        We show that given a chain of \ideal[ideals]
        it must terminate in a finite amount of steps.
        Consider
        \[
            \mathcal{J} = \bigcup_{i\geq 1} I_i \idealin A
        \]
        We have two cases:
        \begin{enumerate}
            \item if \(\mathcal{J} = A\), we have \(1_A \in \mathcal{J}\)
            and thus \(1_A \in I_i\) for some \(i\).
            This also means that \(I_i = A = I_n\)
            and the chain terminates in at most \(i\) steps.
            \item if \(\mathcal{J} \subset A\), then \(\mathcal{J}\)
            is finitely generated by some \(\{x_1, \cdots, x_n\}\).
            Thus, \[x_1, x_2, \cdots, x_n \in \mathcal{J} = \bigcup_{i\geq 1} I_i\]
            and \(x_k \in I_{i_k}\) for \(k \leq n\).
            Let \(m = \max\{i_1, \cdots, i_n\}\).
            Then, \(I_m \supseteq I_{i_k}\)
            and thus \(x_1, \cdots, x_n \in I_m\).
            This means that \(\forall a_1, \cdots, a_n \in A\),
            \[
                \sum_{i=1}^n a_ix_i \in I_m
            \]
            which are all terms in \(I_m\) and \(I_m \supseteq \mathcal{J} \supseteq I_i\).
            Thus, the chain terinates with \(I_n\).
        \end{enumerate}
    \end{enumerate}
\end{snippetproof}

\begin{snippettheorem}{hilbert-basis-theorem}{Hilbert's basis theorem}
    Let \(A\) be Noetherian. Then, \(A[x]\) is Noetherian.
\end{snippettheorem}

\begin{snippetproof}{hilbert-basis-theorem-proof}{hilbert-basis-theorem}{Hilbert's basis theorem}
    Consider an \ideal \(\mathcal{J} \neq A[x]\) of \(A[x]\).
    By \snippetref[noetherian-ring-equivalence-theorem-proof][this equivalence], \(\mathcal{J}\) is finitely generated.
    \(\forall i \geq 0\) define
    \[
        I_i = \{a \in A \suchthat \exists f \in \mathcal{J} \land \polynomialdeg(f) = f \land f = ax^i + \sum_{k=j}^{i-1} \alpha_j x^j \}
    \]
    We have:
    \begin{enumerate}
        \item \(0_A \in I\);
        \item if \(a,b \in I\), then \(\exists f,g \in \mathcal{J}\)
        such that \[
            f = ax^i + \sum_{j=0}^{i-1} \alpha_j x^j, \quad
            g = bx^i + \sum_{j=0}^{i-1} \beta_j x^j
        \]
        and thus \(f-g \in \mathcal{J}\);
        \item if \(c \in A\) (viewed as \(A[x]\)), \(a \in I_i\) then \(\exists f \in \mathcal{J}\)
        such that \(f = ax^i + \sum_j \alpha_jx^j\).
        Thus, \(c\cdot f \in \mathcal{J}\)
    \end{enumerate}
    And thus \(I_i \idealin A\).
    We now show that \(I_0 \subseteq I_1 \subseteq \cdots \subset I_{i} \subseteq I_{i+1} \cdots\).
    Let \(a \in I_i\). This means that \(\exists f \in \mathcal{J}\) where
    \[
        f = ax^i + \sum_{j=0}^{i-1} \alpha_j x^j  
    \]
    such that \(x \cdot f \in \mathcal{J}\) which can be written as
    \[
        c \cdot f = ax^{i+1} + \sum_{j=0}^i \cdots
    \]
    but this also means that \(a \in I_{i+1}\) and by induction \(I_{i+1} \supseteq I_i\).
    The chain terminates with \(I_m\) and each \(I_i\)
    is generated by \(\{a(i)_h \suchthat 1 \leq h \leq r_i\}\) (we can take \(a(i)_h \neq 0_A\)).
    Given \(a(i)_h \in I_i\), \(\forall 0 \leq i \leq m\) and \(\forall 1 \leq h \leq r_i\),
    \(\exists f(i)_h \in \mathcal{J}\) such that \(\polynomialdeg(f(i)_h) = i\)
    and \(f(i)_h = a(i)_h x^i + \sum \cdots\).
    We want to shwo that \(\mathcal{J}\) is generated by all the \(f(i)_h\),
    meaning
    \[
        \mathcal{J} = \sum_{i=0}^m \left(
            \sum_{h=1}^{r_i} f(i)_h
        \right)
    \]
    Let \(g \in \mathcal{J}\) with any \(\polynomialdeg(g) = s\).
    We have
    \[
        g_0 = b_0x^s + \sum \cdots, \quad b_0 \in I_s
    \]
    We can suppose \(m \geq s\) and we have
    \[
        b_0 = \sum_{i=1}^{r_s} c_i a(s)_i
    \]
    for some \(c_i \in A\). We then define
    \begin{align*}
        g_1 &\triangleq g_0 - \sum_{i=1}^{r_s} c_i \cdot f(s)_i
        = g_0 - \sum_{i=1}^{r_s} \left(c_i a(s)_i x^s + \sum \cdots \right) \\
        &= \left(b_0 x^s + \sum \cdots\right) - \left(
            x^s \left(
                \sum_{i=1}^{r_s} c_i a(s)_i
            \right) + \sum \cdots
        \right) \\
        &= \left(b_0 x^s + \sum \cdots\right) - \left(
            b_0 x^s + \sum \cdots
        \right)
    \end{align*}
    since the two \(b_0x^s\) cancel out we have \(\polynomialdeg(g_1) < s\).
    Let \(s_1 = \polynomialdeg(g_1)\) and thus \(s_1 < s = \polynomialdeg(g_0)\).
    We want to show that \(g_0 = g_1 + p\) where \(p\) is a polinomial finitely generated by \(\{f(s)_i, \cdots, f(s)_{r_s}\}\).
    We have two cases:
    \begin{enumerate}
        \item \(g_1 = 0_A\) and \(g_0\) is generated by \(\{f(s)_i, \cdots, f(s)_{r_s}\}\);
        \item \(g_1 \neq 0_A\) and
        \[
            g_1 = b_1 x^{s_1} + \sum \cdots
        \]
        for some \(b_1 \in I_{s_1}\)
        which can be written as \[
            b_1 = \sum_{i=1}^{r_{s_1}} c_i' a(s_1)_i
        \]
        We now define
        \begin{align*}
            g_2 &\triangleq g_1 - \sum_{i=1}^{r_{s_1}} c_i' \cdot f(s_1)_i \\
            &= \left(b_1 x^{s_i} + \sum \cdots\right) - \left(
                x^{s_1} \cdot \left(\sum c_i' a(s_1)_i\right) + \sum \cdots
            \right) \\
            g_1 &= g_2 + p'
        \end{align*}
        where \(p'\) is generated by \(\{f(s_1)_h\}\),
        which means that \(g_0 = g_1 + p''\)
        where \(p''\) is generated by \(\{f(s)_h\}\).
    \end{enumerate}
    By iterating this process, the degree becomes lower and ultimately
    \(g_0\) is generated by 
    \[
        \{
            f(s)_h, f(s_1)_h, \cdots \cdots    
        \}
    \]
    which is a finite \set
    and thus by \snippetref[noetherian-ring-equivalence-theorem-proof][this equivalence]
    it is Noetherian.
\end{snippetproof}

\end{document}
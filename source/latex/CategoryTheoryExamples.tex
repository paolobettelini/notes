\documentclass[preview]{standalone}

\usepackage{amsmath}
\usepackage{amssymb}
\usepackage{stellar}
\usepackage{definitions}

\begin{document}

\id{categorytheory-examples}
\genpage

\section{Examples}

\subsection{Examples of categories}

\begin{snippetdefinition}{categorical-monoid-defintion}{Monoid}
    A \textit{monoid} is a \category with a single object.
\end{snippetdefinition}

\plain{Every element of the monoid is represented by a morphism from the object to itself.
The binary operation of the monoid is the composition of the morphisms.}

\subsection{Examples of objects}

\subsection{Examples of morphism types}

\begin{snippetproposition}{category-set-monomorphism}{Monomorphism in \textbf{Set}}
    In the \category \textbf{Set}, a morphism is a
    \monomorphism \ifandonlyif it is an \injective \function.
\end{snippetproposition}

\begin{snippetproof}{category-set-monomorphism-proof}{category-set-monomorphism}{Monomorphisms in \textbf{Set}}
    \todo % hint: Consideriamo l'oggetto terminale (singoletto).
    % le freccie che vanno da {a} ad X corrispondono agli elementi di X
\end{snippetproof}

\begin{snippetproposition}{category-set-epimorphism}{Epimorphism in \textbf{Set}}
    In the \category \textbf{Set}, a morphism is an
    \epimorphism \ifandonlyif it is a \surjective \function.
\end{snippetproposition}

\begin{snippetproof}{category-set-epimorphism-proof}{category-set-epimorphism}{Epimorphisms in \textbf{Set}}
    \todo
\end{snippetproof}

\begin{snippetproposition}{category-set-isomorphism}{Isomorphism in \textbf{Set}}
    In the \category \textbf{Set}, a morphism is an
    \catisomorphism \ifandonlyif it is a \bijective \function.
\end{snippetproposition}

\begin{snippetproof}{category-set-isomorphism-proof}{category-set-isomorphism}{Isomorphisms in \textbf{Set}}
    \todo
\end{snippetproof}

\end{document}
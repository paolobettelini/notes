\documentclass[preview]{standalone}

\usepackage{amsmath}
\usepackage{amssymb}
\usepackage{stellar}
\usepackage{definitions}

\begin{document}

\id{quotient-spaces}
\genpage

\section{Quotient spaces}

\begin{snippetdefinition}{vector-space-quotient-definition}{Quotient space}
    Let \(V\) be a \vectorspace on a \field \(\mathbb{K}\)
    of dimension \(n\) and let \(W\) be a linear subspace of \(V\).
    Consider an \equivrelation \(\sim\) on \(V\) defined as
    \(v_1 \sim v_2 \iff v_1 - v_2 \in W\).
    The \equivclass is given by
    \[
        {[v]}_\sim = \{v + w \suchthat w \in W\} = v + W
    \]
    The \emph{quotient space} \(V/W\) is defined as \((V/_\sim, \mathbb{K}, +, \cdot)\)
    where
    \[
        \alpha {[x]}_\sim \triangleq {[\alpha x]}_\sim, \quad \forall \alpha \in \mathbb{K}
    \]
    and
    \[
        {[x]}_\sim + {[y]}_\sim \triangleq {[x + y]}_\sim
    \]
\end{snippetdefinition}

\plain{This is analogous to the quotient group: the difference is the inverse,
just like in the condition of membership on the normal subgroup.}

\plain{The subspace is collpased to zero.}

\begin{snippetproposition}{vector-space-quotient-is-vector-space}{Vector space quotient is a vector space}
    The quotient space is a \vectorspace.
\end{snippetproposition}

\begin{snippetproof}{vector-space-quotient-is-vector-space-proof}{vector-space-quotient-is-vector-space}{Vector space quotient is a vector space}
    We first show that the sum of the cosets is well-defined.
    Consider \(v_1, v_2 \in V\) and \(v_1' \in v_1 + W\) and \(v_2' \in v_2 + W\).
    We want to verify that
    \begin{align*}
        \left(v_1' + W\right) + 
        \left(v_2' + W\right)
        = (v_1' + v_2') + W = (v_1 + v_2) + W
    \end{align*}
    By expressing \(v_1'\) and \(v_2'\) as a sum with \(w_1\) and \(w_2\) respectively we get
    \begin{align*}
        v_1' + v_2' = (v_1 + w_1) + (v_2 + w_2) &= (v_1 + v_2) + \underbrace{(w_1 + w_2)}_{\in W}
    \end{align*}
    This means that \(v_1' + v_2' \in (v_1 + v_2) + W\)
    and thus \((v_1' + v_2') + W = (v_1 + v_2) + W\). \\
    We now show that the scalar product is well-defined.
    Consider \(v' \in v + W\), meaning that \(v' = v + w\) with \(w \in W\).
    \begin{align*}
        \alpha \cdot (v' + W) &= (\alpha \cdot v') + W = (\alpha v + \underbrace{\alpha w}_{\in W}) + W \\
        &= \alpha v + W = \alpha \cdot \left(v + W\right)
    \end{align*}
    We now prove the properties of the addition:
    \begin{enumerate}
        \item the \(+\) operator inherites the commutative property;
        \item the identity element is the coset of \(0_V\) meaning \(0_V + W\).
        Indeed, \((v + W) + (0_V + W) = (v + 0_V) + W = v + W\) for all \(v \in V\);
        \item the inverse element of a coset is \(-(v+W) = (-v) + W\) for all \(v \in V\).
        Indeed, \((v + W) + \left((-v) + W\right) = 0_V + W = 0_{V/W}\);
    \end{enumerate}
    We now prove the properties of the scalar multiplication:
    \begin{align*}
        \alpha \cdot \left((v_1 + W) + (v_2 + W)\right)
        &= \alpha \cdot (v_1 + W) + \alpha \cdot (v_2 + W) \\
        &= \alpha \left((v_1 + v_2) + W\right)
        = \left(\alpha(v_1 + v_2)\right) + W \\
        &= \left(\alpha v_1 + W\right)
        + \left(\alpha v_2 + W\right) = \alpha \cdot (v_1 + W) + \alpha \cdot (v_2 + W)
    \end{align*}
    for all \(\alpha \in \mathbb{K}\) and for all \(v_1 + W, v_2 + W \in V/W\)
    (the other is distributivity property analogous).
\end{snippetproof}

\begin{snippetproposition}{dimension-of-quotient-space}{Dimension of quotient space}
    Let \(V\) be a \vectorspace on a \field \(\mathbb{K}\)
    and let \(W\) be a linear subspace of \(V\).
    Then,
    \[
        \lineardim V/W = \lineardim V - \lineardim W
    \]
\end{snippetproposition}

\begin{snippetproof}{dimension-of-quotient-space-proof}{dimension-of-quotient-space}{Dimension of quotient space}
    Let \(\lineardim V = n\) and \(\lineardim W = r\).
    Consider the \basis \(\mathcal{B}_W = \{w_1, w_2, \cdots, w_r\}\) of \(W\) and \(\mathcal{B} = \mathcal{B}_W \union \{v1, v_2, \cdots, v_{n-r}\}\).
    By \linearlyindependent[linear dependency],
    we know that \(v_i + W \neq w_j + W\) if \(i \neq j\).
    We want to show that
    \[
        \mathcal{B}_{V/W} = \left\{v_1 + W, \cdots, v_{n-r} + W\right\}
    \]
    is a \basis for \(V/W\).
    \begin{enumerate}
        \item \emph{linear independence:} we have
        \begin{align*}
            \sum_i^{n-r} \alpha_i (v_i + W) &= 0_{V/W} = 0_V + W = \left(\sum_i^{n-r} v_i\right) + W
        \end{align*}
        this means that \[
            \sum_{i}^{n-r} \alpha_i v_i \in 0_V + W = W
        \]
        meaning that \(\exists \beta_1, \cdots, \beta_r \in \mathbb{K}\) such that
        \[
            \sum_{i}^{n-r} \alpha_i v_i = \sum_{i}^{r} \beta_i v_i
        \]
        which clearly shows that 
        \[
            \sum_{i}^{n-r} \alpha_i(v_i + W)
        \]
        is \linearlyindependent and thus \(\mathcal{B}_{V/W}\) is aswell.
        \item \(\mathcal{B}_{V/W}\) generates \(V/W\).
        Take \(v + W \in V/W\) with \(v \in V\).
        We have
        \[
            v = \beta_1 w_1 + \cdots + \alpha_{n-r} v_{n-r}
        \] 
        for \(\alpha_i, \beta_i \in \mathbb{K}\). Thus,
        \begin{align*}
            v + W &= \left(b_1w_1 + \cdots + b_rw_r + \alpha_1 v_1 + \cdots + \alpha_{n-r} v_{n-r}\right) + W
            = \sum_i^{n-r} \alpha_i v_i + W \\
            &= \sum_i^{n-r} \alpha_i \left(v_i + W\right)
        \end{align*}
        which is a linear combination.
    \end{enumerate}
\end{snippetproof}

\end{document}
\documentclass[preview]{standalone}

\usepackage{amsmath}
\usepackage{amssymb}
\usepackage{stellar}
\usepackage{definitions}
\usepackage{bettelini}

\begin{document}

\id{norms}
\genpage

\section{Definition}

\begin{snippetdefinition}{norm-definition}{Norm}
    Let \(X\) be a \vectorspace over a subfield \(F\) of \(\complexnumbers\).
    Then, a \emph{norm} is a \function \(p\colon X \fromto \realnumbers\) such that
    \begin{enumerate}
        \item \emph{triangle inequality}: \(\forall x,y\in X\colon p(x+y) \leq p(x) + p(y)\);
        \item \emph{absolute homogeneity}: \(\forall x\in X,\forall \lambda\in F\colon p(\lambda x) = |\lambda|p(x)\).
        \item \emph{positivity}: \(\forall x\in X\colon p(x) \geq 0\) and \(p(x) = 0\) \ifandonlyif \(x = 0\);
    \end{enumerate}
\end{snippetdefinition}

\begin{snippetproposition}{norm-is-uniformely-continuous}{}
    A \norm is uniformely continuous.
\end{snippetproposition}

\begin{snippetdefinition}{little-l-p-norm-definition}{\(l^p\) norm}
    Let \(X = F^n\) where \(F\) is a subfield of \(\complexnumbers\) and \(n\in\naturalnumbers\).
    Then, the \emph{\(l^p\) norm} is \norm
    \[
        ||x||_p = \begin{cases}
            \left( \sum_{i=1}^n |x_i|^p \right)^{1/p} & p < \infty\\
            \max_{i=1,\ldots,n} |x_i| & p = \infty
        \end{cases}
    \]
\end{snippetdefinition}

\begin{snippetproposition}{norms-equivalence}{Norms equivalence}
    Let \(||\cdot||_*\) and \(||\cdot||_{**}\) be two \norm[norms] over a finitely dimensional \(X\). Then,
    \(\exists 0 < \alpha \leq \beta < \infty\) such that
    \[
        \forall x\in X\colon \alpha ||x||_* \leq ||x||_{**} \leq \beta ||x||_*
    \]
\end{snippetproposition}

\begin{snippetdefinition}{matrix-norm-definition}{Matrix norm}
    Let \(F\) be a subfield of \(\complexnumbers\) and \(\matrices_{m\times n}(F)\)
    be the \(F\)-\vectorspace of \(m\times n\) \matrix[matrices] over \(F\).
    Then, a \emph{matrix norm} is a \norm on \(\matrices_{m\times n}(F)\).
\end{snippetdefinition}

\begin{snippetdefinition}{sub-multiplicative-matrix-norm-definition}{Sub-multiplicative matrix norm}
    A \mnorm on \(\matrices_{n\times n}(F)\) is said to be \emph{sub-multiplicative} if
    \[
        ||AB|| \leq ||A|| \cdot ||B||
    \]
    for all \(A,B\in \matrices_{n\times n}(F)\).
\end{snippetdefinition}

\begin{snippetdefinition}{l-p-q-matrix-norm-definition}{\(L_{p,q}\) matrix norm}
    The \mnorm \(L_{p,q}\) on \(\matrices_{m\times n}(F)\) is defined as
    \[
        ||A||_{p,q} \triangleq {\left(
            \sum_{j=1}^n {\left(
                \sum_{i=1}^m |A_{i,j}|^p
            \right)}^{\frac{q}{p}}
        \right)}^{\frac{1}{q}}
    \]
\end{snippetdefinition}

\begin{snippetdefinition}{induced-matrix-norm-definition}{Induced matrix norm}
    Let \(||\cdot||_*\) be a \norm on \(F^n\). The \emph{induced matrix norm}
    on \(\matrices_{m\times n}(F)\) is defined as
    \[
        ||A|| \triangleq \sup_{\underset{x\in F^n}{x \neq 0}} \frac{||Ax||_*}{||x||_*}
    \]
\end{snippetdefinition}

\begin{snippettheorem}{induced-matrix-norm-alternate-form-theorem}{}
    \[
        ||A|| = \sup_{\underset{x\in F^n}{||x||_* = 1}} ||Ax||_*
    \]
\end{snippettheorem}

\begin{snippetproof}{induced-matrix-norm-alternate-form-theorem-proof}{induced-matrix-norm-alternate-form-theorem}{}
    \begin{align*}
        ||A|| &= \sup_{\underset{x\in F^n}{x \neq 0}} \frac{||Ax||_*}{||x||_*} \\
        &= \sup_{\underset{x\in F^n}{x \neq 0}} \frac{\left|\left|A \left(||x||_* \cdot \frac{x}{||x||_*}\right)\right|\right|}{||x||_*} \\
        &= \sup_{\underset{x\in F^n}{x \neq 0}} \frac{||x|| \left|\left| A \left(\frac{x}{||x||_*}\right)\right|\right|}{||x||_*} \\
        &= \sup_{\underset{x\in F^n}{x \neq 0}} \left|\left|A \left(\frac{x}{||x||_*}\right)\right|\right| \\
        &= \sup_{\underset{x\in F^n}{||x||_* = 1}} ||Ax||_*
    \end{align*}
\end{snippetproof}

\begin{snippetproposition}{induced-matrix-norm-properties}{}
    \begin{enumerate}
        \item \(||\identmatrix{n}|| = 1\);
        \item \(\spectralradius(A) \leq ||A||\);
        \item \(||Ax|| \leq ||A|| \cdot ||x||\).
    \end{enumerate}
\end{snippetproposition}

\begin{snippetproposition}{metric-induced-by-norm}{Metric induced by norm}
    A \distancefunctiontext \(d\) on a \vectorspace \(V\) is induced by a
    \norm \ifandonlyif \(\forall x,y,z \in V\)
    \begin{enumerate}
        \item \(d(x+z, y + z) = d(x,y)\);
        \item \(d(\lambda x, \lambda y) = |\lambda| d(x,y)\).
    \end{enumerate}
\end{snippetproposition}

\begin{snippetproof}{metric-induced-by-norm-proof}{metric-induced-by-norm}{Metric induced by norm}
    \iffproof{
        Let \(d(x,y) = ||x-y||\) for some \norm. Then, we have that
        \(\forall x,y,z \in V\)
        \[
            d(x+z, y+z) = ||(x+z) - (y+z)|| = ||x-y|| = d(x,y)
        \]
        and \(\forall \lambda\) in the \field
        \[
            d(\lambda x, \lambda y) = ||\lambda x - \lambda y||
            = |\lambda| \cdot ||x-y|| = |\lambda| d(x,y)
        \]
        which are thus necessary conditions.
    }{
        Let \(||x|| \triangleq d(x,0)\).
        The \norm \(||\cdot||\) is clearly positive (since the metric is).
        It is also homogeneous
        \[
            ||\lambda x|| = d(\lambda x, 0) =|\lambda| d(x,0) = |\lambda| \cdot ||x||
        \]
        and satisfies the triangular inequality
        \[
            ||x+y|| = d(x+y,0) \leq d(x+y, x) + d(x,0) = d(y,0) + ||x|| = ||y|| + ||x||
        \]
        Thus, \(||\cdot||\) is a \norm and \(d(x,y) = ||x-y||\).
    }
\end{snippetproof}

\end{document}
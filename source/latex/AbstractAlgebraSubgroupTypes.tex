\documentclass[preview]{standalone}

\usepackage{amsmath}
\usepackage{amssymb}
\usepackage{stellar}
\usepackage{definitions}

\begin{document}

\id{types-of-subgroups}
\genpage

\section{The centralizer subgroup}

\begin{snippetdefinition}{centralizer-group}{The centralizer subgroup}
    Let \((H, \circ) \subgroupleq (G, \circ)\) be \group[groups].
    The \textit{centralizer} of \(H\) is defined as
    the \group with the elements of \(G\) that commute with every element of \(H\)
    with respect to \(\circ\).
    \[
        \text{C}_{(G, \circ)}(H) = \{
            g \in G \suchthat \forall h \in H, g\circ h=h\circ g
        \}
    \]
\end{snippetdefinition}

\begin{snippettheorem}{centralizer-of-subgroup-is-subgroup}{}
    Let \((H, \circ) \subgroupleq G\) be \group[groups]. Then, \(\text{C}_G(H) \subgroupleq G\).
\end{snippettheorem}

\begin{snippetproof}{centralizer-of-subgroup-is-subgroup-proof}{centralizer-of-subgroup-is-subgroup}{Centralizer of subgroup is subgroup}
    Suppose \(a,b \in \text{C}_G(H)\).
    We want to show \(ab^{-1} \in \text{C}_G(H)\). \\
    Note that the condition \(gh=hg \iff hg^{-1}=g^{-1}h\). \\
    Consider the expression \((ab^{-1})h = a(b^{-1}h) = ahb^{-1} = h(ab^{-1})\).
    This means that \(ab^{-1} \in \text{C}_G(H)\) and thus in \(H\).
\end{snippetproof}

\section{Center of a group}

\begin{snippetdefinition}{center-of-group-definition}{Center of a group}
    Let \(G\) be a \group. The \textit{center} of \(G\) is defined as
    the \group with every element of \(G\) that commute with every other element
    \[
        \text{Z}(G) = \{
            g \in G \suchthat \forall x \in G, gx = xg
        \}
    \]
\end{snippetdefinition}

\begin{snippet}{center-of-group-condition-alternative}
    The condition \(gx=xg\) is also sometimes expressed as \(gxg^{-1} = x\).
\end{snippet}

\begin{snippettheorem}{center-of-group-is-subgroup}{}
    Let \(G\) be a \group, then \(\groupcenter(G) \subgroupleq G\).
\end{snippettheorem}

\begin{snippetproof}{center-of-group-is-subgroup-proof}{center-of-group-is-subgroup}{}
    Assume \(a, b \in \groupcenter(G)\) meaning \(a = gag^{-1}\) and \(b = gag^{-1}\) for any \(g \in G\). \\
    We want to show \(ab^{-1} \in \groupcenter(G)\).
    \(ab^{-1} = (gag^{-1}){(gbg^{-1})}^{-1} = gag^{-1}gb^{-1}g^{-1}
    = g ab^{-1} g^{-1}\) which is precisely the requirement to be in \(\groupcenter(G)\).
\end{snippetproof}

\section{The conjugate subgroup}

\begin{snippetdefinition}{conjugate-subgroup-definition}{The conjugate subgroup}
    Let \(H \subgroupleq G\) be \group[groups]. Then, the \textit{conjugate subgroup} is defined as
    \[
        g^{-1}Hg = \{
            g^{-1}hg \suchthat h \in H
        \}
    \]
\end{snippetdefinition}

\begin{snippettheorem}{conjugate-subgroup-is-subgroup}{}
    Let \(H \subgroupleq G\), then \(g^{-1}Hg \subgroupleq G\).
\end{snippettheorem}

\begin{snippetproof}{conjugate-subgroup-is-subgroup-proof}{conjugate-subgroup-is-subgroup}{}
    Suppose \(a,b \in g^{-1}Hg\).
    We want to show \(ab^{-1} \in g^{-1}Hg\).\\
    Note that \(a = g^{-1}h_1g\) and \(b = g^{-1}h_2g\)
    for some \(h_1, h_2 \in H\). \\
    This means that \(ab^{-1}=a{(g^{-1}h_2g)}^{-1} = a(g^{-1}h_2^{-1}g)
    =g^{-1}h_1gg^{-1}h_2^{-1}g = g^{-1} (h_1h_2) g \in g^{-1}Hg \)
    because \(h_1h_2 \in H\).
\end{snippetproof}

\end{document}
\documentclass[preview]{standalone}

\usepackage{amsmath}
\usepackage{amssymb}
\usepackage{stellar}
\usepackage{definitions}
\usepackage{bettelini}

\begin{document}

\id{normed-spaces-exercises}
\genpage

\section{Exercises}

\begin{snippetproposition}{metric-space-diameter-cloure}{}
    Let \((X, d)\) be a \metricspace and \(A \subseteq X\).
    Then,
    \[
        \msdiameter A = \msdiameter \overline{A}
    \]
\end{snippetproposition}

\begin{snippetproof}{metric-space-diameter-cloure-proof}{metric-space-diameter-cloure}{}
    \doubleleqproof{
        Since \(A \subseteq \overline{A}\), 
        \[
            \msdiameter A = \sup_{x,y \in A} d(x,y)
            \leq \sup_{x,y \in \overline{A}} d(x,y)
            = \msdiameter \overline{A}
        \]
    }{
        Let \(x,y \in \overline{A}\).
        By definition of closure, there exist \sequence[sequences]
        \(\{x_n\}\) and \(\{y_n\}\) such that \(x_n \to x\) e \(y_n \to y\).
        Since \(d\) is continuous, we have
        \[
            d(x,y) = \lim d(x_n, y_n)
        \]
        since \(d(x_n, y_n) \leq \msdiameter A\), we get
        \[
            d(x,y) \leq \msdiameter A
        \]
        and by taking the supremum
        \[
            \msdiameter \overline{A} \leq \msdiameter A
        \]
    }
\end{snippetproof}

\begin{snippetexercise}{normed-spaces-ex1}{}
    Let \(A\colon (\realnumbers^2, ||\cdot||_2) \fromto (\realnumbers^3, ||\cdot||_2)\)
    be the operator between \metricspace[metric spaces]
    defined by
    \[
        A \begin{pmatrix}
            x \\ y
        \end{pmatrix}
        \triangleq \begin{pmatrix}
            3 & -2 \\ 0 & 1 \\ 1 & 3
        \end{pmatrix}
        \begin{pmatrix}
            x \\ y
        \end{pmatrix}
    \]
    Determine the operator norm \(||A||\).
\end{snippetexercise}

\begin{snippetsolution}{normed-spaces-ex1-sol}{}
    We need to find the eigenvalues of
    \[
        A^t A = \begin{pmatrix}
            3 & 0 & 1 \\ -2 & 1 & 3
        \end{pmatrix}
        \begin{pmatrix}
            3 & -2 \\ 0 & 1 \\ 1 & 3
        \end{pmatrix}
        = 
        \begin{pmatrix}
            10 & -3 \\ -3 & 14
        \end{pmatrix}
    \]
    meaning that the characteristic polynomial is
    \begin{align*}
        0 &= \lambda^2 - 24\lambda + 131 \\
        \lambda_{1,2} &= \frac{24\pm \sqrt{24^2 - 4\cdot 131}}{2} \\
        \lambda_{1,2} &= 12\pm \sqrt{13}
    \end{align*}
    and thus we obtain the value
    \[
        ||A|| = \sqrt{12 + \sqrt{13}}
    \]
\end{snippetsolution}

\begin{snippetexercise}{normed-spaces-ex2}{}
    Let \(A \in \setoflineartransformations(\realnumbers^n, \realnumbers^n)\).
    Determine whether \(||A^2|| = {||A||}^2\) with the operator norm.
\end{snippetexercise}

\begin{snippetsolution}{normed-spaces-ex2-sol}{}
    We know that
    \[
        ||A|| = \sup_{||x||_2 = 1} ||Ax||_2
    \]
    Let us show that the proposition does not hold by means of a counterexample.
    Let
    \[
        A = \begin{pmatrix}
            1 & 1 \\ 0 & 1
        \end{pmatrix}, \quad
        A^2 = \begin{pmatrix}
            1 & 1 \\ 0 & 1
        \end{pmatrix}
        \begin{pmatrix}
            1 & 1 \\ 0 & 1
        \end{pmatrix}
        = \begin{pmatrix}
            1 & 2 \\ 0 & 1
        \end{pmatrix}
    \]
    We know that \({||A||}^2\) is equal to the largest eigenvalue of \(A^t A\).
    Hence
    \[
        A^t A = \begin{pmatrix}
            1 & 0 \\ 1 & 1
        \end{pmatrix}
        \begin{pmatrix}
            1 & 1 \\ 0 & 1
        \end{pmatrix}
        = \begin{pmatrix}
            1 & 1 \\ 1 & 2
        \end{pmatrix}
    \]
    and for the eigenvalues we have \begin{align*}
        (\lambda -1)(\lambda - 2) - 1 &= 0\\
        \lambda^2 - 3\lambda + 1 &= 0 \\
        \lambda = \frac{3 \pm \sqrt{5}}{2}
    \end{align*}
    Thus \({||A||}^2 = (3 + \sqrt{5})/2\).
    On the other hand, we compute in the same way \({||A^2||}^2\) and then take the square root.
    Therefore
    \[
        (A^2)^t A^2 = \begin{pmatrix}
            1 & 0 \\ 2 & 1
        \end{pmatrix}
        \begin{pmatrix}
            1 & 2 \\ 0 & 1
        \end{pmatrix}
        = \begin{pmatrix}
            1 & 2 \\ 2 & 5
        \end{pmatrix}
    \]
    and for the eigenvalues we have \begin{align*}
        (\lambda - 1)(\lambda-5)-4 &= 0 \\
        \lambda^2 - 6\lambda + 1 &= 0 \\
        \lambda = \frac{6\pm \sqrt{32}}{2}
    \end{align*}
    and therefore
    \[
        {||A^2||} = \sqrt{
            \frac{6 + \sqrt{32}}{2}
        }
    \]
    However, \(||A^2|| \neq {||A||}^2\).
\end{snippetsolution}

\end{document}
\documentclass[preview]{standalone}

\usepackage{amsmath}
\usepackage{amssymb}
\usepackage{tikz}
\usepackage{stellar}
\usepackage{definitions}
\usepackage{bettelini}

\begin{document}

\id{series}
\genpage

\section{Divergence and convergence}

\begin{snippet}{infinite-series-convergence}
    An infinite series converges if the limit
    of its partial sum sequence also converges,
    otherwise it diverges.
\end{snippet}

\section{Properties}

\begin{snippettheorem}{infinite-series-properties}{}
    \[
        \left(
            \sum_{n=0}^\infty a_n
        \right)
        \left(
            \sum_{n=0}^\infty b_n
        \right)
        =
        \sum_{n=0}^\infty \sum_{k=0}^n a_k b_{n-k}
    \]
\end{snippettheorem}

\section{Covergence theorem}

\begin{snippettheorem}{convergence-theorem}{Convergence Theorem}
    If \(\sum a_n\) converges, then \(\lim_{n\to\infty}a_n=0\)
\end{snippettheorem}

\begin{snippetproof}{convergence-theorem-proof}{convergence-theorem}{Convergence Theorem}
    Consider the partial sum
    \[
        s_n = \sum_{k=1}^{n}a_k
    \]
    The sequence \(a_n\) can now be expressed as
    \[
        a_n = s_n - s_{n-1}
    \]
    Since \(\sum a_n\) converges, \(\lim_{n\to\infty}s_n=L\) for \(L\) finite. \\
    The limit \(\lim_{n\to\infty}s_{n-1}=L\) because \(n-1 \to \infty \text{ as } n \to \infty\).
    This implies the following
    \[
        \lim_{n \to \infty} a_n
        = \lim_{n \to \infty} s_n - s_{n-1} = L - L = 0
    \]
\end{snippetproof}

\section{Absolute and conditional convergence}

\begin{snippetdefinition}{absolute-convergence-definition}{Absolute convergence}
    A series \(\sum a_n\) is said to converge absolutely if
    \(\sum |a_n|\) converges.
\end{snippetdefinition}

\plain{This is a stronger type of convergence. Every absolutely convergent series is also convergent.}
\plain{A series that is convergent but not absolutely convergent is called conditionally convergent.}


\section{Riemann rearrangement theorem}

\begin{snippettheorem}{riemann-rearrangement-theorem}{Riemann rearranged theorem}
    If an infinite series is conditionally convergent, then its terms can be rearranged such that
    the series converges to any \(r\in \realnumbers\) or such that it diverges (to infinity or no finite value).
    If the series is absolutely convergent then any rearrangement of its terms will converge to the same value.
\end{snippettheorem}

\end{document}
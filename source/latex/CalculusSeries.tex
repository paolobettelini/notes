\documentclass[preview]{standalone}

\usepackage{amsmath}
\usepackage{amssymb}
\usepackage{tikz}
\usepackage{stellar}
\usepackage{definitions}
\usepackage{bettelini}

\begin{document}

\id{series}
\genpage

\section{Definition}

\begin{snippetdefinition}{series-definition}{Series}
    Let \(\{a_n\}_{n=k}^\infty\) be \sequence over a \field where \(k\in\integers\).
    The \(N\)-th \emph{partial sum} is the \sequence
    \[
        \{S_N\}_{N=k}^\infty \triangleq \sum_{n=k}^\infty a_n, \quad
        S_N \triangleq \sum_{n=k}^N a_n
    \]
    The \emph{series of general terms \(\{a_n\}_{n=k}^\infty\)} is defined as the sum
    \[
        \sum_{n=k}^\infty a_n
    \]
    We say that
    \[
        \sum_{n=k}^\infty a_n = S
    \]
    with \(S=\lim S_N\) if the limit exists:
    \begin{enumerate}
        \item the series \emph{converges} if \(\exists \lim S_N\) and \(S\in\realnumbers\);
        \item the series \emph{diverges} if \(\exists \lim S_N\) and \(S=\pm\infty\);
        \item the series \emph{oscillates} if \(\nexists \lim S_N\).
    \end{enumerate}
\end{snippetdefinition}

\plain{Note that this is a bit of abuse of notation; the same symbol represents both the sequence of partial sums
and its limit.}

\section{Covergence theorem}

\begin{snippettheorem}{series-convergence-theorem}{Convergence Theorem}
    Let \(\{a_n\}_{n=k}^\infty\) be \sequence.
    Then, if \[\sum_{n=k}^\infty a_n\] \seriesconverges
    we have that \[\lim a_n=0\]
\end{snippettheorem}

\begin{snippetproof}{series-convergence-theorem-proof}{convergence-theorem}{Convergence Theorem}
    Consider the \partialsum
    \[
        S_N = \sum_{k=1}^{M}a_k
    \]
    The sequence \(a_n\) can now be expressed as
    \[
        a_n = S_N - S_{N-1}
    \]
    Since \(\sum a_n\) converges, \(\lim S_N=L\) for \(L\) finite. \\
    The limit \(\lim S_{N-1}=L\) because \(N-1 \to \infty \text{ as } N \to \infty\).
    This implies the following
    \[
        \lim a_n = \lim S_N - S_{N-1} = L - L = 0
    \]
\end{snippetproof}

\section{Series arithmetic}

\begin{snippetproposition}{series-arithmetic}{Series arithmetic}
    Let \(\sum_{n=k}^\infty a_n = A\) and \(\sum_{n=k}^\infty b_n = B\) be \series
    where \(A,B\in\extendedrealnumbers\). Then,
    \begin{enumerate}
        \item the \series \[ \sum_{n=k}^\infty \left(a_n+b_n\right) = A+B \]
        as long as \(A+B\neq \pm\infty \mp\infty\);
        \item the \series \[ \sum_{n=k}^\infty ca_n = \begin{cases} cA & c\neq 0 \\ 0 & c=0 \end{cases} \]
            for \(c\in\realnumbers\);
        \item as long as \(A+B\neq \pm\infty \mp\infty\),
            \[
                \sum_{n=k}^\infty (a_n+b_n) = \sum_{n=k}^\infty a_n + \sum_{n=k}^\infty b_n
            \]
            and as long as \(c\neq 0\)
            \[
                c\sum_{n=k}^\infty a_n = \sum_{n=k}^\infty c_an
            \]
    \end{enumerate}
\end{snippetproposition}

\begin{snippettheorem}{tail-of-a-series-theorem}{Tail of a series}
    Let \(\sum_{n=k}^\infty a_n\) and \(\sum_{n=k}^\infty b_n\) be \series
    such that \(a_n = b_h\) \eventually.
    Then, the behavior of the two series are the same.
\end{snippettheorem}

\plain{This means that modifying a finte amount of terms, does not alter the behavior of the series.}

\begin{snippetproof}{tail-of-a-series-proof}{tail-of-a-series}{Tail of a series}
    Assume that \(\forall N>n_0, a_n = b_n\).
    We can write the \partialsum
    \[
        A_N = \sum_{n=k}^N a_n = \sum_{n=k}^{n_0 - 1}a_n
        + \sum_{n=n_0}^N a_n
    \]
    and likewise
    \[
        B_N = \sum_{n=k}^N b_n = \sum_{n=k}^{n_0 - 1}b_n
        + \sum_{n=n_0}^N b_n
    \]
    However, since \(a_n = b_n\) \eventually,
    \[
        B_N = \sum_{n=k}^N b_n = \sum_{n=k}^{n_0 - 1}b_n
        + \sum_{n=n_0}^N a_n
    \]
    We now note that
    \[ \sum_{n=k}^{n_0 - 1}b_n \]
    is a finite value, and thus
    \[
        \exists \lim A_n
        \iff
        \exists \lim \sum_{n=n_0}^N a_n
        \iff
        \exists \lim B_n
    \]
\end{snippetproof}

\begin{snippettheorem}{positive-series-convergence-theorem}{Positive series convergence}
    Let \[ \sum_{n=k}^\infty a_n \]
    be a \series where \(a_n \geq 0\). Then, the \series either \seriesdiverges
    or \seriesconverges. In particular, the \series \seriesconverges \ifandonlyif the \partialsum
    \sequence \(\{S_N\}\) is \upperbound[bounded above].
    If \(a_n \geq 0\) only \eventually for \(\forall n>n_0\),
    the conclusion is the same with
    \[
        S = S_{n_0} + \sup_n
        \sum_{n=n_0}^N a_n
    \]
\end{snippettheorem}

\plain{A series where the terms are eventually non-negative cannot oscillate.}

\begin{snippetproof}{positive-series-convergence-theorem-proof}{positive-series-convergence-theorem}{Positive series convergence}
    Suppose that \(\forall n, a_n \geq 0\). Then,
    \[
        S_{N+1} = \sum_{n=1}^{N+1} a_n = S_n + a_{N+1} \geq S_N
    \]
    the \sequence \(\{S_N\}\) is monotone decreasing and thus % TODOURGENT link
    \[ \exists\, \lim_n S_n = \sup_n S_n \]
    In the case for which \(a_n \geq 0\) only for \(n\geq n_0\)
    \[ S_N = \sum_{j=1}^{n_0} a_j + \sum_{j=n_0 + 1}^N a_j \]
    where the second term is monotone decreasing in \(N\).
\end{snippetproof}

\section{Absolute and conditional convergence}

\begin{snippetdefinition}{absolute-convergence-definition}{Absolute convergence}
    A \series \[\sum_{n=k}^\infty a_n\] is said to \emph{converge absolutely} if
    \[\sum_{n=k}^\infty |a_n|\] \seriesconverges.
\end{snippetdefinition}

\plain{This is a stronger type of convergence. Every absolutely convergent series is also convergent.
A series that is convergent but not absolutely convergent is called conditionally convergent.}

\begin{snippettheorem}{absolute-convergence-implies-convergence-theorem}{Absolute convergence implies convergence}
    A \series that \convergesabsolutely, \seriesconverges.
\end{snippettheorem}

\section{Riemann rearrangement theorem}

\begin{snippettheorem}{riemann-rearrangement-theorem}{Riemann rearranged theorem}
    If an infinite series is conditionally convergent, then its terms can be rearranged such that
    the series converges to any \(r\in \realnumbers\) or such that it diverges (to infinity or no finite value).
    If the series is absolutely convergent then any rearrangement of its terms will converge to the same value.
\end{snippettheorem}

\section{Product of series}

\begin{snippettheorem}{infinite-series-properties}{}
    Let \(\{a_n\}_{n=0}^\infty, \{b_n\}_{n=0}^\infty\) be \sequence[sequences].
    Then,
    \[
        \left(
            \sum_{n=0}^\infty a_n
        \right)
        \left(
            \sum_{n=0}^\infty b_n
        \right)
        =
        \sum_{n=0}^\infty \sum_{k=0}^n a_k b_{n-k}
    \]
\end{snippettheorem}

\begin{snippetproposition}{nth-root-of-n-limit}{}
    \[ \lim_{n \to \infty} n^{\frac{1}{n}} = 1 \]
\end{snippetproposition}

\end{document}
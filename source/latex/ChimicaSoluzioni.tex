\documentclass[preview]{standalone}

\usepackage{amsmath}
\usepackage{amssymb}
\usepackage{bettelini}
\usepackage{stellar}

\hypersetup{
    colorlinks=true,
    linkcolor=black,
    urlcolor=blue,
    pdftitle={Chimica},
    pdfpagemode=FullScreen,
}

\begin{document}

\id{chimica-soluzioni}
\genpage

\begin{snippetdefinition}{solubilita-definizioe}{Solubilità}
    La \textit{solubilità} rappresenta la concentrazione massima di un soluto
    in una soluzione in un certo solvente a temperatura costante.
\end{snippetdefinition}

\begin{snippetdefinition}{saturazione-definizioe}{Saturazione}
    Una soluzione è detta \textit{satura} o \textit{insatura}
    se ha raggiunto il suo quantitativo massimo di soluto disciolto o meno.
\end{snippetdefinition}

\begin{snippet}{solubilita-expl}
    Quando tentiamo di sciogliere una sostanza in un solvente in quantità maggiore
    di quella data dalla sua solubilità, una parte si scioglie e una parte precipita,
    depositandosi come \textit{corpo di fondo}.

    La solubilità dipende dalle proprietà chimica e altri fattori come la temperatura. \\
    La solubilità dei gas diminuisce con l'aumento della temperatura.

    Quando un soluto viene sciolto in un solvente, il volume della soluzione aumenta,
    ma meno della somma dei due volumi. Questo è dato dal fatto che il soluto prende spazio fra le molecole del solvente.
\end{snippet}

\begin{snippetdefinition}{concentrazione-soluzione-definition}{Concentrazione soluzione}
    La \textit{concentrazione} di una soluzione è il rapporto
    fra la quantità di soluto e la quantità di solvente (o soluzione).
\end{snippetdefinition}

\begin{snippet}{concentrazioni-soluzione-expl}
    Le concentrazioni vengono spesse espresse come:
    \begin{itemize}
        \item \textbf{percentuale in massa: }
        \[\%\frac{m}{m} = \frac{n_{\text{soluto}}}{m_{\text{soluzione}}}\cdot100\]
    
        \item \textbf{percentuale in volume: }
        \[\%\frac{V}{V} = \frac{V_{\text{soluto}}}{V_{\text{soluzione}}}\cdot100\]
    
        \item \textbf{percentuale in massa su volume: }
        \[\%\frac{m}{V} = \frac{m_{\text{soluto}}}{V_{\text{soluzione}}}\cdot100\]
    \end{itemize}
\end{snippet}

\begin{snippetdefinition}{legge-di-henry-definition}{Legge di Henry}
    La quantità di gas che si scioglie in un dato volume di liquido, a una data temperatura, è direttamente
    proporzionale alla pressione di quello stesso gas presente sopra la superficie del liquido.
\end{snippetdefinition}

\end{document}

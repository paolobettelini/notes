\documentclass[preview]{standalone}

\usepackage{amsmath}
\usepackage{amssymb}
\usepackage{stellar}
\usepackage{definitions}
\usepackage{tikz}
\usepackage{bettelini}
\usepackage{adjustbox}

\usetikzlibrary{cd}

\begin{document}

\id{categorytheory-universal-properties}
\genpage

\section{Universal Construction}

\begin{snippetdefinition}{universal-construction-defintion}{Universal Construction}
    Universal construction is used to define objects in terms of their
    relationships up to a unique isomorphism.
    This means that the objects that satisfy the same universal property
    are isomorphic.
    Its purpose is to define an object without knowledge about the object itself,
    but rather just the morphisms.
\end{snippetdefinition}

\section{Product}

\begin{snippet}{categorial-product-universal-construction}
The categorical product represents many operations
such as the cartesian product of sets.
A product of two objects \(a\) and \(b\) has the following
morphisms:
\begin{enumerate}
    \item \(p \colon a \cartesianprod b \fromto a\), which returns the first value of the ordered pair
    \item \(q \colon a \cartesianprod b \fromto b\), which returns the second value of the ordered pair
\end{enumerate}

In the following diagram \(c\) is the actual product and \(c'\) is a candidate object
for the cardinal product.
We rank a candidate \(c_i\) higher than \(c_j\) if there is a morphism
\(m \colon c_i \fromto c_j\).

\begin{minipage}{0.5\textwidth}
    \adjustbox{scale=1.5,center}{%
    \begin{tikzcd}
        & c' \arrow[d, "m"] \arrow[rdd, "q'", bend left] \arrow[ldd, "p'"', bend right] &   \\
        & c \arrow[ld, "p"] \arrow[rd, "q"']                                         &   \\
      a &                                                                                  & b
    \end{tikzcd}
    }
\end{minipage}
\begin{minipage}{0.5\textwidth}
    \begin{align*}
        p' &= p \circ m \\
        q' &= q \circ m 
    \end{align*}
\end{minipage}

Whenever \(m\) is \textit{flawed}, such as when it loses information
or does not preserve structure, we discard \(c_i\).

The universal property is that for any other product
\(c'\) with morphisms \(p'\) and \(q'\),
s a unique morphism
\[
    m \colon c'\fromto c
\]
such that
\begin{align*}
    p \circ m &= p' \\
    q \circ m &= q'
\end{align*}
\end{snippet}

\subsection{Coproduct}

\begin{snippet}{categorial-coproduct-universal-construction}
The coproduct represents operations
such as the disjoint union of sets.
The coproduct is the product when the morphisms are inverted.
The coproduct of two objects \(a\) and \(b\) has the following
morphisms:
\begin{enumerate}
    \item \(p \colon a \fromto a\sqcup b\)
    \item \(q \colon b \fromto a\sqcup b\)
\end{enumerate}

Let \(c=a\sqcup b\).

\begin{minipage}{0.5\textwidth}
    \adjustbox{scale=1.5,center}{%
    \begin{tikzcd}
        & c'                &                                                  \\
        & c \arrow[u, "m"'] &                                                  \\
        a \arrow[ruu, "p'", bend left] \arrow[ru, "p"'] &                   & b \arrow[luu, "q'"', bend right] \arrow[lu, "q"]
    \end{tikzcd}
    }
\end{minipage}
\begin{minipage}{0.5\textwidth}
    \begin{align*}
        p' &= m \circ p \\
        q' &= m \circ q
    \end{align*}
\end{minipage}

The universal property is that for any other object \(c'\)
with morphisms \(p' \colon a \fromto c'\) and \(q' \colon b \fromto c'\),
there exists a unique morphism \(m \colon c \fromto c'\)
such that
\begin{align*}
    p' &= m \circ p \\
    q' &= m \circ q
\end{align*}
\end{snippet}

\end{document}
\documentclass[preview]{standalone}

\usepackage{amsmath}
\usepackage{amssymb}
\usepackage{stellar}
\usepackage{definitions}
\usepackage{tikz}
\usepackage{bettelini}
\usepackage{adjustbox}

\usetikzlibrary{cd}

\begin{document}

\id{categorytheory-universal-properties}
\genpage

\section{Universal Construction}

\begin{snippet}{universal-properties-introduction}
    It is a striking fact that one can often define mathematical
    objects not by means of their internal structure (that is, as in
    the classical spirit of set-theoretic foundations) bur rather in
    terms of their relations with the other objects of the
    mathematical environment in which one works (that is, in
    terms of the objects and arrows of the category in which one
    works), by means of so-called universal properties.

    Of course, isomorphic objects in a category are
    indistinguishable from the point of view of the categorical
    properties that they satisfy; in fact, definitions via universal
    property do not determine the relevant objects ``absolutely''
    but only up to isomorphism in the given category.

    Universal construction defines notions without knowledge about the object itself,
    but rather just its relations with other objects.
\end{snippet}

\end{document}
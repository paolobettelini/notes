\documentclass[preview]{standalone}

\usepackage{amsmath}
\usepackage{amssymb}
\usepackage{stellar}
\usepackage{definitions}

\begin{document}

\id{rings-introduction}
\genpage

\section{Definition}

\begin{snippetdefinition}{ring-definition}{Ring}
    A \textit{ring} \((R, +, \circ)\) is a triple containing a \set \(R\) and two \binoperation[binary operations]
    \(+\) and \(\circ\) on \(R\) such that:
    \begin{enumerate}
        \item \((R, +)\) is an \abeliangroup;
        \item \((R, \circ)\) is a \monoid;
        \item \textit{left distributivity}: \(\forall a,b,c\in R, a\circ(b+c) = (a\circ b) + (a \circ c)\);
        \item \textit{right distributivity}: \(\forall a,b,c\in R, (b+c)\circ a = (b\circ a) + (c \circ a)\).
    \end{enumerate}
\end{snippetdefinition}

\begin{snippetdefinition}{commutative-ring-definition}{Commutative ring}
    A \ring \((R, +, \circ)\) is said to be \textit{commutative} if:
    \begin{enumerate}
        \item \textit{distributivity}: \(\forall a,b\in R, a\circ b = b\circ a\).
    \end{enumerate}
\end{snippetdefinition}

\begin{snippetproposition}{ring-zero-multiplication}{}
    Let \(A = (R, +, \circ)\) be a \ring. Then, \[
        a \circ 0_A = 0_A, \quad \forall a \in R
    \]
\end{snippetproposition}

\begin{snippetproof}{ring-zero-multiplication-proof}{ring-zero-multiplication}{}
    We first note that \(0_A = 0_A + 0_A\)
    and that \(0_A \circ a = a \circ 0_A = 0_A\).
    From this we get \(a \circ 0_A = a \circ 0_A + a \circ 0_A\)
    and by subtracting we get \(0_A \circ a = 0_A\).
    Likewise, we find that \(a \circ 0_A = 0_A\).
\end{snippetproof}

\begin{snippetproposition}{ring-inverse-addition-multiplication}{}
    Let \(A = (R, +, \circ)\) be a \ring. Then, \[
        - a = (- 1_A) \circ a = a \circ (- 1_A), \quad \forall a \in R
    \]
\end{snippetproposition}

\begin{snippetproof}{ring-inverse-addition-multiplication-proof}{ring-inverse-addition-multiplication}{}
    We have
    \begin{align*}
        1_A + (-1_A) &= 0_A \\
        0_A &= a \circ 0_A = a \circ (1_A + (1_A)) = a \circ 1_A + a \circ (-1_A) \\
        &= a + a \circ (-1_A) \\
        -a &= a \circ (-1_A)
    \end{align*}
    Likewise, we find \(-a = (-1_A) \circ a\).
\end{snippetproof}

\plain{We can define the multiplication by an integer as repeated addition.}

\begin{snippetproposition}{ring-multiplication-as-addition}{}
    Let \(A = (R, +, \circ)\) be a \ring. Then, \[
        n(ab) = (ab)n = a(nb)
    \]
    and
    \[
        (-n) \circ a = -(n \circ a)
    \]
\end{snippetproposition}

\begin{snippetproof}{ring-multiplication-as-addition-proof}{ring-multiplication-as-addition}{}
    By induction:
    \begin{enumerate}
        \item the base case is \(n=2\),
        \[
            2(ab) = ab + ab = \begin{cases}
                a(b+b) = a(2b) \\
                (a+a)b = (2a)b
            \end{cases}
        \]
    \end{enumerate}
\end{snippetproof}

\plain{Using these properties we can prove prove that a negative times a negative is a positive.}

\begin{snippetdefinition}{ring-zero-divisor-definition}{Zero divisor}
    Let \(A = (R, +, \circ)\) be a \ring and \(a \in R\) such that \(a \neq 0_A\).
    We say that \(a\) is a \emph{left or right zero divisor} if
    \[
        \exists b \in R \suchthat b \neq 0_A \land a \circ b = 0_A
    \]
    or
    \[
        \exists b \in R \suchthat b \neq 0_A \land b \circ a = 0_A
    \]
    respectively.
\end{snippetdefinition}

\begin{snippetdefinition}{integral-domain-definition}{Integral domain}
    Let \(A = (R, +, \circ)\) be a commutative \ring.
    Then, \(A\) is said to be an \emph{integral domain}
    if it does not contain any zero divisor.
\end{snippetdefinition}

\begin{snippetproposition}{zero-divisors-are-not-invertible}{Zero divisors are not invertible}
    Let \(A = (R, +, \circ)\) be a \ring and \(a \in R\)
    a zero divisor. Then, \(a\) is not invertible.
\end{snippetproposition}

\begin{snippetproof}{zero-divisors-are-not-invertible-proof}{zero-divisors-are-not-invertible}{zero divisors are not invertible}
    There exist a non-null \(b \in R\) such that \(ab = 0_A\).
    Suppose that \(a^{-1} \in A\). We then have
    \begin{align*}
        a^{-1} \circ a \circ b &= 1_A \circ b = b = 0_A \circ a^{-1} = 0_A
    \end{align*}
    which is absurd \lightning.
\end{snippetproof}

\begin{snippetproposition}{zero-divisors-simplification}{}
    Let \(A = (R, +, \circ)\) be a \ring and \(a \in R\)
    such that \(a\neq 0_A\) is not a left/right zero-divisor. Then,
    \[
        a \circ b = a \circ c \implies b = c
    \]
    and
    \[
        b \circ a = c \circ a \implies b = c
    \]
    respectively.
\end{snippetproposition}

\begin{snippetproof}{zero-divisors-simplification-proof}{zero-divisors-simplification}{}
    We will prove the left case
    \begin{align*}
        ab &= ac \\
        ab + (-ac) &= 0_A \\
        a\circ (b-c) &= 0_A
    \end{align*}
    Since there are no zero-divisors, either \(a\) or \(b-c\) is null.
    Since \(a \neq 0\), \(b-c = 0_A\).
\end{snippetproof}

\begin{snippetdefinition}{subring-definition}{Subring}
    Let \(A = (R, +, \circ)\) be a \ring.
    A \emph{subring} of \(A\) is a subset \(S \subseteq R\)
    such that \((S, +, \circ)\) is a \ring.
\end{snippetdefinition}

\begin{snippetdefinition}{ring-ideal-definition}{Ideal}
    Let \(A = (R, +, \circ)\) be a \ring.
    A \emph{left/right ideal} of \(A\) is a subset \(I \subseteq R\)
    such that:
    \begin{enumerate}
        \item \((I, +)\) is a \subgroup of \((R, +)\);
        \item \(\forall i \in I, \forall r \in R\), we have \(ir \in I\) (left ideal) and \(ri \in I\) (right ideal).
    \end{enumerate}
    If the ideal is both a left and right ideal, it is called a
    \emph{two-sided ideal}, \emph{bilateral} or simply an \emph{ideal}.
\end{snippetdefinition}

\end{document}
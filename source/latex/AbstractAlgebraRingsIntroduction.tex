\documentclass[preview]{standalone}

\usepackage{amsmath}
\usepackage{amssymb}
\usepackage{stellar}
\usepackage{definitions}

\begin{document}

\id{rings-introduction}
\genpage

\section{Definition}

\begin{snippetdefinition}{ring-definition}{Ring}
    A \textit{ring} \((R, +, \circ)\) is a triple containing a \set \(R\) and two \binoperation[binary operations]
    \(+\) and \(\circ\) on \(R\) such that:
    \begin{enumerate}
        \item \((R, +)\) is an \abeliangroup;
        \item \((R, \circ)\) is a \monoid;
        \item \textit{left distributivity}: \(\forall a,b,c\in R, a\circ(b+c) = (a\circ b) + (a \circ c)\);
        \item \textit{right distributivity}: \(\forall a,b,c\in R, (b+c)\circ a = (b\circ a) + (c \circ a)\).
    \end{enumerate}
\end{snippetdefinition}

\begin{snippetdefinition}{commutative-ring-definition}{Commutative ring}
    A \ring \((R, +, \circ)\) is said to be \textit{commutative} if:
    \begin{enumerate}
        \item \textit{distributivity}: \(\forall a,b\in R, a\circ b = b\circ a\).
    \end{enumerate}
\end{snippetdefinition}

\end{document}
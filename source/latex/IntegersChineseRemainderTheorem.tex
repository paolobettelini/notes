\documentclass[preview]{standalone}

\usepackage{amsmath}
\usepackage{amssymb}
\usepackage{stellar}
\usepackage{definitions}

\begin{document}

\id{chinese-remainder-theorem}
\genpage

\section{Chinese remainder theorem}

\begin{snippet}{temp}
    Consideriamo le equazioni in \(\mathbb{Z} / n\) e congruenze con incognite.
    Supponiamo di avere una congruenza con incognita del tipo \(a x \equiv b \pmod{n}\)
    con \(a,b\) interi fissati e \(n\) inteor positivo fissato.

    Se \(a\) è \invertiblecongclass[invertible] modulo \(n\), cioè se esiste \(c\) tale che
    \(ac \equiv 1 \pmod{n}\), possiamo risolvere la congruenza
    come faremmo per un'equazione di primo grado nei razionali, cioè "dividendo" per \(a\),
    o meglio, moltiplicando per \(c\), e troviamo \(x \equiv bc \pmod{n}\).

    Ad esempio, \(3x \equiv 5 \pmod{20}\), poiché \(3\) è coprimo con \(20\),
    esiste un intero \(c\) tale che \(3c \equiv 1 \pmod{20}\).
    Troviamo quindi facilmente per esempio \(c=7\), e abbiamo allora
    \[
        3\cdot7x \equiv 5\cdot 7 \pmod{20}
    \]
    vale a dire che \(x\equiv 35 \pmod{20} \equiv 15 \pmod{20}\)

    In maniera equivalente possiamo considerare l'equazione in \(\mathbb{Z} / n\),
    \({[a]}_n x = {[b]}_n\) dove cerchiamo in \(x\) in \(\mathbb{Z} / n\).
    dove cerchiamo \(x\) in \(\mathbb{Z} / n\).
    Se come sopra \(a\) è coprimo con \(n\), ed esiste, quindi
    \({[c]}_n\) tale che \({[a]}_n{[c]}_n = {[1]}_n\), l'equazione dat aha come unica soluzione
    \[
        x = {[b]}_n{[c]}_n = {[bc]}_n
    \]

    Nel nostro esempio abbiamo che \({[3]}_{20} x = {[5]_{20}}\) ha un'unica soluzione
    \(x={[15]}_{20}\).
    Nota: la congruenza con incognite ha infinite soluzioni, tutti gli interi 
    congrui a \(15\) modulo \(20\), l'equazione in \(\mathbb{Z} / 20\) ha un'unica soluzione.

    In the general case we find to solve
    \[
        ax\equiv b \pmod{n} \qquad \text{and} \qquad {[a]}_n x = {[b]}_n
    \]
    We are looking for an \(x\) such that \(ax-b\) sia multiplo di  \(n\), cioè
    \(ax-b = ny\) per qualche \(y\) intero.
    This is a diophantine equation
    \[
        ax-ny = b
    \]
    which has solutions \ifandonlyif \(\gcd(a,n) \divides b\)
\end{snippet}

\begin{snippetexample}{chinese-remainder-example-1}{}
    Consider
    \[
        6x \equiv 5 \pmod{14}
    \]
    and, equivalently \[
        {[6]}_{14} x = {[5]}_{14}
    \]
    Since \(\gcd(6,14) = 2\) does not divide \(5\) the congruence
    and the corresponding equation have no solution.
\end{snippetexample}

\begin{snippetexample}{chinese-remainder-example-2}{}
    Consider
    \[
        6x \equiv 10 \pmod{14}
    \]
    and, equivalently \[
        {[6]}_{14} x = {[10]}_{14}
    \]
    Since \(\gcd(6,14) = 2 \divides 10\), the equation ha ssolutions
    which we can find by solving the diophantine equation
    \[
        6x - 14y = 10 \iff 3x - 7y = 5
    \]
    A Bezout's identity between \(3\) and \(7\) is given by
    \(3\cdot (-2) + 7 \cdot 1 = 1\).
    By multiplying by \(5\) we get \(3\cdot (-10) + 7 \cdot 5 = 5\),
    so a particular solution is given by \(x=-10\) and \(y = -5\)
    and the general one is \(x=-10 + 7h\) and \(y-5+3h\).
    This means that the congruence has solutions
    \(x \equiv -10 \pmod{7}\) (when I multiply it by \(6\) I get a number modulo \(14\))
    and the corresponding equation has as solutions the classes modulo \(14\)
    represented by integers congruence to \(-10\) modulo 3.
    \[
        x = {[-10]}_{14} = {[4]}_{14}
    \]
    and
    \[
        x = {[-3]}_{14}
    \]
    The amount of solutions is thus \(2 = \gcd(6, 14)\).
\end{snippetexample}

% questa cosa qua è data dal fatto che diviso l'equazione per d

\begin{snippetproposition}{temp2}{}
    In generale, se considero la congruenza \(ax \equiv b \pmod{n}\)
    e l'equazione \({[a]}_n x = {[b]}_n\) e \(d = \gcd(a,b) \divides n\),
    allora la congruenza è risolubile ed equivalente a
    \[
        a'x \equiv b' \pmod{n'}
    \]
    con \(a=a'd\), \(b = b'd\), \(n=n'd\).
    Tutte le classi rappresentate da interi
    che soddisfano la congruenza data, sono soluzioni
    dell'equazione data in \(\mathbb{Z} / n\).
    Se \(c\) è una soluzione particolare della congruenza,
    abbiamo queste classi
    \[
        {[c]}_n, {[c + n']}_n, {[c + 2n']}_n, \cdots, {[c + (d-1)n']}_n
    \]
    Notiamo che queste classi sono distinte (le differenza fra i loro rappresentanti
    non sono multipli di \(n\)) e ogni altra classe del tipo
    \({[c+kn']}_n\) compare tra queste.
    Più precisamente \({[c+kn']}_n + {[c+rn']}_n\) con \(r\) resto della divisione
    di \(k\) per \(d\).
\end{snippetproposition}

\begin{snippettheorem}{chinese-remainder-theorem}{Chinese remainder theorem}
    Consider the system of congruences
    \[
        \begin{cases}
            x \equiv a_1 \pmod{n_1} \\
            x \equiv a_2 \pmod{n_2} \\
            \cdots \\
            x \equiv a_r \pmod{n_r}
        \end{cases}
    \]
    with \(a_i\in\integers\) and \(n_i \in\integers^+\)
    where \(n_i, n_{i+1}\) are \coprime.
    Then, the given system is equivalent to a congruence
    of the form \(x \equiv a \pmod{n}\)
    for some \(a\) and
    \[
        n = \prod_{i=1}^r n_i
    \]
    
\end{snippettheorem}

\begin{snippetproof}{chinese-remainder-theorem-proof}{chinese-remainder-theorem}{Chinese remainder theorem}
    \begin{itemize}
        \item The base case \(r=1\) is trivial (the congruence is equivalent to itself).
        \item Let \(r > 1\). We consider the first two congruences
        \[
            \begin{cases}
                x \equiv a_1 \pmod{n_1} \\
                x \equiv a_2 \pmod{n_2} \\
            \end{cases}
        \]
        We need to determine the integers \(x\) such that
        \(x = a_1 + n_1h\) for some \(h\in\integers\) and
        \(x = a_2 + n_2k\) for some \(k\in\integers\).
        We thus get
        \[
            a_1 + n_1 h = a_2 + n_2 k
        \]
        which is the diophantine equation
        \[
            n_1h - n_2 k = a_2 - a_1
        \]
        Since \(n_1\) and \(n_2\) are \coprime, \(\gcd(n_1, n_2) = 1 \divides a_2 - a_1\)
        and thus solutions exist.
        However, we want to show that the solutions of the system
        \[
            \begin{cases}
                x \equiv a_1 \pmod{n_1} \\
                x \equiv a_2 \pmod{n_2} \\
            \end{cases}
        \]
        are only the ones of type \(a + tn_1n_2\)
        with a particular solution (determined as done above)
        with \(t\in\integers\).
        In order to achieve this, we consider the following facts:
        \begin{enumerate}
            \item every integers of type \(a+tn_1n_2\) is a solution.
            Indeed, \(a+tn_1n_2 \equiv a \pmod{n_1} \equiv a_1 \pmod{n_1}\)
            and \(a+tn_1n_2 \equiv a \pmod{n_2} \equiv a_2 \pmod{n_2}\);
            \item every solution \(b\) of the system has form \(b = a + tn_1n_2\)
            for some \(t\in\integers\).
            Indeed, \(b\equiv a_1 \pmod{n_1}\), \(a \equiv a_1 \pmod{n_1}\)
            and thus \(b \equiv a \pmod{n_1}\) which means \(b-a\)
            is a multiple of \(n_1\).
            Likewise, \(b-a\) is a multiple of \(n_2\).
            It follows that \(b-a\) is a multiple of \(n_1\) and \(n_2\), which are
            \coprime, and \(b-a = tn_1n_2\).
        \end{enumerate}
        We just showed that the first two congruences are equivalent to a congruence of type
        \[
            x \equiv a \pmod{n_1n_2}
        \]
        Thus, the system is reduced to
        \[
            \begin{cases}
                x \equiv a_1 \pmod{n_1n_2} \\
                x \equiv a_3 \pmod{n_3} \\
                \cdots \\
                x \equiv a_r \pmod{n_r}
            \end{cases}
        \]
        We iterate this procedure to obtain the thesis.
    \end{itemize}
\end{snippetproof}

\end{document}
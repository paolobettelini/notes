\documentclass[preview]{standalone}

\usepackage{amsmath}
\usepackage{amssymb}
\usepackage{stellar}
\usepackage{definitions}

\def\setX{\blue X \clear}
\def\setT{\teal \mathcal{T} \clear}
\def\ts{(\setX, \setT)}

\begin{document}

\id{seconda-countable-spaces}
\genpage

\section{Basis}

\begin{snippetdefinition}{topological-space-basis-definition}{Basis of topological space}
    \def\tbasis{\scolor[orange] \mathcal{B} \clear}
    \def\Ai{\scolor[orange!75] A_{\gray i} \clear}
    \def\setU{\scolor[teal!75] U \clear}
    \def\indexes{\gray i\in I \clear}
    Let \(\ts\) be a \topologicalspace.
    A collection of \topologicalspace[open sets] \(\tbasis \subseteq \setT\)
    is called a \textit{basis} (\textit{base}) of \(\setT\) if for all \(\setU\in \setT\)
    there is a sequence \({\{\Ai\}}_{\indexes}\) with \(\Ai \in \tbasis\) and
    \[ \bigcup_{\indexes} \Ai=\setU \]
\end{snippetdefinition}

\begin{snippetproposition}{topology-is-basis}{Topology is always a basis}
    Let \((X, \mathcal{T})\) be a \topologicalspace.
    Then, \(\mathcal{B} = \mathcal{T}\) is a \topologicalbasis of \(\mathcal{T}\).
\end{snippetproposition}

\begin{snippetproposition}{basis-of-discrete-topology}{Basis of discrete topology}
    Let \((X, \powerset(X))\) be a \topologicalspace.
    Then, \(\mathcal{B} = \left\{\{x\} \suchthat x\in X\right\}\) is a \topologicalbasis of \(\mathcal{T}\).
\end{snippetproposition}

\begin{snippetproposition}{topology-induced-by-metric-basis}{Basis of topology induced by metric space}
    Let \((X,d)\) be a \metricspace and \((X, \metrictopology_d)\)
    be the \topologicalspace induced by \((X,d)\).
    Then, \[\mathcal{B} = \left\{\ball_\epsilon(x) \suchthat x\in X, \epsilon>0\right\}\]
    is a \topologicalbasis of \(\mathcal{T}\).
\end{snippetproposition}

\section{Second-countable space}

\begin{snippetdefinition}{second-countable-space-definition}{Second-countable space}
    A \topologicalspace \((X, \mathcal{T})\) is called \textit{second-countable} if there exists a
    \countable \topologicalbasis of \(\mathcal{T}\).
\end{snippetdefinition}

\end{document}
\documentclass[preview]{standalone}

\usepackage{amsmath}
\usepackage{amssymb}
\usepackage{stellar}
\usepackage{definitions}

\begin{document}

\id{settheory-orders}
\genpage

\section{Types of orders}

\subsection{Reflexive}

\begin{snippetdefinition}{preorder-definition}{Preorder order}
    A \textit{preorder} is a \homrelation \(\leq\) on a \set \(A\)
    with the following properties:
    \begin{enumerate}
        \item \textit{Reflexive}: \(\forall a \in A, a \leq a\)
        \item \textit{Transitive}: \(\forall a,b,c \in A, a \leq b \land b \leq c \implies a \leq c\)
    \end{enumerate}
\end{snippetdefinition}

\begin{snippetdefinition}{partial-order-definition}{Partial order}
    A \textit{partial order} is a \homrelation \(\leq\) on a \set \(A\)
    with the following properties:
    \begin{enumerate}
        \item \textit{Reflexive}: \(\forall a \in A, a \leq a\)
        \item \textit{Transitive}: \(\forall a,b,c \in A, a \leq b \land b \leq c \implies a \leq c\)
        \item \textit{Antisymmetric}: \(\forall a,b \in A, a \leq b \land b \leq a \implies a=b\)
    \end{enumerate}
\end{snippetdefinition}

\begin{snippetdefinition}{total-order-definition}{Total order}
    A \textit{total order} is a \homrelation \(\leq\) on a \set \(A\)
    with the following properties:
    \begin{enumerate}
        \item \textit{Reflexive}: \(\forall a \in A, a \leq a\)
        \item \textit{Transitive}: \(\forall a,b,c \in A, a \leq b \land b \leq c \implies a \leq c\)
        \item \textit{Antisymmetric}: \(\forall a,b \in A, a \leq b \land b \leq a \implies a=b\)
        \item \textit{Strongly connected} (or \textit{total}): \(\forall a,b\in A, a \leq b \lor b\leq a\)
    \end{enumerate}
\end{snippetdefinition}

\begin{snippet}{settheory-3}
    A total order is a partial order where any two elements are comparable.
\end{snippet}

\subsection{Irreflexive}

\begin{snippetdefinition}{strict-preorder-definition}{Strict preorder order}
    A \textit{strict preorder} is a \homrelation \(<\) on a \set \(A\)
    with the following properties:
    \begin{enumerate}
        \item \textit{Irreflexive}: \(\forall a \in A, \lnot (a < a)\)
        \item \textit{Transitive}: \(\forall a,b,c \in A, a < b \land b < c \implies a < c\)
    \end{enumerate}
\end{snippetdefinition}

\begin{snippetdefinition}{strict-partial-order-definition}{Strict partial order}
    A \textit{strict partial order} is a \homrelation \(<\) on a \set \(A\)
    with the following properties:
    \begin{enumerate}
        \item \textit{Irreflexive}: \(\forall a \in A, \lnot (a < a)\)
        \item \textit{Transitive}: \(\forall a,b,c \in A, a < b \land b < c \implies a < c\)
        \item \textit{Antisymmetric}: \(\forall a,b \in A, a < b \land b < a \implies a=b\)
    \end{enumerate}
\end{snippetdefinition}

\begin{snippetdefinition}{strict-total-order-definition}{String total order}
    A \textit{strict total order} is a \homrelation \(<\) on a \set \(A\)
    with the following properties:
    
    \begin{enumerate}
        \item \textit{Irreflexive}: \(\forall a \in A, \lnot (a < a)\)
        \item \textit{Transitive}: \(\forall a,b,c \in A, a < b \land b < c \implies a < c\)
        \item \textit{Antisymmetric}: \(\forall a,b \in A, a < b \land b < a \implies a=b\)
        \item \textit{Strongly connected} (or \textit{total}): \(\forall a,b\in A, a < b \lor b < a\)
    \end{enumerate}
\end{snippetdefinition}

\section{Element characterization}

\subsection{Upper/lower bounds}

\begin{snippetdefinition}{upper-bound-definition}{Upper bound}
    Let \(P\) be a \set, \(\leq\) be a \preorder on \(P\) and let \(S\subseteq P\).
    An \textit{upper bound} of \(S\) in \((P, \leq)\) is an element \(u\in P\) such that
    \[ \forall s\in S, s \leq u \]
    If such a value exists, then \((P, \leq)\) is said to be \textit{bounded above}.
\end{snippetdefinition}

\begin{snippetdefinition}{lower-bound-definition}{Lower bound}
    Let \(P\) be a \set, \(\leq\) be a \preorder on \(P\) and let \(S\subseteq P\).
    A \textit{lower bound} of \(S\) in \((P, \leq)\) is an element \(u\in P\) such that
    \[ \forall s\in S, u \leq s \]
    If such a value exists, then \((P, \leq)\) is said to be \textit{bounded below}.
\end{snippetdefinition}

\begin{snippetdefinition}{bounded-set-definition}{Bounded set}
    Let \(P\) be a \set, \(\leq\) be a \preorder on \(P\) and let \(S\subseteq P\).
    Then, \(S\) is \textit{bounded} if it is both \upperbound[bounded above] and
    \lowerbound[bounded below].
\end{snippetdefinition}

\plain{Note that the upper bound can be outside of the subset.}

\subsection{Greatest/least elements}

\begin{snippetdefinition}{greatest-element-definition}{Greatest element}
    Let \(P\) be a \set, \(\leq\) be a \preorder on \(P\) and let \(S\subseteq P\).
    A \textit{greatest element} of \(S\) in \((P, \leq)\) is an element \(g \in S\) such that
    \[ \forall s\in S, s \leq g \]
\end{snippetdefinition}

\begin{snippetdefinition}{least-element-definition}{Least element}
    Let \(P\) be a \set, \(\leq\) be a \preorder on \(P\) and let \(S\subseteq P\).
    A \textit{least element} of \(S\) in \((P, \leq)\) is an element \(g \in S\) such that
    \[ \forall s\in S, g \leq s \]
\end{snippetdefinition}

\plain{Greatest and least elements are upper and lower bounds in the subset.}

\begin{snippetcorollary}{greatest-least-element-uniqueness}{Greatest and least element uniqueness}
    Let \(P\) be a \set, \(\leq\) be a \preorder on \(P\) and let \(S\subseteq P\).
    If \(\leq\) is also a \partialorder, then \((P, \leq)\) can have at most one \greatestelement
    and at most one \leastelement.
\end{snippetcorollary}

\plain{This is trivial by the anti-symmetric property.}

\subsection{Maximal/minimal elements}

\begin{snippetdefinition}{maximal-element-definition}{Maximal element}
    Let \(P\) be a \set, \(\leq\) be a \preorder on \(P\) and let \(S\subseteq P\).
    A \emph{maximal element} of \(S\) with respect to \(\leq\) is an element \(m\in S\) such that
    if \(s\in S\),
    \[
        m \leq s \implies s \leq m
    \]
\end{snippetdefinition}

\plain{A maximal element is an element such that no other element is strictly greater than it.}

\begin{snippetdefinition}{minimal-element-definition}{Minimal element}
    Let \(P\) be a \set, \(\leq\) be a \preorder on \(P\) and let \(S\subseteq P\).
    A \emph{minimal element} of \(S\) with respect to \(\leq\) is an element \(m\in S\) such that
    if \(s\in S\),
    \[
        s \leq m \implies m \leq s
    \]
\end{snippetdefinition}

\plain{Every greatest/least element is a maximal/minimal element.}

\begin{snippetproposition}{total-order-maximal-elements-are-greatest-elements}{}
    Let \(P\) be a \set, \(\leq\) be a \preorder on \(P\) and let \(S\subseteq P\).
    If \(\leq\) is also a \totalorder, then every \maximalelement[maximal]/\minimalelement[minimal] element
    is a \greatestelement[greatest]/\greatestelement[least] element.
\end{snippetproposition}

\begin{snippetproof}{total-order-maximal-elements-are-maximum-proof}{total-order-maximal-elements-are-maximum}{}
    Let \(a\in S\) be a \maximalelement. Given another element \(b\in S\), either \(a\leq b\) or \(b\leq a\).
    In the former case, \(a\neq b\) by definition of \maximalelement. In the latter case,
    \(b\leq a\) and thus \(a\) is a \greatestelement.
\end{snippetproof}

\section{Other}

\begin{snippettheorem}{only-partial-order-equivalence-relation}{Both Partial order and equivalence relation}
    Let \(A\) be a \set. The only \binrelation on \(A\) that is both an
    \equivrelation and a \partialorder is the relation \(\sim\) defined as
    \(a \sim b \iff a = b\).
\end{snippettheorem}

\begin{snippetproof}{only-partial-order-equivalence-relation-proof}{only-partial-order-equivalence-relation}{Both Partial order and equivalence relation}
    Let \(A\) be a \set and \(\sim\) be a \binrelation that is both a \partialorder and \equivrelation.
    For any \(a,b\in A\) where \(a\sim b\), we have have \(b\sim a\) since the relation is symmetric.
    However, since the relation is also anti-symmetric, we always have that \(a=b\).
\end{snippetproof}

\begin{snippetdefinition}{eventually-definition}{Eventually}
    Let \((S, \leq)\) be a \totalorder and \(P(s)\) be a property that depends on
    \(s\in S\). Then, \(P\) holds \emph{eventually} if
    \[ \exists M \in S \suchthat \forall s\geq M, P(s) \text{ is true} \]
\end{snippetdefinition}

\begin{snippetdefinition}{poset-definition-definition}{Poset}
    A \emph{partially ordered set} or \emph{poset}
    is a pair \((X, \leq)\) where \(X\) is a \set and \(\leq\) is a
    \partialorder on \(X\).
\end{snippetdefinition}

\end{document}
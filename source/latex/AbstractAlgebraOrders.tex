\documentclass[preview]{standalone}

\usepackage{amsmath}
\usepackage{amssymb}
\usepackage{stellar}
\usepackage{definitions}

\begin{document}

\id{settheory-orders}
\genpage

\section{Types of orders}

\subsection{Reflexive}

\begin{snippetdefinition}{preorder-definition}{Preorder order}
    A \textit{preorder} is a \homrelation \(\leq\) on a set \(A\)
    with the following properties:
    \begin{enumerate}
        \item \textit{Reflexive}: \(\forall a \in A, a \leq a\)
        \item \textit{Transitive}: \(\forall a,b,c \in A, a \leq b \land b \leq c \implies a \leq c\)
    \end{enumerate}
\end{snippetdefinition}

\begin{snippetdefinition}{partial-order-definition}{Partial order}
    A \textit{partial order} is a \homrelation \(\leq\) on a set \(A\)
    with the following properties:
    \begin{enumerate}
        \item \textit{Reflexive}: \(\forall a \in A, a \leq a\)
        \item \textit{Transitive}: \(\forall a,b,c \in A, a \leq b \land b \leq c \implies a \leq c\)
        \item \textit{Antisymmetric}: \(\forall a,b \in A, a \leq b \land b \leq a \implies a=b\)
    \end{enumerate}
\end{snippetdefinition}

\begin{snippetdefinition}{total-order-definition}{Total order}
    A \textit{total order} is a \homrelation \(\leq\) on a set \(A\)
    with the following properties:
    
    \begin{enumerate}
        \item \textit{Reflexive}: \(\forall a \in A, a \leq a\)
        \item \textit{Transitive}: \(\forall a,b,c \in A, a \leq b \land b \leq c \implies a \leq c\)
        \item \textit{Antisymmetric}: \(\forall a,b \in A, a \leq b \land b \leq a \implies a=b\)
        \item \textit{Strongly connected} (or \textit{total}): \(\forall a,b\in A, a \leq b \lor b\leq a\)
    \end{enumerate}
\end{snippetdefinition}

\begin{snippet}{settheory-3}
    A total order is a partial order where any two elements are comparable.
\end{snippet}

\subsection{Irreflexive}

\begin{snippetdefinition}{strict-preorder-definition}{Strict preorder order}
    A \textit{strict preorder} is a \homrelation \(<\) on a set \(A\)
    with the following properties:
    \begin{enumerate}
        \item \textit{Irreflexive}: \(\forall a \in A, \lnot (a < a)\)
        \item \textit{Transitive}: \(\forall a,b,c \in A, a < b \land b < c \implies a < c\)
    \end{enumerate}
\end{snippetdefinition}

\begin{snippetdefinition}{strict-partial-order-definition}{Strict partial order}
    A \textit{strict partial order} is a \homrelation \(<\) on a set \(A\)
    with the following properties:
    \begin{enumerate}
        \item \textit{Irreflexive}: \(\forall a \in A, \lnot (a < a)\)
        \item \textit{Transitive}: \(\forall a,b,c \in A, a < b \land b < c \implies a < c\)
        \item \textit{Antisymmetric}: \(\forall a,b \in A, a < b \land b < a \implies a=b\)
    \end{enumerate}
\end{snippetdefinition}

\begin{snippetdefinition}{strict-total-order-definition}{String total order}
    A \textit{strict total order} is a \homrelation \(<\) on a set \(A\)
    with the following properties:
    
    \begin{enumerate}
        \item \textit{Irreflexive}: \(\forall a \in A, \lnot (a < a)\)
        \item \textit{Transitive}: \(\forall a,b,c \in A, a < b \land b < c \implies a < c\)
        \item \textit{Antisymmetric}: \(\forall a,b \in A, a < b \land b < a \implies a=b\)
        \item \textit{Strongly connected} (or \textit{total}): \(\forall a,b\in A, a < b \lor b < a\)
    \end{enumerate}
\end{snippetdefinition}

\section{Elements characterization}

\begin{snippetdefinition}{greatest-element-definition}{Greatest element}
    Given a \partialorder on a set \(A\), an element \(g\) is a \textit{greatest element}
    if \(\forall a\in A, a \leq g\).
\end{snippetdefinition}

\begin{snippetdefinition}{least-element-definition}{Least element}
    Given a \partialorder on a set \(A\), an element \(g\) is a \textit{least element}
    if \(\forall a\in A, g \leq a\).
\end{snippetdefinition}

\begin{snippetdefinition}{maximal-element-definition}{Maximal element}
    Given a \partialorder on a set \(A\) and \(P\subseteq A\), an element \(g\in P\) that is
    a \snippetref[greatest-element-definition][greatest element] of \(P\) is a \textit{maximal element} of \(P\).
\end{snippetdefinition}

\begin{snippetdefinition}{minimal-element-definition}{Minimal element}
    Given a \partialorder on a set \(A\) and \(P\subseteq A\), an element \(g\in A\) that is
    a \snippetref[least-element-definition][least element] of \(P\) is a \textit{minimal element} of \(P\).
\end{snippetdefinition}

\begin{snippetcorollary}{maximal-minimal-element-uniqueness}{Maximal and minimal element uniqueness}
    Given a \partialorder on a set \(A\) and \(P\subseteq A\), there can only exist
    up to one \snippetref[maximal-element-definition][maximal element] and up to one \snippetref[minimal-element-definition][minimal element].
\end{snippetcorollary}

\plain{This is trivial by the anti-symmetric property.}

\begin{snippettheorem}{only-partial-order-equivalence-relation}{Both Partial order and equivalence relation}
    Let \(A\) be a \set. The only \binrelation on \(A\) that is both an
    \equivrelation and a \partialorder is the relation \(\sim\) defined as
    \(a \sim b \iff a = b\).
\end{snippettheorem}

\begin{snippetproof}{only-partial-order-equivalence-relation-proof}{only-partial-order-equivalence-relation}{Both Partial order and equivalence relation}
    Let \(A\) be a \set and \(\sim\) be a \binrelation that is both a \partialorder and \equivrelation.
    For any \(a,b\in A\) where \(a\sim b\), we have have \(b\sim a\) since the relation is symmetric.
    However, since the relation is also anti-symmetric, we always have that \(a=b\).
\end{snippetproof}

\subsection{Infimum and supremum}

\begin{snippetdefinition}{supremum-definition}{Supremum}
    body
\end{snippetdefinition}

\begin{snippetdefinition}{infimum-definition}{Infimum}
    body
\end{snippetdefinition}

\end{document}
\documentclass[preview]{standalone}

\usepackage{amsmath}
\usepackage{amssymb}
\usepackage{bettelini}
\usepackage{stellar}

\hypersetup{
    colorlinks=true,
    linkcolor=black,
    urlcolor=blue,
    pdftitle={Stellar},
    pdfpagemode=FullScreen,
}

\begin{document}

\id{english-frankenstein-letter-1-4}
\genpage

\section{Esercises}

\begin{snippetexercise}{frankenstein-ex-1}
    {What kind of mission has Robert Walton, the author of the letters, embarked on?}
    His mission is to discover the northern Pole with his crew on the north seas.
    He aims for unexplored lands surrounded by ice, snow and mist.
\end{snippetexercise}

\begin{snippetexercise}{frankenstein-ex-2}
    {What do Walton and the crew see one night passing within one-half a mile of their ice bound
    ship?}
    They see a sledge carried by dogs but on the carriage there's a man who
    looks much bigger than normal size, a giant.
\end{snippetexercise}

\begin{snippetexercise}{frankenstein-ex-3}
    {What does the stranger allude to in this passage while talking with Walton?}
    \begin{center}
        \quotes{\textit{Unhappy man! Do you share my madness? Have you drunk also of the} \\
        \textit{intoxicating draught? Hear me; let me reveal my tale and you will dash the cup} \\
        \textit{from your lips!}}
    \end{center}
    He alludes to the fact that he has done something bad/mad and he wants to tell
    his story to the captain, who is alike and wants to know more about
    everything, so as not to commit the same mistake and not being able to
    stop before its too late, as the stranger did.
\end{snippetexercise}

\begin{snippetexercise}{frankenstein-ex-4}
    {The word \quotes{marvellous} appears three times in Walton's letters: in the second letter, it is used as
    a noun, whereas in the fourth letter it is used as an adjective. What does it mean exactly? What
    does it refer to in the letters?}
    In the second letter the meaning is like something outside this world, something
    that people can't conceive. In the fourth letter it's used as normal
    (\(\rightarrow\) something extraordinary / miraculous).
\end{snippetexercise}

\begin{snippetexercise}{frankenstein-ex-5}
    {What does Walton resolve to do with the stranger's story?}
    He wants to write the story in a manuscript. He makes a comparison between the
    story of the stranger and a vessel in a storm which will be destroyed because
    the stranger is in misery when he met Walton.
\end{snippetexercise}

\end{document}

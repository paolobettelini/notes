\documentclass[preview]{standalone}

\usepackage{amsmath}
\usepackage{amssymb}
\usepackage{stellar}
\usepackage{definitions}

\begin{document}

\id{linearalgebra-exercises-batch-9}
\genpage

\section{Exercises}

\begin{snippetexercise}{linear-algebra-batch-9-ex-1}{}
    \todo
\end{snippetexercise}

\begin{snippetsolution}{linear-algebra-batch-9-ex-1-sol}{}
    \todo
\end{snippetsolution}

\begin{snippetexercise}{linear-algebra-batch-9-ex-2}{}
    \todo
\end{snippetexercise}

\begin{snippetsolution}{linear-algebra-batch-9-ex-2-sol}{}
    \todo
\end{snippetsolution}

\begin{snippetexercise}{linear-algebra-batch-9-ex-3}{}
    \todo
\end{snippetexercise}

\begin{snippetsolution}{linear-algebra-batch-9-ex-3-sol}{}
    \todo
\end{snippetsolution}

\begin{snippetexercise}{linear-algebra-batch-9-ex-4}{}
    Let \(g\in \text{End}(\realnumbers^2)\) such that
    \(g(1,5) = (3,7) \land g(2,8) = (-1,1)\). Find the \matrix of \(g\)
    in the canonical \basis of \(\realnumbers^2\).
\end{snippetexercise}

\begin{snippetsolution}{linear-algebra-batch-9-ex-4-sol}{}
    The wanted \matrix \(M\) has form
    \[
        M = \begin{pmatrix}
            a & b \\
            c & d
        \end{pmatrix}
    \]
    We know that
    \[
        \begin{pmatrix}
            a & b \\
            c & d
        \end{pmatrix} \begin{pmatrix}
            1 \\
            5
        \end{pmatrix} = \begin{pmatrix}
            3 \\
            7
        \end{pmatrix}
        \quad \text{and} \quad
        \begin{pmatrix}
            a & b \\
            c & d
        \end{pmatrix} \begin{pmatrix}
            2 \\
            8
        \end{pmatrix} = \begin{pmatrix}
            -1 \\
            1
        \end{pmatrix}
    \]
    This gives the system
    \[
        \begin{cases}
            a + 5b = 3 \\
            c + 5d = 7 \\
            2a + 8b = -1 \\
            2c + 8d = 1
        \end{cases}
    \]
    We apply gaussian elimination
    \[
        \begin{pmatrix}
            1 & 5 & 0 & 0 & 3 \\
            0 & 0 & 1 & 5 & 7 \\
            2 & 8 & 0 & 0 & -1 \\
            0 & 0 & 2 & 8 & 1
        \end{pmatrix}
        \xrightarrow{\text{RREF}}
        \begin{pmatrix}
            1 & 5 & 0 & 0 & 3 \\
            0 & -2 & 0 & 0 & -7 \\
            0 & 0 & 1 & 5 & 7 \\
            0 & 0 & 0 & -2 & -13
        \end{pmatrix}
    \]
    which gives the matrix
    \[
        M = \begin{pmatrix}
            -\frac{29}{2} & \frac{7}{2} \\
            -\frac{51}{2} & \frac{13}{2}
        \end{pmatrix}
    \]
\end{snippetsolution}

\begin{snippetexercise}{linear-algebra-batch-9-ex-5}{}
    \todo
\end{snippetexercise}

\begin{snippetsolution}{linear-algebra-batch-9-ex-5-sol}{}
    \todo
\end{snippetsolution}

\end{document}
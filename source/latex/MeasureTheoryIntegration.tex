\documentclass[preview]{standalone}

\usepackage{amsmath}
\usepackage{amssymb}
\usepackage{stellar}
\usepackage{definitions}

\begin{document}

\id{measuretheory-integration}
\genpage

\section{Integration}

\begin{snippetdefinition}{simple-function-definition}{Siple function}
    Let \((X, \Sigma)\) be a \measurablespace. A \function \(f \colon X \to \complexnumbers\)
    is said to be a \emph{simple function} if it can be written as
    \[
        f(x) = \sum_{k=1}^n \alpha_k \indicator_{A_k}(x)
    \]
    where \(\alpha_k \in \complexnumbers\) and \(A_k \in \Sigma\).
\end{snippetdefinition}

\begin{snippettheorem}{monotone-convergence-measurable-non-negative-function-theorem}{Monotone convergence theorem for non-negative measurable functions}
    Let \((X, \Sigma, \mu)\) be a measurable space and let
    \[
        f_n \colon X \to [0; +\infty)
    \] be measurable
    such that \(f_n \leq f_{n+1}\). Then,
    \[
        \lim_n \int_X f_n\,d\mu = \int_X \lim_n f_n\,d\mu
    \]
\end{snippettheorem}

\begin{snippettheorem}{fatou-lemma-theorem}{Fatou's Lemma}
    Let \(f_n \colon X \fromto [0; +\infty)\) on a measure space \((X, \Sigma, \mu)\). Then,
    \[
        \int_X \liminf f_n \, d\mu \leq \liminf \int_X f_n \, d\mu
    \]
\end{snippettheorem}

\end{document}
\documentclass[preview]{standalone}

\usepackage{amsmath}
\usepackage{amssymb}
\usepackage{stellar}
\usepackage{definitions}

\hypersetup{
    colorlinks=true,
    linkcolor=black,
    urlcolor=blue,
    pdftitle={Stellar},
    pdfpagemode=FullScreen,
}

\begin{document}

\title{Stellar}
\id{row-echelon-form}
\genpage

\section{Row Echelon Forms}

\begin{snippetdefinition}{row-echelon-form-definition}{Row echelon form}
    A \snippetref[matrix-definition][matrix] is said to be in \textit{row echelon form}
    if it fulfils the following properties:
    \begin{enumerate}
        \item all rows having only zero entries are at the bottom.;
        \item the leading entry (that is, the left-most nonzero entry) of every nonzero row,
        called the pivot, is on the right of the leading entry of every row above.
    \end{enumerate}
\end{snippetdefinition}

\begin{snippetproposition}{row-echelon-form-non-uniqueness}{Non uniqueness of row echelon form}
    \snippetref[row-echelon-form-definition][Row echelon form] is not unique.
\end{snippetproposition}

\begin{snippetdefinition}{reduced-row-echelon-form-definition}{Reduced row echelon form}
    A \snippetref[matrix-definition][matrix] is said to be in \textit{reduced row echelon form}
    if it is in \snippetref[row-echelon-form-definition][row echelon form] and
    each every \matrixpivot is \(1\).
\end{snippetdefinition}

\begin{snippetproposition}{reduced-row-echelon-form-uniqueness}{Uniqueness of reduced row echelon form}
    \snippetref[reduced-row-echelon-form-definition][Reduced row echelon form] is unique.
\end{snippetproposition}

\section{Gauss-Jordan Algorithm}

\begin{snippettheorem}{gauss-jordan-algorithm-theorem}{Gauss-Jordan algorithm}
    Let \(A \in \matrices_n(\realnumbers)\). Then, \(A\)
    is invertible \ifandonlyif there exist a finite sequence of elementary row operations
    which transforms \([A|I_n]\) into \([I_n|A^{-1}]\).
\end{snippettheorem}

\end{document}
\documentclass[preview]{standalone}

\usepackage{amsmath}
\usepackage{amssymb}
\usepackage{stellar}
\usepackage{definitions}
\usepackage{bettelini}

\begin{document}

\id{english-atonement-part-three}
\genpage

\section{Exercises}

\subsection{First Part}

\begin{snippetexercise}{atonement-ex-21}
    {What is the significance of the physical discomfort experienced by Briony and the other nurses at
    the hospital?}
    The physical discomfort humanizes Briony and the other nurses.
    It highlights their vulnerabilities and humanity. Regardless of their social
    status or background, they all suffer together in the face of the war's brutality.
    There is a parallel between being a soldier and a nurse. You have to obey
    rules from the higher-ups, you have to wear a uniform and become a number.
\end{snippetexercise}

\begin{snippetexercise}{atonement-ex-22}
    {\quotes{This was her student life now, these four years, this enveloping regime, and she had no will,
    no freedom to leave.} (p. 260, emphasis added) Explain the quote. Why would she express \quotes{no will
    to leave}?}
    She doesn't want to rebel, she just accepts it. She has become like a robot
    following orders.
    She feels guilty and thinks that this punishment is fair,
    she has to \textbf{atone} for her wrongdoings.
\end{snippetexercise}

\begin{snippetexercise}{atonement-ex-23}
    {What is Briony only escape from reality?}
    Briony's only escpape is writing about her patients and her days.
    She enjoys handwriting, which is all that remains left in her spark.
\end{snippetexercise}

\begin{snippetexercise}{atonement-ex-24}
    {What news does his father's letter contain? How does Briony react to it?}
    Her father delivers the news that Lola Quincey and Paul Marshall are to
    be married the following week.
    Though he does not provide any reason for telling her this,
    Briony understands what she has known for some time:
    Paul was the one who assaulted Lola that summer night in 1935.
    After receiving the letter, Briony feels her years-old guilt
    even more acutely and understands that no matter how good a nurse she is
    or whatever opportunities she has given up,
    she will never be able to make up for what she did to Robbie.
    Briony wonders if her father sent her the letter because he, too, has figured out the truth.
    She tries to call him but cannot get through.
    As Briony walks back to the hospital,
    she sees two army medics smiling at her, but she cannot bring herself to meet their eyes.
    Briony wonders what it would be like to live a carefree life.
\end{snippetexercise}

\begin{snippetexercise}{atonement-ex-25}
    {At the hospital, Briony deals with increasingly more severe injuries and her mental and physical
    strength is tested. She even acknowledges that \quotes{all the training she had received […] had been useful
    preparation, especially in obedience, but everything she understood about nursing she learned that
    night. She had never seen men crying before. It shocked her at first, and within the hour she was
    used to it} (286). What is the impact of the French soldier's death on her?}
    Briony has to deal with the death of the French soldier. She would like to cry for him.
    but she wouldn't feel a thing, she was empty.
\end{snippetexercise}

\begin{snippetexercise}{atonement-ex-26}
    {At the end of the section, Briony receives a letter from the publishing company to whom she had
    sent one of her short stories some time before, namely \quotes{Two Figures at a Fountain}. What are their
    main criticisms of Briony's story?}
    They do give her some applause in the sense that it is well written,
    but they say that it's highly unrealistic and nothing really happens.
    This is important because we can see a little spark of hope towards her future as a writer.
\end{snippetexercise}

\subsection{Second Part}

\begin{snippetdefinition}{stream-of-consciousness-definition}{Stream of consciousness}
    \textit{Stream of consciusness} is a narrative technique in fiction that is used to represent a character's
    feelings and thoughts as they experience them, using long, continuous pieces of text without
    obvious organization or structure.
\end{snippetdefinition}

\begin{snippetexercise}{atonement-ex-27}
    {Read the excerpt as well as the definition below. How does Briony's fictional story resemble the
    one used by the author himself when describing the fountain scene in Section One?}
    \hspace{0.1\textwidth}
    \begin{minipage}[r]{0.8\textwidth}
        \itshape
        Might she come between them in some disastrous fashion? Yes, indeed. And having done so, might 
        she obscure the fact by concocting a slight, barely clever fiction and satisfy her vanity by sending it
        off to a magazine? The interminable pages about light and stone and water, a narrative split
        between three different points of view, the hovering stillness of nothing much seeming to 
        happen - none of this could conceal her cowardice.
        Did she really think she could hide behind some borrowed notions of modern writing, 
        and drown her guilt in a stream—three streams!— of consciousness?”
    \end{minipage}
    \\\\
    The description of the fountain scenes are the same, from a visual perspective,
    but each thinks about a different thing.
    They all read the scene in different ways: when you put emphasis on personality, emotion,
    you get different points of view.
    This highlights the difference in perspective.
\end{snippetexercise}

\begin{snippetexercise}{atonement-ex-28}
    {At Lola and Paul Marshall's wedding. What do the following words imply?}
    \hspace{0.1\textwidth}
    \begin{minipage}[r]{0.8\textwidth}
        \itshape
        “By any estimate, it was a very long time until judgment day, and until then the truth that only
        Marshall and his bride knew at first hand was steadily being walled up within the mausoleum of
        their marriage. There it would lie secure in the darkness, long after anyone who cared was dead.
        Every word in the ceremony was another brick in place.” 
    \end{minipage}
    \\\\
    The words imply that their marriage is like a mausoleum.
    The secret is being buried in a tomb, will be kept hidden from the light until
    it will be completely forgotten.
    Given the marriage, there will be no possibility for the truth to come out
    (legally, a married couple cannot sue the other for something that's happened
    before the marriage, so her voice becomes null).
\end{snippetexercise}

\begin{snippetexercise}{atonement-ex-29}
    {Describe Briony's meeting with Cecilia}
    Briony and Cecilia have a small talk. Briony notices that everything in the room
    is a mess, Cecilia was always really chaotic.
    When Robbie goes into the bedroom and sees Briony he becomes furious
    and start arguing. Cecilia tries to calm him his lover down.
    Now Briony is supposed to appeal to the family after her wrongdoings.
    The atmosphere was tense from the start, they don't really want to see and talk to each other.
    They have to, but don't really want to, especially when Briony is around.
    Cecilia starts smoking cigarettes, and even offers one to Briony.
    Cecilia and Robbie have the wrong idea about who has committed the crime.
    They thought Danny Hardman was the assaulter, but he is innocent and now
    Paul is immune to any legal proceeding against him for the rape.
\end{snippetexercise}

\begin{snippetexercise}{atonement-ex-30}
    {What do you think is the meaning of the signature at the end of this section?}
    The signature is there to imply that Briony was the one who wrote the book,
    as an \textbf{atonement} for what she had done.
    This was all a big interpretation of her past,
    written in the far future (1999),
    to redeem herself.
    BT obvious means Briony Tallis.
\end{snippetexercise}

\end{document}
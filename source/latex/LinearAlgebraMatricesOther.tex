\documentclass[preview]{standalone}

\usepackage{amsmath}
\usepackage{amssymb}
\usepackage{stellar}
\usepackage{definitions}

\hypersetup{
    colorlinks=true,
    linkcolor=black,
    urlcolor=blue,
    pdftitle={Stellar},
    pdfpagemode=FullScreen,
}

\begin{document}

\title{Stellar}
\id{matrices-other}
\genpage

\section{Types of matrices}

\subsection{Rotational matrices}

\begin{snippetdefinition}{bidimensional-rotational-matrix-definition}{Bidimensional rotational matrix}
    The standard rotation matrix in two dimensions is defined as:
    \[
        R(\theta)=
        \begin{bmatrix} 
            \cos\theta && -\sin\theta \\
            \sin\theta && \cos\theta
        \end{bmatrix}
    \]
\end{snippetdefinition}

\begin{snippetdefinition}{tridimensional-rotational-matrices-definition}{Tridimensional rotational matrices}
    The rotation matrices for a three-dimensional point are

    \begin{align*}
        R_x(\theta)&=
        \begin{bmatrix} 
            1 & 0 & 0 \\
            0 & \cos\theta & -\sin\theta \\
            0 & \sin\theta & \cos\theta
        \end{bmatrix}
        \\
        R_y(\theta)&=
        \begin{bmatrix} 
            \cos\theta & 0 & \sin\theta \\
            0 & 1 & 0 \\
            -\sin\theta & 0 & \cos\theta
        \end{bmatrix}
        \\
        R_z(\theta)&=
        \begin{bmatrix} 
            \cos\theta & -\sin\theta & 0 \\
            \sin\theta & \cos\theta & 0 \\
            0 & 0 & 1
        \end{bmatrix}
    \end{align*}
\end{snippetdefinition}

\begin{snippetdefinition}{matrix-conjugate-transpose-definition}{Matrix conjugate transpose}
    Let \(M \in \matrices(\complexnumbers)\). Then, the \emph{adjoint} of \(M\)
    is defined as
    \[
        A^\dagger \triangleq {\left(A^\transpose\right)}^{*}
    \]
\end{snippetdefinition}

\begin{snippetdefinition}{hermitian-matrix-definition}{Hermitian matrix}
    Let \(M \in \matrices(\complexnumbers)\). Then, \(M\) is said to be \emph{hermitian} if
    \[
        M^\conjtrans = M
    \]
\end{snippetdefinition}

\begin{snippetdefinition}{vandermonde-matrix-definition}{Vandermonde matrix}
    A \matrix \(M\) is said to be a \emph{Vandermonde matrix} if it contains the terms
    of a geometric progression in each row or column. \\
    \[
        V =
        \begin{bmatrix}
            1 & a_1 & a_1^2 & \cdots & a_1^{n-1} \\
            1 & a_2 & a_2^2 & \cdots & a_2^{n-1} \\
            1 & a_3 & a_3^2 & \cdots & a_3^{n-1} \\
            \vdots & \vdots & \vdots & \ddots & \vdots \\
            1 & a_m & a_m^2 & \cdots & a_m^{n-1} \\
        \end{bmatrix}
    \]
\end{snippetdefinition}

\end{document}
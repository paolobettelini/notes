\documentclass[preview]{standalone}

\usepackage{amsmath}
\usepackage{amssymb}
\usepackage{stellar}
\usepackage{definitions}

\begin{document}

\id{series-terms-with-parameters-exercises}
\genpage

\section{Series with parameters}


\begin{snippetexercise}{series-parameters-ex-1}{}
    Study the following \series
    \[
        \sum_{n=1}^\infty \frac{x^n}{n\cdot 2^n}
    \]
\end{snippetexercise}

\begin{snippetsolution}{series-parameters-ex-1-sol}{}
    Analizziamo la convergenza assoluta
    \begin{align*}
        \sum_{n=1}^\infty \left|\frac{x^n}{n\cdot 2^n}\right|
        &= \sum_{n=1}^\infty \frac{{|x|}^n}{n\cdot 2^n}
    \end{align*}
    Calcoliamo il limite
    \begin{align*}
        \lim_n |a_n| &= \lim_n {\left|\frac{x}{2}\right|}^n \cdot \frac{1}{n}
        \\
        &= \lim_n \frac{q^n}{n} = \begin{cases}
            0 & q \leq 1 \\
            +\infty & q > 1
        \end{cases}, \quad q = \left|\frac{x}{2}\right|
    \end{align*}
    Possiamo quindi notare che se \(q>1\), la serie non converge assolutamente.
    In particolare, il limite del termine non è pari a zero, e quindi la serie non converge.
    Applichiamo il criterio della radice n-esima,
    \begin{align*}
        \lim_n \sqrt[n]{|a_n|} &= \lim_n {\left(
            \frac{{|x|}^n}{n\cdot 2^n}
        \right)}^{1/n} \\
        &= \frac{|x|}{2} = q
    \end{align*}
    Allora, se \(q<1\), oppure \(|x| < 2\), la serie converge assolutamente,
    mentre nel caso \(q>1\), oppure \(|x| > 2\), la serie non converge.
    Infine, nel caso singolo \(q=1\), oppure \(|x| = 2\),
    il caso è inane. In questo caso la serie diventa
    \[
        \sum_{n=1}^\infty \frac{1}{n} {\left(\frac{|x|}{2}\right)}^n
        = \sum_{n=1}^\infty \frac{1}{n}
    \]
    che non converge assolutamente.
    La serie originale è invece
    \[
        \sum_{n=1} \frac{1}{n}
    \]
    che non converge, e
    \[
        \sum_{n=1} {(-1)}^n \frac{1}{n}
    \]
    con \(x=-2\), caso in cui converge.
    In conclusione la serie
    \[
        \begin{cases}
            \text{converge assolutamente} & -2 < x < 2 \\
            \text{non converge} & x < -2 \lor x \geq 2 \\
            \text{converge semplicemente} & x = -2
        \end{cases}
    \]
\end{snippetsolution}

\begin{snippetexercise}{series-parameters-ex-2}{}
    Study the following \series
    \[
        \sum_{n=1}^\infty {(-1)}^n
        \frac{{\eulernumber}^{n \cdot \frac{x+1}{x-1}}}{n+\sqrt{n}}
    \]
\end{snippetexercise}

\begin{snippetsolution}{series-parameters-ex-2-sol}{}
    \todo
\end{snippetsolution}

\begin{snippetexercise}{series-parameters-ex-3}{}
    Study the following \series
    \[
        \sum_{n=1}^\infty \frac{n+1}{n^2 + 1} \frac{x^n}{\frac{1}{n} + x^{2n}}
    \]
\end{snippetexercise}

\begin{snippetsolution}{series-parameters-ex-3-sol}{}
    \todo
\end{snippetsolution}

\begin{snippetexercise}{series-parameters-ex-4}{}
    Study the following \series
    \[
        \sum_{n=1}^\infty \frac{n+\sqrt{n} - {\eulernumber}^{-n}}{n^2 + 1}{(x^2 - 3x)}^{3n}
    \]
\end{snippetexercise}

\begin{snippetsolution}{series-parameters-ex-4-sol}{}
    The series has alternating terms for \(0<x<3\) and non-negative terms otherwise.
    We start by studying the absolute convergence
    \[
        \sum_{n=1}^\infty |a_n| = \sum_{n=1}^\infty \frac{n+\sqrt{n} - {\eulernumber}^{-n}}{n^2 + 1}{|x^2 - 3x|}^{3n}
    \]
    We apply the root test
    \[
        \lim \sqrt[n]{|a_n|}
        = {|x^2 - 3x|}^3 = L
    \]
    To study \(L<1\) we need solve
    \[
    \begin{cases}
        x^2 - 3x - 1 < 0 \\
        x^2 - 3x + 1 > 0
    \end{cases}
    \]
    which has solution
    \[
        \left(\frac{3-\sqrt{13}}{2}; \frac{3-\sqrt{5}}{2}\right)
        \union \left(\frac{3+\sqrt{5}}{2}; \frac{3+\sqrt{13}}{2}\right)
    \]
    so the \series absolutely converges in this interval.
    
    We now study the case \(L=1\). If \(x=\frac{3\pm\sqrt{13}}{2}\),
    then the \series becomes
    \[
        \sum \frac{n+\sqrt{n} + {\eulernumber}^{-n}}{n^2 + 1}
    \] which \seriesdiverges by asymptotic comparison with the \harmonicseries.
    If \(x=\frac{3\pm\sqrt{5}}{2}\), \series becomes
    \[
        \sum {(-1)}^n \frac{n+\sqrt{n} + {\eulernumber}^{-n}}{n^2 + 1}
    \]
    which converges by Leibniz's theorem.
    
    In the exterior interval
    \[
        \left(-\infty; \frac{3-\sqrt{13}}{2}\right) \union
        \left(\frac{3+\sqrt{13}}{2}; +\infty\right)
    \]
    the term \(x^2 - 3x > 0\) and
    \[
        a_n \asymptotic \frac{C^n}{n}, \quad C>0
    \]
    which \seriesdiverges since
    \[
        \sum \frac{C^n}{n} > \sum \frac{1}{n}
    \]

    In the remaining interval
    \[
        \frac{3-\sqrt{5}}{2} < x < \frac{3+\sqrt{5}}{2}
    \]
    the \series becomes
    \[
        \sum {(-1)}^n \frac{n+\sqrt{n} + {\eulernumber}^{-n}}{n^2 + 1} {|x^2 - 3x|}^{3n}
    \]
    and since the term
    \[
        \frac{n+\sqrt{n} + {\eulernumber}^{-n}}{n^2 + 1} {|x^2 - 3x|}^{3n}
    \]
    is positive and monotine increasing, the \series \seriesoscillates.
\end{snippetsolution}





\begin{snippetexercise}{series-parameters-ex-5}{}
    Study the following \series
    \[
        \sum_{n=1}^\infty {(-1)}^n \frac{1-{\eulernumber}^{-n}}{n+1} n^{x^2-2x}
    \]
\end{snippetexercise}

\begin{snippetsolution}{series-parameters-ex-5-sol}{}
    \todo
\end{snippetsolution}

\end{document}
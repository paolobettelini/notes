\documentclass[preview]{standalone}

\usepackage{amsmath}
\usepackage{amssymb}
\usepackage{stellar}
\usepackage{definitions}
\usepackage{bettelini}

\begin{document}

\id{series-tests}
\genpage

\section{Comparison Test}

\begin{snippettheorem}{series-comparison-test}{Comparison test}
    Let \[\sum_{n=k}^\infty a_n\] and \[\sum_{n=k}^\infty b_n\]
    be \series where \(a_n, b_n \geq 0\) \eventually. Then,
    \begin{enumerate}
        \item \textbf{comparison:} if \(a_n \leq b_n\) \eventually we have
        \[
            \sum_{n=k}^\infty a_n \leq \sum_{n=k}^\infty b_n
        \]
        In particular,
        \[
            \sum_{n=1}^\infty b_n < +\infty \implies
            \sum_{n=1}^\infty a_n < +\infty
        \]
        and
        \[
            \sum_{n=1}^\infty a_n = \infty \implies
            \sum_{n=1}^\infty a_n = +\infty
        \]
        \item \textbf{asymptotic comparison:}
        if \(a_n \sim b_n\), we have
        \[
            \sum_{n=k}^\infty a_n < +\infty \iff
            \sum_{n=k}^\infty b_n < +\infty
        \]
    \end{enumerate}
\end{snippettheorem}

\begin{snippetproof}{series-comparison-test-proof}{series-comparison-test}{Comparison test}
    \begin{enumerate}
        \item Let \[
                A_N = \sum_{n=k}^N a_n \quad B_N = \sum_{n=k}^N b_n
            \]
            Since \(a_n \leq b_n\) eventually, we have \(A_N \leq B_N\) with \(N\to\infty\).
            By monotonie of the limit
            \[ \lim_n A_n = \sum_{n=k}^N a_n \leq \lim_n B_n = \sum_{n=k}^N b_n \]
        \item Assume that \(a_n \asymptotic b_n\) so that \(\frac{a_n}{b_n} \to 1\).
            By definition of limit with \(\varepsilon = \frac{1}{2}\), it follows
            \begin{align*}
                \exists n_0 \suchthat \forall n \geq n_0,
                -\frac{1}{2} < \frac{a_n}{b_n} - 1 < \frac{1}{2}
                &\iff \frac{1}{2} < \frac{a_n}{b_n} < \frac{3}{2} \\
                &\iff \frac{1}{2}b_n < a_n < \frac{3}{2}b_n
            \end{align*}
            By using the first point we have
            \[
                \frac{1}{2} \sum_{n=n_0}^\infty b_n < \sum_{n=n_0}^\infty a_n < \frac{3}{2} \sum_{n=n_0}^\infty b_n
            \]
            and the thesis follows.
    \end{enumerate}
\end{snippetproof}

\begin{snippetcorollary}{series-behavior-bigtheta}{}
    Let \[\sum_{n=k}^\infty a_n\] and \[\sum_{n=k}^\infty b_n\]
    be \series where \(a_n = \bigtheta(b_n)\). Then, both \series follow the same behavior.
\end{snippetcorollary}

%\section{Limit comparison test}%
%\begin{snippettheorem}{limit-comparison-test}{Limit comparison test}
%    Let \[
%        \sum_{n=k}^\infty a_n
%    \]
%    and \[
%        \sum_{n=k}^\infty b_n
%    \]
%    be \series with \(a_n \geq 0\) and \(b_n > 0\) and let
%    \[ c = \lim_{n \to \infty} = \frac{a_n}{b_n} \]
%    Then, if \(0 > c > \infty\), either both \series converge or both \series diverge.
%\end{snippettheorem}

\section{Cauchy condensation test}

\begin{snippettheorem}{cauchy-condensation-test-theorem}{Cauchy condensation test}
    Let
    \[
        \sum_{n=1}^\infty a_n
    \]
    be a \series where \(a_n \geq 0\) and \(\forall n, a_n \geq a_{n+1}\).
    Then, the \series follows the same behavior as
    \[
        \sum_{k=0}^\infty 2^k a_{2^k}
    \]
\end{snippettheorem}

\begin{snippetproof}{cauchy-condensation-test-theorem-proof}{cauchy-condensation-test-theorem}{Cauchy condensation test}
    \todo
\end{snippetproof}

\section{Alternating series test (Leibniz's Theorem)}

\begin{snippettheorem}{alternating-series-test}{Alternating series test}
    Let \[\sum_{n=1}^\infty {(-1)}^n a_n\] be a \series
    where:
    \begin{enumerate}
        \item \(a_n \geq 0\);
        \item \(\lim a_n = 0\);
        \item \(\{a_n\}\) is a decreasing sequence.
    \end{enumerate}
    Then, the \series \seriesconverges.
    Furthermore, if \(S\) is the value of the \series and \(S_n\) is its \partialsum,
    we have that the error \(E_n = S-S_n\) satisfies:
    \[ |E_n| \leq a_{n+1} \]
    and the sign of \(E_N\) is the sign of \({(-1)}^{N+1}\). Thus, if \(N\) is even, \(S_N\geq S\) and
    \(S_N \leq S\) is \(N\) is odd (the equality happens when the terms are null).
\end{snippettheorem}

\begin{snippetproof}{alternating-series-test-proof}{alternating-series-test}{Alternating series test}
    Let's separately consider the partial sums with even indices and odd indices.
    Note that:
    \begin{enumerate}
        \item \begin{align*}
            S_{2N} &= \sum_{n=1}^{2N} {(-1)}^n a_n \\
            &= \sum_{n=1}^{2N-1} \left({(-1)}^n a_n \right) + {(-1)}^{2N} - a_{2N} \\
            &\geq S_{2N-1}
        \end{align*}
        \item \begin{align*}
            S_{2N} &= -a_1 + a_2 -a_3 + a_4 + \cdots - a_{2N-1} + a_{2N} \\
            &= -(a_1 - a_2) -(a_3 - a_4) - \cdots - (a_{2N-1} a_{2N})
        \end{align*}
        Since \(a_n \geq a_{n+1}\), all groupings are greater than or equal to zero.
        Therefore,
        \begin{align*}
            S_{2N} &\leq -(a_1 - a_2) - (a_3 - a_4) - \cdots - (a_{2N-3} - a_{2N-2}) \\
            &= S_{2N - 2}
        \end{align*}
        so \(S_{2N}\) is monotonically decreasing.
        \item Now, consider the odd indices:
        \begin{align*}
            S_{2N + 1} &= -a_1 + (a_2 - a_3) + \cdots + (a_{2N-2} - a_{2N-1})
            + (a_{2N} - a_{2N+1})
        \end{align*}
        As before, all groupings are greater than or equal to zero (including the last one),
        so
        \begin{align*}
            S_{2N + 1} &\geq -a_1 + (a_2 - a_3) + \cdots + (a_{2N-2} - a_{2N-1}) \\
            &= S_{2N-1}
        \end{align*}
        thus \(S_{2N+1}\) is monotone increasing.
    \end{enumerate}
    We thus obtain that \(\forall N, S_{2N} \geq S_{2N-1} \geq S_1\). Since the odd terms are increasing, the smallest of all is the first.
    Therefore, \(S_{2N}\) is monotonically decreasing and bounded below.
    So,
    \[
        \exists\, \lim_N S_{2N} = S^+
    \]
    which, being monotonically decreasing, converges from above.
    Since \(S_{2N + 1} = S_{2N} - a_{2N + 1}\), where the first term tends to \(S\) and \(a_{2N+1}\) tends to zero,
    then \(S_{2N+1}\) lso tends to \(S\). In this case, it converges from below, as \(S_{2N+1}\) is monotonically increasing.
    Consequently, since both \(S_{2N}\) and \(S_{2N+1}\) tend to \(S\), the whole sequence converges to \(S\).
    Additionally, as seen, all even terms are greater than \(S\) and all odd terms are less than \(S\).
    Since \(S_{2N+1} \to S^-\) and \(S_{2N} \to S^+\), we have
    \[
        \forall n, M, S_{2M+1} \leq S \leq S_{2N}
    \]
    Now, consider the error \(E_N = S - S_N\).
    If \(N\) is even, then \(N=2k\), so since \(S_{2k+1} \leq S \leq S_{2k}\),
    \[E_N = S - S_{2k} \geq S_{2k + 1} - S_{2k} = {(-1)}^{2k+1} a_{2k+1}\]
    Thus we have
    \[
        |E_{2k}| \leq a_{2k+1}
    \]
    and the sign of \(E_{2k} = -1\).
    The other case \(N=2k+1\) is analogous.
\end{snippetproof}

\section{Ratio test}

\begin{snippettheorem}{ratio-test}{Ratio test}
    Let \(\sum a_n\) be a \series and set
    \[
        L = \lim_{n \to \infty} \left| \frac{a_{n+1}}{a_n} \right|
    \]
    Then,
    \begin{enumerate}
        \item if \(L < 1\) the \series is absolutely convergent;
        \item if \(L > 1\) the \series is divergent;
        \item if \(L = 1\) the \series may be divergent, conditionally convergent, or absolutely convergent.
    \end{enumerate}
\end{snippettheorem}

\section{Root test}

\begin{snippettheorem}{root-test}{Root test}
    Let \(\sum a_n\) be a series and set
    \[
        L = \lim_{n \to \infty} {|a_n|}^{\frac{1}{n}}
    \]
    Then,
    \begin{enumerate}
        \item if \(L < 1\) the \series is absolutely convergent;
        \item if \(L > 1\) the \series is divergent;
        \item if \(L = 1\) the \series may be divergent, conditionally convergent, or absolutely convergent.
    \end{enumerate}
\end{snippettheorem}

\section{Integral Test}

\begin{snippettheorem}{integral-test}{Integral Test}
    Let \(f(x)\) be a continuous \function on \([k;\infty)\)
    such that it is decreasing and positive on the interval \([N; \infty)\)
    for some \(N\).
    \[
        \integral[k][\infty][f(x)][x] \text{ converges } \implies
        \sum_{n=k}^{\infty} f(n) \text{ converges}
    \]
    and
    \[
        \integral[k][\infty][f(x)][x] \text{ diverges } \implies
        \sum_{n=k}^{\infty} f(n) \text{ diverges}
    \]
\end{snippettheorem}

\begin{snippetproof}{integral-test-proof}{integral-test}{Integral Test}{
    \todo
}
\end{snippetproof}

\end{document}
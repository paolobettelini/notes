\documentclass[preview]{standalone}

\usepackage{amsmath}
\usepackage{amssymb}
\usepackage{stellar}
\usepackage{chemfig}

\hypersetup{
    colorlinks=true,
    linkcolor=black,
    urlcolor=blue,
    pdftitle={Stellar},
    pdfpagemode=FullScreen,
}

\begin{document}

\title{Stellar}
\id{chimica-esercizi-idrocarburi}
\genpage

\begin{snippetexercise}{idrocarburi-ex1}
    {Nomina la seguente molecola secondo la nomenclatura IUPAC}
    \begin{center}
        \chemfig{-[1]-[-1](-[-2]-[-1])-[1]-[-1]-[1]-[-1]}
        \\\vspace{0.25cm}
        3-etileptano
    \end{center}
\end{snippetexercise}

\begin{snippetexercise}{idrocarburi-ex2}
    {Nomina la seguente molecola secondo la nomenclatura IUPAC}
    \begin{center}
        \chemfig{-[-1](-[:90])(-[-4])-[-1]-[1]}
        \\\vspace{0.25cm}
        2,2-dimetilbutano
    \end{center}
\end{snippetexercise}

\begin{snippetexercise}{idrocarburi-ex3}
    {Nomina la seguente molecola secondo la nomenclatura IUPAC}
    \begin{center}
        \chemfig{-[1](-[:90])-[-1]-[1](-[:90]-[1]-[:90])-[-1]-[1]-[-1]-[1]}
        \\\vspace{0.25cm}
        2-metil-4-propilottano
    \end{center}
\end{snippetexercise}

\begin{snippetexercise}{idrocarburi-ex4}
    {Nomina la seguente molecola secondo la nomenclatura IUPAC}
    \begin{center}
        \chemfig{*6(---(-[1])-(-[:100]-[1])--)}
        \\\vspace{0.25cm}
        1-etil-2-metilciclopentano
    \end{center}
\end{snippetexercise}

\begin{snippetexercise}{idrocarburi-ex5}
    {Nomina la seguente molecola secondo la nomenclatura IUPAC}
    \begin{center}
        \chemfig{-[1](-[:90])(-[3])-[-1]-[1](-[:90]-[1]-[-1])-[-1]-[1]-[-1]-[1]-[-1]}
        \\\vspace{0.25cm}
        2,2-dimetil-4-propilnonano
    \end{center}
\end{snippetexercise}

\begin{snippetexercise}{idrocarburi-ex6}
    {Nomina la seguente molecola secondo la nomenclatura IUPAC}
    \begin{center}
        \chemfig{=[1]-[-1]=[1]}
        \\\vspace{0.25cm}
        1,3-butadiene
    \end{center}
\end{snippetexercise}

\begin{snippetexercise}{idrocarburi-ex7}
    {Nomina la seguente molecola secondo la nomenclatura IUPAC}
    \begin{center}
        \chemfig{-[-1](-[:-90])-[1]~[1]}
        \\\vspace{0.25cm}
        3-metil-1-butino
    \end{center}
\end{snippetexercise}

\begin{snippetexercise}{idrocarburi-ex8}
    {Nomina la seguente molecola secondo la nomenclatura IUPAC}
    \begin{center}
        \chemfig{-[1]-[-1](=[:-90])-[1]}
        \\\vspace{0.25cm}
        2-metil-1-butene
    \end{center}
\end{snippetexercise}

% pag 12

% pag 24.1
\begin{snippetexercise}{idrocarburi-ex9}
    {Nomina la seguente molecola secondo la nomenclatura IUPAC}
    \begin{center}
        \chemfig{-[1](=[:90])-[-1](-[:-90]-[-1])-[1](-[:110])(-[:70])-[-1]-[1]}
        \\\vspace{0.25cm}
        3-etil-4,4-dimetil-2-esanone
    \end{center}
\end{snippetexercise}

\end{document}
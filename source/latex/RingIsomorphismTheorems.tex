\documentclass[preview]{standalone}

\usepackage{amsmath}
\usepackage{amssymb}
\usepackage{stellar}
\usepackage{definitions}

\begin{document}

\id{ring-isomorphism-theorems}
\genpage

\section{Isomorphism theorems}

\begin{snippettheorem}{ring-first-isomorphism-theorem}{First isomorphism theorem}
    Let \(\varphi\colon G \fromto H\) be a \ringhomomorphism.
    Then, \[G / \ringker_\varphi \ringisomorphic \image \varphi\]
\end{snippettheorem}

\begin{snippettheorem}{ring-second-isomorphism-theorem}{Second isomorphism theorem}
    Let \(A\) be a \ring and let
    \(B \subringleq A\) and \(I \idealin A\).
    Then, the quotient ring
    \[
        \frac{(B+I)}{I} \ringisomorphic \frac{B}{I \intersection B} 
    \]
\end{snippettheorem}

\begin{snippettheorem}{ring-third-isomorphism-theorem}{Third isomorphism theorem}
    Let \(A\) be a \ring, \(I, J \idealin A\) and \(J \subset I\).
    Then,
    \[
        \frac{A/J}{I/J} \ringisomorphic \frac{A}{I}
    \]
    In particular, there is a \bijective \function
    between the \set of \ideal[ideals] of \(A/J\)
    and the \set of \ideal[ideals] \(I\) of \(A\) such that \(I \subseteq J\).
\end{snippettheorem}

\subsection{Proofs}

\begin{snippetproof}{ring-first-isomorphism-theorem-proof}{ring-first-isomorphism-theorem}{First isomorphism theorem}
    \todo
\end{snippetproof}

\begin{snippetproof}{ring-second-isomorphism-theorem-proof}{ring-second-isomorphism-theorem}{Second isomorphism theorem}
    \todo
\end{snippetproof}

\begin{snippetproof}{ring-third-isomorphism-theorem-proof}{ring-third-isomorphism-theorem}{Third isomorphism theorem}
   \todo
\end{snippetproof}

\end{document}
\documentclass[preview]{standalone}

\usepackage{amsmath}
\usepackage{amssymb}
\usepackage{tikz}
\usepackage{graphicx}
\usepackage{pgfplots}
\usepackage{amsmath}
\usepackage{stellar}
\usepackage{definitions}
\usepackage{bettelini}

\begin{document}

\id{vector-spaces}
\genpage

\section{Definitions}

\includesnpt{vector-space-definition}

\begin{snippetproposition}{field-is-vector-space-with-itself}{}
    Let \(\mathbb{F}\) be a \field. Then, \(\mathbb{F}\) is a
    \vectorspace over \(\mathbb{F}\).
\end{snippetproposition}

\begin{snippetproposition}{field-cartesian-prod-vector-space}{}
    Let \(\mathbb{F}\) be a \field. Then, \({\mathbb{F}}^n\) is a
    \vectorspace over \(\mathbb{F}\).
\end{snippetproposition}

\begin{snippetdefinition}{linear-subspace-definition}{Linear subspace}
    Let \(V\) be a \vectorspace over \(\mathbb{F}\). A \set \(U \subseteq V\)
    is a \emph{linear subspace} of \(V\) if \(U\) is a \vectorspace over \(\mathbb{F}\)
    with the operations of \(V\).
\end{snippetdefinition}

\begin{snippettheorem}{linear-subspace-criterion-theorem}{}
    Let \(V\) be a \vectorspace over \(\mathbb{F}\).
    A \set \(U \subseteq V\) is a linear subspace \ifandonlyif
    \begin{enumerate}
        \item \(0_V \in U\);
        \item \(u,v \in U \implies u+v \in U\);
        \item \(\lambda \in \mathbb{F}, u\in U \implies \lambda \cdot u \in U\).
    \end{enumerate}
\end{snippettheorem}

\begin{snippetproof}{linear-subspace-criterion-theorem-proof}{linear-subspace-criterion-theorem}{}
    \iffproof{
        If \(U\) is a linear subspace, then the properties are satisfied by definition.
    }{
        By the listed properties, we have:
        \begin{enumerate}
            \item the additive identity of \(V\) is in \(U\);
            \item addition is closed and well-defined in \(U\);
            \item scalar multiplication is closed and well-defined in \(U\).
        \end{enumerate}
        Given \(u\in U\), \(-1 \cdot u = -u\) is the inverse, which is closed \(-u \in U\).
        The operations of addition, product and scalar multiplication work in \(V\), so they also work in \(U\),
        and by the properties they are also closed in \(U\).
    }
\end{snippetproof}

\begin{snippetproposition}{linear-subspace-intersection}{Linear subspaces intersection}
    Let \(V\) be a \vectorspace and \(U, W\) be linear subspaces of \(V\).
    Then, \(U \intersection W\) is a linear subspace of \(V\), \(U\) and \(W\).
\end{snippetproposition}

\begin{snippetproof}{linear-subspace-intersection-proof}{linear-subspace-intersection}{Linear subspaces intersection}
    Let \(\mathbb{F}\) be the \field over which \(V\) is defined.
    \begin{enumerate}
        \item the additive identity \(0_V \in U \intersection W\);
        \item given \(u,w \in U \intersection W\), then \(u + w \in U\) but also
        \(u + w \in W\), meaning \(u + v \in U \intersection W\);
        \item given \(u \in U \intersection W\) and \(\lambda \in \mathbb{F}\),
        then \(\lambda u \in U\) but also \(\lambda u \in W\), meaning
        \(\lambda u \in U \intersection W\).
    \end{enumerate}
\end{snippetproof}

\plain{The same does not work for the union, so we want to construct the smallest linear subspace
containing two subspaces.}

\begin{snippetdefinition}{vector-space-sum-definition}{Vector space sum}
    Let \(U_1, U_2, \cdots, U_n\) be linear subspaces of
    a \vectorspace \(V\). Then, their \emph{sum}
    is given by
    \[
        U_1 + U_2 + \cdots + U_n \triangleq \{
            u_1 + u_2 + \cdots + u_n \suchthat u_1\in U_1, u_2\in U_2, \cdots, u_n\in U_n    
        \}
    \]
\end{snippetdefinition}

\begin{snippetproposition}{vector-space-sum-is-smallest-vector-space}{}
    Let \(U_1, U_2, \cdots, U_n\) be linear subspaces of
    a \vectorspace \(V\). Then, \(U_1 + U_2 + \cdots + U_n\)
    is the smallest \vectorspace containing each \(U_i\).
\end{snippetproposition}

\begin{snippetproof}{vector-space-sum-is-smallest-vector-space-proof}{vector-space-sum-is-smallest-vector-space}{}
    We first show that the sum is a linear subspace:
    \begin{enumerate}
        \item since \(U_i\) is a linear subspace, \(0_V \in U_i\).
        Thus,
        \[
            0_V = 0_V \in U_1 + 0_V \in U_2 + \cdots + 0_V \in U_n
        \]
        \item given \(u,v \in U_1 + U_2 + \cdots + U_n\), we have
        \(u=u_1 + u_2 + \cdots + u_n\) and \(v=v_1 + v_2 + \cdots + v_n\). Together we have
        \begin{align*}
            u + v = (u_1 + v_1) + (u_2 + v_2) + \cdots + (u_n + v_n)
        \end{align*}
        and since \(u_i + v_i \in U_i\), \(u+v \in U_1 + U_2 + \cdots + U_n\).
        \item \(u \in U_1 + U_2 + \cdots + U_n \iff u = u_1 + \cdots + u_n\).
        Let \(\lambda \in \mathbb{F}\). Then,
        \begin{align*}
            \lambda u = \lambda (u_1 + \cdots + u_n) = \lambda u_1 + \cdots \lambda u_n \in
            U_1 + U_2 + \cdots + U_n
        \end{align*}
    \end{enumerate}
    We now need to prove that such construction is the smallest \vectorspace containing \(U_1, U_2, \cdots, U_n\).
    Let \(W\) be a \vectorspace containing \(U_1, U_2, \cdots, U_n\),
    meaning that \(U_i \subseteq W\) for every \(i\).
    Since \(W\) is closed under addition and scalar multiplication,
    we have that \(u_1 + u_2 + \cdots + u_n \in W\) for every \(u_i \in U_i\).
    Thus, every \vectorspace containing \(U_1, U_2, \cdots, U_n\)
    contains \(U_1 + U_2 + \cdots + U_n\).
    Since \(U_1 + U_2 + \cdots + U_n\) is a \vectorspace containing \(U_1, U_2, \cdots, U_n\),
    it is the smallest \vectorspace containing \(U_1, U_2, \cdots, U_n\).
\end{snippetproof}

\plain{Linear subspaces can never be disjoint. However, something close can happen.
Note that elements of the sum do not have a unique representation, and can be obtained
by adding different terms. We want to restrict ourselves to the case where the sum is uniquely represented.}

\begin{snippetdefinition}{vector-space-direct-sum-definition}{Vector space direct sum}
    Let \(U_1, U_2, \cdots, U_n\) be linear subsets of a \vectorspace \(V\)
    over a \field \(\mathbb{F}\). Then, the sum \(U_1 + U_2 + \cdots + U_n\)
    is said to be a \emph{direct sum} if every element cannot be written as the sum of different elements.
    \[
        U_1 \oplus U_2 \oplus \cdots \oplus U_n
    \]
\end{snippetdefinition}

\begin{snippetproposition}{vector-space-direct-sum-identity}{}
    Let \(U_1, U_2, \cdots, U_n\) be linear subsets of a \vectorspace \(V\)
    over a \field \(\mathbb{F}\). Then, the sum \(U_1 + U_2 + \cdots + U_n\)
    is a direct sum \ifandonlyif \(0_V\) cannot be written as the sum of different terms.
\end{snippetproposition}

\begin{snippetproof}{vector-space-direct-sum-identity-proof}{vector-space-direct-sum-identity}{}
    \iffproof{
        Given by the definition of direct sum.
    }{
        Let \(v \in V\) and consider the representations \(v = v_1 + \cdots v_n\)
        and \(v = w_1 + \cdots + w_n\). We want to show that \(w_i = v_i\) for every \(i\).
        \begin{align*}
            v_1 + \cdots v_n - w_1 - \cdots w_n &= 0_V \\
            (v_1 - w_1) + \cdots + (v_n - w_n) &= 0_V
        \end{align*}
        But since \(0_V\) can only be written as
        \(0_V = 0_V + \cdots + 0_V\), we must have \(v_i - w_i = 0_V\),
        meaning \(v_i = w_i\).
    }
\end{snippetproof}

\begin{snippetproposition}{vector-space-sum-is-direct-iff-intersection-is-trivial}{}
    Let \(U, W\) be linear subspaces of a \vectorspace \(V\).
    Then, \(U + W\) is a direct sum \ifandonlyif \(U \intersection W = \{0_V\}\).
\end{snippetproposition}

\begin{snippetproof}{vector-space-sum-is-direct-iff-intersection-is-trivial-proof}{vector-space-sum-is-direct-iff-intersection-is-trivial}{}
    \iffproof{
        Let \(v \in U \intersection W\). Consider the representation \(0 = v + (-v)\)
        where \(v\in U\) and \(-v \in W\). Since the sum is direct, \(v = -v\) meaning \(v = 0_V\).
    }{
        We need to show that the only representation of \(0_V\) is the trivial one.
        Assume that \(0 = u + w\) for some \(u\in U\) and \(w\in W\).
        Since \(u=-w\), and \(u\in U\land -w \in W\), we must have that \(u, w \in U \intersection W\),
        and since \(U \intersection W = \{0_V\}\), \(u = w = 0_V\). 
    }
\end{snippetproof}

\includesnpt{linear-combination-definition}

\includesnpt{span-definition}

\begin{snippetproposition}{span-is-linear-subspace}{Span is linear subspace}
    Let \(V\) be a \vectorspace over a \field \(K\) and \(S\) a \set of \vector[vectors] in \(V\).
    Then, \(\linearspan(S)\) is the smallest linear subspace of \(V\) containing
    every element of \(S\).
\end{snippetproposition}

\begin{snippetproof}{span-is-linear-subspace-proof}{span-is-linear-subspace}{Span is linear subspace}
    \begin{enumerate}
        \item If \(S = \emptyset\), then \(0_V \in \linearspan(S)\). Otherwise,
        we can construct \(0_V\) as a linear combination where the scalars are \(0\).
        \item Let \(u = a_1v_1 + \cdots + a_nv_n\) and \(w = b_1v_1 + \cdots + b_nv_n\)
        with \(v_i \in S\) such that \(u, w \in \linearspan(S)\). Then,
        \begin{align*}
            u+w &= a_1v_1 + \cdots + a_nv_n + b_1v_1 + \cdots + b_nv_n \\
            &= (a_1 + b_1)v_1 + \cdots + (a_n + b_n)v_n \in \linearspan(S)
        \end{align*}
        \item Let \(u = a_1v_1 + \cdots + a_nv_n\). We have
        \begin{align*}
            \lambda u &= \lambda(a_1v_1 + \cdots + a_nv_n) \\
            &= (\lambda a_1)v_1 + \cdots + (\lambda a_n)v_n \in \linearspan(S)
        \end{align*}
    \end{enumerate}
\end{snippetproof}

% TODO: linear subspaces are closed under linear combination

\includesnpt{linear-independence-definition}

\begin{snippetdefinition}{vector-space-generating-set}{Generating set}
    Let \(A=(V, F, +, \cdot)\) be a \vectorspace.
    A \textit{generating set} \(\mathcal{S}\) is a
    \set of \vector[vectors] \(S\subseteq V\) such that \[ \linearspan(S) = V \]
\end{snippetdefinition}

\begin{snippetdefinition}{basis-definition}{Basis}
    Let \(A=(V, F, +, \cdot)\) be a \vectorspace.
    A \textit{basis} \(\mathcal{B}\) is a finite \linearlyindependent
    generating set of \(V\).
\end{snippetdefinition}

\plain{The basis cannot be redundant.}

\begin{snippetdefinition}{finite-vector-space-definition}{Finite vector space}
    A \vectorspace is said to be \emph{finite} if there exist
    a finite \basis for it.
\end{snippetdefinition}

\begin{snippetlemma}{linearly-dependent-set-reduction}{}
    Let \(A=(V, F, +, \cdot)\) be a \vectorspace
    and consider a linearly dependent \set \(S=\{V_1, V_2, \cdots, V_n\}\).
    Then, there exist a \(j\) such that
    \begin{enumerate}
        \item \(V_j \in \linearspan(S \difference V_j)\);
        \item \(\linearspan(S \difference V_j) = \linearspan(S)\).
    \end{enumerate}
\end{snippetlemma}

\begin{snippetproof}{linearly-dependent-set-reduction-proof}{linearly-dependent-set-reduction}{}
    \begin{enumerate}
        \item We know there there exist scalars \(a_1, a_2, \cdots, a_n\) where \((a_1, a_2, \cdots, a_n) \neq (0, 0, \cdots, 0)\)
        such that \[
            \sum_{i=1}^n a_i \vec{v}_i = 0
        \]
        Let \(j\) be the greatest index such that \(a_j \neq 0\).
        Then,
        \[
            \vec{v}_j = \sum_{i=1}^n \frac{-a_i}{a_j} \vec{v}_i
        \]
        which means that \(\vec{v}_j \in \linearspan(S)\).
        \item Clearly, \(\linearspan(S \difference V_j) \subseteq \linearspan(S)\).
        We want to show the other inclusion.
        Let \(\vec{u} \in \linearspan(S)\)
        \begin{align*}
            \vec{u} &= \sum_{i=1}^n b_i \vec{v}_i \\
            &= \sum_{i \neq j}^n b_i \vec{v}_i
            + b_j \sum_{i \neq j}^n \frac{-a_i}{a_j} \vec{v}_i \\
            &= \sum_{i \neq j}^n \left(b_i - b_j \frac{a_i}{a_j} \right) \vec{v}_i.
        \end{align*}
        Since all terms involve only elements of \(S \difference \{V_j\}\), we conclude that 
        \(\vec{u} \in \linearspan(S \difference \{V_j\})\).
    \end{enumerate}
\end{snippetproof}

\begin{snippettheorem}{vector-space-generative-set-cardinality-inequality}{}
    Let \(A=(V, F, +, \cdot)\) be a \vectorspace.
    Then, if \(A\) is a \linearlyindependent \set and \(B\) is a generator set of \(V\),
    \[
        |A| \leq |B|
    \]
\end{snippettheorem}

\begin{snippetproof}{vector-space-generative-set-cardinality-inequality-proof}{vector-space-generative-set-cardinality-inequality}{}
    Let \(A = \{v_1, v_2, \cdots, v_n\}\)
    and \(B = \{w_1, w_2, \cdots, w_n\}\).
    We iterate the following process:
    add a vector \(\vec{v}_j\) to \(B\), which produces a linearly dependent set.
    By \snippetref[linearly-dependent-set-reduction][this lemma]
    one of the vectors of this collection is in the span of the other vectors,
    remove such vector from \(B\). Eventually, we will repreat this process for every vector of \(A\),
    meaning that at the beginning there are at least many elements in \(A\)
    as there are in \(B\).
\end{snippetproof}

\begin{snippetcorollary}{finite-vector-space-subspace-is-finite}{}
    Let \(V\) be a finitely dimensional \vectorspace. Then,
    a linear subspace of \(V\) is finite.
\end{snippetcorollary}

\begin{snippetproof}{finite-vector-space-subspace-is-finite-proof}{finite-vector-space-subspace-is-finite}{}
    Let \(U\) be a linear subspace of \(V\).
    If \(U={0_V}\) we are done. Otherwise, there exist \(u_1 \in U\) that is not null.
    We can keep adding vectors to \(U\) that are not in the span of the previous ones,
    this is always a linearly independent set.
    This collection cannot have more elements than a generating set, thus it is finite. 
\end{snippetproof}

\begin{snippetproposition}{vector-space-basis-criterion}{}
    Let \(A=(V, F, +, \cdot)\) be a \vectorspace.
    A \set \(\mathcal{B} = \{v_1, v_2, \cdots, v_n\}\)
    of vectors in \(V\) is a \basis \ifandonlyif
    for every \(v\in V\), \(v\) can be written uniquely as a linear combination of the vectors
    in \(\mathcal{B}\).
\end{snippetproposition}

\begin{snippetproof}{vector-space-basis-criterion-proof}{vector-space-basis-criterion}{}
    \iffproof{
        Since \(\mathcal{B}\) generates \(V\),
        let \[
            v = \sum_{i=1}^n \alpha_i v_i
        \]
        or
        \[
            v = \sum_{i=1}^n \beta_i v_i
        \]
        for some \(\alpha_i,\beta_i \in F\).
        But then,
        \begin{align*}
            \sum_{i=1}^n (\alpha_i - \beta_i) v_i = 0 \implies \alpha_i = \beta_i
        \end{align*}
        since \(\mathcal{B}\) is linearly independent.
    }{
        Since every element can be written as a linear combination,
        \[
            \linearspan(\mathcal{B}) = V
        \]
        Assume there exist \(\alpha_i \in F\) such that
        \[
            0 = \sum_{i=1}^n \alpha_i v_i
        \]
        since \[
            0 = \sum_{i=1}^n 0 \cdot v_i
        \]
        by the uniqueness of the linear combination we conclude that
        \(\alpha_i = 0\), meaning that \(\mathcal{B}\) is linearly independent.
    }
\end{snippetproof}

\begin{snippetproposition}{vector-space-generating-set-reduction}{Generating set reduction}
    Let \(A=(V, F, +, \cdot)\) be a \vectorspace.
    Every generating set of \(V\) can be reduced to a \basis of the \vectorspace.
\end{snippetproposition}

\begin{snippetproof}{vector-space-generating-set-reduction-proof}{vector-space-generating-set-reduction}{Generating set reduction}
    Let \(S = \{v_1, v_2, \cdots, v_n\}\) be such a generating set of \(V\).
    We want to iteratively remove an element until \(S\) is a \basis.
    \begin{itemize}
        \item \emph{step zero} if \(v_1 = 0\), we remove \(v_1\) from \(S\).
        \item \emph{step j} if \(v_j = \linearspan(S \difference v_j)\), we remove \(v_j\) from \(S\).
    \end{itemize}
    After \(n\) steps we have a linearly independent set which generates \(V\).
\end{snippetproof}

\begin{snippetcorollary}{every-finite-vector-space-has-basis}{}
    A \vectorspace \(A=(V, F, +, \cdot)\) has a \basis.
\end{snippetcorollary}

\begin{snippetproof}{every-finite-vector-space-has-basis-proof}{every-finite-vector-space-has-basis}{}
    The \set \(V\) is a generator set of \(V\). This \set can be reduced into a \basis.
\end{snippetproof}

\begin{snippetproposition}{linearly-independent-set-basis-expansion}{}
    Let \(A=(V, F, +, \cdot)\) be a finitely dimensional \vectorspace.
    Every \set of vectors that is linearly independent
    in a \vectorspace can be expanded into a \basis.
\end{snippetproposition}

\begin{snippetproof}{linearly-independent-set-basis-expansion-proof}{linearly-independent-set-basis-expansion}{}
    Let \(\mathcal{B} = \{w_1, w_2, \cdots, w_n\}\) be a \basis of \(A\)
    and let \(U = \{u_1, u_2, \cdots, u_m\}\) be a \set of linearly independent vectors.
    Consider the \set \(U \union \mathcal{B}\) and apply a reduction until it is a \basis. 
\end{snippetproof}

\begin{snippetdefinition}{linear-subspace-complement-definition}{Linear subspace complement}
    Let \(V\) be a finitely dimensional \vectorspace and \(W\) a linear subspace of \(V\).
    Then, a \emph{complement} of \(W\) in \(V\) is a linear subspace
    \(U\) of \(V\) such that \(V = W \oplus U\).
\end{snippetdefinition}

\begin{snippetproposition}{linear-subspace-complement-existance}{}
    Let \(V\) be a finitely dimensional \vectorspace and \(W\) a linear subspace of \(V\).
    Then, \(W\) has a complement.
\end{snippetproposition}

\begin{snippetproof}{linear-subspace-complement-existance-proof}{linear-subspace-complement-existance}{}
    We know that \(V\) and \(W\) are both finitely dimensional
    and \(W\) admits a \basis \(\mathcal{B}\) and \(V\) admits a \basis
    \[
        \mathcal{L} = \mathcal{B} \union U, \quad U = \bigcup_i \{u_i\}
    \]
    We want to show that \(W+U=V\) and \(W \intersection U = \{0\}\).
    Let \(v\in V\). \(\mathcal{L}\) generates \(V\) since it is a \basis
    \[
        v = \sum_{i=1}^n \alpha_i w_i \in W + U
    \]
    Let now \(v \in U \intersection W\), meaning
    \begin{align*}
        v &= \sum_{i=1}^n \alpha_i u_i = \sum_{i=1}^n \beta_i w_i \\
        0 &= \sum_{i=1}^n \alpha_i u_i + \sum_{i=1}^m (-\beta_i)w_i
    \end{align*}
    for some \(\alpha_i, \beta_i \in \mathbb{F}\).
    But \(\mathcal{L}\) is linearly independent, meaning \(\alpha, \beta = 0\)
    and \(v=0\).
\end{snippetproof}

\begin{snippetproposition}{basis-have-same-cardinality}{Basis have same cardinality}
    Let \(V\) be a finitely dimensional \vectorspace and \(\mathcal{B}_1, \mathcal{B}_2\) be two \basis
    of \(V\). Then, \(\cardinality{\mathcal{B}_1} = \cardinality{\mathcal{B}_2}\).
\end{snippetproposition}

\begin{snippetproof}{basis-have-same-cardinality-proof}{basis-have-same-cardinality}{Basis have same cardinality}
    Since \(\mathcal{B}_1\) is linearly independent and \(\mathcal{B}_2\)
    generates the \vectorspace, \(\cardinality{\mathcal{B}_1} \leq \cardinality{\mathcal{B}_2}\).
    Likewise, \(\cardinality{\mathcal{B}_1} \geq \cardinality{\mathcal{B}_2}\) meaning that they are equal.
\end{snippetproof}

\begin{snippetdefinition}{vector-space-dimension-definition}{Vector space dimension}
    Let \(V\) be a \vectorspace. Then, its \emph{dimension} is defined as
    \[
        \text{dim}(V) \triangleq \cardinality{\mathcal{B}}
    \]
    where \(\mathcal{B}\) is a \basis.
\end{snippetdefinition}

\begin{snippetproposition}{linear-subspace-has-lower-dimension}{}
    Let \(V\) be a \vectorspace and \(U\) a linear subspace of \(V\).
    Then,
    \[
        \lineardim(U) \leq \lineardim(V)
    \]
\end{snippetproposition}

\begin{snippetproof}{linear-subspace-has-lower-dimension-proof}{linear-subspace-has-lower-dimension}{}
    A \basis of \(U\) is linearly independent on \(V\), a \basis of \(V\) generates \(V\),
    meaning that a \basis of \(U\) has at most as many elements as a \basis of \(V\).
\end{snippetproof}

\begin{snippetproposition}{linearly-independent-set-of-dim-is-basis}{}
    Let \(V\) be a finitely dimensional \vectorspace where \(\lineardim(V) = n\).
    Then, if \(\{v_1, v_2, \cdots, v_n\}\) is \linearlyindependent,
    it is also a \basis.
\end{snippetproposition}

\begin{snippetproof}{linearly-independent-set-of-dim-is-basis-proof}{linearly-independent-set-of-dim-is-basis}{}
    Since \(\{v_1, v_2, \cdots, v_n\}\) is \linearlyindependent, it can be expanded into a \basis.
    However, since the size of every \basis is \(n\), such a collection is already a \basis.
\end{snippetproof}

\begin{snippetproposition}{generating-set-of-dim-is-basis}{}
    Let \(V\) be a finitely dimensional \vectorspace and let \(\lineardim(V) = n\).
    Then, if \(\{v_1, v_2, \cdots, v_n\}\) is a generating set of \(V\), it is also a \basis of \(V\).
\end{snippetproposition}

\begin{snippetproof}{generating-set-of-dim-is-basis-proof}{generating-set-of-dim-is-basis}{}
    Since \(\{v_1, v_2, \cdots, v_n\}\) is a generating set, we can reduce it into a \basis.
    However, since the size of every \basis is \(n\), such a collection is already a \basis.
\end{snippetproof}

\begin{snippettheorem}{grassman-formula-theorem}{Grassman Formula}
    Let \(V\) be a finitely dimensional \vectorspace
    and let \(W, U\) be linear subspaces of \(V\). Then,
    \[
        \lineardim(W + U)
        = \lineardim(W) + \lineardim(U) - \lineardim(U \intersection W)
    \]
\end{snippettheorem}

\begin{snippetproof}{grassman-formula-theorem-proof}{grassman-formula-theorem}{}
    We first note that \(U, W \in U + W\) and \(U \intersection W \in U, W\).
    In particular, \(U \intersection W\) is the smallest.
    Let \(\mathcal{B}_1 = \{v_1, v_2, \cdots, v_n\}\) be a \basis for \(U \intersection W\),
    meaning that \(\lineardim(U \intersection W) = n\). We note that \(\mathcal{B}_1\) is \linearlyindependent
    in \(U\), meaning that it can be expanded into a \basis for \(U\)
    \[
        \mathcal{B}_U = \mathcal{B}_1
        \union \bigcup_i^j u_i
    \]
    meaning that \(\lineardim(U) = n + j\).
    Likewise, we can do the same for \(W\)
    \[
        \mathcal{B}_W = \mathcal{B}_1
        \union \bigcup_i^k w_i
    \]
    meaning that \(\lineardim(W) = n + k\).
    We want to show that \(\mathcal{B} = \mathcal{B}_U + \mathcal{B}_W\) is a \basis
    for \(U + W\).
    We note that \(\mathcal{B}\) generates \(U + W\). Let \(u+w \in U+ W\)
    \begin{align*}
        u &= \sum_i^n \alpha_i v_i
        + \sum_i^j \beta_i u_i \\
        w &= \sum_i^n \gamma_i v_i
        + \sum_i^k \delta_i w_i \\
        u+v &= \sum_i^n (\alpha_i + \gamma_i) v_i
        + \sum_i^j \beta u_i
        + \sum_i^k \delta_i w_i
    \end{align*}
    We now need to show that it is \linearlyindependent.
    Suppose
    \begin{align*}
        \sum_{i=1}^n \alpha_i v_i
        + \sum_{i=1}^j \beta_i u_i
        + \sum_{i=1}^k \delta_i w_i &= 0 \\
        \underbrace{\sum_{i=1}^n \alpha_i v_i
        + \sum_{i=1}^j \beta_i u_i}_{\in U}
        + \underbrace{\sum_{i=1}^k \delta_i w_i}_{\in W} &= 0
    \end{align*}
    so both sides lie in \(U \intersection W = \linearspan(\mathcal{B}_1)\).
    We know that \(\{v_i, u_i\}\) and \(\{v_i, w_i\}\) are linearly independent in \(U\) and \(W\) respectively,
    meaning that \(\alpha_i = 0, \beta_i = 0\) and \(\delta_i = 0\) for every \(i\).
    Since \(\mathcal{B}\) is a \basis,
    \begin{align*}
        \lineardim(U + W) &= n + j + k \\
        &= n + j + n + j - n \\
        &= \lineardim(W) + \lineardim(U) - \lineardim(U \intersection W)
    \end{align*}
\end{snippetproof}

\begin{snippetdefinition}{algebra-definition}{Algebra}
    Let \(\mathbb{F}\) be a \field and \(A\) a \set and let
    \begin{align*}
        + \colon A \cartesianprod A \fromto A \\
        \circ \colon A \cartesianprod A \fromto A \\
        \cdot \colon F \cartesianprod A \fromto A
    \end{align*}
    Then, \(A\) is a \emph{\(\mathbb{F}\)-algebra} if:
    \begin{enumerate}
        \item \((A, +, \circ)\) is a \ring;
        \item \((A, +, \cdot)\) is a \vectorspace over \(\mathbb{F}\);
        \item \(\forall a,b \in A, \forall \lambda \in \mathbb{F}, (\lambda \cdot a) \circ b = \lambda \cdot (a\circ b)\).
    \end{enumerate}
\end{snippetdefinition}

\end{document}
\documentclass[preview]{standalone}

\usepackage{amsmath}
\usepackage{amssymb}
\usepackage{tikz}
\usepackage{graphicx}
\usepackage{pgfplots}
\usepackage{amsmath}
\usepackage{stellar}
\usepackage{definitions}
\usepackage{bettelini}

\begin{document}

\id{vector-spaces}
\genpage

\section{Definitions}

\includesnpt{vector-space-definition}

\begin{snippetproposition}{field-is-vector-space-with-itself}{}
    Let \(\mathbb{F}\) be a \field. Then, \(\mathbb{F}\) is a
    \vectorspace over \(\mathbb{F}\).
\end{snippetproposition}

\begin{snippetproposition}{field-cartesian-prod-vector-space}{}
    Let \(\mathbb{F}\) be a \field. Then, \({\mathbb{F}}^n\) is a
    \vectorspace over \(\mathbb{F}\).
\end{snippetproposition}

\begin{snippetdefinition}{linear-subspace-definition}{Linear subspace}
    Let \(V\) be a \vectorspace over \(\mathbb{F}\). A \set \(U \subseteq V\)
    is a \emph{linear subspace} of \(V\) if \(U\) is a \vectorspace over \(\mathbb{F}\)
    with the operations of \(V\).
\end{snippetdefinition}

\begin{snippettheorem}{linear-subspace-criterion-theorem}{}
    Let \(V\) be a \vectorspace over \(\mathbb{F}\).
    A \set \(U \subseteq V\) is a linear subspace \ifandonlyif
    \begin{enumerate}
        \item \(0_V \in U\);
        \item \(u,v \in U \implies u+v \in U\);
        \item \(\lambda \in \mathbb{F}, u\in U \implies \lambda \cdot u \in U\).
    \end{enumerate}
\end{snippettheorem}

\begin{snippetproof}{linear-subspace-criterion-theorem-proof}{linear-subspace-criterion-theorem}{}
    \iffproof{
        If \(U\) is a linear subspace, then the properties are satisfied by definition.
    }{
        By the listed properties, we have:
        \begin{enumerate}
            \item the additive identity of \(V\) is in \(U\);
            \item addition is closed and well-defined in \(U\);
            \item scalar multiplication is closed and well-defined in \(U\).
        \end{enumerate}
        Given \(u\in U\), \(-1 \cdot u = -u\) is the inverse, which is closed \(-u \in U\).
        The operations of addition, product and scalar multiplication work in \(V\), so they also work in \(U\),
        and by the properties they are also closed in \(U\).
    }
\end{snippetproof}

\begin{snippetproposition}{linear-subspace-intersection}{Linear subspaces intersection}
    Let \(V\) be a \vectorspace and \(U, W\) be linear subspaces of \(V\).
    Then, \(U \intersection W\) is a lienar subspace of \(V\), \(U\) and \(W\).
\end{snippetproposition}

\begin{snippetproof}{linear-subspace-intersection-proof}{linear-subspace-intersection}{Linear subspaces intersection}
    Let \(\mathbb{F}\) be the \field over which \(V\) is defined.
    \begin{enumerate}
        \item the additive identity \(0_V \in U \intersection W\);
        \item given \(u,w \in U \intersection W\), then \(u + w \in U\) but also
        \(u + w \in W\), meaning \(u + v \in U \intersection W\);
        \item given \(u \in U \intersection W\) and \(\lambda \in \mathbb{F}\),
        then \(\lambda u \in U\) but also \(\lambda u \in W\), meaning
        \(\lambda u \in U \intersection W\).
    \end{enumerate}
\end{snippetproof}

\plain{The same does not work for the union, so we want to construct the smallest linear subspace
containing two subspaces.}

\begin{snippetdefinition}{vector-space-sum-definition}{Vector space sum}
    Let \(U_1, U_2, \cdots, U_n\) be subsets of
    a \vectorspace \(V\). Then, their \emph{sum}
    is given by
    \[
        U_1 + U_2 + \cdots + U_n = \{
            u_1 + u_2 + \cdots + u_n \suchthat u_1\in U_1, u_2\in U_2, \cdots, u_n\in U_n    
        \}
    \]
\end{snippetdefinition}

\includesnpt{linear-combination-definition}

\includesnpt{span-definition}

\includesnpt{linear-independence-definition}

\includesnpt{basis-definition}

% We need a Normed vector space for magnitude (+ unitary vector?)



\end{document}
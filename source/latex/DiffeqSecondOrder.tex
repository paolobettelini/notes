\documentclass[preview]{standalone}

\usepackage{amsmath}
\usepackage{amssymb}
\usepackage{parskip}
\usepackage{fullpage}
\usepackage{hyperref}
\usepackage{stellar}
\usepackage{bettelini}

\hypersetup{
    colorlinks=true,
    linkcolor=black,
    urlcolor=blue,
    pdftitle={Differential Equations},
    pdfpagemode=FullScreen,
}

\begin{document}

\id{diffeq-second-order}
\genpage

\section{Second-Order Differential Equations}

\begin{snippet}{diffeq-second-order-defintion}
A second-order differential equation has the form
\[
    y''(t)+a(t)y'(t)+b(t)y(t)=f(t)
\]
if \(f(t)=0\) then the equation is said to be \textit{homogeneous}.
\end{snippet}

\section{Principle of superposition}

\begin{snippettheorem}{diffeq-superposition-principle}{Principle of superposition}
    If \(y_1(t)\) and \(y_2(t)\) are solutions to a linear homogeneous,
    then so is
    \[
        y(t) = c_1 y_1(t) + c_2 y_2(t)
    \]
\end{snippettheorem}

\section{Characteristic equation}

\begin{snippet}{diffeq-characteristic-equation}
Consider a linear, homogeneous differential equation of the form
\[
    a_n y^{(n)} + a_{n-1} y^{(n-1)} + \cdots + a_0y = 0
\]

We may try to solve it by assuming that the solution(s) will have the form \(e^{rt}\),
meaning
\begin{align*}
    a_n r^n e^{rt}
    + a_{n-1} r^{n-1} e^{rt} + \cdots + a_0 e^{rt} & = 0 \\
    e^{rt} \left(
        a_n r^n + a_{n-1} r^{n-1} + \cdots + a_0
    \right) &= 0
\end{align*}
Since \(e^{rt} \neq 0\), then
\begin{align*}
    a_n r^n + a_{n-1} r^{n-1} + \cdots + a_0 = 0
\end{align*}
This is now a polynomial of \(n\)-degree whose solutions are
\(r_n\), \(r_{n-1}\), \(\cdots\), \(r_1\).

The solutions to the differential equation are then \(e^{r_n t}\),
\(e^{r_{n-1}t}\), \(\cdots\), \(e^{r_1 t}\).

Thus, by the principle of superposition, a more generic solution is given by
\[
    c_n e^{r_n t} +
    c_{n-1} e^{r_{n-1}t} +
    \cdots +
    c_1 e^{r_1 t}
\]
\end{snippet}

% https://tutorial.math.lamar.edu/Classes/DE/IntroSecondOrder.aspx

\end{document}
\documentclass[preview]{standalone}

\usepackage{amsmath}
\usepackage{amssymb}
\usepackage{stellar}
\usepackage{definitions}
\usepackage{bettelini}
\usepackage{tikz}

\usetikzlibrary{cd}

\begin{document}

\id{commutative-rings}
\genpage

\section{Fields and fractions}

%\begin{snippet}{fields-and-fractions-expl}
%    Let \(\mathbb{K}\) be a \field.
%    \(\mathbb{K}\) and \(\{0_\mathbb{K}\}\)
%    are ideals for \(\mathbb{K}\).
%\end{snippet}

\plain{Given a domain of integrity we want to find the smallest field containing it.}

\begin{snippetdefinition}{field-of-fractions-definition}{Field of fractions}
    Let \(A\) be an integral domain.
    Then, the \emph{field of fractions} or \emph{quotient field}
    \(\mathbb{K} = \text{Quot}(A)\) is the \set of \equivclass[equivalence classes]
    \({[(a,b)]}_\sim\) defined as:
    \begin{enumerate}
        \item \(0_\mathbb{K} = {[(0_A,b)]}_\sim\);
        \item \(1_\mathbb{K} = {[(a,a)]}_\sim\) with \(a \neq 0_A\);
        \item \({[(a_1,b_1)]}_\sim + {[(a_2,b_2)]}_\sim \triangleq {[(a_1b_2 + a_2b_1,b_1b_2)]}_\sim\);
        \item \({[(a_1,b_1)]}_\sim \cdot {[(a_2,b_2)]}_\sim \triangleq {[(a_1a_2,b_1b_2)]}_\sim\);
    \end{enumerate}        
\end{snippetdefinition}

\begin{snippetproposition}{field-of-fractions-is-field}{}
    \(quotientfield(\mathbb{A})\) a \field.
\end{snippetproposition}

\begin{snippetproof}{field-of-fractions-is-field-proof}{field-of-fractions-is-field}{}
    Let us denote \({[(a,b)]}_\sim \in \quotientfield(A)\) as \(a/b\).
    \begin{enumerate}
        \item \emph{closure under addition:} \[
            \left(\frac{a_1}{b_1}\right)
            + \left(\frac{a_2}{b_2}\right)
            = \frac{a_1b_2 + a_2b_1}{b_1b_2} \in \quotientfield(A)
        \]
        \item \emph{closure under addition inverse:}
        \[ -\frac{a}{b} = \left(\frac{-a}{b}\right) \in \quotientfield(A) \]
        \item \emph{closure under multiplication:} \[
            \left(\frac{a_1}{b_1}\right)
            \cdot
            \left(\frac{a_2}{b_2}\right)
            = 
            \left(\frac{a_1a_2}{b_1b_2}\right)\in \quotientfield(A)
        \]
        Note that since \(b_1, b_2 \neq 0\), \(b_1b_2 \neq 0\).
        \item the other properties are given by the properties of the ring operations.
    \end{enumerate}
\end{snippetproof}

\plain{The domain of integrity can be seen as a subgroup of the quotient field.}

\begin{snippetproposition}{universal-property-field-fraction}{}
    Let \(A\) be a domain of integrity and \(\mathbb{K} = \quotientfield(A)\).
    For each \injective \ringhomomorphism \(\varphi \colon A \fromto \mathbb{L}\)
    where \(\mathbb{L}\) is a \field viewed as a \ring,
    there exist a unique \ringhomomorphism \(\tilde{\varphi} \colon \mathbb{K} \fromto \mathbb{K}\)
    such that \(\varphi = \tilde{\varphi} \circ i\).
    \begin{center}
        % https://tikzcd.yichuanshen.de/#N4Igdg9gJgpgziAXAbVABwnAlgFyxMJZABgBpiBdUkANwEMAbAVxiRAEEQBfU9TXfIRQAmclVqMWbADrSAtnRwALAEYrgAaS7deIDNjwEiARlLHx9Zq0QhZC5WuAAZbV3EwoAc3hFQAMwAnCDkkMhAcCCRTcLosBjYlCAgAaxBqSykbLB1-IJDEMIikUQkrGWl6ALQlbOoGOhUYBgAFfkMhEACsTyUcHJBA4KjqIsQSjOtbaTwGWGBZSuqsVwouIA
        \begin{tikzcd}
        A \arrow[rr, "i", hook] \arrow[rd, "\varphi"'] &            & \mathbb{K} \arrow[ld, "\tilde{\varphi}"] \\
                                                    & \mathbb{L} &                                         
        \end{tikzcd}
    \end{center}
\end{snippetproposition}

\begin{snippetproof}{universal-property-field-fraction-proof}{universal-property-field-fraction}{}
    Given \(a'/b' \sim a/b\) we have \(a'b-b'a = 0\).
    We thus define
    \[
        \tilde{\varphi}\left(\frac{a'}{b'}\right) \triangleq \varphi(a') \cdot {\varphi(b')}^{-1}
    \]
    and we get
    \begin{align*}
        \tilde{\varphi}\left(\frac{a'}{b'}\right) - \tilde{\varphi}\left(\frac{a}{b}\right) 
        = \left(
            \varphi(a') \cdot {\varphi(b')}^{-1} - \varphi(a) {\varphi(b)}^{-1}
        \right)
    \end{align*}
    We want to show that this value is null. We umltiply by \(\varphi(b)\varphi(b')\) on both sides:
    \begin{align*}
        \varphi(b)\varphi(b') \left(
            \tilde{\varphi}\left(\frac{a'}{b'}\right) - \tilde{\varphi}\left(\frac{a}{b}\right)
        \right)
        &=
        \varphi(b)\varphi(b')
        \left(
            \varphi(a') \cdot {\varphi(b')}^{-1} - \varphi(a) {\varphi(b)}^{-1}
        \right) \\
        &= \varphi(a')\varphi(b) - \varphi(a)\varphi(b') = 0 
    \end{align*}
    Since we are working in a domain of integrity
    and \(\varphi(b)\varphi(b')\neq 0\), the other term must be null and thus
    \[ \tilde{\varphi}\left(\frac{a'}{b'}\right) - \tilde{\varphi}\left(\frac{a}{b}\right)  = 0 \]
    We will now show that it is a \ringhomomorphism:
    \begin{enumerate}
        \item \[
            \tilde{\varphi}(0_{\mathbb{K}})
            = \tilde{\varphi}(\frac{0_A}{b}) = \varphi(0_A) \cdot {\varphi(b)}^{-1} = 0_{\mathbb{L}} \cdot
            {\varphi(b)}^{-1} = 0_{\mathbb{L}}
        \]
        \item \[
            \tilde{\varphi}(1_{\mathbb{K}}) = \tilde{\varphi}\left(\frac{1_A}{1_A}\right)
            = \varphi(1_A) \cdot \varphi(1_A)^{-1} = 1_{\mathbb{L}}
        \]
        \item \begin{align*}
            \tilde{\varphi}\left(
                \frac{a_1}{b_1}- \frac{a_2}{b_2}
            \right)
            &= \tilde{\varphi}\left(
                \frac{a_1b_2 - a_2 b_1}{b_1b_2}
            \right)
            = \varphi(a_1b_2 - a_2b_1) \cdot {\varphi(b_1b_2)}^{-1}\\
            &= \left(\varphi(a_1) \varphi(b_2) - \varphi(a_1)\varphi(b_1)\right)
            {\varphi(b_1)}^{-1}{\varphi(b_2)}^{-1} \\
            &= \varphi(a_1){\varphi(b_1)}^{-1}
            - \varphi(a_2) {\varphi(b_2)}^{-1}
        \end{align*}
        and \[
            \tilde{\varphi}\left(\frac{a_1}{b_1}\right)
            - \tilde{\varphi}\left(\frac{a_2}{b_2}\right)
            = \varphi(a_1){\varphi(b_1)}^{-1}
            - \varphi(a_2) {\varphi(b_2)}^{-1}
        \]
        which are equal
        \item the same for the multiplication.
    \end{enumerate}
\end{snippetproof}

\begin{snippetcorollary}{corollary-placeholder}{}
    The quotient \(\quotientfield(A)\) is the only \field satisfying
    the \snippetref[universal-property-field-fraction][universal property of the field of fraction].
\end{snippetcorollary}

\begin{snippetproof}{corollary-placeholder-proof}{corollary-placeholder}{}
    \begin{center}
        % https://tikzcd.yichuanshen.de/#N4Igdg9gJgpgziAXAbVABwnAlgFyxMJZARgBoAGAXVJADcBDAGwFcYkQBBEAX1PU1z5CKcqWLU6TVuwA6MgLb0cACwBGq4AGluPPiAzY8BImXE0GLNohBzFK9Vu4ByXf0NCiAJjESL067ZKahraPBIwUADm8ESgAGYAThDySKIgOBBIAMw0jPSqMIwACgJGwiCMMHE4IOZSViBYriCJyak0GUhkkpbsTbn5hSXuxtYJWJHKNbzxSSmIaZ2I3j3+jS4zLXNdHZnLdb0BMniMsMByDAloylg6AwXFpR5jE1PNrfMrSzmrDXInZzkaGwdwqg0eI3K40m00o3CAA
        \begin{tikzcd}
                                                & A \arrow[rd, "i"] \arrow[ld, "i"'] \arrow[d, "i'"] &            \\
        \mathbb{K} \arrow[r, "\tilde{\varphi}"'] & \mathbb{K}' \arrow[r, "\tilde{\psi}"']             & \mathbb{K}
        \end{tikzcd}
    \end{center}
    \todo % 13 
\end{snippetproof}

\end{document}
\documentclass[preview]{standalone}

\usepackage{amsmath}
\usepackage{amssymb}
\usepackage{stellar}
\usepackage{definitions}

\newcommand\ts{{(X, \mathcal{T})}}

\begin{document}

\id{topology-connectedness}
\genpage

\section{Connectedness}

\begin{snippetdefinition}{connected-topological-space-definition}{Connected topological space}
    Let \(X\) be a \topologicalspace.
    Then, \(X\) is said to be \emph{connected} if the only subsets that are
    both \topologicalspace[open][Open set] and \closedset[closed]
    are \(\emptyset\) and \(X\).
\end{snippetdefinition}

\plain{If there were other such sets, then the space would be homeomorphic to the disjoint union of the set and its complement.}

\begin{snippetlemma}{connected-topological-space-equivalence}{}
    Let \(X\) be a \topologicalspace.
    Then, \(X\) is not connected \ifandonlyif it is disjoint union
    of two non-empty \topologicalspace[open sets][Open set] (or \closedset[closed sets]).
\end{snippetlemma}

\begin{snippetproposition}{connected-topological-space-subspace}{}
    Let \(X\) be a \topologicalspace and \(Y \subseteq X\) and \(Y \neq \emptyset\) be a connected subspace of \(X\).
    Then, for every \(A \subseteq X\) that is both \topologicalspace[open][Open set] and \closedset[closed],
    we have
    \[
        Y \subseteq A \lxor Y \intersection Y = \emptyset
    \]
    If \(Y = \emptyset\) they are both true.
\end{snippetproposition}

\begin{snippetproof}{connected-topological-space-subspace-proof}{connected-topological-space-subspace}{}
    Since \(A\) is both \topologicalspace[open][Open set] and \closedset[closed],
    \(A \intersection Y\) is \topologicalspace[open][Open set] and \closedset[closed] in \(Y\).
    Since \(Y\) is connected, \(A \intersection Y = \emptyset\) or \(A \intersection Y = Y\)
\end{snippetproof}

\begin{snippetproposition}{connectedness-of-real-unital-interval}{Connectedness of real unital interval}
    Consider \([0,1] \subseteq \realnumbers\) with the subspace topology
    induced by the standard real topology.
    Then, \([0,1]\) is connected.
\end{snippetproposition}

\begin{snippetproof}{connectedness-of-real-unital-interval-proof}{connectedness-of-real-unital-interval}{Connectedness of real unital interval}
    Let \(C, D \subseteq [0,1]\)
    be non-empty \closedset[closed] subspaces of \([0,1]\)
    such that \(C \union D = [0,1]\).
    They are also \closedset[closed] in \(\realnumbers\).
    We want to show that \(C \intersection D \neq \emptyset\).
    Let \(d = \inf D\), which exists since \(D \subseteq [0,1]\).
    If \(d=0\), \(C \intersection D \neq \emptyset\).
    Otherwise, let \(d \neq 0\). Then,
    \([0, d) \subseteq C\). This is given by the fact that \(C \union D = [0,1]\)
    and \(d = \inf D\).
    Since \(C\) is \closedset[closed], \(d \in \overline{[0,d)} \subseteq \overline{C} = C\).
    On the other hand, \(d \in D\) by definition of \(\inf D\).
    Thus \(d \in \overline{D} = D\).
    Finally, \(c\in C \intersection D\) and thus \(C \intersection D \neq \emptyset\).
\end{snippetproof}

\begin{snippetproposition}{continuous-map-preserves-connectedness}{}
    Let \(X,Y\) be \topologicalspace[topological spaces]
    such that \(X\) is connected
    and \(f \colon X \fromto Y\) be a map.
    Then, \(f(X)\) is connected.
\end{snippetproposition}

\begin{snippetproof}{continuous-map-preserves-connectedness-proof}{continuous-map-preserves-connectedness}{}
    Let \(Z \subseteq f(X)\)
    be a non-empty subset of \(f(X)\), \topologicalspace[open][Open set]
    and \closedset[closed] in \(f(X)\). We want to show that \(Z = f(X)\).
    We have \(Z = A \intersection f(X) = C \intersection f(X)\)
    for some \topologicalspace[open set][Open set] \(A\) of \(Y\)
    and some \topologicalspace[open set][Open set] \(C\) of \(Y\).
    Then, \(f^{-1}(Z) = f^{-1}(A) = f^{-1}(C)\).
    Note that \(f^{-1}(Z) \neq \emptyset\).
    Indeed, \(Z = f(f^{-1(Z)})\)
    and \(Z \subseteq f(X)\).
    This is true because we restricted the codomain of \(f\) and it is thus \surjective.
    If it were \(f^{-1}(Z) = \emptyset\) we would have \(Z = f(\emptyset) = \emptyset\),
    which would be absurd \lightning.
    Now, since \(X\) is connected, it follows that
    \[
        f^{-1}(Z) = X
    \]
    But then \(f(X) = f(f^{-1}(Z)) = Z\).
\end{snippetproof}

\subsection{Arc connectedness}

\begin{snippetdefinition}{topological-path-definition}{Topological path}
    Let \(X\) be a \topologicalspace.
    A continuous map \(\varphi \colon [0,1] \fromto X\)
    is said to be a \emph{path}.
\end{snippetdefinition}

\begin{snippetdefinition}{arc-connected-topological-space-definition}{Arc connected topological space}
    Let \(X\) be a \topologicalspace.
    Then, \(X\) is said to be \emph{arc connected} if \(\forall x,y \in X\)
    there exists a path \(\varphi\colon [0,1] \to X\)
    such that \(\varphi(0)=x\) and \(\varphi(1)=y\).
\end{snippetdefinition}

\end{document}
\documentclass[preview]{standalone}

\usepackage{amsmath}
\usepackage{amssymb}
\usepackage{stellar}
\usepackage{definitions}
\usepackage{bettelini}

\begin{document}

\id{diffeq-exercises}
\genpage

\section{Exercises}

\begin{snippetexercise}{cauchy-problem-ex1}{}
    Solve the Cauchy problem
    \[
        \begin{cases}
            y = xy' - \log y' \\
            y(1) = 1
        \end{cases}
    \]
\end{snippetexercise}

\begin{snippetsolution}{cauchy-problem-ex1-sol}{}
    This is a Clairaut equation. We differentiate and obtain
    \begin{align*}
        y' &= y' + xy'' - \frac{y''}{y'} \\
        0 &= y''\left(x - \frac{1}{y'}\right)
    \end{align*}
    The first solution is given by \(y' = C\), hence \(y = Cx - \log C\).
    Substituting the initial point gives \(1 = C - \log C\). We can't solve this analytically,
    but by considering the graphs \(t = \log C\) and \(t = C - 1\), we see they intersect only at \(x = 1\),
    hence \(y = x\) is a solution.

    The second solution is given by \(y' = 1/x\), which is separable, giving \(y = \log x + C\).
    Using the initial condition, we find \(C = 1\), so \(y = \log x + 1\).
\end{snippetsolution}

\begin{snippetexercise}{cauchy-problem-ex2}{}
    \[
        \begin{cases}
            y' = y^2 - \frac{y}{x} - \frac{1}{x^2}     \\
            y(1) = 2
        \end{cases}
    \]
    Hint: a particular solution is \(1/x\) for \(x > 0\).
\end{snippetexercise}

\begin{snippetsolution}{cauchy-problem-ex2-sol}{}
    Let's look for a solution of the form \(y(x) = z(x) + 1/x\).
    Differentiating, we get
    \begin{align*}
        y' &= z' - \frac{1}{x^2} \\
        z' - \frac{1}{x^2} &= z^2 + \frac{1}{x^2} + \frac{2}{x}z - \frac{1}{x}z - \frac{1}{x^2} - \frac{1}{x^2} \\
        z' &= z^2 + \frac{1}{x}z
    \end{align*}
    which is a Bernoulli equation:
    \begin{align*}
        \frac{z'}{z^2} &= 1 + \frac{1}{x} \cdot \frac{1}{z} \\
        w' + \frac{w}{x} &= -1
    \end{align*}
    with \(w = 1/z\) and \(w' = -z'/z^2\).
    We now have a first-order linear equation.

    Thus,
    \begin{align*}
        w(x) &= e^{-\log x} \left[
            C - \frac{x^2}{2}
        \right] \\
        &= \frac{2C - x^2}{2x}
    \end{align*}
    and
    \[
        z(x) = \frac{2x}{2C - x^2}
    \]
    so
    \[
        y(x) = \frac{2x}{2C - x^2} + \frac{1}{x}
    \]
    Using the initial condition, we find \(C = 3/2\).
\end{snippetsolution}

\begin{snippetexercise}{cauchy-problem-ex3}{}
    Solve
    \[
        \begin{cases}
            y'' - \frac{{(y')}^2}{y} - yy' = 0 \\
            y(1) = 1 \\
            y'(1) = 2
        \end{cases}
    \]
\end{snippetexercise}

\begin{snippetsolution}{cauchy-problem-ex3-sol}{}
    Let \(z(y) = y'\), then \(y'' = z'y' = z'z\).
    \begin{align*}
        z'z - \frac{z^2}{y} - yz &= 0 \\
        z\left(z' - \frac{z}{y} - y\right) &= 0
    \end{align*}
    The first solution is \(z = 0\), i.e., \(y' = 0\), but this is inconsistent with the initial data,
    so we discard it. What remains is
    \begin{align*}
        z' - \frac{z}{y} - y = 0
    \end{align*}
    which is a first-order linear equation:
    \[
        z(y) = y\left[
            C + y
        \right] = Cy + y^2
    \]
    Using the initial point, \(2 = C + 1\), so \(C = 1\).
    Substituting back, we get \(y' = y + y^2\),
    hence
    \begin{align*}
        \integral[1][y][
            \frac{1}{t+t^2}
        ][t]
        &= \integral[1][x][1][t] \\
        \log \left(\frac{2y}{1 + y}\right)
        &= x - 1 \\
        y &= \frac{e^{x-1}}{2 - e^{x-1}}
    \end{align*}
\end{snippetsolution}

\begin{snippetexercise}{cauchy-problem-ex4}{}
    Solve
    \[
        \begin{cases}
            y' (x\cos y + \sin(2y)) = 1\\
            y(-2) = \pi
        \end{cases}
    \]
\end{snippetexercise}

\begin{snippetsolution}{cauchy-problem-ex4-sol}{}
    Let's consider \(x\) as a function of \(y\) to simplify the equation.
    So
    \[
        x'(y) = \frac{1}{y'(x)}
    \]
    We get
    \begin{align*}
        \frac{1}{x'} \left(x\cos y + \sin(2y)\right) &= 1 \\
        x\cos y + \sin(2y) &= x'
    \end{align*}
    which is linear:
    \begin{align*}
        x(y) &= e^{\int \cos y \,dy} \left[
            C + \int\sin(2y) e^{-\int \cos y\,dy}\,dy
        \right] \\
        x(y) &= e^{\sin y} \left[
            C - 2\sin y e^{-\sin y} - 2e^{-\sin y}
        \right] \\
        &= Ce^{\sin y} - 2\sin y e^{-\sin y} - 2e^{-\sin y}
    \end{align*}
    Where we used the integral:
    \begin{align*}
        \integral[\sin(2y)e^{-\sin y}][y]
        &= 2\integral[\sin y \cos y e^{-\sin y}][y] \\
        &= 2 \integral[te^{-t}][t] = -2 e^{-t}(t + 1) \\
        &= -2\sin y e^{-\sin y} - 2e^{-\sin y}
    \end{align*}
    Applying the initial condition \(x(\pi) = -2\), we find \(C = 0\).
\end{snippetsolution}

\begin{snippetexercise}{cauchy-problem-ex5}{}
    After finding a particular polynomial solution, solve
    \[
        \begin{cases}
            y' - y^2 - \frac{y}{x} + 9x^2 = 0 \\
            y(1) = \alpha
        \end{cases}
    \]
\end{snippetexercise}

\begin{snippetsolution}{cauchy-problem-ex5-sol}{}
    This is a Riccati equation, so we need a particular solution.
    We look for a polynomial particular solution of degree 1: \(y(x) = ax + b\).
    Using the initial point, take \(a = \alpha\).
    Substituting:
    \begin{align*}
        \alpha - \alpha^2x^2 - 2\alpha bx - b^2 - \alpha - \frac{b}{x} + 9x^2 &= 0
    \end{align*}
    Since \(\alpha = \alpha + b\), we get \(b = 0\)
    \begin{align*}
        -\alpha^2 x^2 + 9x^2 &= 0 \\
        x^2(9 - \alpha^2) &= 0
    \end{align*}
    So \(\alpha = \pm 3\).
    A particular solution is \(y(x) = \pm 3x\).
    Now let \(y(x) = z(x) + 3x\), then \(y' = z' + 3\).
    Substituting:
    \begin{align*}
        z' + 3 - z^2 - 9x^2 - 6xz - \frac{z}{x} - 3 + 9x^2 &= 0 \\
        z' - z^2 - \left(6x + \frac{1}{x}\right)z &= 0
    \end{align*}
    This is a Bernoulli equation. Divide by \(z^2\):
    \begin{align*}
        \frac{z'}{z^2} &= 1 + \left(6x + \frac{1}{x}\right) \frac{1}{z} \\
        w' &= -1 - \left(6x + \frac{1}{x}\right) w
    \end{align*}
    where \(w = 1/z\) and \(w' = -z'/z^2\).
    Therefore,
    \begin{align*}
        w(x) &= e^{-3x^2 - \log x} \left[
            C - e^{3x^2 + \log x}
        \right] \\
        &= \frac{e^{-3x^2}}{x}\left[
            C - \frac{1}{6}e^{3x^2}
        \right] \\
        &= \frac{6Ce^{-3x^2} - 1}{6x}
    \end{align*}
    Substituting back:
    \begin{align*}
        z(x) &= \frac{6x}{6Ce^{-3x^2} - 1} \\
        y(x) &= \frac{6x}{6Ce^{-3x^2}-1} + 3x
    \end{align*}
    Using the initial condition, we find
    \[
        C = \frac{e^3}{\alpha - 3} + \frac{e^3}{6}
    \]
    Now we find for which \(\alpha\) the solution is defined on \((0, \infty)\).
    Since we already have the explicit solution, we can analyze its domain.
    We encounter problems if \(\alpha = 3\), and also need
    \begin{align*}
        6Ce^{-3x^2} - 1 \neq 0 \quad \forall x \in (0, \infty)
    \end{align*}
    If \(C\) is negative, then \(e^{-3x^2}\) never equals \(1/(6C)\).
    Also, if \(1/(6C) > 1\), the condition is satisfied, so we require \(C < 1/6\).
    Substituting gives the corresponding condition on \(\alpha\).
\end{snippetsolution}

\end{document}
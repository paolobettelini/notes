\documentclass[preview]{standalone}

\usepackage{amsmath}
\usepackage{amssymb}
\usepackage{xcolor}
\usepackage{stellar}
\usepackage{definitions}
\usepackage{bettelini}
\usepackage{mathtools}

\begin{document}

\id{topology-continuity}
\genpage

\section{Continuity}

\begin{snippetdefinition}{topology-point-continuity-definition}{Topology continuity at a point}
    Let \((X, {\mathcal{T}}_X)\) and \((Y, {\mathcal{T}}_Y)\) be \topologicalspace[topological spaces].
    A \function \(f\colon X \fromto Y\) is \textit{continuous} at \(x_0\) (with respect to \({\mathcal{T}}_X\)
    and \({\mathcal{T}}_Y\)) if
    for every \neighborhood \(V\) of \(f(x_0)\) there exist a
    \neighborhood \(U\) of \(x_0\) such that \(f(U) \subseteq V\).
\end{snippetdefinition}

\begin{snippetdefinition}{topology-continuity-definition}{Topology continuity}
    Let \((X, {\mathcal{T}}_X)\) and \((Y, {\mathcal{T}}_Y)\) be \topologicalspace[topological spaces].
    A \function \(f\colon X \fromto Y\) is \textit{continuous} (with respect to \({\mathcal{T}}_X\)
    and \({\mathcal{T}}_Y\)) if
    \[
        u \in {\mathcal{T}}_Y \implies f^\inversefunction(u) \in {\mathcal{T}}_X
    \]
\end{snippetdefinition}

\plain{The definition of continuity needs to preserve the structure of the topology.
Continuity intuitively means that nearby points in the domain map to nearby points in the codomain.
This is achieved by ensuring that the preimage of every open set in the codomain is an open set in the domain.
If this preimage condition is met, it means that the function doesn't "break apart" or "tear" the
structure of open sets in the domain when mapping to the codomain.}

\plain{This is also equivalent to saying that the map is continuous at every point.}

\begin{snippetproposition}{topology-continuous-map-characterization}{}
    Let \(f \colon X \fromto Y\)
    be a map between \topologicalspace[topological spaces].
    Then, \(f\) is \topologycontinuous \ifandonlyif
    \[
        \forall A \subseteq X, f(\closure[X][A]) \subseteq \closure[Y][f(A)]
    \]
\end{snippetproposition}

\begin{snippetproof}{topology-continuous-map-characterization-proof}{topology-continuous-map-characterization}{}
    \iffproof{
        By the definition of continuity (with the complement)
        \[
            f(\overline{A}) = \overline{f(A)}
            \iff
            \overline{A} \subseteq f^{-1}(\overline{f(A)})
        \]
        By definition of closure of \(A\) (as the smallest closed set containing \(A\)),
        this is equivalent to saying that
        \[
            A \subseteq f^{-1}(\overline{f(A)})
        \]
        This is equivalent to saying \(f(A) \subseteq \overline{f(A)}\) which is verified.
    }{
        We want to show that for each closed set \(C\), its preimage is also closed.
        Let \(A = f^{-1}(C)\). We have
        \begin{align*}
            f(\overline{A}) = f(\overline{f^{-1}(C)})
            \subseteq \overline{f(f^{-1}(C))} \subseteq \overline{C} = C
        \end{align*}
        But this is equivalent to \(f^{-1}C = \overline{f^{-1}(C)}\)
        which is equivalent to saying that \(f^{-1}(C)\) is closed, which is what we wanted.
    }
\end{snippetproof}

\plain{This means that the composition of continuous functions are also continuous.}

\begin{snippettheorem}{topological-continuous-at-every-point-theorem}{}
    Let \(f \colon X \fromto Y\)
    be a map between \topologicalspace[topological spaces].
    Then, \(f\) is \topologycontinuous \ifandonlyif it is continuous at each point.
\end{snippettheorem}

\begin{snippetproof}{topological-continuous-at-every-point-theorem-proof}{topological-continuous-at-every-point-theorem}{}
    \iffproof{
        Let \(x \in X\) and \(V\) a \neighborhood of \(f(x_0)\).
        By definition
        \[
            V \supseteq A \ni f(x)
        \]
        where \(A\) is \topologicalspace[open][Open].
        By continuity at a point \(x \in f^{-1}(A)\)
        is \topologicalspace[open][Open] as \(f(x) \in A\).
        We can take \(U = f^{-1}(A)\) and thus
        \[
            f(U) = f(f^{-1}(A)) \subseteq A \subseteq V
        \]
    }{
        We want to show that for each \topologicalspace[open][Open] \(A\) of \(Y\),
        \(f^{-1}(A)\) is \topologicalspace[open][Open] in \(X\).
        The condition of opennes can be written as
        \[
            \forall x \in f^{-1}(A), f^{-1}(A) \text{ is a \neighborhood of } X
        \]
        which is equivalent to
        \[
            \forall x \in f^{-1}(A), f^{-1}(A) \subseteq U
        \]
        for some \topologicalspace[open][Open] \(U\) such that \(x \in U\).
        By taking \(V = A\) we know that such a \(U\) exists by definition of continuity of \(f\)
        in \(x\). Thus, there exists a \topologicalspace[open][Open] \neighborhood \(U\)
        in \(x\) such that \(f(U) \subseteq V\)
    }
\end{snippetproof}

\begin{snippetdefinition}{homeomorphism-definition}{Homeomorphism}
    Let \((X, {\mathcal{T}}_X)\) and \((Y, {\mathcal{T}}_Y)\) be \topologicalspace[topological spaces].
    A \function \(f\colon X \fromto Y\) is a \textit{homeomorphism} if it is \topologycontinuous, \bijective
    and \(f^\inversefunction\) is \topologycontinuous.
\end{snippetdefinition}

\plain{Homeomorphisms induce equivalence relations between topological spaces.}

\begin{snippetdefinition}{sequentially-continuous-definition}{Sequentially continuous}
    Let \((X, {\mathcal{T}}_X)\) and \((Y, {\mathcal{T}}_Y)\) be \topologicalspace[topological spaces].
    A \function \(f\colon X \fromto Y\) is \textit{sequentially continuous} if for every
    \(x\in X\), for every sequence \({\{x_n\}}_{n \in \naturalnumbers} \subseteq X\) with \(x_n \to x\)
    we have that \({\{f(x_n)\}}_{n \in \naturalnumbers} \subseteq Y\) converges with
    \(f(x_n) \topologyconverges f(x)\).
\end{snippetdefinition}

\begin{snippetdefinition}{open-closed-topological-map-definition}{Open and closed topological map}
    Let \(f \colon X \fromto Y\)
    be a map between \topologicalspace[topological spaces].
    Then, \(f\) is said to be \emph{open/closed}
    if for every \topologicalspace[open set][Open] / \closedset \(S\),
    \(f(S)\) is \topologicalspace[open][Open] / \closedset[closed].
\end{snippetdefinition}

\begin{snippetproposition}{homeomorphism-equivalent-conditions}{}
    Let \(f \colon X \fromto Y\)
    be a map between \topologicalspace[topological spaces].
    The following are equivalent:
    \begin{enumerate}
        \item \(f\) is a \homeomorphism;
        \item \(f\) is open and \bijective;
        \item \(f\) is closed and \bijective;
    \end{enumerate}
\end{snippetproposition}

\begin{snippetproof}{homeomorphism-equivalent-conditions-proof}{homeomorphism-equivalent-conditions}{}
    \begin{itemize}
        \item \((1) \iff (2\&3):\)
        let \(g^{-1} = f\). We have \(g^{-1}(A) = f(A)\).
        \(f\) is open/closed \ifandonlyif \(g\) is continuos,
        and \(g\) is continuous \ifandonlyif \(\forall A\) \topologicalspace[open][Open] / \closedset[closed]
        \(g^{-1}(A) = f(A)\) is \topologicalspace[open][Open] / \closedset[closed].
    \end{itemize}
\end{snippetproof}

\section{Subspaces}

\begin{snippet}{subspace-topology-expl}
    We want a minimal notion of subspace such that the inclusion map from the subset to the original space is continuous.
    We thus want \(i \colon Y \xhookrightarrow{} X\) is continuous.
    For each \topologicalspace[open set][Open] \(A\),
    \(i^{-1}(A) = A \intersection Y\).
    This is equivalent to saying that all the \(A \cap Y\) are in the topology.

    The closed sets in this topology are the ones that can be written as
    \(i^{-1}(C)\) for \(C\) closed in \(X\).
    \[
        Y \difference i^{-1}(A) = i^{-1}(X \difference A)
    \]
\end{snippet}

\begin{snippetdefinition}{subspace-topology-definition}{Subspace topology}
    Let \((X, \tau)\) be a \topologicalspace and \(Y \subseteq X\).
    Then, the \emph{subspace topology} is defined as
    \[
        \pi_Y = \{Y \intersection U \suchthat U \in \tau \}
    \]
\end{snippetdefinition}

\begin{snippetproposition}{subspace-topology-is-topology}{}
    Let \((X, \tau)\) be a \topologicalspace and \(Y \subseteq X\).
    Then, the subspace topology \(\pi_Y\) is a topology for \(Y\).
\end{snippetproposition}

\begin{snippetproof}{subspace-topology-is-topology-proof}{subspace-topology-is-topology}{}
    Let \(i \colon Y \xhookrightarrow{} X\) be the continuous inclusion map.
    \begin{enumerate}
        \item \(\emptyset \in \pi_Y\) as \(i^{-1}(\emptyset) = \emptyset\);
        \item \(Y \in \pi_Y\) as \(i^{-1}(X) = Y\);
        \item \[
            i^{-1}(A_1) \intersection i^{-1}(A_2) = i^{-1}(A_1 \intersection A_2)
        \]
        \item \[
            i^{-1}\left(
                \bigcup_{A \in \pi_Y} A
            \right)
            =
            \bigcup_{A \in \pi_Y} i^{-1}(A)
        \]
    \end{enumerate}
\end{snippetproof}

\begin{snippetproposition}{subspace-topology-basis}{}
    Let \(\mathcal{B}\) be a \topologicalbasis for \((X, \tau)\).
    Let \(i \colon Y \xhookrightarrow{} X\) be the continuous inclusion map.
    Then,
    \[
        \{i^{-1}(B) \suchthat B \in \mathbb{B}\}
    \]
    is a \topologicalbasis for the subspace \((Y, \pi_Y)\).
\end{snippetproposition}

\begin{snippetproof}{subspace-topology-basis-proof}{subspace-topology-basis}{}
    Let \(A\in \tau\). We have \(A = \bigcup B\) for \(B \in \mathbb{B}\).
    \[
        i^{-1}(A) = i^{-1} \left(
            \bigcup B
        \right)
        = \bigcup i^{-1}(B)
    \]
\end{snippetproof}

\begin{snippetproposition}{subspace-topology-universal-property}{Subspace topology universal property}
    Let \(X, Z\) be \topologicalspace[topological spaces]
    and \(i \colon Y \xhookrightarrow{} X\) be the continuous inclusion map
    of a subspace of \(X\).
    Suppose that \(f \colon Z \fromto Y\) is a map.
    Then, \(i \circ f\) is continuous \ifandonlyif \(f\) is continuous.
\end{snippetproposition}

\begin{snippetproof}{subspace-topology-universal-property-proof}{subspace-topology-universal-property}{Subspace topology universal property}
    \ffiproof{
        This is obvious as the composition of continuous functions are continuous.
    }{
        Suppose that \(i \circ f\) is continuous.
        We want to show that \(f\) is continuous, meaning that \(\forall B\)
        \topologicalspace[open][Open] in \(Y\), \(f^{-1}(B)\) is \topologicalspace[open][Open]
        in \(Z\).
        By definition of subspace topology, \(B = i^{-1}(A)\) for some \topologicalspace[open][Open]
        \(A\) of \(X\).
        \[
            f^{-1}(B) = f^{-1}(i^{-1}(A)) = {(i \circ f)}^{-1}(A)
        \]
        \topologicalspace[open][Open] in \(Z\), since \(i \circ f\) is continuous.
    }
\end{snippetproof}

\begin{snippetproposition}{closure-in-subspace-topology}{}
    Let \(X\) be a \topologicalspace and let \(i \colon Y \xhookrightarrow{} X\)
    be the continuous inclusion map
    of a subspace \(Y\) of \(X\). Let \(S \subseteq Y\). Then,
    \[
        (\closure[X][S]) \intersection Y = \closure[Y][S]
    \]
\end{snippetproposition}

\begin{snippetproof}{closure-in-subspace-topology-proof}{closure-in-subspace-topology}{}
    It suffices to note the following:
    \begin{itemize}
        \item the closure of a subset is the intersection of all the \closedset[closed sets]
        containing the subset;
        \item the \closedset[closed sets] for the subspace topology of \(Y\)
        re precisely the intersecitons with \(Y\)
        of \closedset[closed sets] for the topology \(X\).
    \end{itemize}
\end{snippetproof}

\begin{snippetlemma}{closure-and-opennes-in-topological-subsets}{}
    Let \(X\) be a \topologicalspace, \(Z \subseteq Y \subseteq X\). Then,
    \begin{enumerate}
        \item if \(Y\) is \topologicalspace[open][Open] in \(X\),
        then \(Z\) is \topologicalspace[open][Open] in \(Y\)
        \ifandonlyif \(Z\) is \topologicalspace[open][Open] in \(X\);
        \item if \(Y\) is \closedset[closed] in \(X\),
        then \(Z\) is \closedset[closed] in \(Y\)
        \ifandonlyif \(Z\) is \closedset[closed] in \(X\);
        \item if \(y\in Y\) and \(Y\) is a \neighborhood of \(y\) in \(X\),
        then \(Z\) is a \neighborhood of \(y\) in \(Y\) \ifandonlyif \(Z\)
        is a \neighborhood of \(y\) in \(X\).
    \end{enumerate}
\end{snippetlemma}

\begin{snippetproof}{closure-and-opennes-in-topological-subsets-proof}{closure-and-opennes-in-topological-subsets}{}
    \begin{enumerate}
        \item By definition of subspace, \(Z\) is \topologicalspace[open][Open] in \(Y\) \ifandonlyif
        \(Z = A \intersection Y\) for some \(A\) \topologicalspace[open][Open] in \(X\).
        If \(Y\) is \topologicalspace[open][Open] in \(X\), then \(Z\)
        is \topologicalspace[open][Open] in \(X\) as intersection of two \topologicalspace[open sets][Open]
        in \(X\). The other direction is obvious by definition of subspace topology.
        \item Analogous.
        \item There exist an \topologicalspace[open set][Open] \(U\)
        of \(X\) such that \(y \in U \subseteq Y\).
        SUppose that \(Z\) is a \neighborhood of \(y\) in \(X\).
        We have \(Z \supseteq V \ni y\) where \(V\) is \topologicalspace[open][Open]
        in \(X\). This means that \(Z \intersection Y \supseteq V \intersection Y \ni y\)
        which is also \topologicalspace[open][Open] because it is the intersection of two
        \topologicalspace[open sets][Open]. \\
        Viceversa, suppose that \(Z\) is a \neighborhood
        of \(y\) in \(Y\).
        Then, there is a \(V\) \topologicalspace[open][Open] in \(Y\)
        such that \(y \in V \subseteq Z\).
        Consider \(U\) \topologicalspace[open][Open] in \(X\) such that
        \(Y \supseteq U \ni y\) like before. Since \(U\)
        is included in \(Y\) we can consider it a subspace of both \(Y\) and \(X\).
        These two topologies are the same by transitivity.
        This is because the \topologicalspace[open sets][Open] of such topology
        are the subsets of the form \((A \intersection Y) \intersection U\)
        where \(A\) is \topologicalspace[open][Open] in \(X\).
        However, this is equivalent to the intersection with the smallest subset
        \[
            (A \intersection Y) \intersection U
            = A \intersection U
        \]
        since \(U \subseteq Y \subseteq X\).
        The subset \(U \intersection V\) is \topologicalspace[open][Open]
        of \(U\) (viewed as a subspace of \(Y\)) as \(V\)
        is \topologicalspace[open][Open] in \(Y\).
        However, \(U\) is \topologicalspace[open][Open] in \(X\)
        and thus by \((1)\), \(U \intersection V\) is \topologicalspace[open][Open] of \(X\)
        by transitivity (when considering \(U\) as a subspace of \(X\)).
        Thus, \(y\in U \intersection V \subseteq Z\)
        and \(Z\) is a \neighborhood of \(y\) in \(X\).
    \end{enumerate}
\end{snippetproof}

\begin{snippetdefinition}{topological-immersion-definition}{Topological immersion}
    Let \(f \colon X \fromto Y\)
    be a map between \topologicalspace[topological spaces] that is continuous and \injective.
    Then, \(f\) is said to be a \emph{topological immersion}
    if the \topologicalspace[open sets][Open]
    of \(X\) are precisely the subsets of the form \(f^{-1}(A)\)
    where \(A\) is \topologicalspace[open][Open] in \(Y\).
\end{snippetdefinition}

\end{document}
\documentclass[preview]{standalone}

\usepackage{amsmath}
\usepackage{amssymb}
\usepackage{bettelini}
\usepackage{stellar}

\hypersetup{
    colorlinks=true,
    linkcolor=black,
    urlcolor=blue,
    pdftitle={Chimica},
    pdfpagemode=FullScreen,
}

\begin{document}

\title{Chimica}
\id{chimica-acidi-basi-esercizi}
\genpage

\begin{snippetexercise}{acidi-basi-ex-1}{Indica la base coniugata dei seguenti acidi:}
    \begin{itemize}
        \item \textbf{HNO\(_2\):}
        \item \textbf{HClO:}
        \item \textbf{HPO\(_4^{2-}\):}
        \item \textbf{HF:}
    \end{itemize}
\end{snippetexercise}

\begin{snippetexercise}{acidi-basi-ex-2}{La sostanza HCO\(_3^-\) è anfotera?}
    Sì, perché se la metto in acqua può liberare sia H\({}^+\) che H\({}^-\).
    Prendendo un protone si può trasformare in H\({}_2\)CO\({}_3\),
    oppure può liberare un protone e trasformarsi in CO\({}_3^{2-}\).
\end{snippetexercise}

% hCl no perché non puo prendere H+

\begin{snippetexercise}{acidi-basi-ex-3}{Indica l'acido coniugato delle seguenti basi:}
    \begin{itemize}
        \item \textbf{H\({}_2\)PO\({}_4^-\):} H\(_3\)PO\(_4\);
        \item \textbf{HS\({}^-\):} H\(_2\)S;
        \item \textbf{NH\({}_3\):} NH\(_4^+\);
        \item \textbf{CH\({}_3\)COO\({}^-\):} CH\(_3\)COOH.
    \end{itemize}
\end{snippetexercise}

\begin{snippetexercise}{acidi-basi-ex-4}{Considera le seguenti sostanze:
    NH\(_3\); HI; Ca(OH)\({}_2\); KOH; H\({}_2\)CO\({}_3\) e CH\({}_3\)COOH}
    \begin{enumerate}
        \item \textbf{Stabilisci se si tratta di sostanze acide o basiche e motiva le tue risposte:}
            Gli acidi sono: Hi, H\({}_2\)CO\({}_3\) e CH\({}_3\)COOH
            mentre quelle basiche sono NH\(_3\), Ca(OH)\({}_2\) e KOH.
        \item \textbf{Utilizzando quattro sostanze delle sei sopra indicate, scrivi le equazioni chimiche
        complete e bilanciate di due possibili reazioni acido-base (reazioni di
        neutralizzazione):}
        HI + KOH \(\rightarrow\) KI + H\({}_2\)O
        e
        H\({}_2\)CO\({}_3\) + NH\({}_3\) \(\rightarrow\) HCO\({}_3^-\) + NH\({}_4^+\).
        \item \textbf{Scrivi le reazioni di ionizzazione in acqua (protolisi) di un acido e di una base:}

    \end{enumerate}
\end{snippetexercise}

\begin{snippetexercise}{acidi-basi-ex-5}{}
    Considera un lago contenente acqua pura
    \begin{itemize}
        \item Quant'è la concentrazione [H\({}_3\)O\({}^+\)]?
            \(10^{-7} M\).
        \item Quant'è la concentrazione [OH\({}^-\)]?
            \(10^{-7} M\).
        \item Quant'è il pH? 7.
        \item Quanti grammi di H\({}_3\)O\({}^+\) ci sono?
            Non si può sapere.
        \item Se il volume è un km cubo?
            Sono \(10^12\) di metri cubi, quindi
            \(10^12 \text{L} \cdot 10^{-7} \frac{\text{mol}}{\text{L}} \cdot 19 \text{una}\).
    \end{itemize}
    Viene aggiunta una soluzione 0.1 molare di H\({}_2\)SO\({}_4\).
    \begin{itemize}
        \item Come varia il pH? La sostanza è acida e quindi il pH diminuisce.
    \end{itemize}
\end{snippetexercise}

% Fra Co(OH)2 e NaOH qual'è la base più forte?
% La base più forte è Co(OH)2 perché
%   - sono entrambi basi forti
%   - può liberare due idrossidi > 1

% Al contrario, fra HCl e H2CO3, HCl è il più acido
% nonostante l'altro sia diprotico, perché HCl è forte.

% Come faccio a dimostrarlo per certo? Con la tabella della reazione e l'eq
% quadratica. pH = (pKa - log Concentrazione acida) / 2
% In generale, un acido forte libera sempre di più di un acido (non forte) poliprotico.
% Il /2 è perché è diprotico?

% 

\end{document}

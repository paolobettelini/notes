\documentclass[preview]{standalone}

\usepackage{amsmath}
\usepackage{amssymb}
\usepackage{stellar}
\usepackage{bettelini}

\hypersetup{
    colorlinks=true,
    linkcolor=black,
    urlcolor=blue,
    pdftitle={Stellar},
    pdfpagemode=FullScreen,
}

\begin{document}

\title{Geografia economica}
\id{geoeconomica-migrazioni}
\genpage

\section{Migrazioni}

\begin{snippetdefinition}{migrante-definizione}{Migrante}
    Non esiste una definizione formale di migrante internazionale. General
    mente gli esperti tendono a considerare un migrante una persona che
    modifica il proprio Stato di residenza indipendentemente dalla ragione
    o dallo statuto legale.
\end{snippetdefinition}

\begin{snippetdefinition}{rifugiato-definizione}{Rifugiato}
    Persona che si trova fuori dal proprio paese d'origine per paura di persecuzioni
    (nazionalità, razza, religione, appartenenza ad un gruppo sociale o politico),
    a causa di un conflitto, di violenza generalizzata o altre
    circostanze con impatto sull'ordine pubblico e che, pertanto, richiede
    protezione internazionale.
\end{snippetdefinition}

\begin{snippetdefinition}{richiedente-asilo-definizione}{Richiedente l'asilo}
    Coloro che hanno lasciato il loro paese d'origine e hanno inoltrato una
    richiesta di asilo in un paese terzo, ma sono ancora in attesa di una
    de cisione da parte delle autorità competenti riguardo al riconoscimento
    del loro status di rifugiati.
\end{snippetdefinition}

\begin{snippetdefinition}{profugo-definizione}{Profugo}
    Si tratta di una parola usata in modo generico che deriva dal verbo latino
    profugere, «cercare scampo». Talvolta si intende profugo come
    colui che per diverse ragioni (guerra, povertà, calamità naturali, ecc.) ha
    lasciato il proprio Paese ma non è nelle condizioni di chiedere la prote zione internazionale.
\end{snippetdefinition}

% stocks vs flows TODO

\newcommand{\greenbox}{
    \fcolorbox{black}{green}{\rule{0pt}{5pt}\rule{5pt}{0pt}}
}

\newcommand{\redbox}{
    \fcolorbox{black}{red}{\rule{0pt}{5pt}\rule{5pt}{0pt}}
}

\begin{snippet}{conseguenze-migrazioni}
    Le conseguenze sociali e culturali delle migrazioni
    possono essere sia positivhe che negative, sia per quanto
    riguarda i paesi di partenza che, sia per i paesi di arrivo.
    
    I paesi di arrivo hanno
    \begin{itemize}
        \item \greenbox più forza lavoro e diversità;
        \item \redbox più disoccupazione, difficoltà di integrazione,
        segregazinoe, discriminazione.
    \end{itemize}
    
    I paesi di partenza hanno
    \begin{itemize}
        \item \greenbox rimesse finanziarie;
        \item \redbox perdita di tradizioni,
        cambiamento della struttura demografica (meno nati, più invecchiamenti),
        fuga di cervelli.
    \end{itemize}
\end{snippet}

\end{document}

% Correzione Espe 2
% Scaletta :
% Contestualizzazione
% Breve sintesi
% Elencre motivazioni:
% Spiegare/commentare i perché della tesi dell'autore (es. perché età dell'oro della glob.?)
% Situazione post 2016
% Perché conferma/smentisce autore

Spaendo che il testo è stato redatto da Khanna, nel 2016, quali sono
gli elementi del contesto che ci permettono di interpretare
correttamente il documento

% guardare il file su moodle
\documentclass[preview]{standalone}

\usepackage{amsmath}
\usepackage{amssymb}
\usepackage{stellar}
\usepackage{definitions}
\usepackage{bettelini}

\begin{document}

\id{trigonometry}
\genpage

\section{Trigonometric circle}

\begin{snippet}{trigonometric-circle-illustration}
    \begin{center}
        \begin{tikzpicture}[scale=3.75]
            \definecolor{darkgreen}{rgb}{0.0, 0.7, 0.0}
            % Draw the x and y axes
            \draw[thick, ->] (-1.5,0) -- (1.5,0) node[right] {$x$};
            \draw[thick, ->] (0,-1.5) -- (0,1.5) node[above] {$y$};
        
            % Draw the unit circle
            \draw (0,0) circle (1);
        
            % Draw the angle theta
            \draw[thick, blue] (0,0) -- (0.866,0.5) node[right] {$P(\cos\theta,\sin\theta)$};
            \draw[dashed, blue] (0,0) -- (0.866,-0.5) node[right] {$P'(\cos\theta,-\sin\theta)$};
        
            % Draw dashed lines to indicate the projections on the axes
            \draw[dashed] (0.866,-0.5) -- (0.866,0.5) -- (0,0.5);
        
            % Label the angle theta
            \draw (0.3,0) arc (0:30:0.3);
            \node at (0.35,0.10) {$\theta$};
        
            % Label the origin
            \node at (-0.06, -0.06) {$O$};
        
            % Draw the projections on the axes
            \draw[thick, red] (0,0) -- (0.866, 0) node[midway, below] {$\cos \theta$};
            \draw[thick, darkgreen] (0.866,0) -- (0.866, 0.5) node[midway, left] {$\sin \theta$};
        
            % Additional labels and lines
            %\node at (-1.075, 0.05) {-1};
            \node at (-0.1, 1.1) {\(U_2\)};
            \node at (1.075, 0.05) {\(U_1\)};
            %\node at (-0.05, -1.1) {\(U_2\)};
        
            % Labels for each quadrant
            \node at (1, 1) {I};
            \node at (-1, 1) {II};
            \node at (-1, -1) {III};
            \node at (1, -1) {IV};
    
            %others
            \draw (0.866,0) rectangle (0.816,0.05);
        \end{tikzpicture}
    \end{center}
\end{snippet}

\section{Laws}

\begin{snippettheorem}{law-of-sines}{Law of sines}
    Given a triangle with sides \(a\), \(b\) and \(c\) and their respective opposite angles
    \(\alpha\), \(\beta\) and \(\gamma\)
    \[
        \frac{\sin(\alpha)}{a} =
        \frac{\sin(\beta)}{b} =
        \frac{\sin(\gamma)}{c}
    \]
\end{snippettheorem}

\begin{snippettheorem}{law-of-cosines}{Law of cosines}
    Given a triangle with sides \(a\), \(b\) and \(c\)
    \[
        c^2 = a^2 + b^2 - 2ab\cos\gamma
    \]
    where \(\gamma\) is the angle between \(a\) and \(b\) (opposite of \(c\)).
\end{snippettheorem}

\section{Trigonometric Identities}

\begin{snippettheorem}{pythagorean-identity-theorem}{Pythagorean identity}
    For any \(\theta \in \realnumbers\),
    \[
        \sin^2\theta + \cos^2\theta = 1
    \]
\end{snippettheorem}

\begin{snippetproposition}{double-angle-sine}{Double-angle sine}
    Let \(\theta \in \realnumbers\).
    \[ \sin(2\theta) = 2 \sin\theta\cos\theta \]
\end{snippetproposition}

\begin{snippetproposition}{sine-angle-addition}{Sine angle addition}
    Let \(\alpha,\beta \in \realnumbers\).
    \[ \sin(\alpha \pm \beta) = \sin\alpha\cos\beta \pm \cos\alpha\sin\beta \]
\end{snippetproposition}

\begin{snippetproposition}{cosine-angle-addition}{Cosine angle addition}
    Let \(\alpha,\beta \in \realnumbers\).
    \[ \cos(\alpha \pm \beta) = \cos\alpha\cos\beta \mp \sin\alpha\cos\beta \]
\end{snippetproposition}

\end{document}
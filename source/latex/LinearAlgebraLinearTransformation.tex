\documentclass[preview]{standalone}

\usepackage{amsmath}
\usepackage{amssymb}
\usepackage{stellar}
\usepackage{definitions}

\hypersetup{
    colorlinks=true,
    linkcolor=black,
    urlcolor=blue,
    pdftitle={Stellar},
    pdfpagemode=FullScreen,
}

\begin{document}

\title{Stellar}
\id{linearalgebra-linear-transformation}
\genpage

\section{Linear Transformation}

\subsection{Definition}

\begin{snippetdefinition}{linear-transformation-definition}{Linear Transformation}
    Let \(\mathcal{V}\) and \(\mathcal{W}\) be \vectorspace[vector spaces] over a \field \((K, +, \cdot)\).
    A \function \(f \colon \mathcal{V} \fromto \mathcal{W}\) is a
    \emph{linear transformation} if
    \begin{enumerate}
        \item \(\forall u,v \in V, f(u + v) = f(u) + f(v)\);
        \item \(\forall c\in K, f(c\cdot u) = c\cdot f(u)\).
    \end{enumerate}
\end{snippetdefinition}

\begin{snippetdefinition}{linear-projection-definition}{Linear projection}
    Let \(\mathcal{V}\) be a \vectorspace. Then,
    a \lineartransformation \(P \colon \mathcal{V} \fromto \mathcal{V}\) is a
    \emph{linear projection} if \(P \circ P = P\).
\end{snippetdefinition}

\subsection{The basis}

%\plain{Multiplying a vector by a matrix produces another vector. This is a linear transformation.
%The matrix contains the information about the transformation.}

\begin{snippettheorem}{linear-transformation-on-basis}{Linear transformation on basis}
    Given a \basis for a \vectorspace

    \[
        \mathcal{B}=\{\vec{b}_1, \vec{b}_2, \ldots, \vec{b}_n\}
    \]
    and a linear transformation \(T\), it suffices
    to know the value of \(T\mathcal{B}=\{T\vec{b}_1, T\vec{b}_2, \ldots, T\vec{b}_n\}\)
    to determine \(T\) applied to any vector on the vector space.
    Any transformation \(T\) is completely described by \(\mathcal{B}\)
    and \(T\mathcal{B}\).
\end{snippettheorem}

\begin{snippetproof}{linear-transformation-on-basis-proof}{linear-transformation-on-basis}{Linear transformation on basis}
    Given a \basis for a \vectorspace

    \[
        \mathcal{B}=\{b_1, b_2, \ldots, b_n\}
    \]

    we can expand a vector \(a\) along this basis

    \[
        a = \sum_{k=1}^{n} \alpha_k b_k
    \]

    We then apply a transformation \(T\) to the vector \(a\) and use the properties of
    linear transformations
    
    \[
        Ta
        = T\sum_{k=1}^{n} \alpha_k b_k
        = \sum_{k=1}^{n} \alpha_k Tb_k
    \]
\end{snippetproof}

\begin{snippet}{matrix-values-expl}
    Each column of a matrix is indeed the result of applying its transformation
    to the corresponding vector in the basis.
    Intuitively, this is given by the fact that we only need to know where the vectors of the basis
    end up after the transformation in order to represent the whole information.
\end{snippet}

\begin{snippetproposition}{kernel-of-linear-transformation-is-subspace}{}
    Let \(f \in \setoflineartransformations(V, W)\). Then, \(\grpker f\) is a linear subspace of \(V\).
\end{snippetproposition}

\begin{snippetproof}{kernel-of-linear-transformation-is-subspace-proof}{kernel-of-linear-transformation-is-subspace}{}
    \begin{enumerate}
        \item \(f(0_V) = 0_W \in \grpker f\) since \(f(0_V) = f(0_V + 0_V) = f(0_V) +f(0_V)\);
        \item let \(u,v \in \grpker f\). Then,
        \[
            f(u+v) = f(u) + f(v) = 0_W + 0_W = 0_W \in \grpker f
        \]
        \item let \(u \in \grpker f\) and \(\lambda \in \mathbb{F}\). Then,
        \[
            f(\lambda u) = \lambda f(u) = \lambda 0_W = 0_W \in \grpker f
        \]
    \end{enumerate}
\end{snippetproof}

\begin{snippetproposition}{linear-endomorphisms-are-ring}{}
    The structure \((\setoflineartransformations(V, V), \circ, +)\) is a \ring.
\end{snippetproposition}

\end{document}
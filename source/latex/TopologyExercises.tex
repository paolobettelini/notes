\documentclass[preview]{standalone}

\usepackage{amsmath}
\usepackage{amssymb}
\usepackage{stellar}
\usepackage{definitions}
\usepackage{bettelini}

\begin{document}

\id{topology-exercises}
\genpage

\section{Exercises}

% foglio 1

\begin{snippetexercise}{topology-exercise-ex-1}{}
    How many topologies are there on \(\emptyset, \{a\}, \{a,b\}\)?
\end{snippetexercise}

\begin{snippetsolution}{topology-exercise-ex-1-sol}{}
    \begin{enumerate}
        \item The only possible topology for \(\empty\) is \(\{\emptyset\}\).
        \item The topology for \(\{a\}\) has at least the element \(\{\emptyset, \{a\}\}\).
        No element can be added or removed so there is only one topology.
        \item We have \(\{\emptyset, \{a,b\}\}, \{\emptyset, \{a,b\}, \{a\}\}, \{\emptyset, \{a,b\}, \{b\}\}, \{\emptyset, \{a,b\}, \{a\}, \{b\}\}\).
    \end{enumerate}
\end{snippetsolution}

\begin{snippetexercise}{topology-exercise-ex-2}{}
    Let \(X\) be an infinite \set and consider the topology
    \[
        \tau_X = \{A \subseteq X \suchthat X \difference A \text{ is finite}\}
        \union \{\emptyset\}
    \]
    Does every pair of non-empty open sets have an empty intersection? 
\end{snippetexercise}

\begin{snippetsolution}{topology-exercise-ex-2-sol}{}
    The open sets of \(\tau_X\) must be infinite. Since
    \[
        X \difference (A \intersection B)
        = (X \difference A) \union (X \difference B)
    \]
    meaning that \(A \intersection B\) is still infinite,
    since the two differences are finite.
\end{snippetsolution}

\begin{snippetexample}{topology-exercise-ex-3}{}
    Let \(X\) be a \set and \(\infty \in X\). Verify whether
    \[
        \tau_X = \{
            A \subseteq X \suchthat \infty \notin A \lor X \difference A \text{ is finite}    
        \}
    \]
    is a possible topology.
\end{snippetexample}

\begin{snippetsolution}{topology-exercise-ex-3-sol}{}
    \begin{enumerate}
        \item \(\emptyset \in \tau_X\) since \(\infty \notin \emptyset\);
        \item \(X \in \tau_X\) since \(X \difference X = \emptyset\) is finite;
        \item \(\forall A, B \in \tau_X\),
        if \(\infty \notin A \intersection B\) then we are done.
        Otherwise, \(\infty \in A \land \infty \in B\).
        In this case note that
        \[
            X \difference (A \intersection B)
            = (X \difference A) \union (X \difference B)
        \]
        which is the union of two finite sets.
        \item Let \(\{A_i\}_{i \in I}\) be a family of elements of \(\tau_X\),
        if \(\infty \notin \bigcup A_i\) then we are done.
        Otherwise, \[
            X \difference \left(\bigcup_{i \in I} A_i \right)
            = \bigcap_{i\in I} \left(X \difference A_i\right)
        \]
        which is finite as at least one of the \(X \difference A_i\)
        is finite because it contains \(\infty\).
    \end{enumerate}
\end{snippetsolution}

\begin{snippetexercise}{topology-exercise-ex-4}{}
    Consider a \partialorder[partially ordered] \set \((X, \leq)\).
    Let
    \[
        M_x = \{
            y \in X \suchthat x \leq y    
        \}
    \]
    Show that \(\{M_x \suchthat x \in X\}\) forms a \topologicalbasis for \(X\).
\end{snippetexercise}

\begin{snippetsolution}{topology-exercise-ex-4-sol}{}
    We note the obvious double inclusion
    \[
        \tau_X = \left\{
            A \suchthat
            A = \bigcup_{i \in I} A_i = \bigcup_{M_x \subseteq A} M_x
        \right\}
    \]
    We have:
    \begin{enumerate}
        \item \[
            X = \bigcup_{x\in X} M_x
        \]
        \item we now want to show that \(M_x \intersection M_x = \bigcup M_x\).
        We have
        \begin{align*}
            M_x \intersection M_y
            &= \{z \in X \suchthat x \leq z \land y \leq z\}
        \end{align*}
        We will show a double inclusion.
        Let \(z \in M_x \intersection M_y\).
        It is clear that
        \[
            z \in \bigcup_{z' \in M_x \intersection M_y} M_{z'}
        \]
        Now let
        \[
            z' \in \bigcup_{z \in M_x \intersection M_y} M_z
        \]
        This is equivalent to saying that there exist \(z\)
        such that \(z' \in M_z\).
        This means that \(z \leq z'\), and \(x \leq z, y\leq z\).
        Finally, \(x \leq z', y \leq z'\) meaning that \(z' \in M_x \intersection y\).
    \end{enumerate}
\end{snippetsolution}

\begin{snippetexercise}{topology-exercise-ex-5}{}
    Show that a \topologicalspace is \(T_1\)
    \ifandonlyif \(\forall x \in X\),
    \[
        \{x\} = \bigcap_{U \in I(x)} U
    \]
    where \(I(x)\) is the \set of \neighborhood[neighborhoods] of \(x\).
\end{snippetexercise}

\begin{snippetsolution}{topology-exercise-ex-5-sol}{}
    \iffproof{
        Let
        \[
            y\in \bigcap U_x
        \]
        and suppose by way of contradiction that \(x\neq y\).
        Since \(X\) is \(T_1\), there exist an \topologicalspace[open set][Open]
        \(U_y\) such that \(x \in U_y\) and \(y \notin U_y\).
        Then, \(U_y\) is a \neighborhood of \(x\) (since it is open and contains \(x\)).
        But then \(y \notin U_y\) which is a contradiction \lightning.
        Thus, \(y=x\) and \[
            \bigcap_{U \in I(x)} U \subseteq \{x\}  
        \]
        But clearly \(x\in U\) for all \(U \in I(x)\), so \(x\in \bigcap_{U\in I(x)}U\) and thus
        \[
            \{x\} \subseteq \bigcap_{U \in I(x)} U
        \]
        and therefore
        \[
            \{x\} = \bigcap_{U \in I(x)} U
        \]
    }{
        Assume that \(\forall x \in X\)
        \[
            \{x\} = \bigcap_{U \in I(x)} U
        \]
        Let \(x,y\in X\) such that \(x\neq y\).
        Then, \[
            y \neq \{x\} = \bigcap_{U \in I(x)} U
        \]
        so there exists some \neighborhood \(U \in I(x)\) such that \(y\notin U\).
        That is, there is a \neighborhood of \(x\) that does not contain \(y\), which is exactly the \(T_1\) condition.
    }
\end{snippetsolution}

\begin{snippetexercise}{topology-exercise-ex-6}{}
    Let \(A\) be a dense subset of a \topologicalspace \(X\).
    Show that for every \topologicalspace[open][Open]
    \(U \subseteq X\), \(\overline{U \intersection A} = \overline{U}\).
\end{snippetexercise}

\begin{snippetsolution}{topology-exercise-ex-6-sol}{}
    \doubleinclusionproof{
        This inclusion is trivial.
    }{
        We can state the condition of density as: \(U \intersection A \neq 0\), \(\forall U \neq \emptyset\) \topologicalspace[open][Open].
        Let \(y \in U\) so that \(y \notin \overline{U \cap A}\). Then, there exists a \neighborhood of \(y\), \(V_y\),
        such that \(V_y \cap (U \cap A) = \emptyset\). But
        \[
            V_y \cap (U \cap A) = (V_y \cap U) \cap A = \emptyset
        \]
        and \(V_y \cap U \neq 0\) which is a contradiction \lightning. 
    }
\end{snippetsolution}

\begin{snippetexercise}{topology-exercise-ex-7}{}
    Let \(X\) be a \set.
    Consider \(C \colon \powerset(X) \fromto \powerset(X)\) such that:
    \begin{enumerate}
        \item \(A \subseteq X \implies A \subseteq C(A)\);
        \item \(C(\emptyset) = \emptyset\);
        \item \(A \subseteq X \implies C(A) = C(C(A))\);
        \item \(C(A \union B) = C(A) \union C(B)\).
    \end{enumerate}
    Prove that for every topology on \(X\),
    the application of \(C\) is an operatore of closeness
    and, viceversa, that for every such operator there exist a unique
    topological structure such that \(C(A) = \overline{A}\).
\end{snippetexercise}

\begin{snippetsolution}{topology-exercise-ex-7-sol}{}
    \begin{enumerate}
        \item Let \(\tau_X\) be a topology for \(X\) and define \(C(A) = \overline{A}\).
            This clearly satisfies properties (1-3).
            We will now prove (4). \\
            \doubleinclusionproof{
                Since \(A \union B \subseteq \overline{A} \union \overline{B}\),
                then \(\overline{A \union B} \subseteq \overline{A} \union \overline{B}\).
            }{
                Since \(A \subseteq \overline{A \union B}\)
                and \(B \subseteq \overline{A \union B}\)
                we have \(A \union B \subseteq A \subseteq \overline{A \union B}\).
            }
        \item Let \(\tau_X\) be defined by the \(\{B \suchthat B = C(B)\}\).
        We will thus prove that a \set \(B\) is \closedset[closed] \ifandonlyif \(B=C(B)\).
        \\
        \doubleinclusionproof{
            By property (1) and (2), \(C(A)\) is one such \set. Thus,
            \[
                \overline{A} =
                \bigcap_{\substack{B \text{ \closedset[closed]} \\ A \subseteq B}} B \subseteq C(A)
            \]
        }{
            Since \(B \subseteq A\), we have \(A \union B = B\),
            and thus \begin{align*}
                C(A \union B) &= C(B) = B \\
                C(A) \union C(B) &= B \\
                C(A) \union B &= B \\
                &\implies C(A) \subseteq B
            \end{align*}
        }
        We still need to show that \(\tau_X\) is a topology:
        \begin{enumerate}
            \item \(\emptyset = C(\emptyset)\);
            \item \(X = C(X)\);
            \item we need to prove the arbitrary
            intersection, not the finite one as we are working with the \closedset[closed sets].
            \begin{align*}
                \bigcap_{i\in I} B_i &= \bigcap_{i\in I} C(B_i) \supseteq 
                C\left(\bigcap_{i\in I} B_i\right) \supseteq \bigcap_{i\in I} B_i
            \end{align*}
            Thus, their intersection is \closedset[closed].
            \item trivil since \(C(A \union B) = C(A) \union C(B)\).
        \end{enumerate}
    \end{enumerate}
\end{snippetsolution}

% extra

% 3.17 manetti
\begin{snippetproposition}{constant-topological-maps-are-continuous}{}
    Let \(f\colon X \fromto Y\) be a map between \topologicalspace[topological spaces].
    Let \(\tau_X, \tau_Y\) be the topologies associated with \(X,Y\) respectively.
    If \(f(x) = y_0\), then \(f\) is continuous.
\end{snippetproposition}

\begin{snippetproof}{constant-topological-maps-are-continuous-proof}{constant-topological-maps-are-continuous}{}
    Let \(V \in \tau_Y\).
    We have
    \[
        f^{-1}(V) = \begin{cases}
            \emptyset & y_0 \notin V \\
            X & y_0 \in V
        \end{cases}
    \]
    which are both \topologicalspace[open][Open].
\end{snippetproof}

% 3.18 manetti
\begin{snippetproposition}{topological-identity-map-continuity-equivalence}{}
    The topological map \(\text{id}\colon (X, \tau_1) \fromto (X, \tau_2)\)
    is \topologycontinuous \ifandonlyif \(\tau_1\) is coarser than \(\tau_2\).
\end{snippetproposition}

\begin{snippetproof}{topological-identity-map-continuity-equivalence-proof}{topological-identity-map-continuity-equivalence}{}
    \iffproof{
        Let \(A \in \tau_2\). We have
        \[
            \text{id}^{-1}(A) = A \in \tau_1
        \]
    }{
        If \(\tau_2 \subseteq \tau_1\), we have
        \[
            \text{id}^{-1}(A) = A \in \tau_1
        \]
        which is \topologicalspace[open][Open].
    }
\end{snippetproof}

\begin{snippetproposition}{adherent-subsets-properties-with-continuous-maps}{}
    Let \((X, \tau)\) be a \topologicalspace. Two subsets \(A,B\) are said to be \emph{adherent}
    if \[
        (A \intersection \overline{B}) \union (\overline{A} \intersection B) \neq \emptyset
    \]
    Then,
    \begin{enumerate}
        \item if \(f \colon X \fromto Y\) is \topologycontinuous then it preserves adherence relations.
        \item if \(f \colon X \fromto Y\) between \(T_1\) spaces preserves adherence relations,
        then it is continuous.
    \end{enumerate}
\end{snippetproposition}

\begin{snippetproof}{adherent-subsets-properties-with-continuous-maps-proof}{adherent-subsets-properties-with-continuous-maps}{}
    We know that \(f\) is \topologycontinuous \ifandonlyif 
    \[
        \forall A \subseteq, f(\overline{A}) \subseteq \overline{f(A)}
    \]
    \begin{enumerate}
        \item Suppose for simplicity that \(\overline{A} \intersection B \neq \emptyset\).
        Let \(x\in \overline{A} \intersection B\).
        This means that \(f(x) \in \overline{f(A)} \intersection f(B)\).
        \item We want to show that \(f(\overline{A}) \subseteq \overline{f(A)}\).
        Consider \(x\) in a \closedset \(\overline{A}\). This means that
        \[
            \{x\} \intersection \overline{A} \neq \emptyset
        \]
        Since \(f\) preserves the adherence relation, \(f(\{x\}) = \{f(x)\}\)
        which is adherent with \(f(A)\).
        We now have \(\overline{\{f(x)\}} \intersection f(A) \neq \emptyset\)
        or  \(\overline{\{f(x)\}} \intersection \overline{f(A)} \neq \emptyset\).
        Thus, \(f(x) \in \overline{f(A)}\).
    \end{enumerate}
\end{snippetproof}

% foglio 2

\end{document}
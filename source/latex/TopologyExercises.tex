\documentclass[preview]{standalone}

\usepackage{amsmath}
\usepackage{amssymb}
\usepackage{stellar}
\usepackage{definitions}

\begin{document}

\id{topology-exercises}
\genpage

\section{Exercises}

\begin{snippetexercise}{topology-exercise-ex-1}{}
    How many topologies are there on \(\emptyset, \{a\}, \{a,b\}\)?
\end{snippetexercise}

\begin{snippetsolution}{topology-exercise-ex-1-sol}{}
    \begin{enumerate}
        \item The only possible topology for \(\empty\) is \(\{\emptyset\}\).
        \item The topology for \(\{a\}\) has at least the element \(\{\emptyset, \{a\}\}\).
        No element can be added or removed so there is only one topology.
        \item We have \(\{\emptyset, \{a,b\}\}, \{\emptyset, \{a,b\}, \{a\}\}, \{\emptyset, \{a,b\}, \{b\}\}, \{\emptyset, \{a,b\}, \{a\}, \{b\}\}\).
    \end{enumerate}
\end{snippetsolution}

\begin{snippetexercise}{topology-exercise-ex-2}{}
    Let \(X\) be an infinite \set and consider the topology
    \[
        \tau_X = \{A \subseteq X \suchthat X \difference A \text{ is finite}\}
        \union \{\emptyset\}
    \]
    Does every pair of non-empty open sets have an empty intersection? 
\end{snippetexercise}

\begin{snippetsolution}{topology-exercise-ex-2-sol}{}
    The open sets of \(\tau_X\) must be infinite. Since
    \[
        X \difference (A \intersection B)
        = (X \difference A) \union (X \difference B)
    \]
    meaning that \(A \intersection B\) is still infinite,
    since the two differences are finite.
\end{snippetsolution}

\begin{snippetexample}{topology-exercise-ex-3}{}
    Let \(X\) be a \set and \(\infty \in X\). Verify whether
    \[
        \tau_X = \{
            A \subseteq X \suchthat \infty \notin A \lor X \difference A \text{ is finite}    
        \}
    \]
    is a possible topology.
\end{snippetexample}

\begin{snippetsolution}{topology-exercise-ex-3-sol}{}
    \begin{enumerate}
        \item \(\emptyset \in \tau_X\) since \(\infty \notin \emptyset\);
        \item \(X \in \tau_X\) since \(X \difference X = \emptyset\) is finite;
        \item \(\forall A, B \in \tau_X\),
        if \(\infty \notin A \intersection B\) then we are done.
        Otherwise, \(\infty \in A \land \infty \in B\).
        In this case note that
        \[
            X \difference (A \intersection B)
            = (X \difference A) \union (X \difference B)
        \]
        which is the union of two finite sets.
        \item Let \(\{A_i\}_{i \in I}\) be a family of elements of \(\tau_X\),
        if \(\infty \notin \bigcup A_i\) then we are done.
        Otherwise, \[
            X \difference \left(\bigcup_{i \in I} A_i \right)
            = \bigcap_{i\in I} \left(X \difference A_i\right)
        \]
        which is finite as at least one of the \(X \difference A_i\)
        is finite because it contains \(\infty\).
    \end{enumerate}
\end{snippetsolution}

\begin{snippetexercise}{topology-exercise-ex-4}{}
    Consider a \partialorder[partially ordered] \set \((X, \leq)\).
    Let
    \[
        M_x = \{
            y \in X \suchthat x \leq y    
        \}
    \]
    Show that \(\{M_x \suchthat x \in X\}\) forms a \topologicalbasis for \(X\).
\end{snippetexercise}

\begin{snippetsolution}{topology-exercise-ex-4-sol}{}
    We note the obvious double inclusion
    \[
        \tau_X = \left\{
            A \suchthat
            A = \bigcup_{i \in I} A_i = \bigcup_{M_x \subseteq A} M_x
        \right\}
    \]
    We have:
    \begin{enumerate}
        \item \[
            X = \bigcup_{x\in X} M_x
        \]
        \item we now want to show that \(M_x \intersection M_x = \bigcup M_x\).
        We have
        \begin{align*}
            M_x \intersection M_y
            &= \{z \in X \suchthat x \leq z \land y \leq z\}
        \end{align*}
        We will show a double inclusion.
        Let \(z \in M_x \intersection M_y\).
        It is clear that
        \[
            z \in \bigcup_{z' \in M_x \intersection M_y} M_{z'}
        \]
        Now let
        \[
            z' \in \bigcup_{z \in M_x \intersection M_y} M_z
        \]
        This is equivalent to saying that there exist \(z\)
        such that \(z' \in M_z\).
        This means that \(z \leq z'\), and \(x \leq z, y\leq z\).
        Finally, \(x \leq z', y \leq z'\) meaning that \(z' \in M_x \intersection y\).
    \end{enumerate}
\end{snippetsolution}

\begin{snippetexercise}{topology-exercise-ex-5}{}
    \todo
\end{snippetexercise}

\begin{snippetsolution}{topology-exercise-ex-5-sol}{}
    \todo
\end{snippetsolution}

\end{document}
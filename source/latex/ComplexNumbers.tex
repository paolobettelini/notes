\documentclass[preview]{standalone}

\usepackage{amsmath}
\usepackage{amssymb}
\usepackage{parskip}
\usepackage{fullpage}
\usepackage{hyperref}
\usepackage{tikz}
\usepackage{stellar}
\usepackage{definitions}
\usepackage{bettelini}

\begin{document}

\id{complexnumbers}
\genpage

\section{Complex Numbers}

\subsection{Definition}

\begin{snippetdefinition}{complex-numbers-definition}{Complex numbers}
    The set of \textit{complex numbers} is defined as
    \[
    \i
        \mathbb{C} = \{ a+bi \suchthat a,b\in\realnumbers \}
    \]
    where \(i\) is defined as \(i^2=-1\), called the \textit{imaginary unit}.
\end{snippetdefinition}

\begin{snippet}{imaginary-unit-negative-version}
    The equation \(x^2 = -1\) has two solutions in \(\complexnumbers\): \(i\) and \(-i\), however,
    there is not any algebraic difference between these two solutions.
    You can choose \(i\) to arbitrarily be one of this solution as long as it is consistent.
\end{snippet}

\subsection{Properties of imaginary unit}

\begin{snippettheorem}{imaginary-unit-exponents}{Imaginary exponents}
    The imaginary unit forms a \cyclicgroup of order \(4\) under multiplication:
    \[
        \gengrp{i} = \{ 1, i, -1, -i \}
    \]
    given by
    \[
        i^n = i^{n \mod 4}
    \]
\end{snippettheorem}

\subsection{Complex plane}

\begin{snippet}{complex-plane}
    We can represent each complex number on a plane (Argand plane), where the horizontal axis
    represent the real numbers \(\realnumbers\) and the vertical axis represents
    every scalar multiple of the imaginary unit \(i\).
\end{snippet}

%\begin{snippet}{complexplane-illustration}
%\begin{center}
%    \begin{tikzpicture}
%        \begin{scope}[thick,font=\scriptsize]
%
%            \draw [->] (-5,0) -- (5,0) node [above left]  {\(\Re(s)\)};
%            \draw [->] (0,-5) -- (0,5) node [below right] {\(\Im(s)\)};
%
%            \draw (0,0) -- (0,0)   node [above right] {\(0\)};
%            \foreach \n in {-4,...,-1,1,2,...,4}{
%                \draw (\n,-3pt) -- (\n,3pt) node [above] {\(\n\)};
%                \draw (-3pt,\n) -- (3pt,\n) node [right] {\(\n i\)};
%            }
%
%            \draw [color=black, fill=black] (3,2) circle (0.05) node [above] {\(3+2i\)};
%        \end{scope}
%    \end{tikzpicture}
%\end{center}
%\end{snippet}

\includesnpt{complexplane}

\subsection{Operations}

\begin{snippet}{complex-numbers-addition}
Let \(a, b \in \realnumbers\)
\[
    (a+bi)+(c+di)=a+bi+c+di=(a+c)+(b+d)i
\]
\end{snippet}

\begin{snippet}{complex-numbers-subtraction}
Let \(a, b \in \realnumbers\)
\[
    (a+bi)-(c+di)=a+bi-c-di=(a-c)+(b-d)i
\]
\end{snippet}

\begin{snippet}{complex-numbers-multiplication}
Let \(a, b \in \realnumbers\)
\[
    (a+bi)(c+di)=ac+adi+bci+bdi^2=(ac-db)+(ad+bc)i
\]
\end{snippet}

\begin{snippet}{complex-numbers-division}
Let \(a, b \in \realnumbers\)
\begin{align*}
    \frac{a+bi}{c+di} &= \frac{a+bi}{c+di} \cdot \frac{c-di}{c-di} = \frac{ac-adi+bci-bdi^2}{c^2-d^2i^2}
    \\ &= \frac{ac+bd+(bc-ad)i}{c^2+d^2} \\ &= \frac{ac+bd}{c^2+d^2} + \frac{bc-ad}{c^2 + d^2}i
\end{align*}
\end{snippet}

\section{Axiomatic definition}

\begin{snippetaxiom}{complex-number-axiomatic-def}{Complex Number}
    A complex number is a tuple \((a,b)\) where \(a,b\in \realnumbers\).
    
    \begin{enumerate}
        \item \textbf{Equality:} \( (a,b) = (c,d) \iff a=c \land b=d \)
        \item \textbf{Addition:} \((a,b) + (c,d) = (a+c, b+d)\)
        \item \textbf{Multiplication:} \((a,b) \cdot (c,d) = (ac-db, ad+bc)\)
        \item \textbf{Scalar Multiplication:} \(m(a,b) = (ma, mb), \quad m \in \realnumbers\)
    \end{enumerate}
    
    Let \(z_1, z_2, z_3 \in \complexnumbers\)
    \begin{enumerate}
        \item \(z_1+z_2\) and \(z_1z_2\) are also in \(\complexnumbers\)
        \item \(z_1+z_2=z_2+z_1\)
        \item \(z_1 + (z_2 + z_3) = (z_1 + z_2) + z_3\)
        \item \(z_1z_2=z_2z_1\)
        \item \(z_1(z_2z-3)=(z_1z_2)z_3\)
        \item \(z_1(z_2+z_3)=z_1z_2+z_1z_3\)
        \item \(z_1+0=z_1\)
        \item \(z_1\cdot 1=z_1\)
        \item \(\exists_{=1} z \suchthat z+z_1=0\)
        \item \(\exists_{=1} z \suchthat z\cdot z_1=1\)
    \end{enumerate}
\end{snippetaxiom}

\subsection{Vector form}

\begin{snippet}{complex-number-vector-form}
    Any complex number \(a+bi\) can be represented by a vector \((a,b)\).
\end{snippet}

\begin{snippetdefinition}{complex-scalar-product}{Complex scalar product}
    The \textit{scalar product} between \(z_1=a+bi\) and \(z_2=c+di\) is given by
    
    \[
        z_1\circ z_2 = |z_1|\,|z_2|\cos\theta
        = ac+bd = \Re(z_1^*z_2) = \frac{1}{2}(z_1^*z_2+z_1z_2^*)
    \]
    where \(\theta\) is the angle formed by the two vectors.
\end{snippetdefinition}

\begin{snippetdefinition}{complex-vector-product}{Complex vector product}
    The vector product between \(z_1=a+bi\) and \(z_2=c+di\) is given by

    \[
        z_1\times z_2 = |z_1|\,|z_2|\sin\theta
        = ad-cb = \Im(z_1^*z_2) = \frac{1}{2i}(z_1^*z_2+z_1z_2^*)
    \]
    where \(\theta\) is the angle formed by the two vectors.

    We can see that
    \[
        z_1^*z_2 = (z_1\circ z_2) + i(z_1 \times z_2)
    \]
\end{snippetdefinition}

\subsection{Complex conjugate coordinates}

\begin{snippetproposition}{complex-conj-coordinates}{Coordinates in terms of conjugate}
    For any complex number \(z\)
    \begin{align*}
        \Re z&=\frac{z+\overline{z}}{2}
        \\
        \Im z &=\frac{z-\overline{z}}{2i}
    \end{align*}
    %\(z\) can also be represented by the conjugate coordinates \((z, z^*)\).
\end{snippetproposition}

\end{document}
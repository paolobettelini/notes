\documentclass[preview]{standalone}

\usepackage{amsmath}
\usepackage{amssymb}
\usepackage{stellar}
\usepackage{definitions}

\begin{document}

\id{linearalgebra-exercises-batch-4}
\genpage

\section{Exercises}

\begin{snippetexercise}{linear-algebra-batch-4-ex-1}{}
    Let \(v_1 = (i, 2-3i, 5)\) and \(v_2=(7+i, 1+i, -1)\) in \(\complexnumbers^3\)
    and let \(V = \linearspan(v_1, v_2)\).
    \begin{enumerate}
        \item Show that \((-17-2i, 7+4i, 2+15i) \in V\);
        \item Show that \((1,4+3i,6i) \notin V\).
    \end{enumerate}
\end{snippetexercise}

\begin{snippetsolution}{linear-algebra-batch-4-ex-1-sol}{}
    \begin{enumerate}
        \item We want to find \(\alpha, \beta \in \complexnumbers\) such that
        \[  \alpha v_1 + \beta v_2 = (-17-2i, 7+4i, 2+15i) \]
        We thus have the linear system 
        \[
            \begin{cases}
                \alpha i + \beta(7+i) = -17-2i \\
                \alpha(2-3i) + \beta(1 + i) = 7 + 4i \\
                5\alpha - \beta = 2 + 15i
            \end{cases}
        \]
        Substitute the third equation into the second
        \begin{align*}
            \alpha(2-3i) + 5\alpha + 5\alpha i - 2 -2i - 15i + 15 &= 7 + 4i \\
            \alpha &= \frac{-6 + 21i}{7 + 2i} \cdot \frac{7-2i}{7-2i} \\
            &= \frac{159i}{53} = 3i
        \end{align*}
        We then have
        \[
            \beta = 5\alpha - (2 + 15i) = 5(3i) - (2 + 15i) = 15i - 2 - 15i = -2
        \]
        The third equation is satisfied
        \[
            5(3i) - (-2) = 2 + 15i
        \]
        \item We have the system
        \[
            \begin{cases}
                \alpha i + \beta(7+i) = 1 \\
                \alpha(2-3i) + \beta(1 + i) = 4+3i \\
                5\alpha - \beta = 6i
            \end{cases}
        \]
        Substitute the third equation into the second
        \begin{align*}
            \alpha(2-3i) &= 2\alpha - 3i\alpha \\
            7\alpha + 2 + 2i\alpha -9i &= 0
        \end{align*}
        By matching the real and imaginary parts, we get
        \(\alpha = -2/7\) and \(\beta = 9/2\).
    \end{enumerate}
\end{snippetsolution}

\begin{snippetexercise}{linear-algebra-batch-4-ex-2}{}
    Let \(v_1 = (1,2,3), v_2 = (3,2,1) \in \realnumbers^3\)
    and let \(V_1 = \linearspan(v_1)\) and \(V_2 = \linearspan(v_2)\).
    Let
    \[
        R = \{
            (x_1, x_2, x_3) \in \realnumbers^3 \suchthat
            x_1 - 2x_2 + x_3 = 0
        \}
    \]
    Show that \(R = V_1 + V_2\).
\end{snippetexercise}

\begin{snippetsolution}{linear-algebra-batch-4-ex-2-sol}{}
    Let \(v \in V_1 + V_2 = (\alpha + 3\beta, 2\alpha + 2\beta, 3\alpha + \beta)\).
    We have that
    \[
        (\alpha + 3\beta) - 2(2\alpha + 2\beta) + (3\alpha + \beta) = 0
    \]
    and thus \(V_1 + V_2 \subseteq R\).
    We then let \(r = (2x_2 - x_3, x_2, x_3) \in R\). We want
    \[
        \alpha v_1 + \beta v_2 = r
    \]
    This gives the system
    \begin{align*}
        \alpha + 3\beta &= 2x_2 - x_3 \\
        2\alpha + 2\beta &= x_2 \\
        3\alpha + \beta &= x_3
    \end{align*}
    giving the solution
    \[
        \alpha = \frac{x_2}{4} + \frac{x_3}{2}, \quad
        \beta = \frac{3}{4}x_2 - \frac{1}{2}x_3
    \]
\end{snippetsolution}

\begin{snippetexercise}{linear-algebra-batch-4-ex-3}{}
    Consider the linear subspaces of \(\realnumbers^4\)
    \[
        U = \linearspan((1,0,1,0), (1,2,3,4)), \quad
        V = \linearspan((0,1,1,1), (0,0,0,1))
    \]
    \begin{enumerate}
        \item Study whether \(U+V = \realnumbers^4\);
        \item Study whether \(U + V\) is a direct sum.
    \end{enumerate}
\end{snippetexercise}

\begin{snippetsolution}{linear-algebra-batch-4-ex-3-sol}{}
    We have
    \[
        U+V = \linearspan((1,0,1,0), (1,2,3,4), (0,1,1,1), (0,0,0,1))
    \]
    and in order for it to be equal to \(\realnumbers^4\), we need
    \[
        \alpha (1,0,1,0) + \beta (1,2,3,4) + \gamma (0,1,1,1) + \delta (0,0,0,1) = (x_1,x_2,x_3,x_4)
    \]
    \begin{enumerate}
        \item This gives the system
        \begin{align*}
            \alpha + \beta &= x_1 \\
            2\beta + \gamma &= x_2 \\
            \alpha + 3\beta + \gamma &= x_3 \\
            4\beta + \delta &= x_4
        \end{align*}
        This gives the matrix
        \[
            \begin{pmatrix}
                1 & 1 & 0 & 0 \\
                0 & 2 & 1 & 0 \\
                1 & 3 & 1 & 0 \\
                0 & 4 & 0 & 1
            \end{pmatrix}
            \begin{pmatrix}
                \alpha \\
                \beta \\
                \gamma \\
                \delta
            \end{pmatrix}
            =
            \begin{pmatrix}
                x_1 \\
                x_2 \\
                x_3 \\
                x_4
            \end{pmatrix}
        \]
        which is reduced to
        \[
            \begin{pmatrix}
                1 & 1 & 0 & 0 \\
                0 & 2 & 1 & 0 \\
                0 & 0 & -1 & 1 \\
                0 & 0 & 0 & 0
            \end{pmatrix}
            \begin{pmatrix}
                \alpha \\
                \beta \\
                \gamma \\
                \delta
            \end{pmatrix}
            =
            \begin{pmatrix}
                x_1 \\
                x_2 \\
                x_4 - 2x_2 \\
                x_3 -x_1-x_2
            \end{pmatrix}
        \]
        which has no solutions if \(x_3 - x_1 - x_2 \neq 0\).
        \item We want to check if the system
        \[
            \begin{pmatrix}
                1 & 1 & 0 & 0 \\
                0 & 2 & 1 & 0 \\
                1 & 3 & 1 & 0 \\
                0 & 4 & 0 & 1
            \end{pmatrix}
            \begin{pmatrix}
                \alpha \\
                \beta \\
                \gamma \\
                \delta
            \end{pmatrix}
            =
            \begin{pmatrix}
                0 \\
                0 \\
                0 \\
                0
            \end{pmatrix}
        \]
    \end{enumerate}
    has only one solution. The system is reduced to
    \[
        \begin{pmatrix}
            1 & 1 & 0 & 0 \\
            0 & 2 & 1 & 0 \\
            0 & 0 & -1 & 1 \\
            0 & 0 & 0 & 0
        \end{pmatrix}
        \begin{pmatrix}
            \alpha \\
            \beta \\
            \gamma \\
            \delta
        \end{pmatrix}
        =
        \begin{pmatrix}
            0 \\
            0 \\
            0 \\
            0
        \end{pmatrix}
    \]
    which has infinite solutions. Thus, the sum is not direct.
\end{snippetsolution}

\begin{snippetexercise}{linear-algebra-batch-4-ex-4}{}
    Prove or disprove the following statement:
    \begin{enumerate}
        \item Let \(U_1, U_2, W\) be linear subspaces of \(V\) such that \(U_1 \oplus W = U_2 \oplus W\). Then, \(U_1 = U_2\).
        \item There exist linear subspaces \(U, V, W\) of \(\realnumbers^2\) such that \(\realnumbers^2 = U \oplus V = U \oplus W = V \oplus W\).
    \end{enumerate}
\end{snippetexercise}

\begin{snippetsolution}{linear-algebra-batch-4-ex-4-sol}{}
    \begin{enumerate}
        \item Consider \(V = \realnumbers^2\) and let \(W = \linearspan((1,0))\), \(U_2 = \linearspan((-1,1))\)
        and \(U_1 = \linearspan((1,1))\). We have \(\realnumbers^2 = W \oplus U_1\) and \(\realnumbers^2 = W \oplus U_2\)
        but \(U_1 = U_2\).
        \item Let \(W = \linearspan((1,0))\), \(V = \linearspan((1,1))\) and \(W = \linearspan((-1,1))\). Then, \(\realnumbers^2 = U \oplus V = U \oplus W = V \oplus W\).
    \end{enumerate}
    Every sum is direct as the spans only intersect at the origin.
\end{snippetsolution}

\begin{snippetexercise}{linear-algebra-batch-4-ex-5}{}
    Find \(t\in\realnumbers\) such that \(\{(3,1,4), (1,5,9), (2,6,t)\}\) is not \linearlyindependent. 
\end{snippetexercise}

\begin{snippetsolution}{linear-algebra-batch-4-ex-5-sol}{}
    We want the system
    \[
        \alpha \begin{pmatrix}
            3 \\ 1 \\ 4
        \end{pmatrix}  
        + \beta \begin{pmatrix}
            1 \\ 5 \\ 9
        \end{pmatrix}
        + \gamma \begin{pmatrix}
            2 \\ 6 \\ t
        \end{pmatrix}
        = \begin{pmatrix}
            0 \\ 0 \\ 0
        \end{pmatrix}  
    \]
    to have multiple solutions. This gives the matrix
    \[
        \begin{pmatrix}
            3 & 1 & 2 \\
            1 & 5 & 6 \\
            4 & 9 & t
        \end{pmatrix}
        \begin{pmatrix}
            \alpha \\ \beta \\ \gamma
        \end{pmatrix}
        =
        \begin{pmatrix}
            0 \\ 0 \\ 0
        \end{pmatrix}
    \]
    This is reduced to
    \[
        \begin{pmatrix}
            1 & 5 & 6 \\
            0 & -14 & -16 \\
            0 & 0 & t - \frac{165}{14}
        \end{pmatrix}
        \begin{pmatrix}
            \alpha \\ \beta \\ \gamma
        \end{pmatrix}
        =
        \begin{pmatrix}
            0 \\ 0 \\ 0
        \end{pmatrix}
    \]
    and thus the solution has infinite solutions if \(t = \frac{165}{14}\).
\end{snippetsolution}

\begin{snippetexercise}{linear-algebra-batch-4-ex-6}{}
    Consider the linear subspaces of \(\realnumbers^5\)
    \[
        \{(0,1,0,3,1),(1,2,1,0,1),(1,2,1,1,0),(1,3,1,2,3)\}
    \]
    Determine whether this \set is \linearlyindependent.
\end{snippetexercise}

\begin{snippetsolution}{linear-algebra-batch-4-ex-6-sol}{}
    We want the system
    \[
        \alpha \begin{pmatrix}
            0 \\ 1 \\ 0 \\ 3 \\ 1
        \end{pmatrix}  
        + \beta \begin{pmatrix}
            1 \\ 2 \\ 1 \\ 0 \\ 1
        \end{pmatrix}
        + \gamma \begin{pmatrix}
            1 \\ 2 \\ 1 \\ 1 \\ 0
        \end{pmatrix}
        + \delta \begin{pmatrix}
            1 \\ 3 \\ 1 \\ 2 \\ 3
        \end{pmatrix}
        = \begin{pmatrix}
            0 \\ 0 \\ 0 \\ 0 \\ 0
        \end{pmatrix}
    \]
    to have one solution. This gives the matrix
    \[
        \begin{pmatrix}
            0 & 1 & 1 & 1 \\
            1 & 2 & 2 & 3 \\
            0 & 1 & 1 & 1 \\
            3 & 0 & 1 & 2 \\
            1 & 1 & 0 & 3
        \end{pmatrix}
        \begin{pmatrix}
            \alpha \\ \beta \\ \gamma \\ \delta
        \end{pmatrix}
        =
        \begin{pmatrix}
            0 \\ 0 \\ 0 \\ 0 \\ 0
        \end{pmatrix}
    \]
    This is reduced to
    \[
        \begin{pmatrix}
            1 & 1 & 0 & 3 \\
            0 & 1 & 1 & 1 \\
            0 & 0 & 1 & -1 \\
            0 & 0 & 0 & 2
        \end{pmatrix}
        \begin{pmatrix}
            \alpha \\ \beta \\ \gamma \\ \delta
        \end{pmatrix}
        =
        \begin{pmatrix}
            0 \\ 0 \\ 0 \\ 0 \\ 0
        \end{pmatrix}
    \]
    which only admit one solution. Thus, the set is \linearlyindependent.
\end{snippetsolution}

\begin{snippetexercise}{linear-algebra-batch-4-ex-7}{}
    \todo
\end{snippetexercise}

\begin{snippetsolution}{linear-algebra-batch-4-ex-7-sol}{}
    \todo
\end{snippetsolution}

\begin{snippetexercise}{linear-algebra-batch-4-ex-8}{}
    \todo
\end{snippetexercise}

\begin{snippetsolution}{linear-algebra-batch-4-ex-8-sol}{}
    \todo
\end{snippetsolution}

\end{document}
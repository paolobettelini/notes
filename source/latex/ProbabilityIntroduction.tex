\documentclass[preview]{standalone}

\usepackage{amsmath}
\usepackage{amssymb}
\usepackage{stellar}
\usepackage{definitions}

\begin{document}

\id{probability-introduction}
\genpage

\section{Probability}

\subsection{Intuitive construction}

\begin{snippetexample}{classical-probability-example}{Classical approach to probability}
    Consider an urn containing 6 indistinguishable balls numbered from 1 to 6.
    We want to perform the experiment of extracting a ball from the urn
    and studying which ball is more likely to be extracted.
    We thus want to place an ordering on the set of outcomes of this experiment.
    Let \(\Omega\) be the set of all possible outcomes of the experiment.
    Let \(P_i\) be the probability of extracting the number \(i \in \{1,2,3,4,5,6\}\).
    In this case, the natural choice is \(P_i = \frac{1}{6}\) for all \(i \in \{1,2,3,4,5,6\}\).
    Note that \(\sum P_i = 1\), which is the probability of extracting a number from 1 to 6, in this case the certain event.
    This is called the additivity of probabilities for disjoint events.
    Furthermore, the probability \(P_j = 0\) with \(j \notin \{1,2,3,4,5,6\}\).
    We can also note that
    \[
        P_i = \frac{\text{Favorable cases}}{\text{Possible cases}}
        = \frac{
            \cardinality{\{i\}}
        }{\cardinality{\Omega}}
    \]
    In this case, the elementary cases \(\{i\}\) are equiprobable.
\end{snippetexample}

\begin{snippetexample}{frequentist-probability-example}{Frequentist approach to probability}
    Consider the experiment of rolling a 6-sided die
    with faces numbered from 1 to 6.
    We have \(\Omega = \{1,2,3,4,5,6\}\).
    In this case, the events or elementary cases are not necessarily equiprobable.
    Therefore, to assign the probabilities \(P_i\) we could roll the die experimentally.
    \[
        P_i^{(k)} = \frac{N_i^{(k)}}{k}
    \]
    where \(N_i^{(k)}\) is the number of times the event \(i\) occurs out of \(k\) rolls.
    Clearly these values are variable, so we take
    \[
        P_i = \lim_{k \to \infty} P^{(k)}_i
    \]
    for \(i \in \{1,2,3,4,5,6\}\) if the limit exists.
    Note that \(P_i^{(k)} \in [0,1]\) and
    \[
        \sum_{i=1}^6 P_i^{(k)} = 1, \quad k \in \naturalnumbers
    \]
    and by taking the limit \(k \to \infty\):
    \[
        \sum_{i=1}^6 P_i = 1
    \]
\end{snippetexample}

\plain{In both cases we have the same properties,
but in the latter the probabilities do not necessarily coincide.}

\begin{snippetexample}{subjective-probability-example}{Subjective probability}
    Consider a round-robin football tournament involving 6 teams:
    (1) R. Madrid, (2) M. City, (3) Bayern Munich, (4) Atalanta, (5) Porto, (6) Nantes.
    The experiment we are considering studies the winner of the tournament.
    We have \(\Omega = \{1,2,3,4,5,6\}\).
    The classical approach requires equiprobable elementary cases, which is not the case here.
    The frequentist approach requires repeating a tournament many times under the same conditions,
    which is practically impossible.
    The alternative idea is therefore to ask two experts in the field
    to assign probabilities in a consistent manner with respect to the observations we have
    made in the other two approaches.
    The first expert might choose, for example, 
    \[
        P_1 = \frac{1}{4}, \quad
        P_2 = \frac{1}{4}, \quad
        P_3 = \frac{1}{5}, \quad
        P_4 = \frac{1}{5}, \quad
        P_5 = \frac{1}{10}, \quad
        P_6 = 0
    \]
    while the second expert might choose
    \[
        P_1 = \frac{11}{27}, \quad
        P_2 = \frac{1}{3}, \quad
        P_3 = \frac{1}{9}, \quad
        P_4 = \frac{2}{27}, \quad
        P_5 = \frac{1}{27}, \quad
        P_6 = \frac{1}{27}
    \]
    Clearly, there is a subjective nature here.
\end{snippetexample}

\begin{snippetdefinition}{subjective-probability-definition}{Subjective probability}
    The \emph{probability} of an event is defined as
    the measure of the degree of belief, i.e. a real number in \([0,1]\),
    that a coherent individual assigns to the occurrence of the considered event
    based on their knowledge. \\
    In other words, the probability of an event is how much a coherent individual considers it fair to pay
    to receive \(1\) if the event occurs, and \(0\) if it does not occur.
\end{snippetdefinition}

\plain{Each of these approaches covers the previous one.
The subjective one is the most general.}

\subsection{Formalization}

\begin{snippetdefinition}{outcome-definition}{Outcome}
    An \emph{outcome} is any statement
    whose truthfulness can be established
    by observing the outcome of the experiment.
\end{snippetdefinition}

\plain{If we have an urn, we can represent
the events as points on a line segment, i.e. all possible extractions.
With two urns, we can do the same with a discrete grid, and so on.}

\begin{snippetdefinition}{event-definition}{Event}
    An \emph{event} is a \set of outcomes.
\end{snippetdefinition}



\end{document}
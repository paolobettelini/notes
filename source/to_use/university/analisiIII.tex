\documentclass[a4paper]{article}

\usepackage{amsmath}
\usepackage{amssymb}
\usepackage{stellar}
\usepackage{parskip}
\usepackage{fullpage}
\usepackage{wrapfig}

\title{Analisi III}
\author{Paolo Bettelini}
\date{}

\begin{document}

\maketitle
\tableofcontents

\section{Successione di funzioni}

\sdefinition{Successione di funzioni}{
    Una \emph{successione di funzioni} è una famiglia di funzioni \(\{f_n\}_{n\in\mathbb{N}}\)
    definite su un dominio comune \(f_n \colon D \to \mathbb{R}\).
}

\sdefinition{Convergenza in un punto}{
    Sia \(\{f_n\}_{n\in\mathbb{N}}\) una successione di funzioni.
    La successione converge in un punto \(x_0\) se
    \[
        \lim_{n\to\infty} f_n(x_0) < \infty
    \]
}

\sdefinition{Convergenza puntuale}{
    Sia \(\{f_n\}_{n\in\mathbb{N}}\) una successione di funzioni.
    La successione \emph{converge puntualmente} ad una funzione \(f\colon D \to \mathbb{R}\) se
    \[
        \forall x\in D, \lim_{n\to\infty} f_n(x) = f(x)
    \]
}

Quindi la successione converge in ogni punto, ma la velocità di convergenza può dipenderere dal punto.

\sdefinition{Convergenza uniforme}{
    Sia \(\{f_n\}_{n\in\mathbb{N}}\) una successione di funzioni.
    La successione \emph{converge uniformemente} ad una funzione \(f\colon D \to \mathbb{R}\) se
    \[
        \underset{x\in D}{\sup} \left| f_n(x) - f(x) \right| \to 0
    \]
    per \(n\to\infty\).
}

Quindi la velocità di convergenza è la stessa in ogni punto.

\sdefinition{Convergenza uniformemente di Cauchy}{
    Sia \(\{f_n\}_{n\in\mathbb{N}}\) una successione di funzioni.
    La successione è \emph{uniformemente di Cauchy} se
    \[
        \forall \varepsilon > 0,
        \exists N \in \mathbb{N} \,|\,
        \forall n,m > N,
        \underset{x\in D}{\sup} \left| f_n(x) - f_m(x) \right| < \varepsilon
    \]
}

A partire da un certo indice, tutte le funzioni della successione sono molto vicine tra loro in modo uniforme su tutto \(D\),
indipendentemente dalla funzione limite.

\stheorem{Convergenza uniforme e convergenza uniformemente di Cauchy}{
    Sia \(\{f_n\}_{n\in\mathbb{N}}\) una successione di funzioni.
    Se la successione è uniformemente di Cauchy allora è uniformemente convergente.
}

\stheorem{}{
    Sia \(f_n\colon [a,b] \to \mathbb{R}\) una successione di funzioni R-integrabili
    dove \(f_n \to f\) in \([a,b]\). Allora \(f\) è R-integrabile e
    \[
        \lim_{n\to\infty} \integral[a][b][f_n(x)][x]
        = \integral[a][b][\lim_{n\to\infty} f_n(x)][x]
        = \integral[a][b][f(x)][x]
    \]
}

\stheorem{}{
    Sia \(f_n\colon [a,b] \to \mathbb{R}\) una successione di funzioni derivabili.
    Supponiamo che:
    \begin{enumerate}
        \item \(\exists x_0 \in [a,b]\) tale che \(f_n\) converge in \(x_0\);
        \item \(f_n'\) converge uniformemente in \(g\) a \([a,b]\).
    \end{enumerate}
    Allora,
    \begin{enumerate}
        \item \(f_n\) converge uniformemente a \(f\) in \([a,b]\);
        \item \(f\) è derivabile;
        \item \(f'(x) = g(x)\) per ogni \(x \in [a,b]\).
    \end{enumerate}
}

\section{Serie di funzioni}

\sdefinition{Convergenza uniforme}{
    La serie di funzioni \(\sum_{n=1}^\infty f_n(x)\) \emph{converge uniformemente} ad una funzione \(S(x)\)
    se la successione delle somme parziali
    \[
        S_N(x) = \sum_{n=1}^N f_n(x)
    \]
    converge uniformemente a \(S(x)\), ovvero se
    \[
        \underset{x\in D}{S_N(x) - S(x)} \to 0
    \]
    per \(N\to\infty\).
}

È più forte della convergenza puntuale.

\sdefinition{Convergenza totale}{
    Una serie di funzioni \(\sum_{n=1}^\infty f_n(x)\) \emph{converge totalmente} su un insieme \(D\) se la serie di norme
    \[
        \sum ||f_n||_\infty
    \]
    converge.
}

Ricordiamo che in generale la norma

\[
    ||f||_p = {\left(
        \integral[a][b][|f(x)|^p][x]
    \right)}^{\frac{1}{p}}, \quad 1 \leq p < \infty
\]
e per \(p=\infty\)
\[
    ||f||_\infty = \underset{x\in D}{\sup} |f(x)|
\]

\end{document}
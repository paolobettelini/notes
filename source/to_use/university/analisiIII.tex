\documentclass[a4paper]{article}

\usepackage{amsmath}
\usepackage{amssymb}
\usepackage{stellar}
\usepackage{parskip}
\usepackage{fullpage}
\usepackage{wrapfig}

\title{Analisi III}
\author{Paolo Bettelini}
\date{}

\begin{document}

\maketitle
\tableofcontents

\section{Successione di funzioni}

\sdefinition{Successione di funzioni}{
    Una \emph{successione di funzioni} è una famiglia di funzioni \(\{f_n\}_{n\in\mathbb{N}}\)
    definite su un dominio comune \(f_n \colon D \to \mathbb{R}\).
}

\sdefinition{Convergenza in un punto}{
    Sia \(\{f_n\}_{n\in\mathbb{N}}\) una successione di funzioni.
    La successione converge in un punto \(x_0\) se
    \[
        \lim_{n\to\infty} f_n(x_0) < \infty
    \]
}

\sdefinition{Convergenza puntuale}{
    Sia \(\{f_n\}_{n\in\mathbb{N}}\) una successione di funzioni.
    La successione \emph{converge puntualmente} ad una funzione \(f\colon D \to \mathbb{R}\) se
    \[
        \forall x\in D, \lim_{n\to\infty} f_n(x) = f(x)
    \]
}

Quindi la successione converge in ogni punto, ma la velocità di convergenza può dipenderere dal punto.

\sdefinition{Convergenza uniforme}{
    Sia \(\{f_n\}_{n\in\mathbb{N}}\) una successione di funzioni.
    La successione \emph{converge uniformemente} ad una funzione \(f\colon D \to \mathbb{R}\) se
    \[
        \sup_{x\in D} \left| f_n(x) - f(x) \right| \to 0
    \]
    per \(n\to\infty\).
}

Quindi la velocità di convergenza è la stessa in ogni punto.

\sdefinition{Convergenza uniformemente di Cauchy}{
    Sia \(\{f_n\}_{n\in\mathbb{N}}\) una successione di funzioni.
    La successione è \emph{uniformemente di Cauchy} se
    \[
        \forall \varepsilon > 0,
        \exists N \in \mathbb{N} \,|\,
        \forall n,m > N,
        \sup_{x\in D} \left| f_n(x) - f_m(x) \right| < \varepsilon
    \]
}

A partire da un certo indice, tutte le funzioni della successione sono molto vicine tra loro in modo uniforme su tutto \(D\),
indipendentemente dalla funzione limite.

\stheorem{Convergenza uniforme e convergenza uniformemente di Cauchy}{
    Sia \(\{f_n\}_{n\in\mathbb{N}}\) una successione di funzioni.
    Se la successione è uniformemente di Cauchy allora è uniformemente convergente.
}

\stheorem{}{
    Sia \(f_n\colon [a,b] \to \mathbb{R}\) una successione di funzioni R-integrabili
    dove \(f_n \to f\) in \([a,b]\). Allora \(f\) è R-integrabile e
    \[
        \lim_{n\to\infty} \integral[a][b][f_n(x)][x]
        = \integral[a][b][\lim_{n\to\infty} f_n(x)][x]
        = \integral[a][b][f(x)][x]
    \]
}

\stheorem{}{
    Sia \(f_n\colon [a,b] \to \mathbb{R}\) una successione di funzioni derivabili.
    Supponiamo che:
    \begin{enumerate}
        \item \(\exists x_0 \in [a,b]\) tale che \(f_n\) converge in \(x_0\);
        \item \(f_n'\) converge uniformemente in \(g\) a \([a,b]\).
    \end{enumerate}
    Allora,
    \begin{enumerate}
        \item \(f_n\) converge uniformemente a \(f\) in \([a,b]\);
        \item \(f\) è derivabile;
        \item \(f'(x) = g(x)\) per ogni \(x \in [a,b]\).
    \end{enumerate}
}

\section{Serie di funzioni}

\sdefinition{Convergenza uniforme}{
    La serie di funzioni \(\sum_{n=1}^\infty f_n(x)\) \emph{converge uniformemente} ad una funzione \(S(x)\)
    se la successione delle somme parziali
    \[
        S_N(x) = \sum_{n=1}^N f_n(x)
    \]
    converge uniformemente a \(S(x)\), ovvero se
    \[
        \sup_{x\in D}|S_N(x) - S(x)| \to 0
    \]
    per \(N\to\infty\).
}

È più forte della convergenza puntuale.

\sdefinition{Convergenza totale}{
    Una serie di funzioni \(\sum_{n=1}^\infty f_n(x)\) \emph{converge totalmente} su un insieme \(D\) se la serie di norme
    \[
        \sum ||f_n||_\infty
    \]
    converge.
}

Ricordiamo che in generale la norma

\[
    ||f||_p = {\left(
        \integral[a][b][|f(x)|^p][x]
    \right)}^{\frac{1}{p}}, \quad 1 \leq p < \infty
\]
e per \(p=\infty\)
\[
    ||f||_\infty = \sup_{x\in D} |f(x)|
\]

% 6

\stheorem{}{
    Sia \((X, \Sigma, \mu)\) uno spazio di misura e
    \[
        f_n \colon X \to [0; +\infty)
    \] con \(f_n \geq f_{n+1}\)
    Allora
    \[
        \lim_n \int_X f_n\,d\mu = \int_X (\lim_n f_n)\,d\mu
    \]
}

Sia per esempio \(f_n = \chi_{\{1\}} + \chi_{\{n\}}\).
Allora la funzione converge puntualmente in quanto l'1 si sposta sempre più a destra.
Abbiamo
\[
    \int_{\mathbb{N}} f_n \,d\mu = 2 \to 2
\]

Se invece \(f_n \geq f_{n+1}\) allora \(f_n = \chi_{\{n, n+1, \cdots\}}\), allora tende a zero.
Tuttavia, l'integrale di \(f_n\) è infinito in quanto la misura dell'insieme è infinita.

\sproof{}{
    Abbiamo
    \[
        f_n \leq f_{n+1} \cdots \leq f, \quad f = \lim_n f_n
    \]
    Quindi
    \[
        \int_X f_n \,d\mu \leq \int_X f_{n+1}\,d\mu \leq \int_X f\,d\mu
    \]
    quindi anchde la successione degli integrali è monotona e ammette limite.
    Il limite sarà sempre più piccolo dell'ultimo valore.
    \[
        \lim_n \int_X f_n\,d\mu \leq \int_X f\,d\mu
    \]
    Facciamo ora il caso \(\geq\). Sia \(0 \leq \varphi \leq f\) una funzione semplice
    \[
        \varphi = \sum_{i=1}^N \alpha_i \chi_{E_i}, \quad a_i \geq 0
    \]
    e prendiamo \(c\in (0,1)\). Considegliamo gli insiemi
    \[
        A_n = \{f_n \geq c\varphi\}
    \]
    Tali insiemi sono misurabili, in quanto sto moltiplicando una funzione misurabile per una costante
    e l'insieme \(\{f\geq g\}\) è come dire \(\{f-g \geq 0\}\).
    Sappiamo 
    \begin{enumerate}
        \item \(A_n \in A_{n+1}\) in quanto \(c\varphi(x) \leq f_n(x) \leq f_{n+1}(x)\);
        \item \(\bigcup A_n = X\). Sia \(x\in X\). Se \(\varphi(x) = 0\) allora è in \(A_n\).
        Se invece \(\varphi(x) > 0\), ma siccome \(\varphi \leq f\), e \(c<1\), allora 
        \[
            c\varphi(x) < \varphi(x) \leq f(x)
        \]
        La succesione, da un certo posto in poi, è più grande di \(c\varphi(x)\) (ne basta uno),
        quindi \(x\in A_n\).
    \end{enumerate}
    Osserviamo che
    \begin{align*}
        E_i &= E_i \cap X \\
        &= E_i \cap (\bigcup A_n) \\
        &= \bigcup_n (E_i \cap A_n)
    \end{align*}
    Quindi \(E_i \cap A_n \subseteq E_i \cap A_{n+1}\) è una successione di insiemi che si sta allargando.
    Quindi, la misura dell'union è il limite.
    \[
        \mu(E_i) = \lim_{n\to\infty} E_i \cap A_n
    \]
    Consideriamo
    \begin{align*}
        \int_X f_n \, d\mu &\geq \int_{A_n} f_n \, d\mu \geq c \int_{A_n} \varphi \,d\mu \\
        &= c \int_X \varphi \chi_{A_n} \,d\mu = c\sum_{i=1}^N \alpha_i \mu(E_i \cap A_n)
    \end{align*}
    Facciamo ora il limite
    \begin{align*}
        \lim_n \int_X f_n \,d\mu &\geq c\lim_n \sum_{i=1}^N \alpha_i \mu(E_i \cap A_n) \\
        &= c\sum_{i=1}^N \alpha_i \mu(E_i) = c\int_X \varphi\,d\mu
    \end{align*}
    Abbiamo quindi ottenuto che
    \[
        \lim_n \int_X f_n \,d\mu \geq c\int_X \varphi\,d\mu
    \]
    vale per tutti i \(c\in(0,1)\), e allora possiamo usare il supremum
    \[
        \lim_n \int_X f_n \,d\mu \geq \int_X \varphi\,d\mu
    \]
    Non solo vale per ogni \(c\), ma per ogni funzione semplice tale che \(0\leq \varphi \leq f\).
    In particolare anche per il supremum. Il supremum di questi integrali al variare di tutte le funzioni semplici
    minori di \(f\) è l'integrale di \(f\), cioè la definizione
    \[
        \lim_n \int_X f_n \,d\mu \geq \int_X f\,d\mu
    \]
}

Mettendo assieme le due cose otteniamo l'uguaglianza

\[
    \lim_n \int_X f_n \,d\mu = \int_X f\,d\mu
\]

\scorollary{}{
    Allora
    \[
        \sum_{n=1}^\infty \int_X f_n\,d\mu = \int_X \sum_{n=1}^\infty f_n\,d\mu
    \]
}

Siccome i termini sono tutti positivi, la successione delle serie parziale è monotona.

\slemma{Lemma di Fatou}{
    Sia \(f_n \colon X \to [0, +\infty)\) misurabili, allora
    \[
        \int_X \liminf f_n\,d\mu \leq \liminf \int_X f_n\,d\mu
    \]
    (l'integrale esiste sempre)
}

\sproof{}{
    Consideriamo
    \[
        g_n = \inf_{k\geq n} f_k
    \]
    chiaramente \(g_n \leq g_{n+1} \to \liminf f_n\) e sono misurabili.
    Consideriamo allora l'integrale
    \begin{align*}
        \lim \int_X g_n \,d\mu = \int_X \liminf f_n\,d\mu
    \end{align*}
    e per il teorema della convergenza monotona e definizione di lim inf
    \begin{align*}
        \int_X \liminf f_n\,d\mu &= \lim_n \int_X (\inf_n{k\geq n} f_k)\,d\mu \\
        &\leq \liminf \int_X f_n\,d\mu
    \end{align*}
}

\sdefinition{Integrabilità di una funzione positiva}{
    Sia \(f\colon X \to [0; +\infty)\) misurabile.
    Allora \(f\) è \emph{integrabile} su \(X\) se
    \[
        \int_X f\,d\mu \leq \infty
    \]
}

Diciamo che \(f \in L^1(\{X, \Sigma, \mu\})\).
Per esempio \(\{1/n^2\} \in L^1(\mathbb{N})\) ma \(\{1/n\} \notin L^1(\mathbb{N})\)

\sdefinition{Integrabilità di una funzione}{
    Sia \(f\colon X \to \mathbb{R}\) misurabile.
    Allora \(f\) è \emph{integrabile} se \(f^+\) e \(f^-\) sono integrabili (che sono entrambe funzioni positive).
}

Dobbiamo tuttavia definire l'integrale di una funzione di segno arbitraria. Sia allora
\[
    \int_X f\,d\mu = \int_X f^+\,d\mu - \int_X f^-\,d\mu
\]

Consideriamo per esempio
\[
    f = (1,-\frac{1}{2}, \frac{1}{3}, -\frac{1}{4})
\]
Allora
\[
    f^+ = (1,0, \frac{1}{3}, 0)
\]
e
\[
    f^- = (0,\frac{1}{2}, 0, \frac{1}{4})
\]
L'integrale non converge in quanto i due integrali non convergono (le serie divergono per confronto asintotico).

\sproposition{}{
    Siano \(f,g \in L^1\).
    \begin{enumerate}
        \item \(\alpha f + \beta b \in L^1\) e \[
            \int_X (\alpha f + \beta g)\,d\mu = \alpha \int_X f\,d\mu + \beta \int_X g\,d\mu
        \]
        Quindi lo spazio delle funzioni integrabili è uno spazio vettoriale.
        \item  \[
            f \leq g \implies \int_X f\,d\mu \leq \int_X g\,d\mu
        \]
        \item \(f \in L^1 \iff |f| \in L^1\). Infatti \(f^+ f^- = |f|\) e per la direzione inserva
        abbiamo \(0 \leq f^+ \leq |f|\). Ma se l'integrale del modulo è finito allora lo sarà anche quello
        di \(f^+\) che è più piccolo. Lo stesso vale per la parte negativa.
        \item Se \(f\) è misurabile allora lo è anche \(|f|\), ma il viceversa non è vero.
        Per esempio sia \(X = \{a,b,c\}\) e \(\Sigma = \{X, \emptyset, \{a\}, \{b, c\}\}\).
        Sia allora
        \[
            f = \begin{cases}
                1 & x=a \lor x=b \\
                -1 & x = c
            \end{cases}
        \]
        Chiaramente \(\{f < 0\} = \{c\}\) non è misurabile, ma \(|f| = 1\) per tutte le \(x\)
        e le funzioni costanti sono sempre misurabili.
        \item \[
            \left|\int_X f\,d\mu\right| \leq \int_X |f|\,d\mu
        \]
        Infatti
        \begin{align*}
            \left|\int_X (f^+ - f^-)\,d\mu\right| &= \left|\int_X f^+\,d\mu - \int_X f^-\,d\mu\right| \\
            &\leq \left|\int_X f^+\,d\mu\right| + \left|\int_X f^-\,d\mu\right| \\
            &= \int_X f^+ \,d\mu + \int_X f^- \,d\mu \\
            &= \int_X (f^+ + f^-)\,d\mu \\
            &= \int_X |f| \,d\mu 
        \end{align*}
    \end{enumerate}
}

\stheorem{Teorema della convergenza dominante}{
    Sia \(f_n \colon X \to \mathbb{R}\) misurabile e sia \(f = \lim_n f_n\).
    Supponiamo che ci sia \(g\in L^1\) tale che \(|f_n| \leq g\) in \(X\).
    Allora
    \[
        \lim_n \int_X f_n\,d\mu = \int_X f\,d\mu
    \]
}

\sproof{}{
    \(f_n\) sono integrabili in quanto \(|f_n| \leq g\) che è integrabili, quindi sarà finito anche
    l'integrale del modulo, e \(f\) è integrabile perché ciò vale anche per il limite.
    Allora \(|f-f_n| \leq 2g\) quindi \(2g - |f-f_n| \geq 0\). Siccome quest'ultima è una successione positiva
    posso applicare il lemma di Fatou
    \[
        \int_X \liminf (2g - |f-f_n|)\,d\mu \leq \liminf
        \int_X (2g - |f-f_n|)\,d\mu
    \]
    Ma per le proprietà dei lim inf possiamo estrarre la costante
    \begin{align*}
        \int_X 2g - \lim|f-f_n| &= \int_X 2g \\
        &\leq \liminf \left( \int_X 2g\,d\mu - \int_X |f-f_n|\,d\mu \right) \\
        &= \int_X 2g \,d\mu - \limsup \int_X |f-f_n|\,d\mu
    \end{align*}
    Abbiamo quindi
    \begin{align*}
        \int_X 2g\,d\mu &\leq \int_X 2g\,d\mu - \limsup \int_X |f-f_n|\,d\mu \\
        \limsup \int_X |f-f_n|\,d\mu &\leq 0
    \end{align*}
    Ma quindi questo limite deve essere ed essere uguale a zero
    \[
        \int_X |f-f_n|\,d\mu = 0
    \]
    Infine, usando il modulo
    \begin{align*}
        \lim \left|
            \int_X f_n \,d\mu - \int_X f \,d\mu
        \right| \leq \lim \int_X |f_n-f|\,d\mu = 0
    \end{align*}
    siccome è tutto positivo deve essere 
    \[
        \lim \left|
            \int_X f_n \,d\mu - \int_X f \,d\mu
        \right| = 0
    \]
}

Se \(A \subset X\) con \(A\) integrabile e \(f\colon X \to \mathbb{R}\)
misurabile, \(f\) è integrabile in \(A\) se \(f\chi_A\) è integrabile.
Chiaramente definiamo
\[
    \int_A f\,d\mu = \int_X f\chi_A\,d\mu
\]
Quindi per vedere se è integrabile nel sottoinsieme la estendiamo su tutto lo spazio con
la funzione caratteristica e integriamo.

Costruiamo ora una misura su \(R\) (la misura di Lebesuge).
Vogliamo che sia invariante per traslazione \(\mu(A) = \mu(A + c)\) dove \(c\) è una costante.
Vorremmo anche che \(\mu([b,a]) = b-a\). Tuttavia, non è possibile costruire tale misura su tutto
\(\mathbb{R}\). Sia allora \(I=(a,b)\) (non cambia se incluso o meno)
e denotiamo \(l(I) = b-a\). Sia anche \(E \subset \mathbb{R}\). Diamo la \emph{misura esterna}
\[
    \mu^*(E) = \inf \left\{
        \sum_{n=1}^\infty l(I_n) \,|\, E \subset \bigcup_n I_n
    \right\}
\]
Alcune proprietà di questa ipotetica misura
\begin{enumerate}
    \item \(\mu^*(\emptyset) = 0\);
    \item \(\mu^*(\{x\}) = 0\) dove \(\{x\} \subset (x-\varepsilon, x + \varepsilon)\);
    \item se \(E\) numerabile, allora \(\mu^*(E) = 0\)
    \[
        E \subset \{x_n\}
    \]
    \[
        I_n = \left(x_n - \frac{\varepsilon}{2^n}, x_n + \frac{\varepsilon}{2^n}\right)
    \]
    \[
        E \subset \bigcup I_n
    \]
    \begin{align*}
        \sum_{n=1}^\infty l(I_n) &= \sum_{n=1}^\infty \frac{\varepsilon}{2^{n-1}} \\
        &= \varepsilon \sum_{n=0}^\infty \frac{1}{2^n} = 2\varepsilon
    \end{align*}
    \item \(\mu^*(E + x) = \mu^*(E)\) (invariante per traslazione).
    \item subadditività \[
        \mu^*\left(\bigcup E_n\right) \leq \sum_n \mu(E_n)
    \]
    \item \(\mu^*(I) = b-a\)
\end{enumerate}
Se tutto fosse vero, abbiamo quello che cerchiamo, ma in realtà quando gli insiemi sono disgiunti
l'ugualgianza non vale, quindi non esiste tale misura.

Vale sempre \(\mu^*(I) \leq b-a\) perché c'è l'inf, c'è sempre un ricoprimento.
La misura esterna è almeno quel valore, magari più piccolo, vale lo stesso.

\end{document}
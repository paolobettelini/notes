\documentclass[a4paper]{article}

\usepackage{amsmath}
\usepackage{amssymb}
\usepackage{stellar}
\usepackage{parskip}
\usepackage{fullpage}
\usepackage{wrapfig}

\title{Analisi III}
\author{Paolo Bettelini}
\date{}

\begin{document}

\maketitle
\tableofcontents

\section{Successione di funzioni}

\sdefinition{Successione di funzioni}{
    Una \emph{successione di funzioni} è una famiglia di funzioni \(\{f_n\}_{n\in\mathbb{N}}\)
    definite su un dominio comune \(f_n \colon D \to \mathbb{R}\).
}

\sdefinition{Convergenza in un punto}{
    Sia \(\{f_n\}_{n\in\mathbb{N}}\) una successione di funzioni.
    La successione converge in un punto \(x_0\) se
    \[
        \lim_{n\to\infty} f_n(x_0) < \infty
    \]
}

\sdefinition{Convergenza puntuale}{
    Sia \(\{f_n\}_{n\in\mathbb{N}}\) una successione di funzioni.
    La successione \emph{converge puntualmente} ad una funzione \(f\colon D \to \mathbb{R}\) se
    \[
        \forall x\in D, \lim_{n\to\infty} f_n(x) = f(x)
    \]
}

Quindi la successione converge in ogni punto, ma la velocità di convergenza può dipenderere dal punto.

\sdefinition{Convergenza uniforme}{
    Sia \(\{f_n\}_{n\in\mathbb{N}}\) una successione di funzioni.
    La successione \emph{converge uniformemente} ad una funzione \(f\colon D \to \mathbb{R}\) se
    \[
        \sup_{x\in D} \left| f_n(x) - f(x) \right| \to 0
    \]
    per \(n\to\infty\).
}

Oppure possiamo dire che la condizione è che
\[
    \forall \varepsilon > 0, \exists N \,|\, \forall n>N, ||f_n-f||_{\infty, E} < \varepsilon
\]

Dovremmo dire che la differenza
\[
    |f_n(x) - f| \leq \varepsilon
\]
ma siccome ciò deve valere per tutte le \(x\) possiamo utilizzare il supremum.

Quindi la velocità di convergenza è la stessa in ogni punto. 
Ogni cosa che converge uniformemente converge puntualmente.

\sdefinition{Convergenza uniformemente di Cauchy}{
    Sia \(\{f_n\}_{n\in\mathbb{N}}\) una successione di funzioni.
    La successione è \emph{uniformemente di Cauchy} se
    \[
        \forall \varepsilon > 0,
        \exists N \in \mathbb{N} \,|\,
        \forall n,m > N,
        \sup_{x\in D} \left| f_n(x) - f_m(x) \right| < \varepsilon
    \]
}

A partire da un certo indice, tutte le funzioni della successione sono molto vicine tra loro in modo uniforme su tutto \(D\),
indipendentemente dalla funzione limite.

\stheorem{Convergenza uniforme e convergenza uniformemente di Cauchy}{
    Sia \(\{f_n\}_{n\in\mathbb{N}}\) una successione di funzioni.
    Se la successione è uniformemente di Cauchy allora è uniformemente convergente.
}

\stheorem{}{
    Sia \(\{f_n\}\) convergente uniformemente a \(f\) in \(E\) e sia \(x_0\in E\) un punto di accumulazione di \(E\).
    Supponiamo che esista
    \[
        \exists\, \lim_{x\to x_0} f_n(x) = \lambda_n
    \]
    per ogni \(n\), allora
    \begin{enumerate}
        \item \(\lambda_n \to \lambda\),
        \item \[
            \lim_{x\to x_0} f(x) = \lambda
        \]
    \end{enumerate}
}

\sproof{}{
    \begin{enumerate}
        \item \[
            |\lambda_n - \lambda_m| = \lim_{x\to x_0} |f_n(x) - f_n(x)| \leq \lim_{x\to x_0}
            ||f_n-f_m||_{\infty, E} < \varepsilon
        \]
        dunque è di Cauchy e converge al limite \(\lambda_n \to \lambda\).
        \item \begin{align*}
            |f(x) - \lambda| &\leq |f(x) - f_n(x)| + |f_n(x) - \lambda_n| + |\lambda_n - \lambda| \\
            &\leq ||f-f_n||_{\infty, E} + |f_n(x) - \lambda_n|
        \end{align*}
        dunque se \(\overline{n} = \max\{N_1, N_2\}\)
        \[
            |f(x) - \lambda| \leq 2\varepsilon + |f_{\overline{n}}(x) - \lambda_{\overline{n}}| \leq 3\varepsilon
        \]
        quindi
        \[
            f_{\overline{n}}(x) - \lambda_{\overline{n}} \leq \varepsilon
        \]
    \end{enumerate}
}

Quindi, se abbiamo convergenza uniforme,
\begin{align*}
    \lim_{x\to x_0} f(x) &= \lim_{x\to x_0} \left(\lim_{x\to\infty} f_n(x) \right) \\
    &= \lambda = \lim_{n\to\infty} \lambda_n = \lim_{n\to\infty} \left(\lim_{x\to x_0} f_n(x)\right)
\end{align*}
possiamo scambiare l'ordine.

\scorollary{}{
    Se \(f_n\) sono continue e \(f_n\to f\), allora \(f\) è continua.
}

\stheorem{}{
    Sia \(f_n\colon [a,b] \to \mathbb{R}\) una successione di funzioni R-integrabili
    dove \(f_n \to f\) in \([a,b]\). Allora \(f\) è R-integrabile e
    \[
        \lim_{n\to\infty} \integral[a][b][f_n(x)][x]
        = \integral[a][b][\lim_{n\to\infty} f_n(x)][x]
        = \integral[a][b][f(x)][x]
    \]
}

\sproof{}{
    Supponiamo anche che \(f_n\) siano continue.
    \begin{enumerate}
        \item \(f\) è continua e quindi R-integrabile;
        \item mostriamo che vale l'uguale, cioè \(\forall m,n \geq N\),
        \begin{align*}
            \left|\integral[a][b][f_n(x)][x] - \integral[a][b][f(x)][x]\right| &\leq
            \integral[a][b][f_n(x) - f(x)][x] \\
            &\leq \integral[a][b][||f_n - f||_{\infty, [a,b]}][x] \leq \varepsilon(b-a)
        \end{align*}
        (cioè tende a zero) per \(n\geq N\).
    \end{enumerate}
}

\stheorem{}{
    Sia \(f_n\colon [a,b] \to \mathbb{R}\) una successione di funzioni derivabili.
    Supponiamo che:
    \begin{enumerate}
        \item \(\exists x_0 \in [a,b]\) tale che \(f_n\) converge in \(x_0\);
        \item \(f_n'\) converge uniformemente in \(g\) a \([a,b]\).
    \end{enumerate}
    Allora,
    \begin{enumerate}
        \item \(f_n\) converge uniformemente a \(f\) in \([a,b]\);
        \item \(f\) è derivabile;
        \item \(f'(x) = g(x)\) per ogni \(x \in [a,b]\).
    \end{enumerate}
}

\section{Serie di funzioni}

\sdefinition{Convergenza uniforme}{
    La serie di funzioni \(\sum_{n=1}^\infty f_n(x)\) \emph{converge uniformemente} ad una funzione \(S(x)\)
    se la successione delle somme parziali
    \[
        S_N(x) = \sum_{n=1}^N f_n(x)
    \]
    converge uniformemente a \(S(x)\), ovvero se
    \[
        \sup_{x\in D}|S_N(x) - S(x)| \to 0
    \]
    per \(N\to\infty\).
}

È più forte della convergenza puntuale.

\sdefinition{Convergenza totale}{
    Una serie di funzioni \(\sum_{n=1}^\infty f_n(x)\) \emph{converge totalmente} su un insieme \(D\) se la serie di norme
    \[
        \sum ||f_n||_\infty
    \]
    converge.
}

Ricordiamo che in generale la norma

\[
    ||f||_p = {\left(
        \integral[a][b][|f(x)|^p][x]
    \right)}^{\frac{1}{p}}, \quad 1 \leq p < \infty
\]
e per \(p=\infty\) con \(f\) limitata
\[
    ||f||_\infty = \sup_{x\in D} |f(x)|
\]
che è un numero siccome \(f\) è limitata

\stheorem{}{
    XXXX. Se ho la convergenza uniforme posso invertire integrale e serie.
}

% 6

\section{Integrazione}

%%%%%%%%%
\stheorem{Monotone convergence theorem for non-negative measurable functions}{
    Let \((X, \Sigma, \mu)\) be a measurable space and let
    \[
        f_n \colon X \to [0; +\infty)
    \] be measurable
    such that \(f_n \leq f_{n+1}\). Then,
    \[
        \lim_n \int_X f_n\,d\mu = \int_X (\lim_n f_n)\,d\mu
    \]
}

Sia per esempio \(f_n = \chi_{\{1\}} + \chi_{\{n\}}\).
Allora la funzione converge puntualmente in quanto l'1 si sposta sempre più a destra.
Abbiamo
\[
    \int_{\mathbb{N}} f_n \,d\mu = 2 \to 2
\]

Se invece \(f_n \geq f_{n+1}\) allora \(f_n = \chi_{\{n, n+1, \cdots\}}\), allora tende a zero.
Tuttavia, l'integrale di \(f_n\) è infinito in quanto la misura dell'insieme è infinita.

\sproof{}{
    Abbiamo
    \[
        f_n \leq f_{n+1} \cdots \leq f, \quad f = \lim_n f_n
    \]
    Quindi
    \[
        \int_X f_n \,d\mu \leq \int_X f_{n+1}\,d\mu \leq \int_X f\,d\mu
    \]
    quindi anchde la successione degli integrali è monotona e ammette limite.
    Il limite sarà sempre più piccolo dell'ultimo valore.
    \[
        \lim_n \int_X f_n\,d\mu \leq \int_X f\,d\mu
    \]
    Facciamo ora il caso \(\geq\). Sia \(0 \leq \varphi \leq f\) una funzione semplice
    \[
        \varphi = \sum_{i=1}^N \alpha_i \chi_{E_i}, \quad \alpha_i \geq 0
    \]
    e prendiamo \(c\in (0,1)\). Considegliamo gli insiemi
    \[
        A_n = \{f_n \geq c\varphi\}
    \]
    Tali insiemi sono misurabili, in quanto sto moltiplicando una funzione misurabile per una costante
    e l'insieme \(\{f\geq g\}\) è come dire \(\{f-g \geq 0\}\).
    Sappiamo 
    \begin{enumerate}
        \item \(A_n \in A_{n+1}\) in quanto \(c\varphi(x) \leq f_n(x) \leq f_{n+1}(x)\);
        \item \(\bigcup A_n = X\). Sia \(x\in X\). Se \(\varphi(x) = 0\) allora è in \(A_n\).
        Se invece \(\varphi(x) > 0\), ma siccome \(\varphi \leq f\), e \(c<1\), allora 
        \[
            c\varphi(x) < \varphi(x) \leq f(x)
        \]
        La succesione, da un certo posto in poi, è più grande di \(c\varphi(x)\) (ne basta uno),
        quindi \(x\in A_n\).
    \end{enumerate}
    Osserviamo che
    \begin{align*}
        E_i &= E_i \cap X \\
        &= E_i \cap (\bigcup A_n) \\
        &= \bigcup_n (E_i \cap A_n)
    \end{align*}
    Quindi \(E_i \cap A_n \subseteq E_i \cap A_{n+1}\) è una successione di insiemi che si sta allargando.
    Quindi, la misura dell'union è il limite.
    \[
        \mu(E_i) = \lim_{n\to\infty} E_i \cap A_n
    \]
    Consideriamo
    \begin{align*}
        \int_X f_n \, d\mu &\geq \int_{A_n} f_n \, d\mu \geq c \int_{A_n} \varphi \,d\mu \\
        &= c \int_X \varphi \chi_{A_n} \,d\mu = c\sum_{i=1}^N \alpha_i \mu(E_i \cap A_n)
    \end{align*}
    Facciamo ora il limite
    \begin{align*}
        \lim_n \int_X f_n \,d\mu &\geq c\lim_n \sum_{i=1}^N \alpha_i \mu(E_i \cap A_n) \\
        &= c\sum_{i=1}^N \alpha_i \mu(E_i) = c\int_X \varphi\,d\mu
    \end{align*}
    Abbiamo quindi ottenuto che
    \[
        \lim_n \int_X f_n \,d\mu \geq c\int_X \varphi\,d\mu
    \]
    vale per tutti i \(c\in(0,1)\), e allora possiamo usare il supremum
    \[
        \lim_n \int_X f_n \,d\mu \geq \int_X \varphi\,d\mu
    \]
    Non solo vale per ogni \(c\), ma per ogni funzione semplice tale che \(0\leq \varphi \leq f\).
    In particolare anche per il supremum. Il supremum di questi integrali al variare di tutte le funzioni semplici
    minori di \(f\) è l'integrale di \(f\), cioè la definizione
    \[
        \lim_n \int_X f_n \,d\mu \geq \int_X f\,d\mu
    \]
    Mettendo assieme le due cose otteniamo l'uguaglianza
    
    \[
        \lim_n \int_X f_n \,d\mu = \int_X f\,d\mu
    \]
}

\scorollary{}{
    Allora
    \[
        \sum_{n=1}^\infty \int_X f_n\,d\mu = \int_X \sum_{n=1}^\infty f_n\,d\mu
    \]
}

\sproof{}{
    Siccome i termini sono tutti positivi, la successione delle serie parziale è monotona.
}

%%%%%%%
\slemma{Lemma di Fatou}{
    Sia \(f_n \colon X \to [0, +\infty)\) misurabili, allora
    \[
        \int_X \liminf f_n\,d\mu \leq \liminf \int_X f_n\,d\mu
    \]
    (l'integrale esiste sempre)
}

\sproof{}{
    Consideriamo
    \[
        g_n = \inf_{k\geq n} f_k
    \]
    chiaramente \(g_n \leq g_{n+1} \to \liminf f_n\) e sono misurabili.
    Consideriamo allora l'integrale
    \begin{align*}
        \lim \int_X g_n \,d\mu = \int_X \liminf f_n\,d\mu
    \end{align*}
    e per il teorema della convergenza monotona e definizione di lim inf
    \begin{align*}
        \int_X \liminf f_n\,d\mu &= \lim_n \int_X (\inf_n{k\geq n} f_k)\,d\mu \\
        &\leq \liminf \int_X f_n\,d\mu
    \end{align*}
}

\sdefinition{Integrabilità di una funzione positiva}{
    Sia \(f\colon X \to [0; +\infty)\) misurabile.
    Allora \(f\) è \emph{integrabile} su \(X\) se
    \[
        \int_X f\,d\mu < \infty
    \]
}

Diciamo che \(f \in L^1(\{X, \Sigma, \mu\})\).
Per esempio \(\{1/n^2\} \in L^1(\mathbb{N})\) ma \(\{1/n\} \notin L^1(\mathbb{N})\)

\sdefinition{Integrabilità di una funzione}{
    Sia \(f\colon X \to \mathbb{R}\) misurabile.
    Allora \(f\) è \emph{integrabile} se \(f^+\) e \(f^-\) sono integrabili (che sono entrambe funzioni positive).
}

Dobbiamo tuttavia definire l'integrale di una funzione di segno arbitraria. Sia allora
\[
    \int_X f\,d\mu = \int_X f^+\,d\mu - \int_X f^-\,d\mu
\]

Consideriamo per esempio
\[
    f = (1,-\frac{1}{2}, \frac{1}{3}, -\frac{1}{4})
\]
Allora
\[
    f^+ = (1,0, \frac{1}{3}, 0)
\]
e
\[
    f^- = (0,\frac{1}{2}, 0, \frac{1}{4})
\]
L'integrale non converge in quanto i due integrali non convergono (le serie divergono per confronto asintotico).

\sproposition{}{
    Siano \(f,g \in L^1\).
    \begin{enumerate}
        \item \(\alpha f + \beta b \in L^1\) e \[
            \int_X (\alpha f + \beta g)\,d\mu = \alpha \int_X f\,d\mu + \beta \int_X g\,d\mu
        \]
        Quindi lo spazio delle funzioni integrabili è uno spazio vettoriale.
        \item  \[
            f \leq g \implies \int_X f\,d\mu \leq \int_X g\,d\mu
        \]
        \item \(f \in L^1 \iff |f| \in L^1\). Infatti \(f^+ f^- = |f|\) e per la direzione inserva
        abbiamo \(0 \leq f^+ \leq |f|\). Ma se l'integrale del modulo è finito allora lo sarà anche quello
        di \(f^+\) che è più piccolo. Lo stesso vale per la parte negativa.
        \item Se \(f\) è misurabile allora lo è anche \(|f|\), ma il viceversa non è vero.
        Per esempio sia \(X = \{a,b,c\}\) e \(\Sigma = \{X, \emptyset, \{a\}, \{b, c\}\}\).
        Sia allora
        \[
            f = \begin{cases}
                1 & x=a \lor x=b \\
                -1 & x = c
            \end{cases}
        \]
        Chiaramente \(\{f < 0\} = \{c\}\) non è misurabile, ma \(|f| = 1\) per tutte le \(x\)
        e le funzioni costanti sono sempre misurabili.
        \item \[
            \left|\int_X f\,d\mu\right| \leq \int_X |f|\,d\mu
        \]
        Infatti
        \begin{align*}
            \left|\int_X (f^+ - f^-)\,d\mu\right| &= \left|\int_X f^+\,d\mu - \int_X f^-\,d\mu\right| \\
            &\leq \left|\int_X f^+\,d\mu\right| + \left|\int_X f^-\,d\mu\right| \\
            &= \int_X f^+ \,d\mu + \int_X f^- \,d\mu \\
            &= \int_X (f^+ + f^-)\,d\mu \\
            &= \int_X |f| \,d\mu 
        \end{align*}
    \end{enumerate}
}

\stheorem{Teorema della convergenza dominante}{
    Sia \(f_n \colon X \to \mathbb{R}\) misurabile e sia \(f = \lim_n f_n\).
    Supponiamo che ci sia \(g\in L^1\) tale che \(|f_n| \leq g\) in \(X\).
    Allora
    \[
        \lim_n \int_X f_n\,d\mu = \int_X f\,d\mu
    \]
}

\sproof{}{
    \(f_n\) sono integrabili in quanto \(|f_n| \leq g\) che è integrabili, quindi sarà finito anche
    l'integrale del modulo, e \(f\) è integrabile perché ciò vale anche per il limite.
    Allora \(|f-f_n| \leq 2g\) quindi \(2g - |f-f_n| \geq 0\). Siccome quest'ultima è una successione positiva
    posso applicare il lemma di Fatou
    \[
        \int_X \liminf (2g - |f-f_n|)\,d\mu \leq \liminf
        \int_X (2g - |f-f_n|)\,d\mu
    \]
    Ma per le proprietà dei lim inf possiamo estrarre la costante
    \begin{align*}
        \int_X 2g - \lim|f-f_n| &= \int_X 2g \\
        &\leq \liminf \left( \int_X 2g\,d\mu - \int_X |f-f_n|\,d\mu \right) \\
        &= \int_X 2g \,d\mu - \limsup \int_X |f-f_n|\,d\mu
    \end{align*}
    Abbiamo quindi
    \begin{align*}
        \int_X 2g\,d\mu &\leq \int_X 2g\,d\mu - \limsup \int_X |f-f_n|\,d\mu \\
        \limsup \int_X |f-f_n|\,d\mu &\leq 0
    \end{align*}
    Ma quindi questo limite deve essere ed essere uguale a zero
    \[
        \int_X |f-f_n|\,d\mu = 0
    \]
    Infine, usando il modulo
    \begin{align*}
        \lim \left|
            \int_X f_n \,d\mu - \int_X f \,d\mu
        \right| \leq \lim \int_X |f_n-f|\,d\mu = 0
    \end{align*}
    siccome è tutto positivo deve essere 
    \[
        \lim \left|
            \int_X f_n \,d\mu - \int_X f \,d\mu
        \right| = 0
    \]
}

Se \(A \subseteq X\) con \(A\) integrabile e \(f\colon X \to \mathbb{R}\)
misurabile, \(f\) è integrabile in \(A\) se \(f\chi_A\) è integrabile.
Chiaramente definiamo
\[
    \int_A f\,d\mu = \int_X f\chi_A\,d\mu
\]
Quindi per vedere se è integrabile nel sottoinsieme la estendiamo su tutto lo spazio con
la funzione caratteristica e integriamo.

Costruiamo ora una misura su \(R\) (la misura di Lebesuge).
Vogliamo che sia invariante per traslazione \(\mu(A) = \mu(A + c)\) dove \(c\) è una costante.
Vorremmo anche che \(\mu([b,a]) = b-a\). Tuttavia, non è possibile costruire tale misura su tutto
\(\mathbb{R}\). Sia allora \(I=(a,b)\) (non cambia se incluso o meno)
e denotiamo \(l(I) = b-a\). Sia anche \(E \subset \mathbb{R}\). Diamo la \emph{misura esterna}
\[
    \mu^*(E) = \inf \left\{
        \sum_{n=1}^\infty l(I_n) \,|\, E \subset \bigcup_n I_n
    \right\}
\]
Alcune proprietà di questa ipotetica misura
\begin{enumerate}
    \item \(\mu^*(\emptyset) = 0\);
    \item \(\mu^*(\{x\}) = 0\) dove \(\{x\} \subset (x-\varepsilon, x + \varepsilon)\);
    \item se \(E\) numerabile, allora \(\mu^*(E) = 0\)
    \[
        E \subset \{x_n\}
    \]
    \[
        I_n = \left(x_n - \frac{\varepsilon}{2^n}, x_n + \frac{\varepsilon}{2^n}\right)
    \]
    \[
        E \subset \bigcup I_n
    \]
    \begin{align*}
        \sum_{n=1}^\infty l(I_n) &= \sum_{n=1}^\infty \frac{\varepsilon}{2^{n-1}} \\
        &= \varepsilon \sum_{n=0}^\infty \frac{1}{2^n} = 2\varepsilon
    \end{align*}
    \item \(\mu^*(E + x) = \mu^*(E)\) (invariante per traslazione).
    \item subadditività (numerabile) \[
        \mu^*\left(\bigcup E_n\right) \leq \sum_n \mu(E_n)
    \]
    \item \(\mu^*(I) = b-a\)
\end{enumerate}

Se tutto fosse vero, abbiamo quello che cerchiamo, ma in realtà quando gli insiemi sono disgiunti
l'ugualgianza non vale, quindi non esiste tale misura.

Vale sempre \(\mu^*(I) \leq b-a\) perché c'è l'inf, c'è sempre un ricoprimento.
La misura esterna è almeno quel valore, magari più piccolo, vale lo stesso.

Vogliamo mostrare la subadditività (numerabile).
Per definizione possiamo prendere \(E_n\) come un'unione di intervalli numerati
\[
    E_n \subseteq \bigcup_k I^n_k
\]
quindi, per avvicinarsi alla misura
\begin{align*}
    \sum_{k=1} l(I^n_k) \leq \mu^*(E_n) + \frac{\varepsilon}{2^n}
\end{align*}
Chairamente l'unione di \(E_n\) è ricoperta da un unione di unioni
\[
    \bigcup E_n \subseteq \bigcup_n \left(\bigcup_k I_k^n\right)
\]
E per definizione la misura di tale unione
\begin{align*}
    \mu^*\left(\bigcup E_n\right) &\leq \sum_n \left(\sum_k l(I_k^n)\right) \\
    &\leq \sum_n \left(\mu^* (E_n) + \frac{\varepsilon}{2^n}\right) \\
    &= \sum_n \mu^*(E_n) + \varepsilon
\end{align*}
Siccome \([a,b] \subset (a-\varepsilon, b + \varepsilon)\) è una possibile ricopritura,
abbiamo
\[
    \mu^*([b,a]) \leq b-a + 2\varepsilon
\]
Ora facciamo il contrario; mostriamo che per ogni ricoprimento
\([a,b] \subseteq \bigcup I_n\), la serie di tutte quelle lunghezze è almeno \(b-a\).
L'insieme \(\bigcup I_n\) è compatto e quindi ha un ricoprimento finito. Possiamo
estrarre un sottoricoprimento finito che lo ricopre ancora.
Quindi possiamo immaginarci
\[
    [a,b] \subseteq I_1 \cup \cdots \cup I_n
\]
Vogliamo mostrare che se i ricoprimenti finiti hanno lunghezza almeno \(b-a\), quindi anche quelli infiniti.
Siccome usiamo intervalli aperti, vogliamo che gli altri intervalli si sovrappongano per coprire anche gli estremi,
che non sono coperti.
Impostiamo allora la condizione che \(a_1 < a\), \(a_2 < b_1\), \(a_3 < b_2\).
Quindi in generale ci spostiamo verso destra con \(a_k - b_{k-1}\). L'ultimo intervallo
deve contenere \(b\) quindi \(b_n > b\).
Quindi, dato un ricoprimento qualsiasi, possiamo sempre trovare un sottoricoprimento in questa maniera.
Abbiamo allora la sommatoria
\begin{align*}
    \sum_{k=1}^n l(I_k) &= b_n - a_n + b_{n-1} - a_{n-1} + \cdots + b_2 - a_2 + b_1 - a_1 \\
    &= b_n + (b_{n-1} - a_n) + (b_{n-2} - a_{n-1}) + \cdots + (b_1 - a_2) - a_1
\end{align*}
Siccome \(a_k - b_{k-1}\), tutte le parentesi sono strettamente positive.
Se buttiamo via tali termini ci rimane un valore maggiore di \(b_n - a_1\).
\begin{align*}
    \sum_{k=1}^n l(I_k) > b_n - a_1 > b-a
\end{align*}
Abbiamo quindi trovato che \(\mu^*([a,b]) = b-a\).
Possiamo trovare la misura dell'intervallo aperto facendo
\begin{align*}
    b-a = \mu^*([a,b]) &= \mu^*\left((a,b) \cup \{a\} \cup \{b\}\right) \\
    &\leq \mu^*\left((a,b)\right) + \mu^*(\{a\}) + \mu^*(\{b\}) \\
    &= \mu^*\left((a,b)\right) \leq b-a
\end{align*}

\sdefinition{Misurabile secondo Lebesgue}{
    Un insieme \(E \subseteq \mathbb{R}\) è \emph{misurabile secondo Lebesgue}
    se \(\forall A \subseteq \mathbb{R}\),
    \[
        \mu^*(A) = \mu^*(A \cap E) + \mu^*(A \cap E^c)
    \]
}

Questa definizione è data dal fatto che vogliamo che la misura si scomponga in due parti disgiunte
per tutti gli \(A\), quella che si sovrappone con \(E\) e quella che non si sovrappone con \(E\).

\stheorem{}{
    Gli insiemi misurabili secondo Lebesgue sono una \(\sigma\)-algebra.
}

\sproof{}{
    Sia \(\mathcal{M}\) tale insieme.
    \begin{enumerate}
        \item Notiamo un paio di cose. Se \(\mu^*(E) = 0\), allora \(E \in \mathcal{M}\).
        Questo è dato dal fatto che
        \begin{align*}
            0 + \mu^*(A \cap E^c) &= \mu^*(A \cap E) + \mu^*(A \cap E^c) \\
            &\leq \mu^*(A)
        \end{align*}
        Quindi anche tutti gli insiemi misurabili hanno misura zero.
        \item Abbiamo anche che se \(E \in \mathcal{M}\) allora \(E^C \in \mathcal{M}\).
        Questo è dato dalla definizione simmetrica di misura di Lebesgue.
        \item Mostriamo che se \(E_1, E_2 \in \mathcal{M}\), allora \(E_1 \cup E_2 \in \mathcal{M}\).
        Per fare ciò mostriamo \(E_1 \cap E_2 \in \mathcal{M}\), e poi usiamo il complementare due volte
        per tornare al primo caso. Siccome \(E_2\) è misurabile possiamo scomporre
        \begin{align*}
            \mu^*(A) &= \mu^*(A \cap E_1) + \mu^*(A \cap E_1^c) \\
            &= \mu^*(A \cap E_1 \cap E_2) + \mu^*(A \cap E_1 \cap E_2^c) + \mu^*(A \cap E_1^c) \\
            &\geq \mu^*(A \cap (E_1 \cap E_2)) + \mu^*((A \cap E_1 \cap E_2^c) \cup A \cap E_1^c) \\
            &= \mu^*(A \cap (E_1 \cap E_2)) + \mu^*(A \cap (E_1 \cap E_2^c))
        \end{align*}
        il terzo passaggio usa la subadditività per maggiorare. Chiaramente se ciò vale per due insiemi,
        banalmente vale per \(n\) insiemi \(E_1, E_2, \cdots, E_n \in \mathcal{M}\),
        e quindi \(\bigcup_i E_i \in \mathcal{M}\).
        Se quindi \(E_1, E_2, \cdots, E_n\) sono misurabili e sono disgiunti, allora
        % TODO: \subseteq
        \(\forall A \subseteq \mathbb{R}\),
        \begin{align*}
            \mu^*\left(
                A \cap \left(
                    \bigcup_{k=1}^n E_k
                \right)
            \right)
            = \sum_{k=1}^n \mu^*(A \cap E_k)
        \end{align*}
        Per esempio, per \(A= \mathbb{R}\)
        \[
            \mu^*\left(\bigcup_{k=1}^n E_k\right)
            = \sum_{k=1}^n \mu^*(E_k)
        \]
        Per induzione abbiamo \(n+1 \implies n\)
        \begin{align*}
            \mu^*\left(A \cap \left(\bigcup_{k=1}^n E_k\right)\right)
            &= \mu^*\left(A \cap \left(\bigcup_{k=1}^n E_k\right) \cup E_n\right)
            + \mu^*\left(
                A \cap \left(\bigcup_{k=1}^n E_k\right) \cap E_n^c
            \right) \\
            &= \mu^*(A \cap E_n) + \mu^*\left(A \cap \left(\bigcup_{k=1}^{n-1} E _k\right) \right) \\
            &= \mu^*\left(
                A \cap E_n
            \right)
            + \sum_{k=1}^{n-1} \mu^*(A \cap E_k) \\
            &= \sum_{k=1}^n \mu^*(A \cap E_k)
        \end{align*}
        Mostriamo ora il caso infinito. Sia \(\{E_n\}\) con \(E_n \in \mathcal{M}\),
        allora \(\bigcup I_n \in \mathcal{M}\). Sia
        \begin{align*}
            E &= \bigcup E_n = E_1 \cap (E_2 \backslash E_1) \cup (E_3 \backslash (E_1 \cup E_2)) \cup \cdots \\
            &= G_1 \cup G_2 \cup G_3 \cup \cdots
        \end{align*}
        siccome l'intersezione di insiemi misurabili è misurabile, e i \(G_i\) sono una collezione finita
        di quest'ultimi, allora i \(G_i\) sono misurabili.
        Abbiamo allora \(E = \bigcup G_n\) dove \(G_n \in \mathcal{M}\) sono disgiunti.
        Sia
        \[
            F_n = \bigcup_{k=1}^n G_k
        \]
        Chiaramente \(F_n \subseteq E\) e \(F_n^c \supseteq E^c\).
        Abbiamo allora
        \begin{align*}
            \mu^*(A) &= \mu^*(A \cap F_n) + \mu^*(A \cap F_n^c) \\
            &\geq \mu^*\left(A \cap \left(\bigcup_{k=1}^n G_k\right)\right)
             + \mu^*(A \cap E^c) \\
             &= \sum_{k=1}^n \mu^*\left(A \cap G_k\right)
             + \mu^*(A \cap E^c)
        \end{align*}
        Abbiamo quindi questa maggiorazione per ogni \(n\), quindi vale anche per il limite.
        Il limite delle successioni delle somme parziali è la serie.
        \begin{align*}
            \mu^*(A) &\geq \sum_{k=1}^\infty \mu^*(A \cap G_k) + \mu^*(A \cap E^c) \\
            &\geq \mu^*\left(\bigcup_k^\infty (A \cap G_k)\right)
            + \mu^*(A \cap E^c) \\
            &= \mu^*(A \cap E) + \mu^*(A \cap E^c)
        \end{align*}
        per la subadditività.
    \end{enumerate}
}

La \(\sigma\)-algebra \(\mathcal{M}\) viene detta \emph{\(\sigma\)-algebra di Lebesgue}.

\sdefinition{Misura di Lebesgue}{
    Sia \(E\) misurabile secondo Lebesgue.
    Allora
    \[
        \mu(E) \triangleq \mu^*(E)
    \]
    dove \(\mu^*\) è la misura esterna.
}

Dobbiamo assicurarsi che data una collezione \(\{E_n\}\) misurabili secondo Lebesgue e disgiunti,
\[
    \mu\left(\bigcup E_n\right) = \sum_{n=1}^\infty \mu(E_n)
\]
Sicuramente il primo termine è minore o uguale al secondo.
Per il maggiore o uguale abbiamo
\begin{align*}
    \mu\left(\bigcup^\infty E_n\right) &\geq \mu\left(\bigcup_{k=1}^n E_k\right) \\
    &= \mu^*\left(\bigcup_{k=1}^n E_k\right) \\
    &= \sum_{k=1}^n \mu^*(E_k) = \sum_{k=1}^n \mu(E_k)
\end{align*}
che vale siccome vale la subadditività su insiemi finiti disgiunti.
Sicocme ciò vale per ogni \(n\), allora vale anche il limite
\[
    \mu\left(
        \bigcup_{n=1}^\infty E_n
    \right)
     \geq \sum_{n=1}^\infty \mu(E_n)
\]

La misura esterna è additiva per un numero finiti di insiemi disgiunti, ma non è vero nel caso infinito.
La \(\sigma\)-algebra che abbiamo creato è la più grande che gode delle proprietà della misura che vogliamo.

Abbiamo quindi l'algebra \((\mathbb{R}, \mathcal{M}, \mu)\). Abbiamo pronta la teoria dell'integrazione
per definire l'integrale di Legesbue. Dobbiamo tuttavia capire quali insiemi sono misurabili.

\sproposition{}{
    \((a, +\infty)\) è misurabile.
}

\sproof{}{
    Abbiamo
    \begin{align*}
        \mu^*(A) = \mu^*(A \cap (a, +\infty)) + \mu^*(A \cap (-\infty, a])
    \end{align*}
    e \(A \subseteq \bigcup I_n\)
    Siano \[
        I_n^- = I_n \cap (-\infty, a], \quad I_n^+ = I_n \cap (a, +\infty)
    \]
    ovviamente valgono
    \[
        I_n = I_n^- \cup I_n^+, \quad I_n^- \cap I_n^+ = \emptyset
    \]
    quindi
    \(l(I_n) = l(I_n^-) + l(I_n^+)\). Inoltre
    \[
        A \cap (-\infty, a] \subseteq \bigcup I_n^-, \quad
        A \cap (a, +\infty) \subseteq \bigcup I_n^+
    \]
    E per definizione abbiamo
    \begin{align*}
        \mu^*(A \cap (a, +\infty)) + \mu^*(A \cap (-\infty, a])
        &\leq \sum_n l(I_n^+) + \sum_n l(I_n^-) \\
        &= \sum_n l(I_n) \leq \mu^*(A) + \varepsilon
    \end{align*}
}

Quindi tutti anche gli intervalli sono misurabili.
Anche \([a, b)\) è misurabile in quanto

\[
    [a,b) = \bigcap_{n=1}^\infty \left(a-\frac{1}{n}, \infty\right) \cap (-\infty, b)
\]
e \((-\infty, b)\) è misurabile in quanto è il complemento di
\[
    [b, +\infty) = \bigcap (b-\frac{1}{n}, +\infty)
\]
In generale \((a,b) \in \mathcal{M}\).
Se \(A\) è aperto allora è misurabile.
\(\mathbb{R}\) con la misura di Lebesgue è uno spazio di misura completo.

\sproposition{}{
    Sia \(A \subseteq \mathbb{R}\) aperto. Allora
    \(A\) è unione numerabile di intervalli disgiunti.
}

Quindi sono misurabili (non serve nemmeno che siano disgiunti).

\sproof{}{
    Sia \(x\in A\) e consideriamo
    \[
        I_x = \left\{\bigcup I \,|\, x\in I\right\} \subseteq A
    \]
    chiaramente \(I_x\) è un intervallo, il più grande intervallo contenente \(x\).
    Se \(I_x = A\), allora abbiamo finito altrimenti \(I_x \subset A\)
    e consideriamo dunque \(y\in A \backslash I_x\) e \(I_y\).
    Chiaramente \(I_x \cap I_y = \emptyset\). Adesso abbiamo altri due casi,
    o \(I_x \cup I_y = A\), e allora abbiamo scritto l'aperto come unione di intervalli disgiunti,
    oppure c'è \(z \in A \backslash (I_y \cup I_x)\). Consideriando \(I_z\) possiamo fare
    lo stesso ragionamento. Possiamo andare avanti finché non ho consumato tutti i punti di \(A\).
    Dobbiamo tuttavia mostrare che \(A = \bigcup I_{x_i}\) è unione numerabile.
    Per fare ciò consideriamo tutti i razionali \(\{r_n\}\) che stanno in \(A\).
    Ognuno dei \(I_{x_i}\) deve contenere almeno un razionale, ma siccome i razionali sono numerabili,
    ci sarebbero intervalli \(I_{x_i}\) senza razionali, che è impossibile.
}

\saxiom{Assioma della scelta}{
    Sia \(\mathcal{F}\) una collezione di sottoinsieme di \(X\)
    esiste una funzione di scelta \(\varphi\colon \mathcal{F} \to X\)
    tale che \(\forall G \in \mathcal{F}, \varphi(G) \in G\).
}

Vediamo ora un insieme che non è misurabile usando l'assioma della scelta.
In \(\mathbb{R}\) con la misura di Lebesgue, sia \(X = [0, 1)\)
e definiamo \[
    x \overset{\circ}{+} y = \begin{cases}
        x + y & x + y < 1 \\
        x + y - 1 & x + y \geq 1
    \end{cases}
\]
per \(x,y\in X\). Usiamo la relazione di equivalenza \(x \sim y \iff x-y\in\mathbb{Q}\).
Indichiamo con \(P\) tutti gli elementi che estraiamo con la funzione della scelta dalle classi di equivalenza,
cioè i rappresentanti delle varie classi.
Consideriamo ora i razionali \(\{r_n\}\) di \([0, 1)\) e sia
\[
    P_n \triangleq P \overset{\circ}{+} r_n
\]
Abbiamo alcune proprietà:
\begin{enumerate}
    \item \(n \neq m \implies P_n \cap P_m = \emptyset\).
    Se \(z \in P_n \cap P_m\), allora \(z = p \overset{\circ}{+} r_n = q \overset{\circ}{+} r_m\).
    Quindi \(p-q = r_m - r_n\), ma quindi \(p-q\) è razionale, e quindi sono nella stessa classe di equivalenza,
    contro l'ipotesi che sono di classi distinte.
    \item \[
        \bigcup P_n = [0,1)
    \]
    Chiaramente \(\bigcup P_n \subseteq [0,1)\). Sia ora \(x\in [0,1)\) e mostriamo che appartiene
    ad un certo \(P_n\). Ovviamente \(x\in {[x]}_\sim = {[p]}_\sim\), quindi \(p-x\in\mathbb{Q}\).
    \begin{itemize}
        \item se \(x > p\) allora \(x-p = r_{\overline{n}} \in [0,1)\), quindi
        \(x = p + r_{\overline{n}}\) o in altre parole \(x \in ü_{r_{\overline{n}}}\)
        \item se \(x < p\) allora \(x-p+1 \in \mathbb{Q} \cap [0,1)\) e
        \(x-p+1 = r_{\hat{n}} \in [0,1) = x = p + r_{\hat{n}} - 1 = p \overset{\circ}{+} r_{\hat{n}}\)
    \end{itemize}
\end{enumerate}
Supponiamo ora che \(P\) sia misurabile, e quindi \(P_n\) è misurabile.
Allora \(\mu(P) = \mu(P_n)\)
\begin{align*}
    1 = \mu([0,1)) &= \mu\left(\bigcup P_n\right) = \sum_{n=1}^\infty \mu(P_n) \\
    &= \sum_{n=1}^\infty \mu(P)
\end{align*}
quindi la serie di termini costanti o è zero, o diverge, il che è assurdo.
Quindi l'insieme non è misurabile.

Ricordiamo che una funzione è misurabile quando \(\{f < \alpha\} \in \mathcal{M}\).
La funzione \(1_{\mathbb{Q}}\) è misurabile e
\[
    \int_{\mathbb{R}} 1_{\mathbb{Q}}\,d\mu = \mu(\mathbb{Q}) = 0
\]

\stheorem{}{
    Sia \(f\colon [a,b] \to \mathbb{R}\) Riemann-integrabile. Allora
    \begin{enumerate}
        \item \(f\) è misurabile secondo Lebesgue;
        \item \(f\) è Lebesgue-integrabile;
        \item \[
            \integral[a][b][f(x)][x] = \int_{[a,b]} f\,d\mu
        \]
    \end{enumerate}
}

\sproof{}{
    Sia \(f\) misurabile. Vogliamo vedere se
    \[
        \int_{[a,b]} |f|\,d\mu < \infty
    \]
    Ma
    \begin{align*}
        \int_{[a,b]} |f|\,d\mu &\leq \int_{[a,b]} M\,d\mu = M(b-a)
    \end{align*}
    Siccome \(f\) è R-integrabile, sappiamo che \(\forall \varepsilon > 0\),
    esiste una partizione \(P\) di \([a,b]\) tale che
    \[
        |S(f, P) - s(f, P)| < \varepsilon
    \]
    TODO: usare i simboli di integrale superiore e inferiore.
    Ricordiamo che
    \[
        S(f, P) = \sum_{i=1}^n M_i(x_i - x_{i-1})
    \]
    e
    \[
        s(f, P) = \sum_{i=1}^n m_i(x_i - x_{i-1})
    \]
    Prendiamo
    \[
        \varphi_1 = \sum_{i=1}^n m_i 1_{(x_{i-1}, x_i)}
    \]
    e
    \[
        \varphi_2 = \sum_{i=1}^n M_i 1_{(x_{i-1}, x_i)}
    \]
    Una prende l'inf e l'altra prende il sup, quindi \(\varphi_1 \leq f \leq \varphi_2\).
    Allora
    \[
        S(f, P) = \int_{[a,b]} \varphi_2\,d\mu,\quad 
        s(f, P) = \int_{[a,b]} \varphi_1\,d\mu
    \]
    Quindi possiamo rimpiazzare la condizione con gli integrali di Lebesgue
    \[
        \left|
            \int_{[a,b]} \varphi_2\,d\mu -
            \int_{[a,b]} \varphi_1\,d\mu
        \right| < \varepsilon
    \]
    Al posto di \(\varepsilon\) prendiamo \(1/n\). Per ogni \(n\) ci saranno le funzioni semplici
    \(\varphi_1^n \leq f \leq \varphi_2^n\). Possiamo anche fare in modo che
    \(\varphi_1^n \leq \varphi_1^{n+1} \leq f \leq \varphi_2^{n+1} \leq \varphi_2^n\).
    Chiaramente \(\{\varphi_1^n\}\) e \(\{\varphi_2^n\}\) sono monotone e quindi convergono
    a \(\overline{\varphi}_1\) e \(\overline{\varphi}_2\), quindi \(\overline{\varphi}_1 \leq f \leq \overline{\varphi}_2\).
    Vale sempre che \(|\varphi_1^n|, |\varphi_2^n| \leq M\) sono limitate da qualche costante,
    ma allora possiamo applicare il teorema della convergenza dominante
    \begin{align*}
        \int_{[a,b]} (\overline{\varphi}_2-\overline{\varphi}_1)\,d\mu = 0
    \end{align*}
    ma quindi siccome l'integranda \(\overline{\varphi}_2-\overline{\varphi}_1\) è non negativa,
    allora deve essere quasi ovunque uguale a zero, oppure che le due sono uguali quasi ovunque,
    e siccome \(\overline{\varphi}_1 \leq f \leq \overline{\varphi}_2\), allora
    \(\overline{\varphi}_2=f=\overline{\varphi}_1\) quasi ovunque.
    Allora, siccome \(\mathcal{M}\) è completa, \(f\) è misurabile.
    Il terzo punto è immediato in quanto l'integrale rimane monotono e quindi
    \begin{align*}
        \int_{[a,b]} \overline{\varphi}_1 \,d\mu
        \leq
        \int_{[a,b]} f \,d\mu
        \int_{[a,b]} \overline{\varphi}_2 \,d\mu
    \end{align*}
    ma il primo è uguale all'ultimo.
    Per definizione di integrale di Riemann,
    \[
        \int_{[a,b]} \overline{\varphi}_2 \,d\mu
    \]
    è l'integrale di Riemann di \(f\), e quindi
    \[
        \integral[a][b][f(x)][x] = \int_{[a,b]}f\,d\mu
    \]
}

\stheorem{}{
    Sia \(f\colon \mathbb{R} \to [0, +\infty)\) misurabile tale che
    \(f\) sia R-integrabile in \([a,c]\) per \(c>a\).
    Allora,
    \[
        \lim_{c\to\infty} \integral[a][c][f(x)][c]
        = \int_{[a, +\infty)} f\,d\mu
    \]
}

L'ipotesi garantisce che l'integrale esiste per ogni \(c\).
Siccome la funzione è positiva, l'integrale è monotono crescente (potrebbe essere \(+\infty\)).
Quindi, nel caso dell'integrale di Lebesgue non è necessario usare il limite per estendere
l'integrale nell'intervallo illimitato, a differenza dell'integrale di Riemann.

\sproof{}{
    Consideriamo una generica successione \(c_n \to \infty\) e consideriamo
    \[
        f_n(x) = f(x) 1_{[a, c_n]}
    \]
    Chiaramente \(0 \leq f_n \leq f_{n+1}\) è monotona crescente.
    Inoltre, \(f_n \to f\) in \([a, +\infty)\).
    Usiamo il teorema della convergenza monotona che si dice
    \[
        \lim \int_X f_n\,d\mu = \int_X f\,d\mu
    \]
    Quindi
    \begin{align*}
        \lim_n \int_{\mathbb{R}} f_n\,d\mu
        &= \lim_n \int_{\mathbb{R}} f1_{[a, c_n]}\,d\mu \\ 
        &= \lim \int_{[a, c_n]} f\,d\mu \\
        &= \lim \integral[a][c_n][f][\mu] \\
        &= \lim \integral[a][c_n][f(x)][x] \\
        &= \lim \int_{\mathbb{R}} f1_{[a, +\infty)}\,d\mu \\
        &= \int_{[a, +\infty)} \,d\mu
    \end{align*}
}

\pagebreak

\sexample{}{
    La funzione \[\frac{\cos\pi x}{x} \notin L^1((1, +\infty))\]
    Una funzione è integrabile secondo Lebesgue se e solo se lo è il suo modulo.
    Possiamo anche utilizzare il fatto che
    \[
        \int_{\bigcup E_n} f\,d\mu = \sum_n \int_{E_n} f\,d\mu
    \]
    per insiemi \(E_n\) disgiunti se \(f\) è positiva, come in questo caso.
    Consideriamo quindi
    \[
        [1, +\infty) = \bigcup_{k=1}^\infty [k, k+1)
    \]
    che sono disgiunti. E quindi
    \begin{align*}
        \int_{[1, +\infty)} \frac{|\cos\pi x|}{x}\,d\mu
        &= \sum_{k=1}^\infty \int_{[k,k+1)} \frac{|\cos\pi x|}{x}\,d\mu \\
        &= \sum_{k=1}^\infty \integral[k][k+1][\frac{|\cos\pi x|}{x}][x] \\
        &\geq \sum_{k=1}^\infty \integral[k][k+1][\frac{|\cos\pi x|}{k+1}][x] \\
        &= \sum_{k=1}^\infty \frac{1}{k+1} \integral[0][1][|\cos\pi x|][x] = +\infty
    \end{align*}
    che diverge.
    Questa funzione non è integrabile secondo Lebesgue ma lo è secondo Riemann.
}

\sexample{}{
    Studiare, al variare di \(\alpha\), quando
    \[
        f(x) = \frac{x^\alpha \sin\pi x}{(\ln x) \ln(1 + \sqrt{x})}
        \in L^1((0, +\infty))
    \]
    Abbiamo dei problemi in \(x=0, 1, +\infty\).
    In un intorno di zero abbiamo
    \[
        f \sim \frac{C}{x^{-\alpha - \frac{1}{2}} \ln x}
    \]
    quindi è integrabile per \(\alpha> - 3/2\).
    In un intorno di \(1\) abbiamo
    \[
        f \sim C\frac{\sim \pi x}{\ln x}
    \]
    che è ha limite
    \[
        \lim_{x\to 1} C x \cos(\pi x) = 0
    \]
    quindi la nostra funzione è sempre integrabile in un intorno di \(1\).
    In un intorno di infinito abbiamo la maggiorazione
    \[
        |f| \leq \frac{Cx^\alpha}{\ln^2(x)}
    \]
    siccome per \(x\) grande togliamo il \(+1\).
    Quindi la funzione è del tipo 
    \[
        \frac{C}{x^{-\alpha} \ln^2 x}
    \]
    che è integrabile per \(\alpha \leq -1\).
    Quindi la funzione è integrabile per \(-3/2<\alpha \leq -1\).
    Dobbiamo tuttavia capire che cosa succede se \(\alpha \leq -1\), siccome
    abbiamo usato una maggiorazione.
    Dividiamo l'integrale in diversi integrali secondo il periodo
    \begin{align*}
        \integral[a=2][+\infty][\frac{x^\alpha |\sin\pi x|}{(\ln x) \ln(1 + \sqrt{x})}][x]
        &= \sum_{k=2}^\infty \integral[k][k+1][\frac{x^\alpha |\sin\pi x|}{(\ln x) \ln(1 + \sqrt{x})}][x] \\
        &\geq \sum_{k=1}^\infty \frac{1}{{(k+1)}^{-\alpha}\ln(k+1)\ln(1 + \sqrt{k+1})}
        \integral[0][1][|\sin\pi x|][x] 
    \end{align*}
    dove \(a=2\) è quasi arbitrario.
    Della parte periodica sappiamo che l'integrale è costante, del resto della funzione
    abbiamo preso il minimo. Cominciamo guardando \(-1 \leq \alpha \leq 0\).
    Il termine ora ha forma
    \[
        \frac{1}{k^{-\alpha}\ln^2k}
    \]
    allora diverge per \(a > -1\).
    In conclusione, la funzione è integrabile se e solo se
    \[
        -\frac{3}{2} < \alpha \leq -1
    \]
}

Quindi per \(\alpha \leq -1\) abbiamo fatto una maggiorazione,
mentre per il resto abbiamo fatto una minorazione.

\sexample{}{
    Studiare quando
    \[
        f(x) = \frac{\sin^2x^2}{x^\alpha}
        \in L^1((0, +\infty))
    \]
    al variare di \(\alpha\).
    Abbiamo problemi in \(x=0, +\infty\).
    In un intorno di \(0\) abbiamo
    \[
        f\sim \frac{1}{x^{\alpha - 4}}
    \]
    quindi è integrabile per \(\alpha < 5\).
    In un intorno di infinito notiamo che
    \[
        f \leq \frac{1}{x^\alpha}
    \]
    quindi è integrabile se \(\alpha > 1\). Dobbiamo tuttavia studiare
    il caso \(\alpha \leq 1\). Dobbiamo cambiare la variabile in maniera tale da fare diventare
    la funzione periodica \(x^2 = t\). Allora,
    \begin{align*}
        \frac{1}{2} \integral[1][+\infty][\frac{\sin^2t}{t^{\frac{\alpha + 1}{2}}}][t]
        &\geq \frac{1}{2} \sum_{k=1}^\infty \integral[k\pi][(k+1)\pi][
            \frac{\sin^2 t}{t^{\frac{\alpha + 1}{2}}}][t] \\
        &\geq \frac{1}{2} \sum_{k=1}^\infty \frac{1}{{[(k+1)\pi]}^{\frac{\alpha + 1}{2}}}
        \integral[0][\pi][\sin^2t][t]
    \end{align*}
    che non è integrabile \(\frac{\alpha + 2}{2} \leq 1 \iff \alpha \leq 1\).
    Quindi, \(f\in L^1((0, +\infty))\) se e solo se \(1 < \alpha < 5\).
}

\pagebreak

\sexample{}{
    Studiare quando
    \[
        f(x) = \frac{x^\alpha}{(1 + x^2) \sqrt[3]{\sin x}}
        \in L^1((0, +\infty))
    \]
    al variare di \(\alpha\).
    Abbiamo problemi ad infinito ed sicuramente illimitata in quanto il seno si annulla
    periodicamente.
    Tuttavia, i punti critici periodici dipendono solo da
    \[
        \frac{1}{\sqrt[3]{x}}
    \]
    In un intorno di zero abbiamo
    \[
        f \sim \frac{1}{x^{\frac{1}{3} - \alpha}}
    \]
    che è integrabile per \(\alpha > - 2/3\).
    Guardiamo ora cosa succede in un intorno di \(k\pi\)
    \[
        \frac{1}{{|x-k\pi|}^\alpha} \sim f \frac{1}{\sqrt[3]{\sin x}}
    \]
    Dobbiamo fare uno sviluppo per studiare il seno negli intorni di \(k\pi\).
    \[
        \sin(x) = 0 \pm (x-k\pi) + o((x-k\pi))
    \]
    quindi si comporta come
    \[
        \frac{1}{{|x-k\pi|}^{1/3}}
    \]
    che è integrabile. Quindi, non ci sono problemi di integrabilità in tali punti per quel pezzo della funzione.
    In un intorno di infinito
    \begin{align*}
        \integral[0][\infty][|f|][\mu]
        &\geq \integral[\pi][\infty][|f|][\mu] \\
        &= \sum_{k=1}^\infty \integral[k\pi][(k+1)\pi][\frac{x^\alpha}{(1 + x^2) \sqrt[3]{\sin x}}][x] \\
        &\geq \sum_{k=1}^\infty \frac{{(k\pi)}^\alpha}{1 + {(k\pi)}^2}
        \integral[0][\pi][\frac{1}{\sqrt[3]{\sin x}}][x]
    \end{align*}
    quindi il carattere è lo stesso di
    \[
        \frac{1}{k^{2-\alpha}}
    \]
    che diverge per \(\alpha \geq 1\).
    Analogamente minoriamo
    \begin{align*}
        \leq \sum_{k=1}^\infty \frac{{((k+1) \pi)}^\alpha}{1 + \pi^2 {(k + 1)}^2}
        \integral[0][\pi][\frac{1}{\sqrt[3]{\sin x}}][x]
    \end{align*}
    che converge per \(\alpha < 1\).
    Quindi, \(f\in L^1 \iff -2/3 < \alpha < 1\).
}

\pagebreak

\sexercise{}{
    Calcolare
    \[
        \lim_n \integral[0][\infty][\frac{e^x + x^n}{1 + x^ne^{2x}}][x]
    \]
    Calcoliamo il limite (per \(x > 0\)) delle funzioni che stiamo integrando
    \begin{align*}
        \lim_n \frac{e^x + x^n}{1 + x^ne^{2x}}
        &= \begin{cases}
            e^x 0 < x < 1 \\
            e^{-2x}
        \end{cases}
    \end{align*}
    il caso \(x=1\) non ci interessa in quanto per ciò che concerne l'integrale un singolo
    punto è irrilevante.
    Vogliamo applicare il teorema di convergenza dominante.
    Vogliamo trovare una maggiorante integrabile \(g\).
    Per \(x\in (0,1)\) possiamo usare
    \[
        \frac{e^x + x^n}{1 + x^ne^{2x}} \leq e^x + 1
    \]
    Invece, per \(x\in (1, +\infty)\)
    \[
        \frac{e^x + x^n}{1 + x^ne^{2x}} \leq 
        \frac{e^x}{x^n e^{2x}} + \frac{x^n}{x^ne^{2x}}
        = \frac{e^-x}{x^n} + e^{-2x}
        \leq e^{-x} + e^{-2x}
    \]
    Quindi,
    \[
        f_n \leq \begin{cases}
            e^x + 1 & x\in(0,1) \\
            e^{-2x} + e^{-x} & x>1
        \end{cases}
    \]
    allora il limite degli integrali è l'integrale del limite
}

\sexercise{}{
    Calcolare
    \[
        \lim_n \integral[0][n][\frac{n^2 e^{-n/t}}{t^2 \sqrt{1 + t^3}}][t]
    \]
    Il problema è l'intervallo di integazione, ma possiamo sistemarlo
    \begin{align*}
        \lim_n \integral[0][n][\frac{n^2 e^{-n/t}}{t^2 \sqrt{1 + t^3}}][t]
        &= \lim_n \integral[0][\infty][\frac{n^2 e^{-n/t}}{t^2 \sqrt{1 + t^3}}1_{[0, n]}][t]
    \end{align*}
    Il limite è dato da
    \begin{align*}
        \lim_n f_n(t) = \begin{cases}
            \to 0
        \end{cases}
    \end{align*}
    Quindi per dimostrare che l'integrare è nullo dobbiamo trovare una maggiorante integrabile.
    Potrei maggiorare con una constante
    \[
        \frac{n^2 e^{-n/t}}{t^2 \sqrt{1 + t^3}} \leq
        \frac{C}{t^2 \sqrt{1 + t^3}}
    \]
    che tuttavia non è integrabile quando \(t\) è piccolo.
    Consideriamo allora il termine
    \[
        \frac{n^2}{t^2}e^{-n/t} = y^2e^{-y}
    \]
    con \(y = n/t\). Di tale funziona, controlliamo se è veramente limitata superiormente, ha un massimo
    \begin{align*}
        2ye^{-y} - y^2 e^{-y} = ye^{-y}(2-y)
    \end{align*}
    Quindi è sempre limitata da \(4e^{-2}\)
    \[
        \left(
            \frac{n^2}{t^2}e^{-n/t}
        \right)
        \cdot \frac{1}{\sqrt{1 + t^3}}
        \leq \frac{4e^{-2}}{\sqrt{1 + t^3}}
    \]
    che è integrabile sia in zero che in infinito.
}

Lo spazio di probabilità è una misura \((\mathbb{R}, \mathcal{B}, \mathbb{P})\)
dove \(\mathcal{B}\) è la misura generata dagli aperti (di Borel), quindi \(\mathcal{B} \subseteq \mathcal{M}\)
in quanto gli aperti sono nelle \(\sigma\)-algebra di Lebesgue.
Si può mostrare che la misura di Lebesgue è strettamente contenuta in \(\mathcal{M}\) ma è più complicato.
La misura ha la proprietà che \(\mathbb{P}(\mathbb{R}) = 1\).
Se prendiamo \(x\in \mathbb{R}\), \(\delta_x(\mathbb{R}) = 1\) (misura di Dirac).

Costruiamo una funzione \(F \colon \mathbb{R} \to [0,1]\) tale che
\[
    F_{\mathbb{P}}(x) = \mathbb{P}((-\infty, x])
\]
Se prendiamo \(\mathbb{P} = \delta_x\), allora
\(F_{\delta_a}\) vale \(1\) per \(x\geq 1\), altrimenti \(0\).

Prendiamo ora \(\mathbb{P} = \mu(A \cap [0,1])\) che viene chiamata la misura di Lebesgue concentrata in \(1\).
Scriviamo
\[
    F_{\mathbb{P}}(x) = \mu((-\infty, x] \cap [0,1])
\]
fino a zero, la funzione è nulla in quanto l'intersezione è vuota.
La funzione è \(1\) per \(x\geq1\) e una retta in \([0,1]\).
Tale funzioni ha delle proprietà molto importanti:
\begin{enumerate}
    \item \(x<y \implies F(x) \leq F(y)\)
    \item \[
        \lim_{x\to-\infty} F(x) = 0,\qquad \lim_{x\to\infty} F(x) = 1
    \]
    Per dimostrare che il limite tende ad \(1\) prendiamo una successione \(x_n \to \infty\) e consideriamo
    gli intervalli \(I_n = (-\infty, x_n]\).
    Abbiamo che \(I_n \subseteq I_{n+1}\) e
    \[
        \bigcup I_n = \mathbb{R}
    \]
    quindi la probabilità (misura) dell'unione è data dal limite
    \[
        \mathbb{P} \left(
            \bigcup I_n
        \right) = \lim_n \mathbb{P}(I_n) = \mathbb{P}(\mathbb{R})
        = \lim_n F(x_n) = 1
    \]
    \item \emph{continua da destra} \[
        \lim_{x\to x_0^+} F(x) = F(x_0)
    \]
\end{enumerate}

Una funzione \(F \colon \mathbb{R} \to [0,]\) che soddisfa \((1), (2), (3)\) è detta
una funzione di distribuzione (anche funzione di ripartizione della probabilità \(\mathbb{P}\)).
Sia dunque \(F\) una tale funzione. Allora esiste una probabilità \(\mathbb{P}\) su \(\mathbb{R}\)
tale che \(F = F_{\mathbb{P}}\), cioè partendo da \(\mathbb{P}\) posso ritrovare la stessa \(F\).
Quindi, avere una probabiltà o una funzione di distribuzione è la stessa cosa.
Per costruire tale probabilità prendiamo intervalli \(I = (a, b]\) e diciamo che la lunghezza
di tale intervallo relativa alla funzione la calcolo così:
\[
    l_F(I) = F(b) - F(a)
\]
e poi costruiamo
\[
    \mu^*_F(E) = \inf\left\{
        \sum_{n=1}^\infty l_F(I_n), E \subseteq \bigcup_{n=1}^\infty I_n
    \right\}
\]
Un sottoinsieme \(E \subseteq \mathbb{R}\) è \(F\)-misurabile
se
\[
    \forall A \subseteq \mathbb{R}, \mu_F^*(A) = \mu_F^*(A \cap E) + \mu^*_F(A \cap E^c)
\]
esattamente come prima.
Se \(E\) è \(F\)-misurabile allora definiamo
\[
    \mu_F(E) = \mu_F^*(E)
\]
che in realtà è una probabilità, la cui funzione di distribuzione è quella da cui siamo partiti.

\section{Misura di Lebesgue su \(\mathbb{R}^n\)}

Consideriamo ora un insieme della forma (che chiamiamo per semplicità rettangolo)
\[
    J = (a_1, b_1) \times (a_2, b_2) \times \cdots \times (a_n, b_n) \subseteq {\mathbb{R}}^n
\]
Definiamo l'area come
\[
    \text{area}_n(J) = \prod_{i=1}^n (b_i-a_i)
\]
Consideriamo quindi \(E \subseteq \mathbb{R}^n\) e definiamo la misura n-dimensionale
\[
    \mu_n^*(E) = \inf\left\{
        \sum_{n=1}^\infty \text{area}_n (J_h), E \subseteq \bigcup_{h=1}^\infty J_h
    \right\}
\]
Analogamente abbiamo le:
\begin{enumerate}
    \item Proprietà di \(\mu_n^*\)
    \item Definizione di insieme misurabile
    \item Gli insiemi misurabili sono una \(\sigma\)-algebra.
    \item Tale \(\sigma\)-algebra contiene gli aperti.
    \item Se \(E \in \mathcal{M}_n\), allora definiamo
    \[
        \mu_n(E) = \mu_n^*(E)
    \]
    che è la misura di Lebesgue in \({\mathbb{R}}^n\)
\end{enumerate}
è la stessa cosa ma siamo partiti dall'area piuttosto che dalla lunghezza.
Quindi, \(({\mathbb{R}}^n, {\mathcal{M}}_n, \mu_n)\) è l'oggetto con cui abbiamo a che fare,
e abbiamo quindi la teoria dell'integrazione.

Una funzione \(f\in L^1({\mathbb{R}}^n)\),
cioè una funzione integrabile in \({\mathbb{R}}^n\).

\pagebreak

Per calcolare un integrale \(n\)-dimensionale, vogliamo ricondurci al caso unidimensionale. Abbiamo allora il
\stheorem{Teorema di Fubini}{
    Sia \(f \colon {\mathbb{R}}^n = {\mathbb{R}}^k \times {\mathbb{R}}^{n-k} \to {\mathbb{R}}\)
    misurabile e integrabile.
    \begin{enumerate}
        \item \(f_x(y) \colon {\mathbb{R}}^{n-k} \to {\mathbb{R}}^n\)
        è integrabile per quasi ogni \(x\in {\mathbb{R}}^k\).
        \item
        \[
            G(x) = \int_{{\mathbb{R}}^{n-k}} f_x(y)\,d\mu_{n-k}(y)
        \]
        è definita quasi ovunque, è integrabile in \({\mathbb{R}}^{k}\).
        \item \begin{align*}
            \int_{{\mathbb{R}}^{k}} G(x)\,d\mu_n(x) &= \int_{{\mathbb{R}}^{n}} f\,d\mu_n \\
            &= \int_{{\mathbb{R}}^{k}} \left(
                \int_{{\mathbb{R}}^{n-k}}
                f(x,y)\,d\mu_{n-k}(y)
            \right)\,d\mu_k(x)
        \end{align*}
    \end{enumerate}
}

bisogna tuttavia capire quando la funzione è integrabile.

\stheorem{Teorema di Tonelli}{
    Sia \(f \colon {\mathbb{R}}^n = {\mathbb{R}}^k \times {\mathbb{R}}^{n-k} \to [0, +\infty)\) misurabile.
    Allora
    \[
        G(x) = \int_{{\mathbb{R}}^{n-k}} f_x(y)\,d\mu_{n-k}(y)
    \]
    (che potrebbe assumere valore infinito)
    è misurabile in \({\mathbb{R}}^k\) e
    \begin{align*}
        \int_{{\mathbb{R}}^{k}} G(x)\d\mu_k(x) = \int_{{\mathbb{R}}^{n}} f\,d\mu_n
    \end{align*}
}
È importante notare che la funzione debba essere positiva.
Esempio con i quadratini \(\pm 1\), gli integrali unidimensionali fanno zero ma l'integrale bidimensionale non è integrabile.
Se prendiamo il valore assoluto, gli integrali unidimensionali fanno \(2\), e l'integrale doppio diverge a \(+\infty\),
che è corretto nel caso di \(|f|\).

%l'integrabilità di
%\[
%    \frac{1}{{x-a}^\alpha}
%\]
%che è integrabile per \(\alpha < 1\).

\pagebreak

\section{Esercizi}

% TODO gli esercizi di giacomo

% 25 gennaio 2018 tema d'esame
\sexercise{Successioni 1}{
    Per \(x > -1\) studia la successione
    \[
        f_n(x) = \frac{ne^{-n/x}}{x^2\sqrt{1 + x}}
    \]
    \begin{itemize}
        \item \textbf{convergenza puntuale:} controlliamo la convergenza puntuale in \(E = (0, +\infty)\).
        Fissato \(x > 0\) studiamo
        \begin{align*}
            \lim_n f_n(x) = \lim_n \frac{ne^{-n/x}}{x^2\sqrt{1 + x}} = 0 = f(x)
        \end{align*}
        \item \textbf{convergenza uniforme:} controlliamo la convergenza uniforme in \(E\).
        Abbiamo
        \begin{align*}
            {||f_n - f||}_{\infty, E} &= \sup_{x\in (0, +\infty)} \left|
                \frac{ne^{-n/x}}{x^2\sqrt{1 + x}}
            \right|
        \end{align*}
        sostituendo \(t = n/x\) otteniamo una funzione simile all'integranda della funzione gamma, che ha un massimo \(M\)
        \begin{align*}
            \sup_{x\in (0, +\infty)} \left|
                \frac{ne^{-n/x}}{x^2\sqrt{1 + x}}
            \right| &\leq M \sup_{x\in (0, +\infty)} \frac{1}{n\sqrt{1 + x}} \\
            &= \frac{M}{n \varepsilon} \to 0
        \end{align*}
        Chiaramente il sup si ottiene con il denominatore più piccolo, quindi un \(\varepsilon\) molto vicino a \(0\).
        \item \textbf{integrabilità:} mostriamo che \(f_n \in L^1\).
        \begin{align*}
            \integral[0][\infty][|f_n(x)|][x] &=
            \integral[0][\infty][\left|\frac{ne^{-n/x}}{x^2\sqrt{1 + x}}\right|][x] 
        \end{align*}
        In un intorno di \(+\infty\) abbiamo
        \[
            f_n(x) \sim \frac{n}{x^{5/2}}
        \]
        siccome \(\frac{5}{2}>1\) la funzione è integrabile a infinito.
        In un intorno di \(0^+\) maggioriamo
        \[
            f_n(x) \leq \frac{M}{n\sqrt{1 + x}} \sim \frac{M}{n}
        \]
        quindi è integrabile per confronto e confronto asintotico.
    \end{itemize}
}

\pagebreak

\sexercise{Successioni 1}{
    Data la successione
    \[
        f_n(x) = n^\alpha \arctan(x)e^{n^2 x}
    \]
    studiare al variare di \(\alpha\in\mathbb{R}\)
    \begin{enumerate}
        \item \textbf{convergenza puntuale:} studiamo la convergenza puntuale in \((0, +\infty)\).
        Fissato \(x>0\) abbiamo
        \begin{align*}
            \lim_n f_n(x) &= \lim_n n^\alpha \arctan(x)e^{n^2 x} = 0, \forall \alpha \in \mathbb{R}
        \end{align*}
        \item \textbf{convergenza uniforme:} studiamo la convergenza uniforme in \((0, +\infty)\).
        Abbiamo
        \begin{align*}
            {||f_n - f||}_{\infty, E} &= \sup_{x\in (0, +\infty)} \left|
                n^\alpha \arctan(x)e^{n^2 x}
            \right| \\
            &\leq \sup_{x\in (0, +\infty)} \left|
                n^\alpha x e^{n^2 x}
            \right|
        \end{align*}
        siccome \(\arctan(x) \leq x\). Studiamo la derivata della funzione maggiorante \(g_n(x)\).
        \begin{align*}
            g_n'(x) &= n^\alpha e^{-n^2x} - n^\alpha x e^{-n^2 x} n^2 \\
            &= n^\alpha e^{-n^2 x}(1-xn^2)
        \end{align*}
        Per studiare il segno abbiamo
        \begin{align*}
            1 - n^2 x \leq 0 \iff x \leq \frac{1}{n^2}
        \end{align*}
        che è un punto di massimo. Chiaramente \(g_n(0)=0\) e \(\lim_{x\to\infty} g_n(x)\),
        e siccome è sempre positiva, siamo sicuri che tale valore è un punto di massimo.
        Il massimo vale
        \[
            g_n\left(\frac{1}{n^2}\right) = \frac{n^{\alpha-2}}{e}
        \]
        Quindi, la norma infinito è sempre minore di 
        \begin{align*}
            ||f_n-f||_{\infty, E} &\leq \frac{n^{\alpha-2}}{e}
        \end{align*}
        che tende a zero solo quando \(\alpha < 2\) (condizione sufficiente ma non necessaria).
    Cerchiamo ora un limite dal basso
    \begin{align*}
        ||f_n-f||_{\infty, E} \geq f_n\left(\frac{1}{n^2}\right)
        &= n^\alpha \arctan\left(\frac{1}{n^2}\right)e^{-1} \\
        &\sim n^{\alpha - 2}e^{-1}
    \end{align*}
    che non tende a zero. Quindi la convergenza è uniforme per \(\alpha < 2\).
    \end{enumerate}
}

\pagebreak

% 2 sett 2020
\sexercise{Successioni 3}{
    Data la successione
    \[
        f_n(x) = n\left(e^{\frac{x^2}{n}} - 1\right)
    \]
    \begin{enumerate}
        \item \textbf{stabilire in che insieme vi è convergenza puntuale:} fissato \(x\) calcoliamo
        \begin{align*}
            \lim_n f_n &= \lim_n n\left(e^{\frac{x^2}{n}} - 1\right) = x^2
        \end{align*}
        in quanto la parentesi è asintotica all'esponente. Allora l'insieme di convergenza puntuale è \(E = \mathbb{R}\).
        \item \textbf{stabilire se la convergenza è uniforme in tale insieme:} fissato \(x\) abbiamo
        \begin{align*}
            {||f_n - f||}_{\infty, E} &= \sup_{x\in (0, +\infty)} \left|
                n\left(e^{\frac{x^2}{n}} - 1\right) - x^2
            \right| = \sup_{x\in (0, +\infty)} \left|
                g_n(x)
            \right|
        \end{align*}
        Studiamo la derivata di \(g_n(x)\)
        \[
            g_n'(x) = ne^{\frac{x^2}{n}}\frac{2x}{n} - 2x = 2x\left(
                e^{\frac{x^2}{n}} - 1
            \right)
        \]
        Il segno della derivata è lo stesso di \(x\), e \(x=0\) è un punto di minimo.
        \[
            \lim_{x\to\pm\infty} g_n(x) = \pm\infty
        \]
        quindi non è limitata e la convergenza non è assoluta.
        \item \textbf{stabilire se la convergenza è uniforme in un intervallo limitato:} sia \([a,b]\) tale intervallo.
        Dalla forma della funzione, il sup è o in \(x=a\) o in \(x=b\), quindi
        \begin{align*}
            ||f_n - f||_{\infty, E} \leq \max\{|f(a)-f(b)|, |f(b)-f(a)|\}
        \end{align*}
        supponiamo che sia in \(a\)
        \begin{align*}
            ||f_n - f||_{\infty, E} = \left|n\left(e^{\frac{a^2}{n}} - 1\right) - a^2\right|
        \end{align*}
        per risolvere il limite facciamo un espansione di MacLaurin fino al secondo ordine
        \begin{align*}
            ||f_n - f||_{\infty, E} = \left|
                \frac{a^4}{n} + o\left(\frac{1}{n}\right)
            \right| \to 0
        \end{align*}
        Quindi, la convergenza è uniforme in un intervallo limitato.
    \end{enumerate}
}

\pagebreak

\sexercise{Serie 1}{
    Consideriamo la serie
    \[
        \sum_{n=1}^\infty \frac{xe^{-\frac{x^2}{n}}}{n^2 + x^2}
    \]
    verifica la convergenza uniforme in \(\mathbb{R}\). Cominciamo studiando la convergenza totale
    che è più forte, quindi la convergenza di
    \begin{align*}
        \sum_{n=1}^\infty ||f_n||_{\infty, \mathbb{R}}
        &= \sum_{n=1}^\infty \sup_{x\in\mathbb{R}}
        \left|
            \frac{xe^{-\frac{x^2}{n}}}{n^2 + x^2}
        \right|
    \end{align*}
    con \(t = \frac{x}{\sqrt{n}}\) otteniamo la forma \(f(t)=te^{-t^2}\) che ha grafico noto,
    e un massimo \(M\) e minimo
    \begin{align*}
         ||f_n||_{\infty, \mathbb{R}}
        &\leq M \sup_{x\in\mathbb{R}}
        \frac{\sqrt{n}}{n^2 + x^2}
        = \frac{M}{n^{3/2}}
    \end{align*}
    in quanto il supremum è per \(x=0\). Quindi la serie
    \begin{align*}
        \sum_{n=1}^\infty ||f_n||_{\infty, \mathbb{R}}
        &\leq M \sum_{n=1}^\infty \frac{1}{n^{3/2}}
    \end{align*}
    che converge. Quindi la serie converge totalmente e quindi converge anche in maniera
    uniformemente su tutto \(\mathbb{R}\).
}

\sexercise{Serie 2}{
    Conderiamo la serie
    \[
        \sum_{n=1}^\infty \arctan\left(\frac{n^\alpha}{x^2 + n^2}\right)
    \]
    \begin{enumerate}
        \item Stabilire per quali \(\alpha \in \mathbb{R}\) abbiamo convergenza puntuale:
            Fissato \(x\), notiamo che per convergere il termine \(n\)-esimo deve tendere a zero.
            Quindi,
            \[
                \lim_n \arctan\left(\frac{n^\alpha}{x^2 + n^2}\right) \to 0
            \]
            se e solo se \(\alpha < 2\) (condizione necessaria).
            Usiamo il criterio del confronto asintotico: l'argomento dell'arcotangente tende a zero
            e quindi è asintotica al suo argomento.
            La serie
            \[
                \sum \frac{1}{n^{2 - \alpha}}
            \]
            converge se e solo se \(\alpha < 1\).
            Quindi, la serie converge puntualmente per \(\alpha < 1\).
        \item Stabilire per quali \(\alpha \in \mathbb{R}\) abbiamo convergenza uniforme:
        è necessario \(\alpha - 1\). Cominciamo studiando la convergenza totale, che è più forte.
        Abbiamo
        \begin{align*}
            \sum_{n=1}^\infty ||f_n||_{\infty, \mathbb{R}}
            &=\sum_{n=1}^\infty \sup_{x\in\mathbb{R}} \left|
                \arctan\left(\frac{n^\alpha}{x^2 + n^2}\right)
            \right|
        \end{align*}
        Studiamo la derivata del termine
        \begin{align*}
            f_n'(x) = \frac{1}{1 + {\left(\frac{n^\alpha}{x^2 + n^2}\right)}^2}
            &= - \frac{2xn^\alpha}{n^{2\alpha} + {n^2 + x^2}^2} \geq 0 \iff x \leq 0
        \end{align*}
        quindi \(x=0\) è un punto di massimo.
        Infatti, gli estremi sono
        \[
            \lim_{x\to\pm\infty} f_n(x) \to 0
        \]
        Quindi la forma è data da
        \[
            ||f_n||_{\infty, \mathbb{R}} = f_n(0) = \arctan(n^{\alpha - 2})
        \]
        La serie è quindi a termini positivi e usiamo il confronto asintotico
        \begin{align*}
            \sum_{n=1}^\infty ||f_n||_{\infty, \mathbb{R}} &=
            \sum_{n=1}^\infty \arctan(n^{\alpha - 2})
        \end{align*}
        che converge se e solo se \(\alpha < 1\). Quindi abbiamo convergenza uniforme
        per \(\alpha < 1\). Siccome la convergenza puntuale è per \(\alpha < 1\),
        non vi sono altri \(\alpha\) per cui vi è convergenza assoluta.
    \end{enumerate}
}

% tema esame 1 settembre 2022
\sexercise{Serie 3}{
    Consideriamo la serie
    \[
        \sum_{n=1}^\infty \frac{\arctan\left(\frac{x}{n^\alpha + 1}\right)}{\sqrt{n+1} - \sqrt{n}}
    \]
    \begin{enumerate}
        \item Valutare per quali \(\alpha\in\mathbb{R}\) vi è convergenza puntuale in \(\mathbb{R}\).
        Studiamo la condizione necessaria di convergenza.
        Il numeratore tende a zero se e solo se \(\alpha > 0\).
        In tal caso la funzione è assolutamente asintotica a
        \[
            |f_n(x)| \sim \frac{2|x|}{n^{\alpha - 1/2}}
        \]
        E la serie
        \[
            \sum_{n=1}^\infty \frac{2|x|}{n^{\alpha - 1 /2}}
        \]
        converge se e solo se \(\alpha > 3/2\) per confronto asintotico.
        Per \(x<0\), basta notare che \(\arctan(t)\) è simmetrica rispetto all'origine,
        il ché implica convergenza puntiale per \(\alpha > 3/2\).
        \item Stabilire per quali \(\alpha\in\mathbb{R}\) la somma della serie è continua in \(\mathbb{R}\).
        Dobbiamo utilizzare il teorema. Dobbiamo trovare per quali \(\alpha\) converge totale
        per applicare il teorema che dice che se \(f_n\) è continua in \(E\), allora la sua serie converge
        uniformemente a \(S(x)\) in \(E\) e \(S(x)\) è continua.
        Studiamo la convergenza totale in \(E = [-a, a]\), \(a>0\).
        Abbiamo allora
        \begin{align*}
            ||f_n||_{\infty, [-a, a]} &= \sup_{x\in[-a,a]}
            \frac{\arctan\left(\frac{x}{n^\alpha + 1}\right)}{\sqrt{n+1} - \sqrt{n}} \\
            &\leq \sup_{x\in[-a,a]} \frac{|x|}{n^\alpha + 1}2\sqrt{n}
            = \frac{2a\sqrt{n}}{n^{\alpha} + 1} \sim \frac{C}{n^{\alpha - 1/2}}
        \end{align*}
        in quanto \(\arctan(t) \leq t\).
        Ciò implica che
        \[
            \sum_{n=1}^\infty ||f_n||_{\infty, [-a, a]} \leq
            \sum_{n=1}^\infty \frac{C}{n^{\alpha - 1/2}}
        \]
        che converge se e solo se \(\alpha > 3/2\).
        Quindi per confronto la serie converge totalmente, e quindi uniformemente per \(\alpha > 3/2\).
        Poiché la convergenza uniforme implica la convergenza puntuale e la serie converge puntualmente
        per \(\alpha > 3/2\), abbiamo convergenza uniformemente se e solo se \(\alpha > 3/2\).
        Poiché \(f_n\) è continua e la serie converge uniformemente in \([-a, a]\), la serie risulta continua in tale intervallo.
        Poiché \(a\) è arbitrario, possiamo amplificato e la serie è continua su tutto \(\mathbb{R}\).
    \end{enumerate}
}

\end{document}
\documentclass[a4paper]{article}

\usepackage{amsmath}
\usepackage{amssymb}
\usepackage{stellar}
\usepackage{parskip}
\usepackage{fullpage}
\usepackage{wrapfig}
\usepackage{tikz}

\usetikzlibrary{arrows}
\usetikzlibrary{decorations.pathreplacing}
\usetikzlibrary{cd}

\title{Algebra I}
\author{Paolo Bettelini}
\date{}

\begin{document}

\maketitle
\tableofcontents

\section{Funzioni}

Una funzione \(\phi\colon A \to B\) dove \(A\) è il dominio mentre \(B\) è il codominio,
preso un elemento \(a\in A\), la sua immagine viene denotata \(\phi(a)\) oppure \(af\).

Se \(C \subseteq A\), la sua immagine tramite \(\phi\) è indicata come \(C\phi\)
che è un sottoinsieme di \(B\).
\[
    C\phi = \{  c\phi \,|\, c\in C \}
\]

Se \(D\) è un sottoinsieme di \(B\), la sua immagine inversa tramite \(\phi\)
è il sottoinsieme \(D\phi^{-1}\) di \(A\) degli elementi la cui immagine appartiene a \(D\).
\[
    D\phi^{-1} = \{  a \in A \,|\, a\phi \in D \}
\]

\sexample{Funzione}{
    Sia \(\phi\colon \mathbb{R} \to \mathbb{R}\) definita ponendo \(\phi x \triangleq x^2\).
}

Consideriamo ora \(A=\{-1,0,1,2\}\). Abbiamo allora \(A\phi = \{1,0,4\}\).
Consideriamo poi \(B=\{-1, 0, 2, 9\}\). Abbiamo allora \(B\phi^{-1} = \{0, \sqrt{2}, 3, -3\}\).

L'immagine di una funzione è chiaramente l'immagine per il suo dominio come insieme considerato.

\subsection{Proprietà}

\sproposition{}{
    Se \(C \subseteq D \subseteq A\), abbiamo \(C\phi \subseteq D\phi\).
}

\sproof{}{
    Abbiamo che
    \[
        C\phi = \{c\phi \,|\, c\in C\}
    \]
    Dunque \(x\in C\phi\) se e solo se esiste \(c\in C\) tale che \(x=c\phi\).
    Ma \(C \subseteq D\), dunque \(c \in D\). Quindi, \(x = c\phi \in D\phi\).
}

Non è detto che se \(C \subset D\) allora \(C\phi \subset D\phi\).
Mostriamo un esempio in cui \(C \subset D\) ma \(C\phi = D\phi\).
Prendiamo \(C=\{1\} \subset D=\{1,-1\}\). Se prendiamo la funzione del quadrato, in ambo caso
trovo la stessa immagine per via di ambo gli insiemi.

Ciò non avviene nel caso in cui la funzione fosse iniettiva.

\sproposition{}{
    Se \(E\subseteq F \subseteq B\), abbiamo che \(E\phi^{-1} \subseteq F\phi^{-1}\).
}

TODO: esercizio proof.

Anche qui la medesima proposizione ma con l'inclusione stretta non è assicurata.

\sproposition{}{
    Se \(C\subseteq A\), allora \(C\phi \phi^{-1} \supseteq C\).
}

\sproof{}{
    Sia \(x \in C\). Bisogna mostrare \(x\in C\phi\phi^{-1}\).
    Ricordiamo che \(D\phi^{-1} = \{ y \in A \,|\, y\phi \in D \}\). Dunque
    \(Cy\phi = \{ y \in A \,|\, y\phi \in C\phi \}\). Ma ora \(x\phi \in C\phi\),
    perché \(x\in C\). Dunque \(x \in C\phi\phi^{-1}\).
}

Nel solito esempio
\[
    \{1,-1\}\phi\phi^{-1} = \{1,-1\}   
\]
e
\[
    \{1\}\phi\phi^{-1} = \{1,-1\}   
\]

\sproposition{}{
    Se \(D\subseteq B\) allora \(D\phi^{-1}\phi \subseteq D\). L'inclusione può essere stretta.
}
\sproof{}{
    Sia \(x\in D\phi^{-1}\phi\). Ciò significa che \(x = z\phi \)
    per qualche \(z\in D\phi^{-1}\). Ma \(D\phi^{-1} = \{y\,|\, y\phi \in D\}\).
    Dunque, \(z\in D\phi^{-1}\), allora \(z\phi \in D\), cioè \(x\in D\).
}

Con il solito esempio
\[
    \{1,2\}\phi^{-1}\phi = \{1\}
\]
\[
    \{-1\}\phi^{-1}\phi = \emptyset
\]

\sproposition{}{
    Siano \(\phi \colon A \to B\), \(\psi\colon B\to C\) e \(\theta \colon C \to D\) funzioni.
    allora
    \[
        (\phi\psi)\theta = \phi(\psi\theta)
    \]
}

\sproof{}{
    Notiamo che \(\phi\psi \colon A \to C\) e \(\theta\colon C \to D\). Dunque
    \(\phi\psi \colon A \to D\). Analogamente \(\phi \colon A \to B\), \(\psi\theta\colon B\to D\)
    e quindi \(\phi (\psi \theta ) \colon A \to D\). Per mostrare l'uguaglianua devo mostrare che  per ogni \(x\in A\)
    risulta
    \[
        a((\phi\psi)\theta) = a(\phi(\psi \theta))
    \]
    Abbiamo infatti \(a((\phi\psi)\theta) = (a(\phi\psi\theta)) = ((a\phi)\psi)\theta\)
    e \(a(\phi(\psi \theta)) = (a\phi)(\psi\theta) = ((a\phi)\psi)\theta\).
}

Dunque possiamo scrivere semplicemente \(\phi\psi\theta\) senza ambiguità.

Siano \(\phi \colon A \to B\), \(\psi\colon B\to C\) funzioni. 
Ci chiediamo ora che \(\psi\phi = \phi\psi\). Chiaramente, non è detto
che \(\phi\psi\) esista. Possiamo confrontarle solo che \(A=B\).

Allora guardiamo \(\phi \colon A\to A\) e \(\psi \colon A\to A\).
Non è comunque detto che \(\psi\phi = \phi\psi\) siano uguali.

\sdefinition{Funzione identità}{
    Dato un insieme \(A\), la \textit{funzione identica} di \(A\) è la funzione
    \(\text{Id}_A\colon A \to A\) definita come \(a\text{Id}_A \triangleq a\).
}

\sproposition{}{
    Sia \(\phi\colon A \to B\), allora \(\phi\text{Id}_B=\phi\) e \(\text{Id}_A\phi=\phi\).
    TODO: dimostrazione.
}

\subsection{Iniettività suriettività}

La definizione di iniettività è equivalente a dire che \(b\phi^{-1}\) contiene solo un elemento. \\
La definizione di suriettività è equivalente a dire che \(b\phi^{-1}\) contiene almeno un elemento. \\
La definizione di suriettività è equivalente a dire che \(b\phi^{-1}\) e \(b\phi\) contengono solo un elemento.

\subsection{Composizione}

% TODO stellar
Date \(f\) e \(g\) cosa possiamo dire di \(f\) e \(g\) sapendo che \(g(f)\) è suriettiva o iniettiva?

Supponiamo che \(g(f)\) sia suriettiva.
Dunque, per ogni \(c\in C\) esiste \(a\) tale che \(c=g(f(a))\).
In particolare, posto \(b=f(a) \in B\), abbiamo che \(g(b)=c\) cioè \(g\) è suriettiva.
% ma non possiamo dire che f sia suriettiva

Supponiamo che \(g(f)\) sia iniettiva.
Dunque, per ogni \(a_1, a_2 \in A\) dove \(a_1 \neq a_2\), risulta che \(g(f(a_1)) \neq g(f(a_2))\).
Sicuramente la prima funzione non può fare convergere i due elementi, in quando non potrebbero uscire separati
dopo la seconda funzione.
In particolare, \(f(a_1) \neq f(a_2)\). Quindi, \(f\) è iniettiva.
% ma non possiamo dire che g è iniettiva

\sexample{}{
    Siano \(A = \{a\}\) e \(B=\{b, b'\}\) con \(b\neq b'\), \(C=\{c\}\)e \(f \colon A \to B\)
    data \(f(a)=b\) e \(g\colon B \to C\) data \(g(b) = c\) e \(g(b') = c\).
    Allora \(g(f)\) è biettiva. \(f\) è iniettiva e \(g\) non è iniettiva.
    \(f\) non è suriettiva e \(g\) è suriettiva.
}

\subsection{Definizione di invertibilità}

Data \(f\colon A \to B\), allora \(f\) è invertibile se esiste \(g\colon B \to A\)
tale che \(g(f)\) è la funzione identità su \(A\) e \(f(g)\) è la funzione identità su \(B\).

\sproposition{}{
    Se \(f\) è invertibile, allora \(g\) è unica.
}

\sproof{}{
    Prendiamo \(h\colon B \to A\) tale che \(h(f(a)) = a\) e \(f(h(b)) = b\).
    Allora \(g = g\text{Id}_A = g(fh) = (gf)h = \text{Id}_Bh=h\) e quindi è la funzione identità.
}

\sproposition{}{
    Ogni inverso è anch'esso invertibile \({f^{-1}}^{-1}\).
}

\sproposition{}{
    Se \(f\colon A \to B\) e \(g\colon B \to C\) sono invertibili, allora \(g(f)\)
    è invertibile e \({\left(g(f)\right)}^{-1} = f^{-1}(g^{-1})\).
}

\sproof{}{
    Sappiamo che esistono \(f^{-1}\) e \(g^{-1}\). Dunque esiste \(f^{-1}(g^{-1})\).
    Mostriamo che componendo le due in maniera simmetrica si trovano le identità di \(A\) e di \(B\).
}

\sproof{Invertibilità è equivalente a biettività}{
    \iffproof{
        Sia \(f\) invertibile. Allora sappiamo che \(f(f^{-1})\) è la funzione identità di \(A\) e \(f^{-1}(f)\)
        è la funzione identità di \(B\).
        Ora, l'identità di \(A\) è iniettiva (anche biettiva), dunque \(f\) è iniettiva
        e l'identità di \(B\) è suriettiva (dalle due proposizioni di prima), dunque \(f\) è suriettiva.
    }{
        Sia \(f\) biettiva.
        Dobbiamo costruire \(g \colon B \to A\) tale che \(f(g)\) è l'identità di \(A\) e \(f(g)\) è l'identità di \(B\).
        Sappiamo che per ogni \(b\in B\) esiste un unico \(a\in A\) tale che
        \(f(a) = b\). Poniamo allora \(g(b)=a\). Se \(b \in B\), allora \(f(g(b)) = f(a) = b\).
        Se \(a\in A\), abbiamo che \(g(f(a))\) è per definizione di \(g\) l'unico elemento \(a'\in A\)
        tale che  \(f(a') = f(a)\). Siccome \(f\) è iniettiva, \(a'=a\), e quindi \(g(f(a)) = a\)
        e quindi \(g=f^{-1}\).
    }
}

%% Usare anche \inversefunction su stellar

\pagebreak

\section{Matrici}

Data una matrice \(A\) indichiamo con \(A_{i,j}\) l'emenento di posto \((i,j)\).

La trasporta di una triangolare inferiore è triangolare superiore, e viceversa.
La trasporta di una matrice diagonale rimane uguale.

Una matrice uguale alla sua trasporta è detta simmetrica.

\sdefinition{}{
    Dato un anello commutativo \(R\) diciamo \(M_{m,n}(R)\) l'insieme delle matricii \(m \times n\)
    a coefficienti in \(R\).
}

L'addizione è associativa e commutativa (come nell'anello commutativo).
Esiste l'elemento neutro (matrice nulla \(0_{m,n}\)).
Esiste l'elemento inverso \(-A = -1 \cdot A\).
Si dovrebbe dimostrare l'unicità dell'elemento inverso e del neutro.

\sproposition{}{
    Date matrici \(A\) e \(B\) della stessa dimensione, si ha
    \[
        {(A+B)}^t = A^t + B^t
    \]
}

\sproposition{Distributività destra}{
    Con \(A, B \in M_{m,n}(R)\) e \(C \in M_{n,p}(R)\)
    \[
        (A+B)C = AC + BC
    \]
}

\sproposition{Distributività sinistra}{
    Con \(A, B \in M_{m,n}(R)\) e \(C \in M_{n,p}(R)\)
    \[
        A(B+C) = AB + AC
    \]
}

In generale non vale \(AB=BA\). Ambo le operazioni sono definite solo se ambo le matrici sono quadrate
con dimensione \(n\times n\). In tal caso, non è comunque detto che la proprietà valga.
Nel caso in cui \(n=1\) la proprietà commutativa vale necessariamente.

Il principio di annullamento del prodotto non vale.
\[
    \begin{bmatrix}
        1 & 0 \\
        0 & 0
    \end{bmatrix}
    \begin{bmatrix}
        0 & 0 \\
        0 & 1
    \end{bmatrix}
    =
    \begin{bmatrix}
        0 & 0 \\
        0 & 0
    \end{bmatrix}
\]
In questo caso il risultato è la matrice nulla ma nessuno dei due era nulla.

\sproposition{}{
    Se \(A\) e \(B\) sono invertibili e dello stesso ordine, allora
    \(AB\) è invertibile e \({(AB)}^{-1 = B^{-1}A^{-1}}\).
}

\sexample{Matrice non invertibile}{
    \[
        \begin{bmatrix}
            1 & 2 \\
            2 & 4
        \end{bmatrix}
        \begin{bmatrix}
            x & y \\
            z & w
        \end{bmatrix}
        =
        \begin{bmatrix}
            x+2z & y+2w \\
            2x+4z & 2y+4w
        \end{bmatrix}
    \]
    Notiamo che i punti dove dovrebbe esserci uno \(0\) sono il doppio di quelli con \(1\),
    quindi non vi è soluzione e non è invertibile.
}

Se \(A\) e \(B\) sono due matrici quadrate della stessa dimensione
tali che \(AB = I_n\) allora anche \(BA = I_n\) (La dimostrazione non è banale).
% Quindi la definizione vale solo con uno.

\sproposition{}{
    Se \(A \in M_{m,n}(R)\) e \(B \in B_{n,p}(R)\) allora
    \(B^tA^t \in M_{p,m}(R)\). Abbiamo quindi che
    \[
        B^tA^t = {(AB)}^t
    \]
}

\sproposition{}{
    Se \(A\) è invertibile, allora
    \[
        {(A^t)}^{-1} = {(A^{-1})}^t
    \]
}

\pagebreak

\section{Numeri naturali}

\sdefinition{Assiomi di Peano}{
    I numeri naturali sono un insieme \(\mathbb{N}\) dotati di una funzione successore
    \(S \colon \mathbb{N} \to \mathbb{N}\)
    e di un elemento fissato \(0\) tali che:
    \begin{itemize}
        \item la funzione \(S\) è iniettiva;
        \item \(0 \not\in \text{Im}_S\);
        \item se \(A\subseteq \mathbb{N}\) tale che \(0\in A\) e \(As\subseteq A\), allora \(A = \mathbb{N}\);
    \end{itemize}
}

L'esistenza di un tale insieme è garantita dalla teoria assiomatica.
Tuttavia, dobbiamo garantire che i modelli degli assiomi di Peano siano isomorfi,
quindi trovare una funzione biettiva fra tutti i modelli.
Quindi, dati due modelli \((\mathbb{N}, S, 0)\) e \((\mathbb{N}', S', 0')\)
bisogna trovare una funzione biettiva \(f\colon \mathbb{N} \to \mathbb{N}'\)
tale che \(f(0) = 0'\) e \(nfs' = nsf\)

\begin{center}
% https://tikzcd.yichuanshen.de/#N4Igdg9gJgpgziAXAbVABwnAlgFyxMJZABgBpiBdUkANwEMAbAVxiXBAF9T1Nd9CUZAIxVajFmzAIuPbHgJEhpEdXrNWicAHI4WztxAY5-ReVFqJmsHo6iYUAObwioAGYAnCAFskZEDggkAGZVcQ0QV303Tx9EEP9AxAAmUPU2XSiImKQlBKQUsTTNSOoGOgAjGAYABV55ARB3LAcACxxMj29fagCc1MsQBFKKqtrjBU0m1vbbDiA
\begin{tikzcd}
    n \arrow[r, "f"] \arrow[d, "s"'] & n' \arrow[d, "s'"] \\
    ns \arrow[r, "f"']               & n's'              
\end{tikzcd}
\end{center}

Questo può essere fatto con un procedimento cosidetto per ricorrenza, dipende fortemente
dall'assioma 3. In generale gli assiomi di Peano mi permettono di definire successioni di oggetti
per ricorrenza, cioè assegnando un oggetto associato a \(0\)
e il modo di costruire l'oggetto associato (come per esempio il fattoriale o l'addizione nei naturali).

La somma è definita nel seguente modo ricorrente: \(m+n=0\) e \(m+S(n) = S(m+n)\).

Usango gli assiomi posso dimsotrare varie proprietà dell'addizione, detta moltiplicazione
(da definire anch'esso per ricorrenza) e dell'ordine (anch'esso da definire per ricorrenza).

L'ordine è definito solamente da \(n \leq S(n)\).

Le proprietà per \(m,n,p\in\mathbb{N}\) sono:
\begin{enumerate}
    \item \textbf{somma associativa:} \((m+n)+p = m + (n+p)\);
    \item \textbf{somma distributiva:} \(m + n = n + m\);
    \item \textbf{somma nulla:} \(m + 0 = m\);
    \item \textbf{prodotto associativo:} \((mn)p = m(np)\);
    \item \textbf{prodotto distributivo:} \(mn = nm\);
    \item \textbf{prodotto nullo:} \(m S(0)= m\);
    \item \textbf{distributiva:} \((m+n)p = mp + np\);
    \item \textbf{cancellazione somma:} \(m+n = m+p \implies n=p\);
    \item \textbf{cancellazione prodotto:} \(mn = mp \land m \neq 0 \implies n=p\);
    \item \textbf{compatibilità tra somma e ordine:} \(m\leq n \implies m+p \leq n + p\);
    \item \textbf{compatibilità tra prodotto e ordine:} \(m\leq n \implies mp \leq np\);
\end{enumerate}

Detto \(1\) il numero \(S(0)\) risulterà che \(S(n) = n+1\).

\paragraph{Assioma 3:}

\sproposition{}{
    Un altro modo per dire l'assioma 3 è che ogni sottoinsieme non vuoto di \(\mathbb{N}\)
    ammette un minimo.
}

\sproof{}{
    Per dimostrarlo sia \(A\subseteq B\) l'insieme di tutti i minoranti.
    L'insieme \(A\) contiene sicuramente \(0\). Infatti, \(0 \leq n\) per ogni \(n\in\mathbb{N}\).
    L'insieme \(A\) è diverso da \(\mathbb{N}\). Infatti, preso un \(n \in B\),
    sappiamo che \(B \neq \emptyset\), abbiamo che \(n+1\) non è minore o uguale di \(n\),
    quindi non è un minorante di \(B\). Pertanto \(n+1 \notin A\).
    Sappiamo per gli assiomi di Peano che un sottoinsieme di \(\mathbb{N}\) che contiene \(0\)
    e contiene il successore di ogni elemento, coincide con \(\mathbb{N}\).
    Poiché \(0\in A\) e \(A\neq \mathbb{N}\), possiamo concludere che esiste \(k\in A\)
    tale che \(k+1 \notin A\), cioè \(k\) è minorante di \(B\) ma \(k+1\) no.
    Ma allora esiste \(i \in A\) tale che \(k+1 \not\leq i\).
    Poiché l'ordine è totale, ciò significa che \(i < k+1\). D'altra parte
    \(k\) è minorante di \(B\). In particolare, \(k \leq i\), che è minore di \(k+1\).
    Per la proprietà dell'ordine dei naturali, si ha che \(k=1\), cioè \(k\in B\).
    Dunque, \(k\) è minorante di \(B\) che appartiene a \(B\) come volevamo (è il nostro minimo).
}

Non è necessario l'assioma della scelta per prendere \(n\in B\)
in quando \(B\) è ben definito e sappiamo come sceglierlo.

\section{Numeri interi}

Fatto l'anello commutativo degli interi si possono dimostrare delle proprietà come ad esempio
\(n\cdot0 = 0\) per ogni \(n\).

Per dimostrare invece il principio di annullamento del prodotto, cioè che \(mn=0\)
se e solo se almeno uno tra \(m\) e \(n\) è \(0\).
In alcuni anelli commutativi il principio di annullamento del prodotto non vale.

Si dimostra poi che dato \(n \in \mathbb{Z}\), si ha che
\(n\in\mathbb{Z}\) oppure \(0-n n\in\mathbb{Z}\) per ogni \(n\) intero.
Si pone allora
\[
    |n| = \begin{cases}
        n & \text{è un naturale} \\
        -n & \text{altrimenti}
    \end{cases}
\]

Una volta introdotto l'ordine negli interi (compatibile con quello dei naturali),
si dimostrano queste proprietà:
\begin{enumerate}
    \item \(a \leq b \implies a+c \leq b+c\);
    \item \(a \leq b \land c \geq 0 \implies ac \leq bc\);
    \item \(|a+b| \leq |a| + |b|\);
    \item \(|a\cdot b| \leq |a|\cdot|b|\).
\end{enumerate}

\subsection{Divisione con resto}

Estendiamo l'mcd a valori tutti nulli.
Dati due interi il loro mcd è il numero naturali
\(d\) che divide entrambi ed è multiplo di tutti i divisori comuni.
Se almeno uno tra questi è nullo, questo coincide con la definizione precedente.
Se tutti sono zero, definiamo l'mcd come zero, in quanto zero è un multiplo di zero.

\subsection{Massimo comun divisore}

Dimostrare l'esistenza di un massimo comune divisore su più interi
per induzione: il caso base è quello in cui il numero di interi è 2.
Usare l'esistenza del membro a destra e verificare che soddisfa la definizione.

% Dopo prime-number-divides-product
% Viceversa, mostriamo la contronominale, cioè che se \(p \not\divides a\) e \(p \not\divides b\)
% allora \(p \not\divides ab\).
%  I divosiri di \(p\) sono \(1\), \(p\) e i loro opposti. Se \(p\) non divide \(a\),
% i divisori comuni tra \(p\) e \(a\) sono solo \(1\) e \(-1\).
% Dunque gcd(p,a) = 1
% Riassunto:
% Se p divide a, allora gcd(p, a) = p, altrimenti gcd(p, a) = 1.
% Ora, io sto assumendo che p non divide a e p non divide b, cioè che
% p è coprimo con a e p è coprimo con b. Per la proposizione precedente,
% p è coprimo con ab e quindi, e quindi p non divide ab.
% In generale per induzione si trova che (esercizio) che se un primo p
% divide il prodotto di a_1a_2...a_n, allora ne divide almeno uno.

\section{Classi di resto}

Consideriamo i non-multipli di \(3\). La differenza fra un non-multiplo di 3 e quello dopo è 
o \(1\) o \(2\). Dividiamo allora i non-multipli di 3 saltando \(2\) a \(2\), ossia
\[
    -5, -2, +1, +4, +7, +10
\]
e
\[
    -4, -1, +2, +5, +8, +11
\]
La somma di due numeri corrispondenti è sempre un numero di 3.
In generale, se considero le tre liste
\begin{align*}
    -6, -3, +0, +3, +6, +9
    -5, -2, +1, +4, +7, +10 \\
    -4, -1, +2, +5, +8, +11 \\
\end{align*}
Se facciamo la somma di due termini, la lista in cui è il risutato è dato solamente dalle liste
dei due addenti.

\section{Monoidi e gruppi} % TODOURGENT

La moltiplicazione in matrici quadrate è associativa ma non commutativa.
\\
La composizione \(X^X\) è associativa ma non commutativa.
Studiamo la commutatività:
diciamo che se \(|X| = 1\), allora abbiamo solo l'identità \(X^X = \{\text{Id}_X\}\),
in questo caso è quindi commutativa.
Supponiamo ora \(|X| = 2\) e quindi \(X = \{a,b\}\). Allora abbiamo le seguenti funzioni \(X^X = \{\text{Id}_A, \text{Id}_B, \text{Inv}_A, \text{Inv}_B\}\),
che sono \(4\) possibilità.
Per vedere se è commutatia, dovrei considerare
tutte le possibili coppie ordinate \((f,g) \in X^X \times X^X\)
(sono 16) e vedere se \(fg = gf\). Chiaramente, se facciamo la composizione della funzione che manda
sempre in \(a\) con quella che manda sempre \(b\), non commuta.

Se \(AB\) è l'identità delle matrici, allora lo è anche \(BA\).
Se ho un inverso a destra e uno a sinistra sono uguali.

Il \textit{gruppo lineare generale} è dato da
\[
    GL_n(R) = (\text{Inv}(M_n(R)), \cdot)
\]

Nella tabella di Caley: l'operazione è commutative se la tabella
è specchiata sulla diagonale. L'associatività non è facile da vedere.
Vi è un elemento neutro se la riga e la colonna dell'elemento neutro ripetono le etichette.
Un elemento è invertibile nella sua riga e colonna vi è un \(1\).
Il fatto che l'equazione \(ax = b\)
abbia una sola solzione si interpreta dicendo che sulla riga
di \(a\) appaiono tutti gli elementi del gruppo una sola volta
(analogamente per el colonne).
Su ogni riga e colonna ogni elemento può comparire una volta sola.


Esiste un gruppo di ogni ordine \(n\), per esempio \((\mathbb{Z} / n, +)\) ha ordine \(n\).

\pagebreak

\section{Gruppi geometrici}

%\sdefinition{Isometria}{
%    Un'\textit{isometria} del piano \(\pi\) è una funzione
%    \[
%        \sigma\colon \pi \to \pi
%    \]
%    tale che \[
%        \forall P, Q \in \pi, d(P, Q) = d(\sigma(P), \sigma(Q))
%    \]
%}

%Ovviamente un'isometria è iniettiva.
%Infatti, se \(P \neq Q\), allora \(\sigma(P) \neq \sigma(Q)\).

%\sproposition{}{
%    Un'isometria del piano è una funzione biettiva.
%}

%Il fatto che sia suriettiva non è banale.

%Questo si dimostra ocn la disuglianza triangolare notando che il segno uguale
%vale solo quando un punto si trova sulla retta fra gli altri due punti.

%\sproposition{}{
%    Se \(\sigma\) e \(\tau\) sono due isometrie che coincidono
%    su 3 punti non allineati, allora \(\sigma = \tau\).
%}

Altro: un gruppo generato è abeliano se e solo se gli elementi del
sottogruppo commutano fra di loro.

\sexercise{}{
    Dimostra 
    \[
        \langle g^h \rangle \cap \langle g^k \rangle = \langle g^m \rangle
    \]
    con \(m\) minimo comune multiplo di \(h\) e \(k\).
}

%\section{Recuperare}

%!!!!! Se due sottogruppi di un gruppo ciclico hanno lo stesso ordine, sono lo stesso.
%
%1) Dato \(H \leq G\), con \(H = \langle g^n \rangle\), qual'è l'ordine di H, cioè 
%qual'è il periodo di g^n?
%
%2) Come sono "messi" i sottogruppi di G?
%
%Cerchiamo il periodo di g^n. Dobbiamo stabilire quali sono gli esponenti k tali che (g^n)^k= 1.
%Sappiamo che ciò avviene se e solo se d divide nk. Dobbiamo trovare il più piccolo intero positivo k per cui
%ciò avviene. Ovviamente se k=d banalmente funziona. Tuttavia, in generale non è detto che sia il più piccolo.
%Ciò mi dice tuttavia che il periodo che sto cercando è a sua volta un divisore di d.
%Dunque |g^n| divide d.
%
%Facciamo qualche esempio per ipotizzare una relazione tra |g| e |g^n|.
%Prendiamo per esempio \(|g| = 12\). Calcoliamo il periodo di alcuni elementi
%prendeno le potenze successive con esponenti interi positivi.
%
%Con 2 Abbiamo g^2, (g^2)^4, (g^2)^3, (g^2)^4, (g^2)^5, (g^2)^6.
%Siccome dobbiamo trovare il più piccolo numero tale che g^k=g^12, il periodo è 6.
%
%Con g^3 abbiamo 4 e con g^4 con 3.
%Tuttavia, questi sono tutti divisori di 12 e quindi è facile.
%
%Prendiamo 5. Abbiamo (g^5)^2, (g^5)^3, (g^5)^4...
%
%Prendiamo 8. (g^8)^2, (g^8)^3 = g^24 = 1. Quindi il periodo è 3.
%
%Il periodo è quindi dato da p / gcd(p, k)

\section{Altro}

Prendiamo un punto \(P\) del piano e una trasformazione \(\sigma\) del nostro gruppo.
Consideriamo allora \(\sigma(P)\). Questo modo di associare una coppia
\((P, \sigma) \in \pi \times G\) ha alcune regole.
Per esempio \((P, \text{Id}) \to P\) per ogni \(P\),
\((P, \sigma) \to Q\) e \((Q, \tau) \to R\), allora \((P, \tau(\sigma)) \to R\).

\section{Sottogruppi normali}

Vogliamo contare il numero di coniugati di \(H\) in \(G\).
Dato \(H\) elemento di \(X\), cioè sottogruppo in \(G\),
e \(g\in G\), consideriamo l'azione di \(G\) per coniugio.
Ossia, \((H,g) \to H^g\). Chiaramente questa è un'azione in quanto
\(H^1 = H\) e \({(H^{g_1})}^{g_2} = H^{g_1g_2}\).
La sua orbita è l'insieme dei coniugati di \(H\). Lo stabilizzatore
sono gli elementi di \(G\) tali che \(H^g = h\).

\sdefinition{Normalizzante}{
    Il \emph{normalizzante} in \(G\)
    di \(H\) è il sottoinsieme di \(G\) così definito
    \[
        N_G(H) \triangleq \{g\in G \,|\, H^g = H \}
    \]
}

Alcune proprietà:
\begin{enumerate}
    \item \(N_G(H) \leq G\) (è un particolare stabilizzatore rispetto ad un azione, quindi sottogruppo);
    \item \(H \leq N_G(H)\) infatti se \(h \in H\), ovviamente \(H^n = n\);
    \item il numero dei coniugati di \(H\) è uguale all'indice dello stabilizzatore, cioè del normalizzante in \(G\).
    \[
        |G\,:\, N_G(H)|
    \]
    Infatti la cardinalità di un'orbita è uguale all'indice nel gruppo dello stabilizzante di un elemento.
    \item In particolare, se l'indice \(|G\,:\,H|\) è finito, che avviene
    almeno sicuramente se \(G\) è finito, allora abbiamo che
    \[
        |G\,:\, H| = |G \,:\, N_G(H)| \cdot |N_G(H) \,:\, H|
    \]
    Ma allora ciò mi dice che il numero di coniugati è finito e
    divide l'indice \(|G \,:\, H|\).
    \item Se \(H \leq K \leq G\), allora \(H \unlhd K\) se e solo se
    \(K \leq N_G(H)\). In altri termini, i sottogruppi \(K\)
    in cui \(H\) è normale, sono tutti e soli quelli per cui
    \[
        H \leq K \leq K_G(H)
    \]
    \item In particoalre \(H \unlhd N_G(H)\), cioè il normalizzante
    di \(H\) è il più grande sottogruppo di \(G\) in cui \(H\) è normale.
    Si ha \(H \unlhd G\) se e solo se \(N_G(H) = G\).
\end{enumerate}

Ricordiamo che dati due sottogruppi \(H\) e \(K\) di un gruppo \(G\),
si ha che \(HK \leq G\) in particolare se e solo se
\(HK = KH\). Inoltre, \(H \unlhd G\) se e solo se \(Hg = gH\)
per ogni \(g\in G\).
Ora,
\[
    HK = \{hk \,|\, h\in H, k\in K\} = \bigcup_{k\in K} Hk
\]
e
\[
    KH = \{kh \,|\, h\in H, k\in K\} = \bigcup_{k\in K} kH
\]
Quindi, se \(H \unlhd G\), allora \(Hg = gH\)
per ogni \(g \in G\). In particolare,
ciò è vero per ogni \(k \in K\), cioè \(Hk = kH\)
per ogni \(k\in K\), e quindi
\[
    \bigcup_{k\in K} Kh = \bigcup_{k\in K} kH \iff HK = KH
\]
Riassumento, se \(H \unlhd G\) e \(K \leq G\), allora
\(HK = KH\), cioè \(HK \leq G\).
Se anche \(K \unlhd G\), possiamo dire che
\(HK \unlhd G\).

\sproposition{}{
    Se \(H \unlhd G\) e \(K \unlhd G\),
    allora \(HK \unlhd G\).
}

\sproof{}{
    Sappiamo già che \(HK\) è un sottogruppo, siccome almeno uno dei due è normale,
    allora usiamo la proprietà che \(x^g \in HK\) per ogni \(x\in HK\).
    Ora, \(x = hk\) per qualche \(h\in H\) e \(k\in K\).
    Ma allora \(x^g = h^gk^g\). Poiché \(H \unlhd G\), si ha che \(h^g \in H\)
    e analogamente \(h^g k^g \in K\).
    Allora, \(x^g \in HK\) come volevamo.
}

\scorollary{}{
    Se \(H_1, H_2, \cdots, H_r\) sono sottogruppi normali di \(G\),
    allora \(H_1H_2\cdots H_r \unlhd G\).
}

Si dimostra per induzione su \(r\).

\subsection{classi di coniugio nel simmetrico}

\slemma{}{
    Sia \(\sigma = (a_1\;a_2\;\cdots\;a_r)\)
    e \(\tau\) una permutazione qualunque \(n\) lettere.
    Allora,
    \[
        \sigma^\tau = (\tau(a_1)\;\tau(a_2)\;\cdots\;\tau(a_r))
    \]
}

\sproof{}{
    Dobbiamo mostrare che \(\tau^{-1} \sigma \tau^{-1}\) è uguale all'espressione data,
    cioè che
    \[
        \sigma \tau = \tau(a_1\tau\; a_2\tau\; \cdots)
    \]
    oppure
    \[
        (a_1\;a_2\;\cdots) \tau = \tau (a_1 \tau\; a_2\tau\; \cdots)
    \]
    Questo è equivalente a dire
    che per ogni lettera \(i\) si ha che
    \[
        i(a_1\; \cdots\; a_r) \tau = i\tau(a_1\tau\; \cdots)
    \]
    Distinguiamo due casi:
    \begin{enumerate}
        \item \(i\) è una degli \(a_j\). Senza perdita di generalità supponiamo
        \(i=a_1\) (al massimo riordiniamo il ciclo). Allora abbiamo
        \[
            a_1(a_1 \; \cdots \; a_r) \tau = a_2 \tau
        \]
        e
        \[
            a_1\tau(a_1\tau \; \cdots \; a_r\tau) = a_2 \tau
        \]
        e quindi coincidono.
        \item \(i\) non è nessuno degli \(a_j\), allora
        \[
            i(a_1\; \cdots \; a_r) \tau = i\tau
        \]
        e
        \[
            i\tau(a_1\tau\; \cdots \; a_r\tau) = i\tau
        \]
        perché \(i\tau \neq a_j\tau\) per ogni \(j\).
        Infatti, \(i \neq j\) per ogni \(j\), e \(\tau\) è iniettiva.
    \end{enumerate}
    Dunque, il coniugato di un r-ciclo è un r-ciclo.
}

Tutti gli r-cicli sono coniugati tra di loro.
Infatti, dati \(a_1\; \cdots\; a_r\) e \(b_1\; \cdots\; b_r\),
basta prendere \(\tau\) tale che \(a_j \tau = b_j\) per ogni \(j\) da \(1\) a \(r\)
e completare \(\tau\) in modo che sia biettiva, cioè dando valori arbitrari a \(i\tau\)
per ogni \(i\) che non sia uno degli \(a_j\).

\stheorem{}{
    Due elementi di \(\text{Sym}_n\) sono coniugati
    se e solo se hanno lo stesso tipo.
}

\sproof{}{
    Sia \(\sigma = \gamma_1\gamma_2 \cdot \gamma_t\) con \(\gamma_i\)
    cicli disgiunti di lunghezze \(r_1 \geq r_2 \cdots\)
    e sia \(\tau \in \text{Sym}_n\). Allora,
    \[
        \sigma^\tau =
        {(\gamma_1\; \gamma_2\; \cdots \; \gamma_t)}^\tau
        = \gamma_1^\tau \gamma_2^\tau \cdots \gamma_t^\tau
    \]
    e dal lemma precedente segue immediatamente che
    \(\gamma_i^\tau\) sono disgiunti e di lunghezze
    rispettivamente \(r_i \geq r_2 \cdots\).
    Viceversa, se abbiamo 2 permutazioni dello stesso tipo,
    completando come per il lemma, si mostra che
    sono coniugati.
}

\sexercise{}{
    Trovare un \(\tau \in \text{Sym}_8\) tale che
    \[
        {\left((1\;2)(3\;5\;6\;7)\right)}^\tau = (2\;8)(1\;4\;3\;5)
    \]
    Basta prendere \(\tau\) tale che \(1\tau = 2\), \(2\tau = 8\) etc.
    e completare, quindi per esempio \(4\tau = 6\) e \(8\tau = 7\).
    Quindi \(\tau = (1\;2\;8\;7\;6\;4\;6\;3)\)
}

\sexample{}{
    Calcolo delle classi di coniugio di \(\text{Sym}_n\)
    con \(n=1 \cdots 5\) (e calcolo dei centralizzanti).
    % n g |g^G| |C_G(g)|
    % 1 Id    1 1
    % 2 Id    1 2
    %   (12)  1 2
    % 3 Id    1 6
    %   (12)  3 2
    %   (123) 2 3
    % 
    
    % Per esempio |C_g((12))| = 2. Sappiamo che contiene <(12)>, ma l'ordine di quest'ultimo è 2 quindi coincidono.
    % Per il 3-ciclo vale lo stesso ragionamento
    % Sottogruppi normali di Sym_3: I sottogruppi normali sono unione di classi di coniugio,
    % fra cui vi è sicuramente la classe di Id. Abbiamo queste possibilità:
    %   - Id, Id U (12)^G, Id U (123)^G, tutto
    % quali di questi sono sottogruppo? Quelli che lo sono, sono automaticamente anche normali.
    % Il primo e l'ultimo lo sono sicuramente (banali).
    % Il secondo contiene 4 elementi (per la tabella), quindi non è un sottogruppo.
    % Il terzo elemento è A_3, che sappiamo essere sottogruppo. Quindi vi sono 3 sottogruppi normali

    % Non esiste una forula per calcolare quante sono le classi di coniugio nel simmetrico a n lettere
    % (Sono le partizioni di n)
}

Abbiamo visto che le due permutazioni in \(\text{Sym}_n\)
sono coniugate se e solo se hanno lo stesso tipo, dunque le classi di coniugio
sono formate dagli elementi di tipo assegnato

    % n g |g^G| |C_G(g)|
    % 4 Id       1 24
    %   (12)     6 4
    %   (123)    8 3
    %   (1234)   6 4
    %   (12)(34) 3 8

Un sottogruppo normale è unione di classi di coniugio
tra cui perlomeno la classe di identità.
Tali unioni, se sottogruppi, sono normali.

Abbiamo sicuramente il sottogruppo normale identità e tutto il simmetrico \(\text{Sym}_4\).
\textbf{Unione di due classi:} abbiamo possibilmente \(1+6 = 7\)
elementi. Notiamo che \(7\) non divide \(24\) e quindi non può essere un sottogruppo.
Allo stesso modo \(1+6\) di nuovo. Possiamo anche scartare \(1+9\).
Rimane \(1+3=4\), che potrebbe essere sottogruppo.
\[
    \{\text{Id}, (1\;2)(3\;4), (1\;3)(2\;4), (1\;4)(2\;3)\}
\]
\textbf{Unione di tre classi:} abbiamo \(1+8+3 = 12\)
\[
    \{\text{Id}\} \cup {(1\;2\;3)}^G \cup {(1\;2)(3\; 4)}^G = A_4
\]
che è normale in \(\text{Sym}_4\).\\
\textbf{Unione di quattro classi:} non ci sono divisori.

Abbiamo quindi trovato \(3\) sottogruppi normali e 1 candidato \(T\).
Controlliamo se \(T\) è sottogruppo.
Dobbiamo verificare che \(\sigma\tau\) con \(\sigma\tau \in T\) rimangano in \(T\).
Se uno dei due è l'identità, ciò è banale. Quindi rimangono le altre 9 operazioni.
Se \(\sigma = \tau\), abbiamo sempre \(\sigma\sigma = \text{Id}\).
Allora rimangono 6 casi:
\begin{itemize}
    \item \((1\;2)(3\;4)\circ (1\;3)(2\;4) = (1\;4)(2\;4) \in T\);
    \item gli altri sono analoghi;
\end{itemize}
Dunque \(T\) è un sottogruppo ed è quindi normale.
Siccome \(T\) ha 4 elementi è isomorfo a \(C_3\) oppure al gruppo di Klein.
Tuttavia, \(T\) non è ciclico (non contiene elementi di periodo 4) quindi è il gruppo trirettangolo.
Abbiamo quindi \(\text{Id}, T, A_4, \text{Sym}_4\) come sottogruppi.

\textbf{Problema:} siano \(H \unlhd K\)
e \(K \unlhd G\). È vero che \(H \unlhd G\)? Per esempio \(G = \text{Sym}_4\)
e \(K=T\) e \(H = \langle(1\;2)(3\;4)\rangle\).
Ora, \(H \unlhd K\) perché \(K\) è abeliano e \(K \unlhd G\),
\(H\) non è normale in \(G\) perché non è unione di classi di coniugio come visto.

Guardiamo ora il caso \(n=5\).
    % n g |g^G| |C_G(g)|
    % 5 Id        1  120
    %   (12)      10 12
    %   (123)     20 6
    %   (1234)    30 4
    %   (12345    24 5
    %   (12)(34)  15 8
    %   (123)(45) 20 6

Notiamo che il centralizzante \(C_G((1\;2\;3\;4)) \geq \langle(1\;2\;3\;4)\rangle\).
Inoltre sappiamo che entrambi hanno ordine \(4\). Quindi, il centralizzante
è esattamente \(C_G((1\;2\;3\;4)) = \langle(1\;2\;3\;4)\rangle\).
Analogamente per \((1\;2\;3\;4\;5)\). Lo stesso vale per i cicli 3 per 2 che hanno ordine 6.
Facendo i calcoli estensivi si trova che i sottogruppi normali sono
\[
    \text{Id}, A_5, \text{Sym}_5
\]

Si potrebbe dimostrare che che per ogni \(n \geq 5\), i sottogruppi normali di \(\text{Sym}_n\)
sono questi 3. Questo è legato al fatto che none sistano soluzioni di polinomi
del grado quinto in poi.

Classi di coniugio nell'alterno.
Sappiamo che \(A_n \unlhd \text{Sym}_n\).
Gli elementi \((1\;2\;3)\) e \((1\;3\;2)\) sono coniugate nel simmetrico
3 lettere ma non nell'alterno a 3 lettere.

\slemma{}{
    Sia \(H\) un sottogruppo di un gruppo \(G\)
    di indice finito e sia \(K\) un sottogruppo di
    \(G\) qualsiasi. Allora \(H \cap K\) ha indice finito
    ha indice finito in \(K\)
    e
    \[
        |K\,:\, H \cap K| \leq |G\,:\,H|
    \]
}

\sproof{}{
    Supponiamo che \(|G\,:\,H| = n\).
    Devo mostrare che se prendo \(n+1\) laterali (destri) di \(H \cap K\)
    in \(K\), questi non sono tutti diversi.
    Siano allora \(H \cap Kk_1\), \(H \cap Kk_2\), \(\cdots\), \(H \cap Kk_{n+1}\)
    laterali di \(H \cap K\) in \(K\) (cioè siano \(k_1, k_2, \cdots, k_{n+1}\) di \(K\)).
    Consideri questi laterali (destri) di \(H\) in \(G\)
    \[
        HK_1, HK_2 \cdots HK_{n+1}
    \]
    Poiché \(|G\,:\,H| = n\), almeno due di questi coincidono.
    Ad esempio, \(Hk_1 = Hk_2\), cioè \(k_1 = hk_2\)
    per qualche \(h\in H\).
    Ma allora \(h = k_2k_1^{-1} \in K\), cioè \(h \in H \cap K\).
    Ma allora \(H \cap Kk_1 = H\cap Kk_2\).
}

\stheorem{}{
    Sia \(G\) un gruppo finito e sia \(H\)
    un suo sottogruppo di indice \(2\) (dunque \(H \unlhd G\)).
    Se \(x\in H\), vale una e una sola delle seguenti:
    \begin{enumerate}
        \item \(x^H = x^G\) e \(|C_G(x)| = 2 |C_H(x)|\);
        \item \(x^G = x^H \cup {(x')}^H\) con \(x', x\) non coniugati in \(H\),
            \[
                |x^H| = |{(x')}^H| = \frac{1}{2} |x^G|
            \]
            quindi la classe si separa in due con lo stesso numero di elementi,
            e \(C_G(x) = C_H(x)\).
    \end{enumerate}
}

\sproof{}{
    Applichiamo il lemma prendendo come 
    \(K = C_G(x)\). Abbiamo allora, 
    \[
        |C_G(x) \,:\, C_G(x) \cap H| \leq |G\,:\,H| = 2
    \]
    Ora \(C_G(x) \cap H = \{x\in H \,|\, xg = gx\} = C_H(x)\).
    Duque \(|C_G(x)\,:\,C_H(x)| \leq 2\).
    Sappiamo poi che \(|x^G||C_G(x)| = |G|\)
    e che \(|x^H||C_H(x)| = |H|\).
    ma \(|G| = 2|H|\), quindi otteniamo
    \[
        |x^G||C_G(x)| = 2|x^H||C_H(x)|
    \]
    Ma \(|C_G(x) \,:\, C_H(x)| \leq 2\).
    Abbiamo 2 possibilità:
    \begin{enumerate}
        \item \(|C_G(x)| = |C_H(x)|\) cioè \(C_G(x) = C_H(x)\);
        \item \(|C_G(x)| = 2|C_H(x)|\)
    \end{enumerate}
    nel primo caso
    \[
        |x^G||C_G(x)| = 2|x^H||C_H(x)|
    \]
    diventa \(|x^G| = 2|x^H|\), nel secondo caso \(|x^G| = |x^H|\), cioè \(x^G = x^H\).
}

Questo teorema, in particolare, vale per la classe di coniugio dell'alterno.

Sia \(\varphi \colon G \to H\) un omomorfismo.

\subsection{Omomorfismi di sottogruppi normali}

Se \(K \unlhd G\), allora \(K\varphi \unlhd H\)? Non sempre.
Sappiamo che \(K\varphi = \{k\varphi \,|\, k\in K\}\).
Per mostrare che è un sottogruppo normale dobbiamo mostrare che
\({(k\varphi)}^h = k\varphi\) per tutti \(h \in H, k \in K\),
cioè che
\[
    h^{-1}(K\varphi)h = k'\varphi
\]
per qualche \(k' \in K\).
Se sapessimo che \(h = g\varphi\) per qualche \(g\in G\), allora avremmo
\[
    g\varphi^{-1}(k\varphi)(g\varphi) =
    (g^{-1} \varphi)(k\varphi)(g\varphi)
    = (g^{-1}kg)\varphi
\]
che appartiene a \(K\varphi\) perché \(g^{-1} k g = k^g \in K\).
Quindi, \(K\varphi \unlhd \text{Im}\{\varphi\}\) necessariamente solamente per quella condizione.
Prendiamo un esempio per mostrare che in generale la proposizione non è vera.
Sia \(\varphi \colon C_2 \to \text{Sym}_3\) con \(c_2 = \langle g \rangle\) tale che
\[
    \varphi(g) = (1\;2)
\]
Siccome lo scambio ha periodo \(2\) che divide il periodo di \(G\) (sono uguali),
questo omomorfismo è ben definito. Ora \(C_2 \unlhd C_2\), ma l'immagine di \(\varphi(C_2) = \langle (1\;2) \rangle\)
non è normale nel simmetrico su 3 lettere.

Se \(L \unlhd H\), allora \(L\varphi^{-1} \unlhd G\)? Sì.
Ricordiamo che
\[
    L\varphi^{-1} = \{x \in G \,|\, x\varphi\in L \}
\]
Bisogna verificare se per ogni \(x \in L\varphi^{-1}\)
e ogni \(g\in G\) risulta \(x^g \in L \varphi^{-1}\) cioè \((x^g)\varphi \in L\).
\[
    (x^g) \varphi = (g^{-1} x g) \varphi = {(g\varphi)}^{-1} x \varphi(g\varphi)
    = {(x\varphi)}^{(g\varphi)} \in L^{g\varphi} = L
\]

\sproposition{}{
    Sia \(\varphi \colon G \to H\) un omomorfismo di gruppi,
    allora:
    \begin{enumerate}
        \item se \(K \unlhd G\), allora \(K \varphi \unlhd \text{Im} \varphi\);
        \item se \(L \unlhd H\), allora \(L \varphi^{-1} \unlhd G\).
    \end{enumerate}
}

\subsection{A cosa servono i sottogruppi normali}

Proprietà che vengono preservate nelle congruenze in \(X\) (relazioni di equivalenza che sono compatibili rispetto ad un operazione).
Quindi se \(x\sim x'\) e \(y \sim y'\), allora \(x\circ y \sim x'\circ y'\).

\begin{enumerate}
    \item se \(\circ\) in \(X\) è associativa, allora l'operazione in \(X / \sim\) è associativa.
    \[
        ({[x]}_\sim{[y]}_\sim){[z]}_\sim =  {[x]}_\sim({[y]}_\sim{[z]}_\sim) 
    \]
    \item se \(\circ\) in \(X\) è commutativa, allora l'operazione in \(X / \sim\) è commutativa;
    \item se \(\circ\) in \(X\) ha elemento neutro \(1\), allora \({[1]}_\sim\) è elemento neutro in \(X / \sim\);
    \item se \(x\) è invertibile in \(X\), allora \({[x]}_\sim\) è invertibile in \(X / \sim\).
    \item se \(M\) è un monoide e \(\sim\) è una congruenza in \(M\), allora \(M / \sim\) è un monoide.
        Se è commutativo allora \(M/\sim\) è commutativo.
    \item se \(G\) è un gruppo e \(\sim\) è una congruenza in \(G\), allora \(G / \sim\) è un gruppo chaimato \emph{gruppo quoziente}.
        Se è abeliano allora \(G/\sim\) è abeliano.
\end{enumerate}

\sexample{}{
    Consider \((\mathbb{N}, +)\) and let \(n\in \mathbb{N}\)
    and defined \(\sim\) such that \(a\sim a'\) if
    \(a=a'\) or \((a > n) \land (a' > n)\).
    \begin{itemize}
        \item è una relazione di equivalenza;
        \item è una congruenza.
    \end{itemize}
}

\stheorem{}{
    Sia \(G\) un gruppo e sia \(\sim\) una congruenza.
    Risulta allora:
    \begin{enumerate}
        \item \({[1]}_\sim\) è un sottogruppo normale \(H\) di \(G\);
        \item le classi di equivalenza sono esattamente i laterali di \(H\) in \(G\).
        \item Viceversa, se \(H \unlhd G\) allora la relazione di equivalenza le cui classi sono i laterali di \(H\) in \(G\) è una congruenza;
            Dunque, dare una congruenza in \(G\) o un sottogruppo normale
            \(H\) è la stessa informazione.
    \end{enumerate}
    Dato \(H \unlhd G\) scriviamo \(G/H\) per indicare il gruppo quoziente.
}

La medesima classificazione con i sottogruppi normali non funzione bene nei monoidi (come nell'esempio precedente).

\sproof{}{
    \iffproof{
        Data ua congruenza, allora la la classe dell'identità è un sottogruppo normale
        e le classi sono i laterali.
        \begin{enumerate}
            \item \({[1]}_\sim = H \neq \emptyset\) perché \(1_G \in H\);
            \item se \(x\in H\) e \(y\in H\), cioè \(x\sim 1_G\) e \(y \sim 1_G\), allora
            \(xy \sim 1_G 1_G = 1_G\), cioè \(xy\in H\);
            \item se \(x\in H\), cioè \(x \sim 1\). Siccome \(x^{-1} \sim x^{-1}\),
            abbiamo che \(x^{-1} x \sim x^{-1} 1_G\), cioè \(x^{-1} \in H\).
        \end{enumerate}
        Per mostrare che il sottogruppo è normale mosrtriamo che se \(x\in H\) e \(g\in G\),
        cioè \(x\sim 1_G\), allora \(x^g = g^{-1} x g \sim g^{-1} g = 1_G\). Quindi, \(x^g \in H\)
        e allora è normale in \(G\).
        Mostriamo ora che le classi di equivalenza sono i laterali di \(H\)
        in \(G\). Dobbiamo mostrare una per ogni \(g\in G\), risulta \({[g]}_\sim = Hg\).
        \begin{enumerate}
            \item \({[g]}_\sim \subseteq Hg\): sia \(x\sim g\). Abbiamo
            \(x = (xg^{-1})g\) ovviamente. Basta mostrare che \(xg^{-1} \in H\).
            Ora \(xg^{-1} \sim gg^{-1} = 1_G\) cioè \(xg^{-1} \in H\);
            \item \({[g]}_\sim \supseteq Hg\): sia \(x\sim Hg\) cioè
            \(x = hg\) con \(h \in H\), vale a dire \(h \sim 1_G\).
            Ma allora \(x=hg \sim 1_G g = g\), cioè \(x\in {[g]}_\sim\).
        \end{enumerate}
    }{
        sia \(H \unlhd G\) e sia \(\sim\) la relazione di equivalenza
        le cui classi sono i laterali di \(H\) in \(G\).
        Mostriamo che \(\sim\) è una congruenza:
        se \(x\sim x'\) e \(y\sim y'\), allora \(xy\sim x'y'\).
        Sappiamo che \(x\in Hx'\) e \(y \in Hy'\)
        e dobbiamo mostrare che \(xy \in Hx'y'\).
        Allora \(x = h_x x'\) con \(h\in H\) e \(y = h_y y'\).
        Allora, \(xy = h_x x' h_y y'\). Concentriamoci
        sul termine \(x'h_y\): \(x'h_y \in xH = Hx'\),
        cioè \(x'h' = \overline{h} x\). Dunque,
        \[
            xy = h_x \overline{h} x' y' \in Hx'y'
        \]
    }
}

Dunque se \(H \unlhd G\), l'operazione nel gruppo quoziente \(G/H\) è così definita
\[
    Hx \cdot Hy \triangleq Hxy
\]
Se \(H\) non è normale, l'operazione non è ben-definita (se lo fosse avrei una congruenza la cui classe
di \({[1]}_\sim\) dovrebbe essere un sottogruppo normale).

\sproposition{Alcune proprietà}{
    \begin{enumerate}
        \item se \(G\) è abeliano, allora \(G/H\) è abeliano;
        \item se \(G\) è ciclico di generatore \(g\), allora \(G/H\) è ciclico
        con generatore \(Hg\) (infatti, per ogni \(x\in G\) si ha \(x =g^n\)
        per \(n\in\mathbb{N}\) e quindi \(Hx = {(Hg)}^n\)).
        Non vale necessariamente il viceversa.
    \end{enumerate}
}

\sproposition{}{
    Sia \(G\) un gruppo e \(H \unlhd G\),
    la funzione \(\varphi \colon G \to G/H\) che manda
    \(x\) in \(Hx\), è un omomorfismo suriettivo di nucleo \(H\).
    Tale omomorfismo è detto omomorfismo canonico.
}

\sproof{}{
    Chiaramente \(\varphi\) è suriettivo (ogni laterale proviene dai suoi elementi).
    Se \(x,y\) sono elementi di \(G\) si ha che \((xy)\varphi = Hxy = Hx \cdot Hy = x\varphi \cdot y\varphi\).
    Abbiamo anche
    \[
        \text{ker}_\varphi = \{x \in G \,|\, x\varphi = 1_{G/H}\}
        = \{x\in G \,|\, Hx = H\} = H
    \]
}

Quindi, dato un sottogruppo normale c'è almeno un omomorfismo di cui lui è il nucleo: l'omomorfismo
canonico sul quoziente.

I sottogruppi normali di \(G\) sono tutti e soli i nuclei di omomorfismi da \(G\)
in qualche gruppo.

\sproposition{}{
    Se \(G\) è un gruppo non-abeliano,
    allora \(G / Z(G)\) non è ciclico.
}

\sproof{}{
    Per assurdo sia \(G/Z(G)\) ciclico, generato da un centro
    \(Z(G)g\). Per ogni \(x\in G\), si ha allora
    \[
        Z(G)x = {(Z(G) g)}^n
    \]
    per qualche \(n\in\mathbb{N}\), cioè \(x \in {(Z(G) g)}^n = Z(g)g^n\),
    cioè \(x = zg^n\) per qualche \(z\in Z(G)\) e \(n\in\mathbb{N}\).
    Se ora \(y\) è un altro elemento di \(G\), abbiamo che \(y=z'g^m\)
    per qualche \(z'\in Z(G)\) e \(m\in\mathbb{N}\).
    Ora \(xy = zg^nz'g^m = zz'g^{n+m}\) e \(yx = z'g^mzg^n = zz'g^{n+m}\) quindi \(xy=xy\).
    Quindi \(G\) è abeliano lightning.
}

\scorollary{}{
    Dato un gruppo \(G\), l'indice % non abeliano?
    \(|G\,:\,Z(G)|\) non è primo.
}

\sproof{}{
    Se l'indice fosse primo, il quoziente \(G /Z(G)\) avrebbe ordine primo e sarebbe dunque ciclico.
    Allora, \(G\) sarebbe abeliano e \(|G\,:\,Z(G)|\) sarebbe \(1\).
}

\scorollary{}{
    Sia \(G\) di ordine qudrato di un primo \(p\).
    Allora, \(G\) è abeliano.
}

\sproof{}{
    Sappiamo che \(Z(G)\) ha ordine che divide \(|G| = p^2\).
    Le possibilità sarebbero \(|Z(G)| = 1,p, p^2\).
    Tuttavia, abbiamo dimostrato che in un p-gruppo non banale
    il centro non è banale.
    Dunque, non è 1.
    Se fosse \(|Z(G)| = p\), avremmo allora
    \[
        |G\,:\, Z(G)| = \frac{|G|}{|Z(G)|} = \frac{p^2}{p} =p
    \]
    contro il corollario precedente.
    Allora, \(|Z(G)| = p^2\) cioè \(G = Z(G)\) cioè \(G\) è abeliano.
}

% esercizi 5 dicembre
%nota: le liste di coniugi date, sono unioni di classi di coniugi

\subsection{Classificazione dei gruppi di ordine primo quadro}

Abbiamo già visto che sono tutti abeliani.
Gli elementi di un tale gruppo possono avere periodo \(1, p, p^2\). (Solo uno ha periodo \(1\))-

Se \(|G| = p^2\) ed esiste almeno un elemento di periodo
\(p^2\), allora \(G\) è ciclico di ordine \(p^2\).
Se \(|G| = p^2\) e non esistono elementi di periodo \(p^2\),
allora tutti gli elementi non banali di \(G\) hanno periodo \(p\).
Sia \(x\) un tale elemento e sia \(H = \langle x \rangle\).
Allora, \(H\) è ciclico di ordine \(p\) ed è normale in \(G\), in quanto \(G\) è abeliano.
Sia ora \(y \in G \backslash H\). Anche \(K = \langle y \rangle\)
è ciclico di ordine \(p\) ed è normale in \(G\).
Ora \(|H \cap K|\) è un divisore di \(|H|\) e \(|K|\)
cioè di \(p\). Tuttavia, non può essere \(p\) in quanto altrimenti sarebbero uguali,
e quindi \(H \cap K = 1\).
Ora,
\[
    |HK| = \frac{|K||H|}{|H \cap K|} = \frac{p^2}{1} = p^2
\]
Quindi, \(HK = G\).
Riassumento \(G\) è prodotto di due ciclici normali
di periodo \(p\) con intersezione banale,
cioè \(G\) è il prodotto diretto di due ciclici di ordine \(p\).

\subsection{Prodotto semidiretto}

\sexample{}{
    Consideriamo \(\text{Sym}_n\) con \(n \geq 2\)
    e sia \(H = \langle (1\; 2) \rangle\)
    e \(N = A_n\).
    Ora \(A_n \unlhd \text{Sym}_n\),
    \(H \cap N = \text{Id}\):
    infatti \(H = \{\text{Id}\}\) e \((1\;2) \notin A_n\).
    Inoltre
    \[
        |HA_n| = \frac{|H||A_n|}{|H \cap A_n|} = \frac{2\cdot \frac{n!}{2}}{1}
    \]
    cioè \(HA_n = \text{Sym}_n\). \\
    Notiamo che se \(n \geq 3\), allora \(H\) non è normale nel simmetrico,
    perché i coniugati dello scambio di \((1\;2)\) sono tutti e soli gli scambi
    \((i\;j)\) e questi non stanno tutti in \(H\).
    Vogliamo costruire un gruppo non abeliano da due gruppo abeliani (addirittura anche ciclici).
}

Qual'è la differenza fra prodotto diretto e semidiretto?
Se \(G = H \times N\) allora il prodotto di due elementi di \(G\)
è semplice. Se \(g_1 = h_1n_1\) e \(g_2 = h_2n_2\),
allora \(g_1g_2 = h_1h_2n_1n_2\).
Se \(G = H \ltimes N\), allora ho comunque una decomposizione unica per ogni elemento di \(G\),
cioè per ogni \(g\in G\) esistono e sono unici \(h\in H\) e \(n\in N\)
tale che \(g = hn\)
(qui non serve la normalità ma solo intersezione banale e prodotto uguale al gruppo).
Se però \(g_1 = h_1 n_1\) e \(g_2 = h_2n_2\) non posso più garantire che
\(g_1g_2 = h_1h_2n_1n_2\).
Questo fatto è ciò che mi dà più libertà nella costruzione.
La conoscenza dei prodotto in \(H\) e \(N\) non è sufficiente a descrivere
il prodotto in \(G\).

\sexample{}{
    Abbiamo visto che \[
        \text{Sym}_3 = \langle (1\;2)\rangle \ltimes A_3 \cong C_2 \ltimes C_3
    \]
    Ma \(C_6 = C_2 \times C_3\) che è un caso particolare di \(C_2 \ltimes C_3\).
    Quindi a partire dagli stessi strumenti possiamo arrivare a due cose differenti.
}

Nel caso generale, so che \(g_2g_2 = hn\)
per \(h\in H\) e \(n\in N\).
Come calcolo \(h\) e \(n\)? Non è detto che \(h=h_1h_2\)
e \(n=n_1n_2\) (ciò vale per tutti se il prodotto è diretto).
Abbiamo
\begin{align*}
    g_1g_2 = h_1n_1h_2n_2 = hn
\end{align*}
Questo prodotto è unico anche se uno dei due sottogruppi non è normale,
ma in questo caso uno dei due lo è e possiamo usare questa informazione.
Scriviamo infatti
\[
    h_1 n_1 h_2 n_2 = h_1h_2h_2^{-1} n_1 h_2 n_2
    = h_1 h_2 n_1^{h_2} n_2
\]
abbiamo allora un prodotto di un elemento in \(H\) e uno in \(N\).
Quindi \(h = h_1h_2\) e \(n = n_1^{h_2}n_2\).
Pertando, conosco comde moltiplicare in \(G\)
se so come moltiplicare in \(H\) e \(N\)
e so come coniugare elementi di \(N\) tramite elementi
di \(H\).
Nel caso particolare del prodotto diretto, questo coniugio è banale.
Quindi, cosa vuol dire dare questa il coniugio
di elementi di \(N\) traimite elementi di \(H\)?

Ricordiamo che in un gruppo \(G\) il coniugio
tramite un elemento \(g\in G\) definisce un automorfismo
di \(G\) e chese \(N \unlhd G\), il coniugio tramite \(g\)
definisce un'automorfismo di \(N\).
Ho cioè una funzione che manda elementi di \(g\)
in automorfismi di \(N\).
Precisamente,
\[
    g\to (n \to n^g)
\]
Inoltre, se \(h\in G\) è un altro elemento, abbiamo
\(h \to (n \to n^h)\).
Abbiamo allora
\[
    gh \to (n \to n^{gh})
\]
Se indichiamo con \(\varphi_g\) l'automorfismo
di \(N\) che manca \(n\) in \(n^g\)

\begin{center}
% https://tikzcd.yichuanshen.de/#N4Igdg9gJgpgziAXAbVABwnAlgFyxMJZABgBpiBdUkANwEMAbAVxiXBAF9T1Nd9CUARnJVajFmzAA9AOaduIDNjwEiAJhHV6zVohAAKaTICUUgBadRMKDPhFQAMwBOEALZJhIHBCQaQDOgAjGAYABV4VAX8YBxwQLXFdEAAdZPonNDMsAAILLkcXd0QyLx9ET20JPVT0zJy5fJBnNyQS719qYLAoJABaAGYSgOCwiP42JywZMziEnTYaugysgH1gaY5LDiA
\begin{tikzcd}[scale=2]
    n \arrow[r, "\varphi g"] \arrow[rr, "\varphi_{gh}"', bend right] & n^g \arrow[r, "\varphi h"] & (n^g)^h
    \end{tikzcd}
\end{center}
ma questo è equivalente a
\[
    h^{-1} n^g h = h^{-1} g^{-1} ngh = n^{gh}
\]
Dunque \(\varphi_g \varphi_n = \varphi_{gh}\):
abbiamo cioè un omomorfismo di gruppi da \(G\) in \(\text{Aut}(N)\).
Nel caso \(G = H \ltimes N\)
per dire come gli elementi di \(N\) sono coniugati da elementi di \(H\)
mi basta conoscere la restrizione di questo omomorfismo ad \(H\).
Cioè, abbiamo un omomorfismo di gruppi da \(H\) in \(\text{Aut}(N)\).
Se chiamiamo \(\varphi\) tale isomorfismo, abbiamo
\[
    h_1n_1h_2n_2 = h_1h_2n_1^{h_2} n_2 = h_1h_2 n_1 (h_2 \varphi)n_2
\]
Il termine \(h_2\varphi\) è un automorfismo.

Notiamo che nel caso di prodotto diretto
\(h_2 \varphi = \text{Id}_n\) per ogni \(h_2 \in H\).
Vediamo come fare il viceversa

\sdefinition{Prodotto semidiretto esterno}{
    Siano \(H\) e \(N\) due gruppi e sia \(\varphi\colon H \to \text{Aut}(N)\)
    un omomorfismo di gruppi. Nel prodotto cartesiano
    \(H \times N\) definiamo l'operazione nel modo seguente:
    \[
        (h_1, n_1)(h_2, n_2) \triangleq (h_1h_2, n_1(h_2 \varphi) n_2)
    \]
    Il prodotto semidiretto esterno è denotato
    \[
        H \ltimes_\varphi N
    \]
}

Dagli stessi gruppi di partenza potrei avere \(\varphi\) diverse e ottenere
prodotti diversi, anche non isomorfi fra di loro.

\stheorem{Il prodotto semidiretto esterno è un gruppo}{
    Il prodotto semidiretto esterno è un gruppo
    \(H \ltimes_\varphi N\) è un gruppo \(G\). Inoltre,
    \(G\) contiene un sottogruppo \(H' \cong H\) e
    un sottogruppo normale \(N' \cong N\)
    tale che \(G = H'\ltimes N'\) (prodotto semidiretto interno)
}

\sproof{Il prodotto semidiretto esterno è un gruppo}{
    Per semplicità scriviamo \(\varphi_h\) per indicare l'automorfismo
    di \(N\) immagine di \(h\) tramite \(\varphi\).
    Quindi \[
        \varphi_{h_1}\varphi_{h_2} =
        (h_1 \varphi)(h_2 \varphi) = (h_1h_2)\varphi_{h_1h_2}
    \]
    allora
    \[
        (h_1, n_1)(h_2, n_1) = (h_1h_2, n_1\varphi_{n_2} n_2)
    \]
    Verifichiamo le proprietà
    \begin{enumerate}
        \item \emph{associativà:} prendere dal libro;
        \item \emph{associativà:} prendere dal libro
    \end{enumerate}
}

\pagebreak

\section{Teorema di Sylow}

Se \(|G| = n\) e \(d\) divide \(n\), esiste un sottogruppo di ordine \(d\)?
Nei gruppi ciclici sì: per ogni \(d\) che divide \(|G|\) esiste un unico sottogruppo di ordine \(d\).
In generale no, il più piccolo esempio è \(A_4\)
dove non esistono sottogruppi di ordine \(6\).
Infatti, se esistesse \(H \leq A_4\)
con \(|H| = 6\), avremmo \(|A_4 \,:\, H| = 2\)
e \(H\) dovrebbe essere normale, ed essere unione di classi di coniugio di \(A_4\).
Ma in \(A_4\) c'è una classe di ordine \(1\), 2 classi di ordine \(4\),
e una classe di ordine \(3\). Non è possibile far uscire \(6\), non ci sono nemmeno candidati.

\stheorem{}{
    Sia \(G\) un gruppo abeliano finito
    di ordine \(n\) e sia \(d\) un divisore di \(n\).
    Allora, in \(G\) esiste un sottogruppo di ordine \(d\) (non necessariamente unico).
}

Infatti, se per ogni divisore ce n'è esattamente uno, allora il gruppo è ciclico.
Esercizio.

\sproof{}{
    Consideriamo prima il caso in cui \(d\) è primo \(p\), e procediamo per induzione
    su \(\frac{n}{p}\):
    \begin{itemize}
        \item la base è \(\frac{n}{p} = 1\) cioè \(n=p\), \(G\) stesso ha ordine \(p\).
        \item sia ora \(\frac{n}{p} > 1\) cioè \(n>p\). Prendiamo
        un elemento non banale \(y \in G\) e consideriamone il periodo \(|y| = m\).
        Distinguiamo due casi: se \(p\) divide \(m\), abbiamo finito perché l'elemento
        \(y^{\frac{m}{p}}\) ha periodo \(p\) e quindi genera un sottogruppo di ordine \(p\).
        Se \(p\) non divide \(m\), considero il quoziente
        \[
            \frac{G}{\langle y \rangle}
        \]
        (qui usiamo il fatto che \(G\) è abeliano e quindi il sottogruppo è normale).
        Ora, questo quoziente ha ordine \(\frac{n}{m} < n\) in quanto \(m > 1\).
        Tuttavia, tale ordine è multiplo di \(p\).
        Per ipotesi induttiva, tale quoziente contiene un elemento \(x\langle y \rangle\)
        di periodo \(p\). Ora consideriamo \(x^m = z\).
        Poiché \(p\) non divide \(m\), \[
            z \langle y \rangle = x^m \langle y \rangle
            = {(x \langle y \rangle)}^m \neq 1_{G / \langle y \rangle}
        \]
        (se dividesse \(m\) farebbe \(1\)).
        In particolare, \(z \neq 1\). Ora \(z^p = {(x^m)}^p = {(x^p)}^m\).
        Ma \(|x \langle y \rangle| = p\) segue che \(x^p \in \langle y \rangle\).
        Poiché \(|\langle y\rangle| = m\), segue che tutti i suoi elementi elevati alla \(m\)
        danno \(1\), quindi \(z^p = 1\). Ma siccome \(z \neq 1\), allora \(|z| = p\). \\
    \end{itemize}
    Consideriamo ora il caso generale e procediamo per induzione completa su \(d\).
    Il caso in cui \(d=1\) è banale.
    Se \(d>1\), sia \(p\) un divisore di \(d\),
    e sia \(H\) un sottogruppo di \(G\)
    di ordine \(p\). Consderiamo il quoziente \(G/H\) che posso fare in quanto è abeliano.
    Allora l'ordine di tale quoziente è \(n/p\) e \(d/p\) divide \(n/p\).
    Per ipotesi induttiva, \(G/H\) contiene un sottogruppo \(K/H\)
    (per il teorema di isomorfismo) di ordine \(d/p\).
    Ma allora, \(K\) è un sottogruppo di \(G\) di ordine \(d\) (teorema di isomorfismo).
}

% Lemma di Cauchy

\stheorem{}{
    Sia \(G\) un \(p\)-gruppo di ordine \(p^n\).
    Per ogmi \(0 \leq k \leq n\), esiste un sottogruppo normale
    in \(G\) di ordine \(p^k\).
}

\sproof{}{
    Procediamo per induzione su \(k\)
    \begin{enumerate}
        \item il caso banale è \(k=0\) e \(|1| = 1 = p^0\) e \(1\unlhd G\);
        \item se \(k>0\), allora anche \(n>0\). Sappiamo che \(Z(G) \neq 1\).
        Se \(|Z(G)| = p^t\) con \(t\geq k\), poiché \(Z(G)\) è abeliano,
        sappiamo che \(Z(G)\) contiene un sottogruppo di ordine \(p^k\).
        Ma i sottogruppi del centro sono normali in \(G\), quindi abbiamo un sottogruppo
        normale in \(G\) di oridne \(p^k\).
        Se invece \(|Z(G)| = p^t\) con \(t<k\), considero il quoziente
        \[
            \frac{G}{Z(G)}
        \]
        che ha ordine \(p^{n-t}\).
        Questo contiene un sottogruppo normale
        \[
            \frac{N}{Z(G)}
        \]
        di ordine \(p^{k-t}\) e \(0 < k -t \leq n-t\), dove uso l'ipotesi induttiva.
        Per uno dei teoremi di isomorfismo, il terzo,
        \(N \unlhd G\) e
        \[
            |N| = \left|\frac{N}{Z(G)}\right| \cdot
            |Z(G)| = p^{k-t} \cdot p^t = p^k
        \]
    \end{enumerate}
}

\pagebreak

\section{Esercizi}

\sexercise{}{
    Sia \(G = C_{p^3} \times C_{p^2}\) con \(p\) primo.
    \begin{enumerate}
        \item quanti sottogruppi di ordine \(p\) ha \(G\)?
        \item quanti sottogruppi ciclici di ordine \(p^2\) ha \(G\)?
        \item quanti sottogruppi non ciclici di ordine \(p^2\) ha \(G\)?
    \end{enumerate}
}

\sproof{}{
    \begin{enumerate}
        \item ciascun gruppo di ordine \(p\) ciclico contiene \(p-1\)
        elementi di periodo \(p\) e ognuno di questi elementi
        appartiene ad un unico sottogruppo di ordine \(p\).
        Per contare i sottogruppi di ordine \(p\) posso allorare contare
        gli elementi di periodo \(p\) e dividere per \(p-1\).
        Un elemento \((x,y)\) nel prodotto diretto esterno ha periodo
        pari al minimo comune multiplo dei due periodi.
        Quindi, per far sì che il periodo sia \(p\),
        o entrambi sono \(p\) o uno è \(1\) e l'altro è \(p\).
        In \(C_{p^3}\) c'è un unico sottogruppo di ordine \(p\)
        che è formato da tutti gli elementi di periodo che divide \(p\).
        Dunque, abbiamo \(p\) possibilità per \(x\) in modo tale che
        \(|x|\) divide \(p\). Lo stesso vale per \(C_{p^2}\).
        Ma allora, ci sono \(p\cdot p\) coppie
        \((x,y) \in G\) tale che \(|x|\) e \(|y|\) dividano \(p\).
        Siccome almeno uno deve avere periodo \(p\), bisogna togliere
        il caso \(|x| =1\) e \(|y| = 1\), cioè \((1,1)\), quindi il risultato è \(p^2 - 1\)
        elementi di periodo \(p\), e quindi vi sono
        \[
            \frac{p^2 - 1}{p-1} = p+1
        \]
        sottogruppi di oridne \(p\).
        \item ci sono \(\varphi(p^2) = p^2-p\) elementi di periodo \(p^2\), 
        ciascuno in un unico elemento di ordine \(p^2\).
        Quindi sia \(|x|\) e \(|y|\) devono dividere \(p^2\)
        e almeno uno di essi è esattamente \(p^2\).
        Nel \(C_{p^3}\) c'è un unico sottogruppo di ordine \(p^2\) che contiene tutti gli elementi
        di periodo che divide \(p^2\). Analogamente nell'altro.
        Quindi ho \(p^2 \cdot p^2\) elementi \((x, y)\) con \(|x|\) e \(|y|\)
        che divide \(p^2\). Di questi ne togliamo \(p^2\),
        cioè quelli per cui \(|x|\) e \(|x|\) dividono\(p\). Otteniamo
        \[
            \frac{p^4 - p^2}{p^2 - p} = p^2 + p
        \]
        gruppi ciclici di ordine \(p^2\).
        \item Un gruppo di ordine \(p^2\) non ciclico contiene
        \(p^2 - 1\) elementi di periodo \(p\) e un elemento di periodo 1.
        Poiché ho esattamente \(p^2 - 1\) elementi di periodo \(p\) in \(G\),
        ho un'unica possibilità, cioè ho al massimo un sottogruppo non ciclico di ordine
        \(p^2\). Ma un sottogruppo siffatto esiste.
        Prendo l'unico sottogruppo \(H\) di \(C_{p^3}\) di ordine \(p\)
        e l'unico sottogruppo \(K\) di \(C_{p^2}\) di oridne \(p\).
        Ora \(H \cap K = 1\). Quindi, \[
            |HK| = \frac{|H|\cdot|K|}{|H \cap K|} = p^2
        \]
        e \(HK\) è un sottogruppo di ordine \(p^2\) non ciclico (perché contiene gli elementi
        \((x,y)\) con \(x\in H\) e \(y\in K\), cioè \(|x|\) e \(|y|\) dividono \(p\)).
    \end{enumerate}
}

\sexercise{}{
    Sia \(G = \langle a \rangle\)
    con \(|G| = 120\). Siano \(H = \langle a^{33} \rangle\)
    e \(K = \langle a^{28} \rangle\) sottogruppi.
    \begin{enumerate}
        \item trova \(|H|\)
        \item trova \(|H\cap K|\)
        \item esiste \(L \leq G\) tale che \(G = H \times L\)?
        \item esiste \(M \leq G\) tale che \(G = K \times M\)?
    \end{enumerate}
}

\sproof{}{
    \begin{enumerate}
        \item L'ordinato è dato da
        \[
            |H| = |a^{33}| = \frac{120}{\gcd(120, 33)} = 40
        \]
        e
        \[
            |K| = |a^{28}| = \frac{120}{\gcd(120, 28)} = 30
        \]
        \item L'ordine deve dividere sia \(40\) che \(30\) quindi \(10\).
        Vi è solamente un sottogruppo in \(G\) di ordine \(10\)
        e \(H\) e \(K\) contengono ciascuno un sottogruppo di oridne \(10\)
        (tutti e 3 sono il medesimo).
        \item Se esistesse \(L\) come cercato, dovrei avere
        \(|G| = 120 = |H| \cdot |L|\), quindi \(|L| = 3\).
        In \(G\) vi è un unico sottogruppo di ordine \(3\),
        ossia \(L = \langle a^{\frac{120}{3}} \rangle\).
        Vediamo se \(G = H\times L\): la normalità è soddisfatta in quanto siamo in un gruppo ciclico quindi abeliano.
        L'intersezione è banale in quanto \(H \cap L\)
        ha ordine che divide \(|H| = 40\) e \(|L| = 3\).
        Dunque, è 1.
        Per vedere se il prodotto funziona, calcoliamo l'oridne del prodotto
        \[
            |HL| = \frac{|H| \cdot |L|}{|H \cap L|} = \frac{40 \cdot 3}{1} = 120
        \]
        Dunque \(G = H \times L\).
        \item \(M\) dovrebbe avere ordine \(4\).
        L'unico candidato è \(M = \langle a^{30} \rangle\).
        L'intersezione non è banale in quanto \(30\) e \(4\) non sono coprimi,
        quindi non vi è soluzione.
    \end{enumerate}
}

\end{document}
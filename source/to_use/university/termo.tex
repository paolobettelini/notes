\documentclass[a4paper]{article}

\usepackage{amsmath}
\usepackage{amssymb}
\usepackage{stellar}
\usepackage{parskip}
\usepackage{fullpage}
\usepackage{wrapfig}

\title{Termodinamica}
\author{Paolo Bettelini}
\date{}

\begin{document}

\maketitle
\tableofcontents

\section{Termodinamica}

Nello scritto potrebbe esserci la domanda sulla trasformazione adiabatic di un gas perfetto.

La teoria cinetica dei gas è un modello microscopico classico di un gas perfetto.
È il più semplice modello statistico che mette in relazione le proprietà
termodinamiche di un gas all'equilibrio
con le proprietà microscopiche della molecole del gas.
Le ipotesi semplificatrici sono:
\begin{enumerate}
    \item il gas è formato da un numero grande di molecole \(N\)
        in continuo movimento (molto disordindinato);
    \[
        N \sim N_A = 6 \cdot 10^{23}
    \]
    \item Le molecole del gas sono trattate come puntoformi, prive di struttur a interna,
    e soggette alla meccanica Newtoniana.
    \item le molecole in moto nel gas subiscono urti tra loro e con le pareti del contenitore
    con urti perfettamente elastici (si conserva quindi l'energia cinetica).
    Tra un orto e quello successivo le molecole non interagiscono e si muovono di rettilineo uniforme.
    \item La distriuzione spaziale delle molecole è uniforme (mediante)
        e la loro velocità non hanno una direzione preferenziale
        (si ha cioè una distribuzione isotropa della velocità molecolari).
        Queste condizioni sono sicuramente soddisfatte in condizione di equilibrio
        termodinamico.
    \item Le dimensioni delle molecole sono trascurabili rispetto alla XXX
    mmedia che separa le molecole \(\iff\) il volume \(V\) occupato
    dal gas (il volume del contenitore del gas).
    Per pressioni non eccessive tale condizione è soddisfatta (sono le condizioni in cui
    vale l'uguaglianza di stato di un gas perfetto).
\end{enumerate}

Consideriamo \(N\) particelle puntiformi (la molecola del gas)
racchiude in un contenitore cubico di lato \(L\) dove i lati del cubo sono paralleli agli assi
dello spazio Consideriamo la parete \(1\).
Si vuole valutare la forza media esercitata dalle molecole sul tale parete attraverso gli urti.
Consideriamo una singola molecola \(i\)-esima che si schianta
nel punto \(x=L\) con velocità \(\vec{v}_i\). Dopo l'urto si inverte solamente la componente
\(x\) della velocità e il modulo rimane invariato.
Sia \(\delta \vec{q}\) la quantità di moto.
\begin{align*}
    \Delta \vec{q} &= m\vec{v}_f - m\vec{v}_i \\
    &= m(\vec{v}_f - \vec{v}_i) \\
    &= -2m v_{x,i} \hat{x, i}
\end{align*}
Dopo l'urto, per tornare sulla parete \(1\) la molecola impiega
\[
    \Delta t = \frac{2L}{v_{x,i}}
\]
per tornare sulla parete. Invertendo tale dato troviamo anche la frequenza degli urti.
Stiamo ignorando urti fra le molecole, ma possiamo giustificarlo dicendo che se due
molecole di scontrano mediante è come se si scambiassero i ruoli.
In un intervallo di tempo \([t_A, t_B]\) si hanno \(\frac{t_B - t_A}{\Delta t}\)
urti con la parete \(1\).
Siccome la quantità di moto si conserva la parete riceve dalla molecole
\begin{align*}
    \Delta \vec{q}_p &= - \frac{t_B - t_A}{\Delta t} \Delta \vec{q} \\
    &= (t_B - t_A) m \frac{{(\vec{v}_{x,i})}^2}{L} \hat{x}
\end{align*}
quantità di moto.
Per trovare la pressione usiamo il teorema dell'impulso
\begin{align*}
    \frac{\Delta \vec{q}_p}{dt} &= \integral[t_A][t_B][\vec{F}_p][t] \\
    &= \frac{1}{t_B - t_A} \integral[t_A][t_B][\vec{F}_p][t] (t_B - t_A) \\
    &\triangleq \vec{F}_m (t_B - t_A)
\end{align*}
dove \(\vec{F}_m\) è la forza media sulla parete (1) della particelle \(i\)-esima.
Quindi
\begin{align*}
    \vec{F}_m &\triangleq\frac{1}{t_B - t_A} \integral[t_A][t_B][\vec{F}_p][t] \\
    &= \frac{\Delta \vec{q}_p}{t_B - t_A} \\
    &= m \frac{{(\vec{v}_{x,i})}^2}{L} \hat{x}
\end{align*}
Per ottenere la forza totale (macroscopica) devo sommare il contributo di tutte le particelle, per questo abbiamo
usato la media.
\begin{align*}
    \vec{F} &= \sum_{i=1}^N m \frac{{(\vec{v}_{x,i})}^2}{L} \hat{x} \\
    &= \frac{M}{L} N \left(\frac{1}{N} \sum_{i=1}^N {(v_{x,i})}^2\right) \hat{x} \\
    &= \frac{m}{L} N \overline{v_x^2} \hat{x}
\end{align*}
dove
\[
    \overline{v_x^2} \triangleq \frac{1}{N} \sum_{i=1}^N {(v_{x,i})}^2
\]
è il valore medio del quadrato della component \(x\) della velocità molecolare.
Per determinare queste medie è necessario conoscere la distribuzione delle velocità molecolari;
indicheremo con \(f_x(v_x)\) la distribuzione di probabilità (PDF).
In uno stato di equilibrio termodinamico \(f_x(v_x)\) non dipende dal tempo.
Inoltre, siamo nell'ipotesi di isotropia della velocità. Le distribuzioni
delle altre componenti sono analoghe e uguali.
Abbiamo anche
\[
    f(-v_x) = f(v_x)
\]
Chiaramente il valore
\[
    \integral[v_x'][v_x''][f(v_x)][v_x]
\]
è la probabilità di trovare \(v_x\) in \([v_x', v_x'']\).
La funzione \(f\) permette di valutare i diversi momenti. Il momento di ordine \(n\) è
\[
    \overline{v_x^n} = \integral[-\infty][+\infty][f(v_x)v_x^n][v_x]
\]
(il momento di oridne \(1\) è la media). La media, come tutti i momenti dispari,
è zero, siccome la funzione è dispari.
\[
    \overline{v_x} = \integral[-\infty][+\infty][f(v_x)v_x][v_x] = 0
\]
l'integranda è una funzione pari per una funzione dispari. Quindi il prodotto è dispari
e la media è nulla.

\end{document}
\documentclass[a4paper]{article}

\usepackage{amsmath}
\usepackage{amssymb}
\usepackage{stellar}
\usepackage{parskip}
\usepackage{fullpage}
\usepackage{wrapfig}
\usepackage{tikz}

\usetikzlibrary{arrows}
\usetikzlibrary{decorations.pathreplacing}
\usetikzlibrary{cd}

\title{Category theory}
\author{Paolo Bettelini}
\date{}

\begin{document}

\maketitle
\tableofcontents

\section{XXX}

Se un oggetto iniziale esiste in una categoria allora è unico a meno di isomorfismo.
Se esistessero due oggetti iniziali \(0\) e \(0'\), allora ci deve essere
un morfismo fra \(0\) e \(0'\) e uno fra \(0\) e \(0'\).
Dalla definizione tale morfismo è isomorfismo.
Il duale è il medesimo teorema con l'oggetto terminale.

Studiamo cosa sono gli oggetti iniziali nella categoria Sets.
Se consideriamo l'insieme vuoto, vi è una e una sola funzione che colla tale insieme
a tutti gli altri. Quindi l'insieme vuoto è l'oggetto iniziale della categroia Sets.

Invece, l'oggetto terminale è il singoletto della categoria Sets.
Infatti tutti i singoletti sono isomorfi fra loro.

L'uguaglianza degli insiemi diventa isomorfismo nelle categorie.
Possiamo dare un complementare dell'assioma dell'estensionabilità
nelle categoria, cioè due oggetti sono uguali se hanno gli stessi elementi generalizzati.

\subsection{Tipi di categorie}

La categoria Set, Top, Gr, Rng, Vect/K, T-mod(Set).

Possiamo costruire la categoria discreta data un insieme, i cui oggetti sono gli elementi
e le cui freccie sono solo le identità.

Possiamo fare una categoria da un preordine.
Essa ha al massimo un morfismo fra due oggetti distinti.
Esercizio: queste categorie sono tutte quelle indotte da un preordine.

Possiamo fare una categoria con un singolo oggetti (monoide).

Possiamo fare un groupoide: una categoria con soli isomorfismi.
In particolare se un grupoide ha solo un oggetto allora è un gruppo.

La categoria Cat è la categoria di tutte le categorie (piccole)
dove i morfismi sono funtori

Mettere l'esempio del funtore duale fra spazi vettoriali.
Il doppio duale è una trasformazione naturale.

\end{document}
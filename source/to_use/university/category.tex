\documentclass[a4paper]{article}

\usepackage{amsmath}
\usepackage{amssymb}
\usepackage{stellar}
\usepackage{parskip}
\usepackage{fullpage}
\usepackage{wrapfig}
\usepackage{tikz}

\usetikzlibrary{arrows}
\usetikzlibrary{decorations.pathreplacing}
\usetikzlibrary{cd}

\title{Category theory}
\author{Paolo Bettelini}
\date{}

\begin{document}

\maketitle
\tableofcontents

\section{XXX}

Se un oggetto iniziale esiste in una categoria allora è unico a meno di isomorfismo.
Se esistessero due oggetti iniziali \(0\) e \(0'\), allora ci deve essere
un morfismo fra \(0\) e \(0'\) e uno fra \(0\) e \(0'\).
Dalla definizione tale morfismo è isomorfismo.
Il duale è il medesimo teorema con l'oggetto terminale.

Studiamo cosa sono gli oggetti iniziali nella categoria Sets.
Se consideriamo l'insieme vuoto, vi è una e una sola funzione che colla tale insieme
a tutti gli altri. Quindi l'insieme vuoto è l'oggetto iniziale della categroia Sets.

Invece, l'oggetto terminale è il singoletto della categoria Sets.
Infatti tutti i singoletti sono isomorfi fra loro.

L'uguaglianza degli insiemi diventa isomorfismo nelle categorie.
Possiamo dare un complementare dell'assioma dell'estensionabilità
nelle categoria, cioè due oggetti sono uguali se hanno gli stessi elementi generalizzati.

\subsection{Tipi di categorie}

La categoria Set, Top, Gr, Rng, Vect/K, T-mod(Set).

Possiamo costruire la categoria discreta data un insieme, i cui oggetti sono gli elementi
e le cui freccie sono solo le identità.

Possiamo fare una categoria da un preordine.
Essa ha al massimo un morfismo fra due oggetti distinti.
Esercizio: queste categorie sono tutte quelle indotte da un preordine.

Possiamo fare una categoria con un singolo oggetti (monoide).

Possiamo fare un groupoide: una categoria con soli isomorfismi.
In particolare se un grupoide ha solo un oggetto allora è un gruppo.

La categoria Cat è la categoria di tutte le categorie (piccole)
dove i morfismi sono funtori

Mettere l'esempio del funtore duale fra spazi vettoriali.
Il doppio duale è una trasformazione naturale.

\section{Funtori}

Date due categorie \(\mathcal{C},\mathcal{C}'\)
possiamo definire TODO
la categoria \([\mathcal{C}, \mathcal{C}']\)
dei funtori da \(\mathcal{C}\) a \(\mathcal{C}'\)
dove gli oggetti sono i funtori da \(\mathcal{C}\) a \(\mathcal{C}'\),
i morfismi sono le trasformazioni naturali, e la composizione è componente per componente.

\sexercise{}{
    Una trasformazione naturale
    è un isomorfismo naturale se e solo se
    tutte le sue componenti sono degli isomorfimsmi nella categoria d'arrivo.

    %%%%%%%%%
    Dalla definizione \(\alpha\) ammette un inverso nella categoria \textbf{Cat}.
    Cioè due funtori sono naturalmente isomorfi se sono oggetti isomorfi in \([C, C']\).
    L'esercizio richiede di costruire una trasformazione naturale inversa e verificare che sia ancora naturale.
    Quindi esiste \(\alpha'\) tale che \(\alpha \circ \alpha' = 1\) e \(\alpha' \circ \alpha = 1\).
    Se nel diagramma della trasformazione naturale inverto le freccie di \(\alpha\)
    la commutatività vale ancora, ma nell'altra direzione.
}

TODO definizione dull and faithful functors and subcategory.
Non definiamo una nozione di suriettività fra oggetti in quanto l'uguaglianza fra oggetti non è robusta.
Vogliamo non distinguere oggetti essenzialmente uguali. L'ugualignza diventa l'isomorfismo.
Infatti la suriettività essenziale usa un isomorfismo.

Nella definizione del funtore sottocategoria, il functore è sempre federale, ma non necessariamente full.

\sdefinition{Category equivalence}{
    TODO
    Anche qui non usiamo l'uguaglianza ma l'isomorfismo.
}

\stheorem{}{
    Under the axiom of choice a functor is part of an equivalence of categories if and only if
    it is full, faithful and essentially surjective.
}

\sproof{}{
    \iffproof{
        Trivial check. (Exercise) Non richiede l'assioma della scelta.
    }{
        Dobbiamo costruire un funtore \(G\) che sia un inverso.
        Quindi dobbiamo effettuare delle scelte.
    }
}

Questa nozione ci dice che usando AC possiamo definire un quasi-inverso,
che non è unico. Ma in generale i funtori per cui devo eseguire una scelta non sono
particolarmente interessanti.

% sga1 grothendick

Dato un funtore \(F\) possiamo considerare \(\text{End}(F)\)
che è il monoide delle trasformazioni naturali da \(F\) a \(F\).
In particolare, comprende gli automorfismi e quindi il gruppo
\(\text{Aut}(F)\). Nel contesto degli insiemi mancano
le componenti, che sono i morfismi (a parte l'identità).
Quindi questo contetto diventa banale.

Mettere i 3 esempi di functor categories.
La composizione della functor category è componente per componente

% https://tikzcd.yichuanshen.de/#N4Igdg9gJgpgziAXAbVABwnAlgFyxMJZABgBpiBdUkANwEMAbAVxiRADEQBfU9TXfIRQBGclVqMWbAOLdeIDNjwEiAJjHV6zVohAAJbuJhQA5vCKgAZgCcIAWyRkQOCElEgGWMDpBQITACMGVmoACxg6KCQwJgYGahw6LAY2SG8QTUkfAB1sxjRQujkrWwdEdxckdQ8vHz9A4IyQcMjo2PjnJJTdNJCJbTZcgJhE4pAbe0cE10Rq4bAoxABaAGYnLSldAApc-MKAAl24HEPs4cSASi26C-2AXlPzui2AY1vcl6xrF9O955umgw6MMGAAFfjKIQgaxYEyhHCGLhAA
\begin{tikzcd}
F \arrow[r, "\alpha", Rightarrow] \arrow[rr, "(\alpha \ast \beta)(a) = \beta(c) \circ \alpha(a)"', bend right] & G \arrow[r, "\beta", Rightarrow] & H
\end{tikzcd}

\section{3 Esempi}

Esempio 2:
Chiamiamo \(X\) l'insieme.
Visto che \(\tau\) è un endomorfismo di \(X\), prendiamo il prodotto
cartesiano e otteniamo una azione sinistra.
Se volessimo l'azione destra potremmo prendere il duale del monoide.

La condizione di naturalezza corrisponde con
la condizione che \(f\) sia una funzione equivariante, cioè rispetta
l'azione (M-equivariante).
\[
    m \ast' f(\ast) = f(m \ast x)
\]
cioè è compatibile con le due azioni,
di \(M\) su \(X\) e di \(M\) su \(Y\).

Quindi la categoria studiata è la categoria delle azioni sinistre
su insiemi e delle mappe M-equivariante fra loro.

Esempio 3:
Non abbiamo condizioni di naturalità nel senso che escludiamo
le identità che sono banali.
Le trasformazioni naturali corrispondono a famiglie
di funzioni

\sdefinition{Slice category}{
    La composizione è data dai morphism \(h\colon a \to b\)
    tale che il diagramma commuta
}

% https://tikzcd.yichuanshen.de/#N4Igdg9gJgpgziAXAbVABwnAlgFyxMJZABgBpiBdUkANwEMAbAVxiRDpAF9T1Nd9CKAEzkqtRizYAjLjxAZseAkQCMpFWPrNWiEAGMuYmFADm8IqABmAJwgBbJGRA4ISNeO1sAFrKu2HiO4uSCIekromviA29o7UwYihWuHRINQMdFIwDAAKfEqCINZYJl44hpxAA
\begin{tikzcd}
a \arrow[rr, "h"] \arrow[rd, "f"'] &   & b \arrow[ld, "g"] \\
                                   & c &                  
\end{tikzcd}

\sexercise{}{
    Show that for any set \(I\),
    set slice category \(\mathbf{Sets}/I\)
    is equivalent to the category \([I, \mathbf{Set}]\)
    (which is the disjoint union).

    Partiamo da una collezione di insiemi indicizzata e la mando
    nell'unione disgiunta degli \(A_i\) per formare la mappa canonica
    invertibile: \((x, i) \to i\) e nell'altra direzione,
    partendo da un insieme \(A\) con \(f \colon A \to I\),
    associo la controimmagine (che in questo caso posso fare in generale).
    Quindi prendo le fibre \(\{f^{-1}(i) \,|\, i \in I\}\).
    Dibbiamo verificare i dettagli.
}

C'è un equivalenza fra le categorie indicizzate e la nozione di fibrazione
(che sono dei funtori).
Il risultato che forma ciò è Grothendieck's equivalence between
indexed categories and fibrations.
Le indexed categories generalizzano i funtori \(I \to \mathbf{Sets}\),
dove \(I\) viene sostituita dalla categoria delle categorie piccole,
e i funtori dagli pseudofuntori.
Le fibrazioni generalizzano gli oggetti si \(\mathbf{Sets}/I\).

\section{Suboject}

è la generalizzazione categorica dei sottoinsiemi.
Anche qua il triangolo con i due monomorfismi e il morfismo sopra deve commutare.
Ciò è equivalente (piccola verifica esericizio)
al fatto che siano isomorfe come oggetti nelle categoria slice.

Dobbiamo considerare le classi di equivalenza altrimenti non riusciamo
ad identificare...

Dualizzando i monomorfismi otteniamo gli epimorfismi e quindi dualizzando
otteniamo i quozienti (su un categoria che deve essere esatta).

\end{document}
\documentclass[a4paper]{article}

\usepackage{amsmath}
\usepackage{amssymb}
\usepackage{stellar}
\usepackage{parskip}
\usepackage{fullpage}
\usepackage{wrapfig}
\usepackage{tikz}

\usetikzlibrary{arrows}
\usetikzlibrary{decorations.pathreplacing}
\usetikzlibrary{cd}

\title{Computazione I}
\author{Paolo Bettelini}
\date{}

\begin{document}

\maketitle
\tableofcontents

\section{Floating points}

L'insieme dei floating point è
\begin{align*}
    f(\beta, t, m, M) = \{0, \text{NaN}, \pm\infty\} \cup
    \left\{x = \text{sign}(x) \cdot \beta^e \sum_{i=1}^t y_i \beta^{-i} \ \middle|\ 
    t,y_i,m,M \in \mathbb{N}, y_1 \neq 0, -m\leq e \leq M \right\}
\end{align*}
Stimiamo ora l'errore relativo
\[
    \frac{|x-\tilde x|}{|x|}
\]
dove \(x\in\mathbb{R}\) e \(\tilde x \in f(\beta, t, m, M)\) è la sua rappresentazione migliore in un calcolatore.
Consideriamo \(x > 0\).
Chiaramente, se \(\tilde x \in \mathbb{R}\), allora \(|x-\tilde x| = 0\).
Altrimenti, \(x\in [a,b]\) dove \(a,b\in f\) e sono consecutivi in \(f\).
Quindi
\[
    |x-\tilde x| \leq \frac{b-a}{2}
\]
Abbiamo allora
\[
    a = \beta^e \sum_{i=1}^t y_i \beta^{-i}
\]
e
\[
    b = \beta^e \left( \sum_{i=1}^t y_i \beta^{-i} + \beta^{-t} \right)
    = a + \beta^{e - t}
\]
Quindi la differenza è data da
\[
    |x - \tilde x| \leq \frac{1}{2}\beta^{e-t}
\]
Dobbiamo ora minorare l'elemento normalizzante
\[
    |x| = \beta^e \sum_{i=1}^\infty y_i\beta^{-i} \geq \beta^e \cdot y_1 \beta^{-1} \geq \beta^{e-1}
\]
Abbiamo quindi
\[
    \frac{1}{|x|} \leq \beta^{1-e}
\]
Combinando i due risultati otteniamo
\[
    \frac{|x-\tilde x|}{|x|} \leq \frac{1}{2}\beta^{e-t} \beta^{1-e}
    = \frac{1}{2} \beta^{1-t} \triangleq u
\]
Allora \(u\) è la precisione macchina.

\section{Approssimazione di zeri di funzioni}

Sia \(f \in C_{[a,b]}\) tale che \(f(\alpha) = 0\). Vogliamo approssimare \(\alpha\) numericamente.

\section{Metodo di bisezione}

Possiamo applicare ricorsivamente il teorema degli zeri, quindi bisezione.
In questo caso la velocità è indipendente da \(f\) ma solo dipendente dalla grandezza
dell'intervallo. Terminiamo l'algoritmo quando \(|b-a| < \varepsilon\) che è la mia tolleranza.
L'errore relativo è \(|x-\alpha| < \varepsilon|\alpha|\).
Per trovare il numero di operazioni abbiamo
\begin{align*}
    |b_1 - a_1| = \frac{1}{2} |b_2 - a_1|, \cdots, |b_i - a_i| = \frac{1}{2^i}|b_i - a_i|
\end{align*}
Quindi servono
\[
    \lceil \log2 \left(\frac{|b-a|}{\varepsilon}\right) \rceil
\]
Il pro di questo metodo è quindi una convergenza globale ma come contro
abbiamo una convergenza lenta se l'intervallo è grande.

\section{Metodo di iterazione funzionale}

Si definiscono metodi numerici per generare la successione
\(\{x_k\}\) tale che possibilmente
\[
    \lim_{k} x_k = \alpha
\]
Si andranno a definire iterazioni funzionali della forma
\[
    x_{k+1} = g(x_k)
\]
con un \(x_0\) dato. Vogliamo convertire \(f(x)=0\) in un'equazione di punto fisso
\(x=g(x)\), e poi si definisce l'iterazione.
L'iterazione funzionale deve tuttavia convergere. Quindi, data \(g\)
sufficientemente regolare tale che \(\alpha = g(\alpha)\)
e definito lo schema di iterazione \(x_{k+1} = g(x_k)\) con \(x_0\) dato,
si vogliono definire condizioni necessarie e/o sufficienti per la convergenza
\[
    \lim_k x_k = \alpha
\]
La condizione necessaria è
\slemma{}{
    Sia \(g \in C([a,b])\) tale che \(x_0 \in [a,b]\)
    e \(x_{k+1} = g(x_k) \in [a,b]\) e
    \[
        \lim_k x_k = \alpha
    \]
    allora \(\alpha = g(\alpha)\).
}
\sproof{}{
    \[
        \alpha = \lim_k x_k = \lim_k {x_k+1} = \lim_k g(x_k)
    \]
}
La condizione sufficiente è il teorema delle contrazioni.
% serve C^1

% x = -f/f' -x_{k-1}

\end{document}
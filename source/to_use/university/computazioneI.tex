\documentclass[a4paper]{article}

\usepackage{amsmath}
\usepackage{amssymb}
\usepackage{stellar}
\usepackage{parskip}
\usepackage{fullpage}
\usepackage{wrapfig}
\usepackage{tikz}

\usetikzlibrary{arrows}
\usetikzlibrary{decorations.pathreplacing}
\usetikzlibrary{cd}

\title{Algebra I}
\author{Paolo Bettelini}
\date{}

\begin{document}

\maketitle
\tableofcontents

\section{Floating points}

L'insieme dei floating point è
\begin{align*}
    f(\beta, t, m, M) = \{0, \text{NaN}, \pm\infty\} \cup
    \left\{x = \text{sign}(x) \cdot \beta^e \sum_{i=1}^t y_i \beta^{-i} \ \middle|\ 
    t,y_i,m,M \in \mathbb{N}, y_1 \neq 0, -m\leq e \leq M \right\}
\end{align*}
Stimiamo ora l'errore relativo
\[
    \frac{|x-\tilde x|}{|x|}
\]
dove \(x\in\mathbb{R}\) e \(\tilde x \in f(\beta, t, m, M)\) è la sua rappresentazione migliore in un calcolatore.
Consideriamo \(x > 0\).
Chiaramente, se \(\tilde x \in \mathbb{R}\), allora \(|x-\tilde x| = 0\).
Altrimenti, \(x\in [a,b]\) dove \(a,b\in f\) e sono consecutivi in \(f\).
Quindi
\[
    |x-\tilde x| \leq \frac{b-a}{2}
\]
Abbiamo allora
\[
    a = \beta^e \sum_{i=1}^t y_i \beta^{-i}
\]
e
\[
    b = \beta^e \left( \sum_{i=1}^t y_i \beta^{-i} + \beta^{-t} \right)
    = a + \beta^{e - t}
\]
Quindi la differenza è data da
\[
    |x - \tilde x| \leq \frac{1}{2}\beta^{e-t}
\]
Dobbiamo ora minorare l'elemento normalizzante
\[
    |x| = \beta^e \sum_{i=1}^\infty y_i\beta^{-i} \geq \beta^e \cdot y_1 \beta^{-1} \geq \beta^{e-1}
\]
Abbiamo quindi
\[
    \frac{1}{|x|} \leq \beta^{1-e}
\]
Combinando i due risultati otteniamo
\[
    \frac{|x-\tilde x|}{|x|} \leq \frac{1}{2}\beta^{e-t} \beta^{1-e}
    = \frac{1}{2} \beta^{1-t} \triangleq u
\]
Allora \(u\) è la precisione macchina.

\end{document}
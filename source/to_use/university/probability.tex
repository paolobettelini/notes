\documentclass[a4paper]{article}

\usepackage{amsmath}
\usepackage{amssymb}
\usepackage{stellar}
\usepackage{parskip}
\usepackage{fullpage}
\usepackage{wrapfig}
\usepackage{tikz}

\title{Probability}
\author{Paolo Bettelini}
\date{}

\begin{document}

\maketitle
\tableofcontents

\pagebreak

\section{Probabilità}

\subsection{Costruzione intuitiva}

\begin{radioactive}
\sexample{Approccio classico alla probabilità}{
    Consideriamo un'urna contenente 6 palline numerate da \(1\) a \(6\) e per il resto indistinguibili.
    Vogliamo studiare l'esperimento estrazione di una pallina dall'urna.
    Voglio studiare quale pallina sia più probabile che venga estratta.
    Vogliamo quindi mettere un ordinamento sull'insieme degli eventi di questo esperimento.
    Sia \(\Omega\) l'insieme di tutti i possibili risultati dell'esperimento.
    Sia \(P_i\) la probabilità di estrarre il numero \(i \in \{1,2,3,5,6\}\).
    In questo caso la scelta naturale è \(P_i = \frac16\) per tutte le \(i \in \{1,2,3,4,5,6\}\).
    Notiamo che \(\sum P_i = 1\), cioè la probabilità di estrarre un numero da \(1\) a \(6\), cioè in questo
    caso l'evento certo. Questa viene detta additività della probabilità di eventi disgiunti.
    Inoltre, la probabilità \(P_j = 0\) con \(j \notin \{1,2,3,4,5,6\}\).
    Possiamo anche notare che
    \[
        P_i = \frac{\text{Casi favorevoli}}{\text{Casi possibili}}
        = \frac{
            |\{i\}|
        }{|N|}
    \]
    In questo caso i casi elementari \(\{i\}\) sono equiprobabili.
}
\end{radioactive}

\begin{radioactive}
\sexample{Approccio frequentista alla probabilità}{
    Consideriamo l'esperimento lancio di un dado
    a 6 facce numerate da \(1\) a \(6\).
    Abbiamo \(\Omega = \{1,2,3,4,5,6\}\).
    In questo caso non è detto che gli eventi o i casi elementari
    siano equiprobabili. Quindi,
    per assegnare i \(P_i\) potremmo lanciare il dado sperimentalmente.
    \[
        P_i^{(k)} = \frac{N_i^{(k)}}{k}
    \]
    con \(N_i^k\) è il numero di volte che esce l'evento \(i\) su \(k\) lanci.
    Chiaramente questi valori sono variabili, quindi prendiamo
    \[
        P_i = \lim_{k \to +\infty} P^{(k)}_i
    \]
    per  \(i \in \{1,2,3,4,5,6\}\) se il limite esiste.
    Notiamo che \(P_i^{(k)} \in [0,1]\) e
    \[
        \sum_{i=1}^6 P_i^{(k)} = 1, \quad k \in \naturalnumbers
    \]
    e prendendo il limite \(k \to +\infty\)
    \[
        \sum_{i=1}^6 P_i = 1
    \]
}
\end{radioactive}

\begin{radioactive}
In entrambi i casi abbiamo quindi le medesime proprietà
ma nella seconda le probabilità non coincidono necessariamente.
\end{radioactive}

\begin{radioactive}
\sexample{Probabilità soggettiva}{
    Consideriamo un torneo di calcio dove tutti giocano contro tutti, in cui partecipano 6 squadre:
    (1) R. Madrid, (2) M. City, (3) Bayer Monaco, (4) Atalanta, (5) Porto, (6) Nantes.
    L'esperimento che consideriamo è quello che studia il vincitore del torneo.
    Abbiamo \(\Omega = \{1,2,3,4,5,6\}\).
    L'approccio classico richiede casi elementari equiprobabili, e in questo caso non lo sono.
    L'approccio frequentista richiede di richiedere un torneo molte volte sotto le stesse condizioni,
    il che è praticamente impossibile.
    L'idea alternativa è quindi quella di chiedere a due esperti del settore
    di assegnare le probabilità in maniera coerente alle osservazioni che abbiamo
    fatto negli altri due approcci.
    Il primo esperto scegli per esempio \[
        P_1 = \frac14,
        P_2 = \frac14,
        P_3 = \frac15,
        P_4 = \frac15,
        P_5 = \frac{1}{10},
        P_6 = 0
    \]
    mentre il secondo \[
        P_1 = \frac{11}{27},
        P_2 = \frac13,
        P_3 = \frac19,
        P_4 = \frac{2}{27},
        P_5 = \frac{1}{27},
        P_6 = \frac{1}{27}
    \]
    Chiaramente c'è natura soggettiva.
}
\end{radioactive}

\begin{radioactive}
\sdefinition{Probabilità soggettiva}{
    Si definisce probabilità di un evento
    la misura del grado di fiducia, cioè un numero reale in \([0,1]\),
    che in individuo coerente
    assegna al verificarsi dell'evento considerato, in base alle sue conoscenze. \\
    In altro modo, la probabilità di un evento è quanto un individuo coerente ritiene equo pagare
    per ricevere \(1\) se l'evento si verifica e \(0\) se non si verifica.
}
\end{radioactive}

\begin{radioactive}
Ognuno degli approcci ricopre l'approccio precedente.
Quello soggettivo è quello più generale.
\end{radioactive}

\subsection{Formalizzazione}

\begin{radioactive}
\sdefinition{Evento}{
    Un \emph{evento} è una qualsiasi asserzione
    della quale ne si può stabilire la veridicità
    osservando il risultato dell'esperimento. 
}
\end{radioactive}

\begin{radioactive}
Se abbiamo un urna possiamo rappresentare
gli eventi come punti su un segmento di punti, cioè tutte le possibili estrazioni.
Con due urne posso fare lo stesso con una griglia discreta, e così via.
\end{radioactive}

\pagebreak

\section{Temp analisi III}

\stheorem{Criterio di Cauchy}{
    Una successione di funzioni \(\{f_n\}\) converge uniformemente a \(f \colon S \fromto \realnumbers\)
    in \(S\) se e solo se \(\varepsilon > 0\) esiste \(N(\varepsilon)\) naturale tale che
    \[
        \forall m,n > N,
        |f_n(x) - f_m(x)| < \varepsilon, \forall x \in S
    \]
}

\sproof{}{
    \iffproof{
        Fissiamo \(\varepsilon > 0\).
        Dalla definizione di convergenza uniforme,
        esiste \(N\left(\frac{\varepsilon }{2}\right)\) tale che
        \[
            \left|f_n(x) - f(x)\right| < \frac{\varepsilon}{2}
        \]
        per ogni \(n > N\) e \(x \in S\).
        Quindi, \begin{align*}
            \left|f_n(x) - f(x)\right| &\leq \left|f_n(x) - f(x)\right| + \left|f_n(x) - f(x)\right| \\
            &\leq \frac{\varepsilon}{2} + \frac{\varepsilon}{2} = \varepsilon
        \end{align*}
    }{
        Mostriamo la convergenza uniforme. Fissiamo \(x \in S\).
        Allora, per la condizione soddisfatta, \(\{f_n(x)\}\) è una successione
        numerica che verifica il criterio di Cauchy. Di conseguenza, esiste
        \(f_x\) tale che
        \[
            \lim_{n \to \infty} f_n(x) = f_x
        \]
        per ogni \(x \in S\) siccome l'ipotesi vale per tutte le \(x\).
        Definiamo \(f \colon S \fromto \realnumbers\) data da
        \[
            f(x) \triangleq f_x
        \]
        Per iptoesi per ogni \(\varepsilon > 0\) esiste \(N(\varepsilon)\) naturale tale che
        \begin{align*}
            |f_n(x) - f_m(x)| < \varepsilon
        \end{align*}
        per tutte le \(m,n > N\) e \(x \in S\).
        Prendendo il limite con \(m \to \infty\), per definizione di \(f\), otteniamo
        \[
            |f_n(x) - f(x)| < \varepsilon
        \]
        per tutte le\(m,n > N\) e \(x \in S\).
    }
}

Vediamo ora come si comporta la convergenza uniforme rispetto
alle proprietà studiate precedentemente.

\sdefinition{Limitatezza successione di funzioni}{
    Diciamo che la successione di funzioni \(\{f_n\}\)
    è \emph{limitata} in \(S\) se \(\forall n \in \naturalnumbers\), \(\exists r_n > 0\)
    tale che \(|f_n(x) \leq M_n\).
}

\sdefinition{Equilimitatezza successione di funzioni}{
    Diciamo che la successione di funzioni \(\{f_n\}\)
    è \emph{equilimitata} in \(S\) se \(\exists M > 0\) tale che
    \(\forall n \in \naturalnumbers\), \(|f_n(x) \leq M\).
}

Equilimitato implica limitato.

\sproposition{}{
    Sia \(\{f_n\}\) convergente uniformemente a \(f\) in \(S\)
    con \(f \colon S \fromto \realnumbers\).
    Se \(f_n\) è limitata per ogni \(n \in \naturalnumbers\),
    allora \(f\) è limitata in \(S\) e \(\{f_n\}\) è equilimitata.
}

\sproof{}{
    Per la definizione di convergenza uniforme, per ogni \(\varepsilon > 0\)
    esiste \(N(\varepsilon)\) naturale tale che
    \[
        |f_n(x) - f(x)| < \varepsilon, \quad \forall n > N, x \in S
    \]
    Fissiamo arbitrariamente \(\varepsilon = 1\) e prendiamo
    \(N_1 = N(1)\), che verifica la defiinzionoe di convergenza uniforme con \(\varepsilon = 1\).
    Allora,
    \begin{align*}
        |f(x)| &\leq |f(x) - f_{N_1}(x)| + |f_{N_1}(x)| \\
        &\leq 1 + |f_{N_1}(x)| \leq 1 + M_{N_1}
    \end{align*}
    dove \(M_{N_1} > 0\) tale che \(|f_{N_1}(x)| \leq M_{N_1}\) per ogni \(x \in S\), siccome è limitata.
    Siccome \(N_1\) è fissato, esiste \(M>0\) tale che \(|f(x)| \leq M\) per \(x \in S\).
    Proviamo ora la equilimitatezza di \(\{f_n\}\) in \(S\).
    Abbiamo visto che \begin{align*}
        |f_n(x)| \leq |f_n(x) - f(x)| + |f(x)| \leq 1 + M 
    \end{align*}
    Ma per ipotesi sappiamo che per ogni \(n \in \naturalnumbers\),
    esiste \(M_n\) tale che \(|f_n(x)| < M_n\) per ogni \(x \in S\).
    Allora definiamo \(M' = \max\{M_1, \cdots, M_{N_1}, 1 + M\}\), vale che
    \[
        |f_n(x)| \leq M', \quad \forall x \in S, \forall n \in \naturalnumbers
    \]
}

Non vale il viceversa.

\sexample{L'equilimitatezza non implica nemmeno la convergenza puntuale.}{
    Consideriamo \(f_n(x) = \sin(nx)\), che ovviamente è equilimitata da \(M=1\).
    Tuttavia, non converge puntualmente in \(S = [0, 2\pi]\). Basta prendere \(x = \frac{\pi}{2}\)
    \begin{align*}
        \sin(n \frac{\pi}{2}) = \begin{cases}
            1 & n = 1 + 4k \\
            0 & n = 2k \\
            -1 & n = 3 + 4k
        \end{cases}
    \end{align*}
}

Non vale nemmeno un analogo di Bolzano Weierstrass.

\sexample{L'equilimitatezza non implica un analogo di Bolzano Weierstrass con convergenza puntuale.}{
    Consideriamo \(f_n(x) = \sin(nx)\), che ovviamente è equilimitata da \(M=1\).
    Supponiamo che esista una sottosuccessione \(\{f_{n_k}\}\) che converge puntualmente
    con \(n_k \to \infty\) per \(k \to \infty\).
    Definisco
    \[
        g_k(x) = \left(f_{n_k}(x) - f_{n_{k+1}}(x)\right)^2
        = \left(\sin(n_k x) - \sin(n_{k+1}x)\right)^2
    \]
    è immediato verificare che \(g_k\) converge puntualmente a zero in \(S\),
    poiché \(\{f_{n_k}\}\) converge in \(S\).
    Inoltre, \(|g_k| \leq 4\).
    Per il teorema della convergenza dominata, che vedremo in futuro,
    vale \begin{align*}
        \lim_{k \to \infty} \integral[0][2\pi][g_k(x)][x]
        = \integral[0][2\pi][\lim_{k\to\infty} g_k(x)][x] = 0
    \end{align*}
    Tuttavia, abbiamo che \(\forall k \in \naturalnumbers\),
    \[
        \integral[0][2\pi][g_k(x)][x]
        = \integral[0][2\pi][\left(\sin(n_k x) - \sin(n_{k+1}x)\right)^2][x]
        = 2\pi
    \]
    che è assurdo \lightning.
}

\sexercise{}{
    Dimostrare che per ogni \(i,j\) naturali vale \[
        \integral[0][2\pi][\sin(ix)\sin(jx)][x] = \begin{cases}
            0 & i \neq j \\
            \pi & i = j
        \end{cases}
    \]
}

Studiamo ora la continuità e la convergenza uniforme.

\stheorem{Scambio del limite}{
    Siano \(\{f_n\}\) una successione di funzioni uniformemente convergente ad \(f\)
    in \(S\) con \(f \colon S \fromto \realnumbers\).
    Sia \(x_0 \in S\) punto di accumulazione tale che
    \[
        \exists \lim_{x\to x_0} f_n(x) \in \realnumbers, \quad \forall n \in \naturalnumbers
    \]
    Allora, vale che
    \[
        \lim_{n \to \infty} \lim_{x \to \infty} f_n(x)
        = \lim_{x \to \infty} \lim_{n \to \infty} f_n(x)
    \]
}

\sproof{}{
    Fissiamo \(x_0 \in S\) punto di accumulazione
    e definiamo \[ A_n \triangleq \lim_{x \to x_0} f_n(x) \] per \(n \in \naturalnumbers\).
    Per il criterio di Cauchy, siccome abbiamo convergenza uniforme,
    esiste \(N(\varepsilon)\) naturale tale che
    \[
        |f_n(x) - f_m(x)| < \varepsilon
    \]
    per tutte le \(m,n > N\) e \(x \in S\).
    La funzione \(x \to |x|\) è continua quindi se
    facciamo tendere \(x \to x_0\) nell'ultima espressione,
    otteniamo
    \[
        |A_m - A_n| < \varepsilon
    \]
    per tutte le \(m,n > N\) e \(x \in S\).
    Di conseguenza, \(\{A_n\}\) è una successione numerica che verifica il criterio di Cauchy.
    Quindi, converge e definiamo
    \[
        A \triangleq \lim_{n \to \infty} A_n
    \]
    Quindi la parte sinistra di 
        \[
        \lim_{n \to \infty} \lim_{x \to \infty} f_n(x)
        = \lim_{x \to \infty} \lim_{n \to \infty} f_n(x)
    \]
    ha un senso. Ci manca da dimostrare che
    \[
        \lim_{x\to x_0} f(x) = A
    \]
    Cioè, \(\forall \varepsilon > 0\), \(\exists \delta(\varepsilon, x)\)
    tale che \(|f(x) - A| < \epsilon\) per ogni \(x\in S \difference \{x_0\}\) tale che \(|x-x_0| < \delta\).
    Sappiamo che \(f_n\) converge uniformemente a \(f\) quindi esiste \(N\left(\frac{\varepsilon}{3}\right)\)
    naturale tale che
    \begin{align*}
        |f_n(x) - f(x)| < \frac{\varepsilon}{3}, \quad \forall n > N, x \in S
    \end{align*}
    Inoltre, sappiamo che \(A_n\) converge ad \(A\) per \(n \to +\infty\),
    e quindi esiste \(N_2\left(\frac{\varepsilon}{3}, x_0\right)\) naturale tale che 
    \[
        |A_m - A| < \frac{\varepsilon}{3}, \quad \forall n \geq N_2
    \]
    Per definizione, sappiamo che per ogni \(n \in \naturalnumbers\),
    \[
        A_n \triangleq \lim_{x\to x_0} f_n(x)
    \]
    quindi fissato \(M > \max\{N_1, N_2\}\), esiste
    \(\overline\delta(M, x_0, \varepsilon)\) tale che
    
    \[
        |f_{M}(x) - A_{M}| < \frac{\varepsilon}{3},
        \quad \forall x \in S \difference \{x_0\} \suchthat |x-x_0| < \overline\delta
    \]
    Combinando tutto, otteniamo che
    \begin{align*}
        |f(x) - A| &\leq |f(x) - f_{M}(x)| + |f_{M}(x) - A_{M}| + |A_{M} - A| \\
        &\leq \frac{\varepsilon}{3} + \frac{\varepsilon}{3} + \frac{\varepsilon}{3} = \varepsilon
    \end{align*}
    per tutte le \(x \in S \difference \{x_0\}\) tale che \(|x-x_0| < \overline\delta\).
}

Notiamo che se \(\{f_n\}\) converge a \(f\), \(x_0 \in S\)
punto di accumulazione tale che
\[
    \lim_{x\to x_0} f(x) = +\infty, \quad \forall n \in \naturalnumbers
\]
allora
\[
    \lim_{x \to x_0} f(x) = +\infty
\]
e analogamente per \(-\infty\).
Avremmo potuto unificare il tutto con gli intorni della retta reale estesa.


Se richiediamo che \(f_n(x)\) siano continue allora il limite per \(x \to x_0\)
è precisamente \(f_n(x_0)\), e quindi da questo teorema otteniamo immediatamente il corollario.

\scorollary{}{
    Siano \(\{f_n\}\) una successione di funzioni continue uniformemente convergente ad \(f\)
    in \(S\) con \(f \colon S \fromto \realnumbers\).
    Allora \(f\) è continua su \(S\) e \(\forall x_0 \in S\) punto di accumulazione vale
    \[
        \lim_{n \to \infty} \lim_{x \to \infty} f_n(x)
        = \lim_{x \to \infty} \lim_{n \to \infty} f_n(x) = f(x_0)
    \]
}

\sproof{}{
    per esercizio dimostrarla usando la continuità
    per successioni. Ora la dimostriamo con la definizione classica.
    Dobbimo mostrare che per ogni \(x_0 \in S\),
    la funzione \(f\) è continua in \(x_0\),
    cioè \(\forall \varepsilon > 0\) esiste \(\delta(x_0, \varepsilon) > 0\)
    tale che \[
        |f(x) - f(x_0)| < \varepsilon, \quad \forall x \in S \suchthat |x-x_0| < \delta
    \]
    Fissiamo \(x_0 \in S\) e \(\varepsilon > 0\).
    Visto che \(f_n\) converge uniformemente ad \(f\), esiste \(N\left(\frac{\varepsilon}{2}\right)\)
    naturale tale che
    \[
        |f_n(x) - f(x)| < \frac{\varepsilon}{3}, \quad \forall n > N, \forall x \in S
    \]
    Fissiamo \(n_0 > N\).
    Per ipotesi, \(f_{n_0}\) è continua in \(x_0\),
    quindi esiste \(\overline \delta\left(x_0, n_0, \frac{\varepsilon}{3}\right)\) tale che
    \[
        |f_{n_0}(x) - f_{n_0}(x_0)| < \frac{\varepsilon}{3},
        \quad \forall x \in S \suchthat |x-x_0| \leq \overline\delta
    \]
    Combinando il tutto otteniamo che \(\forall x \in S\) tale che \(|x-x_0| < \overline\delta\) vale
    \begin{align*}
        |f(x) - f(x_0)| &\leq |f(x) - f_{n_0}(x)| + |f_{n_0}(x) - f(x_0)|
        + |f_{n_0}(x) - f_{n_0}(x_0)| \\
        &\leq \frac{\varepsilon}{3} + \frac{\varepsilon}{3} + \frac{\varepsilon}{3} = \varepsilon
    \end{align*}
    Quindi scegliendo \(\delta = \overline \delta\) viene verifica la definizione di continuità in \(x_0\)
    per \(f\).
}

\end{document}
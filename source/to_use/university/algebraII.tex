\documentclass[a4paper]{article}

\usepackage{amsmath}
\usepackage{amssymb}
\usepackage{stellar}
\usepackage{parskip}
\usepackage{fullpage}
\usepackage{wrapfig}
\usepackage{tikz}

\usetikzlibrary{arrows}
\usetikzlibrary{decorations.pathreplacing}
\usetikzlibrary{cd}

\title{Algebra II}
\author{Paolo Bettelini}
\date{}

\begin{document}

\maketitle
\tableofcontents

\section{Teoremi di isomorfismo su quozienti di spazi vettoriali}

Let \(V\) be a vector space over \(\mathbb{K}\)
and \(W\) be a linear subspace of \(V\).

We have a map
\[
    \pi \colon V \to V/W
\]
defined as
\[
    \pi(v) \triangleq v + W \in V/W
\]
which is a linear map.

Indeed,
\begin{enumerate}
    \item \[
        \pi(0_V) = 0_V + W = w + W
    \]
    \item \begin{align*}
        \pi(v_1 + v_2) &= \pi(v_1) + \pi(v_2) \\
        (v_1 + v_2) + W &= (v_1 + W) + (v_2 + W) \\
    \end{align*}
    \item \[
        \pi(\lambda v) = (\lambda v) + W = \lambda \left(v + W\right)
    \]
\end{enumerate}

We now consider a morphism \(\varphi \colon V_1 \to V_2\)
between vector spaces. We know that its kernel is a subspace of \(V_1\).
We now construct a new morphism
\[
    \overline{\varphi} \colon V_1 / \text{ker}_\varphi \to V_2
\]
such that
\[
    \overline{\varphi}(v + \text{ker}_\varphi) \triangleq \varphi(v)
\]
We need to ensure that such mapping is well-defined.
Let \(v' \in v + \text{ker}_\varphi\), meaning that \(v' = v + w\) with \(w \in \text{ker}_\varphi\).
\begin{align*}
    \overline{\varphi}(v' + \text{ker}_\varphi)
    &= \varphi(v') = \varphi(v + w) = \varphi(v) + \varphi(w) \\
    &= \varphi(v) = \overline{\varphi}(v + \text{ker}_\varphi)
\end{align*}
We now show that it is also linear:
\begin{enumerate}
    \item \[
        \overline{\varphi}(0_{V_1} + \text{ker}_\varphi) = \varphi(0_{V_1}) = 0_{V_2}
    \]
    \item \begin{align*}
        \overline{\varphi}((v_1 + \text{ker}_\varphi) + (v_2 + \text{ker}_\varphi))
        &= \overline{\varphi}((v_1 + v_2) + \text{ker}_\varphi) \\
        &= \varphi(v_1 + v_2) = \varphi(v_1 + v_2) \\
        &= \overline{\varphi}(v_1 + \text{ker}_\varphi)
        + \overline{\varphi}(v_2 + \text{ker}_\varphi)
    \end{align*}
    \item \[
        \overline{\varphi}(\lambda (v + \text{ker}_\varphi)) =
        \lambda \left(\overline{\varphi}(v + \text{ker}_\varphi)\right)
    \]
\end{enumerate}

Il seguente diagramma commuta e \(\pi\) è suriettiva in quanto \(v+\text{ker}_\varphi = \pi(v)\).
% https://tikzcd.yichuanshen.de/#N4Igdg9gJgpgziAXAbVABwnAlgFyxMJZABgBpiBdUkANwEMAbAVxiRADUB9ARhAF9S6TLnyEU3clVqMWbLgCZ+gkBmx4CRMtyn1mrRBx4ACAPRGAOuZwwAHjmABrGACc+nS-WdoAFln5SYKABzeCJQADNnCABbJDIQHAgkCWk9Ng86L18lCKjYxHjEpHlqbxg6KDYcAHcIMoqEal1ZA0s0P2oGOgAjGAYABWF1MRBnLCDvHByQSJji6iLEFN6wSsQAWgBmeOb9EEsIGhcGLDAYYAysrD4QTp6+wbVRNjGJqb4KPiA
\begin{tikzcd}
V_1 \arrow[r, "\varphi"] \arrow[d, "\pi"', two heads]                  & V_2 \\
V_1 / \text{ker}_\varphi \arrow[ru, "\overline{\varphi}"', bend right] &    
\end{tikzcd}

Quindi \(\varphi = \overline{\varphi} \circ \pi\).

\stheorem{First isomorphism theorem}{
    Let \(\varphi \colon V_1 \to V_2\) be a morphism between vector spaces.
    \[
        \overline{\varphi} \colon V_1 / \text{ker}_\varphi \to \text{im}_\varphi
    \]
    is an isomorphism of vector spaces, meaning
    \[
        V_1 / \text{ker} \cong \text{im}_{\varphi}
    \]
}

\sproof{First isomorphism theorem}{
    We need to show that the morphism is both surjective and injective:
    \begin{enumerate}
        \item let \(v_2 \in \text{im}_\varphi\).
        We want to find a \(v_1 \in V_1\) such that \(v_2 = \varphi(v_1)\).
        This is precisely
        \[
            \overline{\varphi}(v_1 + \text{ker}_\varphi)
        \]
        \item we want to show that the kernel is trivial.
        \begin{align*}
            \text{ker}_{\overline{\varphi}} &= \{
                v + \text{ker}_\varphi \,|\,
                \overline{\varphi}(v + \text{ker}_\varphi) = 0_{V_2}
            \} \\
            &= \{v + \text{ker}_\varphi \,|\, v \in \text{ker}_\varphi\} \\
            &= 0_{V_1} + \text{ker}_\varphi
        \end{align*}
        since \(v + \text{ker}_\varphi = \text{ker}_\varphi\) and we can just choose \(0_{V_1}\).
    \end{enumerate}
}

\sexample{}{
    Consider a vector space \(V = W_1 \oplus W_2\) with \(W_1, W_2 \leq V\)
    and consider the mappings
    \[
        p_1 \colon V \to W_1, \quad p_2 \colon V \to W_2
    \]
    Using the diagrams with \(\overline{p_1},\pi_1\) and \(\overline{p_2},\pi_2\),
    we have 
    \[
        W_1 \cong V/W_2, \quad W_2 \cong V/W_1
    \]
    since \(W_2 = \text{ker}_{p_1}\) and \(W_1 = \text{ker}_{p_2}\).
}

\stheorem{Second isomorphism theorem}{
    Let \(V\) be a vector space over \(\mathbb{K}\) and \(U, W \leq V\).
    Then,
    \[
        \frac{W}{W \cap U} \cong \frac{W + U}{U}
    \]
}

\sproof{Second isomorphism theorem}{
    We apply the first isomoprhism theorem. Construct a surjective mapping
    \[
        \varphi \colon \frac{W}{W \cap U} \to W + U
    \]
    such that \(\text{ker}_\varphi = U\).
    We first note that \[ \frac{W}{W \cap U} \leq V/U \]
    and so we define
    \[
        \varphi(w) \triangleq w + U \in V/U
    \]
    We need to show that it is linear (todo).
    It is surjective as 
    \[
        \text{Im}_\varphi = \frac{W + U}{U}
    \]
    since \(w + u + U = w+ U = \varphi(w)\).
    We now need to study that it is injective
    \begin{align*}
        \text{ker}_\varphi &= \{
            w \in W \,|\, w + U = 0_{V/U} = 0_V + U    
        \} \\
        &= \{
            w \in W \,|\,  w \in U    
        \} = W \cap U
    \end{align*}
    since \(w + U = 0_V + U\) means that \(w \in U\).
}

Notiamo che \(U\) potrebbe non essere sottospazio di \(W\) quindi
non possiamo rimpiazzare \(W + U\) con \(W / U\).

\stheorem{Third isomoprhism theorem}{
    Sia \(V\) uno spazio vettoriale e \(W \leq V\) e \(U \leq W\) dei sottospazi.
    Consideriamo \(V/U\) e \(W/U \leq V/U\).
    e possiamo fare
    \[
        \frac{V/U}{W/U} \cong V/W
    \]
}

\sproof{Third isomoprhism theorem}{
    Costruiamo un morphismo (suriettivo) \(\overline{\varphi} = V/U \to V/W\)
    tale che \(\text{ker}_{\overline{\varphi}} = W/U\).
    Applicando il primo teorema di isomormorfismo otteniamo
    \[
        \frac{V/U}{\text{ker}_{\overline{\varphi}}}
        \cong \text{Im}_{\overline{\varphi}} = V/W
    \]
    Definiamo \(\overline{\varphi}(v+U) = v+W\). Mostriamo che è ben definito:
    dato \(v' \in v + U\) diverso da \(v\), e quindi \(v' = v+u\) con \(u\in U\) vale
    \begin{align*}
        \overline{\varphi}(v' + U) = v' + W = (v + u) + W
        = v + W = \overline{\varphi}(v + U)
    \end{align*}
    siccome \(u \in W\).
    Mostriamo ora che è lineare
    \begin{enumerate}
        \item \begin{align*}
            \overline{\varphi}((v_1 + U) + (v_2 + U))
            &= \overline{\varphi}((v_1 + v_2) + U) = (v_1 + v_2) + W
        \end{align*}
    \end{enumerate}
    Per la suriettività basta prendere un qualsiasi elemento del quoziente
    \(v + W \in V/W\) arbitrario, \(v + W = \overline{\varphi}(v + U)\)
    e quindi \(v + W \in \text{Im}_{\overline{\varphi}}\).
    Per l'iinettività
    \begin{align*}
        \text{ker}_{\overline{\varphi}}
        &= \{
            v + U \in U/V \,|\, v + W = \overline{\varphi}(v + U)
            = 0_{V/W} = 0_V + W
        \} \\
        &= \{
            v + U \in V/U  \,|\, v \in W   
        \} = W/U
    \end{align*}
}

\section{Anelli}

\((Z, +, \cdot)\) è un anello commutativo dove gli elementi invertibili sono solo \(\pm 1\). \\
\((\mathbb{K}[x], +, \cdot)\) è un anello commutativo dove gli elementi invertibili sono solo i polinomi
di grado zero.

Algebra gruppale: Sia \(G\) un gruppo e sia \(\mathbb{K}\) un campo.
\[
    G[\mathbb{K}] = \left\{
        \sum_{g\in G} \lambda g \,|\, \lambda \in \mathbb{K}
    \right\}
\]
(Giusto?)
La addizione è data da.
\begin{align*}
    \left( \sum_{g\in G} \lambda_g \cdot g \right)
    + \left( \sum_{h\in G} \lambda_h \cdot h \right)
    &= \sum_{g,h \in G} (\lambda_g + \lambda_h) (gh) \\
\end{align*}
La moltiplicazione è data da
\begin{align*}
    \left( \sum_{g\in G} \lambda_g \cdot g \right)
    \cdot \left( \sum_{h\in G} \lambda_h \cdot h \right)
    &= \sum_{g,h \in G} (\lambda_g \cdot \lambda_h) (gh) \\
    &= \sum_{k\in G} \left(\sum_{g\cdot h = k} (\lambda_g \lambda_h) \right) \cdot k
\end{align*}
L'elemento neutro è dato da
\[
    0 = \sum_{g\in G} 0 \cdot g
\]
e l'identità
\[
    1 = 1 \cdot 1_G + \sum_{g\in G} 0 \cdot g
\]

\sexample{I quaternioni sono una algebra reale}{
    \[
        \mathbb{H} = \text{span}\{e,i,j,k\}
    \]
    Abbiamo che
    \[
        Z(\mathbb{H}) = \text{span}\{e\} \cong \mathbb{R}
    \]
    Vale \(A_1, A_2, A_3 \leq \mathbb{H}\)
    con
    \begin{align*}
        A_1 &= \text{span}\{e,i\}, \\
        A_2 &= \text{span}\{e,j\}, \\
        A_3 &= \text{span}\{e,k\}, \\
    \end{align*}
    Sono autocentralizzanti e sono isomorfi ai complessi come algebra reale.
}

\section{Esempi di quozienti}

Le class di resto sono un anelli.
Sia \(A = \mathbb{Z}\) e \(I = (n) = \{n \cdot k \,|\, k \in \mathbb{Z}\}\).
Abbiamo quindi il quoziente \(\mathbb{Z}/_{(n)}\) che è dato da
\[
    a+I = a+(n) = {[a]}_n
\]
Chiaramente \(1_{\mathbb{Z}/_{(n)}} = {[1]}_n\).
Questo quoziente non è un dominio di integrità se \(n\) non è un numro primo.
Se non è primo allora esistono interi \(a,b\)
tali che \(a,b \neq \pm 1\)
tale che \(n=ab\). Allora
\begin{align*}
    {[a]}_n \cdot {[b]}_n = {[ab]}_n = {[n]}_n = {[0]}_n
\end{align*}
Se \(n\) è primo allora \(\mathbb{Z}/p\) è addirittura un campo.
Infatti se \({[a]}_p \neq {[0]}_p\)
prendiamo \(0 < a < p\) con \(a,p\) coprimi fra loro.
Allora esistono \(k_1, k_2 \in \mathbb{Z}\)
tale che \(ak_1 + pk_2 = 1\)
quindi \(a_1 = 1 + (-k_2)p\).
Da ciò otteniamo che
\begin{align*}
    {[ak_1]}_p &= ak_1 + (p) = \left(1 + (-k_2)p\right) + (p)
    = 1 + (p) = {[1]}_p \\
    &\implies {[k_1]}_p = {[a]}_p^{-1}
\end{align*}

\subsection{Quozienti di polinomi}

Sia \(f = x^2 - 3x + 2 = (x-2)(x-1)\).
Consideriamo \(\mathbb{Q}[x]/_{(f)}\).
Due laterali sono per esempio \((x-2) + (f)\)
e \((x-1) + (f)\) che non sono lo zero.
Invece il loro prodotto
\[
    \left((x-2) + (f)\right) \cdot
    \left((x-1) + (f)\right)
    = f + (f) = 0 + (f)
\]
che è lo zero, quindi i due termini sono divisori di zero.
Consideriamo \(x^3 - 1 = (x-1)(x^2 + x + 1)\).
Anche in questo caso
\[
    \left((x-2) + (f)\right) \cdot
    \left((x^3 - 1) + (f)\right) = f \cdot (x^2 + x + 1) + (f) = 0 + (f)
\]
Dalla definizione di \(f\) abbiamo \(x^2 = f + (3x-2)\)
\begin{align*}
    x^3 &= x \cdot (f + (3x-2)) \\
    &= x \cdot f + 3f + 9x-6-2x \\
    &= f(x+3) + 7x-6
\end{align*}
Da questo otteniamo che
\begin{align*}
    (x^3 - 1) + (f) &= 7x-6 - 1 + (f) \\
    &= 7x-7 + (f) \\
    &= 7(x+1) + (f)
\end{align*}
che è un suo rappresentante di un grado minore.

\end{document}
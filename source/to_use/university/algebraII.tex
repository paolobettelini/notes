\documentclass[a4paper]{article}

\usepackage{amsmath}
\usepackage{amssymb}
\usepackage{stellar}
\usepackage{parskip}
\usepackage{fullpage}
\usepackage{wrapfig}
\usepackage{tikz}

\usetikzlibrary{arrows}
\usetikzlibrary{decorations.pathreplacing}
\usetikzlibrary{cd}

\title{Algebra II}
\author{Paolo Bettelini}
\date{}

\begin{document}

\maketitle
\tableofcontents

\section{Teoremi di isomorfismo su quozienti di spazi vettoriali}

Let \(V\) be a vector space over \(\mathbb{K}\)
and \(W\) be a linear subspace of \(V\).

We have a map
\[
    \pi \colon V \to V/W
\]
defined as
\[
    \pi(v) \triangleq v + W \in V/W
\]
which is a linear map.

Indeed,
\begin{enumerate}
    \item \[
        \pi(0_V) = 0_V + W = w + W
    \]
    \item \begin{align*}
        \pi(v_1 + v_2) &= \pi(v_1) + \pi(v_2) \\
        (v_1 + v_2) + W &= (v_1 + W) + (v_2 + W) \\
    \end{align*}
    \item \[
        \pi(\lambda v) = (\lambda v) + W = \lambda \left(v + W\right)
    \]
\end{enumerate}

We now consider a morphism \(\varphi \colon V_1 \to V_2\)
between vector spaces. We know that its kernel is a subspace of \(V_1\).
We now construct a new morphism
\[
    \overline{\varphi} \colon V_1 / \text{ker}_\varphi \to V_2
\]
such that
\[
    \overline{\varphi}(v + \text{ker}_\varphi) \triangleq \varphi(v)
\]
We need to ensure that such mapping is well-defined.
Let \(v' \in v + \text{ker}_\varphi\), meaning that \(v' = v + w\) with \(w \in \text{ker}_\varphi\).
\begin{align*}
    \overline{\varphi}(v' + \text{ker}_\varphi)
    &= \varphi(v') = \varphi(v + w) = \varphi(v) + \varphi(w) \\
    &= \varphi(v) = \overline{\varphi}(v + \text{ker}_\varphi)
\end{align*}
We now show that it is also linear:
\begin{enumerate}
    \item \[
        \overline{\varphi}(0_{V_1} + \text{ker}_\varphi) = \varphi(0_{V_1}) = 0_{V_2}
    \]
    \item \begin{align*}
        \overline{\varphi}((v_1 + \text{ker}_\varphi) + (v_2 + \text{ker}_\varphi))
        &= \overline{\varphi}((v_1 + v_2) + \text{ker}_\varphi) \\
        &= \varphi(v_1 + v_2) = \varphi(v_1 + v_2) \\
        &= \overline{\varphi}(v_1 + \text{ker}_\varphi)
        + \overline{\varphi}(v_2 + \text{ker}_\varphi)
    \end{align*}
    \item \[
        \overline{\varphi}(\lambda (v + \text{ker}_\varphi)) =
        \lambda \left(\overline{\varphi}(v + \text{ker}_\varphi)\right)
    \]
\end{enumerate}

Il seguente diagramma commuta e \(\pi\) è suriettiva in quanto \(v+\text{ker}_\varphi = \pi(v)\).
% https://tikzcd.yichuanshen.de/#N4Igdg9gJgpgziAXAbVABwnAlgFyxMJZABgBpiBdUkANwEMAbAVxiRADUB9ARhAF9S6TLnyEU3clVqMWbLgCZ+gkBmx4CRMtyn1mrRBx4ACAPRGAOuZwwAHjmABrGACc+nS-WdoAFln5SYKABzeCJQADNnCABbJDIQHAgkCWk9Ng86L18lCKjYxHjEpHlqbxg6KDYcAHcIMoqEal1ZA0s0P2oGOgAjGAYABWF1MRBnLCDvHByQSJji6iLEFN6wSsQAWgBmeOb9EEsIGhcGLDAYYAysrD4QTp6+wbVRNjGJqb4KPiA
\begin{tikzcd}
V_1 \arrow[r, "\varphi"] \arrow[d, "\pi"', two heads]                  & V_2 \\
V_1 / \text{ker}_\varphi \arrow[ru, "\overline{\varphi}"', bend right] &    
\end{tikzcd}

Quindi \(\varphi = \overline{\varphi} \circ \pi\).

\stheorem{First isomorphism theorem}{
    Let \(\varphi \colon V_1 \to V_2\) be a morphism between vector spaces.
    \[
        \overline{\varphi} \colon V_1 / \text{ker}_\varphi \to \text{im}_\varphi
    \]
    is an isomorphism of vector spaces, meaning
    \[
        V_1 / \text{ker} \cong \text{im}_{\varphi}
    \]
}

\sproof{First isomorphism theorem}{
    We need to show that the morphism is both surjective and injective:
    \begin{enumerate}
        \item let \(v_2 \in \text{im}_\varphi\).
        We want to find a \(v_1 \in V_1\) such that \(v_2 = \varphi(v_1)\).
        This is precisely
        \[
            \overline{\varphi}(v_1 + \text{ker}_\varphi)
        \]
        \item we want to show that the kernel is trivial.
        \begin{align*}
            \text{ker}_{\overline{\varphi}} &= \{
                v + \text{ker}_\varphi \,|\,
                \overline{\varphi}(v + \text{ker}_\varphi) = 0_{V_2}
            \} \\
            &= \{v + \text{ker}_\varphi \,|\, v \in \text{ker}_\varphi\} \\
            &= 0_{V_1} + \text{ker}_\varphi
        \end{align*}
        since \(v + \text{ker}_\varphi = \text{ker}_\varphi\) and we can just choose \(0_{V_1}\).
    \end{enumerate}
}

\sexample{}{
    Consider a vector space \(V = W_1 \oplus W_2\) with \(W_1, W_2 \leq V\)
    and consider the mappings
    \[
        p_1 \colon V \to W_1, \quad p_2 \colon V \to W_2
    \]
    Using the diagrams with \(\overline{p_1},\pi_1\) and \(\overline{p_2},\pi_2\),
    we have 
    \[
        W_1 \cong V/W_2, \quad W_2 \cong V/W_1
    \]
    since \(W_2 = \text{ker}_{p_1}\) and \(W_1 = \text{ker}_{p_2}\).
}

\stheorem{Second isomorphism theorem}{
    Let \(V\) be a vector space over \(\mathbb{K}\) and \(U, W \leq V\).
    Then,
    \[
        \frac{W}{W \cap U} \cong \frac{W + U}{U}
    \]
}

\sproof{Second isomorphism theorem}{
    We apply the first isomoprhism theorem. Construct a surjective mapping
    \[
        \varphi \colon \frac{W}{W \cap U} \to W + U
    \]
    such that \(\text{ker}_\varphi = U\).
    We first note that \[ \frac{W}{W \cap U} \leq V/U \]
    and so we define
    \[
        \varphi(w) \triangleq w + U \in V/U
    \]
    We need to show that it is linear (todo).
    It is surjective as 
    \[
        \text{Im}_\varphi = \frac{W + U}{U}
    \]
    since \(w + u + U = w+ U = \varphi(w)\).
    We now need to study that it is injective
    \begin{align*}
        \text{ker}_\varphi &= \{
            w \in W \,|\, w + U = 0_{V/U} = 0_V + U    
        \} \\
        &= \{
            w \in W \,|\,  w \in U    
        \} = W \cap U
    \end{align*}
    since \(w + U = 0_V + U\) means that \(w \in U\).
}

Notiamo che \(U\) potrebbe non essere sottospazio di \(W\) quindi
non possiamo rimpiazzare \(W + U\) con \(W / U\).

\end{document}
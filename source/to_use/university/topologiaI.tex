\documentclass[a4paper]{article}

\usepackage{amsmath}
\usepackage{amssymb}
\usepackage{stellar}
\usepackage{parskip}
\usepackage{fullpage}
\usepackage{wrapfig}
\usepackage{tikz}

\usetikzlibrary{arrows}
\usetikzlibrary{decorations.pathreplacing}
\usetikzlibrary{cd}

\title{Topologia I}
\author{Paolo Bettelini}
\date{}

\begin{document}

\maketitle
\tableofcontents

\section{Topologia}

Invarianti: \(p_0\) corrisponde al numero di componenti connesse di uno spazio.
Formalmente \(\pi_0(X)\) è l'insieme delle componenti connesse di \(X\) per archi.
Invece, \(p_1\) è il gruppo fondamentale \(\pi_1(X)\), che descrive la struttura dei cammini
chiusi fino a omotopia.

\saxiom{Estensionalità}{
    \[
        A = B \iff \forall x(x\in A \iff x\in B)
    \]
}

% la logica di boole corrisponde alla logica classica come la logica costruttivista corrisponde alle algebre di H

% mettere le leggi di the morgan generalizzate + le due leggi extra (famiglie arbitrarie)

\sproposition{Relazione di aggiunzione}{
    Valgono
    \[
        S \subseteq f^{-1}(T) \iff
        f(S) \subseteq T
    \]
}
Da cui derivano \(f(f^{-1})(T) \subseteq T\).
Ma in generale l'uguaglianza non vale in quanto \(f\) potrebbe non essere suriettiva.
E pure \(S \subseteq f^{-1}(f(S))\).
Ma in generale l'uguaglianza non vale in quanto \(f\) potrebbe non essere iniettiva.

L'operazione di controimmagine preserva tutte le operazioni insiemistiche.
\begin{align*}
    f^{-1} \left(\bigcup_{i\in I} A_i\right) &= \bigcup{i\in I} f^{-1}(A_i) \\
    f^{-1} \left(\bigcap_{i\in I} A_i\right) &= \bigcap{i\in I} f^{-1}(A_i) \\
    X \backslash f^{-1}(T) &= f^{-1}(Y \backslash T)
\end{align*}

L'operazione di immagine preserva in generale solo le unioni.

\begin{align*}
    f\left(\bigcup_{i\in I} A_i\right) &= \bigcup_{i\in I} f(A_i)
\end{align*}
le altre due non valgono necessariamente. Abbiamo solo
\begin{align*}
    f(A\cap B) &\subseteq f(A) \cap f(B) \\
\end{align*}
se \(f\) non è iniettiva la direzione opposta non vale necessariamente.
Infatti potrebbero esistere \(x,x'\) tale che \(x \in A \backslash B\)
e \(x' \in B \backslash A\) tali che \(f(x) = f(x')\).
La medesima logica vale per il complementare.

\sproposition{Proprietà universale del quoziente}{
    Sia \(f\colon X \to Y\) e \(\sim\) relazione di equivalenza su \(X\). Sono equivalenti:
    \begin{enumerate}
        \item \(f\) è costante sulle classi di equivalenza
            \[ x \sim x' \iff f(x) = f(x') \]
        \item \(f\) fattorizza (in modo necessaria unico, essendo \(\pi\) suriettivo) attraverso
            \(\pi\), cioè \(\exists_{=1} \,g \colon X/_\sim \to Y\) tale che \(g \circ \pi = f\).
    \end{enumerate}
}
\begin{center}
    % https://tikzcd.yichuanshen.de/#N4Igdg9gJgpgziAXAbVABwnAlgFyxMJZABgBpiBdUkANwEMAbAVxiRAA0QBfU9TXfIRQBGclVqMWbAJrdeIDNjwEiZYePrNWiDgHoA+gB1D2ALbdxMKAHN4RUADMAThHOIyIHBCQAmapqkdYzQsEGoGOgAjGAYABX5lIRAnLGsACxw5Rxc3Dy8kUQktNgcskGdXX2p8xELosCgkAFoAZg8A7RBrMJAI6LiEwTYU9MyuCi4gA
    \begin{tikzcd}
    X \arrow[d, "\pi"'] \arrow[r, "f"]   & Y \\
    X/_\sim \arrow[ru, "g"', bend right] &  
    \end{tikzcd}
\end{center}

\sproof{}{
    \begin{enumerate}
        \item \((2) \implies (1):\) \(f = g \circ \pi\).
        Abbiamo \[
            x \sim x' \implies
            \pi(x) = \pi(x') \implies g(\pi(x)) = g(\pi(x'))
        \]
        che sono uguali a \(f(x)\) e \(f(x')\).
        \item \((2) \implies (1):\)
        Definiamo \(g\colon X/_\sim \colon Y\) come
        \[
            g([x]) \triangleq f(x)
        \]
        bisogna verificare che sia ben posta.
        Vogliamo quindi che se \([x] = [x']\) allora \(f(x) = f(x')\).
        Ma ciò è garantito dalla ipotesi.
    \end{enumerate}
}

In \(\mathbb{R}^n\).
\[
    d_\infty(x,y) \leq d_2(x,y) \leq d_1(x,y) \leq n \cdot d_\infty(x,y)
\]

\section{08 ottobre 2025 DA METTERE}

\sexercise{}{
    Le topologie con la proprietà che le intersezioni arbitrari di
    aperti sono aperti, possono essere
    caratterizzate esplicitamente.
    Essi sono esattamente, a meno di omeomorfismo,
    gli spazi topologici della seguente forma: dato un insieme preordinato \((P, \leq)\),
    la topologia di Alexandrov \(\mathcal{A}_{\mathcal{P}}\)
    su \(\mathcal{P}\) è la topologia i cui aperti sono i sottoinsiemi
    \(U \subseteq \mathcal{P}\) tale che
    \(\forall p \leq q, p \in U \implies q \in U\).
}

\subsection{Generated topology}

\sproposition{}{
    Data una collezione di topologie \(\{\tau_i\}_{i \in I}\)
    su un isieme \(X\). La famiglia
    \[
        \tau = \bigcup_{i\in I} \tau_i
    \]
    è ancora una topologia su \(X\).
}

\scorollary{}{
    Sia \(X\) un insieme e \(S \subseteq \mathcal{P}(X)\) famiglia fi sottoinsiemi.
    Esiste la topologia meno fine su \(X\) che contiene i sottoinsiemi in \(S\)
    come aperti.
    Tale topologai viene detta la topologia generata da \(S\).
}

\sdefinition{Topologia dell'unione disgiunta}{
    Sia \(\{X_i \,|\, i \in I\}\) una famiglia di spazi topologici.
    Allora lo spazio topologico è definita come
    \[
        \bigsqcup_{i\in I} X_i
    \]

    Possiamo definire astrattamente la topologia dell'unione disgiunta su
    \(\sqcup X_i\) come a topologia più fine che rende tutte le mappe \(\tau_i \colon X_i \to \bigsqcup X_i\)
    continue.

    Alternativamente, possiamo definire la topologia come la topologia generata dalla famiglia di sottoinsiemi
    dell'insieme \(\sqcup X_i\)
    che sono aperti in qualcuno degli \(X_i\).
}

Vediamo una caratteristica esplicita di questa topologia

\sproposition{}{
    Un insieme
    \[
        A \subseteq \bigsqcup_{i\in I} X_i
    \]
    è aperto per la topologia dell'unione disgiunta se e solo se
    \(A \cap X_i\) è aperto in \(X_i\) per ogni \(i \in I\).
}

\sproof{}{
    Definiamo \(\tau\) come la collezione dei sottoinsiemi dati nella proposizione
    tale che \(A \cap X_i\) è aperto in \(X_i\).
    Usando il fatto che su \(X_i\) abbiamo delle topologia possiamo dimostrare
    che tale collezione soddisfa gli assiomi di topologia:
    \begin{enumerate}
        \item Siano \(A, B \in \tau\).
        Allora \(A \cap X_i\) e \(B \cap X_i\) sono entrambi aperti in \(X_i\).
        Di conseguenza la loro unione è ancora aperta in \(X_i\). Possiamo scrivere
        \[
            (A \cap X_i) \cup (B \cap X_i) = (A \cap B) \cap X_i
        \]
        che è appunto aperto.
    \end{enumerate}
    Notiamo che \(\tau\) contiene tutti i sottoinsiemi che sono aperti
    in qualche \(X_i\).
    Infatti, \(A \subseteq X_i\) è aperto di \(X_i\),
    \(A \cap X_i = A\) aperto di \(X_i\)
    e \(A \cap X_j = \emptyset\) aperto di \(X_j\) per \(j \neq i\).
    Quindi \(\tau\) contiene la topologia dell'unione disgiunta
    (per definizione di quest'ultima come topologia generate).
    Viceversa, dato \(A \in \tau\) vogliamo mostrare che \(A\)
    è aperto per la topologia dell'unione disgiunta.
    \begin{align*}
        A &\subseteq \bigsqcup_{i\in I} X_i \\
        A &= A \cap \left(\bigsqcup_{i\in I} X_i\right)
        = \bigsqcup_{i\in I} (A \cap X_i)
    \end{align*}
    che è una disgiunzione di insiemi aperti 
}

Queste due definizioni sono una un po' il duale dell'altra,
da due punti di vista differenti.
Da una parte considerando le applicazioni (ci arriviamo come la topologia più fine),
mentre l'altro è come se costruissimo la topologia dal basso.

Possiamo verificare che questa è effettivamente la topologia più fine che rende queste mappe continue.  
In generale, data una famiglia di applicazioni \(f_i \colon (X_i, \tau_I) \to y\)
si può considerare la topologia più fine che rende le mappe \(f_i\)
continue.
In particolare nel caso di un'unica funzione
la topologia più fine che rende \(f\)
continua è la topologia detta topologia quoziente indotta da \(f\).

\sexample{Topologia di Zaniski}{
    Let \(\mathbb{K}\) be a field
    and consider \(\mathbb{K}[x_1, \cdots, x_n]\).
    Consider the affine space
    given by the cartesian exponent \(\mathbb{K}^n\).
    For each \(f \in \mathbb{K}[x_1, \cdots, x_n]\)
    consider
    \[
        D(f) = \{(a_1, \cdots, a_n) \in \mathbb{K}^n \,|\, f(a_1, \cdots, a_n) \neq 0\}
    \]
    the set \(\{D(f)\}\) is a basis for the topology of \(\mathbb{K}^n\)
    called Zaniski topology.
}

\sproof{Che è una base}{
    Chiaramente \(D(0) = \emptyset\) e \(D(1) = \mathbb{K}^n\).
    The latter already proves that the whole space can be expressed as a union.
    For the intersection, consider
    \begin{align*}
        D(f) \cap D(g) = \{
            \{(a_1, \cdots, a_n) \in \mathbb{K}^n \,|\, f(a_1, \cdots, a_n) \neq 0 \land g(a_1, \cdots, a_n) \neq 0\}
        \}
    \end{align*}
    Siccome un campo è un dominio di integrità la condizione è equivalente a
    \((f \circ g)(a_1, \cdots, a_n) \neq 0\) ma ciò è uguale a \(D(f \circ g)\).
    Quindi \(\{D(f)\}\) forma una base per una topologia sul dato spazio.
}

\sexercise{}{
    Caratterizzare i chiusi di questa topologia.
    Quindi generiamo tutti gli aperti e prendiamo i complementari,
    o usiamo le leggi di de morgan. I chiusi sono generati da un ideale.
}

TODO anche la dimostrazione che la chiusura
è pari ai punti x tali che per ogni U in I(x), U intersection B neq emptyset.

\sdefinition{}{
    Uno spazio topologico si dice \(T_1\) se ogni punto \(\{x\}\)
    (come sottoinsieme dello spazio) è chiuso.
}

Per esempio la retta euclidea.

\sproposition{}{
    Sia \(X\) uno spazio topologico.
    Allora \(X\) è \(T_1\) se e solo se
    \(\forall x\in X\),
    \[
        \bigcap_{U \in I(x)} U = \{x\}
    \]
}

\sproof{}{
    \iffproof{
        Abbiamo ovviamente l'inclusione \(\supseteq\).
        Viceversa, dimostriamo che
        \[
            \bigcap_{U \in I(x)} U \subseteq \{x\}
        \]
        che è euivalente a dire
        \[
            X \backslash \left(
                \bigcup_{U \in I(x)} U
            \right) \geq X \backslash \{x\}
        \]
        Prendiamo quindi un punto \(y \in X \backslash \{x\}\)
        che è come dire \(y\neq x\).
        Siccome lo spazio è \(T_1\),
        abbiamo che \(\{y\}\) è chiuso e quindi
        il suo complementar e \(X \backslash \{x\}\) è un aperto che contiene \(x\)
        in quanto \(x\neq y\).
        Quindi \(X \backslash \{y\} \in I(x)\).
        Ponendo \(U = X \backslash \{y\}\)
        otteniamo quindi che \[y \in X \backslash U = X \backslash (X \backslash \{y\}) = \{y\}\]
    }{
        Applichiamo la caratterizzazione della chiusura del singoletto,
        cioè \(y \in \overline{\{x\}}\) è come dire che per ogni \(U \in I(x)\),
        \(U \cap \{x\} \neq \emptyset\).
        Ma tutto ciò è equivalente a dire che
        \[
            x \in \bigcap_{U \in I(x)} U = \{y\}
        \]
        che è equivalente a dire che \(x=y\).
        Quindi, \(\overline{x} = \{x\}\) e quindi è chiuso.
    }
}

L'operazione di controimmagine fra le topologia presenta tutte le operazioni.

% leobartoli@live.fr

\end{document}
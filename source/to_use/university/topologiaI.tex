\documentclass[a4paper]{article}

\usepackage{amsmath}
\usepackage{amssymb}
\usepackage{stellar}
\usepackage{parskip}
\usepackage{fullpage}
\usepackage{wrapfig}
\usepackage{tikz}

\usetikzlibrary{arrows}
\usetikzlibrary{decorations.pathreplacing}
\usetikzlibrary{cd}

\title{Topologia I}
\author{Paolo Bettelini}
\date{}

\begin{document}

\maketitle
\tableofcontents

\section{Topologia}

Invarianti: \(p_0\) corrisponde al numero di componenti connesse di uno spazio.
Formalmente \(\pi_0(X)\) è l'insieme delle componenti connesse di \(X\) per archi.
Invece, \(p_1\) è il gruppo fondamentale \(\pi_1(X)\), che descrive la struttura dei cammini
chiusi fino a omotopia.

\saxiom{Estensionalità}{
    \[
        A = B \iff \forall x(x\in A \iff x\in B)
    \]
}

% la logica di boole corrisponde alla logica classica come la logica costruttivista corrisponde alle algebre di H

% mettere le leggi di the morgan generalizzate + le due leggi extra (famiglie arbitrarie)

\sproposition{Relazione di aggiunzione}{
    Valgono
    \[
        S \subseteq f^{-1}(T) \iff
        f(S) \subseteq T
    \]
}
Da cui derivano \(f(f^{-1})(T) \subseteq T\).
Ma in generale l'uguaglianza non vale in quanto \(f\) potrebbe non essere suriettiva.
E pure \(S \subseteq f^{-1}(f(S))\).
Ma in generale l'uguaglianza non vale in quanto \(f\) potrebbe non essere iniettiva.

L'operazione di controimmagine preserva tutte le operazioni insiemistiche.
\begin{align*}
    f^{-1} \left(\bigcup_{i\in I} A_i\right) &= \bigcup{i\in I} f^{-1}(A_i) \\
    f^{-1} \left(\bigcap_{i\in I} A_i\right) &= \bigcap{i\in I} f^{-1}(A_i) \\
    X \backslash f^{-1}(T) &= f^{-1}(Y \backslash T)
\end{align*}

L'operazione di immagine preserva in generale solo le unioni.

\begin{align*}
    f\left(\bigcup_{i\in I} A_i\right) &= \bigcup_{i\in I} f(A_i)
\end{align*}
le altre due non valgono necessariamente. Abbiamo solo
\begin{align*}
    f(A\cap B) &\subseteq f(A) \cap f(B) \\
\end{align*}
se \(f\) non è iniettiva la direzione opposta non vale necessariamente.
Infatti potrebbero esistere \(x,x'\) tale che \(x \in A \backslash B\)
e \(x' \in B \backslash A\) tali che \(f(x) = f(x')\).
La medesima logica vale per il complementare.

\sproposition{Proprietà universale del quoziente}{
    Sia \(f\colon X \to Y\) e \(\sim\) relazione di equivalenza su \(X\). Sono equivalenti:
    \begin{enumerate}
        \item \(f\) è costante sulle classi di equivalenza
            \[ x \sim x' \iff f(x) = f(x') \]
        \item \(f\) fattorizza (in modo necessaria unico, essendo \(\pi\) suriettivo) attraverso
            \(\pi\), cioè \(\exists_{=1} \,g \colon X/_\sim \to Y\) tale che \(g \circ \pi = f\).
    \end{enumerate}
}
\begin{center}
    % https://tikzcd.yichuanshen.de/#N4Igdg9gJgpgziAXAbVABwnAlgFyxMJZABgBpiBdUkANwEMAbAVxiRAA0QBfU9TXfIRQBGclVqMWbAJrdeIDNjwEiZYePrNWiDgHoA+gB1D2ALbdxMKAHN4RUADMAThHOIyIHBCQAmapqkdYzQsEGoGOgAjGAYABX5lIRAnLGsACxw5Rxc3Dy8kUQktNgcskGdXX2p8xELosCgkAFoAZg8A7RBrMJAI6LiEwTYU9MyuCi4gA
    \begin{tikzcd}
    X \arrow[d, "\pi"'] \arrow[r, "f"]   & Y \\
    X/_\sim \arrow[ru, "g"', bend right] &  
    \end{tikzcd}
\end{center}

\sproof{}{
    \begin{enumerate}
        \item \((2) \implies (1):\) \(f = g \circ \pi\).
        Abbiamo \[
            x \sim x' \implies
            \pi(x) = \pi(x') \implies g(\pi(x)) = g(\pi(x'))
        \]
        che sono uguali a \(f(x)\) e \(f(x')\).
        \item \((2) \implies (1):\)
        Definiamo \(g\colon X/_\sim \colon Y\) come
        \[
            g([x]) \triangleq f(x)
        \]
        bisogna verificare che sia ben posta.
        Vogliamo quindi che se \([x] = [x']\) allora \(f(x) = f(x')\).
        Ma ciò è garantito dalla ipotesi.
    \end{enumerate}
}

In \(\mathbb{R}^n\).
\[
    d_\infty(x,y) \leq d_2(x,y) \leq d_1(x,y) \leq n \cdot d_\infty(x,y)
\]

\section{08 ottobre 2025 DA METTERE}

\sexercise{}{
    Le topologie con la proprietà che le intersezioni arbitrari di
    aperti sono aperti, possono essere
    caratterizzate esplicitamente.
    Essi sono esattamente, a meno di omeomorfismo,
    gli spazi topologici della seguente forma: dato un insieme preordinato \((P, \leq)\),
    la topologia di Alexandrov \(\mathcal{A}_{\mathcal{P}}\)
    su \(\mathcal{P}\) è la topologia i cui aperti sono i sottoinsiemi
    \(U \subseteq \mathcal{P}\) tale che
    \(\forall p \leq q, p \in U \implies q \in U\).
}

\subsection{Generated topology}

\sproposition{}{
    Data una collezione di topologie \(\{\tau_i\}_{i \in I}\)
    su un isieme \(X\). La famiglia
    \[
        \tau = \bigcup_{i\in I} \tau_i
    \]
    è ancora una topologia su \(X\).
}

\scorollary{}{
    Sia \(X\) un insieme e \(S \subseteq \mathcal{P}(X)\) famiglia fi sottoinsiemi.
    Esiste la topologia meno fine su \(X\) che contiene i sottoinsiemi in \(S\)
    come aperti.
    Tale topologai viene detta la topologia generata da \(S\).
}

\sdefinition{Topologia dell'unione disgiunta}{
    Sia \(\{X_i \,|\, i \in I\}\) una famiglia di spazi topologici.
    Allora lo spazio topologico è definita come
    \[
        \bigsqcup_{i\in I} X_i
    \]

    Possiamo definire astrattamente la topologia dell'unione disgiunta su
    \(\sqcup X_i\) come a topologia più fine che rende tutte le mappe \(\tau_i \colon X_i \to \bigsqcup X_i\)
    continue.

    Alternativamente, possiamo definire la topologia come la topologia generata dalla famiglia di sottoinsiemi
    dell'insieme \(\sqcup X_i\)
    che sono aperti in qualcuno degli \(X_i\).
}

Vediamo una caratteristica esplicita di questa topologia

\sproposition{}{
    Un insieme
    \[
        A \subseteq \bigsqcup_{i\in I} X_i
    \]
    è aperto per la topologia dell'unione disgiunta se e solo se
    \(A \cap X_i\) è aperto in \(X_i\) per ogni \(i \in I\).
}

\sproof{}{
    Definiamo \(\tau\) come la collezione dei sottoinsiemi dati nella proposizione
    tale che \(A \cap X_i\) è aperto in \(X_i\).
    Usando il fatto che su \(X_i\) abbiamo delle topologia possiamo dimostrare
    che tale collezione soddisfa gli assiomi di topologia:
    \begin{enumerate}
        \item Siano \(A, B \in \tau\).
        Allora \(A \cap X_i\) e \(B \cap X_i\) sono entrambi aperti in \(X_i\).
        Di conseguenza la loro unione è ancora aperta in \(X_i\). Possiamo scrivere
        \[
            (A \cap X_i) \cup (B \cap X_i) = (A \cap B) \cap X_i
        \]
        che è appunto aperto.
    \end{enumerate}
    Notiamo che \(\tau\) contiene tutti i sottoinsiemi che sono aperti
    in qualche \(X_i\).
    Infatti, \(A \subseteq X_i\) è aperto di \(X_i\),
    \(A \cap X_i = A\) aperto di \(X_i\)
    e \(A \cap X_j = \emptyset\) aperto di \(X_j\) per \(j \neq i\).
    Quindi \(\tau\) contiene la topologia dell'unione disgiunta
    (per definizione di quest'ultima come topologia generate).
    Viceversa, dato \(A \in \tau\) vogliamo mostrare che \(A\)
    è aperto per la topologia dell'unione disgiunta.
    \begin{align*}
        A &\subseteq \bigsqcup_{i\in I} X_i \\
        A &= A \cap \left(\bigsqcup_{i\in I} X_i\right)
        = \bigsqcup_{i\in I} (A \cap X_i)
    \end{align*}
    che è una disgiunzione di insiemi aperti 
}

Queste due definizioni sono una un po' il duale dell'altra,
da due punti di vista differenti.
Da una parte considerando le applicazioni (ci arriviamo come la topologia più fine),
mentre l'altro è come se costruissimo la topologia dal basso.

Possiamo verificare che questa è effettivamente la topologia più fine che rende queste mappe continue.  
In generale, data una famiglia di applicazioni \(f_i \colon (X_i, \tau_I) \to y\)
si può considerare la topologia più fine che rende le mappe \(f_i\)
continue.
In particolare nel caso di un'unica funzione
la topologia più fine che rende \(f\)
continua è la topologia detta topologia quoziente indotta da \(f\).

\sexample{Topologia di Zaniski}{
    Let \(\mathbb{K}\) be a field
    and consider \(\mathbb{K}[x_1, \cdots, x_n]\).
    Consider the affine space
    given by the cartesian exponent \(\mathbb{K}^n\).
    For each \(f \in \mathbb{K}[x_1, \cdots, x_n]\)
    consider
    \[
        D(f) = \{(a_1, \cdots, a_n) \in \mathbb{K}^n \,|\, f(a_1, \cdots, a_n) \neq 0\}
    \]
    the set \(\{D(f)\}\) is a basis for the topology of \(\mathbb{K}^n\)
    called Zaniski topology.
}

\sproof{Che è una base}{
    Chiaramente \(D(0) = \emptyset\) e \(D(1) = \mathbb{K}^n\).
    The latter already proves that the whole space can be expressed as a union.
    For the intersection, consider
    \begin{align*}
        D(f) \cap D(g) = \{
            \{(a_1, \cdots, a_n) \in \mathbb{K}^n \,|\, f(a_1, \cdots, a_n) \neq 0 \land g(a_1, \cdots, a_n) \neq 0\}
        \}
    \end{align*}
    Siccome un campo è un dominio di integrità la condizione è equivalente a
    \((f \circ g)(a_1, \cdots, a_n) \neq 0\) ma ciò è uguale a \(D(f \circ g)\).
    Quindi \(\{D(f)\}\) forma una base per una topologia sul dato spazio.
}

\sexercise{}{
    Caratterizzare i chiusi di questa topologia.
    Quindi generiamo tutti gli aperti e prendiamo i complementari,
    o usiamo le leggi di de morgan. I chiusi sono generati da un ideale.
}

TODO anche la dimostrazione che la chiusura
è pari ai punti x tali che per ogni U in I(x), U intersection B neq emptyset.

\sdefinition{}{
    Uno spazio topologico si dice \(T_1\) se ogni punto \(\{x\}\)
    (come sottoinsieme dello spazio) è chiuso.
}

Per esempio la retta euclidea.

\sproposition{}{
    Sia \(X\) uno spazio topologico.
    Allora \(X\) è \(T_1\) se e solo se
    \(\forall x\in X\),
    \[
        \bigcap_{U \in I(x)} U = \{x\}
    \]
}

\sproof{}{
    \iffproof{
        Abbiamo ovviamente l'inclusione \(\supseteq\).
        Viceversa, dimostriamo che
        \[
            \bigcap_{U \in I(x)} U \subseteq \{x\}
        \]
        che è euivalente a dire
        \[
            X \backslash \left(
                \bigcup_{U \in I(x)} U
            \right) \geq X \backslash \{x\}
        \]
        Prendiamo quindi un punto \(y \in X \backslash \{x\}\)
        che è come dire \(y\neq x\).
        Siccome lo spazio è \(T_1\),
        abbiamo che \(\{y\}\) è chiuso e quindi
        il suo complementar e \(X \backslash \{x\}\) è un aperto che contiene \(x\)
        in quanto \(x\neq y\).
        Quindi \(X \backslash \{y\} \in I(x)\).
        Ponendo \(U = X \backslash \{y\}\)
        otteniamo quindi che \[y \in X \backslash U = X \backslash (X \backslash \{y\}) = \{y\}\]
    }{
        Applichiamo la caratterizzazione della chiusura del singoletto,
        cioè \(y \in \overline{\{x\}}\) è come dire che per ogni \(U \in I(x)\),
        \(U \cap \{x\} \neq \emptyset\).
        Ma tutto ciò è equivalente a dire che
        \[
            x \in \bigcap_{U \in I(x)} U = \{y\}
        \]
        che è equivalente a dire che \(x=y\).
        Quindi, \(\overline{x} = \{x\}\) e quindi è chiuso.
    }
}

\sdefinition{Definizione di convesso in \(\mathbb{R}^n\)}{
    .
}

Sono proprietà topologiche \(T_1\), proprietà di Hausdorff, connessione,
connessione per archi.

% leobartoli@live.fr

\pagebreak

\section{Lezione del 23}

L'operazione di controimmagine fra le topologia presenta tutte le operazioni insiemistiche.
Sezione sugli invarianti topologici? (Sposando anche la definizione di quest'ultima).

\slemma{}{
    Sia \(f \colon X \to Y\) un omeomorfismo.
    Allora, per ogni sottospazio \(S \subseteq X\),
    la restrizione di \(f\) a \(S\) è un omeomorfismo.
}

\sproof{}{
    Un omeomorfismo è un coperazione continua biettica con inverso continua.
    Inoltre vi è la proprietà universale della topologia di sottospazio.
}

\sexample{}{
    L'intervallo \((0,1)\) e \([0,1)\) non sono omeomorfi.
}

\sproof{}{
    Supponiamo che esiste un tale omeomorfismo \(f \colon [0,1) \to (0,1)\).
    Abbiamo che \(f(0) \in (0,1)\).
    Prendiamo \(S = [0,1) \backslash \{0\} = (0,1)\).
    Then, \(f\) restricted to \(S\) is a homeomorphism
    from \(S\) to \[
        f(S) = (0,1) \backslash \{f(0)\} = (0, f(0)) \sqcup (f(0), 1)
    \]
    But \((0,1)\) is connected and \(f(S)\) is not, which is absurd.
}

\sexample{}{
    Sia \(f \colon S^n \to \mathbb{R}\) continua.
    Allora esiste \(x\) tale che \(f(x) = f(-x)\), in particolare non è iniettiva.
}

\sproof{}{
    Sia \(g \colon S^n \to \mathbb{R}\) data da \(g(x) = f(x) - f(-x)\).
    Chiaramente \(g\) è continua. Chiaramente \(g(x) = 0\)
    se e solo se \(f(x) = f(-x)\).
    \(S^n\) è connesso per archi e quindi connesso.
    Allora \(g(S^n)\) è connesso nei reali, ovvero è un intervallo.
    Let \(y \in S^n\).
    Then, \(g(y), g(-y) \in g(S^n)\). Consider
    \[
        \frac{1}{2}g(y) + \frac{1}{2}g(-y) \in g(S)
    \]
    Then
    \[
        \frac{1}{2}\left(
            f(y) - f(-y)
        \right) + \frac{1}{2}\left(
            f(-y) - f(y)
        \right) = 0
    \]
}

\scorollary{}{
    Aperti di \(\mathbb{R}\) non sono omeomorfi ad aperti di \(\mathbb{R}^n\)
    per \(n>1\).
}

\sproof{}{
    Ogni aperto di \(\mathbb{R}^n\) contiene al suo interno
    un sottospazio omeomorfo a \(S^{n-1}\).
    Suppose that there is a homeomorphism
    \(f \colon A \to f(A)\) where \(A\) is open in \(\mathbb{R}^n\)
    and \(f(A)\) is open in \(\mathbb{R}\).
    La reistrizione di \(f\) ad un sottospazio \(B \cong S^{n-1}\)
    è ancora un omeomorfismo. Ma dal risultato precedente non può esistere una tale applicazione biettiva, assurdo.
}

\slemma{}{
    Sia \(f \colon X \to Y\)
    un applicazione continua e suriettiva verso \(Y\) connesso
    e \(\forall y \in Y\), \(f^{-1}(y)\) connesso.
    Allora, se \(f\) è aperta oppure chiusa, \(X\) è connesso.
}

\sproof{}{
    Supponiamo che \(f\) sia aperta senza perdita di generalità.
    Prendiamo \(A_1, A_2\) aperti non vuoti di \(X\)
    tale che \(X = A_1 \cup A_2\).
    Vogliamo mostrare che sono disgiunti.
    Abbiamo che \(Y = f(X) = f(A_1) \cup f(A_2)\).
    Siccome \(Y\) è connesso, la loro intersezione non può essere vuota.
    Abbiamo quindi almeno un punto \(y \in A_1 \intersection A_2\).
    Consideriamo
    \[
        f^{-1}(y) \intersection A_1 \neq \emptyset
        \qquad
        f^{-1}(y) \intersection A_2 \neq \emptyset
    \]
    Per ipotesi
    \[
        f^{-1}(y) = (f^{-1}(y) \intersection A_1)
        \union (f^{-1}(y) \intersection A_2)
    \]
    è aperto. Quindi
    \[
        (f^{-1}(y) \intersection A_1) \intersection (f^{-1}(y) \intersection A_2)
        \neq \emptyset
    \]
    quindi \(A_1 \intersection A_2 \neq \emptyset\).
}

\stheorem{}{
    Siano \(X, Y\) due spazi topologici coneessi.
    Allora, \(X \times Y\) è connesso.
}

\sproof{}{
    Applichiamo il lemma.
    Prendiamo \(p: X \times Y \to Y\) una delle due proiezioni.
    Applichiamo il lemma. \(P\) è aperta (dai risultato sulla topologia prodotto).
    Inoltre \(P\) è continua e suriettiva (se l'insieme \(X\) non è vuoto, in tal caso
    il risultato è banale). Consideriamo la fibra
    \(p^{-1}(y) = X \times \{y\}\) che è omeomorfo ad \(X\), che è connesso.
    Quindi le ipotesi sono soddisfatte e \(Y \times X\) è connesso.
}

Lo stesso vale per la connessione per archi.

\sproof{}{
    Siano \(X,Y\) connessi per archi.
    Allora, \(X \times Y\) è connesso per archi.
}

\sproof{}{
    Mostriamo che ogni coppia di punti è collegata da un cammino.
    Siano quindi \((x,y), (x',y') \in X \times Y\).
    Siccome \(X\) è connesso per archi, esiste un cammino \(\alpha\colon I \to X\)
    tale che \(\alpha(0) = x\) e \(\alpha(1) = x'\).
    Analogamente \(\beta(0) = y\) e \(\beta(1) = y'\).
    Per la proprietà universale del prodotto con \(I\) come vertice,
    esiste un cammino (unico) che fattorizza il diagramma tale che i due triangoli commutino
    mediante le proiezioni.
    Quindi \((\alpha, \beta) \colon I \to X \times Y\)
    dato da \((\alpha, \beta)(t) = (\alpha(t), \beta(t))\).
}

\sdefinition{Componente connessa}{
    Sia \(X\) spazio topologico.
    Un sottospazio \(C\) di \(X\)
    si dice una componente connessa se soddisfa le seguenti:
    \begin{enumerate}
        \item \(C\) è un sottospazio connesso
        \item se \(C \subseteq A\) e \(A\) è connesso allora \(C=A\).
    \end{enumerate}
}

\sexample{}{
    Sia \(X\) uno spazio e \(C \subseteq X\).
    Se \(C\) è sottospazio aperto, chhiuso, connesso e non vuoto, allora è componente connessa.
    Ciò è dato dal fatto che \(C\) è anche chiuso e aperto in \(A\).
}

\slemma{}{
    Sia \(Y\) un sottospazio connesso di \(X\).
    Sia \(W\) un sottospazio tale che \(Y \subseteq W \subseteq \overline{Y}\).
    Allora \(W\) è connesso.
}

\sproof{}{
    Sia \(Z \subseteq W\) aperto, chiuso e non vuoto.
    Consideriamo \(Z \intersection Y\).
    Questo è aperto e chiuso di \(Y\) per
    "transitività" della topologia di sottospazio,
    (cioè se abbiamo una successione di spazi possiamo indurre la topologia di sottospazi in un colpo solo oppure a step).
    Sappiamo che
    \[
        \overline{Y} = \{x \in X \suchthat \text{for every open } A \text{ of } X \text{ where } x \in A, A \intersection Y \neq \emptyset\}
    \]
    Siccome \(Z \neq \emptyset\) è aperto in \(W\),
    \(Z = A \intersection W\) per qualche \(A\) aperto di \(X\).
    Quindi, \(\forall x \in Z \subseteq A\), \(A \intersection Y \neq \emptyset\)
    (e quindi possono prendere un \(x\in Z\)).
    Siccome \(Y\) è connesso,
    deduciamo che \(Z \intersection Y = Y\).
    Infatti quest'ultima intersezione è aperta e chiusa in \(Y\) per definizione di
    topologia di sottospazio e l'intersezione non è vuota.
    Quindi, \(Y \subseteq Z\). Consideriamo la chiusura (in \(W\)) di entrambi
    \(\overline{Y} \subseteq \overline{Z}\).
    Quindi la chiusura in \(W\) è pari a \(\overline{Y} \intersection W\)
    e l'altro \(\overline{Z}=Z\) siccome \(Z\) è chiuso in \(W\) per ipotesi.
    Quindi \(\overline{Y} \subseteq Z\).
    Siccome \(Z \subseteq W\), abbiamo \(W=Z\).
}

\slemma{}{
    Sia \(x\) un punto di uno spazio topologico \(X\)
    e sia \(\{Z_i\}_{i\in I}\) una famiglia di sottospazio connessi di \(X\)
    tali che \(x\in Z_i\).
    Allora, \(\bigcup_i Z_i\) è un sottospazio connesso.
}

\sproof{}{
    Sia
    \[
        W = \bigcup_{i=1} Z_i
    \]
    Dato \(A \subseteq W\) aperto, chiuso e non vuoto,
    vogliamo mostrare che \(A=W\).
    Per ogni \(i \in I\), andiamo a considerare
    \(A \intersection Z_i\) che è un aperto e chiuso di \(Z_i\)
    per definizione di sottospazio.
    Siccome \(Z_1\) è connesso, \(A \intersection Z_i = \emptyset\)
    oppure \(A \intersection Z_i = Z_i\).
    Quest'ultimo è equivalente a dire che \(Z_i \subseteq A\).
    Supponiamo \(A \neq \emptyset\). Quindi
    \[
        A = \bigcup_{i\in I} Z_i \intersection Z_i
    \]
    Allora esiste almeno un \(i \in I\) tale che \(A \intersection Z_i \neq \emptyset\),
    cioè \(Z_i \subseteq A\).
    Dato il punto \(x\) come nelle ipotesi del lemma,
    abbiamo \(x \in Z_i\).
    Segue quindi che \(x \in A\).
    Allora \(x \in Z_i \intersection A\) per ogni \(i \in I\).
    Per ipotesi\(x\in Z_i\) per tutte le \(i\).
    Quindi, \(Z_i \intersection A \neq \emptyset \iff Z_i \subseteq A\), allora
    \[
        W = \bigcup_{i\in I} Z_i \subseteq A
    \]
    cioè \(A = W\).
}

\scorollary{}{
    Siano \(A,B\) due sottospazi connessi di uno spazio topologico.
    Allora, se \(A \intersection B \neq \emptyset\), \(A \union B\) è connesso.
}

\sproof{}{
    Applichiamo il lemma al caso della famiglia \(\{A,B\}\).
}

\slemma{}{
    Sia \(x \in X\) un punto di uno spazio topologico \(X\).
    Denotiamo \(C(x)\) l'unione di tutti i sottospazi connessi di \(X\)
    ch contengono il punto \(x\).
    Allora \(C(x)\) è una componente connessa di \(X\) contenente il punto \(x\).
}

\sproof{}{
    \begin{enumerate}
        \item \(C(x)\) è connesso per il lemma;
        \item \(C(x) \subseteq A\) con \(A\) connesso.
        \(\{x\}\) è un sottospazio connesso di \(X\) contenente \(x\).
        Per definizione \(\{x\} \in C(x)\).
        Abbiamo allroa che \(x \in C(x) \subseteq A\) e quindi \(x\in A\).
        Ma \(A\) è connesso, quindi \(A \subseteq C(x)\).
        Allora \(A = C(x)\).
    \end{enumerate}
}

\stheorem{}{
    Ogni spazio topologico è unione delle sue componenti connesse.
    Ogni componente connessa è chiusa
    e ogni punto è contenuto in una e una sola componente connessa.
}

\sproof{}{
    Siccome \(x\in C(x)\),
    \[
        X = \bigcup_{x\in X} C(x)
    \]
    Sia \(C\) componente connessa. Da un risultato precedente,
    sappiamo che \(\overline{C}\) è ancora un connesso.
    Tuttavia \(C \subseteq \overline{C}\) e per la seconda condizione
    nella definizione di componente connessa, ci deve essere uguaglianza.
    Siano \(C,D\) componenti connesse.
    Supponiamo che non siano disgiunte.
    Abbiamo che \(C \union D\) è connesso in quanto unione dei connessi
    \[C,D \subseteq C \union D\]
    Ciò implica che \(C = C \union D\) e \(D = C \union D\).
    Allora \(C=D\)
}
Questo teorema giustifica la seguente definizione.

\sdefinition{}{
    Sia \(X\) uno spazio topologico e \(x\in X\).
    Allora \(C(x)\) è detta la componente connessa.
}

\sdefinition{Compattezza}{}



\sdefinition{
    Dato \(X\) spazio topologico,
    il \(\pi_0(X)\) è definito
    come l'insieme delle componenti connesse per archi di \(X\).
}

\stheorem{}{
    Sia \(f \colon X \to Y\) un'applicazione contonua.
    Allora se \(X\) è compatto, \(f(X)\) è compatto come sottospazio di \(Y\).
}

\sproof{}{
    Sia \(\mathcal{A}\) una famiglia di aperti di \(Y\) tale che
    \[
        f(X) \subseteq \bigcup_{A \in \mathcal{A}} A
    \]
    Allora \[X = f^{-1}(f(X)) = \bigcup_{A \in \mathcal{A}} f^{-1}(A)\]
    cioè unione di aperti di \(X\) siccome \(f\) è continua.
    Siccome \(X\) è compatto, esistono \(A_1, \cdots, A_n\)
    tali che \(X = f^{-1}(A_1) \union \cdots \union f^{-1}(A_n)\).
    Applicando \(f\) otteniamo
    \[
        f(X) = f(f^{-1}(A_1)) \union \cdots \union f(f^{-1}(A_n))
    \]
    e quindi \(f(X) \subseteq A_1 \union \cdots \union A_n\) che è un ricoprimento finito.
}

\sproposition{}{
    Ogni sottospazio chiuso di uno spazio compatto è compatto.
}

\sproof{}{
    Sia \(X\) compatto e \(C\) chiuso in \(X\).
    Sia \(\mathcal{A}\) una famiglia di aperti tali che
    \[
        C \subseteq \bigcup_{A \in \mathcal{A}} A
    \]
    Notiamo che \(C = (X \backslash C) \union \bigcup A\), dove il primo termine è aperto in quanto complementare
    di un chiuso.
    Essendo \(X\) compatto,
    esistono \(A_1, \cdots, A_n\)
    tali che
    \[
        X = (X \backslash C) \union \bigcup_{i=1}^n A_i
    \]
    quindi
    \[
        C = C \intersection X = (C \intersection (X \backslash C))
        \union \bigcup (A_i \intersection C)
    \]
    ma il primo termine è l'insieme vuoto quindi \(C \subseteq A_1 \union \cdots \union A_n\).
}

\sproposition{}{
    Unione finita di sottospazi compatti è compatta.
}

\sproof{}{
    Sia \(X\) spazio topologico e \(K_1, \cdots, K_n\)
    sottospazi compatti di \(X\).
    Sia
    \[
        K = \bigcup_{i=1}^n K_i
    \]
    Vogliamo dimostrare che \(K\) è compatto.
    Siccome i \(K_i\) sono compatti, possiamo estrarre
    dei sottoricoprimenti finiti da essi. Ma l'unione di questi
    sottoricoprimenti finiti è un sotoricoprimento finito,
    in quanto unioni finite di sottoinsiemi finiti sono sottoinsiemi finiti.
}

\stheorem{}{
    L'intervallo unitario è compatto
    rispetto alla topologia euclidea reale.
}

\sproof{}{
    Sia \(\mathcal{A}\) un ricoprimento aperto di \([0,1]\).
    Quindi
    \[
        [0,1] \subseteq \bigcup_{A \in \mathcal{A}} A
    \]
    Definiamo il sottoinsieme \(X \subseteq [0, \infty)\) come seguente:
    \[
        X = \{t \in [0, +\infty) \suchthat [0,t] \text{ is contained in a finite union of open sets in } \mathcal{A}\}
    \]
    La tesi equivalente a dire che \(1 \in X\).
    Note that \(X \neq \emptyset\) as \(0 \in X\).
    Questo è dato dal fatto che \(0 \in [0,1] \subseteq \bigcup A\).
    Consideriamo ora il supremum di \(X\).
    Abbiamo due possibilità, o \(\sup X > 1\) oppure \(\sup X \leq 1\).
    Nel secondo caso esiste quindi \(t \in X\)
    tale che \(1 < t < \sup X\),
    quindi \([0,t]\) è un sottoinsieme di un unione finita di aperti di \(\mathcal{A}\).
    In particolare \([0,1]\) è uno di questi quindi vale alla tesi.
    Supponiamo invece l'altro caso dove \(b = \sup x > 1\).
    Se \(b \leq 1\) allora \(b\in [0,1]\), quindi esiste \(A \in \mathcal{A}\)
    tale che \(b \in A\) (\(b \geq 0\) in quanto \(0 \in X\)).
    Essendo \(A\) aperto per la topologia euclidea reale, esiste \(\delta>0\)
    tale che \((b-\delta, b+\delta) \subseteq A\).
    D'altra parte, essendo \(b\) il supremum, esiste \(t \in X\)
    tale che \(t \in (b- \delta, b + \delta)\).
    Siccome \(t \in X\), per definizione di \(X\),
    \[
        [0,t] \subseteq A_1 \union \cdots \union A_n
    \]
    Per ogni \(h\) tale che \(0 \leq h < \delta\),
    \[
        [0, b + h] = [0,t] \union [t, b+h]
    \]
    Il primo termine è in \(A_1 \union \cdots \union A_n\)
    mentre il secondo è in \(A \in \mathcal{A}\).
    Questo contraddice il fatto che \(b\) sia il supremum,
    in quanto ciò implicherebbe che \(b + h \in X\).
}

Quindi \(\realnumbers\) e \([0,1]\) non sono omeomorfi.

\scorollary{}{
    Un sottospazio reale è compatto se e solo se
    è chiuso e limitato.
}

\sproof{}{
    \iffproof{
        Sia \(A \subseteq \realnumbers\) compatto.
        Mostriamo che è limitato.
        Consideriamo il ricoprimento aperto \(\{(-n, n) \suchthat n \in \mathbb{N}\}\)
        che ricopre \(\realnumbers\).
        Allora,
        \[
            A \subseteq \bigcup_{n \in \mathbb{N}} (-n, n)
        \]
        Essendo \(A\) compatto, è possibile estrarre un sottoricoprimento finito.
        Quindi, \(A \subseteq [-N, N]\) per qualche \(N > 0\), quindi è limitato.
        Mostriamo ora che \(A\) è chiuso, in particolare mostriamo che \(\overline{A} \subseteq A\).
        Quindi se \(X \backslash A \subseteq X \backslash \overline{A}\),
        ovvero che se \(p \notin A\) allora \(p \notin \overline{A}\).
        Se \(p \notin A\) abbiamo una funzione continua
        da \(A \subseteq \realnumbers \backslash \{p\}\) in \(\realnumbers\)
        data da \(f(x) = 1 / (x - p)\).
        Se \(f\) è continua, siccome \(A\) è compatto, allora \(f(A)\) è compatto,
        quindi anche limitato come dimostratosi prima.
        Questo implica chiaramente che \(p \notin A\).
    }{
        Supponiamo che \(A \subseteq \realnumbers\) sia chiuso e limitato.
        Siccome è limitato, \(A \subseteq [-a, a]\) per qualche \(a \geq 0\).
        Ma \([-a, a]\) è omeomofo a \([0,1]\) quindi è compatto.
    }
}

\sproposition{}{
    Un sottospazio \(K\) di uno spazio topologico \(X\)
    è compatto (per la topologia di sottospazio)
    se e solo se per ogni famiglia \(\mathcal{A}\) di aperti di \(X\) tali che
    \[
        K \subseteq \bigcup_{A \in \mathcal{A}} A
    \]
    esistono \(A_1, \cdots, A_n \in \mathcal{A}\)
    tali che
    \[
        K \subseteq A_1 \subseteq \cdots \subseteq A_n
    \]
}

\sproof{}{
    \(K\) è compatto per la topologia di sottospazio
    ogni ricoprimento aperto \(\mathcal{B}\)
    di \(K\) ammette un sottoricoprimento finito dove
    \(\forall B \in \mathcal{B}\) con \(B = K \intersection A\) con \(A\) aperto di \(X\) vale
    \[
        K = \bigcup_{B\in\mathcal{B}} B \iff K \subseteq \bigcup_{A \intersection K \in \mathcal{B}} A
    \]
    con \(A\) aperto.
    Quindi \(K = B_1 \union \cdots \union B_n\)
    se e solo se \(K \subseteq A_1 \union \cdots \union A_n\),
    dove per ogni \(i=1,2,\cdots, n\), \(B_i \subseteq A_i \intersection K\).
}

Notiamo che ogni insieme finito è compatto per qualunque topologia.

\scorollary{Teorema di Weierstrass}{
    Sia \(X\) uno spazio topologico compatto e \(f \colon X \to \realnumbers\)
    un applicazione continua. Allora \(f\) ammette massimo e minimo.
}

\sproof{}{
    Siccome \(f\) è continua ed \(X\) è compatto, \(f(X)\) è compatto nei reali.
    Ma allora è chiuso e limitato, quindi se consideriamo infimum e supremum
    sono sicuramente numeri reali.
    Dalla definizione di chiusura, infimum e supremum stanno sempre
    nella chiusura. Ma visto che \(f(X)\) è chiuso, choincide con la sua chiusura, quindi
    infimum e supremum stanno nell'insieme stesso, quindi corrispondono a massimo e minimo.
}

\sproposition{}{
    Sia \(\mathcal{B}\) una base di uno spazio topologico \(X\).
    Supponiamo che ogni ricoprimento aperto di \(X\) formato da elementi
    in \(\mathcal{B}\) ammetta un sottoricoprimento finito.
    Allora, \(X\) è compatto.
}

\sproof{}{
    Sia \(\mathcal{A}\) un ricoprimento aperto di \(X\)
    \[
        X = \bigcup_{A \in \mathcal{A}} A
    \]
    Vogliamo dimostrare che da questo ricoprimento si può estrarre un sottoricoprimento finito.
    Per ogni \(A \in \mathcal{A}\), consideriamo \(\mathcal{B}_A = \{B \in \mathcal{B} \suchthat B \subseteq A\}\).
    Chiaramente per definizione di base, \(A = \bigcup \mathcal{B}_A\).
    Quindi
    \[
        X = \bigcup_{A \in \mathcal{A}} A =
        \bigcup_{A \in \mathcal{A}}
        \bigcup_{B \in \mathcal{B}_A} B
    \]
    quindi esistono \(B_1 \in \mathcal{B}_{A_1}, \cdots\)
    tale che
    \[
        X = B_1 \union \cdots \union B_n
    \]
}

\stheorem{}{
    Sia \(f \colon X \to Y\) un'applicazione chiusa, \(Y\) spazio
    compatto, le fibre \(f^{-1}(y)\) compatte. Allora, \(X\) è compatto.
}

\sproof{}{
    Dato \(A \subseteq X\) consideriamo \(A' \subseteq Y\) definito come seguente:
    \[
        A' = \{y \in Y \suchthat f^{-1}(y) \subseteq A\}
    \]
    Mostriamo alcune proprietà:
    \begin{enumerate}
        \item \(Y \backslash Y' = f(X \backslash A)\):
        abbiamo che
        \begin{align*}
            Y \backslash Y' &= \{y \in Y \suchthat \lnot (f^{-1}(y) \subseteq A)\} \\
            &= \{y\in Y \suchthat \exists x \in f^{-1}(y) \suchthat x \notin A\} \\
            &= f(X \backslash A)
        \end{align*}
        \item \(f^{-1}(A') \subseteq A\)
        sia \(x \in f^{-1}(A')\).
        Quindi dire che \(f(x) \in A'\) è come dire
        \(f^{-1}(f(x)) \subseteq A\).
    \end{enumerate}
    Mostriamo che se \(A\) è aperto in \(X\) allora \(A'\) è aperto in \(Y\).
    \(A'\) è aperto in \(Y\) se e solo se \(Y \backslash Y'\) è chiuso in \(Y\).
    Ma \(Y \backslash A' = f(X \backslash A)\) è chiuso in quanto immagine di un chiuso.
    Prendiamo quindi \(\mathcal{A}\) come ricoprimento aperto di \(X\).
    Definiamo \(\mathcal{B}\)
    come la famiglia dei sottoinsiemi di \(X\) esprimibili come
    unioni finite di aperti in \(A\).
    Consideriamo la famiglia \(\mathcal{B}' = \{B' \suchthat B \in \mathcal{B}\}\).
    Mostriamo che questa famiglia è un ricoprimento (aperto)
    di \(Y\).
    Dato \(y\in y\), consideriamo \(f^{-1}(y)\) che è compatto per ipotesi.
    \[
        f^{-1}(y) \subseteq \bigcup_{A \in \mathcal{A}} A
    \]
    quindi
    \(f^{-1}(y) \subseteq A_1 \union \cdots \union A_n\) per qualche \(A_i\).
    Ponendo \(B = A_1 \union \cdots \union A_n\) abbiamo per definizione di \(\mathcal{B}'\)
    che \(y \in \mathcal{B}'\). La compattezza di \(Y\) implica quindi che esistano
    \(B_1', \cdots, B_n'\) tali che \(Y = B_1' \union \cdots \union B_n'\)
    e quindi \[
        X = f^{-1}(Y) = f^{-1}(B_1') \union \cdots \union f^{-1}(B_n')
    \]
    Usando la seconda proprietà dimostrata prima, troviamo
    \[
        X = B_1 \union \cdots \union B_n
    \]
    visto che sono tutte unione finite di aperti in \(\mathcal{A}\),
    allora \(X\) è esprimibile come unione finita di aperti in \(\mathcal{A}\).
}

\subsection{Verso la compattificazione di Alexandrov}

\sproposition{}{
    Siano \(K_1 \supseteq K_2 \supseteq \cdots K_n \supseteq K_{n+1} \supseteq \cdots\)
    una catena discendente numerabile di chiusi non vuoti
    e compatti di uno spazio topologico ambiente. Allora,
    \[
        \bigcap_{n\in\mathbb{N}} K_n \neq \emptyset
    \]
}

\sproof{}{
    \(K_n\) è chiuso in \(K_1\) in quanto
    \(K_n = K_n \cap K_1\) e \(K_n\) è chiuso nello spazio ambiente.
    Quindi i complementari \(K_1 \backslash K_n\) è aperto.
    Supponiamo che l'intersezione sia vuota
    \[
        \bigcap_{n\in\mathbb{N}} K_n = \emptyset
    \]
    Ciò è equivalente a dire che
    \begin{align*}
        K_1 \difference \left(\bigcap_{n\in\mathbb{N}} K_n\right) &= K_1 \difference \emptyset \\
        \bigcup_{n\in\mathbb{N}} (K_1 \difference K_n) = K_1
    \end{align*}
    quindi i sottoinsieme \(\{K_1 \difference K_n\}\)
    formerebbero un ricoprimento aperto di \(K_1\),
    dal quale, essendo \(K_1\) compatto, si dovrebbe poter estrarre un sottoricoprimento
    finito \(\{K_1 \difference K_{J_1}, \cdots, K_1 \difference K_{J_n}\}\).
    Avremmo quindi che
    \begin{align*}
        K_1 &= \bigcup K_1 \difference K_{J_i} \\
        K_1 &= \bigcap K_{J_i} = \emptyset \\
        K_1 &= K_{\max \{J_i\}}
    \end{align*}
    che è assurdo in quanto tutti i \(K\) sono non-vuoti per ipotesi.
}

\sproposition{}{
    Siano \(X,Y\) spazi topologici e \(A,B\) sottospazi rispettivamente di \(X,Y\).
    Allora \(A \times B\) è sottospazio di \(X \times Y\).

    Sullo spazio \(A \times B\) possiamo definire due topologie:
    la topoloia prodotto, considerando \(A\) e \(B\) come spazi topologici
    secondo le topologie di sottospazio indotte dalle topologie madri,
    oppure la topologia di sottospazio indotta dalla topologia prodotto su \(X \times Y\).
    In realtà, sono uguali.
}

\sproof{}{
    Considerando \(A \times B\) con la topologia di sottospazio,
    abbiamo che una sua base è data dia sottoinsieme della forma
    \[
        (U \times V) \cap (A \times B) = (U \cap A) \times (V \cap B)
    \]
    Il primo termine è aperto della base standard
    per \(X \times Y\), gli ultimi due sono aperti generici di \(A,B\)
    per la topologia di sottospazio.
    Le due topoloigie coincidono, avendo una stessa base.

    Possiamo dimostrarlo con le proprietà universali delle topologie di prodotto e
    delle topologie di sottospazio.
    La topologia prodotto è la topologia meno fine su \(A \times B\)
    che rende le due proiezioni
    \(\pi_A \colon A \times B \to A\) e \(\pi_B \colon A \times B \to B\)
    continue. la topologia di sottospazio su \(A \times B\) è la topologia meno fine che rende l'inclusione
    \(1\colon A \times B \hookrightarrow X \times Y\) continue.
    DISEGNO.
    Sia \(J\) una topologia su \(A \times B\).
    \(i\colon (A \times B, J) \to X \times Y\) è continua
    se e solo se \(\pi_X \circ i\) e \(\pi_Y \circ i\) sono continue.
    Ma \(\pi_X \circ i = i_A \circ \pi_A\) e \(\pi_Y \circ i = i_B \circ \pi_B\).
    La proprietà universale della topologia di sottospazio ciò è equivalente a richiedere che
    le due proiezioni \(\pi_A, \pi_B\) siano continue.
    Quindi le due topologie coincidono.
}

\begin{center}
    % https://tikzcd.yichuanshen.de/#N4Igdg9gJgpgziAXAbVABwnAlgFyxMJZABgBpiBdUkANwEMAbAVxiRAEEQBfU9TXfIRQBGclVqMWbdgAIAOnLwBbeDIBC3XiAzY8BIgCYx1es1aIQGnn12DDpYeNNSLATU02B+kQ6eTzIAAa8opYKnAy7tba-HpCJL4m-myB3OIwUADm8ESgAGYAThBKSGQgOBBIAKzUOHRYDGwAFhAQANYgSWZsWAD6nNQMdABGMAwACrF2FgVYmU04HiCFxUhG5ZWIAMy19Y0WLe2dEt0WfVZaKyWIACy1mzsnLiAKaH1Rl0XXdxvVXc+vPqpQYjMaTWzeECzeaLaJXJCiX6IdbOAKA3oaEGjCZTSHQhZLeGIREVUr-NFyN79QlfBH3JA-OoNZqtDrknppLhAA
    \begin{tikzcd}
    A \arrow[d, "i_A"', hook] & A \times B \arrow[r, "\pi_B"'] \arrow[l, "\pi_A"] \arrow[d, "i", hook] & B \arrow[d, "i_B", hook] \\
    X                         & X \times Y \arrow[r, "\pi_Y"] \arrow[l, "\pi_X"']                      & Y                       
    \end{tikzcd}
\end{center}

\stheorem{teorema di Wallace}{
    Let \(X,Y\) spazi topologici e \(A \subseteq X, B \subseteq Y\)
    sottospazi compatti con la topologia indotta e \(W\)
    un aperto di \(X \times Y\) tale che \(A \times B \subseteq W\).
    Allora, esistono degli aperti \(U\) di \(X\)
    e \(V\) di \(Y\) tali che
    \[
        A \subseteq U \land B \subseteq V \land (U \times V \subseteq W)
    \]
}

\sproof{}{
    Dimostriamo dapprima il caso particolare del teorema per
    \(A = \{a\}\).
    Quindi \(\{a\} \times B \subseteq W\).
    Ciò significa che \((a,b) \in W\) per ogni \(b\in B\).
    Siccome \(W\) è un aperto di \(X\times Y\),
    sappiamo che una base della topologia prodotto p data
    dai prodotti \(U \times V\) dove \(U\) è aperto di \(X\) e \(V\) è aperto di \(Y\),
    quindi esistono \(U_b\) di \(X\) e \(V_b\) di \(Y\)
    tali che \((a,b) in U_b \times V_b \subseteq W\).
    Quindi \(\{V_b\}\) definiscono un ricoprimento aperto del sottospazio \(B\).
    per ipotesi, \(B\) è compatto, quindi possiamo estrarre un sottoricoprimento finito
    \[
        B \subseteq \bigcup V_{b_i}
    \]
    quindi a questo punto
    poniamo \(V = \bigcup V_{b_i}\) e \(U = \bigcap U_{b_i}\).
    Allora se \(A \subseteq U\), cioè \(\{a\} \subseteq U\),
    è come dire che \(a \in U\) cioè \(a \in \bigcup b_i\).
    Inoltre \begin{align*}
        U \times V &= U \times \bigcup B_{b_i} \\
        &= \bigcup (U \times V_{b_i})
    \end{align*}
    che sono tutti termini in \(U_{b_i}\) che sono in \(W\).
    Quindi \(U, V\) soddisfano le condizioni del teorema.
    Adesso, mostriamo il caso generale.
    Sia \(A \subseteq X\) un compatto arbitrario.
    \(\forall a \in A\) esistono aperti \(U_a\) di \(X\) e \(V_a\) di \(Y\)
    tali che \(a \in U_a\), \(B \subseteq V_a\) e \(\{a\} \times B \subseteq W\)
    per il caso particolare del teorema appena dimostrato.
    Similarmente a prima, osservimo che \(\{U_a\}\) è un ricoprimento aperto
    di \(A\), e quindi siccome \(A\) è compatto, esiste un sottoricoprimento finito
    \[
        A \subseteq \bigcup U_{a_i}
    \]
    Quindi poniamo \(U = \bigcup U_{a_i}\) e \(V = \bigcap V_{a_i}\).
    Quindi \(A \subseteq U, B \subseteq V\)
    in quanto \(B \subseteq V_a\) per ogni \(a\).
    Quindi sarà anche contenuto nell'intersezione. Inoltre
    \[
        U \times V = \bigcup U_{a_i} \times V_{a_i} \subseteq W
    \]
}

\scorollary{}{
    Ogni sottospazio compatto di uno spazio di Hausdorff
    è chiuso.
}

\sproof{}{
    Sia \(X\) spazio di Hausdorff e \(K \subseteq X\) compatto.
    Vogliamo mostrare che \(K\) è chiuso, quindi che \(X \difference K\) è aperto.
    Dire che è aperto è come richiedere che sia intorno di ciascun suo punto.
    Quindi \(\forall x \in X \difference K\) tale che \(x\notin K\)
    esiste \(U\) tale che \(x\in U\) e \(U \subseteq X \difference K\), con \(U, K\) disgiunti.
    Ricordiamo che \(X\) è di Hausdorff se e solo se
    \(\Delta X = \{(x,x) \suchthat x\in X\}\) è chiusa in \(X \times X\).
    Applichiamo il teorema di Wallace prendendo \(A = \{x\}\)
    e \(B = K\), che sono compatto in quanto finito e compatto per ipotesi rispettivamente.
    \(W = (X \times X) \difference \Delta x\)
    aperto in quanto \(\Delta X\) è chiusa.
    \(A \times B \subseteq W\) cioè \(\{x\} \times K \subseteq W\) e \(x \notin K\).
    Deduciamo quindi dal teorema di Wallace che esistono aperti
    \(U\) di \(X\) e \(V\) di \(Y\) tali che
    \(\{x\} \subseteq U\), che è come dire \(x \in U\),
    e \(K \subseteq V\).
    Inoltre \(U \times K \subseteq U \times V \subseteq W = (X \times X) \difference \Delta X\).
    Quindi \(U \cap K = \emptyset\).
}

\scorollary{}{
    Siano \(X, Y\) spazi topologici.
    Se \(X\) è compatto allora la poiezione \(p \colon X \times Y \to Y\)
    è chiusa. Inoltre, Se \(X\) e \(Y\) sono compatti allora \(X \times Y\) è compatto.
    Mettere enumerate.
}

\sproof{}{
    \begin{enumerate}
        \item Sia \(C \subseteq X \times Y\) chiuso.
        Vogliamo mostrare che \(p(C)\) è chiuso in \(Y\),
        cioè che \(Y \difference p(C)\) è aperto, cioè
        è un intorno di ogni suo punto,
        quindi \(\forall y \notin p(C)\) esiste un intorno di \(Y\)
        disgiunto da \(p(C)\).
        Applichiamo il teorema di Wallacde prendendo
        \(A = X\), \(B = \{y\}\) e \(W = (X \times Y) \difference C\) aperto
        (in quanto \(C\) è chiuso).
        Allora \(A \times B = X \times \{y\} \subseteq W = (X \times Y) \difference C\).
        Ciò è equivalente a \(a \notin p(C)\)
        che è l'insieme di tutte le \(y \in Y\) tale che esiste \(x \in X\)
        con \((x,y) \in C\).
        Il teorema di Wallace fornisce
        quindi un aperto \(V\) intorno di \(y\) interamento contenuto
        in \(Y \difference p(C)\).
        Quindi \(B = \{y\} \subseteq B\) e
        \[
            X \times V \subseteq (X \times Y) \difference C
            \iff
            V \subseteq Y \difference p(C)
        \]
        \item Ricordiamo il risultato che dice
        che data una mappa chiusa che va in un compatto, e le sue fibre
        sono compatte allora il dominio è compatto.
        Nel nostro caso la mappa è la prozeione, che è chiusa per il punto precedente.
        Le fibre sono
        \[
            p^{-1}(y) \cong X
        \]
        in quanto \(X \times \{y\} \cong X\),
        che è compatto per ipotesi.
        Segue per induzione che prodotti finiti di spazi compatti sono compatti.
    \end{enumerate}
}

\scorollary{}{
    Un sottospazio di \(\realnumbers^n\)
    è compatto (con la topologia indotta dalla topologia euclidea su \(\mathbb{R}^n\)),
    se e solo se esso è chiuso e limitato (quindi incluso in un ipercubo).
}

\sproof{}{
    \ffiproof{
        Supponiamo che \(A \subseteq \realnumbers^n\) sia chiuso e limitato.
        Allora \(A \subseteq [-a, a]^n\) per qualche \(a>0\).
        Siccome \([-a, a]\) è omeomorfo a \([0,1]\) che è compatto,
        \([-a, a]^n\) è compatto in quanto prodotto di spazi compatti.
        Quindi, \(A\) è compatto in quanto chiuso di uno spazio compatto.
    }{
        Supponiamo che \(A \subseteq \realnumbers^n\) sia
        compatto. Consideriamo al funzione
        \(d \colon A \fromto \realnumbers\) data da \(x\fromto ||x||\).
        Chiaramente \(d\) è continua.
        Allora, \(d(A)\) è compatto in \(\realnumbers\), ovvero chiuso e limitato.
        In particolare, \(d(A) \subseteq [-N, N]\) per qualche \(N >0\).
        Quindi, \(d(x) \leq N\) cioè per tutte le \(x\in A\), \(||x| \neq N\).
        Quindi, \(A\) è limitato.
        Allora \(A\) è chiuso in quanto compatto in uno spazio di Hausdorff, cioè
        \(\realnumbers^n\).
    }
}

Per esempio \(S^n \subseteq \realnumbers^{n+1}\) è compatto.
Anche \(D^n = \{x \suchthat ||x||\leq 1\}\) disco è compatto.

\scorollary{}{
    Sia \(f\colon X \fromto Y\) un'applicazione continua con \(X\) compatto
    e \(Y\) di Hausdorff. Allora \(f\) è un'applicazione chiusa.
    In particolare, se \(f\) è biettiva allora \(f\) è un omeomorfismo.
}

\sproof{}{
    Per dimostrarlo passiamo da un chiuso \(C\) in \(X\) verso ad un compatto
    \(C\) in \(X\). Una volta fatto ciò possiamo applicare \(f\) per ottenere \(f(C)\) compatto in \(Y\),
    ed infinite otteniamo \(f(C)\) chiuso in \(Y\).
    Abbiamo utilizzato che ogni chiuso di un compatto è compatto.
}

\sdefinition{Compattificazione di Alexandrov}{
    Sia \(X\) uno spazio topologico, \(\infty \notin X\).
    Sull'insieme \(\hat{X} \triangleq X \union \{\infty\}\)
    definiamo la seguente famiglia di sottoinsiemi
    \[
        J = \{A \text{ aperto } \suchthat A \subseteq X\}
        \union \{\hat{X} \difference K \suchthat K \text{ chiuso e compatto in } X\}
    \]
}

\sproposition{Quest'ultima famiglia è una topologia per \(\hat{X}\) della topologia di Alexandrov}{
    \begin{enumerate}
        \item \(\emptyset \in J\) in quanto \(\emptyset\) è un aperto di \(X\);
        \item sia \(K = \emptyset\) che è chiuso e compatto in \(X\). Allora, \(\hat{X} \in J\);
        \item \emph{intersezione:} abbiamo diversi casi
        \begin{enumerate}
            \item siano \(A_1, A_2 \subseteq X\) aperti di \(X\),
            allora \(A_1 \intersection A_2\) è un aperto di \(X\);
            \item sia \(A\) un aperto di \(X\). Sia \(K\) chiuso e compatto in \(X\).
            Quindi
            \[
                A \intersection (\hat{X} \difference K) = A \intersection (X \difference K)
            \]
            quest'ultimo termine è aperto di \(X\).
            \item nel caso \((\hat{X} \difference K_1) \intersection (\hat{X} \difference K_2)\)
            usiamo le leggi di De Morgan. Quindi l'intersezione è equivalente a
            \[
                \hat{X} \difference (K_1 \union K_2)
            \]
            Il secondo termine è chiuso e compatto.
        \end{enumerate}
        \item \emph{unioni arbitrarie:} abbiamo diversi casi
        \begin{enumerate}
            \item sempre per le leggi di De Morgan
                \begin{align*}
                    \bigcup_{i\in I} (\hat{X} \difference K_i) &=
                    \hat{X} \difference \left(\bigcap_{i\in I} K_i\right)
                \end{align*}
                Se \(I = \emptyset\) la proposizione è banale. Altrimenti,
                esiste \(i_0 \in I\) tale che
                \[
                    \bigcap_{i \in I} K_i \subseteq K_{i_0}
                \]
                a sinistra l'intersezione è chiusa in quanto intersezione di chiusi, a destra è comatto.
                Quindi, è anch'esso compatto in quanto chiuso di un compatto.
            \item nel caso di un unione mista con \(K\) chiuso e compatto in \(X\)
            e \(A\) aperto
            \begin{align*}
                (\hat{X} \difference K) \union A
                &= (\hat{X} \difference K) \union (\hat{X} \difference (\hat{X} \difference A)) \\
                &= \hat{X} \difference (K \union (\hat{X} \difference A))
            \end{align*}
            quindi è anche compatto.
            Siccome \(X \difference A\) è chiuso allora è chiuso e compatto in quanto
            chiuso del compatto \(K\)
        \end{enumerate}
    \end{enumerate}
}

\sproposition{}{
    Sia \(X\) uno spazio topologico e \(\infty\notin X\).
    Allora \(\hat{X}\) è compatto.
}

\sproof{}{
    Supponiamo di avere un ricoprimento aperto
    \(\{U_i\}_{i\in I}\) di \(\hat{X}\).
    Esiste almeno una aperto che contiene \(\infty\).
    Per semplicità sia \(\infty \in U_0\).
    Quindi \(U_0\) è della forma \(U_0 = \hat{X} \difference K\)
    per qualche \(K\) chiuso e compatto in \(X\).
    Quindi, \(K = \hat{X} \difference U_0\).
    Siccome \(K\) è compatto, esiste un sottoricoprimento finito
    \(\{U_1, \cdots, U_n\}\) di \(K\).
    Quindi, aggiungendo anche \(U_0\) a questi, ottengo un ricoprimento finito di tutto lo spazio.
}

Notiamo che l'inclusione \(i \colon X \inclusion \hat{X}\)
è continua ed è una immersione aperta.
Infatti, \(i^{-1}(A) = A\) è aperto di \(X\),
e \(i^{-1}(\hat{X} \difference K) = X \difference K\)
che è aperto di \(X\).
Inoltre se \(A\) è un aperto di \(X\), \(i(A) = A\) che è aperto di \(\hat{X}\).
Siccome \(i\) è iniettiva allora è un immersione aperta.

\sproposition{}{
    Se \(X\) è uno spazio topologico NON compatto
    allora \(i(X)\) è denso in \(\hat{X}\),
    dove \(i\) è l'inclusione sopracitata.
}

\sproof{}{
    Abbiamo due tipi di aperti e quindi due casi.
    Nel primo abbiamo \(i(X) \intersection i(A) = i(A) \neq \emptyset\)
    per \(A \neq \emptyset\).
    Nel secondo caso
    \(i(X) \intersection (\hat{X} \difference K) = X \difference K \neq \emptyset\).
}

\sproposition{}{
    Sia \(X\) uno spazio topoligico.
    Allora \(\hat{X}\) è di Hausdorff se e solo se \(X\) è di Hausdorff
    e ogni suo punto possiede un intorno compatto.
}

\sproof{}{
    Siccome \(X\) è aperto in \(\hat{X}\), due punti di \(X\)
    ammettono intorni disgiunti in \(X\) se e solo se ammettono intorni disgiunti in \(\hat{X}\).
    Dato \(x\in X\), vogliamo studiare la condizione di esistenza di intorni perti disgiunti
    rispettivamente di \(x\) e del punto \(\infty\).
    Gli intorni di \(x\) sono aperti \(U\) di \(X\) tale che \(x\in U\),
    mentre gli altri devono essere della forma \(\hat{X} \difference K\)
    con \(\infty \in \hat{X} \difference K\) con \(K\) chiuso compatto in \(X\).
    Vogliamo che l'intersezione sia disgiunta
    \[
        U \intersection (\hat{X} \difference K)
        = U \intersection (X \difference K) = \emptyset
    \]
    fissato \(K\), un tale aperto \(U\) tale che
    \(x \in U\), esiste se e solo se \(x \in \text{int}(K)\),
    che è l'unione di tutti gli aperti contenuti in \(K\), che è precisamente la condizione
    \[
        U \intersection (X \difference K) = \emptyset
    \]
}

\sproposition{}{
    Sia \(Y\) uno spazio compatto e di Hausdorff e \(y\in Y\).
    Allora, lo spazio topologico \(Y \difference \{y\}\) con la topologia di sottospazio
    indotta da quella su \(Y\), soddisfa le condizioni del risultati precedente,
    cioè è uno spazio di Hausdorff e ogni punto ammette un intorno compatto.
}

\sproof{}{
    \(Y \difference \{y\}\) è di Hausdorff in quanto sottospazio di uno spazio di Hausdorff.
    Resta da dimostrare l'esistenza degli intorni compatti per ogni punto.
    Sia \(y\in Y \difference \{y\}\).
    Essendo \(Y\) di Hausdorff per ipotesi, esistono
    intorni aperti \(U\) di \(y\) e \(U'\) di \(y'\)
    tali che \(U \intersection U' = \emptyset\).
    Allora, \(Y \difference U\) è un chiuso di \(Y\) e essendo \(Y\) compatto,
    \(Y \difference U \subseteq Y\) è compatto in quanto chiuso di un compatto (in \(Y\)),
    ma vale anche \(y' \in Y \difference U \subseteq Y \difference \{y\}\).
    L'appartenenza è data dal fatto che \(y' \in U' \subseteq Y \difference U\),
    e l'inclusione perché \(\{y\} \subseteq U\).
    Quindi, se consideriamo \(Y \difference U\) come sottospazio di \(Y \difference \{y\}\)
    non cambia niente per transitività della topologia di sottospazio.
    Quindi, \(Y \difference U\) è un chiuso e compatto di \(Y \difference \{y\}\).
    Abbiamo quindi trovato un intorno compatto (pure chiuso) di \(y'\).
}

\sproposition{}{
    Sia \(f \colon X \fromto Y\) un'immersione
    aperta di spazi topologici di Hausdorff.
    Allora, l'applicazione \(g\colon Y \fromto \hat{X}\)
    data da \(y \fromto x\) se \(y = f(x)\),
    che è ben posta in quanto \(f\) è iniettiva per ipotesi,
    mentre \(y \fromto \infty\) se \(y \notin f(X)\),
    è continua.
}

\sproof{}{
    Mostriamo la continuità.
    Sia \(U\) un aperto di \(\hat{X}\). Abbiamo due possibilità:
    \begin{enumerate}
        \item \(U \subseteq X\), allora \(g^{-1}(U) = f(U)\) che è aperto essendo
        \(U\) aperto di \(X\) e \(f\) aperta per ipotesi;
        \item \(U = \hat{X} \difference K\), allora \(g^{-1}(\hat{X} \difference K) = Y \difference f(K)\).
        Questo è un aperto in quanto \(f(K)\) è un chiuso di \(Y\)
        in quanto \(K\) è compatto, quindi \(f(K)\) è compatto in quanto continua,
        e allora essendo \(Y\) di Hausdorff è chiuso.
    \end{enumerate}
}

\scorollary{}{
    Sia \(Y\) spazio compatto di Hausdorff
    e \(y \in Y\).
    Allora, \(Y\) è omeomorfo alla compattificazione
    di Alexandrov di \(Y\difference \{y\}\).
}

\sproof{}{
    Applichiamo la proposizione precedente
    con \(X = Y \difference \{y\}\) e \(f = i \colon Y \difference \{y\} \inclusion Y\).
    Questa è un immersione aperta, infatti \(Y\) è di Hausdorff e quindi \(T_1\),
    cioè \(Y \difference \{y\}\) è aperto.
    Abbiamo quindi una mappa continua \(g \colon Y \fromto \hat{Y \difference \{y\}}\).
    Inoltre, è un omeomorfismo in quanto iniettiva ed è chiusa in quanto definita tra spazi entrambi
    compatti e di Hausdorff (corollario precedente).
}

Per esempio \(S^{1} \difference \{p\} \cong \realnumbers\).
Più in generale \(S^n\) è omeomorfa alla compattificazione di Alexandrov di \(\realnumbers^n\).

\section{Topologia quoziente}

\sdefinition{}{
    Sia \(f \colon X \fromto Y\) un applicazione (tipicamente suriettiva) e \(X\)
    uno spazio topologico. La topologia quoziente su \(Y\) è la più fine
    cherende \(f\) continua.
}

Questo è un duale di quando abbiamo considerato la topologia meno fine
per le mappe di inclusione.
Caratterizzare questa dualità nel linguaggio categorico potrebbe essere una tesi.

La collezione di insiemi
\[
    \tau = 
    \{A \subseteq Y \suchthat f^{-1}(A) \text{ è aperto di } X\}
\]
è effettivamente una topologia su \(Y\), ed è quindi la topologia quoziente su \(Y\).

Infatti:
\begin{enumerate}
    \item \(\emptyset \in \tau\) in quanto \(f^{-1}(\emptyset) = \emptyset\) che è aperto di \(X\).
    \item \(Y \in \tau\) in quanto \(f^{-1}(Y) = X\) aperto di \(X\);
    \item se \(A_1, A_2 \in \tau\) allora \[f^{-1}(A_1 \intersection A_2) = f^{-1}(A_1) \intersection f^{-1}(A_2)\]
    \item sia invece \(\{A_i\}_{i\in I}\) allora
    \[
        f^{-1}\left(\bigcup_{i\in I} A_i\right)
        = \bigcup_{i\in I} f^{-1}(A_i)
    \]
    che è unione di aperti di \(X\)
\end{enumerate}

\sdefinition{}{
    Una applicazione continua e suriettiva \(f\colon X \fromto Y\)
    è una \emph{identificazione} se gli aperti di \(Y\) sono tutti e soli i sottoinsiemi
    \(A\) di \(Y\) tale che \(f^{-1}(A)\) è aperto in \(X\).
}

La topologia quoziente è l'unica topologia su \(Y\) che rende \(f\) una identificazione.
Inoltre, \(f\) è un identificazione se vale l'analogo per i chiusi della condizione precedente,
cioè se e solo se i chiusi di \(Y\) sono tutti e soli i sottoinsiemi \(C\) di \(Y\)
tale che \(f^{-1}(C)\) è chiuso in \(X\).

\sdefinition{}{
    Sia \(f \colon X \fromto Y\) una mappa.
    Un sottoinsieme \(A\) di \(X\) si dice
    \(f\)-saturo se \(\forall x,x' \in X\) tali che \(f(x) = f(x')\),
    allora \(x\in A\) se e solo se \(x'\in A\).
}

\(A \subseteq X\) è \(f\)-saturo se e solo se \(A = f^{-1}(B)\)
per qualche sottoinsieme \(B \subseteq Y\).
Equivalentemente, \(A = f^{-1}(f(A))\).

\sproposition{}{
    Se \(f\colon X \fromto Y\) è continua e suriettiva, allora \(f\)
    è un identificazione
    se e solo se gli aperti di \(Y\) sono tutti e soli
    i sottoinsiemi di \(Y\) della forma \(f(A)\) dove \(A\) è un aperto \(f\)-saturo di \(X\).
}

\sproof{}{
    Essendo \(f\) suriettiva, per tutte le \(B \subseteq Y\),
    \(B = f(f^{-1}(B))\).
    D'altro canto, \(A\) è \(f\)-saturo se e solo se \(A = f^{-1}(f(A))\).
}

\sproposition{}{
    Sia \(f \colon X \fromto Y\) un identificazione.
    Se \(A\) è un aperto \(f\)-saturo, allora \(f(A)\)
    è un aperto di \(Y\).
}

\sproof{}{
    \(f(A)\) è aperto di \(Y\) se e solo se
    \(f^{-1}(f(A))\) è aperto in \(X\),
    ma \(f^{-1}(f(A)) = A\) siccome \(A\) è \(f\)-saturo.
}

Questo non implica che \(f\) sia aperta.

\sexample{}{
    Sia \(f \colon [0,2\pi] \fromto S^1\)
    data \(f(t) = (\cos t, \sin t)\).
    \(f\) è continua e suriettiva, quindi è chiusa, e quindi è un'identificazione
    chiusa. Osserviamo che \(f\) non è un'identificazione aperta:
    \([0,1) = (-1, 1) \intersection [0, 2\pi]\) è un aperto di \([0,2\pi]\).
    Ma quindi \(f([0,1))\) è un arco di circonferenza aperto in un estremo e chiuso nell'altro,
    quindi non è un aperto di \(S^1\).
}

Vediamo la proprietà universale delle identificazioni.

\sproposition{}{
    Sia \(f \colon X \fromto Y\) una identificazione e \(g\colon X \fromto Z\)
    un'applicazione continua. Allora esiste una e una sola applicazione continua
    \(h\colon Y \fromto Z\) tale che \(h \circ f = g\) se e solo se \(g\)
    è costante sulle fibre di \(f\), cioè per tutte le \(x,x' \in f^{-1}(y), g(x) = g(x')\).
}

\begin{center}
    % https://tikzcd.yichuanshen.de/#N4Igdg9gJgpgziAXAbVABwnAlgFyxMJZABgBpiBdUkANwEMAbAVxiRAA0QBfU9TXfIRQBGclVqMWbAFrdeIDNjwEiZYePrNWiEAE1u4mFADm8IqABmAJwgBbJGRA4ISAEzVNUnRZDUGdACMYBgAFfmUhECssYwALHDlLG3tER2ckUQktNmNEkGs7N2p0xEzPbRBY3xB-INDwwTZouIS-LDAKqDo4WKMDLiA
    \begin{tikzcd}
    X \arrow[d, "f"'] \arrow[r, "g"] & Z \\
    Y \arrow[ru, "h"', dashed]       &  
    \end{tikzcd}
\end{center}

\sproof{}{
    Per definiziona un'identificazione è surriettiva, quindi l'unicità di \(h\)
    segue direttamente. Cioè dato \(y \in Y\),
    se \(y = f(x)\) allora \(h(y)\) deve necessariamente coincidere con \(g(x)\).
    Iniziamo mostrando che la condizione che \(g\) sia costante sulle fibre è necessaria
    per l'esistenza di \(h\) tale che il diagrammi commuti.
    Dire che \(x,x' \in f^{-1}(y)\) è equivalente a dire che \(f(x) = y = f(x')\),
    e quindi \(h(f(x)) = h(f(x'))\), cioè \(g(x)=g(x')\).
    Mostriamo ora che è anche sufficiente.
    Come osservato sopra, la suriettività di \(f\) obbliga a dfinire \(h\)
    nel modo seguente: sia \(y\in Y\), \(y=f(x)\) allora \(h(y) \triangleq g(x)\).
    Tale definizione è ben posta in quanto \(g\) è costante sulel fibre di \(f\).
    \(y = f(x) = f(x') \implies g(x') = g(x)\).
    Ovviamente abbiamo \(h \circ f = g\) per defiinzione di \(h\).
    Vogliamo anche che \(h\) sia continua,
    Sia \(B\) aperto di \(Z\), quindi \(h^{-1}(B)\) è aperto in \(Y\),
    ma ciò è equivalente, isccome \(f\) è una identificazione, a dire che
    \(f^{-1}(h^{-1}(B))\) è aperto in \(X\).
    Ma ciò è uguale a \((h\circ f)^{-1}(B)\) che è uguale a \(g^{-1}(B)\)
    in quanto \(g\) è continua per ipotesi.
}

\sexample{}{
    Sia \[D^n = \{x \in \realnumbers^n \suchthat 1 \geq ||x||\}\]
    Sia \[
        S^n = \{(x,y) \in \realnumbers^{n} \times \realnumbers \suchthat 1 = {||x||}^2 + y^2\}
    \]
    Consideriamo \(f \colon D^n \fromto S^n\)
    data da
    \[
        x \fromto (2x\sqrt{1-||x||^2}, 2||x||^2 - 1)
    \]
    Chiaramente \(x / ||x|| = 1 \to (0,1)\)
    e quindi \(f\) non è iniettiva. Tuttavia, lo diventa
    se la restringiamo all'interno di \(D^n\).
    Infatti definisce un omeomorfismo sulla sua immagine in quanto
    \[
        y < 1 \iff 2||x||^2 - 1 < 1 \iff ||x|| < 1
    \]
    In tal caso \(x\) può essere chiaramente ricostruito mediante applicazione
    continua da \(z\) a \(y\).
    Se \(y = 2||x||^2 - 1\)
    allora 
    \[
        ||x|| = \sqrt{\frac{y + 1}{2}}
    \]
    Sostituendo tale espressione in quella per \(z\)
    si arriva a ricostruire \(x\) in funzione di \(y\) e \(z\).
    Notiamo che
    \(D^n\) è compatto in quanto chiuso e limitato, e \(S^n\) è di Hausdorff
    (per lo spazio metrico), quindi \(f\) è una identificazione continua da uno
    spazio compatto ad uno spazio di Hausdorff.
}

\sdefinition{}{
    Sia \(A \subseteq X\) di uno spazio topologico \(X\).
    Sia \(\sim_A\) la più piccola relazione di equivalenza
    su \(X\) che ha \(A\) come classe di equivalenza, cioè i punti tali che \(x,y \in A\).
    Denotando \(X / A \triangleq X /_{\sim_A}\).
    Questo spazio, dotato della topologia quoziente,
    indotta dalla proiezione canonica \(\pi_A \colon X \fromto X/A\)
    è detto contrazione di \(A\) ad un punto.
}

\sdefinition{Nastro di Möbius}{
    Let \(I = [0,1]\) be the closed unit interval.
    Define the equivalence relation \(\sim\) on \(I\) generated by
    by \((0,t) \sim (1,1-t)\).
    The \emph{Möbius strip} is defined as \(M = (I \times I)/\sim\).
}

\sdefinition{Klein's Bottle}{
    Let \(I = [0,1]\) be the closed unit interval.
    Define the equivalence relation \(\sim\) on \(I\) generated by
    by \((x,0) \sim (x,1)\) and \((0,y) \sim (1,1-y)\).
    The \emph{Möbius strip} is defined as \(K = (I \times I)/\sim\).
}

\sproposition{}{
    Sia \(f\colon X \fromto Z\) un'identificazione. Allora \(Z\)
    è di Hausdorff se e solo se per ogni coppia di punti \(x,y \in X\) tali
    che \(f(x) \neq f(y)\), esistono due aperti \(f\)-saturi disgiunti \(A\)
    e \(B\) di \(X\) tali che \(x\in A\) e \(y\in B\).
}

\sproof{}{
    \(Z\) è di Hausdorff se esolo se per ogni \(x,y\) di \(X\)
    tali che \(f(x) \neq f(y)\), esistono intorni aperti disgiunti \(U, V\) di \(Z\)
    tali che \(f(x) \in U\) e \(f(y) \in V\). \\
    \iffproof{
        Abbiamo che \(x \in f^{-1}(U)\) e \(y \in f^{-1}(V)\) e
        \[
            f^{-1}(U) \intersection f^{-1}(V)
            = f^{-1}(U \intersection V) = f^{-1}(\emptyset) = \emptyset
        \]
    }{
        Sia \(x\in A\) e \(y\in B\) aperti \(f\)-saturi di \(X\) disgiunti.
        Mostriamo che \(f(A)\) e \(f(B)\) sono degli aperti di \(Z\)
        tali che \(f(A) \intersection f(B) = \emptyset\).
        \(f(A)\) e \(f(B)\) sono aperti in quanto \(f\) è identificazione e \(A,B\)
        sono \(f\)-saturi (risultato precedente).
        Siano \(f(x) \in f(A)\) e \(f(y) \in f(B)\).
        Sia \(z \in f(A) \intersection f(B)\).
        Chiaramente \(z = f(a) = f(b)\) per qualche \(a \in A\) oppure \(b\in B\).
        Allora \(f(a) = z = f(b)\).
        Siccome \(A\) è \(f\)-saturo allora \(b\in A\).
        Siccome \(B\) è \(f\)-saturo allora \(a\in B\).
        Ma quindi \(a,b \in A \intersection B = \emptyset\), che è assurdo.
    }
}

\stheorem{}{
    Sia \(X\) uno spazio topologico compatto e di Hausdorff
    e sia \(f\colon X \fromto Y\) un'identificazione. The following are equivalent:
    \begin{enumerate}
        \item \(Y\) è di Hausdorff;
        \item \(f\) è una identificazione chiusa;
        \item il sottoinsieme di \(X \times X\)
        dato da
        \[
            K = \{(x_1, x_2) \in X \times X \suchthat f(x_1) = f(x_2)\}
        \]
        è chiuso in \(X \times X\).
    \end{enumerate}
}

\sproof{}{
    \begin{enumerate}
        \item \((1) \implies (3):\) \(Y\) è di Hausdorff se esolo se la diagonale
        \(\Delta Y \subseteq Y \times Y\) è chiusa. Possiamo esprimere \(K\)
        come
        \[
            K = (f \times f)^{-1}(\Delta Y)
        \]
        e \(f \times f\) è continua per la proprietà universale del prodotto.
        Quindi \(K\) è chiuso in quanto controimmagine di un chiuso.
        \item \((3) \implies (2):\) sia \(A\) chiuso di \(X\).
        Per ipotesi, le due proiezioni canoniche \(p_1, p_2 \colon X \times X \fromto X\)
        soo chiuse, essendo \(X\) compatto e pure di Hausdorff.
        Possiamo esprimere
        \begin{align*}
            f^{-1}(f(A)) &= p_1\left(
                K \intersection p_2^{-1}(A)
            \right)
        \end{align*}
        % spiegazione 12/11/25
        \item \((2) \implies (1):\)
        siano \(a,b \in Y\) tale che \(a \neq b\).
        Consideriamo le fibre \(A = f^{-1}(a)\) e \(B = f^{-1}(b)\).
        Essendo \(f\) suriettiva, \(a = f(x)\) per qualche \(x\)
        e \(b = f(x')\) per qualche \(x'\).
        Quindi, \(\{a\} = f(\{x\})\) e \(\{b\} = f(\{x'\})\)
        dove i singoletti sono chiusi siccome \(X\) è di Hausdorff e quindi è anche \(T_1\).
        Allora, \(\{a\}\) e \(\{b\}\) sono chiusi in \(Y\)
        essendo \(f\) chiusa.
        Quindi, \(A\) e \(B\) sono dei chiusi di \(X\), in wuanto controimmagini di chiusi,
        e quindi compatti (essendo \(X\) compatto).
        Allora la condizione \(A \intersection B = \emptyset\) può essere espressa come
        \[
            A \times B \subseteq (X \times X \difference \Delta x) = W
        \]
        in quanto quest'ultimo insieme è di Hausdorff per ipotesi.
        Allora possiamo applicare il teorema di Wallace che ci garantisce 
        l'esistenza di aperti \(U, V\) che circondano i due compatti
        \(A \subseteq U, B \subseteq V\) e \(U \times V \subseteq W\) che è equivalente a dire
        \(U \intersection V = \emptyset\).
        Tornando indietro abbiamo
        \[
            Y = f(X) = f((X \difference U) \union (X \difference U)) = f(X \difference U) \union f(X \difference U)
        \]
        quindi abbiamo unione di due chiusi essendo \(f\) identificazione chiusa.
        Ora, per concludere, poniamo \(U' = Y \difference f(X \difference U)\)
        e \(V' = Y \difference f(X \difference V)\).
        Abbiamo quindi trovato due intorni dei miei due punti \(a,b\)
        in \(Y\).
    \end{enumerate}
}

\pagebreak

\section{Esercizi 21 ottobre}

\sexercise{}{
    Siano \(f,g\) omeomorfismi, allora la loro composizione (se esiste)
    è un omomorfismo e l'inverso è un omomorfismo. Quindi \(\text{Omeo}(X)\) è un gruppo.
    Siano \(X,Y,Z\) spazi tali che \(f\colon Y \to Z\) e \(g \colon X \to Y\).
    Per definizione l'inverso è un omeomorfismo.
    Siccome \((f \circ g)^{-1} = g^{-1} \circ f^{-1}\) se gli inversi sono omeomorfismo allora è un omeomorfismo.
    Quindi è un gruppo.
}

% Calcola il gruppo Aut(Q) di omeomorfismi dove Q è il grafo quadrato (sottoinsieme di R² fatto dai lati e vertici)
% Soluzione: Aut(Q) = D_4 gruppo diedrale = isometria del quadrato euclideo.
% DOmanda corretta:
% Dimostrare che D4 è uguale a Omeo(Q) dove Q è l'insieme parzialmente ordinato dotato
% della topologia discreta di insieme parzialmente ordinato
% con base i coni come di un esercizio precedente.

\section{Esercizi 11 novembre}

\sexercise{Esercizio 5.3 Manetti}{
    Siano \(f \colon X \fromto Y\), \(g \colon Z \fromto W\)
    due identificazioni aperte. Mostrare che
    \(f\times g \colon X \times Y \fromto Z \fromto W\) è una identificazione aperta.
}

\ssolution{}{
    Dobbiamo mostrare che \(A \subseteq Z \times W\) è aperto se
    \((f\times g)^{-1}(A)\) è aperto. L'altra direzione è ovvia.
    Notiamo che \(f,g\) sono suriettive, quindi anche il prodotto lo è.
    In particolare \(A = (f\times g)({f\times g})^{-1}(A)\).
    Supponiamo che \((f\times g)^{-1}(A)\) sia aperto, cioè
    che
    \[
        (f\times g)^{-1}(A) = \bigcup (U_i, V_i)
    \]
    dove \(U_i \subseteq X\) e \(V_i \subseteq X\) aperti.
    Abbiamo quindi \begin{align*}
        A &= (f\times g)(\bigcup (U_i, V_i)) \\
        &= \bigcup (f\times g)(U_i, V_i) \\
        &= \bigcup f(U_i) \times g(V_i)
    \end{align*}
    che è aperto in quanto \(U_i, V_i\) sono aperti.
}

\sexercise{5.4 Manetti}{
    Sia \(f\colon X \fromto A\) una identificazione.
    Mostrare che se le componenti connesse di \(X\)
    sono aperte allora anche le componenti
    convesse di \(Y\) sono aperte. 
}

\ssolution{}{
    Sia \(Y_i\) una componente connessa di \(Y\).
    Vogliamo mostrare che \(f^{1}(Y_i)\) è aperta.
    Sia \(x_0 \in f^{-1}(Y_1)\) e sia \(X_j\) la componente conessa di \(x_0\).
    Allora \(f(X_j)\) è connesso e \(f(X_j) \intersection Y_i \ni f(x_0)\)
    cioè \(f(X_j) \intersection Y_u \neq \emptyset\).
    Allora \(f(X_j) \union Y_i\) è connesso e \(f(X_J) \subseteq Y_i\)
    per massimalità di \(Y_i\) connesso.
    Quindi \(X_j \subseteq f^{-1}(Y_i)\). Abbiamo quindi provato che
    \(\forall x_0 \in f^{-1}(Y_1)\) le componenti connesse \(X_j\)
    di \(x_0\) è contenuta in \(f^{-1}(Y_i)\).
    Quindi \(f^{-1}(Y_i)\) è unione disgiunta di comonenti connesse di \(X\),
    che sono aperte per ipotesi e quindi \(Y_i\) è aperto.
}

\sexercise{5.5 Manetti}{
    Sia \(f\colon X \fromto X\) una identificazione tale che le fibre \(f^{-1}(y)\) siano tutte
    connesse. Provare che ogni sottoinsieme aperto, chiuso e non vuoto di X è
    saturo. Dedurre che se \(Y\) è connesso, allora anche \(X\) è connesso
}

\ssolution{}{
    Sia \(Y \subseteq X\) aperto e chiuso e sia \(y \in Y\).
    Vogliamo provare che \(f^{-1}(f(y)) \subseteq Y\).
    Siccome \(f^{-1}(f(y))\) è connesso, \(f^{-1}(f(y)) \intersection Y\) aperto e chiuso in
    \(f^{-1}(f(y))\) e chiaramente contiene \(y\).
    \(f^{-1}(f(y)) \intersection Y = f^{-1}(f(y))\)
    per connessione di quest'ultimo vale \(f^{-1}(f(y)) \subseteq Y\).
    Supponiamo \(Y\) connesso. Per assurdo sia \(X = A \sqcup B\) con \(A,B\) aperti e chiusi.
    Perché sono saturi, \(f(A)\) e \(f(B)\) sono aperti
    \[
        f(A) \union f(B) = Y
    \]
    poiché \(f\) è suriettiva. Se \(x \in f(A) \intersection f(B)\)
    allora \(f^{-1}(X) \subseteq A\) e \(f^{-1}(X) \subseteq B\), ma dovrebbero essere aggiunti che è assurdo.
}

\sexercise{5.6 Manetti}{
    Siano \(f\colon X \fromto Y\) una identificazione aperta ed \(A \subseteq X\) un sottoinsieme saturo;
    dimostrare che \(A^\circ\) e \(\overline{A}\) sono saturi. Mostrare con un esempio che
    ciò è generalmente falso se f è una identificazione chiusa.
}

% 5.8


\end{document}
\documentclass[a4paper]{article}

\usepackage{amsmath}
\usepackage{amssymb}
\usepackage{stellar}
\usepackage{parskip}
\usepackage{fullpage}
\usepackage{wrapfig}
\usepackage{tikz}

\usetikzlibrary{arrows}
\usetikzlibrary{decorations.pathreplacing}
\usetikzlibrary{cd}

\title{Topologia I}
\author{Paolo Bettelini}
\date{}

\begin{document}

\maketitle
\tableofcontents

\section{Topologia}

Invarianti: \(p_0\) corrisponde al numero di componenti connesse di uno spazio.
Formalmente \(\pi_0(X)\) è l'insieme delle componenti connesse di \(X\) per archi.
Invece, \(p_1\) è il gruppo fondamentale \(\pi_1(X)\), che descrive la struttura dei cammini
chiusi fino a omotopia.

\saxiom{Estensionalità}{
    \[
        A = B \iff \forall x(x\in A \iff x\in B)
    \]
}

% la logica di boole corrisponde alla logica classica come la logica costruttivista corrisponde alle algebre di H

% mettere le leggi di the morgan generalizzate + le due leggi extra (famiglie arbitrarie)
% 

\end{document}
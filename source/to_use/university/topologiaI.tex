\documentclass[a4paper]{article}

\usepackage{amsmath}
\usepackage{amssymb}
\usepackage{stellar}
\usepackage{parskip}
\usepackage{fullpage}
\usepackage{wrapfig}
\usepackage{tikz}

\usetikzlibrary{arrows}
\usetikzlibrary{decorations.pathreplacing}
\usetikzlibrary{cd}

\title{Topologia I}
\author{Paolo Bettelini}
\date{}

\begin{document}

\maketitle
\tableofcontents

\section{Topologia}

Invarianti: \(p_0\) corrisponde al numero di componenti connesse di uno spazio.
Formalmente \(\pi_0(X)\) è l'insieme delle componenti connesse di \(X\) per archi.
Invece, \(p_1\) è il gruppo fondamentale \(\pi_1(X)\), che descrive la struttura dei cammini
chiusi fino a omotopia.

\saxiom{Estensionalità}{
    \[
        A = B \iff \forall x(x\in A \iff x\in B)
    \]
}

% la logica di boole corrisponde alla logica classica come la logica costruttivista corrisponde alle algebre di H

% mettere le leggi di the morgan generalizzate + le due leggi extra (famiglie arbitrarie)

\sproposition{Relazione di aggiunzione}{
    Valgono
    \[
        S \subseteq f^{-1}(T) \iff
        f(S) \subseteq T
    \]
}
Da cui derivano \(f(f^{-1})(T) \subseteq T\).
Ma in generale l'uguaglianza non vale in quanto \(f\) potrebbe non essere suriettiva.
E pure \(S \subseteq f^{-1}(f(S))\).
Ma in generale l'uguaglianza non vale in quanto \(f\) potrebbe non essere iniettiva.

L'operazione di controimmagine preserva tutte le operazioni insiemistiche.
\begin{align*}
    f^{-1} \left(\bigcup_{i\in I} A_i\right) &= \bigcup{i\in I} f^{-1}(A_i) \\
    f^{-1} \left(\bigcap_{i\in I} A_i\right) &= \bigcap{i\in I} f^{-1}(A_i) \\
    X \backslash f^{-1}(T) &= f^{-1}(Y \backslash T)
\end{align*}

L'operazione di immagine preserva in generale solo le unioni.

\begin{align*}
    f\left(\bigcup_{i\in I} A_i\right) &= \bigcup_{i\in I} f(A_i)
\end{align*}
le altre due non valgono necessariamente. Abbiamo solo
\begin{align*}
    f(A\cap B) &\subseteq f(A) \cap f(B) \\
\end{align*}
se \(f\) non è iniettiva la direzione opposta non vale necessariamente.
Infatti potrebbero esistere \(x,x'\) tale che \(x \in A \backslash B\)
e \(x' \in B \backslash A\) tali che \(f(x) = f(x')\).
La medesima logica vale per il complementare.

\sproposition{Proprietà universale del quoziente}{
    Sia \(f\colon X \to Y\) e \(\sim\) relazione di equivalenza su \(X\). Sono equivalenti:
    \begin{enumerate}
        \item \(f\) è costante sulle classi di equivalenza
            \[ x \sim x' \iff f(x) = f(x') \]
        \item \(f\) fattorizza (in modo necessaria unico, essendo \(\pi\) suriettivo) attraverso
            \(\pi\), cioè \(\exists_{=1} \,g \colon X/_\sim \to Y\) tale che \(g \circ \pi = f\).
    \end{enumerate}
}
\begin{center}
    % https://tikzcd.yichuanshen.de/#N4Igdg9gJgpgziAXAbVABwnAlgFyxMJZABgBpiBdUkANwEMAbAVxiRAA0QBfU9TXfIRQBGclVqMWbAJrdeIDNjwEiZYePrNWiDgHoA+gB1D2ALbdxMKAHN4RUADMAThHOIyIHBCQAmapqkdYzQsEGoGOgAjGAYABX5lIRAnLGsACxw5Rxc3Dy8kUQktNgcskGdXX2p8xELosCgkAFoAZg8A7RBrMJAI6LiEwTYU9MyuCi4gA
    \begin{tikzcd}
    X \arrow[d, "\pi"'] \arrow[r, "f"]   & Y \\
    X/_\sim \arrow[ru, "g"', bend right] &  
    \end{tikzcd}
\end{center}

\sproof{}{
    \begin{enumerate}
        \item \((2) \implies (1):\) \(f = g \circ \pi\).
        Abbiamo \[
            x \sim x' \implies
            \pi(x) = \pi(x') \implies g(\pi(x)) = g(\pi(x'))
        \]
        che sono uguali a \(f(x)\) e \(f(x')\).
        \item \((2) \implies (1):\)
        Definiamo \(g\colon X/_\sim \colon Y\) come
        \[
            g([x]) \triangleq f(x)
        \]
        bisogna verificare che sia ben posta.
        Vogliamo quindi che se \([x] = [x']\) allora \(f(x) = f(x')\).
        Ma ciò è garantito dalla ipotesi.
    \end{enumerate}
}

In \(\mathbb{R}^n\).
\[
    d_\infty(x,y) \leq d_2(x,y) \leq d_1(x,y) \leq n \cdot d_\infty(x,y)
\]

\section{08 ottobre 2025 DA METTERE}

\sexercise{}{
    Le topologie con la proprietà che le intersezioni arbitrari di
    aperti sono aperti, possono essere
    caratterizzate esplicitamente.
    Essi sono esattamente, a meno di omeomorfismo,
    gli spazi topologici della seguente forma: dato un insieme preordinato \((P, \leq)\),
    la topologia di Alexandrov \(\mathcal{A}_{\mathcal{P}}\)
    su \(\mathcal{P}\) è la topologia i cui aperti sono i sottoinsiemi
    \(U \subseteq \mathcal{P}\) tale che
    \(\forall p \leq q, p \in U \implies q \in U\).
}

\subsection{Generated topology}

\sproposition{}{
    Data una collezione di topologie \(\{\tau_i\}_{i \in I}\)
    su un isieme \(X\). La famiglia
    \[
        \tau = \bigcup_{i\in I} \tau_i
    \]
    è ancora una topologia su \(X\).
}

\scorollary{}{
    Sia \(X\) un insieme e \(S \subseteq \mathcal{P}(X)\) famiglia fi sottoinsiemi.
    Esiste la topologia meno fine su \(X\) che contiene i sottoinsiemi in \(S\)
    come aperti.
    Tale topologai viene detta la topologia generata da \(S\).
}

\sdefinition{Topologia dell'unione disgiunta}{
    Sia \(\{X_i \,|\, i \in I\}\) una famiglia di spazi topologici.
    Allora lo spazio topologico è definita come
    \[
        \bigsqcup_{i\in I} X_i
    \]

    Possiamo definire astrattamente la topologia dell'unione disgiunta su
    \(\sqcup X_i\) come a topologia più fine che rende tutte le mappe \(\tau_i \colon X_i \to \bigsqcup X_i\)
    continue.

    Alternativamente, possiamo definire la topologia come la topologia generata dalla famiglia di sottoinsiemi
    dell'insieme \(\sqcup X_i\)
    che sono aperti in qualcuno degli \(X_i\).
}

Vediamo una caratteristica esplicita di questa topologia

\sproposition{}{
    Un insieme
    \[
        A \subseteq \bigsqcup_{i\in I} X_i
    \]
    è aperto per la topologia dell'unione disgiunta se e solo se
    \(A \cap X_i\) è aperto in \(X_i\) per ogni \(i \in I\).
}

\sproof{}{
    Definiamo \(\tau\) come la collezione dei sottoinsiemi dati nella proposizione
    tale che \(A \cap X_i\) è aperto in \(X_i\).
    Usando il fatto che su \(X_i\) abbiamo delle topologia possiamo dimostrare
    che tale collezione soddisfa gli assiomi di topologia:
    \begin{enumerate}
        \item Siano \(A, B \in \tau\).
        Allora \(A \cap X_i\) e \(B \cap X_i\) sono entrambi aperti in \(X_i\).
        Di conseguenza la loro unione è ancora aperta in \(X_i\). Possiamo scrivere
        \[
            (A \cap X_i) \cup (B \cap X_i) = (A \cap B) \cap X_i
        \]
        che è appunto aperto.
    \end{enumerate}
    Notiamo che \(\tau\) contiene tutti i sottoinsiemi che sono aperti
    in qualche \(X_i\).
    Infatti, \(A \subseteq X_i\) è aperto di \(X_i\),
    \(A \cap X_i = A\) aperto di \(X_i\)
    e \(A \cap X_j = \emptyset\) aperto di \(X_j\) per \(j \neq i\).
    Quindi \(\tau\) contiene la topologia dell'unione disgiunta
    (per definizione di quest'ultima come topologia generate).
    Viceversa, dato \(A \in \tau\) vogliamo mostrare che \(A\)
    è aperto per la topologia dell'unione disgiunta.
    \begin{align*}
        A &\subseteq \bigsqcup_{i\in I} X_i \\
        A &= A \cap \left(\bigsqcup_{i\in I} X_i\right)
        = \bigsqcup_{i\in I} (A \cap X_i)
    \end{align*}
    che è una disgiunzione di insiemi aperti 
}

Queste due definizioni sono una un po' il duale dell'altra,
da due punti di vista differenti.
Da una parte considerando le applicazioni (ci arriviamo come la topologia più fine),
mentre l'altro è come se costruissimo la topologia dal basso.

Possiamo verificare che questa è effettivamente la topologia più fine che rende queste mappe continue.  
In generale, data una famiglia di applicazioni \(f_i \colon (X_i, \tau_I) \to y\)
si può considerare la topologia più fine che rende le mappe \(f_i\)
continue.
In particolare nel caso di un'unica funzione
la topologia più fine che rende \(f\)
continua è la topologia detta topologia quoziente indotta da \(f\).

\sexample{Topologia di Zaniski}{
    Let \(\mathbb{K}\) be a field
    and consider \(\mathbb{K}[x_1, \cdots, x_n]\).
    Consider the affine space
    given by the cartesian exponent \(\mathbb{K}^n\).
    For each \(f \in \mathbb{K}[x_1, \cdots, x_n]\)
    consider
    \[
        D(f) = \{(a_1, \cdots, a_n) \in \mathbb{K}^n \,|\, f(a_1, \cdots, a_n) \neq 0\}
    \]
    the set \(\{D(f)\}\) is a basis for the topology of \(\mathbb{K}^n\)
    called Zaniski topology.
}

\sproof{Che è una base}{
    Chiaramente \(D(0) = \emptyset\) e \(D(1) = \mathbb{K}^n\).
    The latter already proves that the whole space can be expressed as a union.
    For the intersection, consider
    \begin{align*}
        D(f) \cap D(g) = \{
            \{(a_1, \cdots, a_n) \in \mathbb{K}^n \,|\, f(a_1, \cdots, a_n) \neq 0 \land g(a_1, \cdots, a_n) \neq 0\}
        \}
    \end{align*}
    Siccome un campo è un dominio di integrità la condizione è equivalente a
    \((f \circ g)(a_1, \cdots, a_n) \neq 0\) ma ciò è uguale a \(D(f \circ g)\).
    Quindi \(\{D(f)\}\) forma una base per una topologia sul dato spazio.
}

\sexercise{}{
    Caratterizzare i chiusi di questa topologia.
    Quindi generiamo tutti gli aperti e prendiamo i complementari,
    o usiamo le leggi di de morgan. I chiusi sono generati da un ideale.
}

TODO anche la dimostrazione che la chiusura
è pari ai punti x tali che per ogni U in I(x), U intersection B neq emptyset.

\sdefinition{}{
    Uno spazio topologico si dice \(T_1\) se ogni punto \(\{x\}\)
    (come sottoinsieme dello spazio) è chiuso.
}

Per esempio la retta euclidea.

\sproposition{}{
    Sia \(X\) uno spazio topologico.
    Allora \(X\) è \(T_1\) se e solo se
    \(\forall x\in X\),
    \[
        \bigcap_{U \in I(x)} U = \{x\}
    \]
}

\sproof{}{
    \iffproof{
        Abbiamo ovviamente l'inclusione \(\supseteq\).
        Viceversa, dimostriamo che
        \[
            \bigcap_{U \in I(x)} U \subseteq \{x\}
        \]
        che è euivalente a dire
        \[
            X \backslash \left(
                \bigcup_{U \in I(x)} U
            \right) \geq X \backslash \{x\}
        \]
        Prendiamo quindi un punto \(y \in X \backslash \{x\}\)
        che è come dire \(y\neq x\).
        Siccome lo spazio è \(T_1\),
        abbiamo che \(\{y\}\) è chiuso e quindi
        il suo complementar e \(X \backslash \{x\}\) è un aperto che contiene \(x\)
        in quanto \(x\neq y\).
        Quindi \(X \backslash \{y\} \in I(x)\).
        Ponendo \(U = X \backslash \{y\}\)
        otteniamo quindi che \[y \in X \backslash U = X \backslash (X \backslash \{y\}) = \{y\}\]
    }{
        Applichiamo la caratterizzazione della chiusura del singoletto,
        cioè \(y \in \overline{\{x\}}\) è come dire che per ogni \(U \in I(x)\),
        \(U \cap \{x\} \neq \emptyset\).
        Ma tutto ciò è equivalente a dire che
        \[
            x \in \bigcap_{U \in I(x)} U = \{y\}
        \]
        che è equivalente a dire che \(x=y\).
        Quindi, \(\overline{x} = \{x\}\) e quindi è chiuso.
    }
}

\sdefinition{Definizione di convesso in \(\mathbb{R}^n\)}{
    .
}

Sono proprietà topologiche \(T_1\), proprietà di Hausdorff, connessione,
connessione per archi.

% leobartoli@live.fr

\pagebreak

\section{Lezione del 23}

L'operazione di controimmagine fra le topologia presenta tutte le operazioni insiemistiche.
Sezione sugli invarianti topologici? (Sposando anche la definizione di quest'ultima).

\slemma{}{
    Sia \(f \colon X \to Y\) un omeomorfismo.
    Allora, per ogni sottospazio \(S \subseteq X\),
    la restrizione di \(f\) a \(S\) è un omeomorfismo.
}

\sproof{}{
    Un omeomorfismo è un coperazione continua biettica con inverso continua.
    Inoltre vi è la proprietà universale della topologia di sottospazio.
}

\sexample{}{
    L'intervallo \((0,1)\) e \([0,1)\) non sono omeomorfi.
}

\sproof{}{
    Supponiamo che esiste un tale omeomorfismo \(f \colon [0,1) \to (0,1)\).
    Abbiamo che \(f(0) \in (0,1)\).
    Prendiamo \(S = [0,1) \backslash \{0\} = (0,1)\).
    Then, \(f\) restricted to \(S\) is a homeomorphism
    from \(S\) to \[
        f(S) = (0,1) \backslash \{f(0)\} = (0, f(0)) \sqcup (f(0), 1)
    \]
    But \((0,1)\) is connected and \(f(S)\) is not, which is absurd.
}

\sexample{}{
    Sia \(f \colon S^n \to \mathbb{R}\) continua.
    Allora esiste \(x\) tale che \(f(x) = f(-x)\), in particolare non è iniettiva.
}

\sproof{}{
    Sia \(g \colon S^n \to \mathbb{R}\) data da \(g(x) = f(x) - f(-x)\).
    Chiaramente \(g\) è continua. Chiaramente \(g(x) = 0\)
    se e solo se \(f(x) = f(-x)\).
    \(S^n\) è connesso per archi e quindi connesso.
    Allora \(g(S^n)\) è connesso nei reali, ovvero è un intervallo.
    Let \(y \in S^n\).
    Then, \(g(y), g(-y) \in g(S^n)\). Consider
    \[
        \frac{1}{2}g(y) + \frac{1}{2}g(-y) \in g(S)
    \]
    Then
    \[
        \frac{1}{2}\left(
            f(y) - f(-y)
        \right) + \frac{1}{2}\left(
            f(-y) - f(y)
        \right) = 0
    \]
}

\scorollary{}{
    Aperti di \(\mathbb{R}\) non sono omeomorfi ad aperti di \(\mathbb{R}^n\)
    per \(n>1\).
}

\sproof{}{
    Ogni aperto di \(\mathbb{R}^n\) contiene al suo interno
    un sottospazio omeomorfo a \(S^{n-1}\).
    Suppose that there is a homeomorphism
    \(f \colon A \to f(A)\) where \(A\) is open in \(\mathbb{R}^n\)
    and \(f(A)\) is open in \(\mathbb{R}\).
    La reistrizione di \(f\) ad un sottospazio \(B \cong S^{n-1}\)
    è ancora un omeomorfismo. Ma dal risultato precedente non può esistere una tale applicazione biettiva, assurdo.
}

\slemma{}{
    Sia \(f \colon X \to Y\)
    un applicazione continua e suriettiva verso \(Y\) connesso
    e \(\forall y \in Y\), \(f^{-1}(y)\) connesso.
    Allora, se \(f\) è aperta oppure chiusa, \(X\) è connesso.
}

\sproof{}{
    Supponiamo che \(f\) sia aperta senza perdita di generalità.
    Prendiamo \(A_1, A_2\) aperti non vuoti di \(X\)
    tale che \(X = A_1 \cup A_2\).
    Vogliamo mostrare che sono disgiunti.
    Abbiamo che \(Y = f(X) = f(A_1) \cup f(A_2)\).
    Siccome \(Y\) è connesso, la loro intersezione non può essere vuota.
    Abbiamo quindi almeno un punto \(y \in A_1 \intersection A_2\).
    Consideriamo
    \[
        f^{-1}(y) \intersection A_1 \neq \emptyset
        \qquad
        f^{-1}(y) \intersection A_2 \neq \emptyset
    \]
    Per ipotesi
    \[
        f^{-1}(y) = (f^{-1}(y) \intersection A_1)
        \union (f^{-1}(y) \intersection A_2)
    \]
    è aperto. Quindi
    \[
        (f^{-1}(y) \intersection A_1) \intersection (f^{-1}(y) \intersection A_2)
        \neq \emptyset
    \]
    quindi \(A_1 \intersection A_2 \neq \emptyset\).
}

\stheorem{}{
    Siano \(X, Y\) due spazi topologici coneessi.
    Allora, \(X \times Y\) è connesso.
}

\sproof{}{
    Applichiamo il lemma.
    Prendiamo \(p: X \times Y \to Y\) una delle due proiezioni.
    Applichiamo il lemma. \(P\) è aperta (dai risultato sulla topologia prodotto).
    Inoltre \(P\) è continua e suriettiva (se l'insieme \(X\) non è vuoto, in tal caso
    il risultato è banale). Consideriamo la fibra
    \(p^{-1}(y) = X \times \{y\}\) che è omeomorfo ad \(X\), che è connesso.
    Quindi le ipotesi sono soddisfatte e \(Y \times X\) è connesso.
}

Lo stesso vale per la connessione per archi.

\sproof{}{
    Siano \(X,Y\) connessi per archi.
    Allora, \(X \times Y\) è connesso per archi.
}

\sproof{}{
    Mostriamo che ogni coppia di punti è collegata da un cammino.
    Siano quindi \((x,y), (x',y') \in X \times Y\).
    Siccome \(X\) è connesso per archi, esiste un cammino \(\alpha\colon I \to X\)
    tale che \(\alpha(0) = x\) e \(\alpha(1) = x'\).
    Analogamente \(\beta(0) = y\) e \(\beta(1) = y'\).
    Per la proprietà universale del prodotto con \(I\) come vertice,
    esiste un cammino (unico) che fattorizza il diagramma tale che i due triangoli commutino
    mediante le proiezioni.
    Quindi \((\alpha, \beta) \colon I \to X \times Y\)
    dato da \((\alpha, \beta)(t) = (\alpha(t), \beta(t))\).
}

\sdefinition{Componente connessa}{
    Sia \(X\) spazio topologico.
    Un sottospazio \(C\) di \(X\)
    si dice una componente connessa se soddisfa le seguenti:
    \begin{enumerate}
        \item \(C\) è un sottospazio connesso
        \item se \(C \subseteq A\) e \(A\) è connesso allora \(C=A\).
    \end{enumerate}
}

\sexample{}{
    Sia \(X\) uno spazio e \(C \subseteq X\).
    Se \(C\) è sottospazio aperto, chhiuso, connesso e non vuoto, allora è componente connessa.
    Ciò è dato dal fatto che \(C\) è anche chiuso e aperto in \(A\).
}

\slemma{}{
    Sia \(Y\) un sottospazio connesso di \(X\).
    Sia \(W\) un sottospazio tale che \(Y \subseteq W \subseteq \overline{Y}\).
    Allora \(W\) è connesso.
}

\sproof{}{
    Sia \(Z \subseteq W\) aperto, chiuso e non vuoto.
    Consideriamo \(Z \intersection Y\).
    Questo è aperto e chiuso di \(Y\) per
    "transitività" della topologia di sottospazio,
    (cioè se abbiamo una successione di spazi possiamo indurre la topologia di sottospazi in un colpo solo oppure a step).
    Sappiamo che
    \[
        \overline{Y} = \{x \in X \suchthat \text{for every open } A \text{ of } X \text{ where } x \in A, A \intersection Y \neq \emptyset\}
    \]
    Siccome \(Z \neq \emptyset\) è aperto in \(W\),
    \(Z = A \intersection W\) per qualche \(A\) aperto di \(X\).
    Quindi, \(\forall x \in Z \subseteq A\), \(A \intersection Y \neq \emptyset\)
    (e quindi possono prendere un \(x\in Z\)).
    Siccome \(Y\) è connesso,
    deduciamo che \(Z \intersection Y = Y\).
    Infatti quest'ultima intersezione è aperta e chiusa in \(Y\) per definizione di
    topologia di sottospazio e l'intersezione non è vuota.
    Quindi, \(Y \subseteq Z\). Consideriamo la chiusura (in \(W\)) di entrambi
    \(\overline{Y} \subseteq \overline{Z}\).
    Quindi la chiusura in \(W\) è pari a \(\overline{Y} \intersection W\)
    e l'altro \(\overline{Z}=Z\) siccome \(Z\) è chiuso in \(W\) per ipotesi.
    Quindi \(\overline{Y} \subseteq Z\).
    Siccome \(Z \subseteq W\), abbiamo \(W=Z\).
}

\slemma{}{
    Sia \(x\) un punto di uno spazio topologico \(X\)
    e sia \(\{Z_i\}_{i\in I}\) una famiglia di sottospazio connessi di \(X\)
    tali che \(x\in Z_i\).
    Allora, \(\bigcup_i Z_i\) è un sottospazio connesso.
}

\sproof{}{
    Sia
    \[
        W = \bigcup_{i=1} Z_i
    \]
    Dato \(A \subseteq W\) aperto, chiuso e non vuoto,
    vogliamo mostrare che \(A=W\).
    Per ogni \(i \in I\), andiamo a considerare
    \(A \intersection Z_i\) che è un aperto e chiuso di \(Z_i\)
    per definizione di sottospazio.
    Siccome \(Z_1\) è connesso, \(A \intersection Z_i = \emptyset\)
    oppure \(A \intersection Z_i = Z_i\).
    Quest'ultimo è equivalente a dire che \(Z_i \subseteq A\).
    Supponiamo \(A \neq \emptyset\). Quindi
    \[
        A = \bigcup_{i\in I} Z_i \intersection Z_i
    \]
    Allora esiste almeno un \(i \in I\) tale che \(A \intersection Z_i \neq \emptyset\),
    cioè \(Z_i \subseteq A\).
    Dato il punto \(x\) come nelle ipotesi del lemma,
    abbiamo \(x \in Z_i\).
    Segue quindi che \(x \in A\).
    Allora \(x \in Z_i \intersection A\) per ogni \(i \in I\).
    Per ipotesi\(x\in Z_i\) per tutte le \(i\).
    Quindi, \(Z_i \intersection A \neq \emptyset \iff Z_i \subseteq A\), allora
    \[
        W = \bigcup_{i\in I} Z_i \subseteq A
    \]
    cioè \(A = W\).
}

\scorollary{}{
    Siano \(A,B\) due sottospazi connessi di uno spazio topologico.
    Allora, se \(A \intersection B \neq \emptyset\), \(A \union B\) è connesso.
}

\sproof{}{
    Applichiamo il lemma al caso della famiglia \(\{A,B\}\).
}

\slemma{}{
    Sia \(x \in X\) un punto di uno spazio topologico \(X\).
    Denotiamo \(C(x)\) l'unione di tutti i sottospazi connessi di \(X\)
    ch contengono il punto \(x\).
    Allora \(C(x)\) è una componente connessa di \(X\) contenente il punto \(x\).
}

\sproof{}{
    \begin{enumerate}
        \item \(C(x)\) è connesso per il lemma;
        \item \(C(x) \subseteq A\) con \(A\) connesso.
        \(\{x\}\) è un sottospazio connesso di \(X\) contenente \(x\).
        Per definizione \(\{x\} \in C(x)\).
        Abbiamo allroa che \(x \in C(x) \subseteq A\) e quindi \(x\in A\).
        Ma \(A\) è connesso, quindi \(A \subseteq C(x)\).
        Allora \(A = C(x)\).
    \end{enumerate}
}

\stheorem{}{
    Ogni spazio topologico è unione delle sue componenti connesse.
    Ogni componente connessa è chiusa
    e ogni punto è contenuto in una e una sola componente connessa.
}

\sproof{}{
    Siccome \(x\in C(x)\),
    \[
        X = \bigcup_{x\in X} C(x)
    \]
    Sia \(C\) componente connessa. Da un risultato precedente,
    sappiamo che \(\overline{C}\) è ancora un connesso.
    Tuttavia \(C \subseteq \overline{C}\) e per la seconda condizione
    nella definizione di componente connessa, ci deve essere uguaglianza.
    Siano \(C,D\) componenti connesse.
    Supponiamo che non siano disgiunte.
    Abbiamo che \(C \union D\) è connesso in quanto unione dei connessi
    \[C,D \subseteq C \union D\]
    Ciò implica che \(C = C \union D\) e \(D = C \union D\).
    Allora \(C=D\)
}
Questo teorema giustifica la seguente definizione.

\sdefinition{}{
    Sia \(X\) uno spazio topologico e \(x\in X\).
    Allora \(C(x)\) è detta la componente connessa.
}

\sdefinition{Compattezza}{}



\sdefinition{
    Dato \(X\) spazio topologico,
    il \(\pi_0(X)\) è definito
    come l'insieme delle componenti connesse per archi di \(X\).
}

\stheorem{}{
    Sia \(f \colon X \to Y\) un'applicazione contonua.
    Allora se \(X\) è compatto, \(f(X)\) è compatto come sottospazio di \(Y\).
}

\sproof{}{
    Sia \(\mathcal{A}\) una famiglia di aperti di \(Y\) tale che
    \[
        f(X) \subseteq \bigcup_{A \in \mathcal{A}} A
    \]
    Allora \[X = f^{-1}(f(X)) = \bigcup_{A \in \mathcal{A}} f^{-1}(A)\]
    cioè unione di aperti di \(X\) siccome \(f\) è continua.
    Siccome \(X\) è compatto, esistono \(A_1, \cdots, A_n\)
    tali che \(X = f^{-1}(A_1) \union \cdots \union f^{-1}(A_n)\).
    Applicando \(f\) otteniamo
    \[
        f(X) = f(f^{-1}(A_1)) \union \cdots \union f(f^{-1}(A_n))
    \]
    e quindi \(f(X) \subseteq A_1 \union \cdots \union A_n\) che è un ricoprimento finito.
}

\sproposition{}{
    Ogni sottospazio chiuso di uno spazio compatto è compatto.
}

\sproof{}{
    Sia \(X\) compatto e \(C\) chiuso in \(X\).
    Sia \(\mathcal{A}\) una famiglia di aperti tali che
    \[
        C \subseteq \bigcup_{A \in \mathcal{A}} A
    \]
    Notiamo che \(C = (X \backslash C) \union \bigcup A\), dove il primo termine è aperto in quanto complementare
    di un chiuso.
    Essendo \(X\) compatto,
    esistono \(A_1, \cdots, A_n\)
    tali che
    \[
        X = (X \backslash C) \union \bigcup_{i=1}^n A_i
    \]
    quindi
    \[
        C = C \intersection X = (C \intersection (X \backslash C))
        \union \bigcup (A_i \intersection C)
    \]
    ma il primo termine è l'insieme vuoto quindi \(C \subseteq A_1 \union \cdots \union A_n\).
}

\sproposition{}{
    Unione finita di sottospazi compatti è compatta.
}

\sproof{}{
    Sia \(X\) spazio topologico e \(K_1, \cdots, K_n\)
    sottospazi compatti di \(X\).
    Sia
    \[
        K = \bigcup_{i=1}^n K_i
    \]
    Vogliamo dimostrare che \(K\) è compatto.
    Siccome i \(K_i\) sono compatti, possiamo estrarre
    dei sottoricoprimenti finiti da essi. Ma l'unione di questi
    sottoricoprimenti finiti è un sotoricoprimento finito,
    in quanto unioni finite di sottoinsiemi finiti sono sottoinsiemi finiti.
}

\stheorem{}{
    L'intervallo unitario è compatto
    rispetto alla topologia euclidea reale.
}

\sproof{}{
    Sia \(\mathcal{A}\) un ricoprimento aperto di \([0,1]\).
    Quindi
    \[
        [0,1] \subseteq \bigcup_{A \in \mathcal{A}} A
    \]
    Definiamo il sottoinsieme \(X \subseteq [0, \infty)\) come seguente:
    \[
        X = \{t \in [0, +\infty) \suchthat [0,t] \text{ is contained in a finite union of open sets in } \mathcal{A}\}
    \]
    La tesi equivalente a dire che \(1 \in X\).
    Note that \(X \neq \emptyset\) as \(0 \in X\).
    Questo è dato dal fatto che \(0 \in [0,1] \subseteq \bigcup A\).
    Consideriamo ora il supremum di \(X\).
    Abbiamo due possibilità, o \(\sup X > 1\) oppure \(\sup X \leq 1\).
    Nel secondo caso esiste quindi \(t \in X\)
    tale che \(1 < t < \sup X\),
    quindi \([0,t]\) è un sottoinsieme di un unione finita di aperti di \(\mathcal{A}\).
    In particolare \([0,1]\) è uno di questi quindi vale alla tesi.
    Supponiamo invece l'altro caso dove \(b = \sup x > 1\).
    Se \(b \leq 1\) allora \(b\in [0,1]\), quindi esiste \(A \in \mathcal{A}\)
    tale che \(b \in A\) (\(b \geq 0\) in quanto \(0 \in X\)).
    Essendo \(A\) aperto per la topologia euclidea reale, esiste \(\delta>0\)
    tale che \((b-\delta, b+\delta) \subseteq A\).
    D'altra parte, essendo \(b\) il supremum, esiste \(t \in X\)
    tale che \(t \in (b- \delta, b + \delta)\).
    Siccome \(t \in X\), per definizione di \(X\),
    \[
        [0,t] \subseteq A_1 \union \cdots \union A_n
    \]
    Per ogni \(h\) tale che \(0 \leq h < \delta\),
    \[
        [0, b + h] = [0,t] \union [t, b+h]
    \]
    Il primo termine è in \(A_1 \union \cdots \union A_n\)
    mentre il secondo è in \(A \in \mathcal{A}\).
    Questo contraddice il fatto che \(b\) sia il supremum,
    in quanto ciò implicherebbe che \(b + h \in X\).
}

Quindi \(\realnumbers\) e \([0,1]\) non sono omeomorfi.

\scorollary{}{
    Un sottospazio reale è compatto se e solo se
    è chiuso e limitato.
}

\sproof{}{
    \iffproof{
        Sia \(A \subseteq \realnumbers\) compatto.
        Mostriamo che è limitato.
        Consideriamo il ricoprimento aperto \(\{(-n, n) \suchthat n \in \mathbb{N}\}\)
        che ricopre \(\realnumbers\).
        Allora,
        \[
            A \subseteq \bigcup_{n \in \mathbb{N}} (-n, n)
        \]
        Essendo \(A\) compatto, è possibile estrarre un sottoricoprimento finito.
        Quindi, \(A \subseteq [-N, N]\) per qualche \(N > 0\), quindi è limitato.
        Mostriamo ora che \(A\) è chiuso, in particolare mostriamo che \(\overline{A} \subseteq A\).
        Quindi se \(X \backslash A \subseteq X \backslash \overline{A}\),
        ovvero che se \(p \notin A\) allora \(p \notin \overline{A}\).
        Se \(p \notin A\) abbiamo una funzione continua
        da \(A \subseteq \realnumbers \backslash \{p\}\) in \(\realnumbers\)
        data da \(f(x) = 1 / (x - p)\).
        Se \(f\) è continua, siccome \(A\) è compatto, allora \(f(A)\) è compatto,
        quindi anche limitato come dimostratosi prima.
        Questo implica chiaramente che \(p \notin A\).
    }{
        Supponiamo che \(A \subseteq \realnumbers\) sia chiuso e limitato.
        Siccome è limitato, \(A \subseteq [-a, a]\) per qualche \(a \geq 0\).
        Ma \([-a, a]\) è omeomofo a \([0,1]\) quindi è compatto.
    }
}

\sproposition{}{
    Un sottospazio \(K\) di uno spazio topologico \(X\)
    è compatto (per la topologia di sottospazio)
    se e solo se per ogni famiglia \(\mathcal{A}\) di aperti di \(X\) tali che
    \[
        K \subseteq \bigcup_{A \in \mathcal{A}} A
    \]
    esistono \(A_1, \cdots, A_n \in \mathcal{A}\)
    tali che
    \[
        K \subseteq A_1 \subseteq \cdots \subseteq A_n
    \]
}

\sproof{}{
    \(K\) è compatto per la topologia di sottospazio
    ogni ricoprimento aperto \(\mathcal{B}\)
    di \(K\) ammette un sottoricoprimento finito dove
    \(\forall B \in \mathcal{B}\) con \(B = K \intersection A\) con \(A\) aperto di \(X\) vale
    \[
        K = \bigcup_{B\in\mathcal{B}} B \iff K \subseteq \bigcup_{A \intersection K \in \mathcal{B}} A
    \]
    con \(A\) aperto.
    Quindi \(K = B_1 \union \cdots \union B_n\)
    se e solo se \(K \subseteq A_1 \union \cdots \union A_n\),
    dove per ogni \(i=1,2,\cdots, n\), \(B_i \subseteq A_i \intersection K\).
}

Notiamo che ogni insieme finito è compatto per qualunque topologia.

\scorollary{Teorema di Weierstrass}{
    Sia \(X\) uno spazio topologico compatto e \(f \colon X \to \realnumbers\)
    un applicazione continua. Allora \(f\) ammette massimo e minimo.
}

\sproof{}{
    Siccome \(f\) è continua ed \(X\) è compatto, \(f(X)\) è compatto nei reali.
    Ma allora è chiuso e limitato, quindi se consideriamo infimum e supremum
    sono sicuramente numeri reali.
    Dalla definizione di chiusura, infimum e supremum stanno sempre
    nella chiusura. Ma visto che \(f(X)\) è chiuso, choincide con la sua chiusura, quindi
    infimum e supremum stanno nell'insieme stesso, quindi corrispondono a massimo e minimo.
}

\sproposition{}{
    Sia \(\mathcal{B}\) una base di uno spazio topologico \(X\).
    Supponiamo che ogni ricoprimento aperto di \(X\) formato da elementi
    in \(\mathcal{B}\) ammetta un sottoricoprimento finito.
    Allora, \(X\) è compatto.
}

\sproof{}{
    Sia \(\mathcal{A}\) un ricoprimento aperto di \(X\)
    \[
        X = \bigcup_{A \in \mathcal{A}} A
    \]
    Vogliamo dimostrare che da questo ricoprimento si può estrarre un sottoricoprimento finito.
    Per ogni \(A \in \mathcal{A}\), consideriamo \(\mathcal{B}_A = \{B \in \mathcal{B} \suchthat B \subseteq A\}\).
    Chiaramente per definizione di base, \(A = \bigcup \mathcal{B}_A\).
    Quindi
    \[
        X = \bigcup_{A \in \mathcal{A}} A =
        \bigcup_{A \in \mathcal{A}}
        \bigcup_{B \in \mathcal{B}_A} B
    \]
    quindi esistono \(B_1 \in \mathcal{B}_{A_1}, \cdots\)
    tale che
    \[
        X = B_1 \union \cdots \union B_n
    \]
}

\stheorem{}{
    Sia \(f \colon X \to Y\) un'applicazione chiusa, \(Y\) spazio
    compatto, le fibre \(f^{-1}(y)\) compatte. Allora, \(X\) è compatto.
}

\sproof{}{
    Dato \(A \subseteq X\) consideriamo \(A' \subseteq Y\) definito come seguente:
    \[
        A' = \{y \in Y \suchthat f^{-1}(y) \subseteq A\}
    \]
    Mostriamo alcune proprietà:
    \begin{enumerate}
        \item \(Y \backslash Y' = f(X \backslash A)\):
        abbiamo che
        \begin{align*}
            Y \backslash Y' &= \{y \in Y \suchthat \lnot (f^{-1}(y) \subseteq A)\} \\
            &= \{y\in Y \suchthat \exists x \in f^{-1}(y) \suchthat x \notin A\} \\
            &= f(X \backslash A)
        \end{align*}
        \item \(f^{-1}(A') \subseteq A\)
        sia \(x \in f^{-1}(A')\).
        Quindi dire che \(f(x) \in A'\) è come dire
        \(f^{-1}(f(x)) \subseteq A\).
    \end{enumerate}
    Mostriamo che se \(A\) è aperto in \(X\) allora \(A'\) è aperto in \(Y\).
    \(A'\) è aperto in \(Y\) se e solo se \(Y \backslash Y'\) è chiuso in \(Y\).
    Ma \(Y \backslash A' = f(X \backslash A)\) è chiuso in quanto immagine di un chiuso.
    Prendiamo quindi \(\mathcal{A}\) come ricoprimento aperto di \(X\).
    Definiamo \(\mathcal{B}\)
    come la famiglia dei sottoinsiemi di \(X\) esprimibili come
    unioni finite di aperti in \(A\).
    Consideriamo la famiglia \(\mathcal{B}' = \{B' \suchthat B \in \mathcal{B}\}\).
    Mostriamo che questa famiglia è un ricoprimento (aperto)
    di \(Y\).
    Dato \(y\in y\), consideriamo \(f^{-1}(y)\) che è compatto per ipotesi.
    \[
        f^{-1}(y) \subseteq \bigcup_{A \in \mathcal{A}} A
    \]
    quindi
    \(f^{-1}(y) \subseteq A_1 \union \cdots \union A_n\) per qualche \(A_i\).
    Ponendo \(B = A_1 \union \cdots \union A_n\) abbiamo per definizione di \(\mathcal{B}'\)
    che \(y \in \mathcal{B}'\). La compattezza di \(Y\) implica quindi che esistano
    \(B_1', \cdots, B_n'\) tali che \(Y = B_1' \union \cdots \union B_n'\)
    e quindi \[
        X = f^{-1}(Y) = f^{-1}(B_1') \union \cdots \union f^{-1}(B_n')
    \]
    Usando la seconda proprietà dimostrata prima, troviamo
    \[
        X = B_1 \union \cdots \union B_n
    \]
    visto che sono tutte unione finite di aperti in \(\mathcal{A}\),
    allora \(X\) è esprimibile come unione finita di aperti in \(\mathcal{A}\).
}

\pagebreak

\section{Esercizi 21 ottobre}

\sexercise{}{
    Siano \(f,g\) omeomorfismi, allora la loro composizione (se esiste)
    è un omomorfismo e l'inverso è un omomorfismo. Quindi \(\text{Omeo}(X)\) è un gruppo.
    Siano \(X,Y,Z\) spazi tali che \(f\colon Y \to Z\) e \(g \colon X \to Y\).
    Per definizione l'inverso è un omeomorfismo.
    Siccome \((f \circ g)^{-1} = g^{-1} \circ f^{-1}\) se gli inversi sono omeomorfismo allora è un omeomorfismo.
    Quindi è un gruppo.
}

% Calcola il gruppo Aut(Q) di omeomorfismi dove Q è il grafo quadrato (sottoinsieme di R² fatto dai lati e vertici)
% Soluzione: Aut(Q) = D_4 gruppo diedrale = isometria del quadrato euclideo.
% DOmanda corretta:
% Dimostrare che D4 è uguale a Omeo(Q) dove Q è l'insieme parzialmente ordinato dotato
% della topologia discreta di insieme parzialmente ordinato
% con base i coni come di un esercizio precedente.



\end{document}
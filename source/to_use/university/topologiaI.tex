\documentclass[a4paper]{article}

\usepackage{amsmath}
\usepackage{amssymb}
\usepackage{stellar}
\usepackage{parskip}
\usepackage{fullpage}
\usepackage{wrapfig}
\usepackage{tikz}

\usetikzlibrary{arrows}
\usetikzlibrary{decorations.pathreplacing}
\usetikzlibrary{cd}

\title{Topologia I}
\author{Paolo Bettelini}
\date{}

\begin{document}

\maketitle
\tableofcontents

\section{Topologia}

Invarianti: \(p_0\) corrisponde al numero di componenti connesse di uno spazio.
Formalmente \(\pi_0(X)\) è l'insieme delle componenti connesse di \(X\) per archi.
Invece, \(p_1\) è il gruppo fondamentale \(\pi_1(X)\), che descrive la struttura dei cammini
chiusi fino a omotopia.

\saxiom{Estensionalità}{
    \[
        A = B \iff \forall x(x\in A \iff x\in B)
    \]
}

% la logica di boole corrisponde alla logica classica come la logica costruttivista corrisponde alle algebre di H

% mettere le leggi di the morgan generalizzate + le due leggi extra (famiglie arbitrarie)

\sproposition{Relazione di aggiunzione}{
    Valgono
    \[
        S \subseteq f^{-1}(T) \iff
        f(S) \subseteq T
    \]
}
Da cui derivano \(f(f^{-1})(T) \subseteq T\).
Ma in generale l'uguaglianza non vale in quanto \(f\) potrebbe non essere suriettiva.
E pure \(S \subseteq f^{-1}(f(S))\).
Ma in generale l'uguaglianza non vale in quanto \(f\) potrebbe non essere iniettiva.

L'operazione di controimmagine preserva tutte le operazioni insiemistiche.
\begin{align*}
    f^{-1} \left(\bigcup_{i\in I} A_i\right) &= \bigcup{i\in I} f^{-1}(A_i) \\
    f^{-1} \left(\bigcap_{i\in I} A_i\right) &= \bigcap{i\in I} f^{-1}(A_i) \\
    X \backslash f^{-1}(T) &= f^{-1}(Y \backslash T)
\end{align*}

L'operazione di immagine preserva in generale solo le unioni.

\begin{align*}
    f\left(\bigcup_{i\in I} A_i\right) &= \bigcup_{i\in I} f(A_i)
\end{align*}
le altre due non valgono necessariamente. Abbiamo solo
\begin{align*}
    f(A\cap B) &\subseteq f(A) \cap f(B) \\
\end{align*}
se \(f\) non è iniettiva la direzione opposta non vale necessariamente.
Infatti potrebbero esistere \(x,x'\) tale che \(x \in A \backslash B\)
e \(x' \in B \backslash A\) tali che \(f(x) = f(x')\).
La medesima logica vale per il complementare.

\sproposition{Proprietà universale del quoziente}{
    Sia \(f\colon X \to Y\) e \(\sim\) relazione di equivalenza su \(X\). Sono equivalenti:
    \begin{enumerate}
        \item \(f\) è costante sulle classi di equivalenza
            \[ x \sim x' \iff f(x) = f(x') \]
        \item \(f\) fattorizza (in modo necessaria unico, essendo \(\pi\) suriettivo) attraverso
            \(\pi\), cioè \(\exists_{=1} \,g \colon X/_\sim \to Y\) tale che \(g \circ \pi = f\).
    \end{enumerate}
}
\begin{center}
    % https://tikzcd.yichuanshen.de/#N4Igdg9gJgpgziAXAbVABwnAlgFyxMJZABgBpiBdUkANwEMAbAVxiRAA0QBfU9TXfIRQBGclVqMWbAJrdeIDNjwEiZYePrNWiDgHoA+gB1D2ALbdxMKAHN4RUADMAThHOIyIHBCQAmapqkdYzQsEGoGOgAjGAYABX5lIRAnLGsACxw5Rxc3Dy8kUQktNgcskGdXX2p8xELosCgkAFoAZg8A7RBrMJAI6LiEwTYU9MyuCi4gA
    \begin{tikzcd}
    X \arrow[d, "\pi"'] \arrow[r, "f"]   & Y \\
    X/_\sim \arrow[ru, "g"', bend right] &  
    \end{tikzcd}
\end{center}

\sproof{}{
    \begin{enumerate}
        \item \((2) \implies (1):\) \(f = g \circ \pi\).
        Abbiamo \[
            x \sim x' \implies
            \pi(x) = \pi(x') \implies g(\pi(x)) = g(\pi(x'))
        \]
        che sono uguali a \(f(x)\) e \(f(x')\).
        \item \((2) \implies (1):\)
        Definiamo \(g\colon X/_\sim \colon Y\) come
        \[
            g([x]) \triangleq f(x)
        \]
        bisogna verificare che sia ben posta.
        Vogliamo quindi che se \([x] = [x']\) allora \(f(x) = f(x')\).
        Ma ciò è garantito dalla ipotesi.
    \end{enumerate}
}

In \(\mathbb{R}^n\).
\[
    d_\infty(x,y) \leq d_2(x,y) \leq d_1(x,y) \leq n \cdot d_\infty(x,y)
\]

\end{document}
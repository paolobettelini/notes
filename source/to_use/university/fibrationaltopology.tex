\documentclass[a4paper]{article}

\usepackage{amsmath}
\usepackage{amssymb}
\usepackage{stellar}
\usepackage{parskip}
\usepackage{fullpage}
\usepackage{wrapfig}
\usepackage{tikz}

\usetikzlibrary{arrows}
\usetikzlibrary{decorations.pathreplacing}
\usetikzlibrary{cd}

\title{Fibrational topology}
\author{}
\date{}

\begin{document}

\maketitle
\tableofcontents

\section{Fibrational topology}

\emph{Note:} The category \(\mathbf{Top}\) of topological spaces
is the Grothendieck construction of the indexed category
\[
    \mathcal T \colon \mathbf{Set}^{\text{op}} \to \mathbf{Cat}
\]
that assigns to each set the poset of topologies on it, with reindexing given by inverse image.

\begin{center}
    \begin{tikzcd}
    \mathbf{Set} \arrow[r]      & \mathbf{Poset}                                                                                   \\
    X \arrow[d, "f"'] \arrow[r] & {(P_X, \subseteq)} \arrow[d, "(f^\ast)^{-1}", bend left] \arrow[d, "\exists f^\ast"', bend right] \\
    Y                           & {(P_Y, \subseteq)}                                                                               
    \end{tikzcd}
\end{center}
where \((P_X, \subseteq)\) is the collection of all subframes of \(\mathcal P(X)\). \\
We have
\begin{center}
    % https://tikzcd.yichuanshen.de/#N4Igdg9gJgpgziAXAbVABwnAlgFyxMJZABgBpiBdUkANwEMAbAVxiRAAUB9ADRAF9S6TLnyEUARnJVajFmy4BNfoJAZseAkTLjp9Zq0QgAOkYC2dHAAsAxowAE7ABQnzV2wwePuASm-Kh6qJEkjrUenKGLhY29k5RbrGOCr780jBQAObwRKAAZgBOEKZIkiA4EEhkIAxYYAYgUHRwlun+IAVFSADM1OVIAExhsvUmMAAeWHA4cHa5AHomTThtHcWIVX2Ig2V0WAxslhAQANYrhWulmz07ewdHp3wUfEA
    \begin{tikzcd}
    P_X \arrow[d, hook]       & P_Y \arrow[l, dashed] \arrow[d, hook]                 \\
    \mathcal P(\mathcal P(X)) & \mathcal P(\mathcal P(Y)) \arrow[l, "\exists f^\ast"]
    \end{tikzcd}
\end{center}
and also the subframe inclusion
\begin{center}
    % https://tikzcd.yichuanshen.de/#N4Igdg9gJgpgziAXAbVABwnAlgFyxMJZABgBpiBdUkANwEMAbAVxiRAB12BbOnACwDGjAAQAFABQBNAJQgAvqXSZc+QigCM5KrUYs2nHvyEMx4gBqyFS7HgJEy67fWatEIAPIB9SfMUgMNqpEmo7UznpuAGYAepx0cDjiXjLy2jBQAObwRKCRAE4QXEhkIDgQSJo6LmwxcQnCALzCMcAAtOpyvrkFRYgATNRlxYN0WAxsfBAQANZdIPmFSADMg+WIlTij426TM3MLvQOla0tyFHJAA
    \begin{tikzcd}
    \mathcal P(Y) \arrow[r, "f^\ast = f^{-1}"] & \mathcal P(X)               \\
    O_Y \arrow[u, hook] \arrow[r]              & f^\ast(O_Y) \arrow[u, hook]
    \end{tikzcd}
\end{center}
where \(f^\ast(O_Y) = \{f^{-1}(U) \suchthat U \in O_Y\}\). \\
For any \(O_Y \subseteq \mathcal P(Y)\) subframe,
\(\exists f^\ast(O_Y)\) is the ``subspace topology''
on \(X\) induced by \(P_Y\) via \(f\)
(even when \(f\) is not an inclusion).
\(\exists f^\ast (O_Y)\) is a subframe of
\(\mathcal P(X)\) since
\(O_Y\) is a subframe of \(\mathcal P(Y)\) and
\(f^\ast \colon \mathcal P(Y) \to \mathcal P (X)\)
is a frame homomorphism. \\

Dually, we can consider the right adjoint
\((f^\ast)^{-1} \colon \mathcal P (\mathcal P(X)) \fromto \mathcal P(\mathcal P(Y))\)
which sends
\begin{align*}
    O_X \to (f^\ast)^{-1}(O_X) &= \{
        V \in \mathcal P(Y) \suchthat \underbrace{f^\ast(V)}_{f^{-1}} \in O_X
    \}
\end{align*}
This gives the ``quotient topology'' on \(Y\) since \(f^{-1}\) is a frame
induced by \(O_X\) via \(f\). \\
Note that \((f^\ast)^{-1}\) also restricts to a map \(P_X \to P_Y\).

\subsection{General categorical setting}

Let \(\mathcal C\) be a category with pullbacks such that
\(\forall C\), \(\text{Sub}_{\mathcal C}(C)\) is a frame and for any
\(f\colon d \fromto c\), \[f^\ast \colon \text{Sub}_{\mathcal C}(d) \to \text{Sub}_{\mathcal C}(c)\]
is a frame homomorphism
\begin{center}
    % https://tikzcd.yichuanshen.de/#N4Igdg9gJgpgziAXAbVABwnAlgFyxMJZABgBpiBdUkANwEMAbAVxiRAB12BbOnACwDGjAAQBhAHrBOOGAA8cwCGgC+ykMtLpMufIRQBGclVqMWbTj34AjAGbAACphg41GrdjwEiZfcfrNWRBABdU0QDA9dIkNfan8zIPsAfQFhTjgmKzhnGABHNO5eQRF7AAppOQUAZUzlJKlC-iEGMWVSgQBKDtD3HS8UMgAmP1NAkCge8O1PPWRDYbjRtmSoAoysnPyLIubhMor5YBqrOobLYpbRNqgu9WMYKABzeCJQGwAnCC4kMhAcCCQ+jcIA+XyQg2o-yQABZqAw6FYYAxHJF+iB3lhHnwcCBFgE2DZJqDvogIX8AYgAMzA4lISmQikAVmoiLAEypv3hiOR0yiQQYMBsOLxCRApRs4k4dDgOA6kgAtECiZ8SfTyUhmSBWez5ZTfvExpw5FgZXBhBKpTLcSAuUiUX09OjMdi7sogA
    \begin{tikzcd}
    \mathcal C^{\text{op}} \arrow[r] & \mathbf{Poset}                                                                                                                      \\
    c \arrow[d, "f"'] \arrow[r]      & P_c \subseteq \mathcal P(\text{Sub}_{\mathcal C}(c)) \arrow[d, "(f^\ast)^{-1}", bend left] \arrow[d, "\exists f^\ast"', bend right] \\
    d                                & P_d \subseteq \mathcal P(\text{Sub}_{\mathcal C}(d))                                                                               
    \end{tikzcd}
\end{center}
with the pullback \(f\ast \colon \text{Sub}_{\mathcal C}(d) \to \text{Sub}_{\mathcal C}(c)\) along \(f\). \\
One can develop much abstract topology in this setting (e.g. by taking \(P_c\) to be the collection
of all subframes of \(\text{Sub}_{\mathcal C}(c)\)).
Duality exchanges the role of left and right adjoints. \\
This is strictly related to the theory of internal locales.
% https://arxiv.org/pdf/2212.11693

\end{document}
\documentclass[a4paper]{article}

\usepackage{amsmath}
\usepackage{amssymb}
\usepackage{stellar}
\usepackage{parskip}
\usepackage{enumitem}
\usepackage{fullpage}
\usepackage{wrapfig}
\usepackage{tikz}
\usepackage{circuitikz}
\usepackage{siunitx}

\usetikzlibrary{arrows}
\usetikzlibrary{decorations.pathreplacing}
\usetikzlibrary{cd}

\title{Elettromagnetismo}
\author{Paolo Bettelini}
\date{}

\begin{document}

\maketitle
\tableofcontents

\section{Capitolo 1: Forza Elettrostatica e Campo Elettrostatico}

\subsection{1.1 e 1.2 Cariche Elettriche e Struttura della Materia}
La materia è composta da atomi, i quali contengono particelle cariche: \textbf{protoni} (positivi) e \textbf{neutroni} (neutri) nel nucleo, ed \textbf{elettroni} (negativi) che orbitano attorno ad esso.

\begin{itemize}
    \item \textbf{Proprietà della Carica:} La carica elettrica è una proprietà intrinseca della materia. Cariche dello stesso segno si respingono, cariche di segno opposto si attraggono.
    \item \textbf{Conservazione della Carica:} In un sistema isolato, la somma algebrica delle cariche elettriche rimane costante. La carica non viene creata o distrutta, ma solo trasferita.
    \item \textbf{Quantizzazione:} La carica elettrica non è un fluido continuo, ma esiste in "pacchetti" discreti. La carica elementare è quella dell'elettrone:
    $$e \approx 1.602 \times 10^{-19} \text{ C}$$
    Ogni carica $Q$ osservabile in natura è un multiplo intero di $e$ ($Q = ne$, con $n \in \mathbb{Z}$).
    \item \textbf{Conduttori vs Isolanti:}
    \begin{itemize}
        \item \textbf{Conduttori (es. Metalli):} Gli elettroni più esterni (elettroni di valenza) sono debolmente legati e liberi di muoversi nel reticolo cristallino.
        \item \textbf{Isolanti (es. Plastica, Vetro):} Gli elettroni sono fortemente legati ai nuclei e non possono spostarsi facilmente.
    \end{itemize}
\end{itemize}

\subsection{1.3 La Legge di Coulomb}
La forza esercitata da una carica puntiforme $q_1$ su una carica puntiforme $q_2$ posta a distanza $r$ è descritta dalla \textbf{Legge di Coulomb}:

$$\vec{F} = \frac{1}{4\pi\varepsilon_0} \frac{q_1 q_2}{r^2} \hat{r}$$

\begin{itemize}
    \item \textbf{Costante dielettrica del vuoto:} $\varepsilon_0 \approx 8.85 \times 10^{-12} \text{ C}^2/(\text{N}\cdot\text{m}^2)$.
    \item \textbf{Legge dell'inverso del quadrato:} La forza diminuisce con il quadrato della distanza. Se la distanza raddoppia, la forza diventa un quarto.
    \item \textbf{Principio di Sovrapposizione:} Se una carica interagisce con più cariche contemporaneamente, la forza risultante è la somma vettoriale delle singole forze:
    $$\vec{F}_{tot} = \sum_{i} \vec{F}_i$$
\end{itemize}

\subsection{1.4 Il Campo Elettrostatico}
Il campo elettrico $\vec{E}$ è un campo vettoriale che descrive l'influenza che una carica sorgente esercita nello spazio circostante. Invece di un'azione a distanza immediata, pensiamo che la carica modifichi lo spazio e che un'altra carica "senta" questa modifica.

\begin{itemize}
    \item \textbf{Definizione:} Il campo è la forza esercitata su una carica di prova $q_0$ positiva, divisa per il valore della carica stessa:
    $$\vec{E} = \frac{\vec{F}}{q_0} \implies \vec{E} = \frac{1}{4\pi\varepsilon_0} \frac{q}{r^2} \hat{r}$$
    \item \textbf{Utilità:} Una volta noto $\vec{E}$ in un punto, la forza su una generica carica $q$ è semplicemente:
    $$\vec{F} = q\vec{E}$$
\end{itemize}

\subsection{1.5 Campo Elettrostatico Prodotto da Distribuzioni Continue}
Quando la carica non è puntiforme ma distribuita su un corpo esteso, si divide il corpo in elementi infinitesimi di carica $dq$. Il campo totale è l'integrale dei contributi di ogni $dq$:

$$\vec{E} = \frac{1}{4\pi\varepsilon_0} \int \frac{dq}{r^2} \hat{r}$$

\textbf{Modelli di densità di carica:}
\begin{enumerate}
    \item \textbf{Lineare ($\lambda$):} Per fili o sbarre sottili, $dq = \lambda dl$.
    \item \textbf{Superficiale ($\sigma$):} Per fogli o superfici di conduttori, $dq = \sigma dS$.
    \item \textbf{Volumetrica ($\rho$):} Per solidi isolanti carichi uniformemente, $dq = \rho dV$.
\end{enumerate}

\subsection{1.6 Linee di Forza del Campo Elettrostatico}
Le linee di forza sono uno strumento visivo per rappresentare il campo elettrico:
\begin{itemize}
    \item Sono sempre tangenti al vettore $\vec{E}$ in ogni punto.
    \item Escono dalle cariche positive (sorgenti) ed entrano in quelle negative (pozzi).
    \item La \textbf{densità delle linee} (quante linee passano in un'area) è proporzionale all'intensità del campo $E$.
    \item Le linee di forza non si incrociano mai.
\end{itemize}

\subsection{1.7 Moto di una Carica in un Campo Elettrostatico}
L'interazione elettrica produce moto. Applicando la seconda legge di Newton ($F = ma$):
$$\vec{F} = q\vec{E} = m\vec{a} \implies \vec{a} = \frac{q}{m}\vec{E}$$

\begin{itemize}
    \item \textbf{In un campo uniforme:} L'accelerazione è costante. La particella segue le leggi del moto uniformemente accelerato.
    \item Se la velocità iniziale $\vec{v}_0$ è perpendicolare a $\vec{E}$, la traiettoria è \textbf{parabolica} (analogo al moto di un proiettile in un campo gravitazionale).
\end{itemize}

\subsection{1.8 Determinazione della Carica Elementare: Esperienza di Millikan}
L'esperimento di Millikan (1909) provò definitivamente la quantizzazione della carica.
\begin{itemize}
    \item \textbf{Procedura:} Piccole gocce d'olio cariche vengono spruzzate tra due piastre orizzontali. Viene applicato un campo $\vec{E}$ verticale per bilanciare la forza di gravità.
    \item \textbf{Equilibrio delle forze:} Quando la goccia è sospesa:
    $$qE = mg \implies q = \frac{mg}{E}$$
    \item \textbf{Conclusione:} Millikan scoprì che i valori di $q$ misurati erano sempre multipli interi di un valore base $1.6 \times 10^{-19} \text{ C}$, dimostrando che la carica elettrica non è divisibile all'infinito.
\end{itemize}

\section{Capitolo 2: Lavoro Elettrico e Potenziale Elettrostatico}

\subsection{2.1 e 2.6 Lavoro della Forza Elettrica e Conservatività}
Il lavoro compiuto dalla forza elettrica per spostare una carica $q$ da un punto $A$ a un punto $B$ lungo un percorso $L$ è definito come:
$$W_{AB} = \int_{A}^{B} \vec{F} \cdot d\vec{s} = q \int_{A}^{B} \vec{E} \cdot d\vec{s}$$

\begin{itemize}
    \item \textbf{Indipendenza dal percorso:} Una proprietà fondamentale del campo elettrostatico è che questo integrale dipende solo dalle posizioni dei punti $A$ e $B$, non dalla traiettoria seguita.
    \item \textbf{Conseguenza 1 (Integrale di linea):} Per ogni percorso chiuso (linea circuitale $\gamma$), il lavoro totale è nullo:
    $$\oint_{\gamma} \vec{E} \cdot d\vec{s} = 0$$
    \item \textbf{Conseguenza 2 (Rotore):} Applicando il teorema di Stokes, l'irrotazionalità del campo si esprime in forma locale come:
    $$\nabla \times \vec{E} = 0$$
    \item \textbf{Intuizione:} Un campo con rotore nullo è detto \textbf{irrotazionale}. Non presenta "vortici" o linee chiuse. Se una carica compie un giro completo e torna al punto di partenza, l'energia netta scambiata col campo è zero.
\end{itemize}



\subsection{2.1 e 2.2 Il Potenziale Elettrostatico (V)}
Poiché il campo è conservativo, è possibile definire una funzione scalare chiamata \textbf{Potenziale Elettrostatico} $V$. La differenza di potenziale (o tensione) tra due punti è il lavoro per unità di carica (con segno invertito):
$$V(B) - V(A) = - \int_{A}^{B} \vec{E} \cdot d\vec{s}$$

\begin{itemize}
    \item \textbf{Potenziale di una carica puntiforme:} Assumendo convenzionalmente $V = 0$ all'infinito:
    $$V(r) = \frac{1}{4\pi\varepsilon_0} \frac{q}{r}$$
    \item \textbf{Distribuzioni continue:} Il potenziale è la somma scalare dei contributi di ogni elemento di carica $dq$. A differenza del campo $\vec{E}$, qui non serve scomporre vettori:
    $$V(P) = \frac{1}{4\pi\varepsilon_0} \int \frac{dq}{r}$$
\end{itemize}

\subsection{2.3 Energia Potenziale Elettrostatica (U)}
L'energia potenziale $U$ rappresenta il lavoro necessario per "assemblare" una configurazione di cariche portandole dall'infinito (dove non interagiscono) alle loro posizioni attuali.

\begin{itemize}
    \item \textbf{Sistema di due cariche:} $U = \frac{1}{4\pi\varepsilon_0} \frac{q_1 q_2}{r_{12}}$
    \item \textbf{Sistema di $n$ cariche:} 
    $$U = \frac{1}{2} \sum_{i=1}^{n} q_i V_i = \frac{1}{2} \sum_{i \neq j} \frac{1}{4\pi\varepsilon_0} \frac{q_i q_j}{r_{ij}}$$
    \item \textbf{Il fattore $1/2$:} È necessario per evitare di conteggiare due volte l'energia di interazione tra la stessa coppia di cariche (es. l'interazione tra $q_1$ e $q_2$ è identica a quella tra $q_2$ e $q_1$).
\end{itemize}

\subsection{2.4 e 2.5 Il Campo come Gradiente del Potenziale}
Il legame matematico inverso tra $\vec{E}$ e $V$ è dato dall'operatore gradiente. Il campo elettrico è il gradiente del potenziale cambiato di segno:
$$\vec{E} = -\nabla V = - \left( \frac{\partial V}{\partial x} \hat{i} + \frac{\partial V}{\partial y} \hat{j} + \frac{\partial V}{\partial z} \hat{k} \right)$$

\begin{itemize}
    \item \textbf{Significato fisico:} Il campo $\vec{E}$ punta sempre nella direzione in cui il potenziale \textbf{diminuisce} più rapidamente.
    \item \textbf{Superfici Equipotenziali:} Sono superfici su cui il potenziale assume lo stesso valore ($V = \text{cost}$). Poiché $\vec{E}$ è il gradiente di $V$, il vettore campo elettrico è \textbf{sempre perpendicolare} alle superfici equipotenziali in ogni punto.
\end{itemize}



\subsection{2.7 e 2.8 Il Dipolo Elettrico}
Un dipolo è costituito da due cariche di segno opposto $+q$ e $-q$ separate da una distanza $d$.

\begin{itemize}
    \item \textbf{Momento di dipolo ($\vec{p}$):} $\vec{p} = q\vec{d}$ (vettore diretto dalla carica negativa alla positiva).
    \item \textbf{Potenziale a grande distanza ($r \gg d$):} 
    $$V(r, \theta) = \frac{1}{4\pi\varepsilon_0} \frac{\vec{p} \cdot \hat{r}}{r^2} = \frac{p \cos \theta}{4\pi\varepsilon_0 r^2}$$
    \textit{Nota:} Il potenziale del dipolo decade come $1/r^2$, più velocemente del potenziale di una carica singola ($1/r$).
    \item \textbf{Momento meccanico (Torque):} In un campo esterno $\vec{E}$, il dipolo subisce una coppia che tende ad allinearlo al campo:
    $$\vec{\tau} = \vec{p} \times \vec{E}$$
    \item \textbf{Energia Potenziale del dipolo:} $U = -\vec{p} \cdot \vec{E}$. L'energia è minima (stabilità) quando il dipolo è parallelo al campo.
    \item \textbf{Forza:} In un campo \textbf{uniforme}, la forza netta è nulla. In un campo \textbf{non uniforme}, il dipolo subisce una forza netta che lo spinge verso le regioni di campo più intenso.
\end{itemize}

\section{Capitolo 3: La Legge di Gauss}

\subsection{3.1 Il Concetto di Flusso ($\Phi$)}
Il flusso del campo elettrico è una grandezza scalare che misura la "quantità" di campo che attraversa una determinata superficie.
\begin{itemize}
    \item \textbf{Intuizione:} Si può immaginare il campo elettrico come un flusso di un fluido. Il flusso attraverso una superficie dipende dall'intensità del campo, dall'estensione dell'area e dall'orientamento reciproco.
    \item \textbf{Definizione matematica:} Per una superficie infinitesima $d\vec{A}$ (il cui vettore area è normale alla superficie stessa), il flusso infinitesimo è:
    $$d\Phi_E = \vec{E} \cdot d\vec{A} = E \, dA \cos\theta$$
    \item \textbf{Superficie generica $S$:} Il flusso totale è l'integrale di superficie:
    $$\Phi_S(\vec{E}) = \int_S \vec{E} \cdot \hat{n} \, dS$$
\end{itemize}



\subsection{La Legge di Gauss (Forma Integrale)}
La legge di Gauss stabilisce una relazione fondamentale tra il flusso del campo attraverso una superficie chiusa $\Omega$ (detta \textbf{superficie gaussiana}) e la carica netta contenuta al suo interno:
$$\Phi_{\Omega}(\vec{E}) = \oint_{\Omega} \vec{E} \cdot d\vec{A} = \frac{Q_{int}}{\varepsilon_0}$$
\begin{itemize}
    \item \textbf{Nota Fondamentale:} Le cariche poste all'esterno della superficie chiusa non contribuiscono al flusso totale netto. Le loro linee di forza, se entrano nella superficie, devono necessariamente uscirne, portando a un contributo nullo nel calcolo complessivo.
\end{itemize}

\subsection{3.2 Dimostrazione e intuizione del Solido Angolo}
La validità della legge di Gauss (proporzionalità a $1/\varepsilon_0$) deriva direttamente dalla legge di Coulomb.
\begin{itemize}
    \item \textbf{Caso della sfera:} Considerando una carica puntiforme $q$ al centro di una sfera di raggio $r$:
    $$\vec{E} = \frac{1}{4\pi\varepsilon_0} \frac{q}{r^2} \hat{r}, \quad d\vec{A} = dA \, \hat{r}$$
    $$\oint \vec{E} \cdot d\vec{A} = \oint \frac{1}{4\pi\varepsilon_0} \frac{q}{r^2} dA = \frac{q}{4\pi\varepsilon_0 r^2} \oint dA$$
    Poiché l'area della sfera è $4\pi r^2$, i termini $r^2$ si cancellano: $\Phi = \frac{q}{\varepsilon_0}$.
    \item \textbf{Superfici arbitrarie:} Per superfici di forma qualunque, si ricorre al concetto di \textbf{angolo solido} $d\Omega$:
    $$d\Omega = \frac{dA \cos\theta}{r^2}$$
    L'integrale dell'angolo solido su una superficie chiusa che racchiude la sorgente è sempre $4\pi$.
\end{itemize}

\subsection{3.3 Applicazioni della Legge di Gauss}
La legge di Gauss permette di calcolare il campo $\vec{E}$ in modo immediato in presenza di elevate simmetrie, scegliendo una superficie gaussiana su cui il modulo di $E$ sia costante e il vettore sia parallelo o perpendicolare ad ogni elemento di area.

\begin{enumerate}
    \item \textbf{Simmetria Sferica (Sfera carica di raggio $R$ e carica $Q$):}
    \begin{itemize}
        \item \textbf{All'esterno ($r > R$):} $E(4\pi r^2) = \frac{Q}{\varepsilon_0} \implies E = \frac{1}{4\pi\varepsilon_0} \frac{Q}{r^2}$ (comportamento da carica puntiforme).
        \item \textbf{All'interno ($r < R$):} Per una distribuzione uniforme di densità $\rho$, la carica interna è $Q_{int} = \rho \cdot \frac{4}{3}\pi r^3$.
        $$E(4\pi r^2) = \frac{\rho \frac{4}{3}\pi r^3}{\varepsilon_0} \implies E = \frac{\rho r}{3\varepsilon_0}$$
    \end{itemize}
    \item \textbf{Simmetria Cilindrica (Filo rettilineo infinito con densità $\lambda$):}
    Si sceglie un cilindro coassiale di raggio $r$ e altezza $h$. Il flusso attraversa solo la superficie laterale.
    $$E(2\pi r h) = \frac{\lambda h}{\varepsilon_0} \implies E = \frac{\lambda}{2\pi\varepsilon_0 r}$$
    \item \textbf{Simmetria Piana (Lastra infinita con densità $\sigma$):}
    Si sceglie un cilindro ortogonale alla lastra. Il flusso esce solo dalle due basi di area $A$.
    $$2EA = \frac{\sigma A}{\varepsilon_0} \implies E = \frac{\sigma}{2\varepsilon_0}$$
\end{enumerate}

\subsection{3.4 La Divergenza e la Prima Equazione di Maxwell}
Passando dalla formulazione integrale a quella locale (puntuale) tramite il \textbf{Teorema della Divergenza}:
$$\oint_{\Omega} \vec{E} \cdot d\vec{A} = \int_{V} (\nabla \cdot \vec{E}) \, dV$$
Sostituendo la legge di Gauss $\oint \vec{E} \cdot d\vec{A} = \frac{1}{\varepsilon_0} \int \rho \, dV$, si ottiene la \textbf{Prima Equazione di Maxwell}:
$$\nabla \cdot \vec{E} = \frac{\rho}{\varepsilon_0}$$
\begin{itemize}
    \item \textbf{Significato fisico:} La divergenza di $\vec{E}$ in un punto rappresenta la densità locale di sorgenti di campo. 
    \item Se $\rho > 0$, il punto è una \textbf{sorgente} (linee uscenti).
    \item Se $\rho < 0$, il punto è un \textbf{pozzo} (linee entranti).
    \item Se $\rho = 0$, il flusso locale è nullo: tante linee entrano quante ne escono.
\end{itemize}

\section{Capitolo 4: Conduttori, Dielettrici ed Energia Elettrostatica}

\subsection{4.1 e 4.2 Conduttori e Schermo Elettrostatico}
In un conduttore all'equilibrio elettrostatico, le cariche sono libere di muoversi fino a raggiungere una configurazione di stallo. Questo stato impone tre proprietà fondamentali:
\begin{itemize}
    \item \textbf{Il campo interno è nullo:} $\vec{E}_{int} = 0$. Se esistesse un campo, le cariche continuerebbero a muoversi a causa della forza di Lorentz.
    \item \textbf{Il potenziale è costante:} Poiché $\vec{E} = -\nabla V = 0$, il potenziale $V$ deve essere lo stesso in ogni punto del conduttore. Esso è dunque un \textbf{volume equipotenziale}.
    \item \textbf{La carica risiede sulla superficie:} Ogni eccesso di carica si distribuisce esclusivamente sulla "pelle" esterna del conduttore per minimizzare la repulsione reciproca.
    \item \textbf{Teorema di Coulomb:} Il campo elettrico nelle immediate vicinanze della superficie di un conduttore è perpendicolare alla superficie stessa e ha modulo:
    $$\vec{E} = \frac{\sigma}{\varepsilon_0} \hat{n}$$
    Dove $\sigma$ è la densità superficiale locale di carica.
\end{itemize}

\textbf{Schermo Elettrostatico (Gabbia di Faraday):}
In un conduttore cavo, il campo all'interno della cavità è rigorosamente nullo se non sono presenti cariche interne alla cavità stessa. Qualsiasi variazione del campo elettrico esterno viene compensata dal movimento delle cariche sulla superficie esterna, rendendo lo spazio interno isolato dalle influenze elettriche esterne.



\subsection{4.3 e 4.4 Capacità e Condensatori}
Un condensatore è un sistema formato da due conduttori affacciati (armature) con cariche $+Q$ e $-Q$. La capacità $C$ è una grandezza geometrica che misura l'attitudine del sistema ad accumulare carica per unità di differenza di potenziale.

\begin{itemize}
    \item \textbf{Definizione:} $C = \frac{Q}{\Delta V}$. L'unità di misura è il Farad ($[F] = [C/V]$).
    \item \textbf{Condensatore piano:} Per due armature parallele di area $A$ a distanza $d$:
    $$C = \frac{\varepsilon_0 A}{d}$$
\end{itemize}

\textbf{Collegamento di Condensatori:}
\begin{itemize}
    \item \textbf{In Parallelo:} I condensatori condividono la stessa differenza di potenziale $\Delta V$. La capacità equivalente è la somma delle singole capacità:
    $$C_{eq} = C_1 + C_2 + \dots = \sum C_i$$
    \item \textbf{In Serie:} I condensatori accumulano la stessa carica $Q$. Il reciproco della capacità equivalente è la somma dei reciproci:
    $$\frac{1}{C_{eq}} = \frac{1}{C_1} + \frac{1}{C_2} + \dots = \sum \frac{1}{C_i}$$
\end{itemize}



\subsection{4.5 Energia del Campo Elettrostatico}
L'energia elettrostatica non risiede nelle cariche stesse, ma è immagazzinata nel campo elettrico che esse generano nello spazio.

\begin{itemize}
    \item \textbf{Energia in un condensatore:} È il lavoro compiuto per caricare le armature:
    $$U = \frac{1}{2} Q \Delta V = \frac{1}{2} C (\Delta V)^2 = \frac{1}{2} \frac{Q^2}{C}$$
    \item \textbf{Densità di energia ($u_e$):} L'energia per unità di volume in una regione dove è presente un campo $E$:
    $$u_e = \frac{1}{2} \varepsilon_0 E^2$$
\end{itemize}

\subsection{4.6, 4.7 e 4.8 I Dielettrici e il Vettore Induzione Elettrica}
L'inserimento di un materiale isolante (dielettrico) tra le armature di un condensatore ne aumenta la capacità. Gli atomi del dielettrico subiscono una \textbf{polarizzazione}, orientando i propri dipoli atomici in modo da creare un campo interno che si oppone a quello esterno.

\begin{itemize}
    \item \textbf{Costante dielettrica relativa ($\varepsilon_r$):} Parametro adimensionale ($>1$) che indica il fattore di potenziamento della capacità:
    $$C = \varepsilon_r C_0 \implies \varepsilon = \varepsilon_0 \varepsilon_r$$
    \item \textbf{Vettore Polarizzazione ($\vec{P}$):} Momento di dipolo per unità di volume:
    $$\vec{P} = \varepsilon_0 (\varepsilon_r - 1) \vec{E}$$
\end{itemize}

\textbf{Il Vettore Induzione Elettrica ($\vec{D}$):}
Per semplificare i calcoli in presenza di dielettrici (dove convivono cariche libere e cariche di polarizzazione), si introduce il vettore $\vec{D}$, che dipende esclusivamente dalle cariche libere controllabili:
$$\vec{D} = \varepsilon_0 \vec{E} + \vec{P} = \varepsilon \vec{E}$$

\textbf{Legge di Gauss generalizzata:}
Il flusso del vettore $\vec{D}$ attraverso una superficie chiusa è pari alla sola carica libera contenuta all'interno:
$$\oint \vec{D} \cdot d\vec{A} = Q_\text{libere}$$

\section{Capitolo 5: Corrente Elettrica}

\subsection{5.1 e 5.2 Definizione di Corrente e Densità di Corrente}
La corrente elettrica rappresenta il movimento ordinato di cariche elettriche attraverso un conduttore.

\begin{itemize}
    \item \textbf{Corrente Elettrica ($I$):} Definita come la quantità di carica $dQ$ che attraversa una sezione del conduttore nell'intervallo di tempo $dt$:
    $$I = \frac{dQ}{dt}$$
    L'unità di misura è l'Ampere ($[A] = [C/s]$).
    \item \textbf{Densità di corrente ($\vec{J}$):} Grandezza vettoriale che descrive il flusso di carica per unità di superficie:
    $$\vec{J} = nq\vec{v}_d$$
    Dove $n$ è la densità numerica dei portatori, $q$ la carica del portatore e $\vec{v}_d$ la \textbf{velocità di deriva} (la velocità media con cui le cariche avanzano nel conduttore).
    \item \textbf{Corrente stazionaria:} Si ha quando la carica non si accumula in alcuna regione del circuito. Per il principio di conservazione della carica, la divergenza della densità di corrente è nulla:
    $$\nabla \cdot \vec{J} = 0$$
\end{itemize}

\subsection{5.3 e 5.4 Legge di Ohm e Modello di Drude}
La conduzione elettrica nei materiali è regolata dalla risposta dei portatori di carica ai campi elettrici applicati.

\begin{itemize}
    \item \textbf{Legge di Ohm (forma locale):} In molti materiali (detti ohmici), la densità di corrente è proporzionale al campo elettrico:
    $$\vec{J} = \sigma \vec{E}$$
    Dove $\sigma$ è la \textbf{conducibilità} elettrica del materiale. Il suo reciproco è la resistività $\rho = 1/\sigma$.
    \item \textbf{Legge di Ohm (forma macroscopica):} Per un conduttore di lunghezza $L$ e sezione $A$:
    $$V = RI$$
    Dove la \textbf{resistenza} $R$ è data dalla seconda legge di Ohm: $R = \rho \frac{L}{A}$.
    \item \textbf{Modello Classico di Drude:} Spiega perché la velocità di deriva è costante nonostante l'accelerazione del campo. Gli elettroni subiscono urti con i nuclei del reticolo cristallino, subendo una forza di attrito viscoso. Definendo il \textbf{tempo di collisione medio} $\tau$:
    $$\vec{v}_d = \frac{q\tau}{m}\vec{E}$$
\end{itemize}


\subsection{5.5 Resistori in Serie e in Parallelo}
Il comportamento dei resistori è opposto a quello dei condensatori:
\begin{itemize}
    \item \textbf{Serie:} Tutti i resistori sono attraversati dalla stessa corrente $I$. La resistenza equivalente è la somma delle singole resistenze:
    $$R_{eq} = R_1 + R_2 + \dots = \sum R_i$$
    \item \textbf{Parallelo:} Tutti i resistori sono sottoposti alla stessa differenza di potenziale $V$. Il reciproco della resistenza equivalente è la somma dei reciproci:
    $$\frac{1}{R_{eq}} = \frac{1}{R_1} + \frac{1}{R_2} + \dots = \sum \frac{1}{R_i}$$
\end{itemize}

\subsection{5.6 Forza Elettromotrice (f.e.m.)}
Per mantenere una corrente costante, è necessario un dispositivo (generatore) che compia lavoro contro le forze elettrostatiche per riportare le cariche dai punti a potenziale basso a quelli a potenziale alto.
\begin{itemize}
    \item \textbf{Definizione:} La f.e.m. $\mathcal{E}$ è il lavoro speso dal generatore per unità di carica:
    $$\mathcal{E} = \frac{dW}{dq}$$
    \item \textbf{Generatore reale:} Possiede una resistenza interna $r$. La tensione effettiva ai morsetti diminuisce all'aumentare della corrente erogata:
    $$V = \mathcal{E} - rI$$
\end{itemize}

\subsection{5.7 Circuiti RC (Carica e Scarica)}
Analisi della dinamica temporale di un circuito contenente un resistore $R$ e un condensatore $C$.
\begin{itemize}
    \item \textbf{Carica del condensatore:} Partendo da scarico ($Q=0$ a $t=0$):
    $$Q(t) = C\mathcal{E}(1 - e^{-t/\tau}) \quad , \quad I(t) = \frac{\mathcal{E}}{R} e^{-t/\tau}$$
    Dove $\tau = RC$ è la \textbf{costante di tempo}.
    \item \textbf{Scarica del condensatore:} Partendo da carico con carica $Q_0$:
    $$Q(t) = Q_0 e^{-t/\tau} \quad , \quad I(t) = I_0 e^{-t/\tau}$$
\end{itemize}


\subsection{5.8 Corrente di Spostamento}
Nelle armature di un condensatore in fase di carica non fluisce carica fisica, ma il campo elettrico sta variando. Maxwell intuì che questa variazione produce gli stessi effetti magnetici di una corrente reale.
\begin{itemize}
    \item \textbf{Definizione:} La corrente di spostamento $I_d$ è legata alla variazione del flusso del campo elettrico $\Phi_E$:
    $$I_d = \varepsilon_0 \frac{d\Phi_E}{dt}$$
    Questa corrente garantisce la continuità della corrente totale nel circuito.
\end{itemize}

\subsection{5.9 Leggi di Kirchhoff}
Fondamentali per l'analisi delle reti elettriche:
\begin{enumerate}
    \item \textbf{Legge dei Nodi (KCL):} Basata sulla conservazione della carica. La somma algebrica delle correnti in un nodo è zero:
    $$\sum I = 0$$
    \item \textbf{Legge delle Maglie (KVL):} Basata sulla conservazione dell'energia. La somma algebrica delle differenze di potenziale lungo una maglia chiusa è zero:
    $$\sum \Delta V = 0$$
\end{enumerate}

\section{Capitolo 6: Campo Magnetico e Forza Magnetica}

\subsection{6.1 e 6.2 Origine del Campo Magnetico ($\vec{B}$)}
Il magnetismo non è una forza distinta dall'elettricità, ma è la manifestazione delle cariche elettriche in movimento. 

\begin{itemize}
    \item \textbf{Intuizione:} In un sistema di riferimento solidale con una carica in movimento, si osserverebbe solo un campo elettrico; il campo magnetico emerge come effetto relativistico dovuto al moto delle cariche rispetto all'osservatore.
    \item \textbf{Sorgenti:} Mentre le cariche statiche generano campi elettrici, le \textbf{correnti elettriche} (cariche in moto) generano campi magnetici.
    \item \textbf{Unità di misura:} Il campo magnetico si misura in Tesla ($T$). Nota: $1 \, T$ è un'intensità molto elevata (il campo magnetico terrestre è di circa $5 \times 10^{-5} \, T$).
\end{itemize}

\subsection{6.3 La Forza di Lorentz}
Una carica puntiforme $q$ che si muove con velocità $\vec{v}$ all'interno di un campo magnetico $\vec{B}$ subisce una forza data dal prodotto vettoriale:
$$\vec{F}_m = q (\vec{v} \times \vec{B})$$

\begin{itemize}
    \item \textbf{Modulo:} $F = |q| v B \sin\theta$, dove $\theta$ è l'angolo tra i vettori $\vec{v}$ e $\vec{B}$. La forza è nulla se il moto è parallelo alle linee di campo.
    \item \textbf{Direzione:} La forza è sempre \textbf{perpendicolare} sia alla velocità della particella che alla direzione del campo magnetico.
    \item \textbf{Lavoro Nullo:} Poiché $\vec{F}_m$ è costantemente ortogonale allo spostamento ($\vec{F}_m \perp \vec{v}$), la forza magnetica non compie lavoro: $W = 0$. Essa non può variare l'energia cinetica (il modulo della velocità) della particella, ma ne cambia solo la direzione (forza deflettente).
    \item \textbf{Regola della mano destra:} Per determinare il verso di $\vec{F}$ su una carica positiva: pollice su $\vec{v}$, dita tese su $\vec{B}$; il verso della forza esce dal palmo.
\end{itemize}



\subsection{6.4 Forza su un conduttore (Seconda Legge di Laplace)}
Poiché la forza di Lorentz agisce sui singoli portatori di carica, un conduttore percorso da una corrente $I$ immerso in un campo $\vec{B}$ subisce una forza macroscopica. Per un elemento infinitesimo di filo $d\vec{l}$:
$$d\vec{F} = I d\vec{l} \times \vec{B}$$
Per un conduttore rettilineo di lunghezza $L$ in un campo uniforme: $\vec{F} = I (\vec{L} \times \vec{B})$.

\subsection{6.5 Momenti meccanici su circuiti piani}
In una spira chiusa immersa in un campo $\vec{B}$ uniforme, la forza totale è nulla, ma si genera una coppia di forze che tende a far ruotare la spira.

\begin{itemize}
    \item \textbf{Momento di dipolo magnetico ($\vec{m}$):} $\vec{m} = I A \hat{n}$, dove $A$ è l'area della spira e $\hat{n}$ è il versore normale (determinato dalla regola della mano destra applicata al verso della corrente).
    \item \textbf{Momento meccanico (Torque):} $\vec{\tau} = \vec{m} \times \vec{B}$.
    \item \textbf{Energia potenziale magnetica:} $U = -\vec{m} \cdot \vec{B}$. Il sistema raggiunge la stabilità (energia minima) quando il momento di dipolo è allineato con il campo esterno.
\end{itemize}



\subsection{6.6 Effetto Hall}
Fenomeno che si verifica quando un conduttore percorso da corrente è immerso in un campo magnetico trasversale.
\begin{itemize}
    \item \textbf{Meccanismo:} La forza di Lorentz devia i portatori di carica verso un lato del conduttore, accumulandoli e creando una separazione di carica.
    \item \textbf{Campo di Hall ($\vec{E}_H$):} L'accumulo di cariche genera un campo elettrico trasversale che si oppone alla forza magnetica. All'equilibrio: $qE_H = q v_d B$.
    \item \textbf{Tensione di Hall:} $V_H = E_H \cdot w = \frac{I B}{n q d}$, dove $n$ è la densità dei portatori e $d$ lo spessore della lamina.
    \item \textbf{Importanza:} Permette di determinare il segno dei portatori di carica (elettroni o lacune) e la loro densità numerica $n$.
\end{itemize}



\subsection{6.7 e 6.8 Moto di una particella in un campo magnetico}
\begin{itemize}
    \item \textbf{Moto Circolare Uniforme ($\vec{v} \perp \vec{B}$):} La forza di Lorentz agisce come forza centripeta ($qvB = mv^2/r$).
    \begin{itemize}
        \item \textbf{Raggio di ciclotrone:} $r = \frac{mv}{qB}$.
        \item \textbf{Pulsazione di ciclotrone:} $\omega = \frac{qB}{m}$. Si noti che la frequenza non dipende dalla velocità della particella.
    \end{itemize}
    \item \textbf{Moto Elicoidale (angolo generico):} Se la velocità ha una componente $v_{\parallel}$ parallela a $\vec{B}$ e una $v_{\perp}$ perpendicolare, la particella descrive un'elica cilindrica. La componente parallela rimane costante, mentre quella perpendicolare genera il moto circolare.
\end{itemize}

\section{Capitolo 7: Sorgenti del Campo Magnetico e Proprietà della Materia}

\subsection{7.1 e 7.2 La Legge di Biot-Savart e Campi di Circuiti Particolari}
Per calcolare il campo magnetico $\vec{B}$ generato da un conduttore di forma arbitraria, si utilizza la \textbf{Legge di Biot-Savart}, che rappresenta l'equivalente della legge di Coulomb per la magnetostatica. 

\begin{itemize}
    \item \textbf{Formula infinitesima:} Un elemento di filo $d\vec{l}$ percorso da una corrente $I$ produce in un punto a distanza $r$ un campo:
    $$d\vec{B} = \frac{\mu_0}{4\pi} \frac{I d\vec{l} \times \hat{r}}{r^2}$$
    \item \textbf{Permeabilità magnetica del vuoto:} $\mu_0 = 4\pi \times 10^{-7} \text{ T}\cdot\text{m/A}$.
    \item \textbf{Intuizione:} Il campo "avvolge" il filo. Puntando il pollice della mano destra nel verso della corrente, le dita indicano il verso delle linee di campo circolari.
\end{itemize}

\textbf{Casi Classici (Applicazioni dirette):}
\begin{enumerate}
    \item \textbf{Filo rettilineo infinito (Legge di Biot-Savart):} A distanza $r$ dal filo, il modulo è:
    $$B = \frac{\mu_0 I}{2\pi r}$$
    \item \textbf{Spira circolare (sull'asse $z$):} Al centro della spira ($z=0$) il campo è massimo. Lungo l'asse:
    $$B(z) = \frac{\mu_0 I R^2}{2(R^2 + z^2)^{3/2}}$$
    \item \textbf{Solenoide ideale:} All'interno di un solenoide molto lungo con $n$ spire per unità di lunghezza, il campo è uniforme e parallelo all'asse:
    $$B = \mu_0 n I$$
\end{enumerate}



\subsection{7.3 Azioni Elettrodinamiche tra Fili}
Due fili paralleli percorsi da corrente interagiscono tramite i rispettivi campi magnetici.
\begin{itemize}
    \item \textbf{Attrazione:} Se le correnti scorrono nello \textbf{stesso verso}.
    \item \textbf{Repulsione:} Se le correnti scorrono in \textbf{verso opposto}.
    \item \textbf{Forza per unità di lunghezza ($L$):} Date due correnti $I_1$ e $I_2$ a distanza $d$:
    $$\frac{F}{L} = \frac{\mu_0 I_1 I_2}{2\pi d}$$
\end{itemize}

\subsection{7.4 Legge di Ampère}
Strumento fondamentale per calcolare $\vec{B}$ in presenza di simmetrie. La circuitazione del campo lungo una linea chiusa $\gamma$ è proporzionale alla corrente totale concatenata (che attraversa la superficie delimitata da $\gamma$).
\begin{itemize}
    \item \textbf{Forma Integrale:} $\oint_{\gamma} \vec{B} \cdot d\vec{l} = \mu_0 I_{conc}$
    \item \textbf{Forma Locale (Differenziale):} $\nabla \times \vec{B} = \mu_0 \vec{J}$
    \item \textbf{Significato fisico:} Il campo magnetico \textbf{non è conservativo} ($\nabla \times \vec{B} \neq 0$ dove c'è corrente). A differenza del campo elettrico, non è possibile definire un potenziale scalare $V$ per $\vec{B}$ in presenza di correnti.
\end{itemize}



\subsection{7.7 Legge di Gauss per il Magnetismo}
Esprime l'assenza di sorgenti puntiformi (monopoli) per il campo magnetico.
\begin{itemize}
    \item \textbf{Formula:} $\oint_{S} \vec{B} \cdot d\vec{A} = 0 \implies \nabla \cdot \vec{B} = 0$
    \item \textbf{Significato fisico:} Non esistono monopoli magnetici isolati (solo polo Nord o solo polo Sud). Le linee di $\vec{B}$ sono sempre circuiti chiusi o si estendono all'infinito; non hanno mai un punto di inizio o di fine.
\end{itemize}

\subsection{7.5, 7.6 e 7.8 Magnetismo nella Materia}
Nei materiali, la risposta magnetica è dovuta ai moti orbitali e allo spin degli elettroni (micro-correnti).
\begin{itemize}
    \item \textbf{Vettore Magnetizzazione ($\vec{M}$):} Momento di dipolo magnetico per unità di volume.
    \item \textbf{Vettore Intensità Magnetica ($\vec{H}$):} Legato alle sole correnti macroscopiche (libere):
    $$\vec{H} = \frac{\vec{B}}{\mu_0} - \vec{M}$$
    \item \textbf{Relazione fondamentale:} $\vec{B} = \mu_0 (\vec{H} + \vec{M}) = \mu \vec{H}$, dove $\mu = \mu_0 \mu_r$.
\end{itemize}

\textbf{Classificazione dei Materiali:}
\begin{itemize}
    \item \textbf{Diamagnetici ($\mu_r < 1$):} Sostanze che generano una magnetizzazione debole opposta al campo esterno (es. acqua, rame).
    \item \textbf{Paramagnetici ($\mu_r > 1$):} Sostanze i cui dipoli atomici si allineano debolmente al campo esterno (es. alluminio).
    \item \textbf{Ferromagnetici ($\mu_r \gg 1$):} Materiali che presentano una forte magnetizzazione spontanea e \textbf{Isteresi}. La relazione tra $B$ e $H$ non è lineare e dipende dalla storia magnetica del materiale.
\end{itemize}

\section{Capitolo 8: Campi Elettrici e Magnetici Variabili nel Tempo}

\subsection{8.1 e 8.2 Legge di Faraday e Legge di Lenz}
La scoperta di Faraday rompe la separazione tra elettricità e magnetismo: un campo magnetico variabile nel tempo genera un campo elettrico \textbf{indotto}.

\begin{itemize}
    \item \textbf{La Legge di Faraday:} La forza elettromotrice (f.e.m.) indotta in un circuito è pari alla variazione temporale del flusso magnetico $\Phi_B$ attraverso la superficie delimitata dal circuito:
    $$\mathcal{E} = - \frac{d\Phi_B}{dt}$$
    \item \textbf{Flusso Magnetico:} $\Phi_B = \int \vec{B} \cdot d\vec{A}$. La variazione può avvenire cambiando l'intensità di $\vec{B}$, l'area del circuito o l'orientamento tra i due.
    \item \textbf{Legge di Lenz (il segno meno):} La corrente indotta scorre sempre in un verso tale da creare un campo magnetico proprio che \textbf{si oppone} alla variazione di flusso che l'ha generata. 
    \item \textbf{Intuizione:} La natura è "conservativa" e tenta di mantenere il flusso magnetico costante. Se il flusso esterno aumenta, il circuito crea un campo opposto; se diminuisce, il circuito crea un campo concorde per sostenerlo.
\end{itemize}



\subsection{8.4 e 8.5 Autoinduzione ed Energia Magnetica}
Un circuito percorso da corrente variabile interagisce con se stesso poiché produce un campo magnetico che genera un flusso attraverso il circuito stesso.

\begin{itemize}
    \item \textbf{Induttanza ($L$):} Grandezza geometrica che misura la capacità del circuito di opporsi alle variazioni della propria corrente. Si misura in Henry ($[H] = [V \cdot s / A]$).
    $$\Phi_B = L I \implies \mathcal{E}_L = -L \frac{dI}{dt}$$
    \item \textbf{Energia Magnetica:} Per vincere la f.e.m. autoindotta e stabilire una corrente, il generatore deve compiere un lavoro che viene immagazzinato nel campo magnetico dell'induttore:
    $$U_m = \frac{1}{2} L I^2$$
    \item \textbf{Densità di energia magnetica ($u_m$):} L'energia contenuta nell'unità di volume del campo $\vec{B}$:
    $$u_m = \frac{B^2}{2\mu_0}$$
\end{itemize}



\subsection{8.7 La Legge di Ampère-Maxwell}
Maxwell completò la legge di Ampère per renderla coerente con i campi variabili (come nel caso della carica di un condensatore), introducendo la \textbf{corrente di spostamento}.
\begin{itemize}
    \item \textbf{Formula Integrale:} La circuitazione di $\vec{B}$ dipende sia dalle correnti di conduzione che dalla variazione del flusso elettrico:
    $$\oint \vec{B} \cdot d\vec{l} = \mu_0 I + \mu_0 \varepsilon_0 \frac{d\Phi_E}{dt}$$
    \item \textbf{Significato:} Esiste una simmetria fondamentale: così come un campo $\vec{B}$ variabile crea un campo $\vec{E}$ (Faraday), un campo $\vec{E}$ variabile crea un campo $\vec{B}$.
\end{itemize}



\subsection{8.8 e 8.9 Le Equazioni di Maxwell (Forma Differenziale)}
Rappresentano la sintesi definitiva dell'elettromagnetismo classico nel vuoto.

\begin{table}[h]
\centering
\begin{tabular}{|l|l|l|}
\hline
\textbf{Nome} & \textbf{Equazione} & \textbf{Significato Fisico} \\ \hline
Gauss (E) & $\nabla \cdot \vec{E} = \frac{\rho}{\varepsilon_0}$ & Le cariche sono sorgenti del campo elettrico. \\ \hline
Gauss (B) & $\nabla \cdot \vec{B} = 0$ & Non esistono monopoli magnetici isolati. \\ \hline
Faraday & $\nabla \times \vec{E} = -\frac{\partial \vec{B}}{\partial t}$ & Un campo $\vec{B}$ variabile genera un campo $\vec{E}$ rotazionale. \\ \hline
Ampère-Maxwell & $\nabla \times \vec{B} = \mu_0 \vec{J} + \mu_0 \varepsilon_0 \frac{\partial \vec{E}}{\partial t}$ & Correnti e campi $\vec{E}$ variabili generano campi $\vec{B}$. \\ \hline
\end{tabular}
\end{table}



\subsection{Le Onde Elettromagnetiche}
Dalla combinazione delle equazioni di Maxwell (applicando il rotore alle equazioni rotazionali) si ottiene un'equazione delle onde per $\vec{E}$ e $\vec{B}$. 
\begin{itemize}
    \item \textbf{Velocità di propagazione:} Maxwell dimostrò che i campi si propagano nello spazio come onde alla velocità:
    $$v = \frac{1}{\sqrt{\varepsilon_0 \mu_0}} \approx 3 \times 10^8 \text{ m/s}$$
    \item \textbf{Conclusione:} Poiché tale valore coincideva con la velocità della luce misurata sperimentalmente, Maxwell dedusse che \textbf{la luce è un'onda elettromagnetica}.
\end{itemize}

\section{Capitolo 9: Oscillazioni Elettriche e Correnti Alternate}

\subsection{9.1 Oscillazioni Elettriche (Il circuito LC)}
Un circuito composto da un condensatore $C$ inizialmente carico e un'induttanza $L$ costituisce un sistema oscillante, equivalente elettrico del sistema meccanico massa-molla.

\begin{itemize}
    \item \textbf{Analogia Meccanica:}
    \begin{itemize}
        \item Il \textbf{condensatore ($C$)} accumula energia potenziale elettrica (molla): $U_E = \frac{q^2}{2C}$.
        \item L'\textbf{induttanza ($L$)} accumula energia cinetica magnetica tramite la corrente (massa in moto): $U_M = \frac{1}{2}LI^2$.
    \end{itemize}
    \item \textbf{L'equazione differenziale:} Applicando la legge delle maglie ($V_L + V_C = 0$):
    $$L\frac{d^2q}{dt^2} + \frac{1}{C}q = 0$$
    Questa è l'equazione di un oscillatore armonico semplice.
    \item \textbf{Pulsazione propria ($\omega_0$):} La frequenza naturale alla quale il sistema scambia energia tra il campo elettrico e quello magnetico è:
    $$\omega_0 = \frac{1}{\sqrt{LC}}$$
\end{itemize}



\subsection{9.2 Circuiti in Corrente Alternata (AC)}
In regime sinusoidale, le grandezze elettriche variano nel tempo come: $V(t) = V_0 \cos(\omega t + \phi)$.
\begin{itemize}
    \item \textbf{Fasori:} Rappresentazione dei segnali sinusoidali come vettori rotanti nel piano complesso. La lunghezza del vettore è l'ampiezza ($V_0$), l'angolo rispetto all'asse reale è la fase. Permettono di sommare tensioni e correnti come vettori, semplificando i calcoli trigonometrici.
    \item \textbf{Reattanza ($X$):} L'opposizione al passaggio della corrente offerta da induttori e condensatori, che dipende dalla frequenza $\omega$.
    \begin{itemize}
        \item \textbf{Induttore ($L$):} La tensione \textbf{anticipa} la corrente di $90^\circ$ ($\pi/2$). Reattanza induttiva: $X_L = \omega L$.
        \item \textbf{Condensatore ($C$):} La tensione \textbf{ritarda} rispetto alla corrente di $90^\circ$ ($-\pi/2$). Reattanza capacitiva: $X_C = \frac{1}{\omega C}$.
    \end{itemize}
\end{itemize}



\subsection{9.3 Il Circuito RLC in Serie e la Risonanza}
Quando $R, L$ e $C$ sono in serie, l'opposizione totale al passaggio della corrente è definita dall'\textbf{Impedenza ($Z$)}:
$$Z = \sqrt{R^2 + (X_L - X_C)^2} = \sqrt{R^2 + \left(\omega L - \frac{1}{\omega C}\right)^2}$$
\begin{itemize}
    \item \textbf{Legge di Ohm Generalizzata:} $V_0 = Z I_0$, dove $V_0$ e $I_0$ sono i valori di picco (o efficaci).
    \item \textbf{Risonanza:} Si verifica quando la reattanza induttiva e capacitiva si annullano a vicenda ($X_L = X_C$).
    \begin{itemize}
        \item Condizione: $\omega = \omega_0 = \frac{1}{\sqrt{LC}}$.
        \item All'equilibrio di risonanza: $Z = R$ (impedenza minima).
        \item La corrente $I$ nel circuito raggiunge il suo valore \textbf{massimo}.
    \end{itemize}
\end{itemize}



\subsection{9.4 Potenza nei Circuiti AC}
Poiché tensione e corrente oscillano, è necessario definire valori medi e parametri di efficacia.
\begin{itemize}
    \item \textbf{Valori Efficaci (rms):} Rappresentano il valore di una corrente continua che dissiperebbe la stessa potenza. $V_{rms} = \frac{V_0}{\sqrt{2}}$, $I_{rms} = \frac{I_0}{\sqrt{2}}$.
    \item \textbf{Potenza Media (Attiva):} La potenza effettivamente dissipata (solo dalla resistenza):
    $$P = V_{rms} I_{rms} \cos\phi$$
    \item \textbf{Fattore di Potenza ($\cos\phi$):} $\phi$ è lo sfasamento tra $V$ e $I$. 
    \begin{itemize}
        \item Se il circuito è puramente resistivo ($\phi = 0$): $\cos\phi = 1$, tutta l'energia viene consumata.
        \item Se il circuito è puramente reattivo ($\phi = \pm 90^\circ$): $\cos\phi = 0$, la potenza media è nulla; l'energia viene accumulata e restituita ciclicamente.
    \end{itemize}
\end{itemize}

\subsection{9.5 Il Trasformatore Ideale}
Dispositivo basato sull'induzione mutua tra due avvolgimenti accoppiati magneticamente su un nucleo di ferro.
\begin{itemize}
    \item \textbf{Rapporto di trasformazione:} Lega le tensioni al numero di spire $N_1$ (primario) e $N_2$ (secondario):
    $$\frac{V_1}{V_2} = \frac{N_1}{N_2}$$
    \item \textbf{Conservazione della potenza:} In un trasformatore ideale (senza perdite), la potenza in ingresso è uguale a quella in uscita ($P_1 = P_2$):
    $$V_1 I_1 = V_2 I_2$$
    Un aumento della tensione (step-up) comporta una diminuzione proporzionale della corrente.
\end{itemize}

\section{Capitolo 10: Onde Elettromagnetiche}

\subsection{10.1, 10.2 e 10.3 La Nascita dell'Onda (Onde Piane)}
Dalle equazioni di Maxwell nel vuoto (in assenza di sorgenti: $\rho = 0$, $\vec{J} = 0$), applicando l'operatore rotore alle leggi di Faraday e Ampère-Maxwell, si ottengono le \textbf{equazioni delle onde} per i campi elettrico e magnetico:

$$\nabla^2 \vec{E} = \mu_0 \varepsilon_0 \frac{\partial^2 \vec{E}}{\partial t^2} \quad , \quad \nabla^2 \vec{B} = \mu_0 \varepsilon_0 \frac{\partial^2 \vec{B}}{\partial t^2}$$

\textbf{Caratteristiche delle onde piane:}
L'onda piana è la soluzione più semplice, in cui i campi dipendono dal tempo $t$ e da una sola coordinata spaziale di propagazione (es. $z$).
\begin{itemize}
    \item \textbf{Natura trasversale:} I vettori $\vec{E}$ e $\vec{B}$ sono sempre perpendicolari tra loro e sono entrambi perpendicolari alla direzione di propagazione $\hat{k}$.
    \item \textbf{Rapporto costante:} In ogni istante e in ogni punto dello spazio, i moduli dei campi sono legati dalla velocità della luce $c$:
    $$E = cB \quad \text{con } c = \frac{1}{\sqrt{\varepsilon_0 \mu_0}} \approx 3 \times 10^8 \text{ m/s}$$
    \item \textbf{Sincronia:} I campi $\vec{E}$ e $\vec{B}$ sono \textbf{in fase}: raggiungono i valori massimi, minimi e nulli contemporaneamente.
\end{itemize}



\subsection{10.4 Energia e Vettore di Poynting ($\vec{S}$)}
L'onda elettromagnetica trasporta energia attraverso lo spazio. Il flusso di energia istantaneo (potenza per unità di superficie) è descritto dal \textbf{Vettore di Poynting}:

$$\vec{S} = \frac{1}{\mu_0} \vec{E} \times \vec{B}$$

\begin{itemize}
    \item \textbf{Direzione:} $\vec{S}$ punta sempre nella direzione di propagazione dell'onda.
    \item \textbf{Intensità ($I$):} Poiché i campi oscillano ad altissima frequenza, si definisce l'intensità come la media temporale del modulo di $\vec{S}$:
    $$I = \langle S \rangle = \frac{E_{max}^2}{2 \mu_0 c} = \frac{1}{2} c \varepsilon_0 E_{max}^2$$
    L'unità di misura è il Watt su metro quadro ($[W/m^2]$).
\end{itemize}

\subsection{10.5 Quantità di Moto e Pressione di Radiazione}
Le onde elettromagnetiche trasportano anche quantità di moto, esercitando una forza meccanica sulle superfici colpite.
\begin{itemize}
    \item \textbf{Quantità di moto ($p$):} Legata all'energia $U$ trasportata dalla relazione $p = U/c$.
    \item \textbf{Pressione di radiazione ($P_{rad}$):} Forza esercitata per unità di superficie:
    \begin{itemize}
        \item \textbf{Superficie totalmente assorbente:} $P_{rad} = \frac{I}{c}$
        \item \textbf{Superficie totalmente riflettente:} $P_{rad} = \frac{2I}{c}$ (il rimbalzo raddoppia la variazione di quantità di moto).
    \end{itemize}
\end{itemize}



\subsection{10.6 Polarizzazione}
La polarizzazione descrive la direzione di oscillazione del vettore campo elettrico $\vec{E}$ nel piano trasversale alla propagazione.
\begin{itemize}
    \item \textbf{Lineare:} $\vec{E}$ oscilla lungo una retta fissa.
    \item \textbf{Circolare/Ellittica:} Il vettore $\vec{E}$ ruota nel tempo descrivendo un cerchio o un'ellisse (risultato della sovrapposizione di due onde sfasate).
    \item \textbf{Legge di Malus:} Se un'onda polarizzata linearmente con intensità $I_0$ attraversa un filtro polarizzatore ruotato di un angolo $\theta$ rispetto alla direzione di polarizzazione dell'onda, l'intensità trasmessa è:
    $$I = I_0 \cos^2 \theta$$
\end{itemize}



\subsection{10.7 Radiazione di Dipolo}
Le onde elettromagnetiche sono generate esclusivamente da \textbf{cariche accelerate}. Una carica ferma o in moto rettilineo uniforme non irradia.
\begin{itemize}
    \item \textbf{Dipolo oscillante:} Una carica che oscilla armonicamente lungo un asse.
    \item \textbf{Distribuzione spaziale:} La potenza irradiata non è uniforme. È massima nel piano perpendicolare all'asse del dipolo e nulla lungo l'asse stesso.
    \item \textbf{Dipendenza dalla frequenza:} La potenza totale emessa è proporzionale alla quarta potenza della frequenza ($P \propto \omega^4$). Per questo motivo, le alte frequenze (come l'azzurro del cielo) vengono diffuse molto più delle basse.
\end{itemize}



\subsection{10.8 Lo Spettro Elettromagnetico}
Le onde elettromagnetiche sono classificate in base alla frequenza $f$ (o lunghezza d'onda $\lambda = c/f$):

\begin{table}[h]
\centering
\begin{tabular}{|l|l|l|}
\hline
\textbf{Tipo di Onda} & \textbf{Frequenza (Hz)} & \textbf{Utilizzo / Caratteristiche} \\ \hline
Onde Radio & $< 10^9$ & Radio, TV, comunicazioni a lunga distanza. \\ \hline
Microonde & $10^9 - 10^{12}$ & Wi-Fi, telefonia mobile, forni a microonde. \\ \hline
Infrarossi & $10^{12} - 4 \times 10^{14}$ & Radiazione termica, sensori di calore. \\ \hline
Luce Visibile & $4 \times 10^{14} - 8 \times 10^{14}$ & Spettro percepibile (dal Rosso al Violetto). \\ \hline
Ultravioletti & $10^{15} - 10^{17}$ & Effetti biologici, lampade germicide. \\ \hline
Raggi X & $10^{17} - 10^{20}$ & Diagnostica medica (radiografie). \\ \hline
Raggi Gamma & $> 10^{20}$ & Fisica nucleare, altissima energia. \\ \hline
\end{tabular}
\end{table}

\pagebreak

\section{Altro}

\subsection{Strategie esercizi}

\begin{itemize}
    \item \textbf{Scelta del Metodo per il Calcolo dei Campi:}
    \begin{itemize}
        \item \textbf{Gauss vs Coulomb:} Usa Gauss solo se il problema presenta simmetrie evidenti (sfere, cilindri infiniti, piani infiniti). Se la distribuzione è finita (un segmento, un disco, un anello), Gauss non serve: devi integrare il contributo infinitesimo $dq$ usando Coulomb.
        \item \textbf{Ampère vs Biot-Savart:} Stessa logica. Ampère funziona solo per fili infiniti, solenoidi o toroidi. Per una spira o un arco di cerchio, usa Biot-Savart.
    \end{itemize}

    \item \textbf{Logica dei Condensatori e Dielettrici (Il "Tranello" della Batteria):}
    \begin{itemize}
        \item \textbf{Batteria collegata:} La differenza di potenziale $\Delta V$ è \textbf{fissa}. Se inserisci un dielettrico, $C$ aumenta e di conseguenza $Q$ aumenta ($Q=CV$).
        \item \textbf{Batteria scollegata:} La carica $Q$ è \textbf{fissa} (non può scappare). Se inserisci un dielettrico, $C$ aumenta e di conseguenza $\Delta V$ deve diminuire ($\Delta V = Q/C$). Il campo elettrico $E$ diminuisce perché le cariche di polarizzazione creano un campo opposto.
    \end{itemize}

    \item \textbf{Comportamento Limite dei Circuiti (Transitori RC/RL):}
    Non serve risolvere sempre l'equazione differenziale se il problema chiede lo stato iniziale o finale:
    \begin{itemize}
        \item \textbf{All'istante $t=0$ (Chiusura):} Il condensatore $C$ si comporta come un \textbf{cortocircuito} (filo); l'induttanza $L$ si comporta come un \textbf{interruttore aperto} (corrente nulla).
        \item \textbf{A regime ($t \to \infty$):} Il condensatore $C$ si comporta come un \textbf{interruttore aperto}; l'induttanza $L$ si comporta come un \textbf{filo} (resistenza nulla).
    \end{itemize}

    \item \textbf{Check di Sanità Mentale (Segni e Direzioni):}
    \begin{itemize}
        \item \textbf{Lenz:} Prima di calcolare il segno della f.e.m. con Faraday, usa la mano destra. Chiediti: "Il flusso sta aumentando? Allora la corrente indotta deve creare un campo che si oppone all'aumento". Se il risultato matematico non coincide con questa intuizione, hai sbagliato un segno nell'integrale.
        \item \textbf{Prodotto Vettoriale:} In $\vec{F} = q\vec{v} \times \vec{B}$ e $\vec{S} = \frac{1}{\mu_0} \vec{E} \times \vec{B}$, ricorda che se i vettori sono paralleli, il risultato è \textbf{zero}. Se sono perpendicolari, il modulo è il prodotto dei moduli.
    \end{itemize}

    \item \textbf{Analisi delle Onde EM:}
    Ricorda la "Regola del Terzo Vettore": se conosci la direzione di $\vec{E}$ (es. asse $x$) e la direzione di propagazione $\hat{k}$ (es. asse $z$), la direzione di $\vec{B}$ è obbligata dal prodotto vettoriale (deve essere asse $y$ per far sì che $\vec{E} \times \vec{B}$ punti verso $z$). Inoltre, $E$ è sempre $10^8$ volte più grande di $B$ in modulo (per via di $c$).

    \item \textbf{Circuiti AC (Frequenze):}
    \begin{itemize}
        \item \textbf{Basse frequenze ($\omega \to 0$):} Domina il condensatore ($X_C \to \infty$), il circuito tende a diventare un ramo aperto.
        \item \textbf{Alte frequenze ($\omega \to \infty$):} Domina l'induttore ($X_L \to \infty$), l'induttanza blocca il segnale.
        \item \textbf{Risonanza:} In questo punto esatto $L$ e $C$ si "annullano" a vicenda e il circuito vede solo la resistenza $R$.
    \end{itemize}
\end{itemize}

\subsection{Stokes}

Il \textbf{Teorema di Stokes Generalizzato} è l'unificazione moderna del calcolo integrale:
$$\int_{\Omega} d\omega = \int_{\partial \Omega} \omega$$
Dove $\omega$ è una $k$-forma differenziale e $d\omega$ è la sua derivata esterna (una $(k+1)$-forma). Vediamo come si riduce ai due casi fondamentali dell'elettromagnetismo in $\realnumbers^3$.

Il caso $k=1$ si applica a una varietà $\Omega$ di dimensione 2 (una superficie $S$) il cui bordo $\partial \Omega$ è una linea chiusa $\gamma$.

\begin{itemize}
    \item \textbf{La 1-forma $\omega$:} Rappresenta il potenziale di linea del campo vettoriale $\vec{V}$:
    $$\omega = V_x dx + V_y dy + V_z dz \implies \int_{\gamma} \omega = \oint_{\gamma} \vec{V} \cdot d\vec{l}$$
    \item \textbf{La 2-forma $d\omega$:} La derivata esterna di una 1-forma in $\realnumbers^3$ corrisponde al rotore del campo:
    $$d\omega = (\nabla \times \vec{V})_x dy \wedge dz + (\nabla \times \vec{V})_y dz \wedge dx + (\nabla \times \vec{V})_z dx \wedge dy$$
    \item \textbf{Risultato:} L'uguaglianza $\int_{S} d\omega = \int_{\gamma} \omega$ diventa:
    $$\int_{S} (\nabla \times \vec{V}) \cdot d\vec{A} = \oint_{\gamma} \vec{V} \cdot d\vec{l}$$
\end{itemize}

Il caso $k=2$ si applica a una varietà $\Omega$ di dimensione 3 (un volume $V$) il cui bordo $\partial \Omega$ è una superficie chiusa $S$.

\begin{itemize}
    \item \textbf{La 2-forma $\omega$:} Rappresenta il flusso del campo vettoriale $\vec{V}$ attraverso una superficie:
    $$\omega = V_x dy \wedge dz + V_y dz \wedge dx + V_z dx \wedge dy \implies \int_{S} \omega = \oint_{S} \vec{V} \cdot d\vec{A}$$
    \item \textbf{La 3-forma $d\omega$:} La derivata esterna di una 2-forma in $\realnumbers^3$ corrisponde alla divergenza del campo moltiplicata per l'elemento di volume:
    $$d\omega = (\nabla \cdot \vec{V}) dx \wedge dy \wedge dz \implies \int_{V} d\omega = \int_{V} (\nabla \cdot \vec{V}) \, dV$$
    \item \textbf{Risultato:} L'uguaglianza $\int_{V} d\omega = \int_{S} \omega$ diventa:
    $$\int_{V} (\nabla \cdot \vec{V}) \, dV = \oint_{S} \vec{V} \cdot d\vec{A}$$
\end{itemize}


\textbf{Teorema della Divergenza (Gauss):}
Mette in relazione il flusso di un campo vettoriale $\vec{V}$ attraverso una superficie chiusa $S$ con l'integrale della divergenza del campo esteso al volume $V$ racchiuso dalla superficie ($S = \partial V$):
$$\oint_{S} \vec{V} \cdot d\vec{A} = \int_{V} (\nabla \cdot \vec{V}) \, dV$$
\begin{itemize}
    \item \textbf{Significato fisico:} Il flusso netto uscente da una superficie è uguale alla somma di tutte le sorgenti (divergenza positiva) e dei pozzi (divergenza negativa) presenti all'interno del volume.
    \item \textbf{Uso in EM:} Permette di dimostrare la prima equazione di Maxwell partendo dalla legge di Gauss per il campo elettrico: $\oint \vec{E} \cdot d\vec{A} = Q/\varepsilon_0 \implies \nabla \cdot \vec{E} = \rho/\varepsilon_0$.
\end{itemize}



\textbf{Tabella Riassuntiva Operatori:}
\begin{table}[h]
\centering
\begin{tabular}{|l|l|l|}
\hline
\textbf{Operatore} & \textbf{Simbolo} & \textbf{Significato Intuitive} \\ \hline
\textbf{Gradiente} & $\nabla V$ & Indica la direzione di massima salita del potenziale. \\ \hline
\textbf{Divergenza} & $\nabla \cdot \vec{E}$ & Misura se un punto "crea" o "distrugge" linee di campo. \\ \hline
\textbf{Rotore} & $\nabla \times \vec{E}$ & Misura la tendenza del campo a "ruotare" attorno a un punto. \\ \hline
\textbf{Laplaciano} & $\nabla^2 V$ & Misura la concavità o la differenza tra il valore puntuale e la media locale. \\ \hline
\end{tabular}
\end{table}

\pagebreak

\section{Esame gennaio 2025}

\begin{snippetexercise}{1. Induzione Elettromagnetica}
    Una bobina composta da $N=4$ spire ha una superficie di $A=200~\mathrm{cm}^{2}$ ($0,02~\mathrm{m}^2$). Essa è immersa in un campo magnetico uniforme perpendicolare alla superficie. Il campo varia da $25~\mathrm{mT}$ a $10~\mathrm{mT}$ in un intervallo di tempo $\Delta t=5~\mathrm{s}$. Sapendo che la resistenza della bobina è $R=5,0~\Omega$, calcolare l'intensità della corrente indotta.
\end{snippetexercise}

\begin{snippetsolution}{1. Induzione Elettromagnetica}
    Si applica la legge di Faraday-Lenz unita alla legge di Ohm:
    \[
    I = \frac{fem}{R} = \frac{1}{R} \cdot N \cdot \frac{\Delta\Phi}{\Delta t}
    \]
    Considerando la variazione del campo $\Delta B = 15 \cdot 10^{-3}~\mathrm{T}$:
    \[
    I = \frac{4 \cdot 0,02 \cdot 15 \cdot 10^{-3}}{5 \cdot 5} = 0,000048~\mathrm{A} = 48~\mathrm{mA}
    \]
\end{snippetsolution}

\begin{snippetexercise}{2. Direzione della Forza di Lorentz}
    Una carica positiva si muove con velocità lungo l'asse x positivo in presenza di un campo magnetico $B$ diretto lungo l'asse z positivo. Determinare direzione e verso della forza agente sulla carica.
\end{snippetexercise}

\begin{snippetsolution}{2. Direzione della Forza di Lorentz}
    Utilizzando la regola della mano destra per il prodotto vettoriale $\vec{F} = q\vec{v} \times \vec{B}$:
    \begin{itemize}
        \item Vettore velocità $\vec{v}$ verso $+x$.
        \item Vettore campo $\vec{B}$ verso $+z$.
        \item La forza risultante $\vec{F}$ è diretta lungo l'asse $y$ negativo.
    \end{itemize}
\end{snippetsolution}

\begin{snippetexercise}{3. Sovrapposizione di Forze Elettrostatiche}
    Tre cariche sono disposte su una retta: $Q=30\cdot10^{-6}~\mathrm{C}$, $q=5\cdot10^{-6}~\mathrm{C}$ e una carica $-2Q$ (dal calcolo). La carica $q$ è posta centralmente a distanza $d=0,30~\mathrm{m}$ dalle altre due. Calcolare la forza totale agente su $q$.
\end{snippetexercise}

\begin{snippetsolution}{3. Sovrapposizione di Forze Elettrostatiche}
    La forza totale è la somma vettoriale delle forze esercitate da $Q$ e da $-2Q$:
    \[
    F_{tot} = \frac{1}{4\pi\epsilon_{0}}\cdot\frac{q\cdot Q}{d^{2}} + \frac{1}{4\pi\epsilon_{0}}\cdot\frac{q\cdot 2Q}{d^{2}}
    \]
    Inserendo i valori (con $k=9\cdot10^{9}~\mathrm{Nm^{2}/C^{2}}$):
    \[
    F_{tot} = 9\cdot10^{9} \cdot 5\cdot10^{-6} \cdot \left(\frac{30\cdot10^{-6}}{0,09} + \frac{60\cdot10^{-6}}{0,09}\right) = 7,5~\mathrm{N}
    \]
\end{snippetsolution}

\begin{snippetexercise}{4. Energia di un Induttore}
    Calcolare l'energia immagazzinata in un induttore di induttanza $L=0,5\cdot10^{-3}~\mathrm{H}$ quando è attraversato da una corrente $i=4,0~\mathrm{A}$.
\end{snippetexercise}

\begin{snippetsolution}{4. Energia di un Induttore}
    L'energia $U$ è data dalla formula:
    \[
    U = \frac{1}{2}Li^{2} = 0,5 \cdot 0,5\cdot10^{-3} \cdot 4^{2} = 4\cdot10^{-3}~\mathrm{J} \quad (4~\mathrm{mJ})
    \]
\end{snippetsolution}

\begin{snippetexercise}{5. Condensatori in Serie}
    Due condensatori $C_{1}=15~\mu\mathrm{F}$ e $C_{2}=30~\mu\mathrm{F}$ sono collegati in serie a un generatore da $\Delta V=50~\mathrm{V}$. Calcolare la carica presente su $C_{2}$.
\end{snippetexercise}

\begin{snippetsolution}{5. Condensatori in Serie}
    In serie, la carica $q$ è identica su entrambi i condensatori e pari alla carica della capacità equivalente:
    \[
    1/C_{eq} = 1/15 + 1/30 \Rightarrow C_{eq} = 10~\mu\mathrm{F}
    \]
    \[
    q = C_{eq} \cdot \Delta V = 10\cdot10^{-6} \cdot 50 = 5\cdot10^{-4}~\mathrm{C}
    \]
\end{snippetsolution}

\begin{snippetexercise}{6. Flusso del Campo Elettrico (Legge di Gauss)}
    Una sfera ha una densità di carica superficiale $\sigma=4,0~\mathrm{nC/m^{2}}$ e raggio $r=0,02~\mathrm{m}$. Calcolare il flusso elettrico attraverso una superficie gaussiana sferica di raggio $0,04~\mathrm{m}$.
\end{snippetexercise}

\begin{snippetsolution}{6. Flusso del Campo Elettrico (Legge di Gauss)}
    Il flusso dipende solo dalla carica totale racchiusa:
    \[
    Q_{tot} = \sigma \cdot 4\pi r^{2} = 4\cdot10^{-9} \cdot 4\pi \cdot (0,02)^{2}
    \]
    \[
    \Phi = Q_{tot}/\epsilon_{0} \approx 2,27~\mathrm{Vm}
    \]
\end{snippetsolution}

\begin{snippetexercise}{7. Parametri delle Onde Elettromagnetiche}
    Data un'intensità della radiazione solare $I=1340~\mathrm{W/m}^{2}$, determinare l'ampiezza del campo elettrico $E_{0}$ e del campo magnetico $B_{0}$.
\end{snippetexercise}

\begin{snippetsolution}{7. Parametri delle Onde Elettromagnetiche}
    \[
    E_{0} = \sqrt{2I/(c\epsilon_{0})} \approx 1000~\mathrm{V/m}
    \]
    \[
    B_{0} = E_{0}/c \approx 3,33\cdot10^{-6}~\mathrm{T}
    \]
\end{snippetsolution}

\begin{snippetexercise}{8. Campo Magnetico di Fili Infiniti}
    Due fili paralleli distanti $d=5~\mathrm{mm}$ portano correnti opposte $i=60~\mathrm{A}$. Calcolare il campo magnetico $B$ nel punto interno situato a $r_{1}=2~\mathrm{mm}$ dal primo filo.
\end{snippetexercise}

\begin{snippetsolution}{8. Campo Magnetico di Fili Infiniti}
    Poiché le correnti sono opposte, i campi tra i fili si sommano. Con $r_{1}=0,002~\mathrm{m}$ e $r_{2}=0,003~\mathrm{m}$:
    \[
    B_{tot} = \frac{\mu_{0} \cdot i}{2\pi} \cdot \left(\frac{1}{r_{1}} + \frac{1}{r_{2}}\right)
    \]
    \[
    B_{tot} = 2\cdot10^{-7} \cdot 60 \cdot (500 + 333,3) \approx 2\cdot10^{-2}~\mathrm{T} \quad (20~\mathrm{mT})
    \]
\end{snippetsolution}

\begin{snippetexercise}{9. Differenza di Potenziale Elettrostatico}
    Calcolare la differenza di potenziale tra due punti A e B generata da un sistema di due cariche puntiformi $q$ e $Q$.
\end{snippetexercise}

\begin{snippetsolution}{9. Differenza di Potenziale Elettrostatico}
    Si calcola il potenziale in ogni punto come somma dei potenziali delle singole cariche ($V=k\sum q_{i}/r_{i}$). Seguendo i passaggi numerici del foglio:
    \[
    V_{A} - V_{B} = +60~\mathrm{V}
    \]
\end{snippetsolution}

\begin{snippetexercise}{10. Resistenza in un Circuito Serie}
    Un circuito è alimentato da una $f.e.m.=20~\mathrm{V}$. Quando due resistenze $R_{1}$ e $R_{2}$ sono in serie, circola una corrente $i=1,0~\mathrm{A}$. Sapendo che $R_{2}=15~\Omega$, determinare $R_{1}$.
\end{snippetexercise}

\begin{snippetsolution}{10. Resistenza in un Circuito Serie}
    Dalla legge di Ohm per il circuito serie:
    \[
    R_{tot} = R_{1} + R_{2} = \frac{V}{i}
    \]
    \[
    R_{1} + 15 = \frac{20}{1,0} = 20~\Omega
    \]
    \[
    R_{1} = 20 - 15 = 5~\Omega
    \]
\end{snippetsolution}

\section{Esame febbraio 2025}

\begin{snippetexercise}{Esercizio 1: Circuito di Condensatori}
    Dato un circuito alimentato da una tensione $V_{0}=18~\mathrm{V}$ composto da tre condensatori $C_{1}=20\mu \mathrm{F}$, $C_{2}=10\mu \mathrm{F}$ e $C_{3}=30\mu \mathrm{F}$, determinare la capacità equivalente e la carica $q_{1}$ sul primo condensatore.
\end{snippetexercise}

\begin{snippetsolution}{Esercizio 1: Circuito di Condensatori}
    \textbf{Capacità in parallelo:} $C_{2}$ e $C_{3}$ sono in parallelo:
    \[
    C_{23} = C_{2} + C_{3} = 10\mu \mathrm{F} + 30\mu \mathrm{F} = 40\mu \mathrm{F}
    \]
    \textbf{Capacità equivalente:} $C_{1}$ è in serie con $C_{23}$:
    \[
    \frac{1}{C_{eq}} = \frac{1}{C_{1}} + \frac{1}{C_{23}} = \frac{1}{20} + \frac{1}{40} = \frac{3}{40}\mu \mathrm{F}^{-1}
    \]
    \[
    \Rightarrow C_{eq} = \frac{40}{3}\mu \mathrm{F} \approx 13.33~\mu \mathrm{F}
    \]
    
    \textbf{Carica $q_{1}$:} In un circuito in serie, la carica su ogni componente è uguale alla carica totale fornita dal generatore:
    \[
    q_{1} = q_{tot} = C_{eq} \cdot V_{0} = \frac{40}{3} \cdot 18 = 240~\mu \mathrm{C}
    \]
\end{snippetsolution}

\begin{snippetexercise}{Esercizio 2: Accelerazione di una particella carica}
    Un protone ($m_{p}=1.67\cdot10^{-27}$ kg, $q=1.6\cdot10^{-19}\mathrm{C}$) parte da fermo e viene accelerato da una differenza di potenziale $\Delta V=4\cdot10^{3}$ V. Calcolare la velocità finale $v$.
\end{snippetexercise}

\begin{snippetsolution}{Esercizio 2: Accelerazione di una particella carica}
    Applicando il principio di conservazione dell'energia ($K=L$):
    \[
    \frac{1}{2}m_{p}v^{2} = q\Delta V \Rightarrow v = \sqrt{\frac{2q\Delta V}{m_{p}}}
    \]
    \[
    v = \sqrt{\frac{2\cdot1.6\cdot10^{-19}\cdot4000}{1.67\cdot10^{-27}}} \approx 8.76\cdot10^{5}~\mathrm{m/s}
    \]
\end{snippetsolution}

\begin{snippetexercise}{Esercizio 3: Onde Elettromagnetiche}
    In un'onda elettromagnetica piana nel vuoto, il campo elettrico massimo è $E_{max}=600~\mathrm{V/m}$. Calcolare l'ampiezza del campo magnetico $B_{max}$.
\end{snippetexercise}

\begin{snippetsolution}{Esercizio 3: Onde Elettromagnetiche}
    Dalla relazione fondamentale tra i moduli dei campi in un'onda piana ($E=cB$):
    \[
    B_{max} = \frac{E_{max}}{c} = \frac{600}{3\cdot10^{8}} = 2\cdot10^{-6}~\mathrm{T} \quad (\text{ovvero } 2~\mu \mathrm{T})
    \]
\end{snippetsolution}

\begin{snippetexercise}{Esercizio 4: Forza tra cariche puntiformi}
    Tre cariche sono allineate sull'asse x: $q_{1}=40\mu \mathrm{C}$ a $x_{1}=-20$ cm, $q_{2}=50\mu \mathrm{C}$ a $x_{2}=30$ cm, e $q_{3}=4\mu \mathrm{C}$ nell'origine ($x=0$). Calcolare la forza risultante su $q_{3}$.
\end{snippetexercise}

\begin{snippetsolution}{Esercizio 4: Forza tra cariche puntiformi}
    Assumendo che le cariche siano tutte positive, $q_{3}$ subisce una spinta verso destra da $q_{1}$ ($F_{13}$) e una spinta verso sinistra da $q_{2}$ ($F_{23}$):
    \[
    F_{tot} = |F_{13}| - |F_{23}| = \frac{q_{3}}{4\pi\epsilon_{0}}\left(\frac{q_{1}}{r_{1}^{2}} - \frac{q_{2}}{r_{2}^{2}}\right)
    \]
    \[
    F_{tot} = (9\cdot10^{9})\cdot(4\cdot10^{-6})\cdot\left(\frac{40\cdot10^{-6}}{0.2^{2}} - \frac{50\cdot10^{-6}}{0.3^{2}}\right) \approx 16~\mathrm{N} \quad (\text{direzione } +x)
    \]
\end{snippetsolution}

\begin{snippetexercise}{Esercizio 5: Legge di Faraday-Lenz}
    Una spira circolare è immersa in un campo magnetico uniforme $B=1.5~\mathrm{T}$ perpendicolare al piano della spira. Il raggio della spira cresce linearmente nel tempo secondo la legge $r(t) = r_{0}+vt$ con $r_{0}=0.12~\mathrm{m}$ e $v=0.03~\mathrm{m/s}$. Calcolare la f.e.m. indotta istantanea.
\end{snippetexercise}

\begin{snippetsolution}{Esercizio 5: Legge di Faraday-Lenz}
    Il flusso del campo magnetico è $\Phi(B) = B \cdot A = B \cdot \pi r^{2}$.
    \[
    E = -\frac{d\Phi}{dt} = -B\pi\frac{d}{dt}(r^{2}) = -B\pi\left(2r\cdot\frac{dr}{dt}\right)
    \]
    Sostituendo i valori all'istante iniziale:
    \[
    E = -1.5 \cdot \pi \cdot 2 \cdot 0.12 \cdot 0.03 \approx -33.9~\mathrm{mV}
    \]
\end{snippetsolution}

\begin{snippetexercise}{Esercizio 6: Effetto Joule}
    Calcolare l'energia termica dissipata in un tempo $\Delta t=120$ s da un resistore $R=150~\Omega$ collegato a una tensione $\Delta V=20~\mathrm{V}$.
\end{snippetexercise}

\begin{snippetsolution}{Esercizio 6: Effetto Joule}
    \[
    E = P \cdot \Delta t = \frac{\Delta V^{2}}{R} \cdot \Delta t = \frac{20^{2}}{150} \cdot 120 = 320~\mathrm{J}
    \]
\end{snippetsolution}

\begin{snippetexercise}{Esercizio 7: Circuito RLC in regime sinusoidale}
    Un circuito serie RLC ha $R=100~\Omega$, $L=1~\mathrm{H}$, $C=2\mu \mathrm{F}$. Il generatore fornisce $V(t) = 100 \sin(500t)$. Calcolare la corrente efficace $I_{eff}$.
\end{snippetexercise}

\begin{snippetsolution}{Esercizio 7: Circuito RLC in regime sinusoidale}
    \begin{enumerate}
        \item \textbf{Pulsazione:} $\omega = 500~\mathrm{rad/s}$
        \item \textbf{Reattanze:}
        \[
        X_{L} = \omega L = 500 \cdot 1 = 500~\Omega
        \]
        \[
        X_{C} = \frac{1}{\omega C} = \frac{1}{500 \cdot 2 \cdot 10^{-6}} = 1000~\Omega
        \]
        \item \textbf{Impedenza:}
        \[
        Z = \sqrt{R^{2} + (X_{L} - X_{C})^{2}} = \sqrt{100^{2} + (500 - 1000)^{2}} \approx 509.9~\Omega
        \]
        \item \textbf{Corrente efficace:}
        \[
        V_{eff} = \frac{V_{max}}{\sqrt{2}} = \frac{100}{\sqrt{2}} \approx 70.71~\mathrm{V}
        \]
        \[
        I_{eff} = \frac{V_{eff}}{Z} = \frac{70.71}{509.9} \approx 0.139~\mathrm{A}
        \]
    \end{enumerate}
\end{snippetsolution}

\section{Esame gennaio 2026}

\begin{snippetexercise}{}
    Una particella carica si muove con una traiettoria circolare
    in un piano perpendicolare ad un campo magnetico. Quale frase rispecchia meglio il comportamento?
    \begin{enumerate}[label=\Alph*]
        \item Il campo non compie lavoro sulla particella
        \item Il lavoro effettuato sulla particella è negativo
        \item Il lavoro effettuato sulla particelle è costante
        \item Illavoro effettuato sulle particelle diminuisce
        \item Illavoro effettuato sulle particelle aumenta
    \end{enumerate}
\end{snippetexercise}

\begin{snippetsolution}{}
    \textbf{Risposta: A} \\
    La forza magnetica (forza di Lorentz) è data da \(\mathbf{F} = q\mathbf{v} \times \mathbf{B}\). Per definizione del prodotto vettoriale, la forza è sempre perpendicolare alla velocità \(\mathbf{v}\) (e quindi allo spostamento istantaneo). Poiché il lavoro è \(W = \int \mathbf{F} \cdot d\mathbf{l}\) e l'angolo tra forza e spostamento è \(90^\circ\), il lavoro compiuto dal campo magnetico è nullo.
\end{snippetsolution}

\begin{snippetexercise}{}
    Calcola la resistenza in \(\Omega\) di un filo con resistività \(3.2 \cdot 10^{-8}\)
    \(\Omega \text{m}\), con lunghezza \(2.5\) metri e diametro \(0.5 \text{mm}\).
\end{snippetexercise}

\begin{snippetsolution}{}
    Usiamo la seconda legge di Ohm: \(R = \rho \frac{L}{A}\). \\
    Convertiamo il diametro in raggio e in metri: \(r = \frac{d}{2} = 0.25 \, \text{mm} = 2.5 \cdot 10^{-4} \, \text{m}\). \\
    L'area della sezione è \(A = \pi r^2 = \pi (2.5 \cdot 10^{-4})^2 \approx 1.96 \cdot 10^{-7} \, \text{m}^2\). \\
    Calcolo:
    \[
    R = (3.2 \cdot 10^{-8}) \frac{2.5}{1.96 \cdot 10^{-7}} \approx \frac{8 \cdot 10^{-8}}{1.96 \cdot 10^{-7}} \approx 0.408 \, \Omega
    \]
\end{snippetsolution}

\begin{snippetexercise}{}
    Se il valore massimo della componente \(E\) di una onda elettromagnetica è \(600 \text{V/m}\), qual è il massimo della componente \(B\)?
\end{snippetexercise}

\begin{snippetsolution}{}
    In un'onda elettromagnetica nel vuoto, la relazione tra le ampiezze dei campi è \(E = cB\), dove \(c \approx 3 \cdot 10^8 \, \text{m/s}\).
    \[
    B = \frac{E}{c} = \frac{600}{3 \cdot 10^8} = 200 \cdot 10^{-8} = 2.0 \cdot 10^{-6} \, \text{T} = 2.0 \, \mu\text{T}
    \]
\end{snippetsolution}

\begin{snippetexercise}{}
    Su un quadrato di lato \(1.5\) metri sono poste due cariche, una di \(2 \text{nC}\)
    e una di \(3 \text{nC}\) su due vertici opposti. Qual è l'intensità del campo
    elettrico su uno dei due altri vertici?
\end{snippetexercise}

\begin{snippetsolution}{}
    Siano le cariche \(q_1 = 2\,\text{nC}\) e \(q_2 = 3\,\text{nC}\). Il vertice considerato è adiacente ad entrambe, quindi la distanza da ciascuna carica è \(r = 1.5\,\text{m}\). I campi elettrici generati sono perpendicolari tra loro.
    \[ E_1 = k \frac{q_1}{r^2} = (8.99 \cdot 10^9) \frac{2 \cdot 10^{-9}}{1.5^2} \approx 8.0 \, \text{N/C} \]
    \[ E_2 = k \frac{q_2}{r^2} = (8.99 \cdot 10^9) \frac{3 \cdot 10^{-9}}{1.5^2} \approx 12.0 \, \text{N/C} \]
    Il campo totale è la somma vettoriale (Pitagora):
    \[ E_{tot} = \sqrt{E_1^2 + E_2^2} = \sqrt{8^2 + 12^2} = \sqrt{64 + 144} \approx 14.4 \, \text{N/C} \]
\end{snippetsolution}

\begin{snippetexercise}{}
    Una sfera di volume \(12 \text{cm}^3\) viene riempita con un materiale non conduttore
    con carica uniforme distribuita \(30 \text{pC}\) sul volume.
    Qual è l'intensità del campo elettrico a \(1.0 \text{cm}\) dal centro?
\end{snippetexercise}

\begin{snippetsolution}{}
    Troviamo il raggio della sfera \(R\). \(V = \frac{4}{3}\pi R^3 \Rightarrow R = \sqrt[3]{\frac{3V}{4\pi}}\).
    Con \(V=12\), \(R \approx 1.42\,\text{cm}\). Poiché la distanza richiesta \(r=1.0\,\text{cm}\) è minore di \(R\), siamo all'interno della distribuzione.
    Il campo interno di una sfera isolante uniformemente carica è:
    \[ E = \frac{Q_{tot} r}{4\pi \epsilon_0 R^3} = \frac{\rho r}{3\epsilon_0} \]
    Più semplicemente, usando la proporzione di carica racchiusa \(Q_{enc} = Q_{tot} \frac{r^3}{R^3}\):
    \[ E = k \frac{Q_{enc}}{r^2} = k \frac{Q_{tot} r}{R^3} \]
    \[ E = (8.99 \cdot 10^9) \frac{30 \cdot 10^{-12} \cdot 0.01}{(0.0142)^3} \approx 941 \, \text{N/C} \]
\end{snippetsolution}

\begin{snippetexercise}{}
    Considera una spira rettangolare di \(0.2 \text{m}\) con campo magnetico uniforme perpendicolare
    al piano della spira. L'intensità del campo magnetico è \(B = 0.4 T \cdot e^{t/J}\) secondi e
    \(J = 4.0\). Calcola fem indotta a \(t = 2.0\) secondi.
\end{snippetexercise}

\begin{snippetsolution}{}
    Assumiamo che la spira sia quadrata con lato \(l=0.2\,\text{m}\), quindi Area \(A = 0.04\,\text{m}^2\).
    La legge di Faraday-Neumann-Lenz: \(\mathcal{E} = - \frac{d\Phi_B}{dt} = -A \frac{dB}{dt}\).
    Data \(B(t) = 0.4 e^{t/4}\), la derivata è \(\frac{dB}{dt} = 0.4 \cdot \frac{1}{4} e^{t/4} = 0.1 e^{t/4}\).
    A \(t=2.0\): \(\frac{dB}{dt} = 0.1 e^{0.5} \approx 0.165 \, \text{T/s}\).
    \[ |\mathcal{E}| = 0.04 \cdot 0.165 \approx 0.0066 \, \text{V} = 6.6 \, \text{mV} \]
\end{snippetsolution}

\begin{snippetexercise}{}
    Il potenziale elettrico all'interno di un conduttore sferico in pieno equilibrio
    \begin{enumerate}[label=\Alph*]
        \item ha la medesima carica attraverso la superficie per unità di tempo divisa per la resistività
        \item è costante e pari al suo valore sulla superficie
        \item decresce dalla superficie fino al centro con 0
        \item è nulla
        \item decresce dalla superficie fino al centro con un valore multiplo
    \end{enumerate}
\end{snippetexercise}

\begin{snippetsolution}{}
    \textbf{Risposta: B} \\
    All'interno di un conduttore in equilibrio elettrostatico il campo elettrico è nullo (\(E=0\)). Poiché \(E = -\nabla V\), se il campo è nullo, il gradiente del potenziale è nullo, il che significa che il potenziale \(V\) è costante in tutto il volume e uguale al valore che assume sulla superficie.
\end{snippetsolution}

\begin{snippetexercise}{}
    Considera un cavo di \(2\) metri sospeso parallelo ad un campo magnetico uniforme di \(0.5 T\),
    attraversato da una corrente di \(0.6 A\). Trova la forza in Newton applicata al cavo.
\end{snippetexercise}

\begin{snippetsolution}{}
    La forza su un filo percorso da corrente è \(F = I L B \sin(\theta)\).
    Il testo specifica che il cavo è \textbf{parallelo} al campo magnetico, quindi \(\theta = 0^\circ\) (o \(180^\circ\)).
    Poiché \(\sin(0) = 0\), la forza magnetica è:
    \[ F = 0 \, \text{N} \]
\end{snippetsolution}

\begin{snippetexercise}{}
    Se \(5 \times 10^{21}\) elettroni entrano in un resistore di \(20 \Omega\) in \(10 \text{min}\), qual è la differenza di
    potenziale in \(V\) lungo il resistore?
\end{snippetexercise}

\begin{snippetsolution}{}
    Calcoliamo prima la carica totale e poi la corrente.
    \(Q = N \cdot e = 5 \cdot 10^{21} \cdot 1.6 \cdot 10^{-19} = 800 \, \text{C}\).
    Il tempo in secondi è \(t = 10 \cdot 60 = 600 \, \text{s}\).
    Corrente \(I = \frac{Q}{t} = \frac{800}{600} = \frac{4}{3} \, \text{A} \approx 1.33 \, \text{A}\).
    Legge di Ohm:
    \[ V = R \cdot I = 20 \cdot \frac{4}{3} = \frac{80}{3} \approx 26.7 \, \text{V} \]
\end{snippetsolution}

\begin{snippetexercise}{}
    Se \(V_a - V_b = 22 \text V\), qual è l'energia immagazzinata nel condensatore di \(50 \mu F\) ?
    \begin{center}
        \begin{circuitikz}
            \draw (0,2) node[left] {a} 
                to[short, o-] (0.5, 2) % piccolo pezzo di filo iniziale
                to[C, l=$25\,\mu\text{F}$] (3,2) % Condensatore orizzontale
                -- (3,2); % Arrivo all'angolo
            \draw (3,2) 
                to[C, l=$50\,\mu\text{F}$] (3,0); 
            \draw (3,0) 
                to[C, l=$25\,\mu\text{F}$] (0.5, 0)
                to[short, -o] (0,0) node[left] {b}; % chiudo su B
        \end{circuitikz}
    \end{center}
\end{snippetexercise}

\begin{snippetsolution}{}
    I tre condensatori (\(25\mu\text{F}, 50\mu\text{F}, 25\mu\text{F}\)) sono in serie.
    Calcoliamo la capacità equivalente:
    \[ \frac{1}{C_{eq}} = \frac{1}{25} + \frac{1}{50} + \frac{1}{25} = \frac{2+1+2}{50} = \frac{5}{50} = \frac{1}{10} \]
    Quindi \(C_{eq} = 10 \, \mu\text{F}\).
    La carica totale fornita dal generatore è \(Q_{tot} = C_{eq} V = 10\mu\text{F} \cdot 22\text{V} = 220 \, \mu\text{C}\).
    In serie, la carica è identica su ogni condensatore: \(Q_{50} = 220 \, \mu\text{C}\).
    L'energia immagazzinata nel condensatore da \(50\mu\text{F}\) è:
    \[ U = \frac{Q^2}{2C} = \frac{(220 \cdot 10^{-6})^2}{2 \cdot 50 \cdot 10^{-6}} = \frac{48400 \cdot 10^{-12}}{100 \cdot 10^{-6}} = 484 \cdot 10^{-6} \, \text{J} = 484 \, \mu\text{J} \]
\end{snippetsolution}

\section{Esame febbraio 2026}

\begin{snippetexercise}{Esercizio 1}
    Un condensatore da \(15 \mu\mathrm{F}\) e un condensatore da \(25 \mu\mathrm{F}\)
    sono connessi in parallelo e a questa combinazione viene applicata una differenza di potenziale di 60 Volt.
    Quanta energia è immagazzinata in questa combinazione di condensatori?
\end{snippetexercise}

\begin{snippetsolution}{Esercizio 1}
\end{snippetsolution}

\begin{snippetexercise}{Esercizio 2}
    Un campo magnetico
\end{snippetexercise}

\begin{snippetsolution}{Esercizio 2}
\end{snippetsolution}

\begin{snippetexercise}{Esercizio 3}
\end{snippetexercise}

\begin{snippetsolution}{Esercizio 3}
\end{snippetsolution}

\begin{snippetexercise}{Esercizio 4}
\end{snippetexercise}

\begin{snippetsolution}{Esercizio 4}
\end{snippetsolution}

\begin{snippetexercise}{Esercizio 5}
\end{snippetexercise}

\begin{snippetsolution}{Esercizio 5}
\end{snippetsolution}

\begin{snippetexercise}{Esercizio 6}
\end{snippetexercise}

\begin{snippetsolution}{Esercizio 6}
\end{snippetsolution}

\begin{snippetexercise}{Esercizio 7}
\end{snippetexercise}

\begin{snippetsolution}{Esercizio 7}
\end{snippetsolution}

\begin{snippetexercise}{Esercizio 8}
\end{snippetexercise}

\begin{snippetsolution}{Esercizio 8}
\end{snippetsolution}

\begin{snippetexercise}{Esercizio 9}
\end{snippetexercise}

\begin{snippetsolution}{Esercizio 9}
\end{snippetsolution}

\begin{snippetexercise}{Esercizio 10}
\end{snippetexercise}

\begin{snippetsolution}{Esercizio 10}
\end{snippetsolution}

\end{document}
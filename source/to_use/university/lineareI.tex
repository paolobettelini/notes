\documentclass[a4paper]{article}

\usepackage{amsmath}
\usepackage{amssymb}
\usepackage{stellar}
\usepackage{parskip}
\usepackage{fullpage}
\usepackage{wrapfig}

\title{Algebra lineare I}
\author{Paolo Bettelini}
\date{}

\begin{document}

\maketitle
\tableofcontents

\section{Algebra lineare}

La moltiplicazione scalare di un vettore è un \emph{omotetia}.

\sdefinition{Spazio a prodotto interno}{
    Uno \emph{spazio a prodotto interno} è uno spazio vettoriale con la struttura
    aggiuntiva del dot product.
}

\sdefinition{Ortogonalità}{
    In uno spazio a prodotto interno, la nozione di \emph{ortogonalità}
    è definita dalla nullità del product.
}

\sdefinition{Spazio affine}{
    Uno \emph{spazio affine} è uno spazio vettoriale
    senza la nozione del punto di origine.
}

Il prodotto vettoriale in \(\mathbb{R}^3\) necessita di un orientamento.
Se consideriamo il prodotto vettoriale di \(\mathbb{R}^3\)
nei piano bidimensionale \(\mathbb{R}^2\), il prodotto ha forma
di complesso. Il prodotto scalare e il prodotto dei complessi
si uniscono nella struttura quaternionale.

I sistemi lineari possono essere interpretati come trovare l'intersezione delle varie rette,
oppure possiamo vederlo come trovare i coefficienti lineari tali che la combinazione lineare
(con tali coefficienti) dei vettori sia il vettore risultante.
Possiamo quindi anche vederlo cmoe trovare il vettore che moltiplicato dalla matrice del sistema
ci restituisce il vettore voluto.

\section{Algoritmo di eliminazione di Gauss}

\slemma{}{
    Dato un sistema di equazioni con \(m\) equazioni e \(n\) indeterminata,
    l'eliminazione di Gauss non modifica le soluzioni del sistema.
}

\sproof{}{
    Per la prima operazione, la dimostrazione in una direzione è banale.
    Per dimostrarla nell'altra direzione è sufficiente considerare le righe della matrice
    \(E_i' = E_i + \lambda E_j\) dove l'apice indica la riga modificata. Siccome la riga che viene aggiunta
    rimane invariata \(E_j = E_j'\) allora \(E_i = E_i' - \lambda E_j'\) e quindi
    i calcoli nella direzione inversa sono gli stessi in quanto sto sempre aggiungendo un multiplo di un'altra riga.
    Quindi le soluzioni prima e dopo l'operazione rimangono invariate. Se il sistema originario non ha soluzioni,
    non ne ha nemmeno quello nuovo.
}

Nell'eliminazionne di gauss vogliamo raggiungere la matrice triangolare superiore.
Distinguiamo i casi dove ci sono righe con tutti zero includendo o senza includere il termine noto.
In tali casi abbiamo infiniti o finite soluzioni.
Ne caso non è possibile raggiungere la row echelon form, bisogna studiare i termini noti.

L'operazione della moltiplicazione scalare di una matrice ne modifica il determinante.
Usando solo le altre due operazioni non è quindi facile giungere alla reduced row echelon form.

\end{document}
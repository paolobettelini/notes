\documentclass[a4paper]{article}

\usepackage{amsmath}
\usepackage{amssymb}
\usepackage{stellar}
\usepackage{parskip}
\usepackage{fullpage}
\usepackage{wrapfig}

\title{Algebra lineare I}
\author{Paolo Bettelini}
\date{}

\begin{document}

\maketitle
\tableofcontents

\section{Algebra lineare}

La moltiplicazione scalare di un vettore è un \emph{omotetia}.

\sdefinition{Spazio a prodotto interno}{
    Uno \emph{spazio a prodotto interno} è uno spazio vettoriale con la struttura
    aggiuntiva del dot product.
}

\sdefinition{Ortogonalità}{
    In uno spazio a prodotto interno, la nozione di \emph{ortogonalità}
    è definita dalla nullità del product.
}

\sdefinition{Spazio affine}{
    Uno \emph{spazio affine} è uno spazio vettoriale
    senza la nozione del punto di origine.
}

Il prodotto vettoriale in \(\mathbb{R}^3\) necessita di un orientamento.
Se consideriamo il prodotto vettoriale di \(\mathbb{R}^3\)
nei piano bidimensionale \(\mathbb{R}^2\), il prodotto ha forma
di complesso. Il prodotto scalare e il prodotto dei complessi
si uniscono nella struttura quaternionale.

I sistemi lineari possono essere interpretati come trovare l'intersezione delle varie rette,
oppure possiamo vederlo come trovare i coefficienti lineari tali che la combinazione lineare
(con tali coefficienti) dei vettori sia il vettore risultante.
Possiamo quindi anche vederlo cmoe trovare il vettore che moltiplicato dalla matrice del sistema
ci restituisce il vettore voluto.

\end{document}
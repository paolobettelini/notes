\documentclass[a4paper]{article}

\usepackage{amsmath}
\usepackage{amssymb}
\usepackage{stellar}
\usepackage{parskip}
\usepackage{fullpage}
\usepackage{wrapfig}
\usepackage{tikz}

\usetikzlibrary{arrows}
\usetikzlibrary{decorations.pathreplacing}
\usetikzlibrary{cd}

\title{Exercises}
\author{Paolo Bettelini}
\date{}

\begin{document}

\maketitle
\tableofcontents

\section{Esercizi}

\sexercise{}{
    Scrivere la relazione fra \(F_a(x)\) e \(F_b(x)\)
    per \(a \neq b\).
}


\sexercise{}{
    Let
    \[
        E = \left\{ x\in\mathbb{R} \,|\, \frac{1}{2} \leq x < 5 \right\}
    \]
    and the sequence \[
        F = \{ x = x_n \,|\, x_n = \frac{n + 1}{n + 2}, \quad n\in\mathbb{N}^* \}
    \]
    %%%%%%%
    Trova inf, sup, min, max (se esistono) di \(E\), \(F\), \(E \cup F\) e \(E\cap F\).
    \begin{itemize}
        \item \(E\) è limitato superiormente e inferiormente.
        Il minimo è \(\frac{1}{2}\), mentre \(5\) è un maggiorante, è il più piccolo
        dei maggioranti quindi \(\sup E = 5\), ma non vi è un massimo.
        \item \(F\) è limitato superiormente in quanto 
        \[
            x_n = \frac{n+1}{n+2} < \frac{n+2}{n+2}  =1
        \]
        È limitato inferiormente perché \(x_n > 0\).
        Per verificare sup e inf, è comodo riscrivere \[
            x_n = 1 - \frac{1}{n+2}
        \]
        Il temrine \(n+2\) cresce con \(n\), quindi \(\frac{1}{n+2}\) decresce al crescere di \(n\)
        e quindi \(x_n\) cresce approcciando \(1\). Allora con \(n=1\) il termine assume il valore più piccolo,
        ossia \(\frac{2}{3}\), quindi il minimo di \(F\). Allora siccome ci avviciamo arbitrariamente a
        \(1\), è lecito ipotizzare \(\sup F = 1\).
        Il massimo di \(F\) non esiste.
        Rimane da far vedere che se \(z < 1\) allora \(z\) non è maggiorante di \(F\)
        cioè
        \[
            x_n - z = (1-z) - \frac{1}{n+2} > 0
        \]
        purché \(\frac{1}{n+2} < 1-z\) cioè \(n > \frac{1}{1-z}-2\).
        Quindi \(z\) non è maggiorante e \(\sup E = 1\).
        \item Verificare che \(\sup (E \cup F) = \max \{ \sup E,\sup F \}\).
        Abbiamo che \(\sup E \leq \sup F\).
        In sup è il massimo dei due in quanto uno è maggiore dell'altro,
        e fa parte dell'insieme, quindi \(\sup E \cup F = 5\).
        Tuttavia, il max non esiste in quando \(5\notin  E \cup F\).
        Analogamente, \(\inf E \cup F = \frac{1}{2}\). Questo valore è anchde il minimo
        in quanto fa parte dell'insieme.
        \item Mostrare con un esempio che non c'è qualcosa di analogo per l'intersezione.
        \[
            E \cap F = \left\{ x_n = \frac{x+1}{x+2} \ \middle|\ \frac{1}{2} \leq \frac{x+1}{x+2} \leq 5 \right\}
        \]
        Quindi \(F \subseteq E\). Consideriamo allora \(E_1 = [\frac{4}{5}, 5)\)
        \[
            E_1 \cap F = \left\{ x_n = \frac{x+1}{x+2} \ \middle|\ \frac{4}{5} \leq x_n \leq 5 \right\}
        \]
        Per quali \(n\) vale che \(\frac{4}{5} \leq \frac{x+1}{x+2} = x_n\)?
        Abbiamo \(4(n+2) \leq 5(n+1)\) e quindi \(n \geq 3\).
        Allora \(\sup E_1 \cap F = 1\) e non vi è massimo, mentre \(\inf E_1 \cap F = \frac{4}{5}\)
        che è anche il minimo.
        \item Posto \(E+F = \{ x + y \,|\, x \in E, y \in F \}\)
        mostrare \(\sup E + F = \sup E + \sup F\). Supponiamo quindi che \(\sup E\) e \(\sup F\)
        siano finiti. Siccome, per definizione, \(\forall e \in E, e \leq \sup E\)
        e \(\forall f \in F, f \leq \sup F\), abbiamo che \[\forall e \in E, \forall f\in F, e + f \leq \sup E + \sup F\]
        Per mostrare che questo è il più piccolo dei maggioranti, è comodo riscrivere la definizione di
        sup dicendo che \(\mu\) è pari a \(\sup E\) se:
        \begin{enumerate}
            \item \(\forall x \in E, x\leq \mu\);
            \item \(\forall \varepsilon > 0, \mu - \varepsilon\) non è maggiorante.
        \end{enumerate}
        \textbf{Nota:} se \(x < \mu\) allora posto \(\varepsilon = \mu - x\) risulta \(x = \mu - \varepsilon\).
        Allora sia \(\varepsilon > 0\). Diciamo che esistono \(e_1\in E\) e \(f_1\in F\) tali che
        \(e_1 + f_1 > \sup E + \sup F - \varepsilon\).
        Poiché \(\sup E\) è, appunto, il supremum, esiste per definizione una \(e_1 \in E\) tale che
        \(e_1 > e_1 > \sup E . \frac{\varepsilon}{2}\).
        Analogamente, esiste \(f_1 \in F\) tale che \(f_1 > \sup F - \frac{\varepsilon}{2}\).
        Da cui \(e_1 + f_2 > \sup E - \frac{\varepsilon}{2} + \sup F - \frac{\varepsilon}{2} = \sup E + \sup F - \varepsilon\).
        \item Posto \(-E = \{ -x \,|\, x \in E\}\) mostrare che
        \(\sup -E = -\inf E\) e \(\inf -E = -\sup E\).
    \end{itemize}
}

Dimostrare che il max esiste se e solo se \(\sup E\) è finito e appartiene a \(E\).
Analogamente per il min.


\sexercise{}{
    Trovare sup, inf, min, max dell'insieme
    \[
        E = \left\{ x_n = \frac{n-7}{x^2 + 1} \ \middle|\ n \geq 1 \right\}
    \]
    Questa successione ha sicuramente un minimo in quanto ci sono solamente \(6\) numeri negativi.
    Possiamo notare che il denominatore cresce più velocemente del numeratore.

    Studiamo quindi per quali indici vale \(x_n \leq x_{n+1}\). Otteniamo quindi
    \begin{align*}
        \frac{n-7}{n^2 + 1} &\leq \frac{(n+1)-7}{{(n+1)}^2 + 1} \\
        \frac{(n-7)(n^2 + 2n + 2) - (n-6)(n^2+1)}{(n^2 + 1)(n^2 + 2n + 2)} &\leq 0
    \end{align*}
    Il denominatore è positivo, quindi studiamo il numeratore
    \begin{align*}
        n^2 - 13n - 8 \leq 0
    \end{align*}
    Le radici di questo polinomio sono \(n_{1,2}= \frac{13\pm\sqrt{201}}{2}\).
    Di conseguenza, l'espressione è negativa per \(\frac{13-\sqrt{201}}{2} < n < \frac{13+\sqrt{201}}{2}\).
    Notiamo che l'estremo di sinistra è negativo. Notiamo anche che \(14^2 < 201 < 15^2\),
    e quindi l'estremo di destra è compreso fra \(14\) e \(\frac{27}{2}\).
    Allora, tutte le \(n\) intere che soddisfano l'equazione sono
    \(n=13\). Ne consegue che se \(n \geq 14\), \(x_n > x_{n+1}\).
    Il maggiornate e supremum è quindi \(x_{14}\).

}


\sexercise{}{
    \[
        a_n = \frac{
            \log \left(\frac{n^2 + 1}{n}\right) + 1
        }{
            \sqrt{n^3 + 1} + \log n
        }
    \]
}

\sexercise{}{
    \[
        a_n = \frac{
            n^{1/2} + \cos(1/n) + \log n
        }{
            {(n + \sqrt{n})}^2 - \sqrt{n}
        }
    \]
}

\sexercise{}{
    \[
        a_n = \log \left(
            1 + \sin \left(\frac{\sqrt{n}}{n^2 + \log n}\right)
        \right)
        \left(\sqrt[3]{n^6 + 1} - n^2\right)
    \]
}

\sexercise{}{
    \[
        a_n = {\left(\cos \frac{1}{\sqrt{n}}\right)}^{\frac{n^3 - \log n}{\sqrt{n^4 + n}}}
    \]
}


\sexercise{}{
    Sia \(\{b_n\}\) una successione e sia \(\{b_{n\pm k_0}\}\) la successione traslata di
    \(\pm k\).
    Dimostrare che \(\lim b_n\) esiste se e solo se \(\lim b_{n \pm k_0}\) esiste e che i limiti sono uguali.
    % TODOURGENT
}

\sexample{}{
    Considera
    \[
        \sum_{n=1}^\infty \frac{1}{4n^2 - 1}
    \]
    Allora
    \[
        \frac{1}{4n^2 - 1} = \frac{1}{(2n+1)(2n-1)} = \frac{1/2}{2n-1} - \frac{1/2}{2n+1}
    \]
    Quindi
    \begin{align*}
        \frac{1}{2} \sum_{n=1}^\infty \left(\frac{1}{2n-1} - \frac{1}{2n+1}\right) \to
        \frac{1}{2}  
    \end{align*}
}

\sexercise{}{
    Calcolare
    \begin{align*}
        \sum_{n=1}^\infty \frac{3^n + 2\cdot 5^{n+1}}{7^{n+2}}
        &= \sum_{n=1}^\infty \frac{3^{n}}{7^{n+2}}
        + 2 \sum_{n=1}^\infty \frac{5^{n+1}}{7^{n+2}} \\
        &= \frac{1}{7^2} \sum_{n=1}^\infty \frac{3^{n}}{7^n}
        + \frac{2}{7} \sum_{n=1}^\infty \frac{5^{n+1}}{7^{n+1}} \\
        &= \frac{1}{7^2} \sum_{n=0}^\infty {\left(\frac{3}{7}\right)}^{n+1}
        +  \frac{2}{7} \sum_{n=0}^\infty {\left(\frac{3}{7}\right)}^{n+2} \\
        &= \cdots
    \end{align*}
}


\sexercise{}{
    Stabilire il carattere della serie
    \[
        \sum_{n=1}^\infty \frac{n^2 + {(1 + 1/n)}^n + \sin n}{{(n+\sqrt{n})}^3 + \log \left(\frac{n}{n+1}\right)}
    \]
    Notiamo che \(\forall n \geq 1, a_n \geq 0\).
    Notiamo allora che
    \[
        a_n = \frac{n^2 \left(1 + \frac{1}{n^2}{\left(1 + \frac{1}{n}\right)}^n + \frac{\sin n}{n}\right)}
        {n^3 \left\{{\left(1 + \frac{1}{\sqrt{n}}\right)}^3 + \frac{1}{n^3}\log\left(\frac{n}{n+1}\right)\right\}}
        \sim \frac{1}{n}
    \]
    Siccome la serie armonica è una serie-p con \(p=1\), allora la serie diverge.
}


\sexercise{}{
    Calcolare
    \[
        \lim_{x\to 0} = \frac{x^3 -4x^2 + 2x\sin x}{x^3\cos(x) - {(e^x - 1)}^2}
        = 2
    \]
}

\sexercise{}{
    Calcolare
    \[
        \lim_{x\to\infty} = \frac{
            {\left(\frac{x^2-1}{x}\right)}^3 + x^4 \sin\left(\frac{1}{\sqrt{x}}\right)
        }{
            \sqrt{x} {\left(\sqrt{x^2 + 1} - x\right)}^2 + x^3\left(1 - \cos\frac{1}{\sqrt{x}}\right)
        }
    \]
    Poiché \(\frac{1}{\sqrt{x}} \to 0\), \(\sin \frac{1}{\sqrt{x}} \sim \frac{1}{\sqrt{x}}\).
    Inoltre, \(x^3\sin\frac{1}{\sqrt{x}} \sim x^{7/2}\), quindi a numeratore raggruppiamo \(x^{7/2}\).
    Per il denominatore \(1 - \cos\frac{1}{\sqrt{x}} \sim \frac{1}{2x}\).
    \begin{align*}
        \sqrt{x}x^2{(\sqrt{1 + \frac{1}{x^2} - 1})}^2 & \sim x^{5/2} {\left(\frac{1}{2x^2}\right)}^2 \\
        &= \frac{1}{2} \frac{1}{x^{3/2}} \to 0
    \end{align*}
    Riscriviamo allo l'espressione come
    \begin{align*}
        \phantom{ } &= \frac{
            x^{7/2} \left\{x^{-7/2} x^3 {\left(1 - \frac{1}{x}\right)}^3 + x^{1/2}\sin(\frac{1}{x^{1/2}})\right\}
        }{
            x^2 \left\{ x^{-2} x^{1/2} {\left(\sqrt{x^2 + 1} - x\right)}^2 + x\cos\left(1 - \frac{1}{\sqrt{x}}\right)\right\}
        } \\
        &\sim 2x^{3/2} \to +\infty
    \end{align*}
}

\sexercise{}{
    Calcolare
    \[
        \lim_{x\to\infty} = {\left(
            \frac{4x-1}{4x+5}
        \right)}^{2x-1}
    \]
    la forma di indecisione è \(1^\infty\).
    Allora usiamo la forma esponenziale
    \[
        e^{(2x-1) \log \left(\frac{4x-1}{4x+5}\right)}
    \]
    Vogliamo usare \(\log(1 + f(x)) \sim f(x)\) con \(f(x) \to 0\).
    Allora scriviamo
    \[
        e^{(2x-1) \log \left(1 - \frac{6}{4x+5}\right)}
    \]
    dove l'esponente è asintotico a \(-3\).
    Allora il limite è pari a \(e^{-3}\).
}

\sexercise{}{
    Calcolare
    \[
        \lim_{x\to 0} = \frac{
            \sin^2(x)\log(1 + \tan^4(\frac{x}{1 + x^4}))
        }{
            \left(e^{2\sin^4x} - 1\right)
            \left(\sqrt[6]{1 + \frac{x^2}{(1 + x)^{3/7}}} - 1\right)
        }
    \]
    Abbiamo:
    \begin{enumerate}
        \item \(\sin(x^2) \sim x^2\)
        \item \(\tan(1 + \tan^4(\frac{x^2}{1 + x^2})) \sim \tan^4(\frac{x^2}{1 + x^2})\sim {\left(\frac{x^2}{1 + x^2}\right)}^4 \sim x^4\)
        \item \(e^{2\sin^4(x)} - 1 \sim 2\sim x^4 \sim 2x^4\)
        \item \(\sqrt[6]{1 + \frac{x^2}{(1 + x)^{3/7}}} - 1 \sim \frac{1}{6}x^2\)
    \end{enumerate}
    XXX
}

\sexercise{}{
    Calcolare
    \[
        \lim_{x\to 0} \frac{
            \sin x - \log(1 + 2x)
        }{
            \sqrt[6]{1 + x} - \sqrt[6]{1 - x}
        }
    \]
    Scriviamo l'asintotico con l'o-piccolo:
    \begin{enumerate}
        \item \(\sin x= x(1 + o(1))\)
        \item \(\log(1 + 2x) = 2x(1 + o(1))\)
    \end{enumerate}
    Allora
    \begin{align*}
        \sin x - \log(1 + 2x) &= x + xo(1) - 2x - 2xo(1) \\
        &= -x + xo(1) = -x(1 + o(1))
    \end{align*}
    Al denominatore abbiamo
    \[
        {\left(1 + 1x\right)}^{1/6} - 1
        = \frac{1}{6}x(1 + o(1))
    \]
    e allora
    \[
        {\left(1 + 1x\right)}^{1/6} = 1 + \frac{1}{6}x(1 + o(1))
    \]
    Trasformiamo analogamente l'altro termine e troviamo
    \[
        \sqrt[6]{1 + x} - \sqrt[6]{1 - x}
        = \frac{1}{3}x(1 + o(1)) \sim \frac{1}{3}x
    \]
    e quindi il limite fa \(-3\).
}

\sexercise{}{
    Calcolare
    \[
        \lim_{x\to 0^+}
        \frac{
            e^{\frac{2}{3}x} - \cos\sqrt{x}
        }{
            {\left(\tan(2x)\right)}^\alpha
        }
    \]
    Il primo termine è pari a \(1 + \frac{2}{3}x(1 + o(1))\),
    il secondo \(1 - \frac{1}{2}x(1 + o(1))\).
    Abbiamo allora
    \[
        \lim_{x\to 0^+} \frac{
            1 + \frac{2}{3}x(1 + o(1)) - 1 + \frac{1}{2}x(1 + o(1))
        }{
            {(2x)}^\alpha
        }
        \sim \frac{7}{3\cdot 2^{\alpha + 1} x^{1 - \alpha}}
        = \begin{cases}
            0^+ & \alpha < 1 \\
            \frac{7}{12} & \alpha = 1 \\
            + \infty & \alpha > 1
        \end{cases}
    \]
}

\sexercise{}{
    Calcolare
    \[
        \lim_{x\to \frac{\pi}{2}}
        \frac{
            \cos x + {\left(x-\frac{\pi}{2}\right)}^2
        }{
            \sin x \left(\sqrt{x} - \frac{\pi}{2}\right)
        }
    \]
    Conviene razionalizzare
    \begin{align*}
        \lim_{x\to \frac{\pi}{2}}
        \frac{
            \left[\cos x + {\left(x-\frac{\pi}{2}\right)}^2\right]
            \left(\sqrt{x} + \sqrt{\frac{\pi}{2}}\right)
        }{
            \sin x \left[
                \left(\sqrt{x} - \sqrt{\frac{\pi}{2}}\right)
                \left(\sqrt{x} + \sqrt{\frac{\pi}{2}}\right)
            \right]
        } = \frac{
            2\sqrt{\frac{\pi}{2}} \left[\cos x + {\left(x - \frac{\pi}{2}\right)}^2\right]
        }{
            \sin x\left(x-\frac{\pi}{2}\right)
        }
    \end{align*}
    Sostituiamo la variabile \(y = \frac{\pi}{2}\)
    \[
        \lim_{y\to 0} \frac{
        \sqrt{2\pi} \left[
            \cos\left(y + \frac{\pi}{2}\right) + y^2
        \right]
        }{y}
    \]
    Notiamo che \(\cos\left(y + \frac{\pi}{2}\right)  = \sin\left(y\right) \sim -y\).
    Quindi,
    \[
        \lim_{y\to 0} \frac{\sqrt{2\pi}(-y)}{y} = -\sqrt{2\pi}
    \]
}

\sexercise{}{
    Calcolare
    \[
        \lim_{x\to 1} \begin{cases}
            \frac{e^{\frac{1}{x-1}} - 1}{x-1} & x > 1 \\
            \sin(\frac{\pi}{2} x) & x < 1
        \end{cases}
    \]
    Calcoliamo allora i limiti dalla due direzioni.
    \[
        \lim_{x\to 1^-} \sin(\frac{\pi}{2} x) = 1
    \]
    L'altro limite
    \[
        \lim_{x\to 1^+} \frac{e^{\frac{1}{x-1}} - 1}{x-1}
        = \frac{\infty}{0^+} = +\infty
    \]
    Quindi il limite generale non esiste.
}

\sexercise{}{
    Calcolare
    \[
        \lim_{x\to 0^+} {\left[
            1 + \sin\left(\frac{x^\alpha}{x + 1}\right)
        \right]}^{
            \frac{x+1}{x^3 + \tan^2x}
        }
    \]
    Scriviamo la forma esponenziale
    \begin{align*}
        \lim_{x\to 0^+} \exp\left\{
            \frac{x+1}{x^3 + \tan^2x}
            \log\left(
                1 + \sin\left(\frac{x^\alpha}{x + 1}\right)
            \right)      
        \right\}
    \end{align*}
    Il primo termine è asintotico a \(\frac{1}{x^2}\),
    mentre il logaritmo è asintotico a \(\sin(\frac{x^\alpha}{x+1})\)
    che è asintotico a \(\frac{x^\alpha}{x+1}\).
    \[
        \lim_{x\to 0^+} x^{\alpha - 2} = \begin{cases}
            +\infty & \alpha > 2 \\
            e & \alpha = 2 \\
            1 & \alpha < 2
        \end{cases}
    \]
}

\sexercise{}{
    Calcolare
    \[
        \lim_{x\to 0^+}
        \left\{
            \cos\left(\frac{\sqrt{x}}{2 + x}\right)
        \right\}^{
            \frac{\tan x}{\log(1 + 1 + x^2)}
        }
    \]
}

\end{document}
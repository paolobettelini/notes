\documentclass[a4paper]{article}

\usepackage{amsmath}
\usepackage{amssymb}
\usepackage{stellar}
\usepackage{parskip}
\usepackage{fullpage}
\usepackage{wrapfig}
\usepackage{tikz}

\usetikzlibrary{arrows}
\usetikzlibrary{decorations.pathreplacing}
\usetikzlibrary{cd}

\title{Exercises}
\author{Paolo Bettelini}
\date{}

\begin{document}

\maketitle
\tableofcontents

\section{Esercizi}

\sexercise{}{
    Verificare per induzione che
    per ogni \(k\),
    \[
        D^k e^{-\frac{1}{x^2}} = P_k(x) e^{-\frac{1}{x^2}}
    \]
    con \(x\neq 0\), dove \(P_k\) è un polinomio di grado minore o uguale a \(k\)
    in \(\frac{1}{x^3}\).
    Dedurre inoltre che per tutte le \(k\),
    \[
        \lim_{x\to 0} D^k \left(e^{-\frac{1}{x^2}}\right) = 0
        = D^k\left(e^{-\frac{1}{x^2}}\right)(0)
    \]
}


\sexercise{Studiare al variare di \(a,b\in\mathbb{R}\) la derivabilità
della funzione \[
    f(x) = \begin{cases}
        \frac{1-\cos x}{x} + x^{\sqrt{1 + x^2} - 1} = f_+(x) & x > 0 \\
        a \ln\left(\frac{1-x}{\sqrt{1+x^2}}\right) + e^{b \cos(x) - 1} =f_-(x) & x \leq 0
    \end{cases}
\]}{
    Per i teoremi di derivabilità \(f_+(x)\) è derivabile per \(x>0\)
    e quindi \(f\) è derivabile per \(x>0\).
    La funzione \(f_-\) è anch'essa derivabile per \(x<1\), in particolare per \(x\leq 0\).
    Quindi, manca da studiare il punto \(x=0\).
    La continuità deve valere
    \[
       \exists\,\lim_{x\to 0^-} f(x) = \lim_{x\to 0^+} f(x) = f(0) 
    \]
    l limite è dato da
    \begin{align*}
       \lim_{x\to 0^-} = f_-(0) = e^{b-1}
    \end{align*}
    n quanto \(f_-\) è continua. Quindi, \(f\) è sempre continua
    a sinistra. L'altro limite è dato da
    \begin{align*}
        \lim_{x\to 0^+} \frac{1-\cos x}{x} + x^{\sqrt{1 + x^2} - 1} &=
        \lim_{x\to 0^+} \frac{1}{2}\frac{x^2}{x} +
        e^{ \left(\sqrt{1 + x^2} - 1\right)\ln(x)} \\
        &= \lim_{x\to 0^+} e^{\frac{1}{2}x^2 \ln(x)}
    \end{align*}
    Abbiamo che
    \[
        x^2 \ln(x) = - \frac{\ln\left(1/x\right)}{{(1/x)}^2} \to 0
    \]
    Quindi, il secondo limite è \(1\).
    Allora \(f\) è continua se e solo se \(1 = e^{b-1}\), quindi \(b=1\).
    Poiché la derivabilità implica la continuità, deve essere \(b=1\).
    Inoltre devono esistere finite \(D_-f(0) = D_+f(0)\),
    quindi
    \[
        D_-f(0) = D_-(f_-)(0) = \begin{cases}
            \lim_{x\to 0^-} \frac{f_-(x) - f_-(0)}{x-0} = -a \\
            Df_-(0) = -a
        \end{cases}
    \]
    e quella sinsitra
    \[
        D_+f(0) = \lim_{x\to 0^+} \frac{f_+(x) - f(0)}{x-0}
        = \frac{1}{2}
    \]
    Concludiamo allora che \(f\) è derivabile in \(x=0\) 
    se e solo se \(b=1\) e \(a=-\frac{1}{2}\).
    In tal caso \(f'(0) = \frac{1}{2}\).
    Se \(b=1\) e \(a\neq \frac{1}{2}\), \(f\) è continua ma non derivabile e \(x=0\)
    abbiamo un punto angoloso.
}

\sexercise{}{
    Dimostrare che se \(f\) è continua, periodica e non costante, ammette un minimo periodo positivo.
    La funzione di Dirichlet è periodica di ogni razionale, quindi non ha un periodo minimo.
}

\sexercise{}{
    Scrivere la relazione fra \(F_a(x)\) e \(F_b(x)\)
    per \(a \neq b\).
}

\sexercise{Tra tutti i contenitori cilindrici di volume fissato \(V\),
trovare quello che ha superficie minima.}{
    Abbiamo che \[
        V = r^2 \pi h
    \]
    e
    \[
        S = 2 r^2 \pi h + 2r\pi h
    \]
    Vogliamo minimizzare \(S\) con \(V\) costante.
    Troviamo \(h = \frac{V}{\pi r^2}\), e allora
    \[
        \frac{S}{2} = \pi r^2 + \frac{V}{r}
    \]
    I vincoli sono \(r>0\). Minimizziamo allora \(f(r) = \frac{S}{2}\).
    La funzione è derivabile e quindi continua in \((0, +\infty)\).
    Notiamo che \(f(r) \to \infty\) per \(x \to 0^+\) e \(x\to\infty\).
    Per estensione del teorema di Weierstrass al caso di funzioni definite
    su intervalli aperti che tendono ad infinito agli estremi, esiste un minimo assoluto in tale intervallo.
    Poiché \(f\) è derivabile in \((0, +\infty)\), tale minimo è assunto in un punto stazionario,
    per il teorema di Fermat. Abbiamo
    \[
        f'(r) = 2\pi r - \frac{V}{r^2} = \frac{1}{r^2}(2\pi r^3 - V)
    \]
    Risolviamo quindi \(f'(r) = 0\) e quindi
    \[
        r = {\left(\frac{V}{2\pi}\right)}^{1/2}
    \]
    Vi è un unico punto stazionario nell'intervallo, e quindi questo è il minimo assoluto.
}

\sexercise{}{
    Let
    \[
        E = \left\{ x\in\mathbb{R} \,|\, \frac{1}{2} \leq x < 5 \right\}
    \]
    and the sequence \[
        F = \{ x = x_n \,|\, x_n = \frac{n + 1}{n + 2}, \quad n\in\mathbb{N}^* \}
    \]
    %%%%%%%
    Trova inf, sup, min, max (se esistono) di \(E\), \(F\), \(E \cup F\) e \(E\cap F\).
    \begin{itemize}
        \item \(E\) è limitato superiormente e inferiormente.
        Il minimo è \(\frac{1}{2}\), mentre \(5\) è un maggiorante, è il più piccolo
        dei maggioranti quindi \(\sup E = 5\), ma non vi è un massimo.
        \item \(F\) è limitato superiormente in quanto 
        \[
            x_n = \frac{n+1}{n+2} < \frac{n+2}{n+2}  =1
        \]
        È limitato inferiormente perché \(x_n > 0\).
        Per verificare sup e inf, è comodo riscrivere \[
            x_n = 1 - \frac{1}{n+2}
        \]
        Il temrine \(n+2\) cresce con \(n\), quindi \(\frac{1}{n+2}\) decresce al crescere di \(n\)
        e quindi \(x_n\) cresce approcciando \(1\). Allora con \(n=1\) il termine assume il valore più piccolo,
        ossia \(\frac{2}{3}\), quindi il minimo di \(F\). Allora siccome ci avviciamo arbitrariamente a
        \(1\), è lecito ipotizzare \(\sup F = 1\).
        Il massimo di \(F\) non esiste.
        Rimane da far vedere che se \(z < 1\) allora \(z\) non è maggiorante di \(F\)
        cioè
        \[
            x_n - z = (1-z) - \frac{1}{n+2} > 0
        \]
        purché \(\frac{1}{n+2} < 1-z\) cioè \(n > \frac{1}{1-z}-2\).
        Quindi \(z\) non è maggiorante e \(\sup E = 1\).
        \item Verificare che \(\sup (E \cup F) = \max \{ \sup E,\sup F \}\).
        Abbiamo che \(\sup E \leq \sup F\).
        In sup è il massimo dei due in quanto uno è maggiore dell'altro,
        e fa parte dell'insieme, quindi \(\sup E \cup F = 5\).
        Tuttavia, il max non esiste in quando \(5\notin  E \cup F\).
        Analogamente, \(\inf E \cup F = \frac{1}{2}\). Questo valore è anchde il minimo
        in quanto fa parte dell'insieme.
        \item Mostrare con un esempio che non c'è qualcosa di analogo per l'intersezione.
        \[
            E \cap F = \left\{ x_n = \frac{x+1}{x+2} \ \middle|\ \frac{1}{2} \leq \frac{x+1}{x+2} \leq 5 \right\}
        \]
        Quindi \(F \subseteq E\). Consideriamo allora \(E_1 = [\frac{4}{5}, 5)\)
        \[
            E_1 \cap F = \left\{ x_n = \frac{x+1}{x+2} \ \middle|\ \frac{4}{5} \leq x_n \leq 5 \right\}
        \]
        Per quali \(n\) vale che \(\frac{4}{5} \leq \frac{x+1}{x+2} = x_n\)?
        Abbiamo \(4(n+2) \leq 5(n+1)\) e quindi \(n \geq 3\).
        Allora \(\sup E_1 \cap F = 1\) e non vi è massimo, mentre \(\inf E_1 \cap F = \frac{4}{5}\)
        che è anche il minimo.
        \item Posto \(E+F = \{ x + y \,|\, x \in E, y \in F \}\)
        mostrare \(\sup E + F = \sup E + \sup F\). Supponiamo quindi che \(\sup E\) e \(\sup F\)
        siano finiti. Siccome, per definizione, \(\forall e \in E, e \leq \sup E\)
        e \(\forall f \in F, f \leq \sup F\), abbiamo che \[\forall e \in E, \forall f\in F, e + f \leq \sup E + \sup F\]
        Per mostrare che questo è il più piccolo dei maggioranti, è comodo riscrivere la definizione di
        sup dicendo che \(\mu\) è pari a \(\sup E\) se:
        \begin{enumerate}
            \item \(\forall x \in E, x\leq \mu\);
            \item \(\forall \varepsilon > 0, \mu - \varepsilon\) non è maggiorante.
        \end{enumerate}
        \textbf{Nota:} se \(x < \mu\) allora posto \(\varepsilon = \mu - x\) risulta \(x = \mu - \varepsilon\).
        Allora sia \(\varepsilon > 0\). Diciamo che esistono \(e_1\in E\) e \(f_1\in F\) tali che
        \(e_1 + f_1 > \sup E + \sup F - \varepsilon\).
        Poiché \(\sup E\) è, appunto, il supremum, esiste per definizione una \(e_1 \in E\) tale che
        \(e_1 > e_1 > \sup E . \frac{\varepsilon}{2}\).
        Analogamente, esiste \(f_1 \in F\) tale che \(f_1 > \sup F - \frac{\varepsilon}{2}\).
        Da cui \(e_1 + f_2 > \sup E - \frac{\varepsilon}{2} + \sup F - \frac{\varepsilon}{2} = \sup E + \sup F - \varepsilon\).
        \item Posto \(-E = \{ -x \,|\, x \in E\}\) mostrare che
        \(\sup -E = -\inf E\) e \(\inf -E = -\sup E\).
    \end{itemize}
}

Dimostrare che il max esiste se e solo se \(\sup E\) è finito e appartiene a \(E\).
Analogamente per il min.


\sexercise{}{
    Trovare sup, inf, min, max dell'insieme
    \[
        E = \left\{ x_n = \frac{n-7}{x^2 + 1} \ \middle|\ n \geq 1 \right\}
    \]
    Questa successione ha sicuramente un minimo in quanto ci sono solamente \(6\) numeri negativi.
    Possiamo notare che il denominatore cresce più velocemente del numeratore.

    Studiamo quindi per quali indici vale \(x_n \leq x_{n+1}\). Otteniamo quindi
    \begin{align*}
        \frac{n-7}{n^2 + 1} &\leq \frac{(n+1)-7}{{(n+1)}^2 + 1} \\
        \frac{(n-7)(n^2 + 2n + 2) - (n-6)(n^2+1)}{(n^2 + 1)(n^2 + 2n + 2)} &\leq 0
    \end{align*}
    Il denominatore è positivo, quindi studiamo il numeratore
    \begin{align*}
        n^2 - 13n - 8 \leq 0
    \end{align*}
    Le radici di questo polinomio sono \(n_{1,2}= \frac{13\pm\sqrt{201}}{2}\).
    Di conseguenza, l'espressione è negativa per \(\frac{13-\sqrt{201}}{2} < n < \frac{13+\sqrt{201}}{2}\).
    Notiamo che l'estremo di sinistra è negativo. Notiamo anche che \(14^2 < 201 < 15^2\),
    e quindi l'estremo di destra è compreso fra \(14\) e \(\frac{27}{2}\).
    Allora, tutte le \(n\) intere che soddisfano l'equazione sono
    \(n=13\). Ne consegue che se \(n \geq 14\), \(x_n > x_{n+1}\).
    Il maggiornate e supremum è quindi \(x_{14}\).

}


\sexercise{}{
    \[
        a_n = \frac{
            \log \left(\frac{n^2 + 1}{n}\right) + 1
        }{
            \sqrt{n^3 + 1} + \log n
        }
    \]
}

\sexercise{}{
    \[
        a_n = \frac{
            n^{1/2} + \cos(1/n) + \log n
        }{
            {(n + \sqrt{n})}^2 - \sqrt{n}
        }
    \]
}

\sexercise{}{
    \[
        a_n = \log \left(
            1 + \sin \left(\frac{\sqrt{n}}{n^2 + \log n}\right)
        \right)
        \left(\sqrt[3]{n^6 + 1} - n^2\right)
    \]
}

\sexercise{}{
    \[
        a_n = {\left(\cos \frac{1}{\sqrt{n}}\right)}^{\frac{n^3 - \log n}{\sqrt{n^4 + n}}}
    \]
}


\sexercise{}{
    Sia \(\{b_n\}\) una successione e sia \(\{b_{n\pm k_0}\}\) la successione traslata di
    \(\pm k\).
    Dimostrare che \(\lim b_n\) esiste se e solo se \(\lim b_{n \pm k_0}\) esiste e che i limiti sono uguali.
    % TODOURGENT
}

\sexample{}{
    Considera
    \[
        \sum_{n=1}^\infty \frac{1}{4n^2 - 1}
    \]
    Allora
    \[
        \frac{1}{4n^2 - 1} = \frac{1}{(2n+1)(2n-1)} = \frac{1/2}{2n-1} - \frac{1/2}{2n+1}
    \]
    Quindi
    \begin{align*}
        \frac{1}{2} \sum_{n=1}^\infty \left(\frac{1}{2n-1} - \frac{1}{2n+1}\right) \to
        \frac{1}{2}  
    \end{align*}
}

\sexercise{}{
    Calcolare
    \begin{align*}
        \sum_{n=1}^\infty \frac{3^n + 2\cdot 5^{n+1}}{7^{n+2}}
        &= \sum_{n=1}^\infty \frac{3^{n}}{7^{n+2}}
        + 2 \sum_{n=1}^\infty \frac{5^{n+1}}{7^{n+2}} \\
        &= \frac{1}{7^2} \sum_{n=1}^\infty \frac{3^{n}}{7^n}
        + \frac{2}{7} \sum_{n=1}^\infty \frac{5^{n+1}}{7^{n+1}} \\
        &= \frac{1}{7^2} \sum_{n=0}^\infty {\left(\frac{3}{7}\right)}^{n+1}
        +  \frac{2}{7} \sum_{n=0}^\infty {\left(\frac{3}{7}\right)}^{n+2} \\
        &= \cdots
    \end{align*}
}


\sexercise{}{
    Stabilire il carattere della serie
    \[
        \sum_{n=1}^\infty \frac{n^2 + {(1 + 1/n)}^n + \sin n}{{(n+\sqrt{n})}^3 + \log \left(\frac{n}{n+1}\right)}
    \]
    Notiamo che \(\forall n \geq 1, a_n \geq 0\).
    Notiamo allora che
    \[
        a_n = \frac{n^2 \left(1 + \frac{1}{n^2}{\left(1 + \frac{1}{n}\right)}^n + \frac{\sin n}{n}\right)}
        {n^3 \left\{{\left(1 + \frac{1}{\sqrt{n}}\right)}^3 + \frac{1}{n^3}\log\left(\frac{n}{n+1}\right)\right\}}
        \sim \frac{1}{n}
    \]
    Siccome la serie armonica è una serie-p con \(p=1\), allora la serie diverge.
}


\sexercise{}{
    Calcolare
    \[
        \lim_{x\to 0} = \frac{x^3 -4x^2 + 2x\sin x}{x^3\cos(x) - {(e^x - 1)}^2}
        = 2
    \]
}

\sexercise{}{
    Calcolare
    \[
        \lim_{x\to\infty} = \frac{
            {\left(\frac{x^2-1}{x}\right)}^3 + x^4 \sin\left(\frac{1}{\sqrt{x}}\right)
        }{
            \sqrt{x} {\left(\sqrt{x^2 + 1} - x\right)}^2 + x^3\left(1 - \cos\frac{1}{\sqrt{x}}\right)
        }
    \]
    Poiché \(\frac{1}{\sqrt{x}} \to 0\), \(\sin \frac{1}{\sqrt{x}} \sim \frac{1}{\sqrt{x}}\).
    Inoltre, \(x^3\sin\frac{1}{\sqrt{x}} \sim x^{7/2}\), quindi a numeratore raggruppiamo \(x^{7/2}\).
    Per il denominatore \(1 - \cos\frac{1}{\sqrt{x}} \sim \frac{1}{2x}\).
    \begin{align*}
        \sqrt{x}x^2{(\sqrt{1 + \frac{1}{x^2} - 1})}^2 & \sim x^{5/2} {\left(\frac{1}{2x^2}\right)}^2 \\
        &= \frac{1}{2} \frac{1}{x^{3/2}} \to 0
    \end{align*}
    Riscriviamo allo l'espressione come
    \begin{align*}
        \phantom{ } &= \frac{
            x^{7/2} \left\{x^{-7/2} x^3 {\left(1 - \frac{1}{x}\right)}^3 + x^{1/2}\sin(\frac{1}{x^{1/2}})\right\}
        }{
            x^2 \left\{ x^{-2} x^{1/2} {\left(\sqrt{x^2 + 1} - x\right)}^2 + x\cos\left(1 - \frac{1}{\sqrt{x}}\right)\right\}
        } \\
        &\sim 2x^{3/2} \to +\infty
    \end{align*}
}

\sexercise{}{
    Calcolare
    \[
        \lim_{x\to\infty} = {\left(
            \frac{4x-1}{4x+5}
        \right)}^{2x-1}
    \]
    la forma di indecisione è \(1^\infty\).
    Allora usiamo la forma esponenziale
    \[
        e^{(2x-1) \log \left(\frac{4x-1}{4x+5}\right)}
    \]
    Vogliamo usare \(\log(1 + f(x)) \sim f(x)\) con \(f(x) \to 0\).
    Allora scriviamo
    \[
        e^{(2x-1) \log \left(1 - \frac{6}{4x+5}\right)}
    \]
    dove l'esponente è asintotico a \(-3\).
    Allora il limite è pari a \(e^{-3}\).
}

\sexercise{}{
    Calcolare
    \[
        \lim_{x\to 0} = \frac{
            \sin^2(x)\log(1 + \tan^4(\frac{x}{1 + x^4}))
        }{
            \left(e^{2\sin^4x} - 1\right)
            \left(\sqrt[6]{1 + \frac{x^2}{(1 + x)^{3/7}}} - 1\right)
        }
    \]
    Abbiamo:
    \begin{enumerate}
        \item \(\sin(x^2) \sim x^2\)
        \item \(\tan(1 + \tan^4(\frac{x^2}{1 + x^2})) \sim \tan^4(\frac{x^2}{1 + x^2})\sim {\left(\frac{x^2}{1 + x^2}\right)}^4 \sim x^4\)
        \item \(e^{2\sin^4(x)} - 1 \sim 2\sim x^4 \sim 2x^4\)
        \item \(\sqrt[6]{1 + \frac{x^2}{(1 + x)^{3/7}}} - 1 \sim \frac{1}{6}x^2\)
    \end{enumerate}
    XXX
}

\sexercise{}{
    Calcolare
    \[
        \lim_{x\to 0} \frac{
            \sin x - \log(1 + 2x)
        }{
            \sqrt[6]{1 + x} - \sqrt[6]{1 - x}
        }
    \]
    Scriviamo l'asintotico con l'o-piccolo:
    \begin{enumerate}
        \item \(\sin x= x(1 + o(1))\)
        \item \(\log(1 + 2x) = 2x(1 + o(1))\)
    \end{enumerate}
    Allora
    \begin{align*}
        \sin x - \log(1 + 2x) &= x + xo(1) - 2x - 2xo(1) \\
        &= -x + xo(1) = -x(1 + o(1))
    \end{align*}
    Al denominatore abbiamo
    \[
        {\left(1 + 1x\right)}^{1/6} - 1
        = \frac{1}{6}x(1 + o(1))
    \]
    e allora
    \[
        {\left(1 + 1x\right)}^{1/6} = 1 + \frac{1}{6}x(1 + o(1))
    \]
    Trasformiamo analogamente l'altro termine e troviamo
    \[
        \sqrt[6]{1 + x} - \sqrt[6]{1 - x}
        = \frac{1}{3}x(1 + o(1)) \sim \frac{1}{3}x
    \]
    e quindi il limite fa \(-3\).
}

\sexercise{}{
    Calcolare
    \[
        \lim_{x\to 0^+}
        \frac{
            e^{\frac{2}{3}x} - \cos\sqrt{x}
        }{
            {\left(\tan(2x)\right)}^\alpha
        }
    \]
    Il primo termine è pari a \(1 + \frac{2}{3}x(1 + o(1))\),
    il secondo \(1 - \frac{1}{2}x(1 + o(1))\).
    Abbiamo allora
    \[
        \lim_{x\to 0^+} \frac{
            1 + \frac{2}{3}x(1 + o(1)) - 1 + \frac{1}{2}x(1 + o(1))
        }{
            {(2x)}^\alpha
        }
        \sim \frac{7}{3\cdot 2^{\alpha + 1} x^{1 - \alpha}}
        = \begin{cases}
            0^+ & \alpha < 1 \\
            \frac{7}{12} & \alpha = 1 \\
            + \infty & \alpha > 1
        \end{cases}
    \]
}

\sexercise{}{
    Calcolare
    \[
        \lim_{x\to \frac{\pi}{2}}
        \frac{
            \cos x + {\left(x-\frac{\pi}{2}\right)}^2
        }{
            \sin x \left(\sqrt{x} - \frac{\pi}{2}\right)
        }
    \]
    Conviene razionalizzare
    \begin{align*}
        \lim_{x\to \frac{\pi}{2}}
        \frac{
            \left[\cos x + {\left(x-\frac{\pi}{2}\right)}^2\right]
            \left(\sqrt{x} + \sqrt{\frac{\pi}{2}}\right)
        }{
            \sin x \left[
                \left(\sqrt{x} - \sqrt{\frac{\pi}{2}}\right)
                \left(\sqrt{x} + \sqrt{\frac{\pi}{2}}\right)
            \right]
        } = \frac{
            2\sqrt{\frac{\pi}{2}} \left[\cos x + {\left(x - \frac{\pi}{2}\right)}^2\right]
        }{
            \sin x\left(x-\frac{\pi}{2}\right)
        }
    \end{align*}
    Sostituiamo la variabile \(y = \frac{\pi}{2}\)
    \[
        \lim_{y\to 0} \frac{
        \sqrt{2\pi} \left[
            \cos\left(y + \frac{\pi}{2}\right) + y^2
        \right]
        }{y}
    \]
    Notiamo che \(\cos\left(y + \frac{\pi}{2}\right)  = \sin\left(y\right) \sim -y\).
    Quindi,
    \[
        \lim_{y\to 0} \frac{\sqrt{2\pi}(-y)}{y} = -\sqrt{2\pi}
    \]
}

\sexercise{}{
    Calcolare
    \[
        \lim_{x\to 1} \begin{cases}
            \frac{e^{\frac{1}{x-1}} - 1}{x-1} & x > 1 \\
            \sin(\frac{\pi}{2} x) & x < 1
        \end{cases}
    \]
    Calcoliamo allora i limiti dalla due direzioni.
    \[
        \lim_{x\to 1^-} \sin(\frac{\pi}{2} x) = 1
    \]
    L'altro limite
    \[
        \lim_{x\to 1^+} \frac{e^{\frac{1}{x-1}} - 1}{x-1}
        = \frac{\infty}{0^+} = +\infty
    \]
    Quindi il limite generale non esiste.
}

\sexercise{}{
    Calcolare
    \[
        \lim_{x\to 0^+} {\left[
            1 + \sin\left(\frac{x^\alpha}{x + 1}\right)
        \right]}^{
            \frac{x+1}{x^3 + \tan^2x}
        }
    \]
    Scriviamo la forma esponenziale
    \begin{align*}
        \lim_{x\to 0^+} \exp\left\{
            \frac{x+1}{x^3 + \tan^2x}
            \log\left(
                1 + \sin\left(\frac{x^\alpha}{x + 1}\right)
            \right)      
        \right\}
    \end{align*}
    Il primo termine è asintotico a \(\frac{1}{x^2}\),
    mentre il logaritmo è asintotico a \(\sin(\frac{x^\alpha}{x+1})\)
    che è asintotico a \(\frac{x^\alpha}{x+1}\).
    \[
        \lim_{x\to 0^+} x^{\alpha - 2} = \begin{cases}
            +\infty & \alpha > 2 \\
            e & \alpha = 2 \\
            1 & \alpha < 2
        \end{cases}
    \]
}

\sexercise{}{
    Calcolare
    \[
        \lim_{x\to 0^+}
        \left\{
            \cos\left(\frac{\sqrt{x}}{2 + x}\right)
        \right\}^{
            \frac{\tan x}{\log(1 + 1 + x^2)}
        }
    \]
}


\sexample{}{
    Consider
    \[
        \sum_{n=1}^\infty \frac{A^n}{n!}
    \]
    con \(A > 0\).
    Quando ci sono i fattoriali usiamo il criterio dei rapporti.
    Abbiamo che
    \[
        \forall n, a_n = \frac{A^n}{n!} > 0
    \]
    per il criterio del rapporto
    \begin{align*}
        \frac{a_{n+1}}{a_n} = \frac{A^n}{(n+1)!}
        \cdot \frac{n!}{A^n} = \frac{A}{n+1} \to 0
    \end{align*}
    Quindi la serie converge, e converge a \(e^A - 1\).
}

\sexample{}{
    Consider
    \[
        \sum_{n=1}^\infty \frac{e^{n^2} + {(\log n)}^n + {\left(1 + \frac{1}{n}\right)}^n}{n^n + e^{3n\log n} + {\left(n+\frac{1}{n}\right)}^{17}}
    \]
    che è ovviamente positivo.
    Usiamo il criterio asintotito.
    A numeratore l'ultimo termine è finito e tende ad \(e\).
    Dobbiamo verificare quale degli altri due termini è dominante.
    Scriviamo allora \({(\log n)}^n = e^{n\log\log n}\).
    Allora chiaramente \(e^{n^2}\) domincia sull'altro termine.
    Analogamente, a denominatore abbiamo \(n^n = e^{n\log n}\)
    come termine dominante.
    \[
        a_n =  \frac{
            e^{n^2} \left\{
            1 + e^{n\log \log(n) - n^2}
            + {\left(1 + \frac{1}{n}\right)}^{n} \cdot e^{-n^2}
        \right\}
        }{
            e^{3n\log n} \left\{
                1 + e^{-2n\log n} + {\left(1 + \frac{1}{n}\right)}^{17}
                \cdot e^{-3n\log n}
            \right\}
        } \sim \frac{e^{n^2}}{e^{3n\log n}}
    \]
    Allora
    \[
        e^{n\log \log n - n^2} = e^{-n^2 \left\{1 - \frac{\log\log n}{n^2}\right\}}
        \to \infty
    \]
    quindi la serie diverge.
    Oppure, con il criterio della radice
    \[
        {\left(e^{n^2 - 3n \log n}\right)}^{\frac{1}{n}}
        = e^{n\left(1 - \frac{3\log n}{n}\right)} \to \infty > 1
    \]
}

\sexample{}{
    Studiare il carattere di
    \[
        \sum^\infty \frac{
            n^{n\log n}
        }{
            (2n)!
        }
    \]
    che ha termini positivi. Ci sono dei fattoriali quindi conviene utilizzare il criterio
    del rapporto.
    Notiamo che \((2n+2)! = (2n+2)(2n+2)(2n)!\)
    e \((n+1)\log(n+1) = n\log(n+1) + \log(n+1) = n[\log n + \log(1 + 1/n)] + \log(n+1)\).
    Il rapporto è dato da\begin{align*}
        \frac{a_{n+1}}{a_n} &= \frac{
            {(n+1)}^{(n+1)\log(n+1)}
        }{
            (2n+2)!
        }
        \cdot
        \frac{
            (2n)!
        }{
            n^{n\log n}
        } \\
        &= \frac{
            {(n+1)}^{n\log n} \cdot {(n+1)}^{n\log(1 + 1/n) + \log(n+1)}
        }{
            n^{n\log n}
        }
    \end{align*}
    Con
    \[
        {\left(1 + \frac{1}{n}\right)}^{n\log n}
        \cdot {(n+1)}^{n\log(1 + 1/n)}
        \cdot {(n+1)}^{\log (n+1)}
    \]
    troviamo
    \[
        \frac{1}{((2n+2)(2n+1))} \cdot {\left[
            {\left(1 + \frac{1}{n}\right)}^n
        \right]}^{\log n}
        \cdot {(n+1)}^{n\log (1 + 1/n)}{(n+1)}^{\log(n+1)}
    \]
    Dal primo e ultimo termine possiamo notare che la serie va ad infinito.
}

\sexample{}{
    Considera
    \[
        \sum_{n=1}^\infty
        \frac{
            2^{\sqrt{n}}
        }{
            n^{\log n}
        }
    \]
    Il criterio della radice non funziona. Infatti,
    \begin{align*}
        \sqrt[n]{a_n}
        = \frac{
            2^{1 / \sqrt{n}}
        }{
            n^{\frac{\log n}{n}}
        }
    \end{align*}
    Il numeratore tende a \(1\),
    mentre scriviamo il denominatore come
    \begin{align*}
        n^{\frac{\log n}{n}} &=
        e^{\frac{1}{n}\log \left(n^{\log n}\right)} \\
        &= e^{\frac{1}{2}{\left(\log n\right)}^2} \to 1
    \end{align*}
    Allora il limite è \(L=1\), quindi il criterio è inconclusivo.
    Allora
    \begin{align*}
        a_n &= \frac{
            a^{\sqrt{n}\log 2}
        }{
            e^{{(\log n)}^2}
        } \\
        &= e^{\sqrt{n} \log 2 - {\left(\log n\right)}^2} \\
        &= e^{\sqrt{n} \left\{\log 2 - \frac{{\log n}^2}{\sqrt{n}}\right\}}
    \end{align*}
    L'esponente tende a infinito quindi la serie diverge per il criterio
    del termine n-esimo.
}

\sexample{}{
    \[
        \sum_{n=1}^\infty
        \frac{
            n^{\log n}
        }{
            2^{\sqrt{n}}
        }
    \]
    che ha i termini della serie precedente ma invertiti.
    Dobbiamo usare il confronto per mostrare che la serie converge.
    Confrontiamo la serie con una p-serie, per esempio \(\sum\frac{1}{n^2}\).
    Il rapporto è dato da
    \begin{align*}
        \frac{a_n}{\frac{1}{n^2}}
        &= n^2 a_n  \\
        &= e^{2\log n - \sqrt{n}\left\{\log 2 - \frac{{(\log n)}^2}{\sqrt{n}}\right\}}
    \end{align*}
    e abbiamo che
    \begin{align*}
        2\log n - \sqrt{n}\log 2 + {(\log n)}^2
        = -\sqrt{n} \left\{
            \log 2 - \frac{2\log n}{\sqrt{n}}
            - \frac{{(\log n)}^2}{\sqrt{n}} \to -\infty
        \right\}
    \end{align*}
    e quindi il rapporto tende a \(0\). Quindi,
    il rapporto è minore di 1 definitivamente
    e la serie converge per confronto.
}


\sexample{}{
    Studia il carattere di
    \[
        \sum_{n=1}^\infty \frac{n^n}{(2n)!}
    \]
    Il limite è dato da
    \begin{align*}
        \frac{a_{n+1}}{a_n} = \frac{
            {(n+1)}^{n+1}
        }{
            (2n+2)!
        }
        \cdot
        \frac{
            (2n)!
        }{
            n^n
        }
        &=
        {\left(\frac{n+1}{n}\right)}^n
        \frac{
            (n+1)(2n)!
        }{
            (2n+2)(2n+1)(2n)!
        } \\
        &=
        {\left(\frac{n+1}{n}\right)}^n
        \frac{n+1}{(2n+2)(2n)!}
        \sim e \cdot  \frac{
            n+1
        }{
            (2n+2)(2n+1)
        } \\
        &= \frac{e^{n(1+1/n)}}{{(2n)}^2 {\left(1 + \frac{1}{n}\right)}{\left(1 + \frac{1}{2n}\right)}}
        \to 0
    \end{align*}
    Quindi la serie converge.

    Con radici abbiamo
    \begin{align*}
        {\left[\frac{n^n}{(2n)!}\right]}^{1/n}
        &= \frac{n}{
            {\left[
                {(2n)}^{2n}
                \cdot 2^{-2n} \cdot \sqrt{4\pi n}(1 + o(1))
            \right]}^{1/n}
        } \\
        &= \frac{n}{
            {(2n)}^2 \cdot e^{-2} {(4\pi)}^{\frac{1}{2n}}
            \cdot n^{\frac{1}{2n}}{(1 + o(1))}^{\frac{1}{n}}
        } \to 0
    \end{align*}
    E quindi converge
}

\sexample{}{
    Studia il carattere di
    \[
        \sum_{n=1}^\infty \frac{e^{n^2} + n^n}{(n^2)! + {\left(1 + \frac{1}{n}\right)}^{n^2}}
    \]
    A numeratore abbiamo
    \begin{align*}
        e^{n^2} + n^n = e^{n^2} + e^{n \log n} = e^{n^2}
        \left\{1 + e^{n\log n - n^2}\right\}
        \sim e^{n^2}
    \end{align*}
    A denominatore abbiamo
    \begin{align*}
        (n^2)! + {\left(1 + \frac{1}{n}\right)}^{n^2}
        &= {(n^2)}^{n^2} \cdot e^{-n^2}
        \cdot \sqrt{2\pi n^2}(1 + o(1))
        + e^{n^2 \log \left(1 + \frac{1}{n}\right)} \\
        &= e^{2n^2 \left\{
            \log n - \frac{1}{2} + \frac{1}{2n^2} \log \sqrt{2\pi n^2}
        \right\}}(1 + o(1)) + e^{n^2 \log\left(1 + \frac{1}{n}\right)}
        \\
        &= e^{2n^2 \left\{
            \log n - \frac{1}{2} + \frac{1}{2n^2} \log\sqrt{2\pi n^2}
        \right\}}
        \left\{1 + o(1) + e^{n^2\log\left(1 + \frac{1}{n}\right) - 2n^2 \left\{\cdots\right\}}\right\} \\
        &\sim e^{2n^2 \left\{
            \log n - \frac{1}{n} + \frac{1}{2n^2}
            \log\sqrt{2\pi n^2}
        \right\}}
        = {(n^2)}^{n^2} e^{-n^2} \sqrt{2\pi n^2}
    \end{align*}
    Ora possiamo usare il criterio della radice
    \begin{align*}
        a_n \sim \frac{
            e^{n^2}
        }{
            {(n^2)}^{n^2}
            e^{-n^2}
            \sqrt{2\pi n^2}
        }
        = b_n
    \end{align*}
    Abbiamo che \(\sum a_n\) ha lo stesso carattere di \(\sum b_n\)
    e
    \begin{align*}
        \sqrt[n]{b_n} &= \frac{
            e^n
        }{
            {(n^2)}^n e^{-n}
            {(2n)}^{\frac{1}{2n}} n^{1/n}
        } \\
        &= \frac{
            e^{2n}
        }{
            n^{2n} {(2n)}^{1/n} n^{1/n}
        } \to
        0
    \end{align*}
    e quindi la serie converge.
}


\sexample{}{
    Verificare per quali valori di \(\alpha>0\),
    \[
        f(x) = \begin{cases}
            {|x|}^\alpha \cos\left(\frac{1}{x}\right) & x \neq 0 \\
            0 & x=0
        \end{cases}
    \]
    è derivabile in \(x=0\). \(f\) è continua per ogni \(\alpha>0\) per il teorema dei due carabinieri.
    Infatti, \(|f(x)| \leq {|x|}^\alpha \to 0 = f(0)\) per \(x\to 0\).
    Il rapporto è dato da
    \[
        \frac{f(x) - f(0)}{x-0} = {|x|}^{\alpha - 1} \cdot \text{sgn}(x) \cdot \cos\frac{1}{x}
        \to \begin{cases}
            0 & \alpha > 1 \\
            \nexists & \alpha \leq 1
        \end{cases}
    \]
    Per ogni \(\alpha\), \(f\) è derivabile in \(\mathbb{R} \backslash \{0\}\) e \begin{align*}
        f'(x) = D[{|x|}^\alpha \cos\frac{1}{x}]
        = {|x|}^{\alpha - 2} \left\{
            \alpha {|x|} \text{sgn} x \cos\frac{1}{x} + \sin\frac{1}{x}
        \right\}
    \end{align*}
    Quindi \(f'(0)\) esiste per \(\alpha>1\), \(f'\) è continua in \(x=0\) se e solo se \(\alpha > 2\).
}

\sexercise{}{
    \begin{align*}
        \lim_{x\to 0} \frac{x^2 - x\ln(1 + x) + 3x}{\sin(x) + 3x^2}
        &= \lim_{x\to 0} \frac{x(x - \ln(1 + x) + 3)}{x(\frac{\sin x}{x} + 3x)} \\
        &= \lim_{x\to 0} \frac{0-0+3}{1 + 0} = 3
    \end{align*}
}

\sexercise{}{
    \begin{align*}
        \lim_{x\to 0} \frac{2\sin (x) \left(e^{x^2} - 1\right)}{(1-\cos(x)){\left[\ln(1 + \sqrt{x})\right]}^2}
        &= \lim_{x\to 0} \frac{
            2x(1 + o(1))x^2(1 + o(1))
        }{
            \frac{x^2}{2}(1 + o(1)) \sqrt{x}(1 + o(1))
        } \\
        &= \lim_{x\to 0} \frac{4(1 + 2o(1) + o^2(1))}{1 + o^2(1) + 3o(1) + o^3(1)} = 4
    \end{align*}
}

\sexercise{}{
    Considera \(n \in \mathbb{R}\)
    \begin{align*}
        \lim_{x\to 0^+} \frac{2}{x^2} \left[
            e^{-x^n} - 1 + \ln\left(\cos(x^n)\right)
        \right]
        &= \lim_{x\to 0^+} \frac{2}{x^2} \left[
            -x^n(1 + o(1)) + \ln\left(1 + \cos(x^n) - 1\right)
        \right] \\
        &= \lim_{x\to 0^+} \frac{2}{x^2} \left[
            -x^n(1 + o(1)) + \ln\left(1 - \frac{x^{2n}}{2}(1 + o(1))\right)
        \right] \\
        &= \lim_{x\to 0^+} \frac{2}{x^2} \left[
            -x^n(1 + o(1)) -\frac{x^{2n}}{2}(1 + o(1))
        \right] \\
        &= \lim_{x\to 0^+} - \frac{2}{x^2} \left[
            x^n \left(1 + \frac{x^n}{2}\right)(1 + o(1))
        \right]
    \end{align*}
    Nel caso \(n>0\) abbiamo
    \[
        \begin{cases}
            0 & n>2 \\
            -2 & n=2 \\
            -\infty & 0 < m < 2
        \end{cases}
    \]
    Nel caso \(n=0\) il limite è \(-\infty\), mentre se \(n<0\) il limite
    non è ben definito.
}

\sexercise{}{
    \begin{align*}
        \lim_{x\to 0} \ln(\cos x + x^2) \frac{e^{-\frac{x^2}{2}} + 1}{1-e^{-x^2}}
    \end{align*}
    Sostituendo troviamo la forma di indeterminazione \(\frac{0}{0}\).
    \begin{align*}
        \lim_{x\to 0} ln(1 + \cos x - 1 + x^2) \frac{e^{-\frac{x^2}{2}} + 1}{\left(1-e^{-\frac{x^2}{2}}\right)\left(1+e^{-\frac{x^2}{2}}\right)}
        &= \lim_{x\to 0} \frac{
            \ln\left(
                1 - \frac{x^2}{2}(1 + o(1)) + x^2
            \right)
        }{
            \frac{x^2}{2}(1 + o(1))
        } = 1
    \end{align*}
}

\sexercise{}{
    \begin{align*}
        \lim_{x\to \frac{\pi}{4}} {\left(
            \sin(2x)
        \right)}^{\frac{1}{\ln\left(1 + \cos\left(x + \frac{\pi}{4}\right)\right)}}
    \end{align*}
    Sostituendo troviamo la forma di indecisione \(1^\infty\).
    Facciamo un cambio di variabile \(t = x-\frac{\pi}{4}\)
    \begin{align*}
        \lim_{x\to 0} {\left(
            \sin\left(2\left(t + \frac{\pi}{4}\right)\right)
        \right)}^{\frac{1}{\ln\left(1 + \cos\left(x + \frac{\pi}{2}\right)\right)}}
        &= \lim_{t\to 0} \exp\left\{\frac{1}{\ln(1 - \sin t)} \cdot \ln(\cos(2t))\right\} \\
        &= \lim_{t\to 0} \exp\left\{\frac{\ln(1 + \cos(2t) - 1)}{\ln(1 - \sin t)}\right\} \\
        &= \lim_{t\to 0} \exp\left\{\frac{\ln(1 - 2t^2)}{1 - t}\right\} \\
        &= \lim_{t\to 0} \exp\left\{\frac{-2t}{t}\right\} = e^0 = 1
    \end{align*}
}

\sexercise{}{
    Studia la continuità di
    \[
        f(x) = {\left(\ln|x|\right)}^{-1}
    \]
}

\sexercise{}{
    \[
        \integral[\frac{1}{e^x + 1}][x]
    \]
    Allora sostituiamo \(t = e^x\) quindi
    \begin{align*}
        \integral[\frac{1}{e^x + 1}][x] &= 
        \integral[\frac{1}{t}\frac{1}{1+t}][t] \\
        &= \integral[\frac{1}{t} - \frac{1}{t+1}][t] \\
        &= \log|t| - \log|t + 1| + C \\
        &= \log\left(\frac{e^x}{e^x + 1}\right) + C
    \end{align*}
}

\sexercise{}{
    \[
        \integral[\frac{1}{e^x + 2 + e^{-x}}][x]
    \]
    Allora abbiamo
    \begin{align*}
        \integral[\frac{1}{e^x + 2 + e^{-x}}][x] &= \integral[\frac{e^x}{e^{2} + 2e^x + 5}][x]
    \end{align*}
    Sostituiamo \(t = e^x\) e otteniamo
    \begin{align*}
        \int \frac{dt}{t^2 + 2t + 5} &=\frac{1}{4}\int \frac{dt}{1 + {\left(\frac{t+1}{2}\right)}^2} \\
        &= \frac{1}{2} \arctan\left\{\frac{e^x + 1}{2}\right\}
    \end{align*}
}

\sexercise{}{
    \[
        \integral[x^{-\frac{3}{2}} \arctan(x^{-\frac{1}{2}})][x]
    \]
    Cominciamo sostituendo \(t=x^{-\frac{1}{2}}\) e proseguiamo per parti
    \begin{align*}
        \integral[x^{-\frac{3}{2}} \arctan(x^{-\frac{1}{2}})][x] &=
        \integral[t^3 \arctan(t)(-2t^{-3})][t] \\
        &= -2\integral[\arctan(t) \cdot 1][t] \\
        &= -2 \left[
            t\arctan(t) - \integral[\frac{t}{1 + t^2}][t]
        \right] \\
        &= -2t\arctan(t) + \integral[\frac{2t}{1 + t^2}][t]
    \end{align*}
    Sostituiamo \(v = 1 + t^2\)
    \begin{align*}
        \integral[\frac{2t}{1 + t^2}][t] &=
        \int \frac{dv}{v} = \log(1 + t^2) + C
    \end{align*}
    Allora il risultato è
    \[
        -2x^{-\frac{1}{2}}\arctan(x^{-\frac{1}{2}}) +
        \log(1 + x^{-1}) + C
    \]
}

\sexercise{}{
    \begin{align*}
        \integral[\log(x^2 + 4)][x] &=
        x\log(x^2 + 4) - 2\integral[\frac{x^2}{x^2 + 4}][x] \\
        &= x\log(x^2 + 4) - 2 \integral[1 - \frac{4}{x^2 + 4}][x] \\
        &= x\log(x^2 + 4) - 2x + \arctan\left(\frac{x}{2}\right) + C
    \end{align*}
}

\sexercise{}{
    \begin{align*}
        \integral[\frac{1}{x^2 + x + 1}][x] &= \integral[\frac{1}{{\left(x + \frac{1}{2}\right)}^2 +\frac{3}{4}}][x]
    \end{align*}
}

\sexercise{}{
    \begin{align*}
        \integral[\frac{1}{x\sqrt{x + 1}}][x]
    \end{align*}
    Sostituiamo \(x + 1 = t^2\)
    \begin{align*}
        \integral[\frac{1}{x\sqrt{x + 1}}][x]
        &= \integral[\frac{2t}{t\cdot(t^2 - 1)}][t] \\
        &= 2 \int \frac{dt}{(t-1)(t+1)} \\
        &= \int \frac{dt}{t-1} + \int \frac{dt}{t+1} \\
        &= \log|t-1|-\log|t+1| + C \\
        &= \log\left|\frac{\sqrt{x+1} - 1}{\sqrt{x+1} + 1}\right| + C
    \end{align*}
}


\sexercise{}{
    Procediamo per parti
    \begin{align*}
        \integral[x^{-\frac{1}{2}} \arctan(x^{-\frac{1}{2}})][x] &= 
        2x^{\frac{1}{2}}\arctan(x^{-\frac{1}{2}}) - \integral[\frac{2x^{\frac{1}{2}}(-\frac{1}{2}x^{-\frac{3}{2}})}{1 + x^{-1}}][x] \\
        &= 2x^{\frac{1}{2}} \arctan(x^{-\frac{1}{2}}) + \log|1 + x| + C
    \end{align*}
}

\sexercise{}{
    \begin{align*}
        \integral[\frac{\arcsin(\sqrt{x})}{\sqrt{x}}][x]
    \end{align*}
    Sostituiamo \(t = \sqrt{x}\) e poi procediamo per parti
    \begin{align*}
        2 \integral[\arcsin(t)][t] &=
        t\arcsin(t) - 2\integral[\frac{t}{\sqrt{1-t^2}}][t]
    \end{align*}
    Sostituiamo ora \(v = 1-t^2\)
    \begin{align*}
        t\arcsin(t) + \integral[v^{-\frac{1}{2}}][v]
        &= \sqrt{x}\arcsin(\sqrt{x}) + 2\sqrt{1 - x} + C
    \end{align*}
}

\sexercise{}{
    \[
        \int \frac{dx}{e^{2x} + 3e^x + 2}
    \]
    Sostituiamo \(t = e^x\)
    \begin{align*}
        \int \frac{dt}{t(t+1)(t+2)}
    \end{align*}
}

\sexercise{}{
    \[
        \integral[\frac{x + 2}{{(x + 1)}^{\frac{5}{2}}}][x]
    \]
    Sostituiamo \(t = x + 1\).
}

\sexercise{}{
    \[
        \integral[\frac{1 + e^x}{1 + e^{2x}}][x]
    \]
    Sostituiamo \(t = e^x\).
    \begin{align*}
        \integral[\frac{1 + t}{1 + t^2}\frac{1}{t}][t] &=
        \integral[\frac{1}{1 + t^2} \frac{1}{t}][t] + \int \frac{dt}{1 + t^2} \\
        &= \arctan(t) + \integral[\frac{-t}{1 + t^2} + \frac{1}{t}][t] \\
        &= -\frac{1}{2} \ln(1 + e^{2x}) + \ln(e^{x}) + \arctan(e^x)
    \end{align*}
}

\end{document}
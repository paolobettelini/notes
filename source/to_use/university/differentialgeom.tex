\documentclass[a4paper]{article}

\usepackage{amsmath}
\usepackage{amssymb}
\usepackage{stellar}
\usepackage{parskip}
\usepackage{fullpage}
\usepackage{wrapfig}

\title{Differential geometry}
\author{Paolo Bettelini}
\date{}

\begin{document}

\maketitle
\tableofcontents

\section{Differential geometry}

% Aggiornare quella in analisi multivariabile con un intervallo generico non degenere1
\sdefinition{Curva parametrizzata di classe \(\mathcal C^k\)}{
    Una \emph{curva parametrizzata} su \(\realnumbers^n\)
    è una funzione \(\sigma \colon I \fromto \realnumbers^n\)
    con \(I\) intervallo reale (con più di un punto)
    e con \(\sigma\) di classe \(\mathcal C^k\).
}

Le derivate sono generalmente definite per l'interno dell'intervallo.
La classe \(\mathcal C^k\) è definita su un estremo dell'intervallo, per esempio \([a,b)\),
se \(\sigma\) può essere prolungata a funzione \(\tilde \sigma\)
di classe \(\mathcal C^k\) su \((a- \varepsilon, b) = I_\varepsilon\),
cosicché \(a \in I_\varepsilon\).
In tal caso \(\sigma'(a)\) è ben definito come \(\tilde \sigma'(a)\).

Per parametrizzare un segmento fra \(A\) e \(B\)
possiamo scrivere \(P = (1-t)A + tB\) oppure \(P = \lambda A + \mu B\)
dove \(\lambda + \mu = 1\), che viene detta combinazione lineare affine.
Se richiediamo anche che \(\lambda,\mu \geq 0\) allora si chiamano combinazioni convesse.
Quest'ultima si generalizza al \(k\)-simplesso.

% mettere questo in omologia
\sdefinition{\(k\)-simplesso in \(\realnumbers^n\)}{
    Sia \(n \geq k\).
    Consideriamo \(k + 1\) punti \(P_0, \cdots, P_k \in \realnumbers^n\)
    in posizione lineare generale, cioè
    non appartenenti ad un sottospazio affine di dimensione minore o uguale a \(r-1\).
    Allora il \(r\)-simplesso di tali vertici è
    \[
        \Delta^n = \left\{
            \sum_{i=0}^r \lambda_i P_i
            \suchthat
            \lambda_j \geq 0 \land \sum \lambda_i = 1
        \right\}
    \] 
}

\sdefinition{Sottospazi affini in forma parametrica}{
    Sia \(U \subseteq \realnumbers^n\) sottospazio vettoriale di dimensione \(r \leq n\)
    e \(P_0 \in \realnumbers^n\).
    Allora
    \[
        L = \left\{
            P_0 + u \suchthat u \in U
        \right\} = P_0 + U
    \]
    è il sottospazio affine
    di direzione \(U\) e passante per \(P_0\).
}

Questi sono una generalizzazione della retta.
Fissata una base \(u_1, \cdots, u_r\) di \(U\),
otteniamo la parametrizzazione lineare
\[
    L = \left\{
        P_0 + \sum_{i=1}^r \lambda_i u_i
        \suchthat \lambda_i \in \realnumbers
    \right\}
\]
cioè \(\sigma \colon \realnumbers^r \fromto \realnumbers^n\)
dato da
\[
    \sigma(\lambda_1, \cdots, \lambda_r) = P_0 + \sum_{i=1}^r \lambda_i u_i
\]
è una parametrizzazione completa di \(L\).
Chiaramente \(P_0\) può essere arbitrariamente scelto fra i punti di \(L\)
e la base può essere scelta arbitrariamente fra i punti di \(U\).
Il passaggio alle coordinate affini
\begin{align*}
    \sigma(\lambda_1, \cdots, \lambda_r)
    &= P_0 + \sum_{i=1}^r \lambda_i u_i \\
    &= P_0 - \left( \sum_{i=1}^r \lambda_i P_0 \right) + \sum_{i=1}^r t_i(\underbrace{P_0 + \lambda_i}_{\in L})
\end{align*}
Chiamando \(P_i = P_0 + u_i\) possiamo scrivere
\[
    \sigma(\lambda_1, \cdots, \lambda_r) = \sum \lambda_0 P_0 + \sum_{i=0}^r \lambda_i P_i
\]
cosicché
\[
    \sum_{i=0}^r \lambda_i = 1
\]
che sono tanti parametri.
Così si possono ottenere i punti all'interno dei simplessi con i \(\lambda_i \geq 0\).

% Da mettere in analisi assieme agli altri
\sdefinition{Baricentro di un insieme di punti}{
    Dati \(P_0, \cdots, P_n \in \realnumbers^n\) arbitrari, il loro baricentro
    è dato dalla combinazione affine
    \[
        B = \frac{1}{n+1} \sum_{i=0}^n P_i
    \]
}
che è la media aritmetica.

\stheorem{Teorema di geometria euclidea}{
    Il baricentro di un triangolo
    è il punto di incontro delle rette
    mediane, e divide ciascuna mediana in due parti una lunga \(\frac13\) dell'altra.
}

Fisicamente, il motivo per cui si incontrano
è dato dal fatto che il baricentro di un segmento è il suo punto medio.
Il baricentro del segmento di una delle mediane è una media pesata a \(\frac23\).

\sproof{}{
    Sia \(B = \frac13(P_0 + P_1 + P2)\).
    Consideriamo la retta \(P_0B = \{(1-t)P_0 + tB\}\), cioè
    \[
        P_0B = \left(1 - t + \frac{t}{3}\right)P_0 - \frac{t}{3}P_1 + \frac{t}{3}P_2
    \]
    che sta sul lato \([P_1, P2]\) se e solo se \(1-t + t/3 = 0\),
    cioè \(t = 1/2\). Quindi,
    \begin{align*}
        \left(1 - t + \frac{t}{3}\right)P_0 + \frac{t}{3}P_1 + \frac{t}{3}P_2
        = \frac{1}{2}P_1 + \frac{1}{2}P_2
    \end{align*}
    per \(t=1/2\), che è il punto medio \(M_{1,2}\) fra \(P_1\) e \(P_2\).
    Analogamente lo stesso vale per ogni mediana \(P_i M_{j,k}\).
    Quindi le tre mediane si incontrano nel baricentro.
    Inoltre, vediamo come scrivere \(B\) come combinazione affine fra \(P_0\)
    e \(M = M_{1,2}\). Abbiamo
    \begin{align*}
        M &= \frac{1}{2}P_1 + \frac{1}{2}P_2 = \restr{
            (1-t)P_0 + tB
        }{t=\frac32} \\
        &= -\frac12 P_0 + \frac32 B
    \end{align*}
    Da cui ricaviamo \(B = \frac13 P_0 + \frac23 M\).
}

Lo stesso teorema vale per i simplessi di dimensione superiore.
Per il tetraedro la mediana parte dal baricentro di ogni faccia, cioè
il baricentro del triangolo. Quindi abbiamo una sorta di successione di punti di baricentro,
e la mediana viene spezzata in due parti con rapporto \(\frac14\).

\sdefinition{}{
    Due curve parmetrizzate \(\sigma_1 \colon I \fromto \realnumbers^n, \sigma_2 \colon J \fromto \realnumbers^n\) di classe \(\mathcal C^h, \mathcal C^k\)
    rispettivamente si dicono \emph{equivalenti}
    se esiste un diffeomorfismo \(h \colon J \fromto I\) tale che \(\sigma_2 = \sigma_1 \circ h\).
}

Il diffeomorfismo è un omeomorfismo tale che \(h,h^{-1}\) siano di classe \(\mathcal C^{\max\{h,k\}}\).
Quindi se \(h,k \geq 1\) allora \(h,h^{-1}\) sono derivabili e quindi
\(h'(t) \neq 0\) in quanto
\[
    \frac{dh^{-1}}{ds} = \restr{\frac{1}{h'(t)}}{s = h(t)}
\]
ne segue quindi che per \(t \in J\) abbiamo \(h'(t) > 0\) o \(h'(t) < 0\),
cioè è monotona crescente o decrescente.

\sexample{Tutte le parametrizzazioni lineari di una retta sono equivalenti}{
    \(\sigma_1(s) = P_1 + sv\) e \(\sigma_2(t) = P_2 + tw\)
    parametrizzano la stessa retta se e solo se
    \[
        \begin{cases}
            P_2 = P_1 + s_0v \\
            w = \rho v
        \end{cases}
    \]
    In tal caso \begin{align*}
        \sigma_2(t) &= P_1 + s_0 v + t\rho v \\
        &= P_1 + (s_0 + t\rho)v \\
        &= \sigma_1(s_0 + t\rho) = \sigma_1(h(t))
    \end{align*}
    con \(h(t) = s_0 + t\rho\) e \(\rho \neq 0\).
}

D'ora in poi tendiamo a lavorare con curve liscie.

\sdefinition{}{
    Una curva \(\sigma \colon I \fromto \realnumbers^n\) liscia si dice
    \emph{regolare} se \(\sigma' \neq 0\).
}

Una curva regolare ha una forma arrotondata ovunque, in quanto esiste sempre una retta tangente.
Una curva \((x, f(x))\) è sempre regolare.

\sexample{Parametrizzazioni standard di una circonferenza}{
    Sia \(C\) la circonferenza di centro \(P_0 = (x_0, y_0)\) e raggio \(R\).
    Le parametrizzazioni standard sono
    \[
        \sigma(t) = P_0 + R(\cos(\omega t), \sin(\omega t))
    \]
    con \(\omega \in \realnumbers^*\).
    Risulta regolare.
}

\sdefinition{Curva a tratti}{
    Se esiste un ricoprimento di intervallo
    \[
        I \subseteq (-\infty, t_1] \union \left(\bigcup [t_i, t_{i+1}]\right) \union [t_k, +\infty)
    \]
    tale per cui le restrizioni a questi intervalli sono regolari o lisci etc.
}

In una curva liscia le derivate esistono gli estremi, quindi se a tratti, sui punti di cuspide possiamo
avere derivata destra e sinistra usando le derivate dei pezzi di curva prima o dopo rispettivamente.

\sdefinition{}{
    Una curva liscia a tratti è detta regolare se le sue derivate destre e simistri sono non nulli.
}

\sdefinition{Punto angoloso e cuspidi per una curva}{
    Se \(\sigma_{-}'(t) \neq \sigma_{+}'(t)\) e sono enrambi non nulli,
    allora \(\sigma(t)\) è detto:
    \begin{enumerate}
        \item punto angoloso se \(\sigma_{-}'(t), \sigma_{+}'(t)\) sono linearmente indipendenti
        \item cuspide se \(\sigma_{-}'(t) = \lambda \sigma_{+}'(t)\) con \(\lambda < 0\).
    \end{enumerate}
    Disegno.
}

\sdefinition{Angolo di un punto angoloso per curve}{
    Definiamo l'angolo
    \[
        \varepsilon(t) \triangleq \begin{cases}
            +\arccos \left(
                \frac{
                    \langle \sigma_{-}'(t), \sigma_{+}'(t)\rangle
                }{
                    ||\sigma_{-}'(t)|| \cdot ||\sigma_{-}'(t)||
                }
            \right) & \det(\sigma_{-}'(t), \sigma_{+}'(t)) > 0 \\
            -\arccos \left(
                \frac{
                    \langle \sigma_{-}'(t), \sigma_{+}'(t)\rangle
                }{
                    ||\sigma_{-}'(t)|| \cdot ||\sigma_{-}'(t)||
                }
            \right) & \det(\sigma_{-}'(t), \sigma_{+}'(t)) < 0
        \end{cases}
    \]
}

Cioè prendiamo il segno positivo se le due derivate formano una base orientata positivamente in \(\realnumbers^2\),
e negativo altrimenti. È il verso della minima rotazione per portare la derivata sinistra sulla derivata destra.

\sdefinition{Curva semplice}{
    Sia \(\sigma \colon [a,b] \fromto \realnumbers^2\) liscia a tratti
    è detta semplice, o chiusa di Jordan, se
    \begin{enumerate}
        \item \(\sigma(a) = \sigma(b)\)
        \item \(\sigma'(a) = \sigma'(b)\)
        \item sono iniettivi
        \[
            \restr{\sigma}{(a, b]}, \quad \restr{\sigma}{[a, b)}
        \]
    \end{enumerate}
}
La seconda intersezione è quella di auto intersecarsi.
\stheorem{Teorema della tangenti di Hopf}{
    Se \(\sigma \colon [a,b] \fromto \realnumbers^2\) è curva semplice e regolare
    avente solo punti angolosi, allora il vettore
    \(\sigma'(t)\) percorre una rotazione complessiva
    \(\pm 2\pi\).
}

\scorollary{}{
    La somma degli angoli orientati estermi di un poligono chiuso è sempre
    \(\pm 2\pi\).
}

\sexercise{}{
    Usare questo corollario per mostrare che la somma degli angoli interni è \((n-2)\pi\).
}

\sexample{Biliardo rettangolare}{
    Consideriamo un biliardo senza attriti di lati \(a,b\).
    Sia \(\mu\) il coefficiente angolare della direzione del tiro.
    \begin{enumerate}
        \item se \(\mu\) è commensurabile con
        \(b/a\), allora la traiettoria è periodica.
        \item se \(\mu\) non è commensurabile con \(b/s\) allora la traiettoria è densa.
    \end{enumerate}
}

\end{document}
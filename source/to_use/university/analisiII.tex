\documentclass[a4paper]{article}

\usepackage{amsmath}
\usepackage{amssymb}
\usepackage{stellar}
\usepackage{parskip}
\usepackage{fullpage}
\usepackage{wrapfig}
\usepackage{tikz}

\usetikzlibrary{arrows}
\usetikzlibrary{decorations.pathreplacing}
\usetikzlibrary{cd}

\title{Analisi II}
\author{Paolo Bettelini}
\date{}

\begin{document}

\maketitle
\tableofcontents

\section{Bolle}

L'insieme vuoto e l'insieme \(X\) sono aperti e chiusi.

\sproposition{}{
    L'unione di aperti non numerabile è aperta, mentre l'intersezione è aperta solo se finita.
}

\sproof{}{
    Per dimostrare quest'ultima lo facciamo su due insiemi e il resto è per induzione.
    Prendiamo un punto nell'interezione e prendiamo le due bolle dentro gli insiemi centrate
    nel punto. Siccome hanno lo stesso centro la loro intersezione è sempre una bolla di raggio
    il minore fra i due.
}

La metrice discreta può generare una bolla che è un singoletto.

\sproposition{}{
    L'unione di chiusi finiti è chiusa. L'intersezione qualsiasi è chiusa.
}

Ogni singoletto è chiuso. Per dimostrarlo mostriamo che nel complementare
esiste una bolla che non interseca il punto (vero per proprietà di Hausdorff).

Tutti i punti di accumulazione sono dei punti aderenti.
Tutti i punti di un sottoinsieme sono aderenti per il sottoinsieme.
Ogni punto o è di accumulazione o è isolato.

Se \(x_0\) è aderente ad E, \(x_0\) può essere un punto di E oppure no.
Se \(x_0\) è punto di accumulazione per E, in ogni bolla centrata in \(x_0\) cadono inifniti punti.

\sproposition{}{
    \(E^\circ\) è aperto. \(E\) è aperto se e solo se \(E = E^\circ\).
    \(E^\circ\) è il più grande aperto contenuto in \(E\).
    
    \(\overline{E}\) è chiuso. \(E\) è chiuso se e solo se \(E = \overline{E}\).
    La chiusura di \(E\) è il più piccolo chiuso contenente \(E\).
}

\sproof{per l'interno}{
    Dimostriamo che \(E^\circ\) è aperto. Sia \(x_0 \in E^\circ\).
    un punto interno ad \(E\), quindi esiste una bolla centrata in tale punto che è contenuta in \(E\).
    Prendiamo un altro punto \(y\) in questa bolla. Possiamo costruire una inner bolla centrata in \(y\)
    con un raggio sufficientemente piccolo da rimanere nella bolla più grande. Quindi il punto \(y\)
    è a sua volta interno, quindi tutta la bolla centrata in \(x_0\) è in \(E^\circ\) e quindi è aperto.
    
    Dimostriamo ora che se \(E\) è aperto allora \(E=E^\circ\) (l'altra implicazione è ovvia).
    Per fare ciò dimostriamo che \(E^\circ\) è il più grande aperto in \(E\).
    Osserviamo che \(E^\circ\) fa parte della famiglia degli aperti di \(X\)
    contenuti in \(E\). Sia \(A\) un aperto contenuto in \(E\). VOglio dimostrare che \(A \subseteq E^\circ\).
    Sia \(x_0 \in A\). \(A\) èunione di bolle quindi esiste unr aggio tale che la
    bolla centrata in \(x_0\) di tale raggio è contenuta in \(A\) che è contenuto in \(E\).
    Quindi, \(x_0\) è interno ad \(E\) e \(x_0 \in E^\circ\) e \(A \subseteq E^\circ\).
    Supponiamo ora che \(E\) sia aperto. Allora \(E\) fa parte della famiglia degli aperti
    di \(X\) contenuti in \(E\). Devo avere \(E \subseteq E^\circ\).
    Dato che \(E^\circ \subseteq E\) allora \(E^\circ = E\).
}

% lezione 2

Dire che un insieme è dentro in un altro significa dire che la sua chiusura coincide con l'insieme.
Tipo la chiusura di Q è R quindi Q è denso in R.


\sdefinition{Limitato}{
    Se è contenuto in una bolla
}

\sdefinition{Diametro}{
    è il sup della metrica su tutte le coppie.
}

\sdefinition{Ricoprimento}{
    Sia \(E\) un sottoinsieme di uno spazio metrico \(X\). Una famiglia
    \[
        \{G_\alpha\}_{\alpha \in A}
    \]
    è un ricoprimento apert di \(E\) se
    \[
        E \subseteq \bigcup_{\alpha \in A} G_\alpha
    \]
}

\sdefinition{Sottoricoprimento}{
    Un Sottoricoprimento di 
    \[
        \{G_\alpha\}_{\alpha \in A}
    \]
    è una sottofamiglia di \(G_\alpha\)
    tale che continua a ricoprire. Cioè ne scarto alcuni ma deve comunque rimanere
    una copertura.
}

\sdefinition{Compatto}{
    Uno spazio metrico \(X\) è compatto se ogni ricoprimento aperto di \(E\)
    ammette un sottoricoprimento finito.
}

Ogni insieme finito è compatto.


\stheorem{}{
    Sia  \(X\) uno spazio metrico e \(E\) un sottoinsieme di \(X\) compatto.
    \begin{enumerate}
        \item \(E\) è limitato;
        \item \(E\) è chiuso;
        \item Ogni sottoinsieme infinito di \(E\) ha almeno un punto di accumulazione in \(E\).
    \end{enumerate}
}

\sproof{}{
    \begin{enumerate}
        \item Consideriamo \(\{B_1(x) \,|\, x\in E\}\) che è un ricoprimento aperto di \(E\).
        Siccome \(E\) è compatto esiste un sottoricoprimento finito aperto di \(E\), ossia
        \(x_1, \ldots, x_n \in E\) tali che
        \[
            E \subseteq \bigcup_{i=1}^n B_1(x_i)
        \]
        Posto
        \[
            R = 1 + \max_{i=1,\ldots,n} d(x_i, x_1)
        \]
        Allora la bolla di raggio \(R\) centrata in \(x_1\) contiene \(E\), quindi \(E\) è limitato.
        \item Supponiamo che non sia chiuso. Allora esiste \(y\in E'\) ma \(y\notin E\).
        Vogliamo costruire un ricoprimento aperto di \(E\) che non ammette sottoricoprimento finito.
        Sia \(r(x) = \frac{1}{2} d(x,y)\) per ogni \(x\in X\).
        Se \(x\in E\) allora \(r(x) > 0\) perchè \(y\notin E\).
        Abbiamo il ricoprimento
        \[
            \{B_{r(x)}(x) \,|\, x\in E\}
        \]
        Ma per la compattezza esisterebbe un sottoricoprimento finito, cioè \(x_1, \ldots, x_n \in E\) tali che
        \[
            E \subseteq \bigcup_{i=1}^n B_{r(x_i)}(x_i)
        \]
        Sia ora \(R = \min_{i=1,\ldots,n} r(x_i)\). Allora \(R > 0\) e la bolla \(B_R(y)\)
        non interseca nessuna delle \(B_{r(x_i)}(x_i)\), assurdo poiché \(y\) è punto di accumulazione.
        \item Sia \(F\) un sottoinsieme infinito di \(E\). Supponiamo che \(F\) non abbia punti di accumulazione in \(E\).
        Allora ogni punto di \(E\) ha una bolla che interseca \(F\) in al più un punto.
        Queste formano un ricoprimento aperto di \(E\).
        Ma se esistesse un sottoricoprimento finito, \(F\) sarebbe finito, assurdo.
    \end{enumerate}
}

\sproposition{}{
    Sia \(E \subseteq X\) compatto. Se \(F \subseteq E\) è chiuso allora \(F\) è compatto.
}
\sproof{}{
    Sia \(\{G_\alpha\}_{\alpha \in A}\) un ricoprimento aperto di \(F\).
    Dobbiamo aggiungere degli insiemi aperti per coprire il resto.
    Siccome \(F\) è chiuso, \(X \setminus F\) è aperto.
    Quindi \(\{G_\alpha\}_{\alpha \in A} \cup \{X \setminus F\}\) è un ricoprimento aperto di \(E\).
    Per la compattezza di \(E\) esiste un sottoricoprimento finito, che escludendo \(X \setminus F\)
    è un sottoricoprimento finito di \(F\).
}

% ESERCIZIO
Se \(F\subseteq X\) è chiuso, ed \(E\subseteq X\) è compatto, allora \(F \cap E\) è compatto.

\stheorem{Teorema dell'intersezione finita}{
    Sia \(\{E_\alpha\}_{\alpha \in A}\) una famiglia di compatti tale che
    ogni intersezione finita è non vuota. Allora
    \[
        \bigcap_{\alpha \in A} E_\alpha \neq \emptyset
    \]
}

\sproof{}{
    Supponiamo che l'intersezione sia vuota. Allora
    e sa \(E_{\overline{\alpha}}\) un compatto finito nella famiglia.
    % TODO
}

\scorollary{caso particolare}{
    Sia \(\{E_n\}_{n\in \mathbb{N}}\) una famiglia di compatti tale che
    \[
        E_{n+1} \subseteq E_n
    \]
    Allora
    \[
        \bigcap_{n\in \mathbb{N}} E_n \neq \emptyset
    \]
}

\stheorem{Teorema di Heine-Borel}{
    Sia \(E \subseteq R^n\) con la metrica euclidea. Allora \(E\) è compatto se e solo se
    \(E\) è chiuso e limitato.
}

\slemma{}{
    Sia \(\{I_k\}_{k\in \mathbb{N}}\) una famiglia di intervalli
    \(I_k = [a_k, b_k]\) tali che \(I_k \supseteq I_{k+1}\).
    Allora 
    \[
        \bigcap_{k\in \mathbb{N}} I_k \neq \emptyset
    \]
}

\sproof{}{
    Gli intervalli sono annidati, quindi \(a_k\) è crescente e \(b_k\) è decrescente e \(a_k \leq b_k\).
    In particolare \(a_k \leq b_i\). Consideriamo l'insieme \(E = \{a_k \,|\, k\in \mathbb{N}\}\).
    \(E\) è limitato superiormente, e ammette supremum \(x\).
    Per definizione \(x \geq a_k\). Ma \(a_k \leq b_i\) per tutte le \(i\).
    Quindi, \(x \leq b_i\) per ogni \(i\). Allora
    \[
        x \in I_n \implies x \in \bigcap I_k
    \]
}

\sdefinition{}{
    Siano \(a,b \in \mathbb{R}^n\) con \(a_i < b_i\) per ogni \(i=1,\ldots,n\).
    Un rettangolo chiuso è il prodotto cartesiano
    \[
        [a_1, b_1] \times [a_2, b_2] \times \ldots \times [a_n, b_n]
    \]
    che indichiamo con \([a,b]\).
}

\slemma{}{
    Sia \(\{R_k\}_{k\in \mathbb{N}}\) una famiglia di rettangoli chiusi
    tali che \(R_k \supseteq R_{k+1}\) per ogni \(k\).
    Allora
    \[
        \bigcap_{k\in \mathbb{N}} R_k \neq \emptyset
    \]
}

\sproof{}{
    Siccome
    \[
        R_k = I_{k,1} \times I_{k,2} \times \ldots \times I_{k,n}
    \]
    possiamo applicare il primo lemma e quindi
    \[
        \exists y_i \in \bigcap_{k\in \mathbb{N}} I_{k,i}
    \]
    Il punto \(y = (y_1, \ldots, y_n)\) è in ogni \(R_k\).
}

\slemma{Lemma 3}{
    In \(\mathbb{R}^n\) con la metrica euclidea
    ogni rettangolo è compatto.
}

\sproof{Lemma 3}{
    Sia \(R = [a,b]\) un rettangolo e supponiamo che non sia compatto.
    Sia \(\{G_\alpha\}_{\alpha \in A}\) un ricoprimento aperto di \(R\)
    che non ammette sottoricoprimento finito.
    Vogliamo adesso dimezzare ambo i lati (quindi \(n\) tagli).
    Abbiamo adesso \(2^n\) rettangoli.
    \[
        [a_i, b_i] = [a_i, c_i] \cup [c_i, b_i], \quad
        c_i = \frac{a_i + b_i}{2}
    \]
    Il diametro di \(R\) è \(||b-a||\).
    Il diametro di ogni rettangolo ottenuto è la metà.
    Almeno uno di questi rettangoli ha la proprietà di non ammettere sottoricoprimento finito.
    Lo chiamiamo \(R_1\).
    Iterando il procedimento otteniamo una successione di rettangoli
    \[
        R \supseteq R_1 \supseteq R_2 \supseteq \ldots
    \]
    con diametro che tende a zero e che non ammettono sottoricoprimento finito, il diametro
    di \(R^k\) è dato da \(\frac{1}{2^k} ||b-a||\).
    Per il lemma precedente esiste \(x \in \bigcap_{k\in \mathbb{N}} R_k\).
    Siccome \(R_k \subseteq R\) per ogni \(k\), \(x \in R\).
    Siccome \(\{G_\alpha\}_{\alpha \in A}\) è un ricoprimento di \(R\),
    esiste \(\alpha_0 \in A\) tale che \(x \in G_{\alpha_0}\).
    \(G_{\alpha_0}\) è aperto, quindi esiste \(r > 0\) tale che
    \(B_r(x) \subseteq G_{\alpha_0}\).
    Scegliamo \(k\) sufficientemente grande tale che \(2^{-k} ||b-a|| < r\).
    Ma il diametro di \(R_k\) è minore di \(r\), quindi \(R_k \subseteq B_r(x)\).
    Quindi \(R_k \subseteq G_{\alpha_0}\), assurdo perchè \(R_k\) non ammette sottoricoprimento finito.
}

\sproof{Heine-Borel}{
    Dobbiamo dimostrare solo che se \(E\) è chiuso e limitato allora è compatto.
    Siccome \(E\) è limitato esiste \(M\) tale che \(||x|| < M\) per ogni \(x \in E\).
    Quindi,
    \[
        E \subseteq [-M, M] \times [-M, M] \times \ldots \times [-M, M] = R
    \]
    \(E\) è un chiuso contenuto in un compatto, quindi è compatto.
}

\stheorem{Teorema di Bolzano-Weierstrass}{
    Ogni sottoinsieme infinito e limitato di \(\mathbb{R}^n\) ha almeno un punto di accumulazione.
}

\sproof{Teorema di Bolzano-Weierstrass}{
    % todo semplice una riga
}

\sdefinition{Insiemi separati}{
    Sia \((X, d)\) uno spazio metrico e \(A,B\subseteq X\) due sottoinsiemi.
    Diciamo che \(A\) e \(B\) sono separati se
    \[
        A \cap \overline{B} = \emptyset \land \overline{A} \cap B = \emptyset
    \]
}

Devono sicuramente essere disgiunti, ma non basta.
Serve che non ci siano punti di accumulazione in comune.

\sdefinition{}{
    Sia \((X, d)\) uno spazio metrico e \(E\subseteq X\). \(E\) è connesso
    se non può essere scritto come unione di due sottoinsiemi non vuoti e separati.
}

I sottoinsiemi connessi di \(\mathbb{R}\) sono tutti e soli gli intervalli.

Uno spazio metrico è connesso se e solo se l'unico sottoinsieme non vuoto
di \(X\) che è anche aperto e chiuso è \(X\) stesso. (Dimostrazione per esercizio).

\(R^n\) con la metrica euclidea è connesso. (Dimostrazione per esercizio non proprio banale).

\subsection{Successioni in spazi metrici}

Mettere la definizione di convergenza ma con \(d(x_m,y) < \varepsilon\).
Oppure \(x_m \in B_\varepsilon(y)\).

In particolare la successione metrica converge se e solo se \(d(x_m, y) \to 0\)
secondo la convergenza reale.

Il limite è unico per proprietà di Hausdorff.

\sproposition{}{
    Sia \((X,d)\) uno spazio metrico e \(E \subseteq X\) e sia \(y\) un punto di accumulazione
    per \(E\). Allora esiste una successione \(\{x_n\} \subseteq E \setminus \{y\}\)
    che converge ad \(y\).
    In particolare, \(E\) è chiuso se e solo se per ogni successione \(\{x_n\} \subseteq E\)
    che converge ad \(y\) allora \(y \in E\).
}

\sproof{}{
    Dato che \(y\in E'\), \(\forall x_m \in \mathbb{N}\), esiste \(x_m\) tale che
    \(x_m \in B_{\frac{1}{m}}(y) \cap E\) e \(x_m \neq y\).
    La successione così costruita converge ad \(y\).
    Infatti, \(d(x_m, y) < \frac{1}{m} \to 0\).
}

\sproposition{}{
    Sia \((X,d)\) uno spazio metrico e sia \(\{x_n\}\) una successione convergente in \(X\).
    Una condizione necessaria per la convergenza è che ogni sottosuccesione converga allo stesso limite.
    La condizione sufficiente è che ogni sottosuccessione ammetta una sottosuccessione
    che converge allo stesso limite.
}

\sdefinition{Compattezza sequenziale}{
    Uno spazio metrico \(X\) è sequenzialmente compatto se ogni successione in \(X\)
    a valori in \(E\) ammette una sottosuccessione convergente ad un punto di \(E\). 
}

\sproposition{Equivalenza compattezza}{
    \(E\) is compact is and only if \(E\) is sequentially compact.
}

Questa c'è solo negli spazi metrici.

\sproof{}{
    \iffproof{
        Sia \(\{x_n\}\) una successione in \(E\).
        Consideriamo \(F = \{x_n \,|\, n\in \mathbb{N}\}\).
        Se \(F\) è finito, esiste un elemento che compare infiniti volte
        e la successione costante converge a tale elemento.
        Se \(F\) è infinito, per la compattezza \(F\) ammette un punto di accumulazione, \(y\in E\).
        Costruiamo una sottosuccessione che converga ad \(y\).
        Scegliamo \(x_m\) tale che \(d(x_m, y) < 1\).
        Scegliamo \(x_{m_2}\) tale che \(d(x_{m_2}, y) < \frac{1}{2}\) e \(m_2 > m_1\), e così via.
        La sottosuccessione così costruita converge ad \(y\) in quanto \(d(x_{m_k}, y) < \frac{1}{k} \to 0\).
    }{
        XXX
    }
}

Ogni successione convergente è di Cauchy.

Per esempio con la metrica discreta una successione è convergente se e solo se è definitamente costante,
che è equivalente ad essere di Cauchy, quindi è completo.

Nel caso dei razionali nei reali con metrica euclidea, consideriamo la radice di due che è un punto di accumulazione
per i razionali. Esiste una successione di razionali che converge a radice di due, quindi è di Cauchy.
Ma essa non può convergere in Q, altrimenti convergerebbe anche in R e avrebbe due limiti.
Tuttavia è una successione di Cauchy in Q perché è convergente in R e quindi è di Cauchy in R.
(La condizione è la medesima). Quindi Q non è completo.

\sdefinition{Spazio completo}{
    Uno spazio metrico \((X,d)\) è completo se ogni successione di Cauchy in \(X\)
    converge ad un punto di \(X\).
}

\stheorem{}{
    \(R^n\) con la metrica euclidea è completo.
}

\sproof{}{
    Sia \(\{x_n\}\) una successione di Cauchy in \(R^n\).
    Scriviamo \(E_n = \{x_k \,|\, k \geq n\}\).
    Notiamo che \(E_n \supseteq E_{n+1}\).
    Ponendo la chiusura \(\overline{E_n} \supseteq \overline{E_{n+1}}\).
    Inoltre, \(E_n\) è limitato e \(\text{diam} E_n \to 0\).
    Infatti, dato \(\varepsilon > 0\) esiste \(N\) tale che per ogni \(m,n \geq N\)
    \(d(x_n, x_m) < \varepsilon\).
    Notiamo inoltre che
    \[
        \text{diam} E_n = \sup\{d(x_m,x_k)\} < \varepsilon
    \]
    Dimostrazione per esercizio vale che \(\text{diam} F = \text{diam} \overline{F}\).
    Quindi, \(\text{diam} \overline{E_n} \to 0\).
    Adesso \(\{\overline{E_n}\}\) è una successione di compatti in quanto chiusi e limitati, annidati.
    Quindi \[
        E \triangleq \bigcap_{n\in \mathbb{N}} \overline{E_n} \neq \emptyset
    \]
    Siccome \(\text{diam} E = 0\) o è vuoto o contiene un solo punto, quindi contiene un solo punto \(E = \{y\}\).
    Mostriamo che \(x_n \to y\).
    Abbiamo \(d(x_n, y) \leq \text{diam} \overline{E_n} \to 0\).
}

\stheorem{}{
    Sia \((X, d)\) uno spazio metrico compatto.
    Allora \(X\) è completo.
}

\sproof{}{
    Sia \(\{x_n\}\) una successione di Cauchy in \(X\).
    Siccome è compatto è compatto per successioni, quindi esiste una sottosuccessione
    \(\{x_{n_k}\}\) che converge ad un punto \(y \in X\).
    Mostriamo che \(x_n \to y\).
    Dato \(\varepsilon > 0\) esiste \(N_0\) tale che per ogni \(m,n \geq N_0\)
    \(d(x_n, x_m) < \frac{1}{2}\varepsilon\).
    Per la convergenza di \(\{x_{n_k}\}\) esiste \(K\) tale che per ogni \(k \geq K\)
    \(d(x_{n_k}, y) < \frac{1}{2}\varepsilon\).
    Scegliamo \(\overline{N} = \max\{N_0, n_K\}\).
    Allora per ogni \(n \geq \overline{N}\) si ha
    \[
        d(x_n, y) \leq d(x_n, x_{n_K}) + d(x_{n_K}, y) < \frac{1}{2}\varepsilon + \frac{1}{2}\varepsilon = \varepsilon
    \]
}

\end{document}
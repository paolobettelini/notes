\documentclass[a4paper]{article}

\usepackage{amsmath}
\usepackage{amssymb}
\usepackage{stellar}
\usepackage{parskip}
\usepackage{fullpage}
\usepackage{wrapfig}
\usepackage{tikz}

\usetikzlibrary{arrows}
\usetikzlibrary{decorations.pathreplacing}
\usetikzlibrary{cd}

\title{Analisi II}
\author{Paolo Bettelini}
\date{}

% libri
% pagani-salsa
% fusco-marcellini-sbordone 

\begin{document}

\maketitle
\tableofcontents

\section{Spazi metrici}

\subsection{Definizioni}

L'insieme vuoto e l'insieme \(X\) sono aperti e chiusi.

\sproposition{}{
    L'unione di aperti non numerabile è aperta, mentre l'intersezione è aperta solo se finita.
}

\sproof{}{
    Per dimostrare quest'ultima lo facciamo su due insiemi e il resto è per induzione.
    Prendiamo un punto nell'interezione e prendiamo le due bolle dentro gli insiemi centrate
    nel punto. Siccome hanno lo stesso centro la loro intersezione è sempre una bolla di raggio
    il minore fra i due.
}

La metrice discreta può generare una bolla che è un singoletto.

\sproposition{}{
    L'unione di chiusi finiti è chiusa. L'intersezione qualsiasi è chiusa.
}

Ogni singoletto è chiuso. Per dimostrarlo mostriamo che nel complementare
esiste una bolla che non interseca il punto (vero per proprietà di Hausdorff).

Tutti i punti di accumulazione sono dei punti aderenti.
Tutti i punti di un sottoinsieme sono aderenti per il sottoinsieme.
Ogni punto o è di accumulazione o è isolato.

Se \(x_0\) è aderente ad E, \(x_0\) può essere un punto di E oppure no.
Se \(x_0\) è punto di accumulazione per E, in ogni bolla centrata in \(x_0\) cadono inifniti punti.

\sproposition{}{
    \(E^\circ\) è aperto. \(E\) è aperto se e solo se \(E = E^\circ\).
    \(E^\circ\) è il più grande aperto contenuto in \(E\).
    
    \(\overline{E}\) è chiuso. \(E\) è chiuso se e solo se \(E = \overline{E}\).
    La chiusura di \(E\) è il più piccolo chiuso contenente \(E\).
}

\sproof{per l'interno}{
    Dimostriamo che \(E^\circ\) è aperto. Sia \(x_0 \in E^\circ\).
    un punto interno ad \(E\), quindi esiste una bolla centrata in tale punto che è contenuta in \(E\).
    Prendiamo un altro punto \(y\) in questa bolla. Possiamo costruire una inner bolla centrata in \(y\)
    con un raggio sufficientemente piccolo da rimanere nella bolla più grande. Quindi il punto \(y\)
    è a sua volta interno, quindi tutta la bolla centrata in \(x_0\) è in \(E^\circ\) e quindi è aperto.
    
    Dimostriamo ora che se \(E\) è aperto allora \(E=E^\circ\) (l'altra implicazione è ovvia).
    Per fare ciò dimostriamo che \(E^\circ\) è il più grande aperto in \(E\).
    Osserviamo che \(E^\circ\) fa parte della famiglia degli aperti di \(X\)
    contenuti in \(E\). Sia \(A\) un aperto contenuto in \(E\). VOglio dimostrare che \(A \subseteq E^\circ\).
    Sia \(x_0 \in A\). \(A\) èunione di bolle quindi esiste unr aggio tale che la
    bolla centrata in \(x_0\) di tale raggio è contenuta in \(A\) che è contenuto in \(E\).
    Quindi, \(x_0\) è interno ad \(E\) e \(x_0 \in E^\circ\) e \(A \subseteq E^\circ\).
    Supponiamo ora che \(E\) sia aperto. Allora \(E\) fa parte della famiglia degli aperti
    di \(X\) contenuti in \(E\). Devo avere \(E \subseteq E^\circ\).
    Dato che \(E^\circ \subseteq E\) allora \(E^\circ = E\).
}

% lezione 2

Dire che un insieme è dentro in un altro significa dire che la sua chiusura coincide con l'insieme.
Tipo la chiusura di Q è R quindi Q è denso in R.


\sdefinition{Limitato}{
    Se è contenuto in una bolla
}

\sdefinition{Diametro}{
    è il sup della metrica su tutte le coppie.
}

\sdefinition{Ricoprimento}{
    Sia \(E\) un sottoinsieme di uno spazio metrico \(X\). Una famiglia
    \[
        \{G_\alpha\}_{\alpha \in A}
    \]
    è un ricoprimento apert di \(E\) se
    \[
        E \subseteq \bigcup_{\alpha \in A} G_\alpha
    \]
}

\sdefinition{Sottoricoprimento}{
    Un Sottoricoprimento di 
    \[
        \{G_\alpha\}_{\alpha \in A}
    \]
    è una sottofamiglia di \(G_\alpha\)
    tale che continua a ricoprire. Cioè ne scarto alcuni ma deve comunque rimanere
    una copertura.
}

\sdefinition{Compatto}{
    Uno spazio metrico \(X\) è compatto se ogni ricoprimento aperto di \(E\)
    ammette un sottoricoprimento finito.
}

Ogni insieme finito è compatto.


\stheorem{}{
    Sia  \(X\) uno spazio metrico e \(E\) un sottoinsieme di \(X\) compatto.
    \begin{enumerate}
        \item \(E\) è limitato;
        \item \(E\) è chiuso;
        \item Ogni sottoinsieme infinito di \(E\) ha almeno un punto di accumulazione in \(E\).
    \end{enumerate}
}

\sproof{}{
    \begin{enumerate}
        \item Consideriamo \(\{B_1(x) \,|\, x\in E\}\) che è un ricoprimento aperto di \(E\).
        Siccome \(E\) è compatto esiste un sottoricoprimento finito aperto di \(E\), ossia
        \(x_1, \ldots, x_n \in E\) tali che
        \[
            E \subseteq \bigcup_{i=1}^n B_1(x_i)
        \]
        Posto
        \[
            R = 1 + \max_{i=1,\ldots,n} d(x_i, x_1)
        \]
        Allora la bolla di raggio \(R\) centrata in \(x_1\) contiene \(E\), quindi \(E\) è limitato.
        \item Supponiamo che non sia chiuso. Allora esiste \(y\in E'\) ma \(y\notin E\).
        Vogliamo costruire un ricoprimento aperto di \(E\) che non ammette sottoricoprimento finito.
        Sia \(r(x) = \frac{1}{2} d(x,y)\) per ogni \(x\in X\).
        Se \(x\in E\) allora \(r(x) > 0\) perchè \(y\notin E\).
        Abbiamo il ricoprimento
        \[
            \{B_{r(x)}(x) \,|\, x\in E\}
        \]
        Ma per la compattezza esisterebbe un sottoricoprimento finito, cioè \(x_1, \ldots, x_n \in E\) tali che
        \[
            E \subseteq \bigcup_{i=1}^n B_{r(x_i)}(x_i)
        \]
        Sia ora \(R = \min_{i=1,\ldots,n} r(x_i)\). Allora \(R > 0\) e la bolla \(B_R(y)\)
        non interseca nessuna delle \(B_{r(x_i)}(x_i)\), assurdo poiché \(y\) è punto di accumulazione.
        \item Sia \(F\) un sottoinsieme infinito di \(E\). Supponiamo che \(F\) non abbia punti di accumulazione in \(E\).
        Allora ogni punto di \(E\) ha una bolla che interseca \(F\) in al più un punto.
        Queste formano un ricoprimento aperto di \(E\).
        Ma se esistesse un sottoricoprimento finito, \(F\) sarebbe finito, assurdo.
    \end{enumerate}
}

\sproposition{}{
    Sia \(E \subseteq X\) compatto. Se \(F \subseteq E\) è chiuso allora \(F\) è compatto.
}
\sproof{}{
    Sia \(\{G_\alpha\}_{\alpha \in A}\) un ricoprimento aperto di \(F\).
    Dobbiamo aggiungere degli insiemi aperti per coprire il resto.
    Siccome \(F\) è chiuso, \(X \setminus F\) è aperto.
    Quindi \(\{G_\alpha\}_{\alpha \in A} \cup \{X \setminus F\}\) è un ricoprimento aperto di \(E\).
    Per la compattezza di \(E\) esiste un sottoricoprimento finito, che escludendo \(X \setminus F\)
    è un sottoricoprimento finito di \(F\).
}

% ESERCIZIO
Se \(F\subseteq X\) è chiuso, ed \(E\subseteq X\) è compatto, allora \(F \cap E\) è compatto.

\stheorem{Teorema dell'intersezione finita}{
    Sia \(\{E_\alpha\}_{\alpha \in A}\) una famiglia di compatti tale che
    ogni intersezione finita è non vuota. Allora
    \[
        \bigcap_{\alpha \in A} E_\alpha \neq \emptyset
    \]
}

\sproof{}{
    Supponiamo che l'intersezione sia vuota. Allora
    e sia \(E_{\overline{\alpha}}\) un compatto fissato nella famiglia.
    \begin{align*}
        &E_{\overline{\alpha}} \cap \left(
            \bigcap_{\alpha \neq \overline{\alpha}} E_\alpha
        \right) = \emptyset \\
        &\implies
        E_{\overline{\alpha}} \subseteq
        \left(
            \bigcap_{\alpha \neq \overline{\alpha}} E_\alpha
        \right)^c
        =
        \bigcup_{\alpha \neq \overline{\alpha}} E_\alpha^c
    \end{align*}
    \(\{E_\alpha^c\}_{\alpha \neq \overline{\alpha}}\) è un ricoprimento aperto di \(E_{\overline{\alpha}}\).
    Esistono quindi \(\alpha_1, \ldots, \alpha_n \neq \overline{\alpha}\)
    tali che \begin{align*}
        &E_{\overline{\alpha}} \subseteq \bigcup_{i=1}^n E_{\alpha_i}^c
        = \left(
            \bigcap_{i=1}^n E_{\alpha_i}
        \right)^c \\
        &\implies
        E_{\overline{\alpha}} \cap \left(
            \bigcap_{i=1}^n E_{\alpha_i}
        \right) = \emptyset
    \end{align*}
    assurdo. % lightning
}

\scorollary{caso particolare}{
    Sia \(\{E_n\}_{n\in \mathbb{N}}\) una famiglia di compatti tale che
    \[
        E_{n+1} \subseteq E_n
    \]
    Allora
    \[
        \bigcap_{n\in \mathbb{N}} E_n \neq \emptyset
    \]
}

\stheorem{Teorema di Heine-Borel}{
    Sia \(E \subseteq R^n\) con la metrica euclidea. Allora \(E\) è compatto se e solo se
    \(E\) è chiuso e limitato.
}

\slemma{}{
    Sia \(\{I_k\}_{k\in \mathbb{N}}\) una famiglia di intervalli
    \(I_k = [a_k, b_k]\) tali che \(I_k \supseteq I_{k+1}\).
    Allora 
    \[
        \bigcap_{k\in \mathbb{N}} I_k \neq \emptyset
    \]
}

\sproof{}{
    Gli intervalli sono annidati, quindi \(a_k\) è crescente e \(b_k\) è decrescente e \(a_k \leq b_k\).
    In particolare \(a_k \leq b_i\). Consideriamo l'insieme \(E = \{a_k \,|\, k\in \mathbb{N}\}\).
    \(E\) è limitato superiormente, e ammette supremum \(x\).
    Per definizione \(x \geq a_k\). Ma \(a_k \leq b_i\) per tutte le \(i\).
    Quindi, \(x \leq b_i\) per ogni \(i\). Allora
    \[
        x \in I_n \implies x \in \bigcap I_k
    \]
}

\sdefinition{}{
    Siano \(a,b \in \mathbb{R}^n\) con \(a_i < b_i\) per ogni \(i=1,\ldots,n\).
    Un rettangolo chiuso è il prodotto cartesiano
    \[
        [a_1, b_1] \times [a_2, b_2] \times \ldots \times [a_n, b_n]
    \]
    che indichiamo con \([a,b]\).
}

\slemma{}{
    Sia \(\{R_k\}_{k\in \mathbb{N}}\) una famiglia di rettangoli chiusi
    tali che \(R_k \supseteq R_{k+1}\) per ogni \(k\).
    Allora
    \[
        \bigcap_{k\in \mathbb{N}} R_k \neq \emptyset
    \]
}

\sproof{}{
    Siccome
    \[
        R_k = I_{k,1} \times I_{k,2} \times \ldots \times I_{k,n}
    \]
    possiamo applicare il primo lemma e quindi
    \[
        \exists y_i \in \bigcap_{k\in \mathbb{N}} I_{k,i}
    \]
    Il punto \(y = (y_1, \ldots, y_n)\) è in ogni \(R_k\).
}

\slemma{Lemma 3}{
    In \(\mathbb{R}^n\) con la metrica euclidea
    ogni rettangolo è compatto.
}

\sproof{Lemma 3}{
    Sia \(R = [a,b]\) un rettangolo e supponiamo che non sia compatto.
    Sia \(\{G_\alpha\}_{\alpha \in A}\) un ricoprimento aperto di \(R\)
    che non ammette sottoricoprimento finito.
    Vogliamo adesso dimezzare ambo i lati (quindi \(n\) tagli).
    Abbiamo adesso \(2^n\) rettangoli.
    \[
        [a_i, b_i] = [a_i, c_i] \cup [c_i, b_i], \quad
        c_i = \frac{a_i + b_i}{2}
    \]
    Il diametro di \(R\) è \(||b-a||\).
    Il diametro di ogni rettangolo ottenuto è la metà.
    Almeno uno di questi rettangoli ha la proprietà di non ammettere sottoricoprimento finito.
    Lo chiamiamo \(R_1\).
    Iterando il procedimento otteniamo una successione di rettangoli
    \[
        R \supseteq R_1 \supseteq R_2 \supseteq \ldots
    \]
    con diametro che tende a zero e che non ammettono sottoricoprimento finito, il diametro
    di \(R^k\) è dato da \(\frac{1}{2^k} ||b-a||\).
    Per il lemma precedente esiste \(x \in \bigcap_{k\in \mathbb{N}} R_k\).
    Siccome \(R_k \subseteq R\) per ogni \(k\), \(x \in R\).
    Siccome \(\{G_\alpha\}_{\alpha \in A}\) è un ricoprimento di \(R\),
    esiste \(\alpha_0 \in A\) tale che \(x \in G_{\alpha_0}\).
    \(G_{\alpha_0}\) è aperto, quindi esiste \(r > 0\) tale che
    \(B_r(x) \subseteq G_{\alpha_0}\).
    Scegliamo \(k\) sufficientemente grande tale che \(2^{-k} ||b-a|| < r\).
    Ma il diametro di \(R_k\) è minore di \(r\), quindi \(R_k \subseteq B_r(x)\).
    Quindi \(R_k \subseteq G_{\alpha_0}\), assurdo perchè \(R_k\) non ammette sottoricoprimento finito.
}

\sproof{Heine-Borel}{
    Dobbiamo dimostrare solo che se \(E\) è chiuso e limitato allora è compatto.
    Siccome \(E\) è limitato esiste \(M\) tale che \(||x|| < M\) per ogni \(x \in E\).
    Quindi,
    \[
        E \subseteq [-M, M] \times [-M, M] \times \ldots \times [-M, M] = R
    \]
    \(E\) è un chiuso contenuto in un compatto, quindi è compatto.
}

\stheorem{Teorema di Bolzano-Weierstrass}{
    Ogni sottoinsieme infinito e limitato di \(\mathbb{R}^n\) ha almeno un punto di accumulazione.
}

\sproof{Teorema di Bolzano-Weierstrass}{
    % todo semplice una riga
}

\sdefinition{Insiemi separati}{
    Sia \((X, d)\) uno spazio metrico e \(A,B\subseteq X\) due sottoinsiemi.
    Diciamo che \(A\) e \(B\) sono separati se
    \[
        A \cap \overline{B} = \emptyset \land \overline{A} \cap B = \emptyset
    \]
}

Devono sicuramente essere disgiunti, ma non basta.
Serve che nessun punto di uno dei due insiemi è punto di accumulazione
dell'altro.

\sdefinition{}{
    Sia \((X, d)\) uno spazio metrico e \(E\subseteq X\). \(E\) è connesso
    se non può essere scritto come unione di due sottoinsiemi non vuoti e separati.
}

I sottoinsiemi connessi di \(\mathbb{R}\) sono tutti e soli gli intervalli.

Uno spazio metrico è connesso se e solo se l'unico sottoinsieme non vuoto
di \(X\) che è anche aperto e chiuso è \(X\) stesso. (Dimostrazione per esercizio).

\(R^n\) con la metrica euclidea è connesso. (Dimostrazione per esercizio non proprio banale).

\subsection{Successioni in spazi metrici}

Mettere la definizione di convergenza ma con \(d(x_m,y) < \varepsilon\).
Oppure \(x_m \in B_\varepsilon(y)\).

In particolare la successione metrica converge se e solo se \(d(x_m, y) \to 0\)
secondo la convergenza reale.

Il limite è unico per proprietà di Hausdorff.

\sproposition{}{
    Sia \((X,d)\) uno spazio metrico e \(E \subseteq X\) e sia \(y\) un punto di accumulazione
    per \(E\). Allora esiste una successione \(\{x_n\} \subseteq E \setminus \{y\}\)
    che converge ad \(y\).
    In particolare, \(E\) è chiuso se e solo se per ogni successione \(\{x_n\} \subseteq E\)
    che converge ad \(y\) allora \(y \in E\).
}

\sproof{}{
    Dato che \(y\in E'\), \(\forall x_m \in \mathbb{N}\), esiste \(x_m\) tale che
    \(x_m \in B_{\frac{1}{m}}(y) \cap E\) e \(x_m \neq y\).
    La successione così costruita converge ad \(y\).
    Infatti, \(d(x_m, y) < \frac{1}{m} \to 0\).
}

\sproposition{}{
    Sia \((X,d)\) uno spazio metrico e sia \(\{x_n\}\) una successione convergente in \(X\).
    Una condizione necessaria per la convergenza è che ogni sottosuccesione converga allo stesso limite.
    La condizione sufficiente è che ogni sottosuccessione ammetta una sottosuccessione
    che converge allo stesso limite.
}

\sdefinition{Compattezza sequenziale}{
    Uno spazio metrico \(X\) è sequenzialmente compatto se ogni successione in \(X\)
    a valori in \(E\) ammette una sottosuccessione convergente ad un punto di \(E\). 
}

\sproposition{Equivalenza compattezza}{
    \(E\) is compact is and only if \(E\) is sequentially compact.
}

Questa c'è solo negli spazi metrici.

\sproof{}{
    \iffproof{
        Sia \(\{x_n\}\) una successione in \(E\).
        Consideriamo \(F = \{x_n \,|\, n\in \mathbb{N}\}\).
        Se \(F\) è finito, esiste un elemento che compare infiniti volte
        e la successione costante converge a tale elemento.
        Se \(F\) è infinito, per la compattezza \(F\) ammette un punto di accumulazione, \(y\in E\).
        Costruiamo una sottosuccessione che converga ad \(y\).
        Scegliamo \(x_m\) tale che \(d(x_m, y) < 1\).
        Scegliamo \(x_{m_2}\) tale che \(d(x_{m_2}, y) < \frac{1}{2}\) e \(m_2 > m_1\), e così via.
        La sottosuccessione così costruita converge ad \(y\) in quanto \(d(x_{m_k}, y) < \frac{1}{k} \to 0\).
    }{
        XXX
    }
}

Ogni successione convergente è di Cauchy.

Per esempio con la metrica discreta una successione è convergente se e solo se è definitamente costante,
che è equivalente ad essere di Cauchy, quindi è completo.

Nel caso dei razionali nei reali con metrica euclidea, consideriamo la radice di due che è un punto di accumulazione
per i razionali. Esiste una successione di razionali che converge a radice di due, quindi è di Cauchy.
Ma essa non può convergere in Q, altrimenti convergerebbe anche in R e avrebbe due limiti.
Tuttavia è una successione di Cauchy in Q perché è convergente in R e quindi è di Cauchy in R.
(La condizione è la medesima). Quindi Q non è completo.

\sdefinition{Spazio completo}{
    Uno spazio metrico \((X,d)\) è completo se ogni successione di Cauchy in \(X\)
    converge ad un punto di \(X\).
}

\stheorem{}{
    \(R^n\) con la metrica euclidea è completo.
}

\sproof{}{
    Sia \(\{x_n\}\) una successione di Cauchy in \(R^n\).
    Scriviamo \(E_n = \{x_k \,|\, k \geq n\}\).
    Notiamo che \(E_n \supseteq E_{n+1}\).
    Ponendo la chiusura \(\overline{E_n} \supseteq \overline{E_{n+1}}\).
    Inoltre, \(E_n\) è limitato e \(\text{diam} E_n \to 0\).
    Infatti, dato \(\varepsilon > 0\) esiste \(N\) tale che per ogni \(m,n \geq N\)
    \(d(x_n, x_m) < \varepsilon\).
    Notiamo inoltre che
    \[
        \text{diam} E_n = \sup\{d(x_m,x_k)\} < \varepsilon
    \]
    Dimostrazione per esercizio vale che \(\text{diam} F = \text{diam} \overline{F}\).
    Quindi, \(\text{diam} \overline{E_n} \to 0\).
    Adesso \(\{\overline{E_n}\}\) è una successione di compatti in quanto chiusi e limitati, annidati.
    Quindi \[
        E \triangleq \bigcap_{n\in \mathbb{N}} \overline{E_n} \neq \emptyset
    \]
    Siccome \(\text{diam} E = 0\) o è vuoto o contiene un solo punto, quindi contiene un solo punto \(E = \{y\}\).
    Mostriamo che \(x_n \to y\).
    Abbiamo \(d(x_n, y) \leq \text{diam} \overline{E_n} \to 0\).
}

\stheorem{}{
    Sia \((X, d)\) uno spazio metrico compatto.
    Allora \(X\) è completo.
}

\sproof{}{
    Sia \(\{x_n\}\) una successione di Cauchy in \(X\).
    Siccome è compatto è compatto per successioni, quindi esiste una sottosuccessione
    \(\{x_{n_k}\}\) che converge ad un punto \(y \in X\).
    Mostriamo che \(x_n \to y\).
    Dato \(\varepsilon > 0\) esiste \(N_0\) tale che per ogni \(m,n \geq N_0\)
    \(d(x_n, x_m) < \frac{1}{2}\varepsilon\).
    Per la convergenza di \(\{x_{n_k}\}\) esiste \(K\) tale che per ogni \(k \geq K\)
    \(d(x_{n_k}, y) < \frac{1}{2}\varepsilon\).
    Scegliamo \(\overline{N} = \max\{N_0, n_K\}\).
    Allora per ogni \(n \geq \overline{N}\) si ha
    \[
        d(x_n, y) \leq d(x_n, x_{n_K}) + d(x_{n_K}, y) < \frac{1}{2}\varepsilon + \frac{1}{2}\varepsilon = \varepsilon
    \]
}

%%%% 1 ott


%%% ????
Sia \(X\) uno spazio metrico completo, \(Y \subseteq X\).
\(Y\) è completo se e solo se \(Y\) è chiuso in \(X\).

\stheorem{}{
    \(E\) sequenzialmente comaptto implica \(E\) compatto.
}

\sproof{}{
    Sia \(\{G_\alpha\}_{\alpha \in A}\) un ricoprimento aperto di \(E\).
    Esiste \(\delta > 0\) tale che \(\forall x \in E\) esiste \(\overline{\alpha}\)
    tale che \(B_\delta(x) \subseteq G_{\overline{\alpha}}\).
    \begin{enumerate}
        \item \emph{claim 1:} \(\forall m \in \mathbb{N}\), esiste \(x_m\) tale che
            \(B_{1/m}(x_m)\) non è sottoinsieme di \(G_\alpha\) per tutte le \(\alpha\).
            \(\{x_m\}\) è una successione in \(E\) e quindi posso estrarre una sottosuccessione convrgente
            \(x_{m_k} \to p \in E\). Esiste \(\hat{\alpha}\) tale che \(p \in G_{\hat{\alpha}}\).
            \(G_{\hat{\alpha}}\) è aperto e quindi esiste una \(\varepsilon > 0\)
            tale che \(B_\varepsilon(p) \in G_{\hat{\alpha}}\).
            Ma \(x_{m_k} \to p\) quindi con \(k\) sufficientemente grande
            \[B_{1/m_k}(x_{m_k}) \subseteq B_\varepsilon(p) \subseteq G_{\hat{\alpha}}\]
            che è assurdo lightning.
        \item \emph{claim 2:} \(E\) è contenuto nel'unione di un numero finito di bolle di raggio
            \(\delta\) centrate in punto di \(E\).
            Per assurdo, sia \(x_1 \in E\). Sicuramente \(B_\delta(x_1)\)
            non ricopre \(E\) quindi esiste \(x_2 \in E \backslash B_\delta(x_1)\).
            Ma assieme \(B_\delta(x_1) \cup B_\delta(x_2)\) non ricoprono \(E\),
            quindi esiste un \(x_3 \in E \backslash (B_\delta(x_1) \cup B_\delta(x_2))\)
            e così via. La successione \(\{x_m\}\) deve ammettere una sottosuccessione
            convergente. Ma \(d(x_i, x_j) \geq \delta\) se \(i \neq j\) quindi la sucessione
            \(\{x_m\}\) non è di Cauchy Lightning.
            Quindi \(E \subseteq B_\delta(x_1) \cup B_\delta(x_2) \cup \cdots\).
    \end{enumerate}
}

In realtà abbiamo mostrato anche la terza.

\stheorem{}{
    Sia \(X\) uno spazio metrico. Sono equivalenti:
    \begin{enumerate}
        \item \(X\) è compatto;
        \item \(X\) è sequenzialmente compatto;
        \item \emph{limit point compact:} ogni sottoinsieme infinito
        di \(X\) ha almeno un punto di accumulazione.
    \end{enumerate}
}

Solo negli spazi metrici.

\subsection{Funzioni}

\sdefinition{}{
    Siano \((X_1, d_1), (X_2, d_2)\) due spazi metrici e sia \(E \subseteq X_1\).
    Sia \(f \colon E \to X_2\) e \(p \in E'\).
    Diciamo che \(l\in X_2\) è limite di \(f(x)\) per \(x\to p\) e diciamo
    \[
        \forall \varepsilon > 0, \exists \delta > 0 \,|\, x \in E
        \land 0 < d_1(x_1, p) < \delta \implies d_2(f(x), l) < \varepsilon
    \]
    Equivalentemente \(\forall \varepsilon >0\) esiste \(\delta > 0\)
    tale che \[f((B_\delta(p) \cap E) \backslash \{p\}) \subseteq B_\varepsilon(l)\]
}

\sproposition{}{
    Sia \(f\colon E \subseteq X_1 \to X_2\).
    Allora \(f(x) \to l\) per \(x\to p\) se e solo se \(f(x_n) = l\)
    per ogni successione \(\{x_n\}\) tale che \(x_n \in E\)e \(x_n \neq p\)
    per tutte le \(n\) e \(x_n \to p\).
}

Valgono i medesimi teoremi tipo l'unicità del limite e i teoremi di permanenza del segno, confronto etc.

\sproposition{}{
    Sia \(f \colon E \subseteq X \to \mathbb{R}^n\) per \(n > 1\).
    Allora
    \[
        f(x) \to l \iff f_i(x) \to l_i
    \]
    per \(x \to p\).
}

\sproof{Sketch}{
    Conderiamo la norma per tutte le \(i\)
    \begin{align*}
        |f_i(x)-l_i| \leq ||f(x) - l|| \leq \sum_k |f_k(x) - l_k|
    \end{align*}
}

\sdefinition{Continuità}{
    Siano \((X_1, d_1), (X_2, d_2)\) due spazi metrici,
    \(f \colon E \subseteq X_1 \to X_2\), \(p \in E\).
    Diciamo che \(f\) è \emph{continua} in \(p\)
    se \(\forall \varepsilon > 0\) esiste \(\delta > 0\) tale che
    \[\forall x \in E \cap B_\delta(p) \implies f(x) \in B_\varepsilon((f(p)))\]
    Euivalentemente \(\forall \varepsilon > 0\) esiste \(\delta > 0\)
    tale che \(x\in E\) e \(d_1(x,p) <\delta\) implica che \(d_2(f(x), f(p)) < \varepsilon\).
    Oppure ancora \((f(B_\delta(p) \cap E)) \subseteq B_\varepsilon(f(p))\).
}

Se \(p\) è un punto isolato di \(E\) allora \(\exists r > 0\)
tale che \(B_r(p) \cap E = \{p\}\). Scegliendo \(\delta \leq r\)
la definizione di continuità è automaticamente soddisfatta.
Se \(p\) non è i solato, allora è un punto di accumulazione per \(E\).
In questo caso \(f\) è continua in \(p\) e vale che \(f(x) \to f(p)\) per \(x\to p\).

\sproposition{}{
    \(f\) è continua in \(p\) se e solo se
    \[
        \lim_{x\to p} f(x_n) = f(p)
    \]
    per ogni successione \(\{x_n\}\) tale che \(x_n \in E\)
    per tutte le \(n\) e \(x_n \to p\).
}

\sdefinition{}{
    Sia \(f \colon E \subseteq X_1 \to X_2\). Diciamo che \(f\) è continua nell'insieme \(E\)
    se \(f\) è continua in ogni punto di \(E\).
}

\sproposition{}{
    Siano \((X_1, d_1), (X_2, d_2)\) spazi metrici e sia \(f \colon X_1 \to X_2\).
    Allora \(f\) è continua in \(X\) se e solo se \(f^{-1}(V)\) è aperto in \(X_1\)
    per tutti i \(V\) aperti in \(X_2\).
}

\sproof{}{
    \iffproof{
        Sia \(V\) un aperto di \(X_2\). Se \(f^{-1}(V) = \emptyset\)
        in questo caso abbiamo finito. Altrimenti,
        sia \(p \in f^{-1}(V)\), cioè \(f(p) \in V\).
        Essendo \(V\) aperto, riesco a trovare \[ B_\varepsilon(f(p)) \in V \]
        Ma \(f\) è continua quindi riesco anche a trovare \(\delta>0\) tale che
        \[
            f(B_\delta(p)) \subseteq B_\varepsilon(f(p))
        \]
        Quindi \(B_\delta(p) \subseteq f^{-1}(V)\) quindi \(p\) è un punto interno a
        \(f^{-1}(V)\). Per l'arbitrarietà di \(p\) segue che \(f^{-1}(V)\) è aperto.
    }{
        Sia \(p \in X\) e dimostriamo che \(f\) è continua in \(p\).
        Sia \(\varepsilon > 0\) fissato. \(B_\varepsilon(f(p))\)
        è un aperto di \(X_2\). \(f^{-1}(B_\varepsilon(f(p)))\) è un aperto di \(X_1\)
        e \(p \in f^{-1}(B_\varepsilon(f(p)))\) e quindi esiste \(\delta > 0\)
        tale che
        \[
            B_\delta(p) \subseteq f^{-1}(B_\varepsilon(f(p)))
        \]
        cioè
        \[
            f(B_\varepsilon(p)) \subset B_\varepsilon(f(p))
        \]
        che è la definizione di continuità.
    }
}

Siccome \(f^{-1}(E^c) = \left(f^{-1}(E)\right)^c\) allora \(f\)
è continua se e solo se \(f^{-1}(C)\) è chiuso in \(X_1\)
per ogni chiuso in \(C \in X_2\). Molto utile.

In generale le funzioni continue non mandano aperti in aperti.
Per esempio \(f \colon \mathbb{R} \to \mathbb{R}\)
data da \(x \to x^2\).
Abbiamo che
\[
    f((-1,1)) = [0,1)
\]

\sdefinition{Funzione aperta}{
    Una funzione viene detta \emph{aperta}
    se \(f(U)\) è aperto in \(X_2\) per tutti gli insiemi \(U\) aperto om \(X_1\).
    Analogamente funzione chiusa.
}

Sia \(f \colon (X, d) \to \mathbb{R}^n\) con \(n>1\)
è continua se e solo se tutte le sue componenti sono continue.

\sproposition{}{
    Siano \((X_1, d_1), (X_2, d_2)\)
    spazi metrici, \(f \colon X_1 \to X_2\) una funzione continua.
    Se \(X_1\) è compatto, allora \(f(X_1)\) è compatto.
}

\sproof{}{
    Sia \(\{G_\alpha\}_{\alpha \in A}\) un ricoprimento aperto di \(f(X_1)\).
    Consideriamo \(\{f^{-1}(G_\alpha)\}_{\alpha \in A}\) che sono degli aperti.
    Queste preimmagini sono un ricoprimento di \(X_1\),
    che è compatto e quindi posso estrarre un sottoricoprimento finito
    \(f^{-1}(G_{\alpha_1}), \cdots, f^{-1}(G_{\alpha_n})\).
    Vogliamo mostrare che \(\{G_{\alpha_1}, \cdots, G_{\alpha_n}\}\)
    sono un ricoprimento di \(f(X_1)\).
    \[
        f(X_1) = f \left(
            \bigcup_{i=1}^n f^{-1}( G_{\alpha_i})
        \right)
        = \bigcup_{i=1}^n f\left(
            f^{-1}(G_{\alpha_i})
        \right)
        \subseteq \bigcup_{i=1}^n G_{\alpha_i}
    \]
}

\stheorem{Teorema di Weierstrass}{
    Sia \((X,d)\) uno spazio metrico compatto e sia
    \(f \colon X \to \mathbb{R}\) una funzione continua.
    Allora, \(\exists x_1, x_2 \in X\) tali che
    \[
        f(x_1) \leq f(x) \leq f(x_2), \quad \forall x \in X
    \]
    cioè \(f\) possiede massimo e minimo assoluto.
}

\sproof{Teorema di Weierstrass}{
    \(f(X)\) è compatto in \(\mathbb{R}\), quindi è chiuso e limitato.
    Siccome limitato, \(f(X)\) ammette infimum e supremum reali.
    Siccome \(\inf f(x)\) e \(\sup f(x)\)
    appartengono a \(\overline{f(x)}\) e \(f(x)\) è chiuso, appartengono allora ad \(f(X)\)
    e quindi sono massimi e minimi.
}

\stheorem{Teorema da compatto ad Hausdorff}{
    Siano \((X_1, d_1), (X_2, d_2)\) spazi metrici
    con \(X_1\) compatto e \(f \colon X_1 \to X_2\) continua.
    Allora, \(f\) è chiusa.
}

In realtà questo funziona con domini compatti e codomini di Hausdorff.

\sproof{Teorema da compatto ad Hausdorff}{
    Sia \(C\) un chiuso di \(X_1\).
    Voglio dimostrare che \(f(C)\) è un chiuso di \(X_2\).
    Sappiamo che \(C\) è chiuso in un compatto, quindi è compatto.
    La funzione è continua e quindi \(f(C)\) è compatto.
    Siccome i compatti sono chiusi allora è chiuso.
}

\scorollary{}{
    Sia \(f \colon (X_1, d_1) \to (X_2, d_2)\) continua, \(X_1\) compatto e
    \(f\) biunivoca. Allora, \(f^{-1}\) è continua.
}

\sproof{}{
    Dobbiamo mostrare che \((f^{-1})^{-1}(C)\) è chiuso per ogni \(C\) chiuso di \(X_2\).
    Ma questo coincide con \(f(C)\) che è chiusa per il teroema da compatto ad Hausdorff.
}

\stheorem{}{
    Sia \(f\colon (X_1, d_1) \to (X_2, d_2)\) continua e sia \(E \subseteq X\) connesso.
    Allora \(f(E)\) è connesso.
}

\sproof{}{
    Supponiamo che \(f(E)\) non sia connesso. Esistono quindi
    due sottoinsiemi non vuoti disgiunti e separati tali che
    \[
        f(E) = A \cup B
    \]
    Poniamo \(F = f^{-1}(A) \cap E\) e \(G = f^{-1}(B) \cap E\).
    Sicuramente \(F,G \neq \emptyset\) e \(E = F\cup G\). Vogliamo mostrare che \(E\) e \(G\)
    sono separati.
    Siccome \(A \subseteq \overline{A}\) vale anche \(f^{-1}(A) \subseteq f^{-1}(\overline{\overline{A}})\).
    L'applicazione \(f\) è continua e la chiusura di \(A\) è un chiuso.
    Quindi la preimmagine del chiuso \(\overline{A}\) è un chiuso.
    Consideriamo ora \[\overline{F} \subseteq \overline{f^{-1}(A)} = f^{-1}(\overline{A})\]
    perché \(f\) è continua se \(\overline{A}\) è chiuso.
    Quindi \(\overline{F} \subseteq f^{-1}(\overline{A})\) che implica
    \(f(\overline{F}) \subseteq \overline{A}\). D'altro canto \(f(G) \subseteq B\)
    e \(\overline{A} \cap B \neq 0\), e quindi \(\overline{F} \cap G \neq 0\)
    perché altrimenti vi sarebbe un elemento sia in \(\overline{A}\) che in \(B\).
    Dovrebbe essere \(f(x) \in \overline{A}\) e \(f(x) \in B\) lightinng.
    Analogamente si dimostra che \(F \cap \overline{G} = \emptyset\)
    cioè abbiamo scritto \(E\) come unione di due sottoinsiemi non vuoti e separati.
    Ma \(E\) è connesso lightning.
}

\sdefinition{}{
    Siano \((X_1, d_1),(X_2, d_2)\) spazi metrici e \(f\colon X_1 \to X_2\).
    Allora \(f\) è uniformemente continua se
    \(\forall \varepsilon > 0\), esiste \(\delta > 0\)
    tale che \(\forall x,y \in X_1\)
    \[
        d_1(x, y) < \delta \implies d_2(f(x), f(y)) < \varepsilon
    \]
}

\stheorem{Theorema di Heine-Cantor}{
    Siano \((X_1, d_1),(X_2, d_2)\) spazi metrici e 
    \(f\colon (X_1, d_1) \to (X_2, d_2)\)
    continua e \(X_1\) compatto.
    Alorra, \(f\) è uniformemente continua.
}

La dimostrazione è la medesima rispetto al caso banale.

\sdefinition{Funzione di Lipschitz}{
    Siano \((X_1, d_1),(X_2, d_2)\) spazi metrici,
    \(f\colon X_1 \to X_2\).
    Diciamo che \(f\) è \emph{lipschitz-continua} o \emph{lipschitziana}
    se \(\exists \alpha > 0\) tale che
    \[
        d_2(f(x), f(y)) \leq \alpha d_1(x,y)
    \]
    per tutte le \(x,y \in X_1\).
}

\sproposition{}{
    Se \(f\) è Lipschitz-continua, allora è uniformemente continua.
}

\sdefinition{}{
    Siano \((X_1, d_1),(X_2, d_2)\) spazi metrici e 
    \(f\colon (X_1, d_1) \to (X_2, d_2)\).
    Diciamo che \(f\) è una \emph{contrazione} se
    \(f \in \text{Lip}_\alpha(X_1, X_2)\) con \(\alpha < 1\).
}

Se il supremum è finito, allora questa è la miglior costante di Lipschitz (in generale)
\begin{align*}
    \sup_{x\neq y} \frac{|f(x)-f(y)|}{|x-y|}
\end{align*}

\sexample{}{
    Consideriamo \(f \colon \mathbb{R} \to \mathbb{R}\)
    data da \(f(x) = x^2\). Non è di lipschitz inquanto non è uniformemente continua.
    Per mostrarlo possiamo dire
    \begin{align*}
        |f(x)-f(y)| &= |x^2 - y^2| = |x+y| \cdot |x-y|
    \end{align*}
    Se restringessimo il dominio di questa funzione ad un intervallo limitato,
    allora sarebbe di Lipschitz, per il supremum.
}

\sproposition{}{
    Sia \(f\colon I \subseteq \mathbb{R} \to \mathbb{R}\) differenziabile.
    Allora, \(f \in \text{Lip}_\alpha(I, \mathbb{R})\)
    se e solo se \(|f'(x)| \leq \alpha\) per tutte le \(x\).
}

\sproof{}{
    \iffproof{
        Cominciamo con
        \begin{align*}
            |f'(x)| &= \left| \lim_{t\to 0} \frac{f(x+t)-f(x)}{t}\right| \\
            &= \lim_{t\to 0} \left| \frac{f(x+t)-f(x)}{t}\right| \\
            &= \lim_{t\to 0} \frac{|f(x+t)-f(x)|}{|t|} \\
            &\leq \lim_{t\to 0} \frac{\alpha |x+t-x|}{|t|} = \alpha
        \end{align*}
        Possiamo togliere il limite dal modulo in quanto il modulo è una funzione continua.
    }{
        Per il teorema di Lagrange esiste \(\theta \in (\min\{x,y\}, \max\{x,y\})\)
        \begin{align*}
            f(x) - f(y) &= f'(\theta) (x-y) \\
            |f(x) - f(y)| &= |f'(\theta)|\cdot |x-y| \\
            &\leq \alpha|x-y|
        \end{align*}
    }
}

\sexercise{}{
    Per quali \(a \leq b\) la funzione \(f(t) = 1 + t - \arctan(t)\)
    è una contradizione in \([a,b]\).
    Stabiliamo quindi se la derivata è limitata
    \begin{align*}
        f'(t) = 1 + \frac{1}{1 + t^2} = \frac{t^2}{1 + t^2}
    \end{align*}
    notiamo quindi che \(0 \leq f'(t) \leq 1\). Quindi è sicuramente lipschitziana.
    Notiamo allora che
    \[
        \sup_{\mathbb{R}} |f'(t)| = 1 = \alpha
    \]
    Quindi per far sì che \(\alpha < 1\) dobbiamo limitare il dominio ad un intervallo limitato.
    Quindi \(-\infty < a \leq b < \infty\).
    Porta l'intervallo \([a,b]\) in sè?
    Siccome la funzione è crescente porta intervalli a intervalli
    di estremi \(f(a)\) e \(f(b)\). Mi basta quindi imporre
    che \(f(a) \geq a\) e \(f(b) \leq b\). Abbiamo quindi
    \begin{align*}
        \begin{cases}
            1 + a - \arctan a \geq a \\
            1 + b - \arctan b \leq b
        \end{cases}
        =
        \begin{cases}
            \arctan a \leq 1 \\
            \arctan b \geq 1
        \end{cases}
    \end{align*}
    e quindi \(a \leq \tan 1 \leq b\).
    Notiamo che \(f(\tan 1) = \tan 1\) quindi è un punto fisso
    per il teorema delle contrazioni.
}

\sexample{}{
    Sia \(v \in \mathbb{R}^n\) e consideriamo
    \(f\colon \mathbb{R}^n \to \mathbb{R}\) data da
    \(f(x) = v \cdot x\).
    Dobbiamo studiare \(|f(x) - f(y) = |v\cdot x - v \cdot y|\)
    usando la bilinearità del prodotto scalare ottengo
    \(|v\cdot (x-y)|\). Per Cacuhy-Schwarz
    \[
        |v\cdot (x-y)| \leq ||v|| \cdot ||x-y||
    \]
    che è quindi di Lipschitz.
}

\stheorem{Teorema di Banach-Cacciopoli o delle contrazioni}{
    Sia \((X, d)\) uno spazio metrico completo
    e sia \(f \colon X \to X\) una contrazione.
    Allora \(\exists_{=1}\, x \in X\) tale che \(f(x) = x\).
}

Le ipotesi sono necessarie.
Togliamo per esempio la completezza e consideriamo quindi \(X=(0, +\infty)\)
con la contrazione \(f(x) = x/2\). Questa contrazione non ha punti fissi.
Togliamo invece l'ipotesi che sia una contrazione.
Richiediamo solamente che sia una contrazione debole, cioè
\[
    d_2(f(x), f(y)) \leq f_1(x,y)
\]
Consideriamo \(X = [0, +\infty)\) e prendiamo
\(f(t) = t + e^{-t}\). 
La derivata è \(f'(t) = 1-e^{-t}\) che è nulla nell'origine
e poi tende ad \(1\) dal sotto.
Chiaramente non ci sono punti fissi in quanto \(f(t)=t\) è come dire \(e^{-t} = 0\).

\sproof{}{
    Cominciamo mostrando l'esistenza del punto fisso.
    Sia \(x_0 \in X\) un punto fissato e
    consideriamo la successione \(x_{n+1} = f(x_n)\).
    \begin{enumerate}
        \item Mostriamo che \(\{x_n\}\) è di Cauchy, quindi siccome lo spazio è completo
        converge. Dobbiamo mostrare che \(d(x_n, x_m)\) tende a zero quando \(n,m\) crescono.
        Consideriamo inizialmente
        \begin{align*}
            d(x_{n+1}, x_n) &= d(f(x_n), f(x-{n-1})) \leq \alpha d(x_n, x_{n-1}) \\
            &= \alpha d(f(x_{n-1}), f(x_{n-2})) \leq \alpha^2 d(x_{n-1}, x_{n-2}) \\
            &\leq \alpha^n d(x_1, x_0)
        \end{align*}
        Calcoliamo ora la distanza generica e usiamo la disuguaglianza triangolare
        ripetutamente per ogni step
        \begin{align*}
            d(x_{n+k}, d_n) &\leq \sum_{i=0}^{k-1} d(x_{n+k-i}, d_{n+k-i-1}) \\
            &\leq d(x_1, x_0) \sum_{i=0}^{k-1} \alpha^{n+k-i} \\
            &= \alpha^n d(x_1, x_0) \sum_{i=0}^{k-1} \alpha^{n+k-i-1} \\
            &= \alpha^n \frac{\alpha^k - 1}{\alpha - 1} d(x_1, x_0) \\
            &= \frac{\alpha^n}{1 - \alpha} d(x_1, x_0) \to 0
        \end{align*}
        Per sbarazzarci di \(k\) (siccome vogliamo \(k\) arbitrario e il \(\varepsilon\)
        nella definizione di Cauchy deve essere uniforme rispetto ad esso)
        maggioriamo la somma parziale della serie geometrica
        con il valore della serie geometrica. Siccome \(0 < \alpha < 1\) il termine non esplode
        e la serie geometrica converge.
        \item Detto \(x\) il limite di \(\{x_n\}\) mostriamo che è un punto fisso di \(f\).
        Consideriamo il limite per \(n\to \infty\)
        \[
            \lim_{n \to \infty} f(x_n) = f\left(\lim_{n \to \infty} x_n\right)
            = f(x)
        \]
        perché \(f\) è continua.
    \end{enumerate}
    Mostriamo ora l'unicità del punto.
    Supponiamo che \(x,y\) siano due punti fissi. Vogliamo mostrare che \(d(x,y) = 0\).
    Abbiamo
    \begin{align*}
        d(x,y) &= d(f(x), f(y)) \leq \alpha d(x,y)
        (1-\alpha) d(x,y) &\leq 0
    \end{align*}
    siccome \(1-\alpha > 0\) ciò succede solo se \(d(x,y) = 0\).
}

Abbiamo notato che
\[
    d(x_{n+k}, x_n) \leq \frac{\alpha^n}{1 - \alpha} d(x_1, x_0)
\]
Con \(k \to \infty\) otteniamo
\[
    d(x, x_n) \leq \frac{\alpha^n}{1 - \alpha} d(x_1, x_0)
\]
quindi tende al punto fisso in maniera esponenziale.

Denotiamo \(f^n = f \circ f \cdots f\).
Se \(f\) è una contrazione, una qualsiasi sua iterazione è anch'essa una contrazione.
\begin{align*}
    d(f(f(x)), f(f(y))) \leq \alpha d(f(x), f(y)) \leq \alpha^2 d(x,y)
\end{align*}
Per induzione segue il resto. In generale la costante è \(\alpha^n\).
Ci chiediamo se nel caso in cui \(f\) non sia una contrazione, una sua iterata lo possa essere.

\sexample{}{
    Per esempio \(f(x) = \cos x\), che non è una contrazione in quanto il supremum
    della derivata è \(1\). Invece, \(\cos(\cos(x))\) ha derivata
    \[
        \sin(\cos(x)) \cdot \sin
    \]
    Il suo modulo è dato da
    \[
        |\sin(\cos(x))| \cdot |\sin x|
        \leq \sin(1) < 1
    \]
    Il secondo termine può solamente essere maggiorato da \(1\), mentre il secondo,
    siccome \(-1 \leq \cos(x) \leq 1\), può essere maggiorato da \(\sin 1\).
    Quindi è una contrazione.

    Con questo possiamo per esempio mostrare che il coseno ha un punto fisso,
    siccome una sua iterata è una contrazione.
}


\scorollary{Indebolimento del teorema delle contrazioni: teorema delle iterate contrazioni}{
    Sia \((X, d)\) uno spazio metrico completo e sia
    \(f \colon X \to X\) un'applicazione tale che \(\exists n \in \mathbb{N}\)
    tale che \(f^n\) sia una contrazione. Allora \(\exists_{=1}\, x\in X\)
    tale che \(f(x)=x\).
}

\sproof{}{
    Mostriamo che i punti fissi di \(f\) (che sono uno solo) sono i punti fissi di \(f^n\).
    Sia \(x\) un punto fisso di \(f\). Allora \(f^n(x) = f(f(\cdots(x)) = f(x) = x\).
    Quindi tutti i punti fissi di \(f\) sono anche punti fissi di \(f^n\).
    Sia ora \(x\) tale che \(f^n(x)=x\). Componendo otteniamo
    \begin{align*}
        f(f^n(x)) &= f(x) \\
        f^n(f(x)) &= f(x)
    \end{align*}
    quindi \(f(x)\) è un punto fisso di \(f^n\), ma siccome \(f\) è una contrazione
    ha solo un punto fisso, quindi coincidono \(f(x)=x\).
    Quindi tutti i punti fissi di \(f^n\) sono anche punti fissi di \(f\).
}

Parametrizziamo ora la funzione

Consideriamo \(T \colon X \times Y \to X\) come operatore parametrizzato dai valori di \(Y\).
Fissato \(y\) imponiamo che \(T(-, y): X \to X\) sia una contrazione.
Per tutte le \(y\) esiste un solo \(x\in X\) tale che \(T(x,y) = x\).
Data la dipendenza funzionale \(x=\varphi(y)\) vogliamo capire come il punto fisso dipende dal parametro.
In particolare, vogliamo mostrare che \(\varphi\) è continua sotto alcune ipotesi.

\stheorem{di dipendenza del punto tfisso del parametro}{
    Sia \(X\) uno spazio metrico completo e sia \(Y\) uno spazio metrico (topologico).
    Sia \(T \colon X \times Y \to X\) tale che \(\exists \alpha < 1\) tale che
    \(\forall y \in Y\), \(T(-,y)\) è in \(\text{Lip}_\alpha(X)\). (\(\alpha\) deve
    essere uniforme rispetto a \(y\)).
    Sia \(y_0 \in Y\) tale che \(\forall x \in X\), \(T(x,y)\) sia continua in \(y_0\).
    Allora \(f\) è continua in \(y_0\).
}

\sproof{}{
    Vogliamo mostrare che \(d(f(y), f(y_0)) \to 0\) se \(y \to y_0\).
    Vogliamo stimare \(d(f(y), f(y_0))\) con \(f(y) = T(f(y), y)\).
    Sia \(x = f(y)\) e \(f(y_0) = x_0\). Allora
    \[
        d(f(y), f(y_0)) = d(T(x,y), T(x_0, y_0))
    \]
    Usando la disuguaglianza triangolare
    \begin{align*}
        d(T(f(y_0), y_0))
        &\leq d(T(f(y), y), T(f(y_0), y)) +
        d(T(f(y_0), y), T(f(y_0), y_0)) \\
        &\leq \alpha d(f(y), f(y_0))
        + d(T(f(y_0), y), T(f(y_0), y_0)) \\
        (1-\alpha) d(f(y), f(y_0))
        &\leq d(T(f(y_0), y), T(f(y_0), y_0)) \to 0
    \end{align*}
    La costante è positiva e indipendente da \(y\).
    La funzione \(T(f(y_0), -)\) è continua in \(y\).
}

\slemma{Sugli spazi normati}{
    \[
        \left|
            ||y|| - ||x||
        \right|
        \leq ||y-x||
    \]
}

\sproof{}{
    Sia \(y = x + (y-x)\). Allora
    \[
        ||y|| = ||x + (y-x)|| \leq
        ||x|| + ||y-x||
    \]
    Scambiando i ruoli di \(x\) e \(y\) si ottiene la proposizione.
}

Ciò mostra che la norma è lipschitz continua.

Ogni spazio normato è uno spazio metrico, ma non il viceversa.

\sdefinition{Spazio di Banach}{
    Uno \emph{spazio di Banach} è uno spazio normato
    completo rispetto alla norma.
}

\sdefinition{Equivalenza di norme}{
    Diciamo che due norme \(||\cdot||, |\cdot|\) sono equivalenti
    se \(\exists 0 < \alpha \leq \beta\) tale che
    \[
        \alpha|x| \leq ||x|| \leq \beta|x|, \quad \forall x \in X
    \]
}

Questa è una relazione di equivalenza.

\pagebreak

\stheorem{Equivalenza di norme reali}{
    Tutte le norme in \(\mathbb{R}^n\) sono equivalenti.
}

\sproof{}{
    Basta mostrare che una norma \(||\cdot||\) questa è equivalente alla \(||\cdot||_2\).
    Dobbiamo trovare \(\alpha,\beta\) tale che
    \[
        \alpha||x||_2 \leq ||x|| \leq \beta||x||_2, \quad \forall x \in \mathbb{R}^n
    \]
    Consideriamo la base canonica \(\{e_1, \cdots, e_n\}\) che è finito-dimensionale
    e quindi \(x = (x_1, \cdots, x_n)\).
    \begin{align*}
        ||x|| &= \left|\left|
            \sum_{i=1}^n x_i \cdot e_i
        \right|\right| \leq \sum_{i=1}^n
        |x_1| \cdot ||e_i|| \\
        &\leq
        \left(n \max \bigcup_{i=1}^n \{||e_i||\}\right)
        \left(
            \sum_{i=1}^n |x_i|
        \right) \\
        &\leq \left(n \max \bigcup_{i=1}^n \{||e_i||\}\right) ||x||_1
        \leq \underbrace{\left(n \max \bigcup_{i=1}^n \{||e_i||\}\right)}_\beta ||x||_2
    \end{align*}
    Questo ci dice anche che \(||\cdot||\) è continua rispetto
    alla topologia indotta da \(||\cdot||_2\), e pure lipschitziana.
    Dobbiamo ora dimostrare l'altra metà della disuguaglianza e trovare \(\alpha\).
    Poniamo \(\alpha\) in funzione dei vettori
    \[
        \alpha = \inf_{||x||_2 = 1} ||x||
    \]
    Mostriamo che \(\alpha>0\).
    Una volta fatto questo, possiam ottenere che la definizione
    di \(\alpha\) dice che \(||x||_2 = 1 \implies ||x|| \geq \alpha\).
    Voglio dimostrare che \(||x|| \geq \alpha ||x||_2\) per tutte le \(x \in \mathbb{R}^n\).
    Se \(x=0\) la disuguaglianza è soddisfstta. Altrimnti, normalizziamo \(z = x / ||x||_2\).
    Usando l'omogeneità assoluta
    \[
        ||z||_2 = \left|\left|\frac{x}{||x||_2}\right|\right|
        = \frac{1}{||x||_2}||x||_2 \implies ||z||_2 = 1
    \]
    Quindi \(||z|| \geq \alpha\) and furthermore
    \[
        \left|\left|\frac{x}{||x||_2}\right|\right| \geq \alpha \implies
        ||x|| \geq \alpha ||x||_2
    \]
    We now need to show that \(\alpha\) is positive.
    Siccome le normi sono non-negative, alla peggio sono nulle.
    In realtà \(\alpha\) è un minimo
    \[
        \alpha = \min_{||x||_2 = 1} ||x||
    \]
    since the norm is continuous with the respect to the topology induced by \(||\cdot||_2\).
    The set over which we are taking the minimum is clearly closed and bounded.
    Siccome siamo nella topologia reale con norma euclidea è quindi anche compatto.
    Per Weierstrass, \(\alpha\) è un minimo.
    Quindi deve essere \(\alpha > 0\). Se fosse \(\alpha = 0\)
    allora esisterebbe \(\hat{x}\) tale che \(||\hat{x}||_2 = 1\) e \(||\hat{x}|| = 0\), che è assurdo lightning.
}

\section{Operatori lineari fra spazi vettoriali}

Lo spazio \(\mathcal{L}(\mathbb{R}^n, \mathbb{R}^m) \cong \text{Mat}_{m\times n}(\mathbb{R})\).

Studiamo la continuità degli operatori lineari in spazi normati.

\sproposition{}{
    Ogni \(A \in \mathcal{L}(\mathbb{R}^n, \mathbb{R}^m)\)
    è continua rispetto alla topologia indotta dalla norma (qualsiasi visto che sono equivalenti in \(\mathbb{R}^n\)).
}

\sproof{}{
    \begin{enumerate}
        \item Mostriamo che la continuità dell'operatore in un singolo punto, come l'origine,
        implica la continuità di \(A\) in tutto \(\mathbb{R}^n\). Questo è un fatto generale.
        Abbiamo quindi che se \(\{x_n\} \to 0\)
        allora \(\{Ax_n\} \to A0 = 0\). La continuità generale è data dal fatto che se \(\{y_n\} \to x\)
        allora \(\{Ay_n\} \to Ax\).
        Ma \(\{Ay_n\} \to Ax\) se e solo se \(\{Ay_m - Ax\} \to 0\) cioè \(\{A(y_m - x)\} \to 0\) e
        per linearità \(\{y_m - x\}\) è una successione che tende a zero, quindi \(A\) è continuo ovunque.
        Inoltre, per la continuità uniforme possiamo msotrare che \(\forall \varepsilon > 0, \exists \delta > 0\)
        tale che
        \[
            ||y-x|| < \delta \implies ||Ay - Ax|| < \varepsilon
        \]
        Ma \(||Ay - Ax|| = ||A(x-y)||\).
        Poniamo quindi \(z = y-x\). Dobbiamo mostrare che se \(||z|| < \delta\)
        allora \(||Az|| < \varepsilon\). Ma questa è la continuità nell'origine che stiamo presupponendo.
        \item Mostriamo ora la continuità nell'origine.
        Usiamo il fatto che \(\mathbb{R}^n\) ha dimensione finita.
        Sia \(x = (x_1, \cdots, x_n)\) secondo la base canonica \((e_1, \cdots, e_n)\).
        Calcolo usando la disuguaglianza triangolare
        \begin{align*}
            ||Ax|| &= \left|\left|
                \sum_{i=1}^n x_i A e_i
            \right|\right|
            \leq \sum_{i=1}^n |x_i| \cdot ||Ae_i|| \\
            &\leq C \sum_{i=1}^n |x_i|_1, \quad C = \max \bigcup_{i=1}^n \{||Ae_i||\}
        \end{align*}
        Ciò dimostra quindi che la funzione è Lipschitz-continua.
    \end{enumerate}
}

Nel passo secondo abbiamo usato il fatto che lo spazio fosse finitamente generato (il dominio).
In generale, con \(A \colon X \to Y\) è un operatore lineare fra spazi normati qualunque,
on è detto che \(A\) sia continuo.

\sexample{Controesempio}{
    Consideriamo lo spazio delle funzioni \(f\colon \mathbb{R} \to \mathbb{R}\)
    limitate, in \(\mathcal{C}^1\) e con derivata limitata \(\mathcal{BC}^1(\mathbb{R})\).
    Come secondo spazio prendiamo delle funzioni continue e limitate \(f\colon \mathbb{R} \to \mathbb{R}\),
    \(\mathcal{BC}(\mathbb{R})\).
    Consideriamo quindi l'operatore della derivata \(\mathcal{BC}^1(\mathbb{R}) \to \mathcal{BC}(\mathbb{R})\).
    Siccome questi non sono spazi finitamente generati, dobbiamo scegliere delle norme.
    Scegliamo come norma sia nel dominio che nel codominio \(||\cdot||_\infty\),
    che ha senso siccome le funzioni sono limitate. Mostriamo quindi che l'operatore lineare non è continuo.
    Scegliamo l'origine. Vogliamo quindi trovare \(\{f_n\} \to 0\) ma tale
    che \(\{f_n'\}\) non tende a zero.
    Per farlo prendiamo una funzione oscillante che si schiaccia sull'ascisse, e quindi la sua derivata
    non si schiaccia come la funzione. Prendiamo
    \[
        f_n(x) = \frac{1}{n}\sin(nx)
    \]
    Abbiamo la norma
    \begin{align*}
        ||f||_\infty &= \sup_{\mathbb{R}} \frac{1}{n}|\sin(nx)|
        = \frac{1}{n}
    \end{align*}
    Mentre la norma della derivata
        \begin{align*}
        ||f'||_\infty &= \sup_{\mathbb{R}} |\cos(nx)| = 1
    \end{align*}
}

Andiamo a definire una norma speciale su questo spazio, la norma operatoriale.

\sdefinition{Norma operatoriale}{
    \[
        ||A||_* = \sup_{||x|| \leq 1} ||Ax||
    \]
}

\sproposition{}{
    Norma operatoriale è una norma.
}

Mostriamo che la norma è ben definita, e che questo è un numero reale.
Infatti, \(||A||_* < +\infty\), siccome l'insieme del supremum è chiuso e limitato e, quindi, compatto,
e la funzione è continua allora il supremum è un massimo.
Mostriamo che il massimo si ottiene sulla frontiera della bolla.

\sproof{}{
    Mostriamo le due disuguaglianze.
    Il fatto che \(||A||_* \geq \max ||Ax||\) è banale, infatti \(\{x \,|\, ||x||=1\}\)
    è un sottoinsieme di \(\{x \,|\, ||x|| \leq 1\}\).
    D'altra parte se il massimo è ottenuto per \(x = 0\) allora il max è 0 e quindi
    \(Ax = 0\) sempre, e la disuguaglianza è banalmente soddisfatta.
    Supponiamo ora che il massimo sia ottenuto in un punto non nullo \(\hat{x}\),
    possiamo normalizzare e ottenere norma unitaria.
    \begin{align*}
        ||A\hat{x}|| &= \left|\left|A\left(||\hat{x}|| \frac{\hat{x}}{||\hat{x}||}\right)\right|\right|
        = \left|\left|||\hat{x}|| A \left(\frac{\hat{x}}{||\hat{x}||}\right)\right|\right|
        = ||\hat{x}|| \left|\left|A \left(\frac{\hat{x}}{||\hat{x}||}\right)\right|\right|
        \leq \left|\left|A\left(\frac{\hat{x}}{||\hat{x}||}\right)\right|\right| \\
        ||A||_* &= \max_{||x|| = 1 ||Ax||} = |A\hat{x}|
        \leq \left|\left|A\frac{\hat{x}}{||\hat{x}||}\right|\right| \leq \max_{||x||=1} ||Ax||
    \end{align*}
}

Quindi il valore può essere calcolato solo sulla buccia in quanto lì viene raggiunto il massimo.

\sproposition{Stima fondamentale}{
    Per ogni \(x\in \mathbb{R}^n\) vale
    \[
        ||Ax||_* \leq ||A|| \cdot ||x||
    \]
}

\sproof{}{
    Se \(||x|| = 1\) vale \(||Ax|| \leq ||A||\)
    perché per la proprietà precedente, \[||A|| = \max_{||x||=1} ||Ax||\]
    Se \(x=0\), la disuguaglianza vale.
    Altrimenti, normalizziamo \(x\) per ritrovarci sulla frontiera.
    \begin{align*}
        ||A||_* \geq \left|\left|
        A\left(\frac{x}{||x||}\right)\right|\right|
        = \frac{1}{||x||} ||Ax||
    \end{align*}
    moltiplicando entrmambi i membri epr \(||x||\) si ottiene la tesi.
}

In realtà questa costante è la migliore.

\sproposition{}{
    Se \(\exists \alpha \in \mathbb{R}\)
    tale che \(||Ax|| \leq \alpha ||x||\) per tutte le \(x \in \mathbb{R}^n\),
    allora \[||A||_* \leq \alpha\]
}

In realtà questo risultato vale anche se supponiamo che \(||Ax|| \leq \alpha ||x||\)
solo per x tale che \(||x|| \leq 1\) oppure tale che \(||x|| = \varepsilon\)
per qualche \(\varepsilon > 0\).

\sproof{}{
    Supponiamo di avere una stima del tipo \(||Ax|| \leq \alpha ||x||\) per tutte le x.
    Sappiamo che
    \begin{align*}
        ||A||_* &= \max_{||x|| = 1} ||Ax|| \leq \alpha \max_{||x|| \leq 1} ||x|| = \alpha
    \end{align*}
    che è quindi chiaramente \(1\).
}

La medesima dimostrazione funziona supponendo che \(||Ax|| \leq \alpha ||x||\)
per tutte le \(||x|| \leq 1\).
Se invece sappiamo che \(||Ax|| \leq \alpha ||x||\) solo per gli \(x\)
tale che \(||x|| = \varepsilon\), allora è sufficiente normalizzare (per esercizio).

\sproof{La norma operatoriale è una norma}{
    \begin{enumerate}
        \item \emph{annullamento:} Supponiamo che \(||A||_* = 0\).
        Devo mostrare che \(A=0\). Siccome la norma è nulla,
        \[
            \max_{||x|| \leq 1} ||Ax|| = 0 \implies ||Ax|| = 0, \quad \forall x \,|\, ||x|| \leq 1
        \]
        e quindi anche per tutti le altre \(x\) visto che possiamo normalizzare. Quindi, \(Ax = 0\).
        \item \emph{positiva omogeneità:} 
        \begin{align*}
            ||\lambda A||_* &= \max_{||x|| = 1} ||\lambda Ax||
            = \max_{||x|| = 1} |\lambda| |Ax| = |\lambda| \max_{||x|| = 1} ||Ax|| = |\lambda| ||A||_*
        \end{align*}
        \item \emph{disuguaglianza triangolare:}
        \begin{align*}
            ||(A+B)x|| &= ||Ax + Bx|| \leq ||Ax|| + ||Bx|| \\
            &\leq ||A||_* ||x|| + ||B||_* ||x| = ||x|| \left(
                ||A||_* + ||B||_*
            \right)
        \end{align*}
        Per la proposizion precedente, \(||A+B||_*\) è la più piccola costante
        per cui vale una disuguaglianza di questo tipo.
    \end{enumerate}
}

La norma operatoriale è scelta tale precisamente per la stima.

\stheorem{}{
    \((\mathcal{L}(\mathbb{R}^n, \mathbb{R}^m), ||\cdot||)\) è uno spazio di Banach.
}

\sproof{}{
    Siccome questo spazio è finito dimensionale è isomorfismo allo spazio \(\mathbb{R}^{m \times n}\)
    che è completo per esempio rispetto alla norma seconda.
    Tuttavia, dimostramolo con le successioni di Cauchy. Cosideriamo quindi una successione di operatori lineari.
    Per tutte le \(\varepsilon > 0\) esiste \(N_\varepsilon\) tale che \(\forall m,n > N_\varepsilon\)
    tale che
    \[
        ||A_m - A_n|| < \varepsilon
    \]
    cioè per tutte le \(x\)
    \[
        ||(A_n - A_n)x|| \leq \varepsilon(x)
    \]
    Questo vuole dire che per \(x\) fissato la successione \(\{A_n x\}\) è una successione di Cauchy
    in \(\mathbb{R}^n\). Siccome \(\mathbb{R}^n\) è completo, la successione converge.
    Chiamiamo il limite di tale successione \(Ax\).
    Verifichiamo che in questo modo abbiamo definito un operatore lineare \(x \to Ax\).
    Sappiamo che \(A_n\) è lineare per ogni \(n\), quindi \(A_n(x + y) = A_nx + A_n y\).
    Sappiamo che \(A(x+y)\) tende ad \(Ax + Ay\) e quindi
    l'espressione sopra tende ad \(Ax + Ay\).
    Analogamente per l'omogeneità.
    Mostriamo ora che \(\{A_n\}\) effettivamente converge ad \(A\), quindi \(||A_n - A|| \to 0\).
    Sappiamo che \(\{A_n\}\) è una successione di Cauchy.
    Quindi \(\forall \varepsilon > 0\) esiste \(N_\varepsilon\) tale che \(\forall n,m > N_\varepsilon\)
    \[
        ||A_m x - A_n x| \leq \varepsilon||x||
    \]
    Ma \(A_n x \to A x\) per \(n \to \infty\).
    passando al limite si ha che per tutte le \(\varepsilon > 0\) esiste \(N_\varepsilon\)
    tale che \(\forall n > N_\varepsilon\), \[
        ||A_nx - Ax|| \leq \varepsilon|x|
    \]
    cioè abbiamo trovato che \(||A_n - A|| - \varepsilon\).
}

Notiamo che abbiamo sfruttato solo la completezza di \(\mathbb{R}^n\).

\sexercise{}{
    Sia \(A \colon \mathbb{R}^2 \to \mathbb{R}^3\)
    dato da
    \[
        A = \begin{pmatrix}
            3 & -2 \\ 0 & 1 \\ 1 & 3
        \end{pmatrix}
    \]
    Prendiamo la norma infinito in ambo gli spazi.
    Il massimo è dato dal valore sulla frontiera della bolla.
    \[
        ||A||_*
        = \max_{\max \{|x|, |y|\} = 1}
        \max \{|3x-2y|, |y|, |x+3y|\}
    \]
    Quindi \(|3x - 2x| \leq 3|x| + 2|y|\)
    edeve essere \(\max\{||x||,||y||\} = 1\).
    In particolare i due moduli sono minori o uguali ad uno.
    \(|3x-2y| \leq 3+2 = 5\) e il valore \(5\)
    è assunto per \(x=1\) e \(y=-1\) (oppure cil contrario).
    Quindi \(||A||=5\).

    Consideriamo invece ora la norma \(1\)
    quindi
    \[
        ||A|| = \max_{|x| + |y| = 1}
        (|3x-2y| + |y| + |x+3y|)
    \]
    Allora
    \[
        |3x - 2y| + |y| + |x + 3y|
        \leq 3|x| + 2|y| + |y| + |x| + 3|y|
        = 4|x| + 6|y|
    \]
    Siccome \(|x| + |y| = 1\)
    quest'ultima è un interpolazione lineare
    tipo \(ta + (1-t)b\).
    In questo caso abbiamo l'espressione 
    \(4|x| + 6|y|\)
    che è sempre minore o uguale di \(6\).
    Quindi abbiamo trovato che \(|3x-2y| + |y| + |x + 3y| \leq 6\).
    Tale valore può essere raggiunto per esempio scegliendo \(x=0\) e \(y=1\).
}

Consideriamo ora operatori lineari invertibili,
quindi fra spazi con la medesima dimensione.

\stheorem{}{
    \(\text{GL}_n(\mathbb{R})\) è un aperto 
    in \(\mathcal{L}(\mathbb{R}^n, \mathbb{R}^n)\)
    dove consideriamo lo spazio topologico indotto dalla metrica indotta
    dalla norma associata allo spazio. Inoltre, la mappa dell'inversione è continua.
}

\sproof{}{
    Visto che il determinante è non nullo
    \[
        \text{GL}_n(\mathbb{R}) = \text{det}^{-1}(\mathbb{R} \backslash \{0\})
    \]
    ma visto che è la preimmagine di un aperto tramite il determinante, che è continuo.

    Dimostriamo tuttavia la proposizione con una argomento che può funzionare anche
    nel caso infinito dimensionale.
    \begin{enumerate}
        \item \emph{identità è un punto interno dell'insieme:} mostriamo che esiste \(\delta\)
        tale che \(||A-\mathbf{1}|| < \delta\) implica che \(A \in \text{GL}(\mathbb{R}^n)\).
        Stiamo usando la norma operatoriale.
        Posiamo \(B = \mathbf{1} - A\) e quindi \(A^{-1} = {(\mathbf{1} - B)}^{-1}\)
        che possiamo scrivere come la serie geometrica di \(B^k\).
        Ovviamente dobbiamo usare la convergenza della somma parziale della serie
        secondo la norma dei vettori. (Purché la serie converga).
        Infatti, se la serie converge usiamo la proprietà distributiva per mostrare che è veramente l'inverso di \(\mathbf{1} - B\)
        \begin{align*}
            (\mathbf{1}-B) \sum_{k=0}^\infty B^k
            &= \sum_{k=0}^\infty B^k - \sum_{k=0}^\infty B^{k+1} \\
            &= \sum_{k=0}^\infty B^k - \sum_{k=1}^\infty B^k = B^0 = \mathbf{1} 
        \end{align*}
        Quindi la stessa cosa che vale per i numeri, vale anche con gli operatori (negli spazi vettoriali in generale).
        Dobbiamo tuttavia studiare quando la serie converge.
        Usando il criterio di Cauchy, troviamo che la serie converge se e solo se \(\forall \varepsilon > 0, \exists N_\varepsilon \in \mathbb{N}\)
        tale che \(\forall p \in \mathbb{N}\) \begin{align*}
            \left|\left|
                \sum_{k=m}^{m+p} B^k
            \right|\right| \leq \varepsilon
        \end{align*}
        Mediante la stima otteniamo una serie numerica.
        \begin{align*}
            \left|\left|
                \sum_{k=m}^{m+p} B^k
            \right|\right| \leq \sum_{k=m}^{m+p}
            \left|\left| B^k \right|\right|
        \end{align*}
        Preferiremmo lavorare la potenza di una norma piuttosto che la norma di una potenza,
        ma coincidono?

        Calcoliamo quindi \(||B \circ A||_*\)
        con \(A,B\) operatori lineari, e usando la norma operatoriale.
        Siccome abbiamo la norma operatoriale vale la stima fondamentale
        \begin{align*}
            ||(B\circ A) x|| &= ||B(Ax)|| \leq ||B||_* ||Ax|| \leq ||B||_* ||A||_* ||x||
        \end{align*}
        Siccome la norma operatoriale è la più piccola costante per cui vale una stima del tipo dato.
        Quindi \(||B \circ A||_* \leq ||B||_* ||A||_*\).
        Quindi in particolare \(||B^k||_* \leq {||B||_*}^k\).
        Tornando a prima otteniamo
        \begin{align*}
            \left|\left|
                \sum_{k=m}^{m+p} B^k
            \right|\right|_* \leq
            \sum_{k=m}^{m+p} ||B^k||_*
            \leq \sum_{k=m}^{m+p} {(||B||_*)}^k \leq \varepsilon
        \end{align*}
        pur di considerare \(||B||_* < 1\).
        \item Sia \(A_0 \in \text{GL}_n(\mathbb{R})\).
        Vogliamo mostrare che \(\exists \delta > 0\) tale che
        \[
            ||A-A_0||_* < \delta \implies A \in \text{GL}_n(\mathbb{R})
        \]
        Calcoliamo
        \begin{align*}
            A = A_0 - (A_0 - A) = A_0 (\mathbf{1}  A_0^{-1}(A_0 - A))
            = A_0 (\mathbf{1} - B), \quad B = \mathbf{1} - B
        \end{align*}
        per il punto precedente \(\mathbf{1} - B\) è invertibile purché \(||B||_* < 1\).
        Studiamo quando ciò avviene
        \begin{align*}
            ||B||_* = ||A_0^{-1}(A_0 - A)||_* \leq ||A_0^{-1}|| \cdot ||A_0 - A|| < 1
            \text{ se } ||A_0 - A||_* < \frac{1}{||A_0^{-1}||}
        \end{align*}
        la divisione la possiamo fare in quanto la funzione nulla, che ha norma nulla, non è invertibile quindi non siamo
        in quel caso.
        Quindi, scegliendo \(\delta \leq 1 / ||A_0^{-1}||\)
        si ha che \(A\) è invertibile, quindi è un aperto.
        Cioè inq uesto caso \(||B|| < 1\) quindi \(\mathbf{1} - B\) è invertibile
        e visto che abbiamo scritto \(A\) come prodoto di operatori invertibili, è anch'esso invertibile.
    \end{enumerate}

    Mostriamo ora che l'inversa è continua.
    Da \(||B\circ A||_* \leq ||B||_* ||A||_*\)
    da qui segue che la composizione di questi lineari è continua.
    Siano \(B_n \to B\) e \(A_n \to A\) quindi \(B_n \circ A_n \to B\circ A\).
    Mostriamo che la mappa \(A \to A^{-1}\) è continua, cioè che
    \(A_n \to A \implies A_n^{-1} \to A^{-1}\). Abbiamo
    \begin{align*}
        A = A_0(\mathbf{1} - A_0^{-1}(A_0 - A)) = A_0(\mathbf{1} - B), \quad A^{-1} = (\mathbf{1} - B)^{-1} A_0^{-1}
    \end{align*}
    Vogliamo stimare \(||A^{-1} - A_0^{-1}||_*\).
    \begin{align*}
        A^{-1} - A_0^{-1} &=
        {(\mathbf{1} - B)}^{-1} A_0^{-1} - A_0^{-1}
        = \left[
            {(\mathbf{1} - B)}^{-1} - \mathbf{1}
        \right]A_0^{-1} = \left[
            \sum_{k=0}^\infty B^k - \mathbf{1}
        \right]A_0^{-1} = \left[\sum_{k=1}^\infty B^k\right] A_0^{-1} \\
        ||A^{-1} - A_0^{-1}||_*
        &= \left|\left| \left(\sum_{k=1}^\infty B^k\right) \circ A_0^{-1}
        \right|\right|
        \leq \left(\sum_{k=1}^\infty {(||B||_*)}^k\right) ||A_0^{-1}||
        = \frac{||B||_*}{1 - ||B||_*} ||A_0^{-1} ||
    \end{align*}
    pur di prendere \(||B||_* < 1\).
    Se \(A \to A_0 \implies ||B||_* \to 0\) allora \( A^{-1} - A_0^{-1} \to 0\).
}

\pagebreak

\section{Calcoli differenziale multivariabile}

\sdefinition{Limite multivariabile}{
    Sia \(f \colon \Omega \subseteq \mathbb{R}^n \to \mathbb{R}^m\)
    e sia \(a\) un punto di accimulazione per \(\Omega\).
    Allora
    \[
        \lim_{x\to a} f(x) = b \in \mathbb{R}^m
    \]
    se \(\forall \varepsilon > 0, \exists \delta > 0\)
    tale che \(\forall x \in \Omega\)
    \[
        0 < ||x-a||_1 < \delta
        \implies
        ||f(x) - b||_2 < \varepsilon
    \]
    dove le norme sono generiche.
}

Il concetto di limite ad infinito vuol dire uscire dai compatti.

\sdefinition{Limite a infinito multivariabile}{
    Sia \(f \colon \Omega \subseteq \mathbb{R}^n \to \mathbb{R}^m\)
    con \(\Omega\) illimitato.
    Allora
    \[
        \lim_{x\to \infty} f(x) = b \in \mathbb{R}^m
    \]
    se \(\forall \varepsilon > 0, \exists M > 0\)
    tale che \(\forall x \in \Omega\)
    \[
        ||x|| > M \implies ||f(x) - b|| < \varepsilon
    \]
}

e per il codominio

\sdefinition{Limite infinito multivariabile}{
    Sia \(f \colon \Omega \subseteq \mathbb{R}^n \to \mathbb{R}^m\)
    e \(a\) punto di accumulazione.
    Allora
    \[
        \lim_{x\to a} f(x) = \infty
    \]
    se \(\forall K > 0, \exists \delta > 0\)
    tale che per tutte le \(x \in \Omega\)
    \[
        0 < ||x-a|| < \delta
        \implies ||f(x)|| > K
    \]
}

\sdefinition{Cammino}{
    Un \emph{cammino} è un applicazione \(\varphi: (0,1) \to \mathbb{R}^n\) continua.
}

Sia \(\Phi_\Omega(a)\) l'insieme dei
cammini con sostegni in \(\Omega\)
e tale che \(\varphi(t) \to a\)
per \(t \to 0^+\)
e \(\varphi(t) \neq\) per tutte le \(t\).

\stheorem{}{
    \[
        \lim_{x\to a} f(x) = b
        \iff
        \lim_{t \to 0^+}
        f(\varphi(t)) = b
    \]
    per tutte le \(\varphi \in \Phi_\omega(a)\).
}

\sdefinition{Derivata direzionale}{
    TODO
    \[
        D_v f(a) = \lim_{t\to 0} \frac{f(a+vt) - f(a)}{t}
    \]
}

%%%%%%% 15/ott

La derivata parziale è come prendere una fetta della funzione multidimensionale,
nella direzione data dal vettore direzionale, e calcolare la verivata ad una singola variabile.

Le derivate parziali lungo i versori della base canonica si chiamano derivate parziali.

\sexample{}{
    Sia \[
        f(x,y) = \begin{cases}
            \frac{xy^2}{x^2 + y^2} & (x,y) \neq 0 \\
            0 & (x,y) = (0,0)
        \end{cases}
    \]
    Stabiliamo se le derivate direzionali
    di \(f\) esistono nell'origine e calcoliamo.
    Questa funzione è costante lungo gli assi cartesiani e vale zero.
    Quindi, le derivate parziali sono nulle.
    Sia \(v\) un qualsiasi versore. Possiamo già escludere i casi
    con \(v_i = 0\) in quanto sono le derivate parziali.
    Calcoliam quindi
    \begin{align*}
        \lim_{t\to 0} \frac{f(0 + tv_1, 0 + tv_2) - f(0,0)}{t}
        &= \lim_{t\to 0}
        \frac{1}{t} \frac{tv_1 t^2 v_2^2}{t^2v_1^2 + t^2v_2^2} \\
        &= \lim_{t\to 0} \frac{t^3 v_1v_2^2}{t^3(v_1^2 + t^2v_2^4)} \\
        &= \lim_{t\to 0} \frac{v_1v_2^2}{v_1^2 + t^2v_2^4} \\
        &= \frac{v_1v_2^2}{v_1^2} = \frac{v_2^2}{v_1}
    \end{align*}
    Quindi \(f\) ammette tutte le derivate direzionali.
    Controlliamo il limite di questa funzione nell'origine, per controllare
    se è quindi continua in quel punto.
    \begin{align*}
        \frac{\rho^3 |\cos\theta| \sin^2\theta}{
            \rho^2 (\cos^2 \theta + \rho^2 \sin^2 \theta)
        }
        &\leq \frac{\rho}{\cos^2\theta + \rho^2 \sin^4 \rho}
    \end{align*}
    Consideriamo
    \[
        f(y^2, y) = \frac{1}{2}
    \]
    Quindi abbimo trovato un cammino lungo il qualr la funzione non tende a zero.
    Avremmo potuto semplicemente considerare la stima
    \[
        \frac{|xy^2|}{x^2 + y^4} \leq \frac{1}{2}
    \]
}

L'esistenza delle derivate parziali
non implica la differenziabilità.

Nelle funzioni di una variabili l'esistenza del rapporto incrementale
e l'approssimazione lineare local di una funzione coincidono e sono equivalenti alla
differenziabilità, ma in più variaibli non coincidono.

\sdefinition{}{
    Sia \(f \colon \Omega \subseteq \mathbb{R}^n \to \mathbb{R}\)
    con \(\Omega\) aperto e \(a \in \Omega\).
    Diremo che \(f\) è differenziabile in \(a\)
    se esiste una forma lineare (funzionale lineare),
    o equivalentemente se esiste \(\lambda \in \mathbb{R}^n\) tale che
    \[
        f(a + h) = f(a) + \lambda \lambda + o(||h||)
    \]
    per \(h \to 0\).
    Tale forma lieare è detta differenziale di \(f\) in \(a\),
    denotata \(\text{d}f(a)\), che quindi manda \(h\) in \(h\lambda\).
}

(Teorema di Riesz, infatti il fatto che la derivata direzionale è data da \(\nabla f(a) \cdot v\)).
Se tale \(\lambda\) esiste, è unico.
Supponiamo che \(\lambda,\mu\) soddisfino l'uguaglianza.
Sottraendo membro a membro otteniamo
\begin{align*}
    (\lambda - \mu) h = o(||h||)
\end{align*}
per \(h\to 0\).
Quindi vogliamo
\begin{align*}
    \lim_{h\to 0} \frac{(\lambda - \mu)h}{||h||} = 0
\end{align*}
ma 
\[
    (\lambda - \mu) \frac{h}{||h||}
\]
non tende a zero per \(h\to 0\).
Per mostrarlo scegliamo il cammino \(\varphi(t) = t(\mu-\nu)\) e
\[
    (\lambda - \mu) \frac{t(\lambda - \mu)}{|t| \cdot ||\lambda - \mu||}
    = \frac{t}{|t|}||\lambda - \mu||
\]
che nont ende a zero perché \(\lambda \neq \mu\).
Quindi non esistono \(\lambda \neq \mu\) tale che valga l'uguaglianza.

\stheorem{}{
    Se \(f\) è differenziabile in \(a\), allora
    \begin{enumerate}
        \item \(f\) è continua in \(a\);
        \item \(\forall v \in \mathbb{R}^n\), esiste la derivata direzionale
        \(D_v f(a) = \lambda \cdot v\);
        \item \(\lambda = \nabla f(a)\)
    \end{enumerate}
}

\sproof{}{
    \begin{enumerate}
        \item Per ipotesi esiste \(\lambda \in \mathbb{R}^n\)
        tale che
        \[
            f(a + h) - f(a) = \lambda \cdot h + \varepsilon(h)
        \]
        con \(\varepsilon(h) = o(||h||)\).
        Quindi, \begin{align*}
            |f(a+h) - f(a)| &= 
            |\lambda \cdot h + \varepsilon(h)| \\
            &= |\lambda h| + |\varepsilon(h)| < ||\lambda||
            ||h|| + ||h|| \frac{|\varepsilon(h)|}{||h||} \\
            &\leq ||h|| \left[
                ||\lambda|| + \frac{|\varepsilon(h)|}{||h||}
            \right] \to 0
        \end{align*}
        Abbiamo usato la disuguaglianza di Cauchy-Schwarz.
        \item Calcoliamo
        \begin{align*}
            \frac{f(a + tv) - f(a)}t
            &= \frac{\lambda \cdot tv + \varepsilon(tv)}{t} \\
            &= \lambda v + \frac{\varepsilon(tv)}{t} \\
            &= \lambda v + \frac{\varepsilon(tv)}{t ||v||} \\
            &= \lambda v + \frac{|t|}{t} \frac{\varepsilon(tv)}{||tv||} \\
            &\to \lambda v
        \end{align*}
        \item \[
            \frac{\partial f}{\partial x_j}(a)
            = D_v f(a) = \lambda e_j = \lambda_j
        \]
        quindi \(\lambda = \nabla f(a)\).
    \end{enumerate}
}

Per sabilire se una funzione \(f\) è differenziabile in \(a\),
basta controllare se le derivate parziali esistono.

\scorollary{Formula del gradiente}{
    Se \(f\) è differenziabile in \(a\),
    allora \(D_v f(a) = \nabla f(a) \cdot v\).
}

%\newcommand\restr[2]{{% we make the whole thing an ordinary symbol
%\left.\kern-\nulldelimiterspace % automatically resize the bar with \right
%#1 % the function
%\vphantom{\big|} % pretend it's a little taller at normal size
%\right|_{#2} % this is the delimiter
%}}
%
%\[
%    \restr{\frac{\partial f}{\partial x_i}}{a} \qquad
%    \frac{\partial f}{\partial x_i} \tikz{
%        \draw (0,0) -- (0, -0.5);
%    }
%\]

\sexample{}{
    Studia se \[
        f(x,y) = x^5 \sqrt{|y|}
    \]
    è differenziabile nei punti del tipo \((x_0, 0)\) lungo l'asse \(x\).
    Nel punto zero potremo avere dei problemi per la derivata parziale di \(y\),
    quindi dovremo usare la definizione e non le regole delle derivate.
    Siccome \(f(x,0) = 0\) la derivata nei punti della forma \((x_0, 0)\) è 0.
    Calcoliamo
    \begin{align*}
        \frac{\partial f}{\partial y}(x_0, 0)
        &= \lim_{y \to 0} \frac{f(x_0, y) - f(y_0, 0)}{y} \\
        &= \lim_{y\to 0} \frac{x_0^5 \sqrt{||y||} - 0}{y} \\
        &= \lim_{y\to 0} x_0^5 \frac{\sqrt{|y|}}{y} \\
        &= \lim_{y\to 0} x_0^5 \,\text{sgn} y |y|^{-1/2} = \begin{cases}
            0 & x_0 = 0 \\
            \nexists & x_0 \neq 0
        \end{cases}
    \end{align*}
    Ma quindi \(f\) non è differenziabile nei punti \((x_0, 0)\) se \(x_0 \neq 0\).
    Invece nel punto dell'origine abbiamo \(\nabla f(0,0) = (0,0)\)
    \begin{align*}
        \lim_{(h,k) \to (0,0)} \frac{f(h,k) - f(0,0) - \nabla f(0,0) \cdot (h,k)}{\sqrt{h^2 + k^2}}
        &= \lim_{(h,k) \to (0,0)} \frac{h^5 \sqrt{|k|}}{\sqrt{h^2 + k^2}} \\
    \end{align*}
    Siccome \[
        \frac{|ab|}{a^2 + b^2} \leq \frac{1}{2}
    \]
    Abbiamo
    \[
        \left|\frac{h^5 \sqrt{|k|}}{\sqrt{h^2 + k^2}}\right|
        = {\left(
            \frac{|h| \cdot |k|}{h^2 + k^2}
        \right)}^{1/2} {|h|}^{9/2} \leq \frac{1}{\sqrt{2}} {|h|}^{9/2}
    \]
    e quindi tende a zero. Allora \(f\) è differenziabile nell'origine.
}

Se \(f\) è differenziabile in \(a\) possiamo definire l'iperpiano tangente
\[
    f(a) + \nabla f(a) \cdot (x-a)
\]

\stheorem{Teorema del differenziale totale}{
    Sia \(f \colon \Omega \to \mathbb{R}\), con \(\Omega \in \mathbb{R}^n\) aperto,
    \(x_0 \in \Omega\). Se \(\exists r > 0\)
    tale che \(B_r(x_0) \subseteq \Omega\) e tutte le derivate parziali di \(f\)
    esistono in \(B_r(x_0)\) e sono continue in \(x_0\),
    allora \(f\) è differenziabile in \(x_0\).
}
È una condizione sufficiente.
\sproof{}{
    Facciamo il caso \(n=2\).
    Per ipotesi sappiamo che \(\nabla f(x_0)\) esiste.
    Dobbiamo msotrare che
    \[
        \lim_{(x,y) \to (x_0,y_0)} \frac{
            f(x,y) - f(x_0, y_0) - \nabla f(x) \cdot (x-x_0, y-y_0)
        }{\sqrt{{(x-x_0)}^2 + {(y-y_0)}^2}} = 0
    \]
    Calcoliamo l'incremento \(f(x,y) - f(x_0, y_0)\). Piuttosto che congiungere
    i due punti mediante un segmento, facciamo lo stesso percorso con due segmenti nella direzione degli assi,
    quindi ho una variabile sola al posto di due, due volte. Quindi passiamo dal punto intermedio
    \((x,y_0)\). Abbiamo allora
    \begin{align*}
        f(x,y) - f(x_0, y_0) &= \left[
            f(x,y) - f(x,y_0)
        \right] + \left[
            f(x,y_0) - f(x_0,y)
        \right]
    \end{align*}
    Possiamo applicare il teorema di Lagrange
    \begin{align*}
        f(x,y) - f(x,y_0) = \frac{\partial f}{\partial y}(x, \xi) (y-y_0)
    \end{align*}
    con \(\xi \in [y, y_0]\), visto che la derivata è parziale.
    Considerando allo stesso modo \(f(x, y_0)\) e applicando Lagrange si ottiene
    \[
        f(x, y_0) - f(x_0, y_0) = \frac{\partial f}{\partial x}(\mu, y_0)(x-x_0)
    \]
    dove \(\eta \in [x, x_0]\). Sostituendo troviamo
    \begin{align*}
        &\frac{
            f(x,y) - f(x_0, y_0) - \frac{\partial f}{\partial x}(x_0, y_0)(x-x_0)
            - \frac{\partial f}{\partial y}(x_0, y_0)(y-y_0)
        }{
            \sqrt{
                {(x-x_0)}^2 + {(y-y_0)}^2
            }
        }
        \\ &= \left|
            \frac{
                \left[
                \frac{\partial f}{\partial x}(\eta, y_0) -
                \frac{\partial f}{\partial x}(x_0, y_0)
            \right](x-x_0)
            +
            \left[
                \frac{\partial f}{\partial y}(x, \xi) -
                \frac{\partial f}{\partial y}(x_0, y_0)
            \right](x-x_0)
            }{
                \sqrt{
                    {(x-x_0)}^2 + {(y-y_0)}^2
                }
            }
        \right| \\
        &= \left|
            \frac{\partial f}{\partial x}(\eta, y_0)
            - \frac{\partial f}{\partial x}(x_0, y_0)
        \right|
        \frac{|x-x_0|}{\sqrt{
            {(x-x_0)}^2 + {(y-y_0)}^2
        }}
        +
        \left|
            \frac{\partial f}{\partial y}(x, \xi)
            - \frac{\partial f}{\partial y}(x_0, y_0)
        \right|
        \frac{|y-y_0|}{\sqrt{
            {(x-x_0)}^2 + {(y-y_0)}^2
        }} \\
        &\leq
        \left|\frac{\partial f}{\partial x}(\eta, y_0) - \frac{\partial f}{\partial x}(x_0, y_0)\right|
        +\left|\frac{\partial f}{\partial y}(x, \xi) - \frac{\partial f}{\partial y}(x_0, y_0)\right|
    \end{align*}
    I due termini con le radici a denominatori sono minori di 1.
    Per il teorema del confronto, se \((x,y) \to (x_0, y_0)\)
    implica che \(\xi \to y_0\) e \(\eta \to x_0\).
    Per la continuità della derivate parziali segue la tesi che il valore tende a zero.
}

\scorollary{}{
    Se in \(\Omega\) tutte le derivate parziali
    di \(f\) esistono e sono continue,
    allora \(f\) è differenziabile in \(\Omega\).
}

\sdefinition{}{
    Sia \(f \colon \Omega \to \mathbb{R}\), con \(\Omega \in \mathbb{R}^n\) aperto.
    Se le derivate parziali esistono e sono continue in \(\Omega\), diremo che
    \(f\) è differenziabile con continuità e scriviamo \(f \in C^1(\Omega)\).
}

Per il teorema del differenziale totale
le classi \(C\) sono annidate.

\sexercise{}{
    Studiare la differenziabilità di \[f(x,y) = \ln(1 + xy)-x^2y\].
    Il dominio massimalel è \(\{(x,y) \,|\, xy > -1\}\).
    Troviamo le due derivate parziali
    \[
        \frac{\partial f}{\partial x} = \frac{y}{1 + xy} - 2xy,
        \quad
        \frac{\partial f}{\partial y} = \frac{x}{1 + xy} - x^2
    \]
    Siccome le derivate sono continue ed esistono ovunque nel dominio (in particolare nell'origine)
    la funzione è differenziabile.
}

\sexercise{}{
    Studiare continuità e differenziabilità nell'origine di \[
        f(x,y) = \begin{cases}
            \sqrt{x^2 + y^2} \left[ 1 - e^{-\frac{x^2 + y^2}{|x|}}\right] & x \neq 0 \\
            0 & x=0
        \end{cases}
    \]
    Non conviene usare il tereoma del differenziale totale perché la funzione è a tratti
    e quindi dovremmo usare la definizione di derivata.
    Cominciamo con la continuità.
    Calcoliamo il limite \((x,y) \to (0,0)\)
    notando che \(|f| \leq \sqrt{x^2 + y^2} \to 0\).
    Quindi è continua nell'origine. Calcoliamo adesso le derivate direzionali.
    Sia \(v = (v_1, v_2)\) un versore
    \begin{align*}
        \frac{f(tv) - f(0)}{t} &= \frac{1}{t} \left(
            |t| \left[1 - e^{\frac{t^2}{|t| \cdot |v_1|}}\right]
        \right) \\
        &= \text{sgn} t \left[1 - e^{- \frac{|t|}{|v_1|}}\right] \to 0
    \end{align*}
    quando \(v_1 \neq 0\).
    Nel caso \(v_1 = 0\) abbiamo il vettore \((0, \pm 1)\) della direzione dell'asse \(y\).
    In tal caso abbiamo
    \[
        \frac{\partial f}{\partial y} (0,0) = 0
    \]
    in quanto \(f\) è identicamente nulla in quell'asse.
    Ma \(f\) non è differenziabile in quanto il seguente non tende a zero
    \begin{align*}
        \left|
            \frac{
                f(x,y) - f(0,0) - \nabla f(0,0) \cdot (x,y)
            }{\sqrt{x^2 +y^2}}
        \right|
        &= 1 - e^{-\frac{x^2 + y^2}{|x|}}
    \end{align*}
    e considerando l'esponente. Possiamo trovare un cammino per il quale il limite non tende a zero.
    Le rette non vanno bene, dobbiamo prendere un cammino che è tangente a \(x=0\).
    Prendiamo per esempio \((t^2, t)\). L'esponente diventa \(-(t^4 + t^2)/t^2 \to -1\).
    Quindi tutto il limite tende a \(1-e^{-1} \neq 0\).
}

La forma differenziabile \(\text{d}f(-) \colon \Omega \to \mathcal{L}(\mathbb{R}^n, \mathbb{R})\)
e quindi associa ad ogni punto il funzionale lineare differenziale.
Quindi
\[
    \text{d}f(a) \colon \mathbb{R}^n \to \mathbb{R}
\]
che associa \(h \to \nabla f(a) \cdot h\).
In due variabili per esempio
\[
    \text{d}f(a)(x-a) = \nabla f(a)(x-a)= \frac{\partial f}{\partial x}(a)(x-e_1) + \frac{\partial f}{\partial x}(x-e_2)
\]
la cui applicazuine può essere scritta con le proiezioni
\[
    \text{d}f(a) = \frac{\partial f}{\partial x}(a)\,dx + \frac{\partial f}{\partial x}\,dy
\]

Nota: \(\nabla f(a)\) è il vettore che punta nella direzione di massima crescita locale.
Infatti, \(f(a+h) - f(a) = \nabla f(a) \cdot h + \varepsilon(h)\)
con \(\varepsilon(h) = o(||h||)\).
Abbiamo
\begin{align*}
    f(a+h) - f(a) &= ||\nabla f(a)|| \cdot ||h|| \cos\theta + \varepsilon(h) \\
    &= ||h|| \left[
        ||\nabla f(a)|| \cos\theta + \frac{o(||h||)}{||h||}
    \right]
\end{align*}
dove \(\theta\) è l'nagolo fra \(h\) e \(\nabla f(a)\).
Quindi ciò assume valore massimo quando \(\theta = 0\),
cioè quando la direzione è il verso di \(\nabla f(a)\).

\pagebreak

\subsection{Funzioni a valori vettoriali}

Consideriamo \(f\colon \Omega \subseteq \mathbb{R}^n \to \mathbb{R}^m\) con \(\Omega\)
aperto. Abbiamo quindi
\[
    f = \sum_{i=1}^n f_i e_i, \quad f_i \colon \Omega \mathbb{R}
\]
Sia \(a\in \Omega\) e sia \(v \in \mathbb{R}^n\). Definiamo la derivata direzionale
\[
    D_v f(a) = \lim_{t\to 0} \frac{
        f(a+tv) - f(a)
    }{t} \in \mathbb{R}^m
\]
e le componenti sono le derivate direzionali delle funzioni a valore reale.
Posso definire un totale di \(mn\) derivate parziali
\[
    \frac{\partial f_i}{\partial x_j} (a)
\]
Possiamo quindi definire la matrice Jacobiana come la matrice
\(J(a)_{i,j} = \frac{\partial f_i}{\partial x_j} (a)\).
Quindi le righe della matrice sono i trasporti dei gradienti \(\nabla f_j(a)\).

\sdefinition{Differenziabilità}{
    Una funzione \(f\colon \Omega \subseteq \mathbb{R}^n \to \mathbb{R}^m\) con \(\Omega\)
    aperto è differenziabile in \(a\in \Omega\) se \(\exists L \in \mathcal{L}(\mathbb{R}^m, \mathbb{R}^n)\)
    tale che
    \[
        f(a+h) - f(a) = L(h) + \varepsilon(h)
    \]
    dove \(\varepsilon(h) = o(||h||)\).
    Tale applicazione lineare di chiama differenziale.
}

Equivalentemente, \(f\) è differenziabile
in \(a\) se esiste una matrice \(\Lambda \in M_{m\times n}(\mathbb{R})\)
tale che \[ f(a+h) - f(a) = \Lambda h + \varepsilon(h) \]
Equivalentemente \(f\) è differenziabile se e solo se \(f_i\) sono tutte differenziabili.

\stheorem{Teorema del differenziale totale}{
    Vale l'analogo teorema: se tutte le derivate parziali delle componenti di \(f\)
    esistono in un intorno di \(a\) e sono continue in a allora \(f\) è differenziabile in \(a\)
}

\stheorem{}{
    Se \(f\) è differenziabile in \(a\) allora \(f\) è continua
    in \(a\), \(\forall v \in \mathbb{R}^n\) tale che \(||v|| = 1\),
    \[
        \exists D_vf(a) = \Lambda v
    \]
    e \(\Lambda = J(a)\).
}
Cioè il differenziale (differenziale primo) \(L\) è rappresentato nella base canonica dalla matrice Jacobiana,
\[
    \text{d}f_a h = Jf(a) h = \sum_{i=1}^m \left(
        \sum_{j=1}^n \frac{\partial f_i}{\partial x_j}(a)h_j
    \right)e_i
\]

\subsection{Chain rule}

\stheorem{Chain rule}{
    Siano \(f \colon \mathbb{R}^n \to \mathbb{R}^m\)
    e \(g \colon \mathbb{R}^m \to \mathbb{R}^p\) funzioni
    tali che \(f\) è differenziabile nel punto \(a\) e \(g\)
    differenziabile in \(f(a)\).
    Allora, \(g\circ f\) è differenziabile in \(a\) e vale
    \[
        \text{d}(g\circ g)(a) = \text{d}g(f(a)) \circ \text{d}f(a)
    \]
    O equivalentemente in forma matriciale
    \[
        J(g \circ f)(a) = Jg(f(a)) Jf(a)
    \]
}

Sia \(h = g\circ f \colon \mathbb{R}^m \to \mathbb{R}^n\).
Calcoliamo il seguente, che è il prodotto della \(i\)-esima riga di \(Jg(f(a))\)
per la \(j\)-esima colonna di \(Jf(a)\)
\begin{align*}
    \frac{\partial h_i}{\partial x_i}(a)
    &= \sum_{k=1}^m \frac{\partial g_i}{\partial y_k} (f(a))
    \cdot \frac{\partial f_k}{\partial x_j}(a)
\end{align*}

\sexample{}{
    Siano \(f\colon \mathbb{R}^2 \to \mathbb{R}^3\) e 
    \(g\colon \mathbb{R}^3 \to \mathbb{R}^2\). Chiamiano \(x,y\) le coordinate di \(f\)
    e \(u, v, w\) quelle di \(g\). Sia \(h = g \circ f\) quindi \(h(x,y) = (h_1(x,y), h_2(x,y))\).
    Calcoliamo per esempio
    \begin{align*}
        \frac{\partial f_2}{\partial x}(a)
        &= \frac{\partial g_2}{\partial u}(f(a))
        \cdot \frac{\partial f_1}{\partial x}(a)
        + \frac{\partial g_2}{\partial v}(f(a)) \cdot
        \frac{\partial f_2}{\partial x}(a)
        + \frac{\partial g_2}{\partial w}(f(a))
        \cdot \frac{\partial f_3}{\partial x}(a)
    \end{align*}
}

\sexample{}{
    Siano \(f\colon \Omega \subseteq \mathbb{R}^2 \to \mathbb{R}^2\) e 
    \(g\colon \mathbb{R}^2 \to \mathbb{R}\). Chiamiamo le variabili di \(f\)
    \((\rho, \theta)\) e di \(g\) dicimo \((x,y)\).
    Definiamo \(f(p, \theta) = (\rho\cos\theta,\rho\sin\theta)\).
    Sia \(h = g \circ f\) quindi \(h(\rho, \theta) = g(\rho\cos\theta, \rho\sin\theta)\).
    Calcoliamo
    \begin{align*}
        \frac{\partial h}{\partial \rho}
        = \frac{\partial g}{\partial x}(\rho\cos\theta, \rho\sin\theta) \cos\theta
        + \frac{\partial g}{\partial y}(\rho\cos\theta, \rho\sin\theta) \sin\theta
    \end{align*}
    e
    \begin{align*}
        \frac{\partial h}{\partial \theta}
        = -\frac{\partial g}{\partial x}(\rho\cos\theta, \rho\sin\theta) \rho\sin\theta
        + \frac{\partial g}{\partial y}(\rho\cos\theta, \rho\sin\theta) \rho\cos\theta
    \end{align*}
}

\sexercise{}{
    Studiare la continuità e differenziabilità nell'origine di
    \[
        f(x,y) = \begin{cases}
            \frac{|x|^{1/4} \ln(1 + x^2 + y^2)}{{(x^2 + y^2)}^\alpha} & (x,y) \neq (0,0) \\
            0 & (x,y) = (0,0)
        \end{cases}
    \]
    al variare di \(\alpha\). Il dominio è tutto \(\mathbb{R}^2\).
    Cominciamo studiando il limite nell'origine. In coordinate polari
    \begin{align*}
        F(\rho,\theta) &= \frac{\rho^{1/4} {|\cos\theta|}^{1/4} \ln(1 + \rho^2)}{\rho^{2\alpha}}
        \leq \frac{\rho^{1/4} \ln(1 + \rho^2)}{\rho^{2\alpha}} \\
        &= \rho^{\frac{1}{4} - 2 \alpha} \ln(1 + \rho^2) \leq \rho^{\rho^{\frac{1}{4} - 2 \alpha + 2}}
    \end{align*}
    Quindi sicuramente tende a zero per \(\alpha < 9/8\), dove \(f\) è continua.
    Per dimostrare che il bound è sharp supponiamo \(\alpha \geq 9/8\) e consideriamo il cammino
    \(x=0\). Abbiamo
    \begin{align*}
        f(x,0) &= \frac{{|x|}^{1/4} \ln(1 + x^2)}{{(x^2)}^\alpha} \\
        &= \frac{{|x|}^{1/4} \ln(1 + x^2)}{{|x|}^{2\alpha}}
        \sim {|x|}^{\frac{1}{4} + 2 - 2\alpha}
    \end{align*}
    Da notare il modo necessario.
    Tale valore non tende mai a zero.
    Supponiamo \(\alpha < 9/8\)
    e studiamo la differenziabilità.
    Non conviene sare il teorema del differenziale totale quindi calcoliamo le derivate parziali.
    Chiaramente
    \[
        \frac{\partial f}{\partial y}(0,0) = 0
    \]
    Mentre
    \begin{align*}
        \frac{\partial f}{\partial x}(0,0)
        &= \lim_{x\to 0} \frac{f(x,0) - f(0,0)}{x} \\
        &= \lim_{x\to 0} \frac{{|x|}^{1/4} \ln(1 + x^2)}{x{|x|}^{2\alpha}} \\
        &= \lim_{x\to 0} \text{sgn} x \cdot {|x|}^{\frac{1}{4} + 1 - 2\alpha} \\
        &= \lim_{x\to 0} \text{sgn} x \cdot {|x|}^{\frac54 - 2\alpha}
    \end{align*}
    che tende a zero per \(\alpha < \frac58\), altrimeti il limite non esiste.
    Quindi se \(\alpha \geq \frac58\) la derivata parziale in 0 non esiste.
    Sia \(\alpha < 5/8\), dove abbiamo il gradiente. Calcoliamo quindi questo limite.
    \begin{align*}
        \frac{f(h,k) - f(0,0) - \nabla f(0,0) \cdot (h,k)}{\sqrt{h^2 + k^2}}
        &= \frac{f(h,k)}{\sqrt{h^2 + k^2}} \\
        &= \frac{{|x|}^{1/4} \ln(1 + h^2 + k^2)}{{(h^2 + k^2)}^{\alpha + 1/2}}
    \end{align*}
    che tende a zero se e solo se \(\alpha + \frac{1}{2} < 9/8\) come detto prima,
    quindi \(\alpha < 5/8\).
}

\sexercise{}{
    Studiare se la seguente funzione è differenziabile nell'origine
    \[
        f(x,y) = \begin{cases}
            e^{-\frac{1}{x^2}} \sin \sqrt{x^2 + y^2} & x \neq 0 \\
            0 & x = 0
        \end{cases}
    \]
}

\stheorem{Teorema di Lagrange}{
    Sia \(\Omega \subseteq \mathbb{R}^n\) open and \([a,b]\subseteq \Omega\).
    Let \(f\colon \Omega \to \mathbb{R}\) differentiable in \(\Omega\).
    Then, \(\exists \alpha \in (a,b)\) such that
    \[
        f(b)-f(a) = \nabla f(\alpha) (b-a)
    \]
}

\sproof{}{
    Costruiamo una funzione che parametrizza il segmento
    \(\varphi \colon [0,1] \to \Omega\) e quindi
    \[
        \varphi(t) = a+t(b-a)
    \]
    che è differenziabile. COnsideriamo la funzione \(F = f \circ \varphi \colon [0,1] \to \mathbb{R}\).
    \(F\) è continua in \([0,1]\) ed è differenziabile in \((0,1)\).
    Ci ricolleghiamo quindi al teorema di Lagrange in una variabile.
    \begin{enumerate}
        \item \(F(1) - F(0) = F'(\theta)(1-0)\);
        \item \(F(1) = f(\varphi(1)) = f(b)\);
        \item \(F(0) = f(a)\);
        \item \(F'(t) = (f\circ \varphi)'(t) = Jf(\varphi(t)) J \varphi(t) = \nabla f(\varphi(t)) \varphi'(t)\);
        \item \(F'(\theta) = \nabla f(\varphi(\theta))\varphi'(\theta) = \nabla f(\varphi(\theta))(b-a)\), basta porre \(a = \varphi(\theta)\).
    \end{enumerate}
}

È falso per le funzioni a valori vettoriali.

\stheorem{Teorema dell'increment fissato}{
    Sia \(\Omega \subseteq \mathbb{R}^n\) aperto e sia \(f\colon \Omega \to \mathbb{R}^m\)
    differenziabile in \(\Omega\). Sia \([x,y] \subseteq \Omega\).
    Allora 
    \[
        ||f(x)-f(y)|| \leq ||x-y|| \sup_{0 < t < 1} || df(x + t(y-x)) ||
    \]
}

\sproof{Teorema dell'increment fissato}{
    Consideriamo \(n=1\). Il teorema
    dice che data \(g\colon [a,b] \to \mathbb{R}^m\) continua su \([a,b]\)
    e differenziabile su \((a,b)\). Se \(||g'(x)|| \leq \alpha\) per tutte le \(x\)
    in \((a,b)\), allora
    \[
        ||g(b) - g(a)|| \leq \alpha(b-a)
    \]
    Poniamo \(v = g(b) - g(a)\) e definiamo la funzione ausiliaria \(\varphi(x) = g(x) \cdot v\)
    che ha forma \([a,b] \to \mathbb{R}\). \(\varphi\) è continua su \([a,b]\) e differenziabile su
    \((a,b)\). Per il tereoam di Lagrange \(\exists \theta \in (a,b)\) tale che
    \(\varphi(b) - \varphi(a) = \varphi'(\theta)(b-a)\).
    Quindi \[
        g(b) \cdot v - g(a) \cdot v = g'(\theta) \cdot v(b-a)
        \leq ||g'(\theta)|| \cdot || v || \cdot (b-a)
    \]
    per Cauchy-Schwarz.
    Usimao il fatto che \(v = g(b) - g(a)\).
    Sostituendo otteniamo
    \[
        ||g(b) - g(a)||^2 \leq ||g'(\theta)|| \cdot || g(b) - g(a) || (b-a)
    \]
    Se \(g(b) = g(a)\) la disuguaglianza è automaticamente verificata. Supponiamo \(g(b) \neq g(a)\), dividendo per
    \(||g(b) - g(a)||\) ottiene la tesi.
    Consideriamo ora il naso \(n>1\).
    Consideriamo la funzione ausiliaria \(\Psi(t) = f(x + t(y-x))\)
    che ha forma \([0,1] \to \mathbb{R}^m\) continua in \([0,1]\) e differenziabile in \((0,1)\).
    Per ciò che abbiamo appena mostrato sappiamo che
    \(||\Psi(1) - \Psi(0)|| \leq \alpha(1-0)\) for
    \[
        \alpha = \sup_{0<t<1} ||\Psi'(t)||
    \]
    Calcoliamo
    \[
        \Psi'(t) = df(x + t(y+x))(y-x)
    \]
    quindi otteniamo
    \[
        ||f(y)-f(x)|| \leq \sup_{0<t<1} ||df(x + t(y-x))(y-x)|| \leq
        ||y-x|| \leq ||y-x|| \sup_{0<t<1} ||df(x + t(y-x))||
    \]
}

\scorollary{}{
    Let \(f\colon \Omega \to \mathbb{R}^m\), \(\Omega \subseteq \mathbb{R}^n\) aperto connesso,
    \(f\) differenziabile in \(\Omega\).
    Allora \(f\) è costante in \(\Omega\) se e solo se
    \[
        \frac{\partial f_i}{\partial x_j} =0
    \]
    in \(\Omega\) per tutte le \(i,j\).
}

\sproof{}{
    \iffproof{Ovvia}{
        Sia \(x_0\in \Omega\) e poniamo \(E_{x_0} = \{x \in \Omega \,|\, f(x) = f(x_0) \}\).
        Vogliamo mostrare che \(E_{x_0} = \Omega\).
        Basta mostrare che \(E_{x_0} \neq \emptyset\) in quanto \(x_0 \in E_{x_0}\)
        e che \(E_{x_0}\) è aperto e chiuso in \(\mathbb{R}^m\).
        Mostriamo che \(E_{x_0}\) è aperto. Sia \(y \in E_{x_0}\). Sicuramente
        \(\exists r > 0\) tale che \(B_r(y) \subseteq \Omega\).
        Sia \(x \in B_r(y)\). \(B_r(y)\) è convessa e quindi
        \([x,y] \subseteq B_r(v)\). Per il torema dell'incremento finito 
        \[
            ||f(x) - f(y)|| \leq ||x-y|| \sup_{0 < t <1} ||df(x + t(y-x))|| = 0
        \]
        perché \(df = 0\).
        Quindi \(f(x)=f(y) = f(x_0)\) e \(x \in E_{x_0}\), cioè \(y\) è interno
        e quindi \(E_{x_0}\) è aperto.
    }
}

\end{document}
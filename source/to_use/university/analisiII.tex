\documentclass[a4paper]{article}

\usepackage{amsmath}
\usepackage{amssymb}
\usepackage{stellar}
\usepackage{parskip}
\usepackage{fullpage}
\usepackage{wrapfig}
\usepackage{tikz}

\usetikzlibrary{arrows}
\usetikzlibrary{decorations.pathreplacing}
\usetikzlibrary{cd}

\title{Analisi II}
\author{Paolo Bettelini}
\date{}

\begin{document}

\maketitle
\tableofcontents

\section{Bolle}

L'insieme vuoto e l'insieme \(X\) sono aperti e chiusi.

\sproposition{}{
    L'unione di aperti non numerabile è aperta, mentre l'intersezione è aperta solo se finita.
}

\sproof{}{
    Per dimostrare quest'ultima lo facciamo su due insiemi e il resto è per induzione.
    Prendiamo un punto nell'interezione e prendiamo le due bolle dentro gli insiemi centrate
    nel punto. Siccome hanno lo stesso centro la loro intersezione è sempre una bolla di raggio
    il minore fra i due.
}

La metrice discreta può generare una bolla che è un singoletto.

\sproposition{}{
    L'unione di chiusi finiti è chiusa. L'intersezione qualsiasi è chiusa.
}

Ogni singoletto è chiuso. Per dimostrarlo mostriamo che nel complementare
esiste una bolla che non interseca il punto (vero per proprietà di Hausdorff).

Tutti i punti di accumulazione sono dei punti aderenti.
Tutti i punti di un sottoinsieme sono aderenti per il sottoinsieme.
Ogni punto o è di accumulazione o è isolato.

Se x_0 è aderente ad E, x_0 può essere un punto di E oppure no.
Se x_0 è punto di accumulazione per E, in ogni bolla centrata in x_0 cadono inifniti punti.

\sproposition{}{
    \(E^\circ\) è aperto. \(E\) è aperto se e solo se \(E = E^\circ\).
    \(E^\circ\) è il più grande aperto contenuto in \(E\).
    
    \(\overline{E}\) è chiuso. \(E\) è chiuso se e solo se \(E = \overline{E}\).
    La chiusura di \(E\) è il più piccolo chiuso contenente \(E\).
}

\sproof{per l'interno}{
    Dimostriamo che \(E^\circ\) è aperto. Sia \(x_0 \in E^\circ\).
    un punto interno ad \(E\), quindi esiste una bolla centrata in tale punto che è contenuta in \(E\).
    Prendiamo un altro punto \(y\) in questa bolla. Possiamo costruire una inner bolla centrata in \(y\)
    con un raggio sufficientemente piccolo da rimanere nella bolla più grande. Quindi il punto \(y\)
    è a sua volta interno, quindi tutta la bolla centrata in \(x_0\) è in \(E^\circ\) e quindi è aperto.
    
    Dimostriamo ora che se \(E\) è aperto allora \(E=E^\circ\) (l'altra implicazione è ovvia).
    Per fare ciò dimostriamo che \(E^\circ\) è il più grande aperto in \(E\).
    Osserviamo che \(E^\circ\) fa parte della famiglia degli aperti di \(X\)
    contenuti in \(E\). Sia \(A\) un aperto contenuto in \(E\). VOglio dimostrare che \(A \subseteq E^\circ\).
    Sia \(x_0 \in A\). \(A\) èunione di bolle quindi esiste unr aggio tale che la
    bolla centrata in \(x_0\) di tale raggio è contenuta in \(A\) che è contenuto in \(E\).
    Quindi, \(x_0\) è interno ad \(E\) e \(x_0 \in E^\circ\) e \(A \subseteq E^\circ\).
    Supponiamo ora che \(E\) sia aperto. Allora \(E\) fa parte della famiglia degli aperti
    di \(X\) contenuti in \(E\). Devo avere \(E \subseteq E^\circ\).
    Dato che \(E^\circ \subseteq E\) allora \(E^\circ = E\).
}

% lezione 2

Dire che un insieme è dentro in un altro significa dire che la sua chiusura coincide con l'insieme.
Tipo la chiusura di Q è R quindi Q è denso in R.


\sdefinition{Limitato}{
    Se è contenuto in una bolla
}

\sdefinition{Diametro}{
    è il sup della metrica su tutte le coppie.
}

\sdefinition{Ricoprimento}{
    Sia \(E\) un sottoinsieme di uno spazio metrico \(X\). Una famiglia
    \[
        \{G_\alpha\}_{\alpha \in A}
    \]
    è un ricoprimento apert di \(E\) se
    \[
        E \subseteq \bigcup_{\alpha \in A} G_\alpha
    \]
}

\sdefinition{Sottoricoprimento}{
    Un Sottoricoprimento di 
    \[
        \{G_\alpha\}_{\alpha \in A}
    \]
    è una sottofamiglia di \(G_\alpha\)
    tale che continua a ricoprire. Cioè ne scarto alcuni ma deve comunque rimanere
    una copertura.
}

\sdefinition{Compatto}{
    Uno spazio metrico \(X\) è compatto se ogni ricoprimento aperto di \(E\)
    ammette un sottoricoprimento finito.
}

Ogni insieme finito è compatto.


\stheorem{}{
    Sia  \(X\) uno spazio metrico e \(E\) un sottoinsieme di \(X\) compatto.
    \begin{enumerate}
        \item \(E\) è limitato;
        \item \(E\) è chiuso;
        \item Ogni sottoinsieme infinito di \(E\) ha almeno un punto di accumulazione in \(E\).
    \end{enumerate}
}

\sproof{}{
    \begin{enumerate}
        \item Consideriamo \(\{B_1(x) \,|\, x\in E\}\) che è un ricoprimento aperto di \(E\).
        Siccome \(E\) è compatto esiste un sottoricoprimento finito aperto di \(E\), ossia
        \(x_1, \ldots, x_n \in E\) tali che
        \[
            E \subseteq \bigcup_{i=1}^n B_1(x_i)
        \]
        Posto
        \[
            R = 1 + \max_{i=1,\ldots,n} d(x_i, x_1)
        \]
        Allora la bolla di raggio \(R\) centrata in \(x_1\) contiene \(E\), quindi \(E\) è limitato.
        \item Supponiamo che non sia chiuso. Allora esiste \(y\in E'\) ma \(y\notin E\).
        Vogliamo costruire un ricoprimento aperto di \(E\) che non ammette sottoricoprimento finito.
        Sia \(r(x) = \frac{1}{2} d(x,y)\) per ogni \(x\in X\).
        Se \(x\in E\) allora \(r(x) > 0\) perchè \(y\notin E\).
        Abbiamo il ricoprimento
        \[
            \{B_{r(x)}(x) \,|\, x\in E\}
        \]
        Ma per la compattezza esisterebbe un sottoricoprimento finito, cioè \(x_1, \ldots, x_n \in E\) tali che
        \[
            E \subseteq \bigcup_{i=1}^n B_{r(x_i)}(x_i)
        \]
        Sia ora \(R = \min_{i=1,\ldots,n} r(x_i)\). Allora \(R > 0\) e la bolla \(B_R(y)\)
        non interseca nessuna delle \(B_{r(x_i)}(x_i)\), assurdo poiché \(y\) è punto di accumulazione.
        \item Sia \(F\) un sottoinsieme infinito di \(E\). Supponiamo che \(F\) non abbia punti di accumulazione in \(E\).
        Allora ogni punto di \(E\) ha una bolla che interseca \(F\) in al più un punto.
        Queste formano un ricoprimento aperto di \(E\).
        Ma se esistesse un sottoricoprimento finito, \(F\) sarebbe finito, assurdo.
    \end{enumerate}
}

\sproposition{}{
    Sia \(E \subseteq X\) compatto. Se \(F \subseteq E\) è chiuso allora \(F\) è compatto.
}
\sproof{}{
    Sia \(\{G_\alpha\}_{\alpha \in A}\) un ricoprimento aperto di \(F\).
    Dobbiamo aggiungere degli insiemi aperti per coprire il resto.
    Siccome \(F\) è chiuso, \(X \setminus F\) è aperto.
    Quindi \(\{G_\alpha\}_{\alpha \in A} \cup \{X \setminus F\}\) è un ricoprimento aperto di \(E\).
    Per la compattezza di \(E\) esiste un sottoricoprimento finito, che escludendo \(X \setminus F\)
    è un sottoricoprimento finito di \(F\).
}

% ESERCIZIO
Se \(F\subseteq X\) è chiuso, ed \(E\subseteq X\) è compatto, allora \(F \cap E\) è compatto.

\stheorem{Teorema dell'intersezione finita}{
    Sia \(\{E_\alpha\}_{\alpha \in A}\) una famiglia di compatti tale che
    ogni intersezione finita è non vuota. Allora
    \[
        \bigcap_{\alpha \in A} E_\alpha \neq \emptyset
    \]
}

\sproof{}{
    Supponiamo che l'intersezione sia vuota. Allora
    e sa \(E_{\overline{\alpha}}\) un compatto finito nella famiglia.
    % TODO
}

\scorollary{caso particolare}{
    Sia \(\{E_n\}_{n\in \mathbb{N}}\) una famiglia di compatti tale che
    \[
        E_{n+1} \subseteq E_n
    \]
    Allora
    \[
        \bigcap_{n\in \mathbb{N}} E_n \neq \emptyset
    \]
}

\stheorem{Teorema di Heine-Borel}{
    Sia \(E \subseteq R^n\) con la metrica euclidea. Allora \(E\) è compatto se e solo se
    \(E\) è chiuso e limitato.
}

\slemma{}{
    Sia \(\{I_k\}_{k\in \mathbb{N}}\) una famiglia di intervalli
    \(I_k = [a_k, b_k]\) tali che \(I_k \supseteq I_{k+1}\).
    Allora 
    \[
        \bigcap_{k\in \mathbb{N}} I_k \neq \emptyset
    \]
}

\sproof{}{
    Gli intervalli sono annidati, quindi \(a_k\) è crescente e \(b_k\) è decrescente e \(a_k \leq b_k\).
    In particolare \(a_k \leq b_i\). Consideriamo l'insieme \(E = \{a_k \,|\, k\in \mathbb{N}\}\).
    \(E\) è limitato superiormente, e ammette supremum \(x\).
    Per definizione \(x \geq a_k\). Ma \(a_k \leq b_i\) per tutte le \(i\).
    Quindi, \(x \leq b_i\) per ogni \(i\). Allora
    \[
        x \in I_n \implies x \in \bigcap I_k
    \]
}

\end{document}
\documentclass[a4paper]{article}

\usepackage{amsmath}
\usepackage{amssymb}
\usepackage{stellar}
\usepackage{parskip}
\usepackage{fullpage}
\usepackage{wrapfig}
\usepackage{tikz}

\usetikzlibrary{arrows}
\usetikzlibrary{decorations.pathreplacing}
\usetikzlibrary{cd}

\title{Analisi II}
\author{Paolo Bettelini}
\date{}

% libri
% pagani-salsa
% fusco-marcellini-sbordone 

\begin{document}

\maketitle
\tableofcontents

\section{Spazi metrici}

\subsection{Definizioni}

L'insieme vuoto e l'insieme \(X\) sono aperti e chiusi.

\sproposition{}{
    L'unione di aperti non numerabile è aperta, mentre l'intersezione è aperta solo se finita.
}

\sproof{}{
    Per dimostrare quest'ultima lo facciamo su due insiemi e il resto è per induzione.
    Prendiamo un punto nell'interezione e prendiamo le due bolle dentro gli insiemi centrate
    nel punto. Siccome hanno lo stesso centro la loro intersezione è sempre una bolla di raggio
    il minore fra i due.
}

La metrice discreta può generare una bolla che è un singoletto.

\sproposition{}{
    L'unione di chiusi finiti è chiusa. L'intersezione qualsiasi è chiusa.
}

Ogni singoletto è chiuso. Per dimostrarlo mostriamo che nel complementare
esiste una bolla che non interseca il punto (vero per proprietà di Hausdorff).

Tutti i punti di accumulazione sono dei punti aderenti.
Tutti i punti di un sottoinsieme sono aderenti per il sottoinsieme.
Ogni punto o è di accumulazione o è isolato.

Se \(x_0\) è aderente ad E, \(x_0\) può essere un punto di E oppure no.
Se \(x_0\) è punto di accumulazione per E, in ogni bolla centrata in \(x_0\) cadono inifniti punti.

\sproposition{}{
    \(E^\circ\) è aperto. \(E\) è aperto se e solo se \(E = E^\circ\).
    \(E^\circ\) è il più grande aperto contenuto in \(E\).
    
    \(\overline{E}\) è chiuso. \(E\) è chiuso se e solo se \(E = \overline{E}\).
    La chiusura di \(E\) è il più piccolo chiuso contenente \(E\).
}

\sproof{per l'interno}{
    Dimostriamo che \(E^\circ\) è aperto. Sia \(x_0 \in E^\circ\).
    un punto interno ad \(E\), quindi esiste una bolla centrata in tale punto che è contenuta in \(E\).
    Prendiamo un altro punto \(y\) in questa bolla. Possiamo costruire una inner bolla centrata in \(y\)
    con un raggio sufficientemente piccolo da rimanere nella bolla più grande. Quindi il punto \(y\)
    è a sua volta interno, quindi tutta la bolla centrata in \(x_0\) è in \(E^\circ\) e quindi è aperto.
    
    Dimostriamo ora che se \(E\) è aperto allora \(E=E^\circ\) (l'altra implicazione è ovvia).
    Per fare ciò dimostriamo che \(E^\circ\) è il più grande aperto in \(E\).
    Osserviamo che \(E^\circ\) fa parte della famiglia degli aperti di \(X\)
    contenuti in \(E\). Sia \(A\) un aperto contenuto in \(E\). VOglio dimostrare che \(A \subseteq E^\circ\).
    Sia \(x_0 \in A\). \(A\) èunione di bolle quindi esiste unr aggio tale che la
    bolla centrata in \(x_0\) di tale raggio è contenuta in \(A\) che è contenuto in \(E\).
    Quindi, \(x_0\) è interno ad \(E\) e \(x_0 \in E^\circ\) e \(A \subseteq E^\circ\).
    Supponiamo ora che \(E\) sia aperto. Allora \(E\) fa parte della famiglia degli aperti
    di \(X\) contenuti in \(E\). Devo avere \(E \subseteq E^\circ\).
    Dato che \(E^\circ \subseteq E\) allora \(E^\circ = E\).
}

% lezione 2

Dire che un insieme è dentro in un altro significa dire che la sua chiusura coincide con l'insieme.
Tipo la chiusura di Q è R quindi Q è denso in R.


\sdefinition{Limitato}{
    Se è contenuto in una bolla
}

\sdefinition{Diametro}{
    è il sup della metrica su tutte le coppie.
}

\sdefinition{Ricoprimento}{
    Sia \(E\) un sottoinsieme di uno spazio metrico \(X\). Una famiglia
    \[
        \{G_\alpha\}_{\alpha \in A}
    \]
    è un ricoprimento apert di \(E\) se
    \[
        E \subseteq \bigcup_{\alpha \in A} G_\alpha
    \]
}

\sdefinition{Sottoricoprimento}{
    Un Sottoricoprimento di 
    \[
        \{G_\alpha\}_{\alpha \in A}
    \]
    è una sottofamiglia di \(G_\alpha\)
    tale che continua a ricoprire. Cioè ne scarto alcuni ma deve comunque rimanere
    una copertura.
}

\sdefinition{Compatto}{
    Uno spazio metrico \(X\) è compatto se ogni ricoprimento aperto di \(E\)
    ammette un sottoricoprimento finito.
}

Ogni insieme finito è compatto.


\stheorem{}{
    Sia  \(X\) uno spazio metrico e \(E\) un sottoinsieme di \(X\) compatto.
    \begin{enumerate}
        \item \(E\) è limitato;
        \item \(E\) è chiuso;
        \item Ogni sottoinsieme infinito di \(E\) ha almeno un punto di accumulazione in \(E\).
    \end{enumerate}
}

\sproof{}{
    \begin{enumerate}
        \item Consideriamo \(\{B_1(x) \,|\, x\in E\}\) che è un ricoprimento aperto di \(E\).
        Siccome \(E\) è compatto esiste un sottoricoprimento finito aperto di \(E\), ossia
        \(x_1, \ldots, x_n \in E\) tali che
        \[
            E \subseteq \bigcup_{i=1}^n B_1(x_i)
        \]
        Posto
        \[
            R = 1 + \max_{i=1,\ldots,n} d(x_i, x_1)
        \]
        Allora la bolla di raggio \(R\) centrata in \(x_1\) contiene \(E\), quindi \(E\) è limitato.
        \item Supponiamo che non sia chiuso. Allora esiste \(y\in E'\) ma \(y\notin E\).
        Vogliamo costruire un ricoprimento aperto di \(E\) che non ammette sottoricoprimento finito.
        Sia \(r(x) = \frac{1}{2} d(x,y)\) per ogni \(x\in X\).
        Se \(x\in E\) allora \(r(x) > 0\) perchè \(y\notin E\).
        Abbiamo il ricoprimento
        \[
            \{B_{r(x)}(x) \,|\, x\in E\}
        \]
        Ma per la compattezza esisterebbe un sottoricoprimento finito, cioè \(x_1, \ldots, x_n \in E\) tali che
        \[
            E \subseteq \bigcup_{i=1}^n B_{r(x_i)}(x_i)
        \]
        Sia ora \(R = \min_{i=1,\ldots,n} r(x_i)\). Allora \(R > 0\) e la bolla \(B_R(y)\)
        non interseca nessuna delle \(B_{r(x_i)}(x_i)\), assurdo poiché \(y\) è punto di accumulazione.
        \item Sia \(F\) un sottoinsieme infinito di \(E\). Supponiamo che \(F\) non abbia punti di accumulazione in \(E\).
        Allora ogni punto di \(E\) ha una bolla che interseca \(F\) in al più un punto.
        Queste formano un ricoprimento aperto di \(E\).
        Ma se esistesse un sottoricoprimento finito, \(F\) sarebbe finito, assurdo.
    \end{enumerate}
}

\sproposition{}{
    Sia \(E \subseteq X\) compatto. Se \(F \subseteq E\) è chiuso allora \(F\) è compatto.
}
\sproof{}{
    Sia \(\{G_\alpha\}_{\alpha \in A}\) un ricoprimento aperto di \(F\).
    Dobbiamo aggiungere degli insiemi aperti per coprire il resto.
    Siccome \(F\) è chiuso, \(X \setminus F\) è aperto.
    Quindi \(\{G_\alpha\}_{\alpha \in A} \cup \{X \setminus F\}\) è un ricoprimento aperto di \(E\).
    Per la compattezza di \(E\) esiste un sottoricoprimento finito, che escludendo \(X \setminus F\)
    è un sottoricoprimento finito di \(F\).
}

% ESERCIZIO
Se \(F\subseteq X\) è chiuso, ed \(E\subseteq X\) è compatto, allora \(F \cap E\) è compatto.

\stheorem{Teorema dell'intersezione finita}{
    Sia \(\{E_\alpha\}_{\alpha \in A}\) una famiglia di compatti tale che
    ogni intersezione finita è non vuota. Allora
    \[
        \bigcap_{\alpha \in A} E_\alpha \neq \emptyset
    \]
}

\sproof{}{
    Supponiamo che l'intersezione sia vuota. Allora
    e sia \(E_{\overline{\alpha}}\) un compatto fissato nella famiglia.
    \begin{align*}
        &E_{\overline{\alpha}} \cap \left(
            \bigcap_{\alpha \neq \overline{\alpha}} E_\alpha
        \right) = \emptyset \\
        &\implies
        E_{\overline{\alpha}} \subseteq
        \left(
            \bigcap_{\alpha \neq \overline{\alpha}} E_\alpha
        \right)^c
        =
        \bigcup_{\alpha \neq \overline{\alpha}} E_\alpha^c
    \end{align*}
    \(\{E_\alpha^c\}_{\alpha \neq \overline{\alpha}}\) è un ricoprimento aperto di \(E_{\overline{\alpha}}\).
    Esistono quindi \(\alpha_1, \ldots, \alpha_n \neq \overline{\alpha}\)
    tali che \begin{align*}
        &E_{\overline{\alpha}} \subseteq \bigcup_{i=1}^n E_{\alpha_i}^c
        = \left(
            \bigcap_{i=1}^n E_{\alpha_i}
        \right)^c \\
        &\implies
        E_{\overline{\alpha}} \cap \left(
            \bigcap_{i=1}^n E_{\alpha_i}
        \right) = \emptyset
    \end{align*}
    assurdo. % lightning
}

\scorollary{caso particolare}{
    Sia \(\{E_n\}_{n\in \mathbb{N}}\) una famiglia di compatti tale che
    \[
        E_{n+1} \subseteq E_n
    \]
    Allora
    \[
        \bigcap_{n\in \mathbb{N}} E_n \neq \emptyset
    \]
}

\stheorem{Teorema di Heine-Borel}{
    Sia \(E \subseteq R^n\) con la metrica euclidea. Allora \(E\) è compatto se e solo se
    \(E\) è chiuso e limitato.
}

\slemma{}{
    Sia \(\{I_k\}_{k\in \mathbb{N}}\) una famiglia di intervalli
    \(I_k = [a_k, b_k]\) tali che \(I_k \supseteq I_{k+1}\).
    Allora 
    \[
        \bigcap_{k\in \mathbb{N}} I_k \neq \emptyset
    \]
}

\sproof{}{
    Gli intervalli sono annidati, quindi \(a_k\) è crescente e \(b_k\) è decrescente e \(a_k \leq b_k\).
    In particolare \(a_k \leq b_i\). Consideriamo l'insieme \(E = \{a_k \,|\, k\in \mathbb{N}\}\).
    \(E\) è limitato superiormente, e ammette supremum \(x\).
    Per definizione \(x \geq a_k\). Ma \(a_k \leq b_i\) per tutte le \(i\).
    Quindi, \(x \leq b_i\) per ogni \(i\). Allora
    \[
        x \in I_n \implies x \in \bigcap I_k
    \]
}

\sdefinition{}{
    Siano \(a,b \in \mathbb{R}^n\) con \(a_i < b_i\) per ogni \(i=1,\ldots,n\).
    Un rettangolo chiuso è il prodotto cartesiano
    \[
        [a_1, b_1] \times [a_2, b_2] \times \ldots \times [a_n, b_n]
    \]
    che indichiamo con \([a,b]\).
}

\slemma{}{
    Sia \(\{R_k\}_{k\in \mathbb{N}}\) una famiglia di rettangoli chiusi
    tali che \(R_k \supseteq R_{k+1}\) per ogni \(k\).
    Allora
    \[
        \bigcap_{k\in \mathbb{N}} R_k \neq \emptyset
    \]
}

\sproof{}{
    Siccome
    \[
        R_k = I_{k,1} \times I_{k,2} \times \ldots \times I_{k,n}
    \]
    possiamo applicare il primo lemma e quindi
    \[
        \exists y_i \in \bigcap_{k\in \mathbb{N}} I_{k,i}
    \]
    Il punto \(y = (y_1, \ldots, y_n)\) è in ogni \(R_k\).
}

\slemma{Lemma 3}{
    In \(\mathbb{R}^n\) con la metrica euclidea
    ogni rettangolo è compatto.
}

\sproof{Lemma 3}{
    Sia \(R = [a,b]\) un rettangolo e supponiamo che non sia compatto.
    Sia \(\{G_\alpha\}_{\alpha \in A}\) un ricoprimento aperto di \(R\)
    che non ammette sottoricoprimento finito.
    Vogliamo adesso dimezzare ambo i lati (quindi \(n\) tagli).
    Abbiamo adesso \(2^n\) rettangoli.
    \[
        [a_i, b_i] = [a_i, c_i] \cup [c_i, b_i], \quad
        c_i = \frac{a_i + b_i}{2}
    \]
    Il diametro di \(R\) è \(||b-a||\).
    Il diametro di ogni rettangolo ottenuto è la metà.
    Almeno uno di questi rettangoli ha la proprietà di non ammettere sottoricoprimento finito.
    Lo chiamiamo \(R_1\).
    Iterando il procedimento otteniamo una successione di rettangoli
    \[
        R \supseteq R_1 \supseteq R_2 \supseteq \ldots
    \]
    con diametro che tende a zero e che non ammettono sottoricoprimento finito, il diametro
    di \(R^k\) è dato da \(\frac{1}{2^k} ||b-a||\).
    Per il lemma precedente esiste \(x \in \bigcap_{k\in \mathbb{N}} R_k\).
    Siccome \(R_k \subseteq R\) per ogni \(k\), \(x \in R\).
    Siccome \(\{G_\alpha\}_{\alpha \in A}\) è un ricoprimento di \(R\),
    esiste \(\alpha_0 \in A\) tale che \(x \in G_{\alpha_0}\).
    \(G_{\alpha_0}\) è aperto, quindi esiste \(r > 0\) tale che
    \(B_r(x) \subseteq G_{\alpha_0}\).
    Scegliamo \(k\) sufficientemente grande tale che \(2^{-k} ||b-a|| < r\).
    Ma il diametro di \(R_k\) è minore di \(r\), quindi \(R_k \subseteq B_r(x)\).
    Quindi \(R_k \subseteq G_{\alpha_0}\), assurdo perchè \(R_k\) non ammette sottoricoprimento finito.
}

\sproof{Heine-Borel}{
    Dobbiamo dimostrare solo che se \(E\) è chiuso e limitato allora è compatto.
    Siccome \(E\) è limitato esiste \(M\) tale che \(||x|| < M\) per ogni \(x \in E\).
    Quindi,
    \[
        E \subseteq [-M, M] \times [-M, M] \times \ldots \times [-M, M] = R
    \]
    \(E\) è un chiuso contenuto in un compatto, quindi è compatto.
}

\stheorem{Teorema di Bolzano-Weierstrass}{
    Ogni sottoinsieme infinito e limitato di \(\mathbb{R}^n\) ha almeno un punto di accumulazione.
}

\sproof{Teorema di Bolzano-Weierstrass}{
    % todo semplice una riga
}

\sdefinition{Insiemi separati}{
    Sia \((X, d)\) uno spazio metrico e \(A,B\subseteq X\) due sottoinsiemi.
    Diciamo che \(A\) e \(B\) sono separati se
    \[
        A \cap \overline{B} = \emptyset \land \overline{A} \cap B = \emptyset
    \]
}

Devono sicuramente essere disgiunti, ma non basta.
Serve che nessun punto di uno dei due insiemi è punto di accumulazione
dell'altro.

\sdefinition{}{
    Sia \((X, d)\) uno spazio metrico e \(E\subseteq X\). \(E\) è connesso
    se non può essere scritto come unione di due sottoinsiemi non vuoti e separati.
}

I sottoinsiemi connessi di \(\mathbb{R}\) sono tutti e soli gli intervalli.

Uno spazio metrico è connesso se e solo se l'unico sottoinsieme non vuoto
di \(X\) che è anche aperto e chiuso è \(X\) stesso. (Dimostrazione per esercizio).

\(R^n\) con la metrica euclidea è connesso. (Dimostrazione per esercizio non proprio banale).

\subsection{Successioni in spazi metrici}

Mettere la definizione di convergenza ma con \(d(x_m,y) < \varepsilon\).
Oppure \(x_m \in B_\varepsilon(y)\).

In particolare la successione metrica converge se e solo se \(d(x_m, y) \to 0\)
secondo la convergenza reale.

Il limite è unico per proprietà di Hausdorff.

\sproposition{}{
    Sia \((X,d)\) uno spazio metrico e \(E \subseteq X\) e sia \(y\) un punto di accumulazione
    per \(E\). Allora esiste una successione \(\{x_n\} \subseteq E \setminus \{y\}\)
    che converge ad \(y\).
    In particolare, \(E\) è chiuso se e solo se per ogni successione \(\{x_n\} \subseteq E\)
    che converge ad \(y\) allora \(y \in E\).
}

\sproof{}{
    Dato che \(y\in E'\), \(\forall x_m \in \mathbb{N}\), esiste \(x_m\) tale che
    \(x_m \in B_{\frac{1}{m}}(y) \cap E\) e \(x_m \neq y\).
    La successione così costruita converge ad \(y\).
    Infatti, \(d(x_m, y) < \frac{1}{m} \to 0\).
}

\sproposition{}{
    Sia \((X,d)\) uno spazio metrico e sia \(\{x_n\}\) una successione convergente in \(X\).
    Una condizione necessaria per la convergenza è che ogni sottosuccesione converga allo stesso limite.
    La condizione sufficiente è che ogni sottosuccessione ammetta una sottosuccessione
    che converge allo stesso limite.
}

\sdefinition{Compattezza sequenziale}{
    Uno spazio metrico \(X\) è sequenzialmente compatto se ogni successione in \(X\)
    a valori in \(E\) ammette una sottosuccessione convergente ad un punto di \(E\). 
}

\sproposition{Equivalenza compattezza}{
    \(E\) is compact is and only if \(E\) is sequentially compact.
}

Questa c'è solo negli spazi metrici.

\sproof{}{
    \iffproof{
        Sia \(\{x_n\}\) una successione in \(E\).
        Consideriamo \(F = \{x_n \,|\, n\in \mathbb{N}\}\).
        Se \(F\) è finito, esiste un elemento che compare infiniti volte
        e la successione costante converge a tale elemento.
        Se \(F\) è infinito, per la compattezza \(F\) ammette un punto di accumulazione, \(y\in E\).
        Costruiamo una sottosuccessione che converga ad \(y\).
        Scegliamo \(x_m\) tale che \(d(x_m, y) < 1\).
        Scegliamo \(x_{m_2}\) tale che \(d(x_{m_2}, y) < \frac{1}{2}\) e \(m_2 > m_1\), e così via.
        La sottosuccessione così costruita converge ad \(y\) in quanto \(d(x_{m_k}, y) < \frac{1}{k} \to 0\).
    }{
        XXX
    }
}

Ogni successione convergente è di Cauchy.

Per esempio con la metrica discreta una successione è convergente se e solo se è definitamente costante,
che è equivalente ad essere di Cauchy, quindi è completo.

Nel caso dei razionali nei reali con metrica euclidea, consideriamo la radice di due che è un punto di accumulazione
per i razionali. Esiste una successione di razionali che converge a radice di due, quindi è di Cauchy.
Ma essa non può convergere in Q, altrimenti convergerebbe anche in R e avrebbe due limiti.
Tuttavia è una successione di Cauchy in Q perché è convergente in R e quindi è di Cauchy in R.
(La condizione è la medesima). Quindi Q non è completo.

\sdefinition{Spazio completo}{
    Uno spazio metrico \((X,d)\) è completo se ogni successione di Cauchy in \(X\)
    converge ad un punto di \(X\).
}

\stheorem{}{
    \(R^n\) con la metrica euclidea è completo.
}

\sproof{}{
    Sia \(\{x_n\}\) una successione di Cauchy in \(R^n\).
    Scriviamo \(E_n = \{x_k \,|\, k \geq n\}\).
    Notiamo che \(E_n \supseteq E_{n+1}\).
    Ponendo la chiusura \(\overline{E_n} \supseteq \overline{E_{n+1}}\).
    Inoltre, \(E_n\) è limitato e \(\text{diam} E_n \to 0\).
    Infatti, dato \(\varepsilon > 0\) esiste \(N\) tale che per ogni \(m,n \geq N\)
    \(d(x_n, x_m) < \varepsilon\).
    Notiamo inoltre che
    \[
        \text{diam} E_n = \sup\{d(x_m,x_k)\} < \varepsilon
    \]
    Dimostrazione per esercizio vale che \(\text{diam} F = \text{diam} \overline{F}\).
    Quindi, \(\text{diam} \overline{E_n} \to 0\).
    Adesso \(\{\overline{E_n}\}\) è una successione di compatti in quanto chiusi e limitati, annidati.
    Quindi \[
        E \triangleq \bigcap_{n\in \mathbb{N}} \overline{E_n} \neq \emptyset
    \]
    Siccome \(\text{diam} E = 0\) o è vuoto o contiene un solo punto, quindi contiene un solo punto \(E = \{y\}\).
    Mostriamo che \(x_n \to y\).
    Abbiamo \(d(x_n, y) \leq \text{diam} \overline{E_n} \to 0\).
}

\stheorem{}{
    Sia \((X, d)\) uno spazio metrico compatto.
    Allora \(X\) è completo.
}

\sproof{}{
    Sia \(\{x_n\}\) una successione di Cauchy in \(X\).
    Siccome è compatto è compatto per successioni, quindi esiste una sottosuccessione
    \(\{x_{n_k}\}\) che converge ad un punto \(y \in X\).
    Mostriamo che \(x_n \to y\).
    Dato \(\varepsilon > 0\) esiste \(N_0\) tale che per ogni \(m,n \geq N_0\)
    \(d(x_n, x_m) < \frac{1}{2}\varepsilon\).
    Per la convergenza di \(\{x_{n_k}\}\) esiste \(K\) tale che per ogni \(k \geq K\)
    \(d(x_{n_k}, y) < \frac{1}{2}\varepsilon\).
    Scegliamo \(\overline{N} = \max\{N_0, n_K\}\).
    Allora per ogni \(n \geq \overline{N}\) si ha
    \[
        d(x_n, y) \leq d(x_n, x_{n_K}) + d(x_{n_K}, y) < \frac{1}{2}\varepsilon + \frac{1}{2}\varepsilon = \varepsilon
    \]
}

%%%% 1 ott


%%% ????
Sia \(X\) uno spazio metrico completo, \(Y \subseteq X\).
\(Y\) è completo se e solo se \(Y\) è chiuso in \(X\).

\stheorem{}{
    \(E\) sequenzialmente comaptto implica \(E\) compatto.
}

\sproof{}{
    Sia \(\{G_\alpha\}_{\alpha \in A}\) un ricoprimento aperto di \(E\).
    Esiste \(\delta > 0\) tale che \(\forall x \in E\) esiste \(\overline{\alpha}\)
    tale che \(B_\delta(x) \subseteq G_{\overline{\alpha}}\).
    \begin{enumerate}
        \item \emph{claim 1:} \(\forall m \in \mathbb{N}\), esiste \(x_m\) tale che
            \(B_{1/m}(x_m)\) non è sottoinsieme di \(G_\alpha\) per tutte le \(\alpha\).
            \(\{x_m\}\) è una successione in \(E\) e quindi posso estrarre una sottosuccessione convrgente
            \(x_{m_k} \to p \in E\). Esiste \(\hat{\alpha}\) tale che \(p \in G_{\hat{\alpha}}\).
            \(G_{\hat{\alpha}}\) è aperto e quindi esiste una \(\varepsilon > 0\)
            tale che \(B_\varepsilon(p) \in G_{\hat{\alpha}}\).
            Ma \(x_{m_k} \to p\) quindi con \(k\) sufficientemente grande
            \[B_{1/m_k}(x_{m_k}) \subseteq B_\varepsilon(p) \subseteq G_{\hat{\alpha}}\]
            che è assurdo lightning.
        \item \emph{claim 2:} \(E\) è contenuto nel'unione di un numero finito di bolle di raggio
            \(\delta\) centrate in punto di \(E\).
            Per assurdo, sia \(x_1 \in E\). Sicuramente \(B_\delta(x_1)\)
            non ricopre \(E\) quindi esiste \(x_2 \in E \backslash B_\delta(x_1)\).
            Ma assieme \(B_\delta(x_1) \cup B_\delta(x_2)\) non ricoprono \(E\),
            quindi esiste un \(x_3 \in E \backslash (B_\delta(x_1) \cup B_\delta(x_2))\)
            e così via. La successione \(\{x_m\}\) deve ammettere una sottosuccessione
            convergente. Ma \(d(x_i, x_j) \geq \delta\) se \(i \neq j\) quindi la sucessione
            \(\{x_m\}\) non è di Cauchy Lightning.
            Quindi \(E \subseteq B_\delta(x_1) \cup B_\delta(x_2) \cup \cdots\).
    \end{enumerate}
}

In realtà abbiamo mostrato anche la terza.

\stheorem{}{
    Sia \(X\) uno spazio metrico. Sono equivalenti:
    \begin{enumerate}
        \item \(X\) è compatto;
        \item \(X\) è sequenzialmente compatto;
        \item \emph{limit point compact:} ogni sottoinsieme infinito
        di \(X\) ha almeno un punto di accumulazione.
    \end{enumerate}
}

Solo negli spazi metrici.

\subsection{Funzioni}

\sdefinition{}{
    Siano \((X_1, d_1), (X_2, d_2)\) due spazi metrici e sia \(E \subseteq X_1\).
    Sia \(f \colon E \to X_2\) e \(p \in E'\).
    Diciamo che \(l\in X_2\) è limite di \(f(x)\) per \(x\to p\) e diciamo
    \[
        \forall \varepsilon > 0, \exists \delta > 0 \,|\, x \in E
        \land 0 < d_1(x_1, p) < \delta \implies d_2(f(x), l) < \varepsilon
    \]
    Equivalentemente \(\forall \varepsilon >0\) esiste \(\delta > 0\)
    tale che \[f((B_\delta(p) \cap E) \backslash \{p\}) \subseteq B_\varepsilon(l)\]
}

\sproposition{}{
    Sia \(f\colon E \subseteq X_1 \to X_2\).
    Allora \(f(x) \to l\) per \(x\to p\) se e solo se \(f(x_n) = l\)
    per ogni successione \(\{x_n\}\) tale che \(x_n \in E\)e \(x_n \neq p\)
    per tutte le \(n\) e \(x_n \to p\).
}

Valgono i medesimi teoremi tipo l'unicità del limite e i teoremi di permanenza del segno, confronto etc.

\sproposition{}{
    Sia \(f \colon E \subseteq X \to \mathbb{R}^n\) per \(n > 1\).
    Allora
    \[
        f(x) \to l \iff f_i(x) \to l_i
    \]
    per \(x \to p\).
}

\sproof{Sketch}{
    Conderiamo la norma per tutte le \(i\)
    \begin{align*}
        |f_i(x)-l_i| \leq ||f(x) - l|| \leq \sum_k |f_k(x) - l_k|
    \end{align*}
}

\sdefinition{Continuità}{
    Siano \((X_1, d_1), (X_2, d_2)\) due spazi metrici,
    \(f \colon E \subseteq X_1 \to X_2\), \(p \in E\).
    Diciamo che \(f\) è \emph{continua} in \(p\)
    se \(\forall \varepsilon > 0\) esiste \(\delta > 0\) tale che
    \[\forall x \in E \cap B_\delta(p) \implies f(x) \in B_\varepsilon((f(p)))\]
    Euivalentemente \(\forall \varepsilon > 0\) esiste \(\delta > 0\)
    tale che \(x\in E\) e \(d_1(x,p) <\delta\) implica che \(d_2(f(x), f(p)) < \varepsilon\).
    Oppure ancora \((f(B_\delta(p) \cap E)) \subseteq B_\varepsilon(f(p))\).
}

Se \(p\) è un punto isolato di \(E\) allora \(\exists r > 0\)
tale che \(B_r(p) \cap E = \{p\}\). Scegliendo \(\delta \leq r\)
la definizione di continuità è automaticamente soddisfatta.
Se \(p\) non è i solato, allora è un punto di accumulazione per \(E\).
In questo caso \(f\) è continua in \(p\) e vale che \(f(x) \to f(p)\) per \(x\to p\).

\sproposition{}{
    \(f\) è continua in \(p\) se e solo se
    \[
        \lim_{x\to p} f(x_n) = f(p)
    \]
    per ogni successione \(\{x_n\}\) tale che \(x_n \in E\)
    per tutte le \(n\) e \(x_n \to p\).
}

\sdefinition{}{
    Sia \(f \colon E \subseteq X_1 \to X_2\). Diciamo che \(f\) è continua nell'insieme \(E\)
    se \(f\) è continua in ogni punto di \(E\).
}

\sproposition{}{
    Siano \((X_1, d_1), (X_2, d_2)\) spazi metrici e sia \(f \colon X_1 \to X_2\).
    Allora \(f\) è continua in \(X\) se e solo se \(f^{-1}(V)\) è aperto in \(X_1\)
    per tutti i \(V\) aperti in \(X_2\).
}

\sproof{}{
    \iffproof{
        Sia \(V\) un aperto di \(X_2\). Se \(f^{-1}(V) = \emptyset\)
        in questo caso abbiamo finito. Altrimenti,
        sia \(p \in f^{-1}(V)\), cioè \(f(p) \in V\).
        Essendo \(V\) aperto, riesco a trovare \[ B_\varepsilon(f(p)) \in V \]
        Ma \(f\) è continua quindi riesco anche a trovare \(\delta>0\) tale che
        \[
            f(B_\delta(p)) \subseteq B_\varepsilon(f(p))
        \]
        Quindi \(B_\delta(p) \subseteq f^{-1}(V)\) quindi \(p\) è un punto interno a
        \(f^{-1}(V)\). Per l'arbitrarietà di \(p\) segue che \(f^{-1}(V)\) è aperto.
    }{
        Sia \(p \in X\) e dimostriamo che \(f\) è continua in \(p\).
        Sia \(\varepsilon > 0\) fissato. \(B_\varepsilon(f(p))\)
        è un aperto di \(X_2\). \(f^{-1}(B_\varepsilon(f(p)))\) è un aperto di \(X_1\)
        e \(p \in f^{-1}(B_\varepsilon(f(p)))\) e quindi esiste \(\delta > 0\)
        tale che
        \[
            B_\delta(p) \subseteq f^{-1}(B_\varepsilon(f(p)))
        \]
        cioè
        \[
            f(B_\varepsilon(p)) \subset B_\varepsilon(f(p))
        \]
        che è la definizione di continuità.
    }
}

Siccome \(f^{-1}(E^c) = \left(f^{-1}(E)\right)^c\) allora \(f\)
è continua se e solo se \(f^{-1}(C)\) è chiuso in \(X_1\)
per ogni chiuso in \(C \in X_2\). Molto utile.

In generale le funzioni continue non mandano aperti in aperti.
Per esempio \(f \colon \mathbb{R} \to \mathbb{R}\)
data da \(x \to x^2\).
Abbiamo che
\[
    f((-1,1)) = [0,1)
\]

\sdefinition{Funzione aperta}{
    Una funzione viene detta \emph{aperta}
    se \(f(U)\) è aperto in \(X_2\) per tutti gli insiemi \(U\) aperto om \(X_1\).
    Analogamente funzione chiusa.
}

Sia \(f \colon (X, d) \to \mathbb{R}^n\) con \(n>1\)
è continua se e solo se tutte le sue componenti sono continue.

\sproposition{}{
    Siano \((X_1, d_1), (X_2, d_2)\)
    spazi metrici, \(f \colon X_1 \to X_2\) una funzione continua.
    Se \(X_1\) è compatto, allora \(f(X_1)\) è compatto.
}

\sproof{}{
    Sia \(\{G_\alpha\}_{\alpha \in A}\) un ricoprimento aperto di \(f(X_1)\).
    Consideriamo \(\{f^{-1}(G_\alpha)\}_{\alpha \in A}\) che sono degli aperti.
    Queste preimmagini sono un ricoprimento di \(X_1\),
    che è compatto e quindi posso estrarre un sottoricoprimento finito
    \(f^{-1}(G_{\alpha_1}), \cdots, f^{-1}(G_{\alpha_n})\).
    Vogliamo mostrare che \(\{G_{\alpha_1}, \cdots, G_{\alpha_n}\}\)
    sono un ricoprimento di \(f(X_1)\).
    \[
        f(X_1) = f \left(
            \bigcup_{i=1}^n f^{-1}( G_{\alpha_i})
        \right)
        = \bigcup_{i=1}^n f\left(
            f^{-1}(G_{\alpha_i})
        \right)
        \subseteq \bigcup_{i=1}^n G_{\alpha_i}
    \]
}

\stheorem{Teorema di Weierstrass}{
    Sia \((X,d)\) uno spazio metrico compatto e sia
    \(f \colon X \to \mathbb{R}\) una funzione continua.
    Allora, \(\exists x_1, x_2 \in X\) tali che
    \[
        f(x_1) \leq f(x) \leq f(x_2), \quad \forall x \in X
    \]
    cioè \(f\) possiede massimo e minimo assoluto.
}

\sproof{Teorema di Weierstrass}{
    \(f(X)\) è compatto in \(\mathbb{R}\), quindi è chiuso e limitato.
    Siccome limitato, \(f(X)\) ammette infimum e supremum reali.
    Siccome \(\inf f(x)\) e \(\sup f(x)\)
    appartengono a \(\overline{f(x)}\) e \(f(x)\) è chiuso, appartengono allora ad \(f(X)\)
    e quindi sono massimi e minimi.
}

\stheorem{Teorema da compatto ad Hausdorff}{
    Siano \((X_1, d_1), (X_2, d_2)\) spazi metrici
    con \(X_1\) compatto e \(f \colon X_1 \to X_2\) continua.
    Allora, \(f\) è chiusa.
}

In realtà questo funziona con domini compatti e codomini di Hausdorff.

\sproof{Teorema da compatto ad Hausdorff}{
    Sia \(C\) un chiuso di \(X_1\).
    Voglio dimostrare che \(f(C)\) è un chiuso di \(X_2\).
    Sappiamo che \(C\) è chiuso in un compatto, quindi è compatto.
    La funzione è continua e quindi \(f(C)\) è compatto.
    Siccome i compatti sono chiusi allora è chiuso.
}

\scorollary{}{
    Sia \(f \colon (X_1, d_1) \to (X_2, d_2)\) continua, \(X_1\) compatto e
    \(f\) biunivoca. Allora, \(f^{-1}\) è continua.
}

\sproof{}{
    Dobbiamo mostrare che \((f^{-1})^{-1}(C)\) è chiuso per ogni \(C\) chiuso di \(X_2\).
    Ma questo coincide con \(f(C)\) che è chiusa per il teroema da compatto ad Hausdorff.
}

\stheorem{}{
    Sia \(f\colon (X_1, d_1) \to (X_2, d_2)\) continua e sia \(E \subseteq X\) connesso.
    Allora \(f(E)\) è connesso.
}

\sproof{}{
    Supponiamo che \(f(E)\) non sia connesso. Esistono quindi
    due sottoinsiemi non vuoti disgiunti e separati tali che
    \[
        f(E) = A \cup B
    \]
    Poniamo \(F = f^{-1}(A) \cap E\) e \(G = f^{-1}(B) \cap E\).
    Sicuramente \(F,G \neq \emptyset\) e \(E = F\cup G\). Vogliamo mostrare che \(E\) e \(G\)
    sono separati.
    Siccome \(A \subseteq \overline{A}\) vale anche \(f^{-1}(A) \subseteq f^{-1}(\overline{\overline{A}})\).
    L'applicazione \(f\) è continua e la chiusura di \(A\) è un chiuso.
    Quindi la preimmagine del chiuso \(\overline{A}\) è un chiuso.
    Consideriamo ora \[\overline{F} \subseteq \overline{f^{-1}(A)} = f^{-1}(\overline{A})\]
    perché \(f\) è continua se \(\overline{A}\) è chiuso.
    Quindi \(\overline{F} \subseteq f^{-1}(\overline{A})\) che implica
    \(f(\overline{F}) \subseteq \overline{A}\). D'altro canto \(f(G) \subseteq B\)
    e \(\overline{A} \cap B \neq 0\), e quindi \(\overline{F} \cap G \neq 0\)
    perché altrimenti vi sarebbe un elemento sia in \(\overline{A}\) che in \(B\).
    Dovrebbe essere \(f(x) \in \overline{A}\) e \(f(x) \in B\) lightinng.
    Analogamente si dimostra che \(F \cap \overline{G} = \emptyset\)
    cioè abbiamo scritto \(E\) come unione di due sottoinsiemi non vuoti e separati.
    Ma \(E\) è connesso lightning.
}

\sdefinition{}{
    Siano \((X_1, d_1),(X_2, d_2)\) spazi metrici e \(f\colon X_1 \to X_2\).
    Allora \(f\) è uniformemente continua se
    \(\forall \varepsilon > 0\), esiste \(\delta > 0\)
    tale che \(\forall x,y \in X_1\)
    \[
        d_1(x, y) < \delta \implies d_2(f(x), f(y)) < \varepsilon
    \]
}

\stheorem{Theorema di Heine-Cantor}{
    Siano \((X_1, d_1),(X_2, d_2)\) spazi metrici e 
    \(f\colon (X_1, d_1) \to (X_2, d_2)\)
    continua e \(X_1\) compatto.
    Alorra, \(f\) è uniformemente continua.
}

La dimostrazione è la medesima rispetto al caso banale.

\sdefinition{Funzione di Lipschitz}{
    Siano \((X_1, d_1),(X_2, d_2)\) spazi metrici,
    \(f\colon X_1 \to X_2\).
    Diciamo che \(f\) è \emph{lipschitz-continua} o \emph{lipschitziana}
    se \(\exists \alpha > 0\) tale che
    \[
        d_2(f(x), f(y)) \leq \alpha d_1(x,y)
    \]
    per tutte le \(x,y \in X_1\).
}

\sproposition{}{
    Se \(f\) è Lipschitz-continua, allora è uniformemente continua.
}

\sdefinition{}{
    Siano \((X_1, d_1),(X_2, d_2)\) spazi metrici e 
    \(f\colon (X_1, d_1) \to (X_2, d_2)\).
    Diciamo che \(f\) è una \emph{contrazione} se
    \(f \in \text{Lip}_\alpha(X_1, X_2)\) con \(\alpha < 1\).
}

Se il supremum è finito, allora questa è la miglior costante di Lipschitz (in generale)
\begin{align*}
    \sup_{x\neq y} \frac{|f(x)-f(y)|}{|x-y|}
\end{align*}

\sexample{}{
    Consideriamo \(f \colon \mathbb{R} \to \mathbb{R}\)
    data da \(f(x) = x^2\). Non è di lipschitz inquanto non è uniformemente continua.
    Per mostrarlo possiamo dire
    \begin{align*}
        |f(x)-f(y)| &= |x^2 - y^2| = |x+y| \cdot |x-y|
    \end{align*}
    Se restringessimo il dominio di questa funzione ad un intervallo limitato,
    allora sarebbe di Lipschitz, per il supremum.
}

\sproposition{}{
    Sia \(f\colon I \subseteq \mathbb{R} \to \mathbb{R}\) differenziabile.
    Allora, \(f \in \text{Lip}_\alpha(I, \mathbb{R})\)
    se e solo se \(|f'(x)| \leq \alpha\) per tutte le \(x\).
}

\sproof{}{
    \iffproof{
        Cominciamo con
        \begin{align*}
            |f'(x)| &= \left| \lim_{t\to 0} \frac{f(x+t)-f(x)}{t}\right| \\
            &= \lim_{t\to 0} \left| \frac{f(x+t)-f(x)}{t}\right| \\
            &= \lim_{t\to 0} \frac{|f(x+t)-f(x)|}{|t|} \\
            &\leq \lim_{t\to 0} \frac{\alpha |x+t-x|}{|t|} = \alpha
        \end{align*}
        Possiamo togliere il limite dal modulo in quanto il modulo è una funzione continua.
    }{
        Per il teorema di Lagrange esiste \(\theta \in (\min\{x,y\}, \max\{x,y\})\)
        \begin{align*}
            f(x) - f(y) &= f'(\theta) (x-y) \\
            |f(x) - f(y)| &= |f'(\theta)|\cdot |x-y| \\
            &\leq \alpha|x-y|
        \end{align*}
    }
}

\sexercise{}{
    Per quali \(a \leq b\) la funzione \(f(t) = 1 + t - \arctan(t)\)
    è una contradizione in \([a,b]\).
    Stabiliamo quindi se la derivata è limitata
    \begin{align*}
        f'(t) = 1 + \frac{1}{1 + t^2} = \frac{t^2}{1 + t^2}
    \end{align*}
    notiamo quindi che \(0 \leq f'(t) \leq 1\). Quindi è sicuramente lipschitziana.
    Notiamo allora che
    \[
        \sup_{\mathbb{R}} |f'(t)| = 1 = \alpha
    \]
    Quindi per far sì che \(\alpha < 1\) dobbiamo limitare il dominio ad un intervallo limitato.
    Quindi \(-\infty < a \leq b < \infty\).
    Porta l'intervallo \([a,b]\) in sè?
    Siccome la funzione è crescente porta intervalli a intervalli
    di estremi \(f(a)\) e \(f(b)\). Mi basta quindi imporre
    che \(f(a) \geq a\) e \(f(b) \leq b\). Abbiamo quindi
    \begin{align*}
        \begin{cases}
            1 + a - \arctan a \geq a \\
            1 + b - \arctan b \leq b
        \end{cases}
        =
        \begin{cases}
            \arctan a \leq 1 \\
            \arctan b \geq 1
        \end{cases}
    \end{align*}
    e quindi \(a \leq \tan 1 \leq b\).
    Notiamo che \(f(\tan 1) = \tan 1\) quindi è un punto fisso
    per il teorema delle contrazioni.
}

\sexample{}{
    Sia \(v \in \mathbb{R}^n\) e consideriamo
    \(f\colon \mathbb{R}^n \to \mathbb{R}\) data da
    \(f(x) = v \cdot x\).
    Dobbiamo studiare \(|f(x) - f(y) = |v\cdot x - v \cdot y|\)
    usando la bilinearità del prodotto scalare ottengo
    \(|v\cdot (x-y)|\). Per Cacuhy-Schwarz
    \[
        |v\cdot (x-y)| \leq ||v|| \cdot ||x-y||
    \]
    che è quindi di Lipschitz.
}

\stheorem{Teorema di Banach-Cacciopoli o delle contrazioni}{
    Sia \((X, d)\) uno spazio metrico completo
    e sia \(f \colon X \to X\) una contrazione.
    Allora \(\exists_{=1}\, x \in X\) tale che \(f(x) = x\).
}

Le ipotesi sono necessarie.
Togliamo per esempio la completezza e consideriamo quindi \(X=(0, +\infty)\)
con la contrazione \(f(x) = x/2\). Questa contrazione non ha punti fissi.
Togliamo invece l'ipotesi che sia una contrazione.
Richiediamo solamente che sia una contrazione debole, cioè
\[
    d_2(f(x), f(y)) \leq f_1(x,y)
\]
Consideriamo \(X = [0, +\infty)\) e prendiamo
\(f(t) = t + e^{-t}\). 
La derivata è \(f'(t) = 1-e^{-t}\) che è nulla nell'origine
e poi tende ad \(1\) dal sotto.
Chiaramente non ci sono punti fissi in quanto \(f(t)=t\) è come dire \(e^{-t} = 0\).

\sproof{}{
    Cominciamo mostrando l'esistenza del punto fisso.
    Sia \(x_0 \in X\) un punto fissato e
    consideriamo la successione \(x_{n+1} = f(x_n)\).
    \begin{enumerate}
        \item Mostriamo che \(\{x_n\}\) è di Cauchy, quindi siccome lo spazio è completo
        converge. Dobbiamo mostrare che \(d(x_n, x_m)\) tende a zero quando \(n,m\) crescono.
        Consideriamo inizialmente
        \begin{align*}
            d(x_{n+1}, x_n) &= d(f(x_n), f(x-{n-1})) \leq \alpha d(x_n, x_{n-1}) \\
            &= \alpha d(f(x_{n-1}), f(x_{n-2})) \leq \alpha^2 d(x_{n-1}, x_{n-2}) \\
            &\leq \alpha^n d(x_1, x_0)
        \end{align*}
        Calcoliamo ora la distanza generica e usiamo la disuguaglianza triangolare
        ripetutamente per ogni step
        \begin{align*}
            d(x_{n+k}, d_n) &\leq \sum_{i=0}^{k-1} d(x_{n+k-i}, d_{n+k-i-1}) \\
            &\leq d(x_1, x_0) \sum_{i=0}^{k-1} \alpha^{n+k-i} \\
            &= \alpha^n d(x_1, x_0) \sum_{i=0}^{k-1} \alpha^{n+k-i-1} \\
            &= \alpha^n \frac{\alpha^k - 1}{\alpha - 1} d(x_1, x_0) \\
            &= \frac{\alpha^n}{1 - \alpha} d(x_1, x_0) \to 0
        \end{align*}
        Per sbarazzarci di \(k\) (siccome vogliamo \(k\) arbitrario e il \(\varepsilon\)
        nella definizione di Cauchy deve essere uniforme rispetto ad esso)
        maggioriamo la somma parziale della serie geometrica
        con il valore della serie geometrica. Siccome \(0 < \alpha < 1\) il termine non esplode
        e la serie geometrica converge.
        \item Detto \(x\) il limite di \(\{x_n\}\) mostriamo che è un punto fisso di \(f\).
        Consideriamo il limite per \(n\to \infty\)
        \[
            \lim_{n \to \infty} f(x_n) = f\left(\lim_{n \to \infty} x_n\right)
            = f(x)
        \]
        perché \(f\) è continua.
    \end{enumerate}
    Mostriamo ora l'unicità del punto.
    Supponiamo che \(x,y\) siano due punti fissi. Vogliamo mostrare che \(d(x,y) = 0\).
    Abbiamo
    \begin{align*}
        d(x,y) &= d(f(x), f(y)) \leq \alpha d(x,y)
        (1-\alpha) d(x,y) &\leq 0
    \end{align*}
    siccome \(1-\alpha > 0\) ciò succede solo se \(d(x,y) = 0\).
}

Abbiamo notato che
\[
    d(x_{n+k}, x_n) \leq \frac{\alpha^n}{1 - \alpha} d(x_1, x_0)
\]
Con \(k \to \infty\) otteniamo
\[
    d(x, x_n) \leq \frac{\alpha^n}{1 - \alpha} d(x_1, x_0)
\]
quindi tende al punto fisso in maniera esponenziale.

Denotiamo \(f^n = f \circ f \cdots f\).
Se \(f\) è una contrazione, una qualsiasi sua iterazione è anch'essa una contrazione.
\begin{align*}
    d(f(f(x)), f(f(y))) \leq \alpha d(f(x), f(y)) \leq \alpha^2 d(x,y)
\end{align*}
Per induzione segue il resto. In generale la costante è \(\alpha^n\).
Ci chiediamo se nel caso in cui \(f\) non sia una contrazione, una sua iterata lo possa essere.

\sexample{}{
    Per esempio \(f(x) = \cos x\), che non è una contrazione in quanto il supremum
    della derivata è \(1\). Invece, \(\cos(\cos(x))\) ha derivata
    \[
        \sin(\cos(x)) \cdot \sin
    \]
    Il suo modulo è dato da
    \[
        |\sin(\cos(x))| \cdot |\sin x|
        \leq \sin(1) < 1
    \]
    Il secondo termine può solamente essere maggiorato da \(1\), mentre il secondo,
    siccome \(-1 \leq \cos(x) \leq 1\), può essere maggiorato da \(\sin 1\).
    Quindi è una contrazione.

    Con questo possiamo per esempio mostrare che il coseno ha un punto fisso,
    siccome una sua iterata è una contrazione.
}


\scorollary{Indebolimento del teorema delle contrazioni: teorema delle iterate contrazioni}{
    Sia \((X, d)\) uno spazio metrico completo e sia
    \(f \colon X \to X\) un'applicazione tale che \(\exists n \in \mathbb{N}\)
    tale che \(f^n\) sia una contrazione. Allora \(\exists_{=1}\, x\in X\)
    tale che \(f(x)=x\).
}

\sproof{}{
    Mostriamo che i punti fissi di \(f\) (che sono uno solo) sono i punti fissi di \(f^n\).
    Sia \(x\) un punto fisso di \(f\). Allora \(f^n(x) = f(f(\cdots(x)) = f(x) = x\).
    Quindi tutti i punti fissi di \(f\) sono anche punti fissi di \(f^n\).
    Sia ora \(x\) tale che \(f^n(x)=x\). Componendo otteniamo
    \begin{align*}
        f(f^n(x)) &= f(x) \\
        f^n(f(x)) &= f(x)
    \end{align*}
    quindi \(f(x)\) è un punto fisso di \(f^n\), ma siccome \(f\) è una contrazione
    ha solo un punto fisso, quindi coincidono \(f(x)=x\).
    Quindi tutti i punti fissi di \(f^n\) sono anche punti fissi di \(f\).
}

Parametrizziamo ora la funzione

Consideriamo \(T \colon X \times Y \to X\) come operatore parametrizzato dai valori di \(Y\).
Fissato \(y\) imponiamo che \(T(-, y): X \to X\) sia una contrazione.
Per tutte le \(y\) esiste un solo \(x\in X\) tale che \(T(x,y) = x\).
Data la dipendenza funzionale \(x=\varphi(y)\) vogliamo capire come il punto fisso dipende dal parametro.
In particolare, vogliamo mostrare che \(\varphi\) è costante sotto alcune ipotesi.

\end{document}
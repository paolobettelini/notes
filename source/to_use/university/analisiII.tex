\documentclass[a4paper]{article}

\usepackage{amsmath}
\usepackage{amssymb}
\usepackage{stellar}
\usepackage{parskip}
\usepackage{fullpage}
\usepackage{wrapfig}
\usepackage{tikz}

\usetikzlibrary{arrows}
\usetikzlibrary{decorations.pathreplacing}
\usetikzlibrary{cd}

\title{Analisi II}
\author{Paolo Bettelini}
\date{}

\begin{document}

\maketitle
\tableofcontents

\section{Bolle}

L'insieme vuoto e l'insieme \(X\) sono aperti e chiusi.

\sproposition{}{
    L'unione di aperti non numerabile è aperta, mentre l'intersezione è aperta solo se finita.
}

\sproof{}{
    Per dimostrare quest'ultima lo facciamo su due insiemi e il resto è per induzione.
    Prendiamo un punto nell'interezione e prendiamo le due bolle dentro gli insiemi centrate
    nel punto. Siccome hanno lo stesso centro la loro intersezione è sempre una bolla di raggio
    il minore fra i due.
}

La metrice discreta può generare una bolla che è un singoletto.

\sproposition{}{
    L'unione di chiusi finiti è chiusa. L'intersezione qualsiasi è chiusa.
}

Ogni singoletto è chiuso. Per dimostrarlo mostriamo che nel complementare
esiste una bolla che non interseca il punto (vero per proprietà di Hausdorff).

Tutti i punti di accumulazione sono dei punti aderenti.
Tutti i punti di un sottoinsieme sono aderenti per il sottoinsieme.
Ogni punto o è di accumulazione o è isolato.

Se x_0 è aderente ad E, x_0 può essere un punto di E oppure no.
Se x_0 è punto di accumulazione per E, in ogni bolla centrata in x_0 cadono inifniti punti.

\sproposition{}{
    \(E^\circ\) è aperto. \(E\) è aperto se e solo se \(E = E^\circ\).
    \(E^\circ\) è il più grande aperto contenuto in \(E\).
    
    \(\overline{E}\) è chiuso. \(E\) è chiuso se e solo se \(E = \overline{E}\).
    La chiusura di \(E\) è il più piccolo chiuso contenente \(E\).
}

\sproof{per l'interno}{
    Dimostriamo che \(E^\circ\) è aperto. Sia \(x_0 \in E^\circ\).
    un punto interno ad \(E\), quindi esiste una bolla centrata in tale punto che è contenuta in \(E\).
    Prendiamo un altro punto \(y\) in questa bolla. Possiamo costruire una inner bolla centrata in \(y\)
    con un raggio sufficientemente piccolo da rimanere nella bolla più grande. Quindi il punto \(y\)
    è a sua volta interno, quindi tutta la bolla centrata in \(x_0\) è in \(E^\circ\) e quindi è aperto.
    
    Dimostriamo ora che se \(E\) è aperto allora \(E=E^\circ\) (l'altra implicazione è ovvia).
    Per fare ciò dimostriamo che \(E^\circ\) è il più grande aperto in \(E\).
    Osserviamo che \(E^\circ\) fa parte della famiglia degli aperti di \(X\)
    contenuti in \(E\). Sia \(A\) un aperto contenuto in \(E\). VOglio dimostrare che \(A \subseteq E^\circ\).
    Sia \(x_0 \in A\). \(A\) èunione di bolle quindi esiste unr aggio tale che la
    bolla centrata in \(x_0\) di tale raggio è contenuta in \(A\) che è contenuto in \(E\).
    Quindi, \(x_0\) è interno ad \(E\) e \(x_0 \in E^\circ\) e \(A \subseteq E^\circ\).
    Supponiamo ora che \(E\) sia aperto. Allora \(E\) fa parte della famiglia degli aperti
    di \(X\) contenuti in \(E\). Devo avere \(E \subseteq E^\circ\).
    Dato che \(E^\circ \subseteq E\) allora \(E^\circ = E\).
}

\end{document}
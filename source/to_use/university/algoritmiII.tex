\documentclass[a4paper]{article}

\usepackage{amsmath}
\usepackage{amssymb}
\usepackage{stellar}
\usepackage{parskip}
\usepackage{fullpage}
\usepackage{wrapfig}
\usepackage{tikz}

\usetikzlibrary{arrows}
\usetikzlibrary{decorations.pathreplacing}
\usetikzlibrary{cd}

\title{Algoritmi II}
\author{Paolo Bettelini}
\date{}

\begin{document}

\maketitle
\tableofcontents

\section{Notazioni asintotiche}

O-grande e \(\Theta\)-grande sono riflessive e transitive.
\(\Theta\)-grande è una relazione di equivalenza.
Alcune proprietà sono:
\begin{itemize}
    \item \[
        f(n) = O(g(n)) \implies cf(n) = O(g(n))
    \]
    per \(c>0\).
    \item se \(f_1(n) = O(g_1(n))\) e \(f_2(n) = O(g_2(n))\), allora
    \[
        f_1(n) + f_2(n) = O(g_1(n) + g_2(n))
    \]
    e
    \[
        f_1(n) \cdot f_2(n) = O(g_1(n) \cdot g_2(n))
    \]
    ma non con la sottrazione e divisione.
\end{itemize}

\sproof{La proprietà non vale con la sottrazione}{
    Consideriamo \(f(n) = n^2\) e \(f'(n) = n\), con
    \(g(n) = g'(n) = n^3\).
    Abbiamo quindi che \(f(n) = O(g(n))\) e \(f'(n) = O(g'(n))\).
    Tuttavia, \(n^2 - n \neq O(n^3 - n^3)\).
}

Alcune proprietà dell'asintotico sono

\begin{itemize}
    \item \[
        f(n) \sim g(n) \iff |f(n) - g(n)| = o(g(n))
    \]
    \item \[
        f(n) \sim g(n) \implies f(n) = \Theta(g(n))
    \]
    \item \[
        f(n) = o(g(n)) \implies f(n) = O(g(n))
    \]
\end{itemize}

\section{Il modello RAM (Random Access Machine)}

La random access memory ha complessità \(O(1)\) a differenza di quella sequenziale che
ha complessità \(O(n)\) (più semplice).

La Random Access machine ha una RAM formata da celle di registri.
I canali di input e output sono sequenziali. Si possono effettuare salti, salti condizionali e operazioni aritmetiche.

Per ora, approssimiamo il costo degli operatori come logaritmico rispetto alla dimensione
degli operandi. Il primo registro \(R_0\) prende il nome di \emph{accumulatore},
in quanto è l'unico registro attaccato all'ALU, ed è quindi la destinazione delle operazioni e
contiene uno dei due operandi iniziali.

\sdefinition{Programma}{
    Un \emph{programma} è una sequenza finita di istruzioni.
    Ogni istriuzione ha un'etichetta (l'indirizzo contenuto in lc).
    Ogni istruzione è una coppia (opcode, indirizzo).
    Ogni indirizzo può essere un'operando o un'etichetta.
}

% TODO lezione venerdì 15 boh

% tesi di church turing
% tesi di church turing estesa


\end{document}
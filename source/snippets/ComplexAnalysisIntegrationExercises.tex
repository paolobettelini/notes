\documentclass[preview]{standalone}

\usepackage{amsmath}
\usepackage{amssymb}
\usepackage{parskip}
\usepackage{fullpage}
\usepackage{hyperref}
\usepackage{wrapfig}
\usepackage{bettelini}
\usepackage{makecell}
\usepackage{stellar}

\hypersetup{
    colorlinks=true,
    linkcolor=black,
    urlcolor=blue,
    pdftitle={ComplexAnalysis},
    pdfpagemode=FullScreen,
}

\begin{document}

\title{Complex Analysis}
\id{complexanalysis-integration-exercises}
\genpage

\section{Complex integration}

\subsection{Complex integrals}

\begin{snippetexercise}{complex-analysis-ex-1}{}
    Compute \[ \oint_\Omega \frac{e^z}{{(z^2 + \pi^2)}^2} \,dz = 0 \]
    where \(\Omega\) is the circle \(|z|=4\).
\end{snippetexercise}

\begin{snippetsolution}{complex-analysis-ex-1-sol}{}
    The poles of \(\frac{e^z}{{(z^2 + \pi^2)}^2} = \frac{e^z}{(z-\pi i)^2(z+\pi i)^2}\)
    are at \(z=\pm \pi i\) in \(\Omega\) and are both of order 2.

    The residue in \(z=\pm \pi i\) is
    \[ \lim_{z \to \pi i} \frac{1}{1!} \frac{d}{dz} \left[
        (z - \pi i)^2 \frac{e^z}{(z-\pi i)^2(z+\pi i)^2}
    \right] = \frac{\pi + i}{4\pi^3}\]

    The residue in \(z=-\pi i\) is
    \[ \lim_{z \to \pi i} \frac{1}{1!} \frac{d}{dz} \left[
        (z - \pi i)^2 \frac{e^z}{(z-\pi i)^2(z+\pi i)^2}
    \right] = \frac{\pi - i}{4\pi^3}\]

    Thus, the integral is given by \(2\pi i \left( \frac{\pi + i}{4\pi^3} + \frac{\pi - i}{4\pi^3} \right) = \frac{i}{\pi}\).
\end{snippetsolution}

\end{document}
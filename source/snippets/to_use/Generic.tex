\documentclass[a4paper]{article}

\usepackage{amsmath}
\usepackage{amssymb}
\usepackage{parskip}
\usepackage{fullpage}
\usepackage{hyperref}

\hypersetup{
    colorlinks=true,
    linkcolor=black,
    urlcolor=blue,
    pdftitle={Generic},
    pdfpagemode=FullScreen,
}

\title{Generic}
\author{Paolo Bettelini}
\date{}

\begin{document}

\section{Plane}

A plane can be uniquely represented by its
normal vector \(\vec{n}\)
and a point on the plane \(P_0\).

To describe the plane using an equation, we can
consider an arbitrary point \(P=(x,y,z)\) on the plane.
There is always a 90 degrees angle between the normal
vector and the vector from \(P_0\) to \(P\) (i.e., their dot product is zero)

\[
    \vec{n} \cdot \overrightarrow{P_0 P} = 0
\]
By plugging in the values for \(\vec{n}\) and \(P_0\)
we get an equation in the form
\[
    Ax+By+Cz+D=0
\]

\section{Vector-Valued Function}

A vector-valued function is a function of a real parameter
which returns a vector
\[
    r(t) = \begin{pmatrix}
        f(t) \\
        g(t) \\
        h(t)
    \end{pmatrix}
\]

\section{Tangent Vector Vector-Valued Function}

Given a vector-valued function
\[
    r(t) = \begin{pmatrix}
        f(t) \\
        g(t) \\
        h(t)
    \end{pmatrix}
\]

where \(f\), \(g\) and \(h\) are differentiable,
then the Tangent vector to the curve is given b
\[
    r'(t) = \begin{pmatrix}
        f'(t) \\
        g'(t) \\
        h'(t)
    \end{pmatrix}
\]

\section{Curve length}

The length of the curve beteen a and b is the integral from a to b of \(\sqrt{f'(t)^2+g'(t)^2+h'(t)^2}\).

\end{document}
% https://youtu.be/40r56pX4mqA?list=PLHXZ9OQGMqxc_CvEy7xBKRQr6I214QJcd
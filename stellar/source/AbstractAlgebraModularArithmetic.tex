\documentclass[preview]{standalone}

\usepackage{amsmath}
\usepackage{amssymb}
\usepackage{parskip}
\usepackage{fullpage}
\usepackage{hyperref}
\usepackage{bettelini}
\usepackage{stellar}

\hypersetup{
    colorlinks=true,
    linkcolor=black,
    urlcolor=blue,
    pdftitle={Integers},
    pdfpagemode=FullScreen,
}

\newcommand{\divides}{\,|\,}

\begin{document}

\title{Integers}
\id{integers-modular-arithmetic}
\genpage

\section{Modular arithmetic}

\subsection{Congruence}

Let \(a,b,n\in\mathbb{Z}\).
We say that \(a\) and \(b\) are said to be congruent modulo \(n\),
denoted as \(a \equiv b \pmod{n}\), if \(a-b\) is a multiple of \(n\).

Note that \(\forall a,b \in \mathbb{Z}, a \equiv b \pmod{1}\textbackslash\).

\subsection{Congruence relation}

The congruence relation modulo \(n\) is an equivalence relation.

\begin{itemize}
    \item \textbf{Reflexive}: \(\forall a, a-a = 0\), which is always a multiple of \(n\).
    \item \textbf{Symmetric}: \(a \equiv b \pmod{n} \implies \exists k \suchthat a-b=kn \implies b-a=-kn\).
    Since \(-kn\) is a multiple of \(n\), then \(b \equiv a \pmod{n}\).
    \item \textbf{Transitive}: \(a \equiv b \pmod{n} \land b \equiv c \pmod{n}\) implies that both
    \(a-b\) and \(b - c\) are multiples of \(n\).
    \(\exists h, k \suchthat nh=a-b \land nk=b-c \implies a-b+b-c=nh+nk \implies a-c=n(h+k)\)
    which means that \(a-c\) is also a multiple of \(n\), so \(a \equiv c \pmod{n}\).
\end{itemize}

\subsection{Equivalence of summation and multiplication}

If \(a \equiv a' \pmod{n}\) and If \(b \equiv b' \pmod{n}\), then
\(a+b \equiv a' + b' \pmod{n}\) and \(ab \equiv a'b' \pmod{n}\).

We can prove this by noting that there exist integers \(h and k\) such that
\(a=a'+hn\) and \(b=b'+kn\).
Now \(a+b = a'+b'+n(h+k)\) and \(ab=a'b' + n(hb0+ka'+hkn)\), meaning
\(a+b \equiv a' + b' \pmod{n}\) and \(ab \equiv a'b' \pmod{n}\).

\subsection{Congruence class}

The equivalence class of an integer \(a\) with respect to modulo \(n\)
is said to be a \textbf{congruence class}, denoted \({[a]}_n\).
\[
    {[a]}_n = \{a+kn \suchthat k \in \mathbb{Z}\}
\]

Note that
\[
    {[a]}_n = {[a + kn]}_n,\quad k \in \mathbb{Z}
\]

\subsection{Quotient set}

The set of all congruence classes modulo \(n\) is denoted \(\mathbb{Z} / n\).

Note that \(\mathbb{Z} / n\) has \(n\) elements:
\[
    {[0]}_n,{[1]}_n,\cdots,{[n-1]}_n
\]
% TODO proof

\subsection{Operations with congruent classes}

\begin{align*}
    {[a]}_n + {[b]}_n &\triangleq {[a+b]}_n \\
    {[a]}_n \cdot {[b]}_n &\triangleq {[a \cdot b]}_n
\end{align*}

\subsection{Properties of congruent classes}

For all \({[a]}_n, {[b]}_n, {[c]}_n \in \mathbb{Z}/n\).

\begin{enumerate}
    \item \textbf{Associative addition}: \({[a]}_n + ({[b]}_n + {[c]}_n) = ({[a]}_n + {[b]}_n) + {[c]}_n\)
    \item \textbf{Associative multiplication}: \({[a]}_n ({[b]}_n {[c]}_n) = ({[a]}_n {[b]}_n) {[c]}_n\)
    \item \textbf{Commutative addition}: \({[a]}_n + {[b]}_n = {[b]}_n + {[a]}_n\)
    \item \textbf{Commutative multiplication}: \({[a]}_n {[b]}_n = {[b]}_n {[a]}_n\)
    \item \textbf{Neutral addition element}: \({[a]}_n + {[0]}_n = {[a]}_n\)
    \item \textbf{Neutral multiplication element}: \({[a]}_n {[1]}_n = {[a]}_n\)
    \item \textbf{Inverse addition element}: \((-{[a]}_n) + {[a]}_n = {[0]}_n\)
    \item \textbf{Distributive property}: \(({[a]}_n + {[b]}_n) {[c]}_n = {[a]}_n{[c]}_n + {[b]}_n{[c]}_n\)
    \item \textbf{Cancellation law}: \({[a]}_n + {[b]}_n = {[a]}_n + {[c]}_n \implies {[b]}_n = {[c]}_n\)
\end{enumerate}

% TODO proofs, especially cancellation law

% TODO proof of [a][0] = [0] ?

\subsection{Invertible congruent classes}

A congruent class \({[a]}_n\) is \textbf{invertible}
if there exist an \({[b]}_n\) such that \({[a]}_n{[b]}_n={[1]}_n\).

The inverse of \({[a]}_n\) is denoted \({[a]}_n^{-1}\).

\subsection{Properties of inverses}

\begin{enumerate}
    \item If \({[a]}_n\) is invertible, then \({[a]}_n^{-1}\) is unique.
    \item If \({[a]}_n\) is invertible, then \({[a]}_n^{-1}\) is invertible and \(({[a]}_n^{-1})^{-1}={[a]}_n\).
    \item If \({[a]}_n\) and \({[b]}_n\) are invertible, then \({[a]}_n{[b]}_n\) is invertible and
    \({({[a]}_n{[b]}_n)}^{-1} = {[a]}_n^{-1}{[b]}_n^{-1}\).8
\end{enumerate}

% PROOF

% pag 49

% https://algebrainsubria.altervista.org/algebra.pdf

\end{document}

\documentclass[preview]{standalone}

\usepackage{amsmath}
\usepackage{amssymb}
\usepackage{parskip}
\usepackage{fullpage}
\usepackage{hyperref}
\usepackage{tikz}
\usepackage{stellar}

\hypersetup{
    colorlinks=true,
    linkcolor=black,
    urlcolor=blue,
    pdftitle={ComplexNumbers},
    pdfpagemode=FullScreen,
}

\begin{document}

\title{Complex Numbers}
\id{complexnumbers}
\genpage

\section{Imaginary unit}

\subsection{Definition}

The imaginary unit or imaginary number \(i\) is a solution to the quadratic
equation \(x^2=-1\) and is defined as

\[
    i^2 = -1
\]

The equation \(x^2 = -1\) has two solutions: \(i\) and \(-i\), however,
there is not any algebraic difference between these two solutions.

\subsection{Properties}

The imaginary number \(i\) has some amazing properties when it comes to exponentiation.

\[
	\begin{cases}
		i^0=+1\\
		i^1=+i\\
		i^2=-1\\
		i^3=-i\\
	\end{cases}
	\quad
	\begin{cases}
		i^4=+1\\
		i^5=+i\\
		i^6=-1\\
		i^7=-i\\
	\end{cases}
	\quad
	\cdots
\]

The multiplicative inverse of \(i\) is \(-i\).

\[
    \frac{1}{i} = \frac{1}{i} \cdot \frac{i}{i}
    = \frac{i}{i^2} = -i
\]

\pagebreak

\section{Complex Numbers}

\subsection{Definition}

Complex numbers are numbers in the form \(a + bi\),
where \(a,b\in\mathbb{R}\) and \(i\) is the imaginary unit.
\\
This set of numbers is called \(\mathbb{C}\).

Since every number \(n\in\mathbb{R}\) can be represented as
a complex number in the form \(n+0i\), \(\mathbb{R}\subset\mathbb{C}\).

\subsection{Complex plane}

We can represent each complex number on a plane (Argand plane), where the horizontal axis
represent the real numbers \(\mathbb{R}\) and the vertical axis represents
every scalar multiple of the imaginary unit \(i\).

\begin{center}
    \begin{tikzpicture}
        \begin{scope}[thick,font=\scriptsize]

            \draw [->] (-5,0) -- (5,0) node [above left]  {\(\Re(s)\)};
            \draw [->] (0,-5) -- (0,5) node [below right] {\(\Im(s)\)};

            \draw (0,0) -- (0,0)   node [above right] {\(0\)};
            \foreach \n in {-4,...,-1,1,2,...,4}{
                \draw (\n,-3pt) -- (\n,3pt) node [above] {\(\n\)};
                \draw (-3pt,\n) -- (3pt,\n) node [right] {\(\n i\)};
            }

            \draw [color=black, fill=black] (3,2) circle (0.05) node [above] {\(3+2i\)};
        \end{scope}
    \end{tikzpicture}
\end{center}

\subsection{Operations}

\subsubsection{Addition}
\[
    (a+bi)+(c+di)=a+bi+c+di=(a+c)+(b+d)i
\]

\subsubsection{Subtraction}
\[
    (a+bi)-(c+di)=a+bi-c-di=(a-c)+(b-d)i
\]

\subsubsection{Multiplication}
\[
    (a+bi)(c+di)=ac+adi+bci+bdi^2=(ac-db)+(ad+bc)i
\]

\subsubsection{Division}
\begin{align*}
    \frac{a+bi}{c+di} &= \frac{a+bi}{c+di} \cdot \frac{c-di}{c-di} = \frac{ac-adi+bci-bdi^2}{c^2-d^2i^2}
    \\ &= \frac{ac+bd+(bc-ad)i}{c^2+d^2} \\ &= \frac{ac+bd}{c^2+d^2} + \frac{bc-ad}{c^2 + d^2}i
\end{align*}

\subsubsection{Real part}

The real part of a complex number \(s\) is denoted as \(\text{Re}(s)\) or \(\Re(s)\).

\[
    \text{Re}(a+bi) = a
\]

\subsubsection{Imaginary part}

The imaginary part of a complex number \(s\) is denoted as \(\text{Im}(s)\) or \(\Im(s)\).

\[
    \text{Im}(a+bi) = b
\]

\subsubsection{Absolute value}

The absolute value (or module) of a complex number is its distance from the origin.

\[
    |a+bi| = \sqrt{a^2 + b^2}
\]

\subsubsection{Conjugate}

The complex conjugate of a number \(s=a+bi\) is denoted as \(s^*\) or \(\overline{s}\).
It is defined as

\[
    \overline{a+bi} = a-bi
\]

Geometrically, \(s^*\) is the reflection about the real axis in the complex plane.

We also have the following trivial properties.

\begin{align*}
    \overline{\overline{s}} &= s
    \\
    \text{Re}(\overline{s}) &= \text{Re}(s)
    \\
    \text{Im}(\overline{s}) &= -\text{Im}(s)
\end{align*}

\subsubsection{Argument}

The argument of a complex number is the angle formed with the x-axis in
the complex plane
\[
    \arg(a+bi)=\arctan \left(\frac{b}{a}\right)
\]

\subsubsection{Axiomatic definition}

A complex number is a tuple \((a,b)\) where \(a\in \mathbb{R}\) and \(b\in \mathbb{R}\).

\paragraph{equality}
\[
    (a,b) = (c,d) \implies a=c \land b=d
\]

\paragraph{Addition}
\[
    (a,b) + (c,d) = (a+c, b+d)
\]

\paragraph{Multiplication}
\begin{align*}
    (a,b) \cdot (c,d) &= (ac-db, ad+bc)
    \\
    m(a,b) &= (ma, mb)
\end{align*}

If \(z_1, z_2, z_3 \in \mathbb{C}\).
\begin{enumerate}
    \item \(z_1+z_2\) and \(z_1z_2\) are also in \(\mathbb{C}\)
    \item \(z_1+z_2=z_2+z_1\)
    \item \(z_1 + (z_2 + z_3) = (z_1 + z_2) + z_3\)
    \item \(z_1z_2=z_2z_1\)
    \item \(z_1(z_2z-3)=(z_1z_2)z_3\)
    \item \(z_1(z_2+z_3)=z_1z_2+z_1z_3\)
    \item \(z_1+0=z_1\)
    \item \(z_1\cdot 1=z_1\)
    \item \(\exists ! z \,|\, z+z_1=0\)
    \item \(\exists ! z \,|\, z\cdot z_1=1\)
\end{enumerate}

\subsection{Trigonometric form}

Any complex number can be represented in a trigonometric form
\[
    a+bi=r(\cos\theta+i\sin\theta)
\]
where \(r\) is the absolute value and \(\theta\) is the argument.

\subsection{Vector form}

Any complex number \(a+bi\) can be represented by a vector \((a,b)\).

\paragraph{Scalar product}

The scalar product between \(z_1=a+bi\) and \(z_2=c+di\) is given by

\[
    z_1\circ z_2 = |z_1|\,|z_2|\cos\theta
    = ac+bd = \Re(z_1^*z_2) = \frac{1}{2}(z_1^*z_2+z_1z_2^*)
\]
where \(\theta\) is the angle formed by the two vectors.

\paragraph{Vector product}

The vector product between \(z_1=a+bi\) and \(z_2=c+di\) is given by

\[
    z_1\times z_2 = |z_1|\,|z_2|\sin\theta
    = ad-cb = \Im(z_1^*z_2) = \frac{1}{2i}(z_1^*z_2+z_1z_2^*)
\]
where \(\theta\) is the angle formed by the two vectors.

We can see that
\[
    z_1^*z_2 = (z_1\circ z_2) + i(z_1 \times z_2)
\]

\subsection{Complex conjugate coordinates}

Since for any complex number \(z=a+bi\)
\begin{align*}
    a&=\frac{1}{2}(z+z^*)
    \\
    b&=\frac{1}{2i}(z-z^*)
\end{align*}
\(z\) can also be represented by the conjugate coordinates \((z, z^*)\).

% TODO: exponentiation
% TODO: Cauchy - Schwarz inequality
% TODO: also polar form with Euler's formula

\end{document}

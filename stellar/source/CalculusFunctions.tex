\documentclass[preview]{standalone}

\usepackage{amsmath}
\usepackage{amssymb}
\usepackage{parskip}
\usepackage{fullpage}
\usepackage{hyperref}
\usepackage{stellar}

\hypersetup{
    colorlinks=true,
    linkcolor=black,
    urlcolor=blue,
    pdftitle={Functions},
    pdfpagemode=FullScreen,
}

\begin{document}

\title{Functions}
\id{functions}
\genpage

\section{Definition}

\begin{snippet}{function-definition}
\sdefinition{Function}{
    Let \(A\) and \(B\) be sets.
    A \textit{function} is a set \(f \subset A \times B\) where
    \[
        \forall x \in A \exists_{=1} \, y \in B \,|\, (x,y) \in f
    \]
}
\end{snippet}

\section{Properties}

\subsection{Injectivity}

\begin{snippet}{injectivity-definition}
\sdefinition{Injectivity}{
    A function \(f:A\to B\) is \textit{injective} if
    \[
        \forall a,b \in A, f(a) = f(b) \implies a = b
    \]
}
\end{snippet}

\subsection{Surjectivity}

\begin{snippet}{surjectivity-definition}
\sdefinition{Surjectivity}{
    A function \(f:A\to B\) is \textit{surjectiv} if
    \[
        \forall b \in B \exists a \,|\, f(a)=b
    \]
}
\end{snippet}

\subsection{Bijectivity}

\begin{snippet}{bijectivity-definition}
\sdefinition{Bijectivity}{
    A function \(f:A\to B\) is \textit{bijective} if
    it has a one-to-one correspondence between each element of \(A\) and  \(B\).
}
\end{snippet}

\begin{snippet}{bijectivity-equiv-inj-and-surj}
\scorollary{Bijectivity properties}{
    A function \(f:A\to B\) is bijective iff it is both injective and surjective.
}
\end{snippet}

\subsection{Invertibility}

%% TODO invertibility and then a theorem <=> bijection
\begin{snippet}{invertibility-definition}
\sdefinition{Invertibility}{
    A function \(f\) is \textit{invertible} iff it is a bijection.
}
\end{snippet}

\subsection{Periodic functions}

\begin{snippet}{periodic-function-definition}
\sdefinition{Periodic Function}{
    A function \(f\) is periodic with a period \(T\) if
    \[
        f(x) = f(x + kT), \quad k \in \mathbb{Z}
    \]
}
\end{snippet}

\subsection{Odd functions}

\begin{snippet}{odd-function-definition}
\sdefinition{Odd Function}{
    A function \(f\) is \textit{odd} if
    \[
        f(-x) = -f(x)
    \]
}
\end{snippet}

\subsection{Even functions}

\begin{snippet}{even-function-definition}
\sdefinition{Even Function}{
    A function \(f\) is even if
    \[
        f(-x) = f(x)
    \]
}
\end{snippet}

\end{document}
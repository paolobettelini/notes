\documentclass[preview]{standalone}

\usepackage{amsmath}
\usepackage{amssymb}
\usepackage{parskip}
\usepackage{fullpage}
\usepackage{hyperref}
\usepackage{bettelini}
\usepackage{stellar}

\hypersetup{
    colorlinks=true,
    linkcolor=black,
    urlcolor=blue,
    pdftitle={Measure Theory},
    pdfpagemode=FullScreen,
}

\begin{document}

\title{Measure Theory}
\id{measuretheory}
\genpage

% https://courses.maths.ox.ac.uk/course/view.php?id=1041
% Zhongmin Qian's Notes from 2017 

\section{XXX}

\begin{snippetdefinition}{dirichlet-function-definition}{Dirichlet function}
    The \textit{Dirichlet function} is defined as the indicator function of the rational numbers
    \[
        1_{\mathbb{Q}}(x) \triangleq \begin{cases}
            1 & x \in \mathbb{Q} \\
            0 & x \notin \mathbb{Q}
        \end{cases}
    \]
\end{snippetdefinition}

% Since there are uncountably many irrational numbers and countably many ration numbers,
% and thus there is a 0\% probability of picking a rational number, the integral
% of this function in this interval is 1.
% However, the Dirichlet function is not Riemann integratable.
% the lebesgue measure extends the class of integfratable functions.

\end{document}

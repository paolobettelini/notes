\documentclass[preview]{standalone}

\usepackage{amsmath}
\usepackage{amssymb}
\usepackage{parskip}
\usepackage{fullpage}
\usepackage{hyperref}
\usepackage{bettelini}
\usepackage{stellar}

\hypersetup{
    colorlinks=true,
    linkcolor=black,
    urlcolor=blue,
    pdftitle={Cyclic groups},
    pdfpagemode=FullScreen,
}

\begin{document}

\title{Fourier Analysis}
\id{fourieranalysis-introduction}
\genpage

\section{What is Fourier Analysis?}

\begin{snippet}{fourier-analysis-introduction}
    Fourier analysis is the study of how a function can be represented as a sum of waves. Take a look at the animation playing at the side, a shape is being drawn using a chain of rotating circles of different sizes. You can even try drawing your own shape, it's interactive! This article will cover in detail how this animation works, and what math is behind it. The concepts that we'll discover are widely used in electronics, acoustics and communications. Operators such as the Fourier Transform are constantly used in the real world, without these discoveries the world would not be the same. Much software relies in Fourier Analysis, such as for instance Shazam, the famous service for identifying songs. Any audio spectrum visualized processes the signal using Fourier Transform, these are just a few of the many application of this analysis.
\end{snippet}

\includesnpt{fourier-lib}
\includesnpt{fourier-series-2d}

% difference between fourier series and transform

\end{document}

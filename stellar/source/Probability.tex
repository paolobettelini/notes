\documentclass[preview]{standalone}

\usepackage{amsmath}
\usepackage{amssymb}
\usepackage{parskip}
\usepackage{fullpage}
\usepackage{hyperref}
\usepackage{stellar}
\usepackage{bettelini}

\hypersetup{
    colorlinks=true,
    linkcolor=black,
    urlcolor=blue,
    pdftitle={Probability},
    pdfpagemode=FullScreen,
}

\begin{document}

\title{Probability}
\id{probability}
\genpage

% 1) https://courses.maths.ox.ac.uk/pluginfile.php/11726/mod_resource/content/4/ProbabilityNotes2021-10-11.pdf
% 2) https://courses.maths.ox.ac.uk/pluginfile.php/7683/mod_resource/content/5/A8LectureNotes_MT21.pdf

\section{Definition}

\begin{snippet}{probability-space-definition}
\sdefinition{Probability Space}{
    A \textit{probability space} is a tuple \((\Omega, \mathcal{F}, \mathbb{P})\)
    where
    \begin{itemize}
        \item \(\Omega\) is the set of all possible outcomes, called \textit{sample space};
        \item \(\mathcal{F}\) is a set of subsets of \(\Omega\), called \textit{event};
        \item \(\mathbb{P}\) is a function \(\mathbb{P}\colon \mathcal{F} \to [0;1]\), called a \textit{probability measure}. 
    \end{itemize}
    The set \(\mathcal{F}\) satisfies the following axioms:
    \begin{enumerate}
        \item it contains the sample space: \(\Omega \in \mathcal{F}\);
        \item it is closed under complements: \(\forall A \in \mathcal{F}, \Omega \setminus A \in \mathcal{F}\);
        \item it is closed under countable unions: \({\{A_i\}}_{i \in I} \subseteq \mathcal{F} \implies \bigcup_{i \in I}A_i \in \mathcal{F}\).
    \end{enumerate}
    The probability measure \(\mathbb{P}\) follows the probability axioms:
    \begin{enumerate}
        \item \(\mathbb{P}(\Omega) = 1\);
        \item it is countably additive: if \({\{A_i\}}_{i \in I} \subseteq \mathcal{F}\)
        is a countable collection of pairwise disjoint sets, then
        \[
            \mathbb{P}\left(\bigcup_{i \in I}A_i\right) = \sum_{i \in I} \mathbb{P}(A_i)
        \]
    \end{enumerate}
}
\end{snippet}

\begin{snippet}{measure-of-inverse-theorem}
\stheorem{Measure of inverse}{
    Let \((\Omega, \mathcal{F}, \mathbb{P})\) be a probability space and \(A, B \in \mathcal{F}\).
    \[
        \mathbb{P}(\Omega \setminus A) = 1 - \mathbb{P}(A)
    \]
}
\end{snippet}

\begin{snippet}{measure-of-inverse-proof}
\sproof{Measure of inverse}{
    Since \(\mathbb{P}(\Omega) = \mathbb{P}(A \cup (\Omega \setminus A)) = 1\),
    we have \(\mathbb{P}(A) + \mathbb{P}(\Omega \setminus A) = 1\)
    and thus the theorem.
}
\end{snippet}

\begin{snippet}{measure-of-subset-theorem}
\stheorem{Measure of subset}{
    Let \((\Omega, \mathcal{F}, \mathbb{P})\) be a probability space and \(A, B \in \mathcal{F}\).
    \[
        A \subseteq B \implies \mathbb{P}(A) \leq \mathbb{P}(B)
    \]
}
\end{snippet}

\begin{snippet}{measure-of-subset-proof}
\sproof{Measure of subset}{
    Assume \(A \subseteq B\). Since \(\mathbb{P}(B) = \mathbb{P}(A) + \mathbb{P}(B \cap (\Omega \setminus A))\)
    and \(\mathbb{P}(B \cap (\Omega \setminus A)) \geq 0\), we have \(\mathbb{P}(B) \geq \mathbb{P}(A)\).
}
\end{snippet}

\begin{snippet}{conditional-probability-definition}
\sdefinition{Conditional Probability}{
    Let \((\Omega, \mathcal{F}, \mathbb{P})\) be a probability space
    and \(A, B \in \mathcal{F}\).
    If \(\mathbb{P}(B) > 0\), then the \textit{conditional probability} of \(A\) given \(B\)
    is given by
    \[
        \mathbb{P}(A|B) = \frac{\mathbb{P}(A \cap B)}{\mathbb{P}(B)}
    \]
}
\end{snippet}

\end{document}